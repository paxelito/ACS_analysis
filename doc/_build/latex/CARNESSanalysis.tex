% Generated by Sphinx.
\def\sphinxdocclass{report}
\documentclass[letterpaper,10pt,english]{sphinxmanual}
\usepackage[utf8]{inputenc}
\DeclareUnicodeCharacter{00A0}{\nobreakspace}
\usepackage{cmap}
\usepackage[T1]{fontenc}
\usepackage{babel}
\usepackage{times}
\usepackage[Bjarne]{fncychap}
\usepackage{longtable}
\usepackage{sphinx}
\usepackage{multirow}


\title{CARNESS analysis package}
\date{April 23, 2015}
\release{20150424.001}
\author{Alessandro Filisetti}
\newcommand{\sphinxlogo}{\includegraphics{logo.png}\par}
\renewcommand{\releasename}{Release}
\makeindex

\makeatletter
\def\PYG@reset{\let\PYG@it=\relax \let\PYG@bf=\relax%
    \let\PYG@ul=\relax \let\PYG@tc=\relax%
    \let\PYG@bc=\relax \let\PYG@ff=\relax}
\def\PYG@tok#1{\csname PYG@tok@#1\endcsname}
\def\PYG@toks#1+{\ifx\relax#1\empty\else%
    \PYG@tok{#1}\expandafter\PYG@toks\fi}
\def\PYG@do#1{\PYG@bc{\PYG@tc{\PYG@ul{%
    \PYG@it{\PYG@bf{\PYG@ff{#1}}}}}}}
\def\PYG#1#2{\PYG@reset\PYG@toks#1+\relax+\PYG@do{#2}}

\expandafter\def\csname PYG@tok@gd\endcsname{\def\PYG@tc##1{\textcolor[rgb]{0.63,0.00,0.00}{##1}}}
\expandafter\def\csname PYG@tok@gu\endcsname{\let\PYG@bf=\textbf\def\PYG@tc##1{\textcolor[rgb]{0.50,0.00,0.50}{##1}}}
\expandafter\def\csname PYG@tok@gt\endcsname{\def\PYG@tc##1{\textcolor[rgb]{0.00,0.27,0.87}{##1}}}
\expandafter\def\csname PYG@tok@gs\endcsname{\let\PYG@bf=\textbf}
\expandafter\def\csname PYG@tok@gr\endcsname{\def\PYG@tc##1{\textcolor[rgb]{1.00,0.00,0.00}{##1}}}
\expandafter\def\csname PYG@tok@cm\endcsname{\let\PYG@it=\textit\def\PYG@tc##1{\textcolor[rgb]{0.25,0.50,0.56}{##1}}}
\expandafter\def\csname PYG@tok@vg\endcsname{\def\PYG@tc##1{\textcolor[rgb]{0.73,0.38,0.84}{##1}}}
\expandafter\def\csname PYG@tok@m\endcsname{\def\PYG@tc##1{\textcolor[rgb]{0.13,0.50,0.31}{##1}}}
\expandafter\def\csname PYG@tok@mh\endcsname{\def\PYG@tc##1{\textcolor[rgb]{0.13,0.50,0.31}{##1}}}
\expandafter\def\csname PYG@tok@cs\endcsname{\def\PYG@tc##1{\textcolor[rgb]{0.25,0.50,0.56}{##1}}\def\PYG@bc##1{\setlength{\fboxsep}{0pt}\colorbox[rgb]{1.00,0.94,0.94}{\strut ##1}}}
\expandafter\def\csname PYG@tok@ge\endcsname{\let\PYG@it=\textit}
\expandafter\def\csname PYG@tok@vc\endcsname{\def\PYG@tc##1{\textcolor[rgb]{0.73,0.38,0.84}{##1}}}
\expandafter\def\csname PYG@tok@il\endcsname{\def\PYG@tc##1{\textcolor[rgb]{0.13,0.50,0.31}{##1}}}
\expandafter\def\csname PYG@tok@go\endcsname{\def\PYG@tc##1{\textcolor[rgb]{0.20,0.20,0.20}{##1}}}
\expandafter\def\csname PYG@tok@cp\endcsname{\def\PYG@tc##1{\textcolor[rgb]{0.00,0.44,0.13}{##1}}}
\expandafter\def\csname PYG@tok@gi\endcsname{\def\PYG@tc##1{\textcolor[rgb]{0.00,0.63,0.00}{##1}}}
\expandafter\def\csname PYG@tok@gh\endcsname{\let\PYG@bf=\textbf\def\PYG@tc##1{\textcolor[rgb]{0.00,0.00,0.50}{##1}}}
\expandafter\def\csname PYG@tok@ni\endcsname{\let\PYG@bf=\textbf\def\PYG@tc##1{\textcolor[rgb]{0.84,0.33,0.22}{##1}}}
\expandafter\def\csname PYG@tok@nl\endcsname{\let\PYG@bf=\textbf\def\PYG@tc##1{\textcolor[rgb]{0.00,0.13,0.44}{##1}}}
\expandafter\def\csname PYG@tok@nn\endcsname{\let\PYG@bf=\textbf\def\PYG@tc##1{\textcolor[rgb]{0.05,0.52,0.71}{##1}}}
\expandafter\def\csname PYG@tok@no\endcsname{\def\PYG@tc##1{\textcolor[rgb]{0.38,0.68,0.84}{##1}}}
\expandafter\def\csname PYG@tok@na\endcsname{\def\PYG@tc##1{\textcolor[rgb]{0.25,0.44,0.63}{##1}}}
\expandafter\def\csname PYG@tok@nb\endcsname{\def\PYG@tc##1{\textcolor[rgb]{0.00,0.44,0.13}{##1}}}
\expandafter\def\csname PYG@tok@nc\endcsname{\let\PYG@bf=\textbf\def\PYG@tc##1{\textcolor[rgb]{0.05,0.52,0.71}{##1}}}
\expandafter\def\csname PYG@tok@nd\endcsname{\let\PYG@bf=\textbf\def\PYG@tc##1{\textcolor[rgb]{0.33,0.33,0.33}{##1}}}
\expandafter\def\csname PYG@tok@ne\endcsname{\def\PYG@tc##1{\textcolor[rgb]{0.00,0.44,0.13}{##1}}}
\expandafter\def\csname PYG@tok@nf\endcsname{\def\PYG@tc##1{\textcolor[rgb]{0.02,0.16,0.49}{##1}}}
\expandafter\def\csname PYG@tok@si\endcsname{\let\PYG@it=\textit\def\PYG@tc##1{\textcolor[rgb]{0.44,0.63,0.82}{##1}}}
\expandafter\def\csname PYG@tok@s2\endcsname{\def\PYG@tc##1{\textcolor[rgb]{0.25,0.44,0.63}{##1}}}
\expandafter\def\csname PYG@tok@vi\endcsname{\def\PYG@tc##1{\textcolor[rgb]{0.73,0.38,0.84}{##1}}}
\expandafter\def\csname PYG@tok@nt\endcsname{\let\PYG@bf=\textbf\def\PYG@tc##1{\textcolor[rgb]{0.02,0.16,0.45}{##1}}}
\expandafter\def\csname PYG@tok@nv\endcsname{\def\PYG@tc##1{\textcolor[rgb]{0.73,0.38,0.84}{##1}}}
\expandafter\def\csname PYG@tok@s1\endcsname{\def\PYG@tc##1{\textcolor[rgb]{0.25,0.44,0.63}{##1}}}
\expandafter\def\csname PYG@tok@gp\endcsname{\let\PYG@bf=\textbf\def\PYG@tc##1{\textcolor[rgb]{0.78,0.36,0.04}{##1}}}
\expandafter\def\csname PYG@tok@sh\endcsname{\def\PYG@tc##1{\textcolor[rgb]{0.25,0.44,0.63}{##1}}}
\expandafter\def\csname PYG@tok@ow\endcsname{\let\PYG@bf=\textbf\def\PYG@tc##1{\textcolor[rgb]{0.00,0.44,0.13}{##1}}}
\expandafter\def\csname PYG@tok@sx\endcsname{\def\PYG@tc##1{\textcolor[rgb]{0.78,0.36,0.04}{##1}}}
\expandafter\def\csname PYG@tok@bp\endcsname{\def\PYG@tc##1{\textcolor[rgb]{0.00,0.44,0.13}{##1}}}
\expandafter\def\csname PYG@tok@c1\endcsname{\let\PYG@it=\textit\def\PYG@tc##1{\textcolor[rgb]{0.25,0.50,0.56}{##1}}}
\expandafter\def\csname PYG@tok@kc\endcsname{\let\PYG@bf=\textbf\def\PYG@tc##1{\textcolor[rgb]{0.00,0.44,0.13}{##1}}}
\expandafter\def\csname PYG@tok@c\endcsname{\let\PYG@it=\textit\def\PYG@tc##1{\textcolor[rgb]{0.25,0.50,0.56}{##1}}}
\expandafter\def\csname PYG@tok@mf\endcsname{\def\PYG@tc##1{\textcolor[rgb]{0.13,0.50,0.31}{##1}}}
\expandafter\def\csname PYG@tok@err\endcsname{\def\PYG@bc##1{\setlength{\fboxsep}{0pt}\fcolorbox[rgb]{1.00,0.00,0.00}{1,1,1}{\strut ##1}}}
\expandafter\def\csname PYG@tok@kd\endcsname{\let\PYG@bf=\textbf\def\PYG@tc##1{\textcolor[rgb]{0.00,0.44,0.13}{##1}}}
\expandafter\def\csname PYG@tok@ss\endcsname{\def\PYG@tc##1{\textcolor[rgb]{0.32,0.47,0.09}{##1}}}
\expandafter\def\csname PYG@tok@sr\endcsname{\def\PYG@tc##1{\textcolor[rgb]{0.14,0.33,0.53}{##1}}}
\expandafter\def\csname PYG@tok@mo\endcsname{\def\PYG@tc##1{\textcolor[rgb]{0.13,0.50,0.31}{##1}}}
\expandafter\def\csname PYG@tok@mi\endcsname{\def\PYG@tc##1{\textcolor[rgb]{0.13,0.50,0.31}{##1}}}
\expandafter\def\csname PYG@tok@kn\endcsname{\let\PYG@bf=\textbf\def\PYG@tc##1{\textcolor[rgb]{0.00,0.44,0.13}{##1}}}
\expandafter\def\csname PYG@tok@o\endcsname{\def\PYG@tc##1{\textcolor[rgb]{0.40,0.40,0.40}{##1}}}
\expandafter\def\csname PYG@tok@kr\endcsname{\let\PYG@bf=\textbf\def\PYG@tc##1{\textcolor[rgb]{0.00,0.44,0.13}{##1}}}
\expandafter\def\csname PYG@tok@s\endcsname{\def\PYG@tc##1{\textcolor[rgb]{0.25,0.44,0.63}{##1}}}
\expandafter\def\csname PYG@tok@kp\endcsname{\def\PYG@tc##1{\textcolor[rgb]{0.00,0.44,0.13}{##1}}}
\expandafter\def\csname PYG@tok@w\endcsname{\def\PYG@tc##1{\textcolor[rgb]{0.73,0.73,0.73}{##1}}}
\expandafter\def\csname PYG@tok@kt\endcsname{\def\PYG@tc##1{\textcolor[rgb]{0.56,0.13,0.00}{##1}}}
\expandafter\def\csname PYG@tok@sc\endcsname{\def\PYG@tc##1{\textcolor[rgb]{0.25,0.44,0.63}{##1}}}
\expandafter\def\csname PYG@tok@sb\endcsname{\def\PYG@tc##1{\textcolor[rgb]{0.25,0.44,0.63}{##1}}}
\expandafter\def\csname PYG@tok@k\endcsname{\let\PYG@bf=\textbf\def\PYG@tc##1{\textcolor[rgb]{0.00,0.44,0.13}{##1}}}
\expandafter\def\csname PYG@tok@se\endcsname{\let\PYG@bf=\textbf\def\PYG@tc##1{\textcolor[rgb]{0.25,0.44,0.63}{##1}}}
\expandafter\def\csname PYG@tok@sd\endcsname{\let\PYG@it=\textit\def\PYG@tc##1{\textcolor[rgb]{0.25,0.44,0.63}{##1}}}

\def\PYGZbs{\char`\\}
\def\PYGZus{\char`\_}
\def\PYGZob{\char`\{}
\def\PYGZcb{\char`\}}
\def\PYGZca{\char`\^}
\def\PYGZam{\char`\&}
\def\PYGZlt{\char`\<}
\def\PYGZgt{\char`\>}
\def\PYGZsh{\char`\#}
\def\PYGZpc{\char`\%}
\def\PYGZdl{\char`\$}
\def\PYGZhy{\char`\-}
\def\PYGZsq{\char`\'}
\def\PYGZdq{\char`\"}
\def\PYGZti{\char`\~}
% for compatibility with earlier versions
\def\PYGZat{@}
\def\PYGZlb{[}
\def\PYGZrb{]}
\makeatother

\begin{document}

\maketitle
\tableofcontents
\phantomsection\label{index::doc}


Contents:


\chapter{Chemistry Graph Analysis}
\label{graph_chemistry_analysis:chemistry-graph-analysis}\label{graph_chemistry_analysis:carness-analysis-python-package-documentation}\label{graph_chemistry_analysis::doc}\label{graph_chemistry_analysis:module-graph_chemistry_analysis}\index{graph\_chemistry\_analysis (module)}
This python tool evaluates a particular chemistry findind RAF, SCC and 
saving the multigraph bipartite network and the catalyst-product network

NETWORKX formats :: \href{http://networkx.lanl.gov/reference/readwrite.html}{http://networkx.lanl.gov/reference/readwrite.html}
\index{beta() (in module graph\_chemistry\_analysis)}

\begin{fulllineitems}
\phantomsection\label{graph_chemistry_analysis:graph_chemistry_analysis.beta}\pysiglinewithargsret{\code{graph\_chemistry\_analysis.}\bfcode{beta}}{\emph{a}, \emph{b}, \emph{size=None}}{}
The Beta distribution over \code{{[}0, 1{]}}.

The Beta distribution is a special case of the Dirichlet distribution,
and is related to the Gamma distribution.  It has the probability
distribution function
\begin{gather}
\begin{split}f(x; a,b) = \frac{1}{B(\alpha, \beta)} x^{\alpha - 1}
(1 - x)^{\beta - 1},\end{split}\notag
\end{gather}
where the normalisation, B, is the beta function,
\begin{gather}
\begin{split}B(\alpha, \beta) = \int_0^1 t^{\alpha - 1}
(1 - t)^{\beta - 1} dt.\end{split}\notag
\end{gather}
It is often seen in Bayesian inference and order statistics.
\begin{description}
\item[{a}] \leavevmode{[}float{]}
Alpha, non-negative.

\item[{b}] \leavevmode{[}float{]}
Beta, non-negative.

\item[{size}] \leavevmode{[}tuple of ints, optional{]}
The number of samples to draw.  The output is packed according to
the size given.

\end{description}
\begin{description}
\item[{out}] \leavevmode{[}ndarray{]}
Array of the given shape, containing values drawn from a
Beta distribution.

\end{description}

\end{fulllineitems}

\index{binomial() (in module graph\_chemistry\_analysis)}

\begin{fulllineitems}
\phantomsection\label{graph_chemistry_analysis:graph_chemistry_analysis.binomial}\pysiglinewithargsret{\code{graph\_chemistry\_analysis.}\bfcode{binomial}}{\emph{n}, \emph{p}, \emph{size=None}}{}
Draw samples from a binomial distribution.

Samples are drawn from a Binomial distribution with specified
parameters, n trials and p probability of success where
n an integer \textgreater{}= 0 and p is in the interval {[}0,1{]}. (n may be
input as a float, but it is truncated to an integer in use)
\begin{description}
\item[{n}] \leavevmode{[}float (but truncated to an integer){]}
parameter, \textgreater{}= 0.

\item[{p}] \leavevmode{[}float{]}
parameter, \textgreater{}= 0 and \textless{}=1.

\item[{size}] \leavevmode{[}\{tuple, int\}{]}
Output shape.  If the given shape is, e.g., \code{(m, n, k)}, then
\code{m * n * k} samples are drawn.

\end{description}
\begin{description}
\item[{samples}] \leavevmode{[}\{ndarray, scalar\}{]}
where the values are all integers in  {[}0, n{]}.

\end{description}
\begin{description}
\item[{scipy.stats.distributions.binom}] \leavevmode{[}probability density function,{]}
distribution or cumulative density function, etc.

\end{description}

The probability density for the Binomial distribution is
\begin{gather}
\begin{split}P(N) = \binom{n}{N}p^N(1-p)^{n-N},\end{split}\notag
\end{gather}
where \(n\) is the number of trials, \(p\) is the probability
of success, and \(N\) is the number of successes.

When estimating the standard error of a proportion in a population by
using a random sample, the normal distribution works well unless the
product p*n \textless{}=5, where p = population proportion estimate, and n =
number of samples, in which case the binomial distribution is used
instead. For example, a sample of 15 people shows 4 who are left
handed, and 11 who are right handed. Then p = 4/15 = 27\%. 0.27*15 = 4,
so the binomial distribution should be used in this case.

Draw samples from the distribution:

\begin{Verbatim}[commandchars=\\\{\}]
\PYG{g+gp}{\PYGZgt{}\PYGZgt{}\PYGZgt{} }\PYG{n}{n}\PYG{p}{,} \PYG{n}{p} \PYG{o}{=} \PYG{l+m+mi}{10}\PYG{p}{,} \PYG{o}{.}\PYG{l+m+mi}{5} \PYG{c}{\PYGZsh{} number of trials, probability of each trial}
\PYG{g+gp}{\PYGZgt{}\PYGZgt{}\PYGZgt{} }\PYG{n}{s} \PYG{o}{=} \PYG{n}{np}\PYG{o}{.}\PYG{n}{random}\PYG{o}{.}\PYG{n}{binomial}\PYG{p}{(}\PYG{n}{n}\PYG{p}{,} \PYG{n}{p}\PYG{p}{,} \PYG{l+m+mi}{1000}\PYG{p}{)}
\PYG{g+go}{\PYGZsh{} result of flipping a coin 10 times, tested 1000 times.}
\end{Verbatim}

A real world example. A company drills 9 wild-cat oil exploration
wells, each with an estimated probability of success of 0.1. All nine
wells fail. What is the probability of that happening?

Let's do 20,000 trials of the model, and count the number that
generate zero positive results.

\begin{Verbatim}[commandchars=\\\{\}]
\PYG{g+gp}{\PYGZgt{}\PYGZgt{}\PYGZgt{} }\PYG{n+nb}{sum}\PYG{p}{(}\PYG{n}{np}\PYG{o}{.}\PYG{n}{random}\PYG{o}{.}\PYG{n}{binomial}\PYG{p}{(}\PYG{l+m+mi}{9}\PYG{p}{,}\PYG{l+m+mf}{0.1}\PYG{p}{,}\PYG{l+m+mi}{20000}\PYG{p}{)}\PYG{o}{==}\PYG{l+m+mi}{0}\PYG{p}{)}\PYG{o}{/}\PYG{l+m+mf}{20000.}
\PYG{g+go}{answer = 0.38885, or 38\PYGZpc{}.}
\end{Verbatim}

\end{fulllineitems}

\index{chisquare() (in module graph\_chemistry\_analysis)}

\begin{fulllineitems}
\phantomsection\label{graph_chemistry_analysis:graph_chemistry_analysis.chisquare}\pysiglinewithargsret{\code{graph\_chemistry\_analysis.}\bfcode{chisquare}}{\emph{df}, \emph{size=None}}{}
Draw samples from a chi-square distribution.

When \emph{df} independent random variables, each with standard normal
distributions (mean 0, variance 1), are squared and summed, the
resulting distribution is chi-square (see Notes).  This distribution
is often used in hypothesis testing.
\begin{description}
\item[{df}] \leavevmode{[}int{]}
Number of degrees of freedom.

\item[{size}] \leavevmode{[}tuple of ints, int, optional{]}
Size of the returned array.  By default, a scalar is
returned.

\end{description}
\begin{description}
\item[{output}] \leavevmode{[}ndarray{]}
Samples drawn from the distribution, packed in a \emph{size}-shaped
array.

\end{description}
\begin{description}
\item[{ValueError}] \leavevmode
When \emph{df} \textless{}= 0 or when an inappropriate \emph{size} (e.g. \code{size=-1})
is given.

\end{description}

The variable obtained by summing the squares of \emph{df} independent,
standard normally distributed random variables:
\begin{gather}
\begin{split}Q = \sum_{i=0}^{\mathtt{df}} X^2_i\end{split}\notag
\end{gather}
is chi-square distributed, denoted
\begin{gather}
\begin{split}Q \sim \chi^2_k.\end{split}\notag
\end{gather}
The probability density function of the chi-squared distribution is
\begin{gather}
\begin{split}p(x) = \frac{(1/2)^{k/2}}{\Gamma(k/2)}
x^{k/2 - 1} e^{-x/2},\end{split}\notag
\end{gather}
where \(\Gamma\) is the gamma function,
\begin{gather}
\begin{split}\Gamma(x) = \int_0^{-\infty} t^{x - 1} e^{-t} dt.\end{split}\notag
\end{gather}
\href{http://www.itl.nist.gov/div898/handbook/eda/section3/eda3666.htm}{NIST/SEMATECH e-Handbook of Statistical Methods}

\begin{Verbatim}[commandchars=\\\{\}]
\PYG{g+gp}{\PYGZgt{}\PYGZgt{}\PYGZgt{} }\PYG{n}{np}\PYG{o}{.}\PYG{n}{random}\PYG{o}{.}\PYG{n}{chisquare}\PYG{p}{(}\PYG{l+m+mi}{2}\PYG{p}{,}\PYG{l+m+mi}{4}\PYG{p}{)}
\PYG{g+go}{array([ 1.89920014,  9.00867716,  3.13710533,  5.62318272])}
\end{Verbatim}

\end{fulllineitems}

\index{exponential() (in module graph\_chemistry\_analysis)}

\begin{fulllineitems}
\phantomsection\label{graph_chemistry_analysis:graph_chemistry_analysis.exponential}\pysiglinewithargsret{\code{graph\_chemistry\_analysis.}\bfcode{exponential}}{\emph{scale=1.0}, \emph{size=None}}{}
Exponential distribution.

Its probability density function is
\begin{gather}
\begin{split}f(x; \frac{1}{\beta}) = \frac{1}{\beta} \exp(-\frac{x}{\beta}),\end{split}\notag
\end{gather}
for \code{x \textgreater{} 0} and 0 elsewhere. \(\beta\) is the scale parameter,
which is the inverse of the rate parameter \(\lambda = 1/\beta\).
The rate parameter is an alternative, widely used parameterization
of the exponential distribution {\color{red}\bfseries{}{[}3{]}\_}.

The exponential distribution is a continuous analogue of the
geometric distribution.  It describes many common situations, such as
the size of raindrops measured over many rainstorms {\color{red}\bfseries{}{[}1{]}\_}, or the time
between page requests to Wikipedia {\color{red}\bfseries{}{[}2{]}\_}.
\begin{description}
\item[{scale}] \leavevmode{[}float{]}
The scale parameter, \(\beta = 1/\lambda\).

\item[{size}] \leavevmode{[}tuple of ints{]}
Number of samples to draw.  The output is shaped
according to \emph{size}.

\end{description}

\end{fulllineitems}

\index{f() (in module graph\_chemistry\_analysis)}

\begin{fulllineitems}
\phantomsection\label{graph_chemistry_analysis:graph_chemistry_analysis.f}\pysiglinewithargsret{\code{graph\_chemistry\_analysis.}\bfcode{f}}{\emph{dfnum}, \emph{dfden}, \emph{size=None}}{}
Draw samples from a F distribution.

Samples are drawn from an F distribution with specified parameters,
\emph{dfnum} (degrees of freedom in numerator) and \emph{dfden} (degrees of freedom
in denominator), where both parameters should be greater than zero.

The random variate of the F distribution (also known as the
Fisher distribution) is a continuous probability distribution
that arises in ANOVA tests, and is the ratio of two chi-square
variates.
\begin{description}
\item[{dfnum}] \leavevmode{[}float{]}
Degrees of freedom in numerator. Should be greater than zero.

\item[{dfden}] \leavevmode{[}float{]}
Degrees of freedom in denominator. Should be greater than zero.

\item[{size}] \leavevmode{[}\{tuple, int\}, optional{]}
Output shape.  If the given shape is, e.g., \code{(m, n, k)},
then \code{m * n * k} samples are drawn. By default only one sample
is returned.

\end{description}
\begin{description}
\item[{samples}] \leavevmode{[}\{ndarray, scalar\}{]}
Samples from the Fisher distribution.

\end{description}
\begin{description}
\item[{scipy.stats.distributions.f}] \leavevmode{[}probability density function,{]}
distribution or cumulative density function, etc.

\end{description}

The F statistic is used to compare in-group variances to between-group
variances. Calculating the distribution depends on the sampling, and
so it is a function of the respective degrees of freedom in the
problem.  The variable \emph{dfnum} is the number of samples minus one, the
between-groups degrees of freedom, while \emph{dfden} is the within-groups
degrees of freedom, the sum of the number of samples in each group
minus the number of groups.

An example from Glantz{[}1{]}, pp 47-40.
Two groups, children of diabetics (25 people) and children from people
without diabetes (25 controls). Fasting blood glucose was measured,
case group had a mean value of 86.1, controls had a mean value of
82.2. Standard deviations were 2.09 and 2.49 respectively. Are these
data consistent with the null hypothesis that the parents diabetic
status does not affect their children's blood glucose levels?
Calculating the F statistic from the data gives a value of 36.01.

Draw samples from the distribution:

\begin{Verbatim}[commandchars=\\\{\}]
\PYG{g+gp}{\PYGZgt{}\PYGZgt{}\PYGZgt{} }\PYG{n}{dfnum} \PYG{o}{=} \PYG{l+m+mf}{1.} \PYG{c}{\PYGZsh{} between group degrees of freedom}
\PYG{g+gp}{\PYGZgt{}\PYGZgt{}\PYGZgt{} }\PYG{n}{dfden} \PYG{o}{=} \PYG{l+m+mf}{48.} \PYG{c}{\PYGZsh{} within groups degrees of freedom}
\PYG{g+gp}{\PYGZgt{}\PYGZgt{}\PYGZgt{} }\PYG{n}{s} \PYG{o}{=} \PYG{n}{np}\PYG{o}{.}\PYG{n}{random}\PYG{o}{.}\PYG{n}{f}\PYG{p}{(}\PYG{n}{dfnum}\PYG{p}{,} \PYG{n}{dfden}\PYG{p}{,} \PYG{l+m+mi}{1000}\PYG{p}{)}
\end{Verbatim}

The lower bound for the top 1\% of the samples is :

\begin{Verbatim}[commandchars=\\\{\}]
\PYG{g+gp}{\PYGZgt{}\PYGZgt{}\PYGZgt{} }\PYG{n}{sort}\PYG{p}{(}\PYG{n}{s}\PYG{p}{)}\PYG{p}{[}\PYG{o}{\PYGZhy{}}\PYG{l+m+mi}{10}\PYG{p}{]}
\PYG{g+go}{7.61988120985}
\end{Verbatim}

So there is about a 1\% chance that the F statistic will exceed 7.62,
the measured value is 36, so the null hypothesis is rejected at the 1\%
level.

\end{fulllineitems}

\index{gamma() (in module graph\_chemistry\_analysis)}

\begin{fulllineitems}
\phantomsection\label{graph_chemistry_analysis:graph_chemistry_analysis.gamma}\pysiglinewithargsret{\code{graph\_chemistry\_analysis.}\bfcode{gamma}}{\emph{shape}, \emph{scale=1.0}, \emph{size=None}}{}
Draw samples from a Gamma distribution.

Samples are drawn from a Gamma distribution with specified parameters,
\emph{shape} (sometimes designated ``k'') and \emph{scale} (sometimes designated
``theta''), where both parameters are \textgreater{} 0.
\begin{description}
\item[{shape}] \leavevmode{[}scalar \textgreater{} 0{]}
The shape of the gamma distribution.

\item[{scale}] \leavevmode{[}scalar \textgreater{} 0, optional{]}
The scale of the gamma distribution.  Default is equal to 1.

\item[{size}] \leavevmode{[}shape\_tuple, optional{]}
Output shape.  If the given shape is, e.g., \code{(m, n, k)}, then
\code{m * n * k} samples are drawn.

\end{description}
\begin{description}
\item[{out}] \leavevmode{[}ndarray, float{]}
Returns one sample unless \emph{size} parameter is specified.

\end{description}
\begin{description}
\item[{scipy.stats.distributions.gamma}] \leavevmode{[}probability density function,{]}
distribution or cumulative density function, etc.

\end{description}

The probability density for the Gamma distribution is
\begin{gather}
\begin{split}p(x) = x^{k-1}\frac{e^{-x/\theta}}{\theta^k\Gamma(k)},\end{split}\notag
\end{gather}
where \(k\) is the shape and \(\theta\) the scale,
and \(\Gamma\) is the Gamma function.

The Gamma distribution is often used to model the times to failure of
electronic components, and arises naturally in processes for which the
waiting times between Poisson distributed events are relevant.

Draw samples from the distribution:

\begin{Verbatim}[commandchars=\\\{\}]
\PYG{g+gp}{\PYGZgt{}\PYGZgt{}\PYGZgt{} }\PYG{n}{shape}\PYG{p}{,} \PYG{n}{scale} \PYG{o}{=} \PYG{l+m+mf}{2.}\PYG{p}{,} \PYG{l+m+mf}{2.} \PYG{c}{\PYGZsh{} mean and dispersion}
\PYG{g+gp}{\PYGZgt{}\PYGZgt{}\PYGZgt{} }\PYG{n}{s} \PYG{o}{=} \PYG{n}{np}\PYG{o}{.}\PYG{n}{random}\PYG{o}{.}\PYG{n}{gamma}\PYG{p}{(}\PYG{n}{shape}\PYG{p}{,} \PYG{n}{scale}\PYG{p}{,} \PYG{l+m+mi}{1000}\PYG{p}{)}
\end{Verbatim}

Display the histogram of the samples, along with
the probability density function:

\begin{Verbatim}[commandchars=\\\{\}]
\PYG{g+gp}{\PYGZgt{}\PYGZgt{}\PYGZgt{} }\PYG{k+kn}{import} \PYG{n+nn}{matplotlib.pyplot} \PYG{k+kn}{as} \PYG{n+nn}{plt}
\PYG{g+gp}{\PYGZgt{}\PYGZgt{}\PYGZgt{} }\PYG{k+kn}{import} \PYG{n+nn}{scipy.special} \PYG{k+kn}{as} \PYG{n+nn}{sps}
\PYG{g+gp}{\PYGZgt{}\PYGZgt{}\PYGZgt{} }\PYG{n}{count}\PYG{p}{,} \PYG{n}{bins}\PYG{p}{,} \PYG{n}{ignored} \PYG{o}{=} \PYG{n}{plt}\PYG{o}{.}\PYG{n}{hist}\PYG{p}{(}\PYG{n}{s}\PYG{p}{,} \PYG{l+m+mi}{50}\PYG{p}{,} \PYG{n}{normed}\PYG{o}{=}\PYG{n+nb+bp}{True}\PYG{p}{)}
\PYG{g+gp}{\PYGZgt{}\PYGZgt{}\PYGZgt{} }\PYG{n}{y} \PYG{o}{=} \PYG{n}{bins}\PYG{o}{*}\PYG{o}{*}\PYG{p}{(}\PYG{n}{shape}\PYG{o}{\PYGZhy{}}\PYG{l+m+mi}{1}\PYG{p}{)}\PYG{o}{*}\PYG{p}{(}\PYG{n}{np}\PYG{o}{.}\PYG{n}{exp}\PYG{p}{(}\PYG{o}{\PYGZhy{}}\PYG{n}{bins}\PYG{o}{/}\PYG{n}{scale}\PYG{p}{)} \PYG{o}{/}
\PYG{g+gp}{... }                     \PYG{p}{(}\PYG{n}{sps}\PYG{o}{.}\PYG{n}{gamma}\PYG{p}{(}\PYG{n}{shape}\PYG{p}{)}\PYG{o}{*}\PYG{n}{scale}\PYG{o}{*}\PYG{o}{*}\PYG{n}{shape}\PYG{p}{)}\PYG{p}{)}
\PYG{g+gp}{\PYGZgt{}\PYGZgt{}\PYGZgt{} }\PYG{n}{plt}\PYG{o}{.}\PYG{n}{plot}\PYG{p}{(}\PYG{n}{bins}\PYG{p}{,} \PYG{n}{y}\PYG{p}{,} \PYG{n}{linewidth}\PYG{o}{=}\PYG{l+m+mi}{2}\PYG{p}{,} \PYG{n}{color}\PYG{o}{=}\PYG{l+s}{\PYGZsq{}}\PYG{l+s}{r}\PYG{l+s}{\PYGZsq{}}\PYG{p}{)}
\PYG{g+gp}{\PYGZgt{}\PYGZgt{}\PYGZgt{} }\PYG{n}{plt}\PYG{o}{.}\PYG{n}{show}\PYG{p}{(}\PYG{p}{)}
\end{Verbatim}

\end{fulllineitems}

\index{geometric() (in module graph\_chemistry\_analysis)}

\begin{fulllineitems}
\phantomsection\label{graph_chemistry_analysis:graph_chemistry_analysis.geometric}\pysiglinewithargsret{\code{graph\_chemistry\_analysis.}\bfcode{geometric}}{\emph{p}, \emph{size=None}}{}
Draw samples from the geometric distribution.

Bernoulli trials are experiments with one of two outcomes:
success or failure (an example of such an experiment is flipping
a coin).  The geometric distribution models the number of trials
that must be run in order to achieve success.  It is therefore
supported on the positive integers, \code{k = 1, 2, ...}.

The probability mass function of the geometric distribution is
\begin{gather}
\begin{split}f(k) = (1 - p)^{k - 1} p\end{split}\notag
\end{gather}
where \emph{p} is the probability of success of an individual trial.
\begin{description}
\item[{p}] \leavevmode{[}float{]}
The probability of success of an individual trial.

\item[{size}] \leavevmode{[}tuple of ints{]}
Number of values to draw from the distribution.  The output
is shaped according to \emph{size}.

\end{description}
\begin{description}
\item[{out}] \leavevmode{[}ndarray{]}
Samples from the geometric distribution, shaped according to
\emph{size}.

\end{description}

Draw ten thousand values from the geometric distribution,
with the probability of an individual success equal to 0.35:

\begin{Verbatim}[commandchars=\\\{\}]
\PYG{g+gp}{\PYGZgt{}\PYGZgt{}\PYGZgt{} }\PYG{n}{z} \PYG{o}{=} \PYG{n}{np}\PYG{o}{.}\PYG{n}{random}\PYG{o}{.}\PYG{n}{geometric}\PYG{p}{(}\PYG{n}{p}\PYG{o}{=}\PYG{l+m+mf}{0.35}\PYG{p}{,} \PYG{n}{size}\PYG{o}{=}\PYG{l+m+mi}{10000}\PYG{p}{)}
\end{Verbatim}

How many trials succeeded after a single run?

\begin{Verbatim}[commandchars=\\\{\}]
\PYG{g+gp}{\PYGZgt{}\PYGZgt{}\PYGZgt{} }\PYG{p}{(}\PYG{n}{z} \PYG{o}{==} \PYG{l+m+mi}{1}\PYG{p}{)}\PYG{o}{.}\PYG{n}{sum}\PYG{p}{(}\PYG{p}{)} \PYG{o}{/} \PYG{l+m+mf}{10000.}
\PYG{g+go}{0.34889999999999999 \PYGZsh{}random}
\end{Verbatim}

\end{fulllineitems}

\index{get\_state() (in module graph\_chemistry\_analysis)}

\begin{fulllineitems}
\phantomsection\label{graph_chemistry_analysis:graph_chemistry_analysis.get_state}\pysiglinewithargsret{\code{graph\_chemistry\_analysis.}\bfcode{get\_state}}{}{}
Return a tuple representing the internal state of the generator.

For more details, see \emph{set\_state}.
\begin{description}
\item[{out}] \leavevmode{[}tuple(str, ndarray of 624 uints, int, int, float){]}
The returned tuple has the following items:
\begin{enumerate}
\item {} 
the string `MT19937'.

\item {} 
a 1-D array of 624 unsigned integer keys.

\item {} 
an integer \code{pos}.

\item {} 
an integer \code{has\_gauss}.

\item {} 
a float \code{cached\_gaussian}.

\end{enumerate}

\end{description}

set\_state

\emph{set\_state} and \emph{get\_state} are not needed to work with any of the
random distributions in NumPy. If the internal state is manually altered,
the user should know exactly what he/she is doing.

\end{fulllineitems}

\index{gumbel() (in module graph\_chemistry\_analysis)}

\begin{fulllineitems}
\phantomsection\label{graph_chemistry_analysis:graph_chemistry_analysis.gumbel}\pysiglinewithargsret{\code{graph\_chemistry\_analysis.}\bfcode{gumbel}}{\emph{loc=0.0}, \emph{scale=1.0}, \emph{size=None}}{}
Gumbel distribution.

Draw samples from a Gumbel distribution with specified location and scale.
For more information on the Gumbel distribution, see Notes and References
below.
\begin{description}
\item[{loc}] \leavevmode{[}float{]}
The location of the mode of the distribution.

\item[{scale}] \leavevmode{[}float{]}
The scale parameter of the distribution.

\item[{size}] \leavevmode{[}tuple of ints{]}
Output shape.  If the given shape is, e.g., \code{(m, n, k)}, then
\code{m * n * k} samples are drawn.

\end{description}
\begin{description}
\item[{out}] \leavevmode{[}ndarray{]}
The samples

\end{description}

scipy.stats.gumbel\_l
scipy.stats.gumbel\_r
scipy.stats.genextreme
\begin{quote}

probability density function, distribution, or cumulative density
function, etc. for each of the above
\end{quote}

weibull

The Gumbel (or Smallest Extreme Value (SEV) or the Smallest Extreme Value
Type I) distribution is one of a class of Generalized Extreme Value (GEV)
distributions used in modeling extreme value problems.  The Gumbel is a
special case of the Extreme Value Type I distribution for maximums from
distributions with ``exponential-like'' tails.

The probability density for the Gumbel distribution is
\begin{gather}
\begin{split}p(x) = \frac{e^{-(x - \mu)/ \beta}}{\beta} e^{ -e^{-(x - \mu)/
\beta}},\end{split}\notag
\end{gather}
where \(\mu\) is the mode, a location parameter, and \(\beta\) is
the scale parameter.

The Gumbel (named for German mathematician Emil Julius Gumbel) was used
very early in the hydrology literature, for modeling the occurrence of
flood events. It is also used for modeling maximum wind speed and rainfall
rates.  It is a ``fat-tailed'' distribution - the probability of an event in
the tail of the distribution is larger than if one used a Gaussian, hence
the surprisingly frequent occurrence of 100-year floods. Floods were
initially modeled as a Gaussian process, which underestimated the frequency
of extreme events.

It is one of a class of extreme value distributions, the Generalized
Extreme Value (GEV) distributions, which also includes the Weibull and
Frechet.

The function has a mean of \(\mu + 0.57721\beta\) and a variance of
\(\frac{\pi^2}{6}\beta^2\).

Gumbel, E. J., \emph{Statistics of Extremes}, New York: Columbia University
Press, 1958.

Reiss, R.-D. and Thomas, M., \emph{Statistical Analysis of Extreme Values from
Insurance, Finance, Hydrology and Other Fields}, Basel: Birkhauser Verlag,
2001.

Draw samples from the distribution:

\begin{Verbatim}[commandchars=\\\{\}]
\PYG{g+gp}{\PYGZgt{}\PYGZgt{}\PYGZgt{} }\PYG{n}{mu}\PYG{p}{,} \PYG{n}{beta} \PYG{o}{=} \PYG{l+m+mi}{0}\PYG{p}{,} \PYG{l+m+mf}{0.1} \PYG{c}{\PYGZsh{} location and scale}
\PYG{g+gp}{\PYGZgt{}\PYGZgt{}\PYGZgt{} }\PYG{n}{s} \PYG{o}{=} \PYG{n}{np}\PYG{o}{.}\PYG{n}{random}\PYG{o}{.}\PYG{n}{gumbel}\PYG{p}{(}\PYG{n}{mu}\PYG{p}{,} \PYG{n}{beta}\PYG{p}{,} \PYG{l+m+mi}{1000}\PYG{p}{)}
\end{Verbatim}

Display the histogram of the samples, along with
the probability density function:

\begin{Verbatim}[commandchars=\\\{\}]
\PYG{g+gp}{\PYGZgt{}\PYGZgt{}\PYGZgt{} }\PYG{k+kn}{import} \PYG{n+nn}{matplotlib.pyplot} \PYG{k+kn}{as} \PYG{n+nn}{plt}
\PYG{g+gp}{\PYGZgt{}\PYGZgt{}\PYGZgt{} }\PYG{n}{count}\PYG{p}{,} \PYG{n}{bins}\PYG{p}{,} \PYG{n}{ignored} \PYG{o}{=} \PYG{n}{plt}\PYG{o}{.}\PYG{n}{hist}\PYG{p}{(}\PYG{n}{s}\PYG{p}{,} \PYG{l+m+mi}{30}\PYG{p}{,} \PYG{n}{normed}\PYG{o}{=}\PYG{n+nb+bp}{True}\PYG{p}{)}
\PYG{g+gp}{\PYGZgt{}\PYGZgt{}\PYGZgt{} }\PYG{n}{plt}\PYG{o}{.}\PYG{n}{plot}\PYG{p}{(}\PYG{n}{bins}\PYG{p}{,} \PYG{p}{(}\PYG{l+m+mi}{1}\PYG{o}{/}\PYG{n}{beta}\PYG{p}{)}\PYG{o}{*}\PYG{n}{np}\PYG{o}{.}\PYG{n}{exp}\PYG{p}{(}\PYG{o}{\PYGZhy{}}\PYG{p}{(}\PYG{n}{bins} \PYG{o}{\PYGZhy{}} \PYG{n}{mu}\PYG{p}{)}\PYG{o}{/}\PYG{n}{beta}\PYG{p}{)}
\PYG{g+gp}{... }         \PYG{o}{*} \PYG{n}{np}\PYG{o}{.}\PYG{n}{exp}\PYG{p}{(} \PYG{o}{\PYGZhy{}}\PYG{n}{np}\PYG{o}{.}\PYG{n}{exp}\PYG{p}{(} \PYG{o}{\PYGZhy{}}\PYG{p}{(}\PYG{n}{bins} \PYG{o}{\PYGZhy{}} \PYG{n}{mu}\PYG{p}{)} \PYG{o}{/}\PYG{n}{beta}\PYG{p}{)} \PYG{p}{)}\PYG{p}{,}
\PYG{g+gp}{... }         \PYG{n}{linewidth}\PYG{o}{=}\PYG{l+m+mi}{2}\PYG{p}{,} \PYG{n}{color}\PYG{o}{=}\PYG{l+s}{\PYGZsq{}}\PYG{l+s}{r}\PYG{l+s}{\PYGZsq{}}\PYG{p}{)}
\PYG{g+gp}{\PYGZgt{}\PYGZgt{}\PYGZgt{} }\PYG{n}{plt}\PYG{o}{.}\PYG{n}{show}\PYG{p}{(}\PYG{p}{)}
\end{Verbatim}

Show how an extreme value distribution can arise from a Gaussian process
and compare to a Gaussian:

\begin{Verbatim}[commandchars=\\\{\}]
\PYG{g+gp}{\PYGZgt{}\PYGZgt{}\PYGZgt{} }\PYG{n}{means} \PYG{o}{=} \PYG{p}{[}\PYG{p}{]}
\PYG{g+gp}{\PYGZgt{}\PYGZgt{}\PYGZgt{} }\PYG{n}{maxima} \PYG{o}{=} \PYG{p}{[}\PYG{p}{]}
\PYG{g+gp}{\PYGZgt{}\PYGZgt{}\PYGZgt{} }\PYG{k}{for} \PYG{n}{i} \PYG{o+ow}{in} \PYG{n+nb}{range}\PYG{p}{(}\PYG{l+m+mi}{0}\PYG{p}{,}\PYG{l+m+mi}{1000}\PYG{p}{)} \PYG{p}{:}
\PYG{g+gp}{... }   \PYG{n}{a} \PYG{o}{=} \PYG{n}{np}\PYG{o}{.}\PYG{n}{random}\PYG{o}{.}\PYG{n}{normal}\PYG{p}{(}\PYG{n}{mu}\PYG{p}{,} \PYG{n}{beta}\PYG{p}{,} \PYG{l+m+mi}{1000}\PYG{p}{)}
\PYG{g+gp}{... }   \PYG{n}{means}\PYG{o}{.}\PYG{n}{append}\PYG{p}{(}\PYG{n}{a}\PYG{o}{.}\PYG{n}{mean}\PYG{p}{(}\PYG{p}{)}\PYG{p}{)}
\PYG{g+gp}{... }   \PYG{n}{maxima}\PYG{o}{.}\PYG{n}{append}\PYG{p}{(}\PYG{n}{a}\PYG{o}{.}\PYG{n}{max}\PYG{p}{(}\PYG{p}{)}\PYG{p}{)}
\PYG{g+gp}{\PYGZgt{}\PYGZgt{}\PYGZgt{} }\PYG{n}{count}\PYG{p}{,} \PYG{n}{bins}\PYG{p}{,} \PYG{n}{ignored} \PYG{o}{=} \PYG{n}{plt}\PYG{o}{.}\PYG{n}{hist}\PYG{p}{(}\PYG{n}{maxima}\PYG{p}{,} \PYG{l+m+mi}{30}\PYG{p}{,} \PYG{n}{normed}\PYG{o}{=}\PYG{n+nb+bp}{True}\PYG{p}{)}
\PYG{g+gp}{\PYGZgt{}\PYGZgt{}\PYGZgt{} }\PYG{n}{beta} \PYG{o}{=} \PYG{n}{np}\PYG{o}{.}\PYG{n}{std}\PYG{p}{(}\PYG{n}{maxima}\PYG{p}{)}\PYG{o}{*}\PYG{n}{np}\PYG{o}{.}\PYG{n}{pi}\PYG{o}{/}\PYG{n}{np}\PYG{o}{.}\PYG{n}{sqrt}\PYG{p}{(}\PYG{l+m+mi}{6}\PYG{p}{)}
\PYG{g+gp}{\PYGZgt{}\PYGZgt{}\PYGZgt{} }\PYG{n}{mu} \PYG{o}{=} \PYG{n}{np}\PYG{o}{.}\PYG{n}{mean}\PYG{p}{(}\PYG{n}{maxima}\PYG{p}{)} \PYG{o}{\PYGZhy{}} \PYG{l+m+mf}{0.57721}\PYG{o}{*}\PYG{n}{beta}
\PYG{g+gp}{\PYGZgt{}\PYGZgt{}\PYGZgt{} }\PYG{n}{plt}\PYG{o}{.}\PYG{n}{plot}\PYG{p}{(}\PYG{n}{bins}\PYG{p}{,} \PYG{p}{(}\PYG{l+m+mi}{1}\PYG{o}{/}\PYG{n}{beta}\PYG{p}{)}\PYG{o}{*}\PYG{n}{np}\PYG{o}{.}\PYG{n}{exp}\PYG{p}{(}\PYG{o}{\PYGZhy{}}\PYG{p}{(}\PYG{n}{bins} \PYG{o}{\PYGZhy{}} \PYG{n}{mu}\PYG{p}{)}\PYG{o}{/}\PYG{n}{beta}\PYG{p}{)}
\PYG{g+gp}{... }         \PYG{o}{*} \PYG{n}{np}\PYG{o}{.}\PYG{n}{exp}\PYG{p}{(}\PYG{o}{\PYGZhy{}}\PYG{n}{np}\PYG{o}{.}\PYG{n}{exp}\PYG{p}{(}\PYG{o}{\PYGZhy{}}\PYG{p}{(}\PYG{n}{bins} \PYG{o}{\PYGZhy{}} \PYG{n}{mu}\PYG{p}{)}\PYG{o}{/}\PYG{n}{beta}\PYG{p}{)}\PYG{p}{)}\PYG{p}{,}
\PYG{g+gp}{... }         \PYG{n}{linewidth}\PYG{o}{=}\PYG{l+m+mi}{2}\PYG{p}{,} \PYG{n}{color}\PYG{o}{=}\PYG{l+s}{\PYGZsq{}}\PYG{l+s}{r}\PYG{l+s}{\PYGZsq{}}\PYG{p}{)}
\PYG{g+gp}{\PYGZgt{}\PYGZgt{}\PYGZgt{} }\PYG{n}{plt}\PYG{o}{.}\PYG{n}{plot}\PYG{p}{(}\PYG{n}{bins}\PYG{p}{,} \PYG{l+m+mi}{1}\PYG{o}{/}\PYG{p}{(}\PYG{n}{beta} \PYG{o}{*} \PYG{n}{np}\PYG{o}{.}\PYG{n}{sqrt}\PYG{p}{(}\PYG{l+m+mi}{2} \PYG{o}{*} \PYG{n}{np}\PYG{o}{.}\PYG{n}{pi}\PYG{p}{)}\PYG{p}{)}
\PYG{g+gp}{... }         \PYG{o}{*} \PYG{n}{np}\PYG{o}{.}\PYG{n}{exp}\PYG{p}{(}\PYG{o}{\PYGZhy{}}\PYG{p}{(}\PYG{n}{bins} \PYG{o}{\PYGZhy{}} \PYG{n}{mu}\PYG{p}{)}\PYG{o}{*}\PYG{o}{*}\PYG{l+m+mi}{2} \PYG{o}{/} \PYG{p}{(}\PYG{l+m+mi}{2} \PYG{o}{*} \PYG{n}{beta}\PYG{o}{*}\PYG{o}{*}\PYG{l+m+mi}{2}\PYG{p}{)}\PYG{p}{)}\PYG{p}{,}
\PYG{g+gp}{... }         \PYG{n}{linewidth}\PYG{o}{=}\PYG{l+m+mi}{2}\PYG{p}{,} \PYG{n}{color}\PYG{o}{=}\PYG{l+s}{\PYGZsq{}}\PYG{l+s}{g}\PYG{l+s}{\PYGZsq{}}\PYG{p}{)}
\PYG{g+gp}{\PYGZgt{}\PYGZgt{}\PYGZgt{} }\PYG{n}{plt}\PYG{o}{.}\PYG{n}{show}\PYG{p}{(}\PYG{p}{)}
\end{Verbatim}

\end{fulllineitems}

\index{hypergeometric() (in module graph\_chemistry\_analysis)}

\begin{fulllineitems}
\phantomsection\label{graph_chemistry_analysis:graph_chemistry_analysis.hypergeometric}\pysiglinewithargsret{\code{graph\_chemistry\_analysis.}\bfcode{hypergeometric}}{\emph{ngood}, \emph{nbad}, \emph{nsample}, \emph{size=None}}{}
Draw samples from a Hypergeometric distribution.

Samples are drawn from a Hypergeometric distribution with specified
parameters, ngood (ways to make a good selection), nbad (ways to make
a bad selection), and nsample = number of items sampled, which is less
than or equal to the sum ngood + nbad.
\begin{description}
\item[{ngood}] \leavevmode{[}int or array\_like{]}
Number of ways to make a good selection.  Must be nonnegative.

\item[{nbad}] \leavevmode{[}int or array\_like{]}
Number of ways to make a bad selection.  Must be nonnegative.

\item[{nsample}] \leavevmode{[}int or array\_like{]}
Number of items sampled.  Must be at least 1 and at most
\code{ngood + nbad}.

\item[{size}] \leavevmode{[}int or tuple of int{]}
Output shape.  If the given shape is, e.g., \code{(m, n, k)}, then
\code{m * n * k} samples are drawn.

\end{description}
\begin{description}
\item[{samples}] \leavevmode{[}ndarray or scalar{]}
The values are all integers in  {[}0, n{]}.

\end{description}
\begin{description}
\item[{scipy.stats.distributions.hypergeom}] \leavevmode{[}probability density function,{]}
distribution or cumulative density function, etc.

\end{description}

The probability density for the Hypergeometric distribution is
\begin{gather}
\begin{split}P(x) = \frac{\binom{m}{n}\binom{N-m}{n-x}}{\binom{N}{n}},\end{split}\notag
\end{gather}
where \(0 \le x \le m\) and \(n+m-N \le x \le n\)

for P(x) the probability of x successes, n = ngood, m = nbad, and
N = number of samples.

Consider an urn with black and white marbles in it, ngood of them
black and nbad are white. If you draw nsample balls without
replacement, then the Hypergeometric distribution describes the
distribution of black balls in the drawn sample.

Note that this distribution is very similar to the Binomial
distribution, except that in this case, samples are drawn without
replacement, whereas in the Binomial case samples are drawn with
replacement (or the sample space is infinite). As the sample space
becomes large, this distribution approaches the Binomial.

Draw samples from the distribution:

\begin{Verbatim}[commandchars=\\\{\}]
\PYG{g+gp}{\PYGZgt{}\PYGZgt{}\PYGZgt{} }\PYG{n}{ngood}\PYG{p}{,} \PYG{n}{nbad}\PYG{p}{,} \PYG{n}{nsamp} \PYG{o}{=} \PYG{l+m+mi}{100}\PYG{p}{,} \PYG{l+m+mi}{2}\PYG{p}{,} \PYG{l+m+mi}{10}
\PYG{g+go}{\PYGZsh{} number of good, number of bad, and number of samples}
\PYG{g+gp}{\PYGZgt{}\PYGZgt{}\PYGZgt{} }\PYG{n}{s} \PYG{o}{=} \PYG{n}{np}\PYG{o}{.}\PYG{n}{random}\PYG{o}{.}\PYG{n}{hypergeometric}\PYG{p}{(}\PYG{n}{ngood}\PYG{p}{,} \PYG{n}{nbad}\PYG{p}{,} \PYG{n}{nsamp}\PYG{p}{,} \PYG{l+m+mi}{1000}\PYG{p}{)}
\PYG{g+gp}{\PYGZgt{}\PYGZgt{}\PYGZgt{} }\PYG{n}{hist}\PYG{p}{(}\PYG{n}{s}\PYG{p}{)}
\PYG{g+go}{\PYGZsh{}   note that it is very unlikely to grab both bad items}
\end{Verbatim}

Suppose you have an urn with 15 white and 15 black marbles.
If you pull 15 marbles at random, how likely is it that
12 or more of them are one color?

\begin{Verbatim}[commandchars=\\\{\}]
\PYG{g+gp}{\PYGZgt{}\PYGZgt{}\PYGZgt{} }\PYG{n}{s} \PYG{o}{=} \PYG{n}{np}\PYG{o}{.}\PYG{n}{random}\PYG{o}{.}\PYG{n}{hypergeometric}\PYG{p}{(}\PYG{l+m+mi}{15}\PYG{p}{,} \PYG{l+m+mi}{15}\PYG{p}{,} \PYG{l+m+mi}{15}\PYG{p}{,} \PYG{l+m+mi}{100000}\PYG{p}{)}
\PYG{g+gp}{\PYGZgt{}\PYGZgt{}\PYGZgt{} }\PYG{n+nb}{sum}\PYG{p}{(}\PYG{n}{s}\PYG{o}{\PYGZgt{}}\PYG{o}{=}\PYG{l+m+mi}{12}\PYG{p}{)}\PYG{o}{/}\PYG{l+m+mf}{100000.} \PYG{o}{+} \PYG{n+nb}{sum}\PYG{p}{(}\PYG{n}{s}\PYG{o}{\PYGZlt{}}\PYG{o}{=}\PYG{l+m+mi}{3}\PYG{p}{)}\PYG{o}{/}\PYG{l+m+mf}{100000.}
\PYG{g+go}{\PYGZsh{}   answer = 0.003 ... pretty unlikely!}
\end{Verbatim}

\end{fulllineitems}

\index{laplace() (in module graph\_chemistry\_analysis)}

\begin{fulllineitems}
\phantomsection\label{graph_chemistry_analysis:graph_chemistry_analysis.laplace}\pysiglinewithargsret{\code{graph\_chemistry\_analysis.}\bfcode{laplace}}{\emph{loc=0.0}, \emph{scale=1.0}, \emph{size=None}}{}
Draw samples from the Laplace or double exponential distribution with
specified location (or mean) and scale (decay).

The Laplace distribution is similar to the Gaussian/normal distribution,
but is sharper at the peak and has fatter tails. It represents the
difference between two independent, identically distributed exponential
random variables.
\begin{description}
\item[{loc}] \leavevmode{[}float{]}
The position, \(\mu\), of the distribution peak.

\item[{scale}] \leavevmode{[}float{]}
\(\lambda\), the exponential decay.

\end{description}

It has the probability density function
\begin{gather}
\begin{split}f(x; \mu, \lambda) = \frac{1}{2\lambda}
\exp\left(-\frac{|x - \mu|}{\lambda}\right).\end{split}\notag
\end{gather}
The first law of Laplace, from 1774, states that the frequency of an error
can be expressed as an exponential function of the absolute magnitude of
the error, which leads to the Laplace distribution. For many problems in
Economics and Health sciences, this distribution seems to model the data
better than the standard Gaussian distribution

Draw samples from the distribution

\begin{Verbatim}[commandchars=\\\{\}]
\PYG{g+gp}{\PYGZgt{}\PYGZgt{}\PYGZgt{} }\PYG{n}{loc}\PYG{p}{,} \PYG{n}{scale} \PYG{o}{=} \PYG{l+m+mf}{0.}\PYG{p}{,} \PYG{l+m+mf}{1.}
\PYG{g+gp}{\PYGZgt{}\PYGZgt{}\PYGZgt{} }\PYG{n}{s} \PYG{o}{=} \PYG{n}{np}\PYG{o}{.}\PYG{n}{random}\PYG{o}{.}\PYG{n}{laplace}\PYG{p}{(}\PYG{n}{loc}\PYG{p}{,} \PYG{n}{scale}\PYG{p}{,} \PYG{l+m+mi}{1000}\PYG{p}{)}
\end{Verbatim}

Display the histogram of the samples, along with
the probability density function:

\begin{Verbatim}[commandchars=\\\{\}]
\PYG{g+gp}{\PYGZgt{}\PYGZgt{}\PYGZgt{} }\PYG{k+kn}{import} \PYG{n+nn}{matplotlib.pyplot} \PYG{k+kn}{as} \PYG{n+nn}{plt}
\PYG{g+gp}{\PYGZgt{}\PYGZgt{}\PYGZgt{} }\PYG{n}{count}\PYG{p}{,} \PYG{n}{bins}\PYG{p}{,} \PYG{n}{ignored} \PYG{o}{=} \PYG{n}{plt}\PYG{o}{.}\PYG{n}{hist}\PYG{p}{(}\PYG{n}{s}\PYG{p}{,} \PYG{l+m+mi}{30}\PYG{p}{,} \PYG{n}{normed}\PYG{o}{=}\PYG{n+nb+bp}{True}\PYG{p}{)}
\PYG{g+gp}{\PYGZgt{}\PYGZgt{}\PYGZgt{} }\PYG{n}{x} \PYG{o}{=} \PYG{n}{np}\PYG{o}{.}\PYG{n}{arange}\PYG{p}{(}\PYG{o}{\PYGZhy{}}\PYG{l+m+mf}{8.}\PYG{p}{,} \PYG{l+m+mf}{8.}\PYG{p}{,} \PYG{o}{.}\PYG{l+m+mo}{01}\PYG{p}{)}
\PYG{g+gp}{\PYGZgt{}\PYGZgt{}\PYGZgt{} }\PYG{n}{pdf} \PYG{o}{=} \PYG{n}{np}\PYG{o}{.}\PYG{n}{exp}\PYG{p}{(}\PYG{o}{\PYGZhy{}}\PYG{n+nb}{abs}\PYG{p}{(}\PYG{n}{x}\PYG{o}{\PYGZhy{}}\PYG{n}{loc}\PYG{o}{/}\PYG{n}{scale}\PYG{p}{)}\PYG{p}{)}\PYG{o}{/}\PYG{p}{(}\PYG{l+m+mf}{2.}\PYG{o}{*}\PYG{n}{scale}\PYG{p}{)}
\PYG{g+gp}{\PYGZgt{}\PYGZgt{}\PYGZgt{} }\PYG{n}{plt}\PYG{o}{.}\PYG{n}{plot}\PYG{p}{(}\PYG{n}{x}\PYG{p}{,} \PYG{n}{pdf}\PYG{p}{)}
\end{Verbatim}

Plot Gaussian for comparison:

\begin{Verbatim}[commandchars=\\\{\}]
\PYG{g+gp}{\PYGZgt{}\PYGZgt{}\PYGZgt{} }\PYG{n}{g} \PYG{o}{=} \PYG{p}{(}\PYG{l+m+mi}{1}\PYG{o}{/}\PYG{p}{(}\PYG{n}{scale} \PYG{o}{*} \PYG{n}{np}\PYG{o}{.}\PYG{n}{sqrt}\PYG{p}{(}\PYG{l+m+mi}{2} \PYG{o}{*} \PYG{n}{np}\PYG{o}{.}\PYG{n}{pi}\PYG{p}{)}\PYG{p}{)} \PYG{o}{*} 
\PYG{g+gp}{... }     \PYG{n}{np}\PYG{o}{.}\PYG{n}{exp}\PYG{p}{(} \PYG{o}{\PYGZhy{}} \PYG{p}{(}\PYG{n}{x} \PYG{o}{\PYGZhy{}} \PYG{n}{loc}\PYG{p}{)}\PYG{o}{*}\PYG{o}{*}\PYG{l+m+mi}{2} \PYG{o}{/} \PYG{p}{(}\PYG{l+m+mi}{2} \PYG{o}{*} \PYG{n}{scale}\PYG{o}{*}\PYG{o}{*}\PYG{l+m+mi}{2}\PYG{p}{)} \PYG{p}{)}\PYG{p}{)}
\PYG{g+gp}{\PYGZgt{}\PYGZgt{}\PYGZgt{} }\PYG{n}{plt}\PYG{o}{.}\PYG{n}{plot}\PYG{p}{(}\PYG{n}{x}\PYG{p}{,}\PYG{n}{g}\PYG{p}{)}
\end{Verbatim}

\end{fulllineitems}

\index{logistic() (in module graph\_chemistry\_analysis)}

\begin{fulllineitems}
\phantomsection\label{graph_chemistry_analysis:graph_chemistry_analysis.logistic}\pysiglinewithargsret{\code{graph\_chemistry\_analysis.}\bfcode{logistic}}{\emph{loc=0.0}, \emph{scale=1.0}, \emph{size=None}}{}
Draw samples from a Logistic distribution.

Samples are drawn from a Logistic distribution with specified
parameters, loc (location or mean, also median), and scale (\textgreater{}0).

loc : float

scale : float \textgreater{} 0.
\begin{description}
\item[{size}] \leavevmode{[}\{tuple, int\}{]}
Output shape.  If the given shape is, e.g., \code{(m, n, k)}, then
\code{m * n * k} samples are drawn.

\end{description}
\begin{description}
\item[{samples}] \leavevmode{[}\{ndarray, scalar\}{]}
where the values are all integers in  {[}0, n{]}.

\end{description}
\begin{description}
\item[{scipy.stats.distributions.logistic}] \leavevmode{[}probability density function,{]}
distribution or cumulative density function, etc.

\end{description}

The probability density for the Logistic distribution is
\begin{gather}
\begin{split}P(x) = P(x) = \frac{e^{-(x-\mu)/s}}{s(1+e^{-(x-\mu)/s})^2},\end{split}\notag
\end{gather}
where \(\mu\) = location and \(s\) = scale.

The Logistic distribution is used in Extreme Value problems where it
can act as a mixture of Gumbel distributions, in Epidemiology, and by
the World Chess Federation (FIDE) where it is used in the Elo ranking
system, assuming the performance of each player is a logistically
distributed random variable.

Draw samples from the distribution:

\begin{Verbatim}[commandchars=\\\{\}]
\PYG{g+gp}{\PYGZgt{}\PYGZgt{}\PYGZgt{} }\PYG{n}{loc}\PYG{p}{,} \PYG{n}{scale} \PYG{o}{=} \PYG{l+m+mi}{10}\PYG{p}{,} \PYG{l+m+mi}{1}
\PYG{g+gp}{\PYGZgt{}\PYGZgt{}\PYGZgt{} }\PYG{n}{s} \PYG{o}{=} \PYG{n}{np}\PYG{o}{.}\PYG{n}{random}\PYG{o}{.}\PYG{n}{logistic}\PYG{p}{(}\PYG{n}{loc}\PYG{p}{,} \PYG{n}{scale}\PYG{p}{,} \PYG{l+m+mi}{10000}\PYG{p}{)}
\PYG{g+gp}{\PYGZgt{}\PYGZgt{}\PYGZgt{} }\PYG{n}{count}\PYG{p}{,} \PYG{n}{bins}\PYG{p}{,} \PYG{n}{ignored} \PYG{o}{=} \PYG{n}{plt}\PYG{o}{.}\PYG{n}{hist}\PYG{p}{(}\PYG{n}{s}\PYG{p}{,} \PYG{n}{bins}\PYG{o}{=}\PYG{l+m+mi}{50}\PYG{p}{)}
\end{Verbatim}

\#   plot against distribution

\begin{Verbatim}[commandchars=\\\{\}]
\PYG{g+gp}{\PYGZgt{}\PYGZgt{}\PYGZgt{} }\PYG{k}{def} \PYG{n+nf}{logist}\PYG{p}{(}\PYG{n}{x}\PYG{p}{,} \PYG{n}{loc}\PYG{p}{,} \PYG{n}{scale}\PYG{p}{)}\PYG{p}{:}
\PYG{g+gp}{... }    \PYG{k}{return} \PYG{n}{exp}\PYG{p}{(}\PYG{p}{(}\PYG{n}{loc}\PYG{o}{\PYGZhy{}}\PYG{n}{x}\PYG{p}{)}\PYG{o}{/}\PYG{n}{scale}\PYG{p}{)}\PYG{o}{/}\PYG{p}{(}\PYG{n}{scale}\PYG{o}{*}\PYG{p}{(}\PYG{l+m+mi}{1}\PYG{o}{+}\PYG{n}{exp}\PYG{p}{(}\PYG{p}{(}\PYG{n}{loc}\PYG{o}{\PYGZhy{}}\PYG{n}{x}\PYG{p}{)}\PYG{o}{/}\PYG{n}{scale}\PYG{p}{)}\PYG{p}{)}\PYG{o}{*}\PYG{o}{*}\PYG{l+m+mi}{2}\PYG{p}{)}
\PYG{g+gp}{\PYGZgt{}\PYGZgt{}\PYGZgt{} }\PYG{n}{plt}\PYG{o}{.}\PYG{n}{plot}\PYG{p}{(}\PYG{n}{bins}\PYG{p}{,} \PYG{n}{logist}\PYG{p}{(}\PYG{n}{bins}\PYG{p}{,} \PYG{n}{loc}\PYG{p}{,} \PYG{n}{scale}\PYG{p}{)}\PYG{o}{*}\PYG{n}{count}\PYG{o}{.}\PYG{n}{max}\PYG{p}{(}\PYG{p}{)}\PYG{o}{/}\PYGZbs{}
\PYG{g+gp}{... }\PYG{n}{logist}\PYG{p}{(}\PYG{n}{bins}\PYG{p}{,} \PYG{n}{loc}\PYG{p}{,} \PYG{n}{scale}\PYG{p}{)}\PYG{o}{.}\PYG{n}{max}\PYG{p}{(}\PYG{p}{)}\PYG{p}{)}
\PYG{g+gp}{\PYGZgt{}\PYGZgt{}\PYGZgt{} }\PYG{n}{plt}\PYG{o}{.}\PYG{n}{show}\PYG{p}{(}\PYG{p}{)}
\end{Verbatim}

\end{fulllineitems}

\index{lognormal() (in module graph\_chemistry\_analysis)}

\begin{fulllineitems}
\phantomsection\label{graph_chemistry_analysis:graph_chemistry_analysis.lognormal}\pysiglinewithargsret{\code{graph\_chemistry\_analysis.}\bfcode{lognormal}}{\emph{mean=0.0}, \emph{sigma=1.0}, \emph{size=None}}{}
Return samples drawn from a log-normal distribution.

Draw samples from a log-normal distribution with specified mean,
standard deviation, and array shape.  Note that the mean and standard
deviation are not the values for the distribution itself, but of the
underlying normal distribution it is derived from.
\begin{description}
\item[{mean}] \leavevmode{[}float{]}
Mean value of the underlying normal distribution

\item[{sigma}] \leavevmode{[}float, \textgreater{} 0.{]}
Standard deviation of the underlying normal distribution

\item[{size}] \leavevmode{[}tuple of ints{]}
Output shape.  If the given shape is, e.g., \code{(m, n, k)}, then
\code{m * n * k} samples are drawn.

\end{description}
\begin{description}
\item[{samples}] \leavevmode{[}ndarray or float{]}
The desired samples. An array of the same shape as \emph{size} if given,
if \emph{size} is None a float is returned.

\end{description}
\begin{description}
\item[{scipy.stats.lognorm}] \leavevmode{[}probability density function, distribution,{]}
cumulative density function, etc.

\end{description}

A variable \emph{x} has a log-normal distribution if \emph{log(x)} is normally
distributed.  The probability density function for the log-normal
distribution is:
\begin{gather}
\begin{split}p(x) = \frac{1}{\sigma x \sqrt{2\pi}}
e^{(-\frac{(ln(x)-\mu)^2}{2\sigma^2})}\end{split}\notag
\end{gather}
where \(\mu\) is the mean and \(\sigma\) is the standard
deviation of the normally distributed logarithm of the variable.
A log-normal distribution results if a random variable is the \emph{product}
of a large number of independent, identically-distributed variables in
the same way that a normal distribution results if the variable is the
\emph{sum} of a large number of independent, identically-distributed
variables.

Limpert, E., Stahel, W. A., and Abbt, M., ``Log-normal Distributions
across the Sciences: Keys and Clues,'' \emph{BioScience}, Vol. 51, No. 5,
May, 2001.  \href{http://stat.ethz.ch/~stahel/lognormal/bioscience.pdf}{http://stat.ethz.ch/\textasciitilde{}stahel/lognormal/bioscience.pdf}

Reiss, R.D. and Thomas, M., \emph{Statistical Analysis of Extreme Values},
Basel: Birkhauser Verlag, 2001, pp. 31-32.

Draw samples from the distribution:

\begin{Verbatim}[commandchars=\\\{\}]
\PYG{g+gp}{\PYGZgt{}\PYGZgt{}\PYGZgt{} }\PYG{n}{mu}\PYG{p}{,} \PYG{n}{sigma} \PYG{o}{=} \PYG{l+m+mf}{3.}\PYG{p}{,} \PYG{l+m+mf}{1.} \PYG{c}{\PYGZsh{} mean and standard deviation}
\PYG{g+gp}{\PYGZgt{}\PYGZgt{}\PYGZgt{} }\PYG{n}{s} \PYG{o}{=} \PYG{n}{np}\PYG{o}{.}\PYG{n}{random}\PYG{o}{.}\PYG{n}{lognormal}\PYG{p}{(}\PYG{n}{mu}\PYG{p}{,} \PYG{n}{sigma}\PYG{p}{,} \PYG{l+m+mi}{1000}\PYG{p}{)}
\end{Verbatim}

Display the histogram of the samples, along with
the probability density function:

\begin{Verbatim}[commandchars=\\\{\}]
\PYG{g+gp}{\PYGZgt{}\PYGZgt{}\PYGZgt{} }\PYG{k+kn}{import} \PYG{n+nn}{matplotlib.pyplot} \PYG{k+kn}{as} \PYG{n+nn}{plt}
\PYG{g+gp}{\PYGZgt{}\PYGZgt{}\PYGZgt{} }\PYG{n}{count}\PYG{p}{,} \PYG{n}{bins}\PYG{p}{,} \PYG{n}{ignored} \PYG{o}{=} \PYG{n}{plt}\PYG{o}{.}\PYG{n}{hist}\PYG{p}{(}\PYG{n}{s}\PYG{p}{,} \PYG{l+m+mi}{100}\PYG{p}{,} \PYG{n}{normed}\PYG{o}{=}\PYG{n+nb+bp}{True}\PYG{p}{,} \PYG{n}{align}\PYG{o}{=}\PYG{l+s}{\PYGZsq{}}\PYG{l+s}{mid}\PYG{l+s}{\PYGZsq{}}\PYG{p}{)}
\end{Verbatim}

\begin{Verbatim}[commandchars=\\\{\}]
\PYG{g+gp}{\PYGZgt{}\PYGZgt{}\PYGZgt{} }\PYG{n}{x} \PYG{o}{=} \PYG{n}{np}\PYG{o}{.}\PYG{n}{linspace}\PYG{p}{(}\PYG{n+nb}{min}\PYG{p}{(}\PYG{n}{bins}\PYG{p}{)}\PYG{p}{,} \PYG{n+nb}{max}\PYG{p}{(}\PYG{n}{bins}\PYG{p}{)}\PYG{p}{,} \PYG{l+m+mi}{10000}\PYG{p}{)}
\PYG{g+gp}{\PYGZgt{}\PYGZgt{}\PYGZgt{} }\PYG{n}{pdf} \PYG{o}{=} \PYG{p}{(}\PYG{n}{np}\PYG{o}{.}\PYG{n}{exp}\PYG{p}{(}\PYG{o}{\PYGZhy{}}\PYG{p}{(}\PYG{n}{np}\PYG{o}{.}\PYG{n}{log}\PYG{p}{(}\PYG{n}{x}\PYG{p}{)} \PYG{o}{\PYGZhy{}} \PYG{n}{mu}\PYG{p}{)}\PYG{o}{*}\PYG{o}{*}\PYG{l+m+mi}{2} \PYG{o}{/} \PYG{p}{(}\PYG{l+m+mi}{2} \PYG{o}{*} \PYG{n}{sigma}\PYG{o}{*}\PYG{o}{*}\PYG{l+m+mi}{2}\PYG{p}{)}\PYG{p}{)}
\PYG{g+gp}{... }       \PYG{o}{/} \PYG{p}{(}\PYG{n}{x} \PYG{o}{*} \PYG{n}{sigma} \PYG{o}{*} \PYG{n}{np}\PYG{o}{.}\PYG{n}{sqrt}\PYG{p}{(}\PYG{l+m+mi}{2} \PYG{o}{*} \PYG{n}{np}\PYG{o}{.}\PYG{n}{pi}\PYG{p}{)}\PYG{p}{)}\PYG{p}{)}
\end{Verbatim}

\begin{Verbatim}[commandchars=\\\{\}]
\PYG{g+gp}{\PYGZgt{}\PYGZgt{}\PYGZgt{} }\PYG{n}{plt}\PYG{o}{.}\PYG{n}{plot}\PYG{p}{(}\PYG{n}{x}\PYG{p}{,} \PYG{n}{pdf}\PYG{p}{,} \PYG{n}{linewidth}\PYG{o}{=}\PYG{l+m+mi}{2}\PYG{p}{,} \PYG{n}{color}\PYG{o}{=}\PYG{l+s}{\PYGZsq{}}\PYG{l+s}{r}\PYG{l+s}{\PYGZsq{}}\PYG{p}{)}
\PYG{g+gp}{\PYGZgt{}\PYGZgt{}\PYGZgt{} }\PYG{n}{plt}\PYG{o}{.}\PYG{n}{axis}\PYG{p}{(}\PYG{l+s}{\PYGZsq{}}\PYG{l+s}{tight}\PYG{l+s}{\PYGZsq{}}\PYG{p}{)}
\PYG{g+gp}{\PYGZgt{}\PYGZgt{}\PYGZgt{} }\PYG{n}{plt}\PYG{o}{.}\PYG{n}{show}\PYG{p}{(}\PYG{p}{)}
\end{Verbatim}

Demonstrate that taking the products of random samples from a uniform
distribution can be fit well by a log-normal probability density function.

\begin{Verbatim}[commandchars=\\\{\}]
\PYG{g+gp}{\PYGZgt{}\PYGZgt{}\PYGZgt{} }\PYG{c}{\PYGZsh{} Generate a thousand samples: each is the product of 100 random}
\PYG{g+gp}{\PYGZgt{}\PYGZgt{}\PYGZgt{} }\PYG{c}{\PYGZsh{} values, drawn from a normal distribution.}
\PYG{g+gp}{\PYGZgt{}\PYGZgt{}\PYGZgt{} }\PYG{n}{b} \PYG{o}{=} \PYG{p}{[}\PYG{p}{]}
\PYG{g+gp}{\PYGZgt{}\PYGZgt{}\PYGZgt{} }\PYG{k}{for} \PYG{n}{i} \PYG{o+ow}{in} \PYG{n+nb}{range}\PYG{p}{(}\PYG{l+m+mi}{1000}\PYG{p}{)}\PYG{p}{:}
\PYG{g+gp}{... }   \PYG{n}{a} \PYG{o}{=} \PYG{l+m+mf}{10.} \PYG{o}{+} \PYG{n}{np}\PYG{o}{.}\PYG{n}{random}\PYG{o}{.}\PYG{n}{random}\PYG{p}{(}\PYG{l+m+mi}{100}\PYG{p}{)}
\PYG{g+gp}{... }   \PYG{n}{b}\PYG{o}{.}\PYG{n}{append}\PYG{p}{(}\PYG{n}{np}\PYG{o}{.}\PYG{n}{product}\PYG{p}{(}\PYG{n}{a}\PYG{p}{)}\PYG{p}{)}
\end{Verbatim}

\begin{Verbatim}[commandchars=\\\{\}]
\PYG{g+gp}{\PYGZgt{}\PYGZgt{}\PYGZgt{} }\PYG{n}{b} \PYG{o}{=} \PYG{n}{np}\PYG{o}{.}\PYG{n}{array}\PYG{p}{(}\PYG{n}{b}\PYG{p}{)} \PYG{o}{/} \PYG{n}{np}\PYG{o}{.}\PYG{n}{min}\PYG{p}{(}\PYG{n}{b}\PYG{p}{)} \PYG{c}{\PYGZsh{} scale values to be positive}
\PYG{g+gp}{\PYGZgt{}\PYGZgt{}\PYGZgt{} }\PYG{n}{count}\PYG{p}{,} \PYG{n}{bins}\PYG{p}{,} \PYG{n}{ignored} \PYG{o}{=} \PYG{n}{plt}\PYG{o}{.}\PYG{n}{hist}\PYG{p}{(}\PYG{n}{b}\PYG{p}{,} \PYG{l+m+mi}{100}\PYG{p}{,} \PYG{n}{normed}\PYG{o}{=}\PYG{n+nb+bp}{True}\PYG{p}{,} \PYG{n}{align}\PYG{o}{=}\PYG{l+s}{\PYGZsq{}}\PYG{l+s}{center}\PYG{l+s}{\PYGZsq{}}\PYG{p}{)}
\PYG{g+gp}{\PYGZgt{}\PYGZgt{}\PYGZgt{} }\PYG{n}{sigma} \PYG{o}{=} \PYG{n}{np}\PYG{o}{.}\PYG{n}{std}\PYG{p}{(}\PYG{n}{np}\PYG{o}{.}\PYG{n}{log}\PYG{p}{(}\PYG{n}{b}\PYG{p}{)}\PYG{p}{)}
\PYG{g+gp}{\PYGZgt{}\PYGZgt{}\PYGZgt{} }\PYG{n}{mu} \PYG{o}{=} \PYG{n}{np}\PYG{o}{.}\PYG{n}{mean}\PYG{p}{(}\PYG{n}{np}\PYG{o}{.}\PYG{n}{log}\PYG{p}{(}\PYG{n}{b}\PYG{p}{)}\PYG{p}{)}
\end{Verbatim}

\begin{Verbatim}[commandchars=\\\{\}]
\PYG{g+gp}{\PYGZgt{}\PYGZgt{}\PYGZgt{} }\PYG{n}{x} \PYG{o}{=} \PYG{n}{np}\PYG{o}{.}\PYG{n}{linspace}\PYG{p}{(}\PYG{n+nb}{min}\PYG{p}{(}\PYG{n}{bins}\PYG{p}{)}\PYG{p}{,} \PYG{n+nb}{max}\PYG{p}{(}\PYG{n}{bins}\PYG{p}{)}\PYG{p}{,} \PYG{l+m+mi}{10000}\PYG{p}{)}
\PYG{g+gp}{\PYGZgt{}\PYGZgt{}\PYGZgt{} }\PYG{n}{pdf} \PYG{o}{=} \PYG{p}{(}\PYG{n}{np}\PYG{o}{.}\PYG{n}{exp}\PYG{p}{(}\PYG{o}{\PYGZhy{}}\PYG{p}{(}\PYG{n}{np}\PYG{o}{.}\PYG{n}{log}\PYG{p}{(}\PYG{n}{x}\PYG{p}{)} \PYG{o}{\PYGZhy{}} \PYG{n}{mu}\PYG{p}{)}\PYG{o}{*}\PYG{o}{*}\PYG{l+m+mi}{2} \PYG{o}{/} \PYG{p}{(}\PYG{l+m+mi}{2} \PYG{o}{*} \PYG{n}{sigma}\PYG{o}{*}\PYG{o}{*}\PYG{l+m+mi}{2}\PYG{p}{)}\PYG{p}{)}
\PYG{g+gp}{... }       \PYG{o}{/} \PYG{p}{(}\PYG{n}{x} \PYG{o}{*} \PYG{n}{sigma} \PYG{o}{*} \PYG{n}{np}\PYG{o}{.}\PYG{n}{sqrt}\PYG{p}{(}\PYG{l+m+mi}{2} \PYG{o}{*} \PYG{n}{np}\PYG{o}{.}\PYG{n}{pi}\PYG{p}{)}\PYG{p}{)}\PYG{p}{)}
\end{Verbatim}

\begin{Verbatim}[commandchars=\\\{\}]
\PYG{g+gp}{\PYGZgt{}\PYGZgt{}\PYGZgt{} }\PYG{n}{plt}\PYG{o}{.}\PYG{n}{plot}\PYG{p}{(}\PYG{n}{x}\PYG{p}{,} \PYG{n}{pdf}\PYG{p}{,} \PYG{n}{color}\PYG{o}{=}\PYG{l+s}{\PYGZsq{}}\PYG{l+s}{r}\PYG{l+s}{\PYGZsq{}}\PYG{p}{,} \PYG{n}{linewidth}\PYG{o}{=}\PYG{l+m+mi}{2}\PYG{p}{)}
\PYG{g+gp}{\PYGZgt{}\PYGZgt{}\PYGZgt{} }\PYG{n}{plt}\PYG{o}{.}\PYG{n}{show}\PYG{p}{(}\PYG{p}{)}
\end{Verbatim}

\end{fulllineitems}

\index{logseries() (in module graph\_chemistry\_analysis)}

\begin{fulllineitems}
\phantomsection\label{graph_chemistry_analysis:graph_chemistry_analysis.logseries}\pysiglinewithargsret{\code{graph\_chemistry\_analysis.}\bfcode{logseries}}{\emph{p}, \emph{size=None}}{}
Draw samples from a Logarithmic Series distribution.

Samples are drawn from a Log Series distribution with specified
parameter, p (probability, 0 \textless{} p \textless{} 1).

loc : float

scale : float \textgreater{} 0.
\begin{description}
\item[{size}] \leavevmode{[}\{tuple, int\}{]}
Output shape.  If the given shape is, e.g., \code{(m, n, k)}, then
\code{m * n * k} samples are drawn.

\end{description}
\begin{description}
\item[{samples}] \leavevmode{[}\{ndarray, scalar\}{]}
where the values are all integers in  {[}0, n{]}.

\end{description}
\begin{description}
\item[{scipy.stats.distributions.logser}] \leavevmode{[}probability density function,{]}
distribution or cumulative density function, etc.

\end{description}

The probability density for the Log Series distribution is
\begin{gather}
\begin{split}P(k) = \frac{-p^k}{k \ln(1-p)},\end{split}\notag
\end{gather}
where p = probability.

The Log Series distribution is frequently used to represent species
richness and occurrence, first proposed by Fisher, Corbet, and
Williams in 1943 {[}2{]}.  It may also be used to model the numbers of
occupants seen in cars {[}3{]}.

Draw samples from the distribution:

\begin{Verbatim}[commandchars=\\\{\}]
\PYG{g+gp}{\PYGZgt{}\PYGZgt{}\PYGZgt{} }\PYG{n}{a} \PYG{o}{=} \PYG{o}{.}\PYG{l+m+mi}{6}
\PYG{g+gp}{\PYGZgt{}\PYGZgt{}\PYGZgt{} }\PYG{n}{s} \PYG{o}{=} \PYG{n}{np}\PYG{o}{.}\PYG{n}{random}\PYG{o}{.}\PYG{n}{logseries}\PYG{p}{(}\PYG{n}{a}\PYG{p}{,} \PYG{l+m+mi}{10000}\PYG{p}{)}
\PYG{g+gp}{\PYGZgt{}\PYGZgt{}\PYGZgt{} }\PYG{n}{count}\PYG{p}{,} \PYG{n}{bins}\PYG{p}{,} \PYG{n}{ignored} \PYG{o}{=} \PYG{n}{plt}\PYG{o}{.}\PYG{n}{hist}\PYG{p}{(}\PYG{n}{s}\PYG{p}{)}
\end{Verbatim}

\#   plot against distribution

\begin{Verbatim}[commandchars=\\\{\}]
\PYG{g+gp}{\PYGZgt{}\PYGZgt{}\PYGZgt{} }\PYG{k}{def} \PYG{n+nf}{logseries}\PYG{p}{(}\PYG{n}{k}\PYG{p}{,} \PYG{n}{p}\PYG{p}{)}\PYG{p}{:}
\PYG{g+gp}{... }    \PYG{k}{return} \PYG{o}{\PYGZhy{}}\PYG{n}{p}\PYG{o}{*}\PYG{o}{*}\PYG{n}{k}\PYG{o}{/}\PYG{p}{(}\PYG{n}{k}\PYG{o}{*}\PYG{n}{log}\PYG{p}{(}\PYG{l+m+mi}{1}\PYG{o}{\PYGZhy{}}\PYG{n}{p}\PYG{p}{)}\PYG{p}{)}
\PYG{g+gp}{\PYGZgt{}\PYGZgt{}\PYGZgt{} }\PYG{n}{plt}\PYG{o}{.}\PYG{n}{plot}\PYG{p}{(}\PYG{n}{bins}\PYG{p}{,} \PYG{n}{logseries}\PYG{p}{(}\PYG{n}{bins}\PYG{p}{,} \PYG{n}{a}\PYG{p}{)}\PYG{o}{*}\PYG{n}{count}\PYG{o}{.}\PYG{n}{max}\PYG{p}{(}\PYG{p}{)}\PYG{o}{/}
\PYG{g+go}{             logseries(bins, a).max(), \PYGZsq{}r\PYGZsq{})}
\PYG{g+gp}{\PYGZgt{}\PYGZgt{}\PYGZgt{} }\PYG{n}{plt}\PYG{o}{.}\PYG{n}{show}\PYG{p}{(}\PYG{p}{)}
\end{Verbatim}

\end{fulllineitems}

\index{multinomial() (in module graph\_chemistry\_analysis)}

\begin{fulllineitems}
\phantomsection\label{graph_chemistry_analysis:graph_chemistry_analysis.multinomial}\pysiglinewithargsret{\code{graph\_chemistry\_analysis.}\bfcode{multinomial}}{\emph{n}, \emph{pvals}, \emph{size=None}}{}
Draw samples from a multinomial distribution.

The multinomial distribution is a multivariate generalisation of the
binomial distribution.  Take an experiment with one of \code{p}
possible outcomes.  An example of such an experiment is throwing a dice,
where the outcome can be 1 through 6.  Each sample drawn from the
distribution represents \emph{n} such experiments.  Its values,
\code{X\_i = {[}X\_0, X\_1, ..., X\_p{]}}, represent the number of times the outcome
was \code{i}.
\begin{description}
\item[{n}] \leavevmode{[}int{]}
Number of experiments.

\item[{pvals}] \leavevmode{[}sequence of floats, length p{]}
Probabilities of each of the \code{p} different outcomes.  These
should sum to 1 (however, the last element is always assumed to
account for the remaining probability, as long as
\code{sum(pvals{[}:-1{]}) \textless{}= 1)}.

\item[{size}] \leavevmode{[}tuple of ints{]}
Given a \emph{size} of \code{(M, N, K)}, then \code{M*N*K} samples are drawn,
and the output shape becomes \code{(M, N, K, p)}, since each sample
has shape \code{(p,)}.

\end{description}

Throw a dice 20 times:

\begin{Verbatim}[commandchars=\\\{\}]
\PYG{g+gp}{\PYGZgt{}\PYGZgt{}\PYGZgt{} }\PYG{n}{np}\PYG{o}{.}\PYG{n}{random}\PYG{o}{.}\PYG{n}{multinomial}\PYG{p}{(}\PYG{l+m+mi}{20}\PYG{p}{,} \PYG{p}{[}\PYG{l+m+mi}{1}\PYG{o}{/}\PYG{l+m+mf}{6.}\PYG{p}{]}\PYG{o}{*}\PYG{l+m+mi}{6}\PYG{p}{,} \PYG{n}{size}\PYG{o}{=}\PYG{l+m+mi}{1}\PYG{p}{)}
\PYG{g+go}{array([[4, 1, 7, 5, 2, 1]])}
\end{Verbatim}

It landed 4 times on 1, once on 2, etc.

Now, throw the dice 20 times, and 20 times again:

\begin{Verbatim}[commandchars=\\\{\}]
\PYG{g+gp}{\PYGZgt{}\PYGZgt{}\PYGZgt{} }\PYG{n}{np}\PYG{o}{.}\PYG{n}{random}\PYG{o}{.}\PYG{n}{multinomial}\PYG{p}{(}\PYG{l+m+mi}{20}\PYG{p}{,} \PYG{p}{[}\PYG{l+m+mi}{1}\PYG{o}{/}\PYG{l+m+mf}{6.}\PYG{p}{]}\PYG{o}{*}\PYG{l+m+mi}{6}\PYG{p}{,} \PYG{n}{size}\PYG{o}{=}\PYG{l+m+mi}{2}\PYG{p}{)}
\PYG{g+go}{array([[3, 4, 3, 3, 4, 3],}
\PYG{g+go}{       [2, 4, 3, 4, 0, 7]])}
\end{Verbatim}

For the first run, we threw 3 times 1, 4 times 2, etc.  For the second,
we threw 2 times 1, 4 times 2, etc.

A loaded dice is more likely to land on number 6:

\begin{Verbatim}[commandchars=\\\{\}]
\PYG{g+gp}{\PYGZgt{}\PYGZgt{}\PYGZgt{} }\PYG{n}{np}\PYG{o}{.}\PYG{n}{random}\PYG{o}{.}\PYG{n}{multinomial}\PYG{p}{(}\PYG{l+m+mi}{100}\PYG{p}{,} \PYG{p}{[}\PYG{l+m+mi}{1}\PYG{o}{/}\PYG{l+m+mf}{7.}\PYG{p}{]}\PYG{o}{*}\PYG{l+m+mi}{5}\PYG{p}{)}
\PYG{g+go}{array([13, 16, 13, 16, 42])}
\end{Verbatim}

\end{fulllineitems}

\index{multivariate\_normal() (in module graph\_chemistry\_analysis)}

\begin{fulllineitems}
\phantomsection\label{graph_chemistry_analysis:graph_chemistry_analysis.multivariate_normal}\pysiglinewithargsret{\code{graph\_chemistry\_analysis.}\bfcode{multivariate\_normal}}{\emph{mean}, \emph{cov}\optional{, \emph{size}}}{}
Draw random samples from a multivariate normal distribution.

The multivariate normal, multinormal or Gaussian distribution is a
generalization of the one-dimensional normal distribution to higher
dimensions.  Such a distribution is specified by its mean and
covariance matrix.  These parameters are analogous to the mean
(average or ``center'') and variance (standard deviation, or ``width,''
squared) of the one-dimensional normal distribution.
\begin{description}
\item[{mean}] \leavevmode{[}1-D array\_like, of length N{]}
Mean of the N-dimensional distribution.

\item[{cov}] \leavevmode{[}2-D array\_like, of shape (N, N){]}
Covariance matrix of the distribution.  Must be symmetric and
positive semi-definite for ``physically meaningful'' results.

\item[{size}] \leavevmode{[}int or tuple of ints, optional{]}
Given a shape of, for example, \code{(m,n,k)}, \code{m*n*k} samples are
generated, and packed in an \emph{m}-by-\emph{n}-by-\emph{k} arrangement.  Because
each sample is \emph{N}-dimensional, the output shape is \code{(m,n,k,N)}.
If no shape is specified, a single (\emph{N}-D) sample is returned.

\end{description}
\begin{description}
\item[{out}] \leavevmode{[}ndarray{]}
The drawn samples, of shape \emph{size}, if that was provided.  If not,
the shape is \code{(N,)}.

In other words, each entry \code{out{[}i,j,...,:{]}} is an N-dimensional
value drawn from the distribution.

\end{description}

The mean is a coordinate in N-dimensional space, which represents the
location where samples are most likely to be generated.  This is
analogous to the peak of the bell curve for the one-dimensional or
univariate normal distribution.

Covariance indicates the level to which two variables vary together.
From the multivariate normal distribution, we draw N-dimensional
samples, \(X = [x_1, x_2, ... x_N]\).  The covariance matrix
element \(C_{ij}\) is the covariance of \(x_i\) and \(x_j\).
The element \(C_{ii}\) is the variance of \(x_i\) (i.e. its
``spread'').

Instead of specifying the full covariance matrix, popular
approximations include:
\begin{itemize}
\item {} 
Spherical covariance (\emph{cov} is a multiple of the identity matrix)

\item {} 
Diagonal covariance (\emph{cov} has non-negative elements, and only on
the diagonal)

\end{itemize}

This geometrical property can be seen in two dimensions by plotting
generated data-points:

\begin{Verbatim}[commandchars=\\\{\}]
\PYG{g+gp}{\PYGZgt{}\PYGZgt{}\PYGZgt{} }\PYG{n}{mean} \PYG{o}{=} \PYG{p}{[}\PYG{l+m+mi}{0}\PYG{p}{,}\PYG{l+m+mi}{0}\PYG{p}{]}
\PYG{g+gp}{\PYGZgt{}\PYGZgt{}\PYGZgt{} }\PYG{n}{cov} \PYG{o}{=} \PYG{p}{[}\PYG{p}{[}\PYG{l+m+mi}{1}\PYG{p}{,}\PYG{l+m+mi}{0}\PYG{p}{]}\PYG{p}{,}\PYG{p}{[}\PYG{l+m+mi}{0}\PYG{p}{,}\PYG{l+m+mi}{100}\PYG{p}{]}\PYG{p}{]} \PYG{c}{\PYGZsh{} diagonal covariance, points lie on x or y\PYGZhy{}axis}
\end{Verbatim}

\begin{Verbatim}[commandchars=\\\{\}]
\PYG{g+gp}{\PYGZgt{}\PYGZgt{}\PYGZgt{} }\PYG{k+kn}{import} \PYG{n+nn}{matplotlib.pyplot} \PYG{k+kn}{as} \PYG{n+nn}{plt}
\PYG{g+gp}{\PYGZgt{}\PYGZgt{}\PYGZgt{} }\PYG{n}{x}\PYG{p}{,}\PYG{n}{y} \PYG{o}{=} \PYG{n}{np}\PYG{o}{.}\PYG{n}{random}\PYG{o}{.}\PYG{n}{multivariate\PYGZus{}normal}\PYG{p}{(}\PYG{n}{mean}\PYG{p}{,}\PYG{n}{cov}\PYG{p}{,}\PYG{l+m+mi}{5000}\PYG{p}{)}\PYG{o}{.}\PYG{n}{T}
\PYG{g+gp}{\PYGZgt{}\PYGZgt{}\PYGZgt{} }\PYG{n}{plt}\PYG{o}{.}\PYG{n}{plot}\PYG{p}{(}\PYG{n}{x}\PYG{p}{,}\PYG{n}{y}\PYG{p}{,}\PYG{l+s}{\PYGZsq{}}\PYG{l+s}{x}\PYG{l+s}{\PYGZsq{}}\PYG{p}{)}\PYG{p}{;} \PYG{n}{plt}\PYG{o}{.}\PYG{n}{axis}\PYG{p}{(}\PYG{l+s}{\PYGZsq{}}\PYG{l+s}{equal}\PYG{l+s}{\PYGZsq{}}\PYG{p}{)}\PYG{p}{;} \PYG{n}{plt}\PYG{o}{.}\PYG{n}{show}\PYG{p}{(}\PYG{p}{)}
\end{Verbatim}

Note that the covariance matrix must be non-negative definite.

Papoulis, A., \emph{Probability, Random Variables, and Stochastic Processes},
3rd ed., New York: McGraw-Hill, 1991.

Duda, R. O., Hart, P. E., and Stork, D. G., \emph{Pattern Classification},
2nd ed., New York: Wiley, 2001.

\begin{Verbatim}[commandchars=\\\{\}]
\PYG{g+gp}{\PYGZgt{}\PYGZgt{}\PYGZgt{} }\PYG{n}{mean} \PYG{o}{=} \PYG{p}{(}\PYG{l+m+mi}{1}\PYG{p}{,}\PYG{l+m+mi}{2}\PYG{p}{)}
\PYG{g+gp}{\PYGZgt{}\PYGZgt{}\PYGZgt{} }\PYG{n}{cov} \PYG{o}{=} \PYG{p}{[}\PYG{p}{[}\PYG{l+m+mi}{1}\PYG{p}{,}\PYG{l+m+mi}{0}\PYG{p}{]}\PYG{p}{,}\PYG{p}{[}\PYG{l+m+mi}{1}\PYG{p}{,}\PYG{l+m+mi}{0}\PYG{p}{]}\PYG{p}{]}
\PYG{g+gp}{\PYGZgt{}\PYGZgt{}\PYGZgt{} }\PYG{n}{x} \PYG{o}{=} \PYG{n}{np}\PYG{o}{.}\PYG{n}{random}\PYG{o}{.}\PYG{n}{multivariate\PYGZus{}normal}\PYG{p}{(}\PYG{n}{mean}\PYG{p}{,}\PYG{n}{cov}\PYG{p}{,}\PYG{p}{(}\PYG{l+m+mi}{3}\PYG{p}{,}\PYG{l+m+mi}{3}\PYG{p}{)}\PYG{p}{)}
\PYG{g+gp}{\PYGZgt{}\PYGZgt{}\PYGZgt{} }\PYG{n}{x}\PYG{o}{.}\PYG{n}{shape}
\PYG{g+go}{(3, 3, 2)}
\end{Verbatim}

The following is probably true, given that 0.6 is roughly twice the
standard deviation:

\begin{Verbatim}[commandchars=\\\{\}]
\PYG{g+gp}{\PYGZgt{}\PYGZgt{}\PYGZgt{} }\PYG{k}{print} \PYG{n+nb}{list}\PYG{p}{(} \PYG{p}{(}\PYG{n}{x}\PYG{p}{[}\PYG{l+m+mi}{0}\PYG{p}{,}\PYG{l+m+mi}{0}\PYG{p}{,}\PYG{p}{:}\PYG{p}{]} \PYG{o}{\PYGZhy{}} \PYG{n}{mean}\PYG{p}{)} \PYG{o}{\PYGZlt{}} \PYG{l+m+mf}{0.6} \PYG{p}{)}
\PYG{g+go}{[True, True]}
\end{Verbatim}

\end{fulllineitems}

\index{negative\_binomial() (in module graph\_chemistry\_analysis)}

\begin{fulllineitems}
\phantomsection\label{graph_chemistry_analysis:graph_chemistry_analysis.negative_binomial}\pysiglinewithargsret{\code{graph\_chemistry\_analysis.}\bfcode{negative\_binomial}}{\emph{n}, \emph{p}, \emph{size=None}}{}
Draw samples from a negative\_binomial distribution.

Samples are drawn from a negative\_Binomial distribution with specified
parameters, \emph{n} trials and \emph{p} probability of success where \emph{n} is an
integer \textgreater{} 0 and \emph{p} is in the interval {[}0, 1{]}.
\begin{description}
\item[{n}] \leavevmode{[}int{]}
Parameter, \textgreater{} 0.

\item[{p}] \leavevmode{[}float{]}
Parameter, \textgreater{}= 0 and \textless{}=1.

\item[{size}] \leavevmode{[}int or tuple of ints{]}
Output shape. If the given shape is, e.g., \code{(m, n, k)}, then
\code{m * n * k} samples are drawn.

\end{description}
\begin{description}
\item[{samples}] \leavevmode{[}int or ndarray of ints{]}
Drawn samples.

\end{description}

The probability density for the Negative Binomial distribution is
\begin{gather}
\begin{split}P(N;n,p) = \binom{N+n-1}{n-1}p^{n}(1-p)^{N},\end{split}\notag
\end{gather}
where \(n-1\) is the number of successes, \(p\) is the probability
of success, and \(N+n-1\) is the number of trials.

The negative binomial distribution gives the probability of n-1 successes
and N failures in N+n-1 trials, and success on the (N+n)th trial.

If one throws a die repeatedly until the third time a ``1'' appears, then the
probability distribution of the number of non-``1''s that appear before the
third ``1'' is a negative binomial distribution.

Draw samples from the distribution:

A real world example. A company drills wild-cat oil exploration wells, each
with an estimated probability of success of 0.1.  What is the probability
of having one success for each successive well, that is what is the
probability of a single success after drilling 5 wells, after 6 wells,
etc.?

\begin{Verbatim}[commandchars=\\\{\}]
\PYG{g+gp}{\PYGZgt{}\PYGZgt{}\PYGZgt{} }\PYG{n}{s} \PYG{o}{=} \PYG{n}{np}\PYG{o}{.}\PYG{n}{random}\PYG{o}{.}\PYG{n}{negative\PYGZus{}binomial}\PYG{p}{(}\PYG{l+m+mi}{1}\PYG{p}{,} \PYG{l+m+mf}{0.1}\PYG{p}{,} \PYG{l+m+mi}{100000}\PYG{p}{)}
\PYG{g+gp}{\PYGZgt{}\PYGZgt{}\PYGZgt{} }\PYG{k}{for} \PYG{n}{i} \PYG{o+ow}{in} \PYG{n+nb}{range}\PYG{p}{(}\PYG{l+m+mi}{1}\PYG{p}{,} \PYG{l+m+mi}{11}\PYG{p}{)}\PYG{p}{:}
\PYG{g+gp}{... }   \PYG{n}{probability} \PYG{o}{=} \PYG{n+nb}{sum}\PYG{p}{(}\PYG{n}{s}\PYG{o}{\PYGZlt{}}\PYG{n}{i}\PYG{p}{)} \PYG{o}{/} \PYG{l+m+mf}{100000.}
\PYG{g+gp}{... }   \PYG{k}{print} \PYG{n}{i}\PYG{p}{,} \PYG{l+s}{\PYGZdq{}}\PYG{l+s}{wells drilled, probability of one success =}\PYG{l+s}{\PYGZdq{}}\PYG{p}{,} \PYG{n}{probability}
\end{Verbatim}

\end{fulllineitems}

\index{noncentral\_chisquare() (in module graph\_chemistry\_analysis)}

\begin{fulllineitems}
\phantomsection\label{graph_chemistry_analysis:graph_chemistry_analysis.noncentral_chisquare}\pysiglinewithargsret{\code{graph\_chemistry\_analysis.}\bfcode{noncentral\_chisquare}}{\emph{df}, \emph{nonc}, \emph{size=None}}{}
Draw samples from a noncentral chi-square distribution.

The noncentral \(\chi^2\) distribution is a generalisation of
the \(\chi^2\) distribution.
\begin{description}
\item[{df}] \leavevmode{[}int{]}
Degrees of freedom, should be \textgreater{}= 1.

\item[{nonc}] \leavevmode{[}float{]}
Non-centrality, should be \textgreater{} 0.

\item[{size}] \leavevmode{[}int or tuple of ints{]}
Shape of the output.

\end{description}

The probability density function for the noncentral Chi-square distribution
is
\begin{gather}
\begin{split}P(x;df,nonc) = \sum^{\infty}_{i=0}
\frac{e^{-nonc/2}(nonc/2)^{i}}{i!}P_{Y_{df+2i}}(x),\end{split}\notag
\end{gather}
where \(Y_{q}\) is the Chi-square with q degrees of freedom.

In Delhi (2007), it is noted that the noncentral chi-square is useful in
bombing and coverage problems, the probability of killing the point target
given by the noncentral chi-squared distribution.

Draw values from the distribution and plot the histogram

\begin{Verbatim}[commandchars=\\\{\}]
\PYG{g+gp}{\PYGZgt{}\PYGZgt{}\PYGZgt{} }\PYG{k+kn}{import} \PYG{n+nn}{matplotlib.pyplot} \PYG{k+kn}{as} \PYG{n+nn}{plt}
\PYG{g+gp}{\PYGZgt{}\PYGZgt{}\PYGZgt{} }\PYG{n}{values} \PYG{o}{=} \PYG{n}{plt}\PYG{o}{.}\PYG{n}{hist}\PYG{p}{(}\PYG{n}{np}\PYG{o}{.}\PYG{n}{random}\PYG{o}{.}\PYG{n}{noncentral\PYGZus{}chisquare}\PYG{p}{(}\PYG{l+m+mi}{3}\PYG{p}{,} \PYG{l+m+mi}{20}\PYG{p}{,} \PYG{l+m+mi}{100000}\PYG{p}{)}\PYG{p}{,}
\PYG{g+gp}{... }                  \PYG{n}{bins}\PYG{o}{=}\PYG{l+m+mi}{200}\PYG{p}{,} \PYG{n}{normed}\PYG{o}{=}\PYG{n+nb+bp}{True}\PYG{p}{)}
\PYG{g+gp}{\PYGZgt{}\PYGZgt{}\PYGZgt{} }\PYG{n}{plt}\PYG{o}{.}\PYG{n}{show}\PYG{p}{(}\PYG{p}{)}
\end{Verbatim}

Draw values from a noncentral chisquare with very small noncentrality,
and compare to a chisquare.

\begin{Verbatim}[commandchars=\\\{\}]
\PYG{g+gp}{\PYGZgt{}\PYGZgt{}\PYGZgt{} }\PYG{n}{plt}\PYG{o}{.}\PYG{n}{figure}\PYG{p}{(}\PYG{p}{)}
\PYG{g+gp}{\PYGZgt{}\PYGZgt{}\PYGZgt{} }\PYG{n}{values} \PYG{o}{=} \PYG{n}{plt}\PYG{o}{.}\PYG{n}{hist}\PYG{p}{(}\PYG{n}{np}\PYG{o}{.}\PYG{n}{random}\PYG{o}{.}\PYG{n}{noncentral\PYGZus{}chisquare}\PYG{p}{(}\PYG{l+m+mi}{3}\PYG{p}{,} \PYG{o}{.}\PYG{l+m+mo}{0000001}\PYG{p}{,} \PYG{l+m+mi}{100000}\PYG{p}{)}\PYG{p}{,}
\PYG{g+gp}{... }                  \PYG{n}{bins}\PYG{o}{=}\PYG{n}{np}\PYG{o}{.}\PYG{n}{arange}\PYG{p}{(}\PYG{l+m+mf}{0.}\PYG{p}{,} \PYG{l+m+mi}{25}\PYG{p}{,} \PYG{o}{.}\PYG{l+m+mi}{1}\PYG{p}{)}\PYG{p}{,} \PYG{n}{normed}\PYG{o}{=}\PYG{n+nb+bp}{True}\PYG{p}{)}
\PYG{g+gp}{\PYGZgt{}\PYGZgt{}\PYGZgt{} }\PYG{n}{values2} \PYG{o}{=} \PYG{n}{plt}\PYG{o}{.}\PYG{n}{hist}\PYG{p}{(}\PYG{n}{np}\PYG{o}{.}\PYG{n}{random}\PYG{o}{.}\PYG{n}{chisquare}\PYG{p}{(}\PYG{l+m+mi}{3}\PYG{p}{,} \PYG{l+m+mi}{100000}\PYG{p}{)}\PYG{p}{,}
\PYG{g+gp}{... }                   \PYG{n}{bins}\PYG{o}{=}\PYG{n}{np}\PYG{o}{.}\PYG{n}{arange}\PYG{p}{(}\PYG{l+m+mf}{0.}\PYG{p}{,} \PYG{l+m+mi}{25}\PYG{p}{,} \PYG{o}{.}\PYG{l+m+mi}{1}\PYG{p}{)}\PYG{p}{,} \PYG{n}{normed}\PYG{o}{=}\PYG{n+nb+bp}{True}\PYG{p}{)}
\PYG{g+gp}{\PYGZgt{}\PYGZgt{}\PYGZgt{} }\PYG{n}{plt}\PYG{o}{.}\PYG{n}{plot}\PYG{p}{(}\PYG{n}{values}\PYG{p}{[}\PYG{l+m+mi}{1}\PYG{p}{]}\PYG{p}{[}\PYG{l+m+mi}{0}\PYG{p}{:}\PYG{o}{\PYGZhy{}}\PYG{l+m+mi}{1}\PYG{p}{]}\PYG{p}{,} \PYG{n}{values}\PYG{p}{[}\PYG{l+m+mi}{0}\PYG{p}{]}\PYG{o}{\PYGZhy{}}\PYG{n}{values2}\PYG{p}{[}\PYG{l+m+mi}{0}\PYG{p}{]}\PYG{p}{,} \PYG{l+s}{\PYGZsq{}}\PYG{l+s}{ob}\PYG{l+s}{\PYGZsq{}}\PYG{p}{)}
\PYG{g+gp}{\PYGZgt{}\PYGZgt{}\PYGZgt{} }\PYG{n}{plt}\PYG{o}{.}\PYG{n}{show}\PYG{p}{(}\PYG{p}{)}
\end{Verbatim}

Demonstrate how large values of non-centrality lead to a more symmetric
distribution.

\begin{Verbatim}[commandchars=\\\{\}]
\PYG{g+gp}{\PYGZgt{}\PYGZgt{}\PYGZgt{} }\PYG{n}{plt}\PYG{o}{.}\PYG{n}{figure}\PYG{p}{(}\PYG{p}{)}
\PYG{g+gp}{\PYGZgt{}\PYGZgt{}\PYGZgt{} }\PYG{n}{values} \PYG{o}{=} \PYG{n}{plt}\PYG{o}{.}\PYG{n}{hist}\PYG{p}{(}\PYG{n}{np}\PYG{o}{.}\PYG{n}{random}\PYG{o}{.}\PYG{n}{noncentral\PYGZus{}chisquare}\PYG{p}{(}\PYG{l+m+mi}{3}\PYG{p}{,} \PYG{l+m+mi}{20}\PYG{p}{,} \PYG{l+m+mi}{100000}\PYG{p}{)}\PYG{p}{,}
\PYG{g+gp}{... }                  \PYG{n}{bins}\PYG{o}{=}\PYG{l+m+mi}{200}\PYG{p}{,} \PYG{n}{normed}\PYG{o}{=}\PYG{n+nb+bp}{True}\PYG{p}{)}
\PYG{g+gp}{\PYGZgt{}\PYGZgt{}\PYGZgt{} }\PYG{n}{plt}\PYG{o}{.}\PYG{n}{show}\PYG{p}{(}\PYG{p}{)}
\end{Verbatim}

\end{fulllineitems}

\index{noncentral\_f() (in module graph\_chemistry\_analysis)}

\begin{fulllineitems}
\phantomsection\label{graph_chemistry_analysis:graph_chemistry_analysis.noncentral_f}\pysiglinewithargsret{\code{graph\_chemistry\_analysis.}\bfcode{noncentral\_f}}{\emph{dfnum}, \emph{dfden}, \emph{nonc}, \emph{size=None}}{}
Draw samples from the noncentral F distribution.

Samples are drawn from an F distribution with specified parameters,
\emph{dfnum} (degrees of freedom in numerator) and \emph{dfden} (degrees of
freedom in denominator), where both parameters \textgreater{} 1.
\emph{nonc} is the non-centrality parameter.
\begin{description}
\item[{dfnum}] \leavevmode{[}int{]}
Parameter, should be \textgreater{} 1.

\item[{dfden}] \leavevmode{[}int{]}
Parameter, should be \textgreater{} 1.

\item[{nonc}] \leavevmode{[}float{]}
Parameter, should be \textgreater{}= 0.

\item[{size}] \leavevmode{[}int or tuple of ints{]}
Output shape. If the given shape is, e.g., \code{(m, n, k)}, then
\code{m * n * k} samples are drawn.

\end{description}
\begin{description}
\item[{samples}] \leavevmode{[}scalar or ndarray{]}
Drawn samples.

\end{description}

When calculating the power of an experiment (power = probability of
rejecting the null hypothesis when a specific alternative is true) the
non-central F statistic becomes important.  When the null hypothesis is
true, the F statistic follows a central F distribution. When the null
hypothesis is not true, then it follows a non-central F statistic.

Weisstein, Eric W. ``Noncentral F-Distribution.'' From MathWorld--A Wolfram
Web Resource.  \href{http://mathworld.wolfram.com/NoncentralF-Distribution.html}{http://mathworld.wolfram.com/NoncentralF-Distribution.html}

Wikipedia, ``Noncentral F distribution'',
\href{http://en.wikipedia.org/wiki/Noncentral\_F-distribution}{http://en.wikipedia.org/wiki/Noncentral\_F-distribution}

In a study, testing for a specific alternative to the null hypothesis
requires use of the Noncentral F distribution. We need to calculate the
area in the tail of the distribution that exceeds the value of the F
distribution for the null hypothesis.  We'll plot the two probability
distributions for comparison.

\begin{Verbatim}[commandchars=\\\{\}]
\PYG{g+gp}{\PYGZgt{}\PYGZgt{}\PYGZgt{} }\PYG{n}{dfnum} \PYG{o}{=} \PYG{l+m+mi}{3} \PYG{c}{\PYGZsh{} between group deg of freedom}
\PYG{g+gp}{\PYGZgt{}\PYGZgt{}\PYGZgt{} }\PYG{n}{dfden} \PYG{o}{=} \PYG{l+m+mi}{20} \PYG{c}{\PYGZsh{} within groups degrees of freedom}
\PYG{g+gp}{\PYGZgt{}\PYGZgt{}\PYGZgt{} }\PYG{n}{nonc} \PYG{o}{=} \PYG{l+m+mf}{3.0}
\PYG{g+gp}{\PYGZgt{}\PYGZgt{}\PYGZgt{} }\PYG{n}{nc\PYGZus{}vals} \PYG{o}{=} \PYG{n}{np}\PYG{o}{.}\PYG{n}{random}\PYG{o}{.}\PYG{n}{noncentral\PYGZus{}f}\PYG{p}{(}\PYG{n}{dfnum}\PYG{p}{,} \PYG{n}{dfden}\PYG{p}{,} \PYG{n}{nonc}\PYG{p}{,} \PYG{l+m+mi}{1000000}\PYG{p}{)}
\PYG{g+gp}{\PYGZgt{}\PYGZgt{}\PYGZgt{} }\PYG{n}{NF} \PYG{o}{=} \PYG{n}{np}\PYG{o}{.}\PYG{n}{histogram}\PYG{p}{(}\PYG{n}{nc\PYGZus{}vals}\PYG{p}{,} \PYG{n}{bins}\PYG{o}{=}\PYG{l+m+mi}{50}\PYG{p}{,} \PYG{n}{normed}\PYG{o}{=}\PYG{n+nb+bp}{True}\PYG{p}{)}
\PYG{g+gp}{\PYGZgt{}\PYGZgt{}\PYGZgt{} }\PYG{n}{c\PYGZus{}vals} \PYG{o}{=} \PYG{n}{np}\PYG{o}{.}\PYG{n}{random}\PYG{o}{.}\PYG{n}{f}\PYG{p}{(}\PYG{n}{dfnum}\PYG{p}{,} \PYG{n}{dfden}\PYG{p}{,} \PYG{l+m+mi}{1000000}\PYG{p}{)}
\PYG{g+gp}{\PYGZgt{}\PYGZgt{}\PYGZgt{} }\PYG{n}{F} \PYG{o}{=} \PYG{n}{np}\PYG{o}{.}\PYG{n}{histogram}\PYG{p}{(}\PYG{n}{c\PYGZus{}vals}\PYG{p}{,} \PYG{n}{bins}\PYG{o}{=}\PYG{l+m+mi}{50}\PYG{p}{,} \PYG{n}{normed}\PYG{o}{=}\PYG{n+nb+bp}{True}\PYG{p}{)}
\PYG{g+gp}{\PYGZgt{}\PYGZgt{}\PYGZgt{} }\PYG{n}{plt}\PYG{o}{.}\PYG{n}{plot}\PYG{p}{(}\PYG{n}{F}\PYG{p}{[}\PYG{l+m+mi}{1}\PYG{p}{]}\PYG{p}{[}\PYG{l+m+mi}{1}\PYG{p}{:}\PYG{p}{]}\PYG{p}{,} \PYG{n}{F}\PYG{p}{[}\PYG{l+m+mi}{0}\PYG{p}{]}\PYG{p}{)}
\PYG{g+gp}{\PYGZgt{}\PYGZgt{}\PYGZgt{} }\PYG{n}{plt}\PYG{o}{.}\PYG{n}{plot}\PYG{p}{(}\PYG{n}{NF}\PYG{p}{[}\PYG{l+m+mi}{1}\PYG{p}{]}\PYG{p}{[}\PYG{l+m+mi}{1}\PYG{p}{:}\PYG{p}{]}\PYG{p}{,} \PYG{n}{NF}\PYG{p}{[}\PYG{l+m+mi}{0}\PYG{p}{]}\PYG{p}{)}
\PYG{g+gp}{\PYGZgt{}\PYGZgt{}\PYGZgt{} }\PYG{n}{plt}\PYG{o}{.}\PYG{n}{show}\PYG{p}{(}\PYG{p}{)}
\end{Verbatim}

\end{fulllineitems}

\index{normal() (in module graph\_chemistry\_analysis)}

\begin{fulllineitems}
\phantomsection\label{graph_chemistry_analysis:graph_chemistry_analysis.normal}\pysiglinewithargsret{\code{graph\_chemistry\_analysis.}\bfcode{normal}}{\emph{loc=0.0}, \emph{scale=1.0}, \emph{size=None}}{}
Draw random samples from a normal (Gaussian) distribution.

The probability density function of the normal distribution, first
derived by De Moivre and 200 years later by both Gauss and Laplace
independently {\color{red}\bfseries{}{[}2{]}\_}, is often called the bell curve because of
its characteristic shape (see the example below).

The normal distributions occurs often in nature.  For example, it
describes the commonly occurring distribution of samples influenced
by a large number of tiny, random disturbances, each with its own
unique distribution {\color{red}\bfseries{}{[}2{]}\_}.
\begin{description}
\item[{loc}] \leavevmode{[}float{]}
Mean (``centre'') of the distribution.

\item[{scale}] \leavevmode{[}float{]}
Standard deviation (spread or ``width'') of the distribution.

\item[{size}] \leavevmode{[}tuple of ints{]}
Output shape.  If the given shape is, e.g., \code{(m, n, k)}, then
\code{m * n * k} samples are drawn.

\end{description}
\begin{description}
\item[{scipy.stats.distributions.norm}] \leavevmode{[}probability density function,{]}
distribution or cumulative density function, etc.

\end{description}

The probability density for the Gaussian distribution is
\begin{gather}
\begin{split}p(x) = \frac{1}{\sqrt{ 2 \pi \sigma^2 }}
e^{ - \frac{ (x - \mu)^2 } {2 \sigma^2} },\end{split}\notag
\end{gather}
where \(\mu\) is the mean and \(\sigma\) the standard deviation.
The square of the standard deviation, \(\sigma^2\), is called the
variance.

The function has its peak at the mean, and its ``spread'' increases with
the standard deviation (the function reaches 0.607 times its maximum at
\(x + \sigma\) and \(x - \sigma\) {\color{red}\bfseries{}{[}2{]}\_}).  This implies that
\emph{numpy.random.normal} is more likely to return samples lying close to the
mean, rather than those far away.

Draw samples from the distribution:

\begin{Verbatim}[commandchars=\\\{\}]
\PYG{g+gp}{\PYGZgt{}\PYGZgt{}\PYGZgt{} }\PYG{n}{mu}\PYG{p}{,} \PYG{n}{sigma} \PYG{o}{=} \PYG{l+m+mi}{0}\PYG{p}{,} \PYG{l+m+mf}{0.1} \PYG{c}{\PYGZsh{} mean and standard deviation}
\PYG{g+gp}{\PYGZgt{}\PYGZgt{}\PYGZgt{} }\PYG{n}{s} \PYG{o}{=} \PYG{n}{np}\PYG{o}{.}\PYG{n}{random}\PYG{o}{.}\PYG{n}{normal}\PYG{p}{(}\PYG{n}{mu}\PYG{p}{,} \PYG{n}{sigma}\PYG{p}{,} \PYG{l+m+mi}{1000}\PYG{p}{)}
\end{Verbatim}

Verify the mean and the variance:

\begin{Verbatim}[commandchars=\\\{\}]
\PYG{g+gp}{\PYGZgt{}\PYGZgt{}\PYGZgt{} }\PYG{n+nb}{abs}\PYG{p}{(}\PYG{n}{mu} \PYG{o}{\PYGZhy{}} \PYG{n}{np}\PYG{o}{.}\PYG{n}{mean}\PYG{p}{(}\PYG{n}{s}\PYG{p}{)}\PYG{p}{)} \PYG{o}{\PYGZlt{}} \PYG{l+m+mf}{0.01}
\PYG{g+go}{True}
\end{Verbatim}

\begin{Verbatim}[commandchars=\\\{\}]
\PYG{g+gp}{\PYGZgt{}\PYGZgt{}\PYGZgt{} }\PYG{n+nb}{abs}\PYG{p}{(}\PYG{n}{sigma} \PYG{o}{\PYGZhy{}} \PYG{n}{np}\PYG{o}{.}\PYG{n}{std}\PYG{p}{(}\PYG{n}{s}\PYG{p}{,} \PYG{n}{ddof}\PYG{o}{=}\PYG{l+m+mi}{1}\PYG{p}{)}\PYG{p}{)} \PYG{o}{\PYGZlt{}} \PYG{l+m+mf}{0.01}
\PYG{g+go}{True}
\end{Verbatim}

Display the histogram of the samples, along with
the probability density function:

\begin{Verbatim}[commandchars=\\\{\}]
\PYG{g+gp}{\PYGZgt{}\PYGZgt{}\PYGZgt{} }\PYG{k+kn}{import} \PYG{n+nn}{matplotlib.pyplot} \PYG{k+kn}{as} \PYG{n+nn}{plt}
\PYG{g+gp}{\PYGZgt{}\PYGZgt{}\PYGZgt{} }\PYG{n}{count}\PYG{p}{,} \PYG{n}{bins}\PYG{p}{,} \PYG{n}{ignored} \PYG{o}{=} \PYG{n}{plt}\PYG{o}{.}\PYG{n}{hist}\PYG{p}{(}\PYG{n}{s}\PYG{p}{,} \PYG{l+m+mi}{30}\PYG{p}{,} \PYG{n}{normed}\PYG{o}{=}\PYG{n+nb+bp}{True}\PYG{p}{)}
\PYG{g+gp}{\PYGZgt{}\PYGZgt{}\PYGZgt{} }\PYG{n}{plt}\PYG{o}{.}\PYG{n}{plot}\PYG{p}{(}\PYG{n}{bins}\PYG{p}{,} \PYG{l+m+mi}{1}\PYG{o}{/}\PYG{p}{(}\PYG{n}{sigma} \PYG{o}{*} \PYG{n}{np}\PYG{o}{.}\PYG{n}{sqrt}\PYG{p}{(}\PYG{l+m+mi}{2} \PYG{o}{*} \PYG{n}{np}\PYG{o}{.}\PYG{n}{pi}\PYG{p}{)}\PYG{p}{)} \PYG{o}{*}
\PYG{g+gp}{... }               \PYG{n}{np}\PYG{o}{.}\PYG{n}{exp}\PYG{p}{(} \PYG{o}{\PYGZhy{}} \PYG{p}{(}\PYG{n}{bins} \PYG{o}{\PYGZhy{}} \PYG{n}{mu}\PYG{p}{)}\PYG{o}{*}\PYG{o}{*}\PYG{l+m+mi}{2} \PYG{o}{/} \PYG{p}{(}\PYG{l+m+mi}{2} \PYG{o}{*} \PYG{n}{sigma}\PYG{o}{*}\PYG{o}{*}\PYG{l+m+mi}{2}\PYG{p}{)} \PYG{p}{)}\PYG{p}{,}
\PYG{g+gp}{... }         \PYG{n}{linewidth}\PYG{o}{=}\PYG{l+m+mi}{2}\PYG{p}{,} \PYG{n}{color}\PYG{o}{=}\PYG{l+s}{\PYGZsq{}}\PYG{l+s}{r}\PYG{l+s}{\PYGZsq{}}\PYG{p}{)}
\PYG{g+gp}{\PYGZgt{}\PYGZgt{}\PYGZgt{} }\PYG{n}{plt}\PYG{o}{.}\PYG{n}{show}\PYG{p}{(}\PYG{p}{)}
\end{Verbatim}

\end{fulllineitems}

\index{pareto() (in module graph\_chemistry\_analysis)}

\begin{fulllineitems}
\phantomsection\label{graph_chemistry_analysis:graph_chemistry_analysis.pareto}\pysiglinewithargsret{\code{graph\_chemistry\_analysis.}\bfcode{pareto}}{\emph{a}, \emph{size=None}}{}
Draw samples from a Pareto II or Lomax distribution with specified shape.

The Lomax or Pareto II distribution is a shifted Pareto distribution. The
classical Pareto distribution can be obtained from the Lomax distribution
by adding the location parameter m, see below. The smallest value of the
Lomax distribution is zero while for the classical Pareto distribution it
is m, where the standard Pareto distribution has location m=1.
Lomax can also be considered as a simplified version of the Generalized
Pareto distribution (available in SciPy), with the scale set to one and
the location set to zero.

The Pareto distribution must be greater than zero, and is unbounded above.
It is also known as the ``80-20 rule''.  In this distribution, 80 percent of
the weights are in the lowest 20 percent of the range, while the other 20
percent fill the remaining 80 percent of the range.
\begin{description}
\item[{shape}] \leavevmode{[}float, \textgreater{} 0.{]}
Shape of the distribution.

\item[{size}] \leavevmode{[}tuple of ints{]}
Output shape.  If the given shape is, e.g., \code{(m, n, k)}, then
\code{m * n * k} samples are drawn.

\end{description}
\begin{description}
\item[{scipy.stats.distributions.lomax.pdf}] \leavevmode{[}probability density function,{]}
distribution or cumulative density function, etc.

\item[{scipy.stats.distributions.genpareto.pdf}] \leavevmode{[}probability density function,{]}
distribution or cumulative density function, etc.

\end{description}

The probability density for the Pareto distribution is
\begin{gather}
\begin{split}p(x) = \frac{am^a}{x^{a+1}}\end{split}\notag
\end{gather}
where \(a\) is the shape and \(m\) the location

The Pareto distribution, named after the Italian economist Vilfredo Pareto,
is a power law probability distribution useful in many real world problems.
Outside the field of economics it is generally referred to as the Bradford
distribution. Pareto developed the distribution to describe the
distribution of wealth in an economy.  It has also found use in insurance,
web page access statistics, oil field sizes, and many other problems,
including the download frequency for projects in Sourceforge {[}1{]}.  It is
one of the so-called ``fat-tailed'' distributions.

Draw samples from the distribution:

\begin{Verbatim}[commandchars=\\\{\}]
\PYG{g+gp}{\PYGZgt{}\PYGZgt{}\PYGZgt{} }\PYG{n}{a}\PYG{p}{,} \PYG{n}{m} \PYG{o}{=} \PYG{l+m+mf}{3.}\PYG{p}{,} \PYG{l+m+mf}{1.} \PYG{c}{\PYGZsh{} shape and mode}
\PYG{g+gp}{\PYGZgt{}\PYGZgt{}\PYGZgt{} }\PYG{n}{s} \PYG{o}{=} \PYG{n}{np}\PYG{o}{.}\PYG{n}{random}\PYG{o}{.}\PYG{n}{pareto}\PYG{p}{(}\PYG{n}{a}\PYG{p}{,} \PYG{l+m+mi}{1000}\PYG{p}{)} \PYG{o}{+} \PYG{n}{m}
\end{Verbatim}

Display the histogram of the samples, along with
the probability density function:

\begin{Verbatim}[commandchars=\\\{\}]
\PYG{g+gp}{\PYGZgt{}\PYGZgt{}\PYGZgt{} }\PYG{k+kn}{import} \PYG{n+nn}{matplotlib.pyplot} \PYG{k+kn}{as} \PYG{n+nn}{plt}
\PYG{g+gp}{\PYGZgt{}\PYGZgt{}\PYGZgt{} }\PYG{n}{count}\PYG{p}{,} \PYG{n}{bins}\PYG{p}{,} \PYG{n}{ignored} \PYG{o}{=} \PYG{n}{plt}\PYG{o}{.}\PYG{n}{hist}\PYG{p}{(}\PYG{n}{s}\PYG{p}{,} \PYG{l+m+mi}{100}\PYG{p}{,} \PYG{n}{normed}\PYG{o}{=}\PYG{n+nb+bp}{True}\PYG{p}{,} \PYG{n}{align}\PYG{o}{=}\PYG{l+s}{\PYGZsq{}}\PYG{l+s}{center}\PYG{l+s}{\PYGZsq{}}\PYG{p}{)}
\PYG{g+gp}{\PYGZgt{}\PYGZgt{}\PYGZgt{} }\PYG{n}{fit} \PYG{o}{=} \PYG{n}{a}\PYG{o}{*}\PYG{n}{m}\PYG{o}{*}\PYG{o}{*}\PYG{n}{a}\PYG{o}{/}\PYG{n}{bins}\PYG{o}{*}\PYG{o}{*}\PYG{p}{(}\PYG{n}{a}\PYG{o}{+}\PYG{l+m+mi}{1}\PYG{p}{)}
\PYG{g+gp}{\PYGZgt{}\PYGZgt{}\PYGZgt{} }\PYG{n}{plt}\PYG{o}{.}\PYG{n}{plot}\PYG{p}{(}\PYG{n}{bins}\PYG{p}{,} \PYG{n+nb}{max}\PYG{p}{(}\PYG{n}{count}\PYG{p}{)}\PYG{o}{*}\PYG{n}{fit}\PYG{o}{/}\PYG{n+nb}{max}\PYG{p}{(}\PYG{n}{fit}\PYG{p}{)}\PYG{p}{,}\PYG{n}{linewidth}\PYG{o}{=}\PYG{l+m+mi}{2}\PYG{p}{,} \PYG{n}{color}\PYG{o}{=}\PYG{l+s}{\PYGZsq{}}\PYG{l+s}{r}\PYG{l+s}{\PYGZsq{}}\PYG{p}{)}
\PYG{g+gp}{\PYGZgt{}\PYGZgt{}\PYGZgt{} }\PYG{n}{plt}\PYG{o}{.}\PYG{n}{show}\PYG{p}{(}\PYG{p}{)}
\end{Verbatim}

\end{fulllineitems}

\index{permutation() (in module graph\_chemistry\_analysis)}

\begin{fulllineitems}
\phantomsection\label{graph_chemistry_analysis:graph_chemistry_analysis.permutation}\pysiglinewithargsret{\code{graph\_chemistry\_analysis.}\bfcode{permutation}}{\emph{x}}{}
Randomly permute a sequence, or return a permuted range.

If \emph{x} is a multi-dimensional array, it is only shuffled along its
first index.
\begin{description}
\item[{x}] \leavevmode{[}int or array\_like{]}
If \emph{x} is an integer, randomly permute \code{np.arange(x)}.
If \emph{x} is an array, make a copy and shuffle the elements
randomly.

\end{description}
\begin{description}
\item[{out}] \leavevmode{[}ndarray{]}
Permuted sequence or array range.

\end{description}

\begin{Verbatim}[commandchars=\\\{\}]
\PYG{g+gp}{\PYGZgt{}\PYGZgt{}\PYGZgt{} }\PYG{n}{np}\PYG{o}{.}\PYG{n}{random}\PYG{o}{.}\PYG{n}{permutation}\PYG{p}{(}\PYG{l+m+mi}{10}\PYG{p}{)}
\PYG{g+go}{array([1, 7, 4, 3, 0, 9, 2, 5, 8, 6])}
\end{Verbatim}

\begin{Verbatim}[commandchars=\\\{\}]
\PYG{g+gp}{\PYGZgt{}\PYGZgt{}\PYGZgt{} }\PYG{n}{np}\PYG{o}{.}\PYG{n}{random}\PYG{o}{.}\PYG{n}{permutation}\PYG{p}{(}\PYG{p}{[}\PYG{l+m+mi}{1}\PYG{p}{,} \PYG{l+m+mi}{4}\PYG{p}{,} \PYG{l+m+mi}{9}\PYG{p}{,} \PYG{l+m+mi}{12}\PYG{p}{,} \PYG{l+m+mi}{15}\PYG{p}{]}\PYG{p}{)}
\PYG{g+go}{array([15,  1,  9,  4, 12])}
\end{Verbatim}

\begin{Verbatim}[commandchars=\\\{\}]
\PYG{g+gp}{\PYGZgt{}\PYGZgt{}\PYGZgt{} }\PYG{n}{arr} \PYG{o}{=} \PYG{n}{np}\PYG{o}{.}\PYG{n}{arange}\PYG{p}{(}\PYG{l+m+mi}{9}\PYG{p}{)}\PYG{o}{.}\PYG{n}{reshape}\PYG{p}{(}\PYG{p}{(}\PYG{l+m+mi}{3}\PYG{p}{,} \PYG{l+m+mi}{3}\PYG{p}{)}\PYG{p}{)}
\PYG{g+gp}{\PYGZgt{}\PYGZgt{}\PYGZgt{} }\PYG{n}{np}\PYG{o}{.}\PYG{n}{random}\PYG{o}{.}\PYG{n}{permutation}\PYG{p}{(}\PYG{n}{arr}\PYG{p}{)}
\PYG{g+go}{array([[6, 7, 8],}
\PYG{g+go}{       [0, 1, 2],}
\PYG{g+go}{       [3, 4, 5]])}
\end{Verbatim}

\end{fulllineitems}

\index{poisson() (in module graph\_chemistry\_analysis)}

\begin{fulllineitems}
\phantomsection\label{graph_chemistry_analysis:graph_chemistry_analysis.poisson}\pysiglinewithargsret{\code{graph\_chemistry\_analysis.}\bfcode{poisson}}{\emph{lam=1.0}, \emph{size=None}}{}
Draw samples from a Poisson distribution.

The Poisson distribution is the limit of the Binomial
distribution for large N.
\begin{description}
\item[{lam}] \leavevmode{[}float{]}
Expectation of interval, should be \textgreater{}= 0.

\item[{size}] \leavevmode{[}int or tuple of ints, optional{]}
Output shape. If the given shape is, e.g., \code{(m, n, k)}, then
\code{m * n * k} samples are drawn.

\end{description}

The Poisson distribution
\begin{gather}
\begin{split}f(k; \lambda)=\frac{\lambda^k e^{-\lambda}}{k!}\end{split}\notag
\end{gather}
For events with an expected separation \(\lambda\) the Poisson
distribution \(f(k; \lambda)\) describes the probability of
\(k\) events occurring within the observed interval \(\lambda\).

Because the output is limited to the range of the C long type, a
ValueError is raised when \emph{lam} is within 10 sigma of the maximum
representable value.

Draw samples from the distribution:

\begin{Verbatim}[commandchars=\\\{\}]
\PYG{g+gp}{\PYGZgt{}\PYGZgt{}\PYGZgt{} }\PYG{k+kn}{import} \PYG{n+nn}{numpy} \PYG{k+kn}{as} \PYG{n+nn}{np}
\PYG{g+gp}{\PYGZgt{}\PYGZgt{}\PYGZgt{} }\PYG{n}{s} \PYG{o}{=} \PYG{n}{np}\PYG{o}{.}\PYG{n}{random}\PYG{o}{.}\PYG{n}{poisson}\PYG{p}{(}\PYG{l+m+mi}{5}\PYG{p}{,} \PYG{l+m+mi}{10000}\PYG{p}{)}
\end{Verbatim}

Display histogram of the sample:

\begin{Verbatim}[commandchars=\\\{\}]
\PYG{g+gp}{\PYGZgt{}\PYGZgt{}\PYGZgt{} }\PYG{k+kn}{import} \PYG{n+nn}{matplotlib.pyplot} \PYG{k+kn}{as} \PYG{n+nn}{plt}
\PYG{g+gp}{\PYGZgt{}\PYGZgt{}\PYGZgt{} }\PYG{n}{count}\PYG{p}{,} \PYG{n}{bins}\PYG{p}{,} \PYG{n}{ignored} \PYG{o}{=} \PYG{n}{plt}\PYG{o}{.}\PYG{n}{hist}\PYG{p}{(}\PYG{n}{s}\PYG{p}{,} \PYG{l+m+mi}{14}\PYG{p}{,} \PYG{n}{normed}\PYG{o}{=}\PYG{n+nb+bp}{True}\PYG{p}{)}
\PYG{g+gp}{\PYGZgt{}\PYGZgt{}\PYGZgt{} }\PYG{n}{plt}\PYG{o}{.}\PYG{n}{show}\PYG{p}{(}\PYG{p}{)}
\end{Verbatim}

\end{fulllineitems}

\index{power() (in module graph\_chemistry\_analysis)}

\begin{fulllineitems}
\phantomsection\label{graph_chemistry_analysis:graph_chemistry_analysis.power}\pysiglinewithargsret{\code{graph\_chemistry\_analysis.}\bfcode{power}}{\emph{a}, \emph{size=None}}{}
Draws samples in {[}0, 1{]} from a power distribution with positive
exponent a - 1.

Also known as the power function distribution.
\begin{description}
\item[{a}] \leavevmode{[}float{]}
parameter, \textgreater{} 0

\item[{size}] \leavevmode{[}tuple of ints{]}\begin{description}
\item[{Output shape.  If the given shape is, e.g., \code{(m, n, k)}, then}] \leavevmode
\code{m * n * k} samples are drawn.

\end{description}

\end{description}
\begin{description}
\item[{samples}] \leavevmode{[}\{ndarray, scalar\}{]}
The returned samples lie in {[}0, 1{]}.

\end{description}
\begin{description}
\item[{ValueError}] \leavevmode
If a\textless{}1.

\end{description}

The probability density function is
\begin{gather}
\begin{split}P(x; a) = ax^{a-1}, 0 \le x \le 1, a>0.\end{split}\notag
\end{gather}
The power function distribution is just the inverse of the Pareto
distribution. It may also be seen as a special case of the Beta
distribution.

It is used, for example, in modeling the over-reporting of insurance
claims.

Draw samples from the distribution:

\begin{Verbatim}[commandchars=\\\{\}]
\PYG{g+gp}{\PYGZgt{}\PYGZgt{}\PYGZgt{} }\PYG{n}{a} \PYG{o}{=} \PYG{l+m+mf}{5.} \PYG{c}{\PYGZsh{} shape}
\PYG{g+gp}{\PYGZgt{}\PYGZgt{}\PYGZgt{} }\PYG{n}{samples} \PYG{o}{=} \PYG{l+m+mi}{1000}
\PYG{g+gp}{\PYGZgt{}\PYGZgt{}\PYGZgt{} }\PYG{n}{s} \PYG{o}{=} \PYG{n}{np}\PYG{o}{.}\PYG{n}{random}\PYG{o}{.}\PYG{n}{power}\PYG{p}{(}\PYG{n}{a}\PYG{p}{,} \PYG{n}{samples}\PYG{p}{)}
\end{Verbatim}

Display the histogram of the samples, along with
the probability density function:

\begin{Verbatim}[commandchars=\\\{\}]
\PYG{g+gp}{\PYGZgt{}\PYGZgt{}\PYGZgt{} }\PYG{k+kn}{import} \PYG{n+nn}{matplotlib.pyplot} \PYG{k+kn}{as} \PYG{n+nn}{plt}
\PYG{g+gp}{\PYGZgt{}\PYGZgt{}\PYGZgt{} }\PYG{n}{count}\PYG{p}{,} \PYG{n}{bins}\PYG{p}{,} \PYG{n}{ignored} \PYG{o}{=} \PYG{n}{plt}\PYG{o}{.}\PYG{n}{hist}\PYG{p}{(}\PYG{n}{s}\PYG{p}{,} \PYG{n}{bins}\PYG{o}{=}\PYG{l+m+mi}{30}\PYG{p}{)}
\PYG{g+gp}{\PYGZgt{}\PYGZgt{}\PYGZgt{} }\PYG{n}{x} \PYG{o}{=} \PYG{n}{np}\PYG{o}{.}\PYG{n}{linspace}\PYG{p}{(}\PYG{l+m+mi}{0}\PYG{p}{,} \PYG{l+m+mi}{1}\PYG{p}{,} \PYG{l+m+mi}{100}\PYG{p}{)}
\PYG{g+gp}{\PYGZgt{}\PYGZgt{}\PYGZgt{} }\PYG{n}{y} \PYG{o}{=} \PYG{n}{a}\PYG{o}{*}\PYG{n}{x}\PYG{o}{*}\PYG{o}{*}\PYG{p}{(}\PYG{n}{a}\PYG{o}{\PYGZhy{}}\PYG{l+m+mf}{1.}\PYG{p}{)}
\PYG{g+gp}{\PYGZgt{}\PYGZgt{}\PYGZgt{} }\PYG{n}{normed\PYGZus{}y} \PYG{o}{=} \PYG{n}{samples}\PYG{o}{*}\PYG{n}{np}\PYG{o}{.}\PYG{n}{diff}\PYG{p}{(}\PYG{n}{bins}\PYG{p}{)}\PYG{p}{[}\PYG{l+m+mi}{0}\PYG{p}{]}\PYG{o}{*}\PYG{n}{y}
\PYG{g+gp}{\PYGZgt{}\PYGZgt{}\PYGZgt{} }\PYG{n}{plt}\PYG{o}{.}\PYG{n}{plot}\PYG{p}{(}\PYG{n}{x}\PYG{p}{,} \PYG{n}{normed\PYGZus{}y}\PYG{p}{)}
\PYG{g+gp}{\PYGZgt{}\PYGZgt{}\PYGZgt{} }\PYG{n}{plt}\PYG{o}{.}\PYG{n}{show}\PYG{p}{(}\PYG{p}{)}
\end{Verbatim}

Compare the power function distribution to the inverse of the Pareto.

\begin{Verbatim}[commandchars=\\\{\}]
\PYG{g+gp}{\PYGZgt{}\PYGZgt{}\PYGZgt{} }\PYG{k+kn}{from} \PYG{n+nn}{scipy} \PYG{k+kn}{import} \PYG{n}{stats}
\PYG{g+gp}{\PYGZgt{}\PYGZgt{}\PYGZgt{} }\PYG{n}{rvs} \PYG{o}{=} \PYG{n}{np}\PYG{o}{.}\PYG{n}{random}\PYG{o}{.}\PYG{n}{power}\PYG{p}{(}\PYG{l+m+mi}{5}\PYG{p}{,} \PYG{l+m+mi}{1000000}\PYG{p}{)}
\PYG{g+gp}{\PYGZgt{}\PYGZgt{}\PYGZgt{} }\PYG{n}{rvsp} \PYG{o}{=} \PYG{n}{np}\PYG{o}{.}\PYG{n}{random}\PYG{o}{.}\PYG{n}{pareto}\PYG{p}{(}\PYG{l+m+mi}{5}\PYG{p}{,} \PYG{l+m+mi}{1000000}\PYG{p}{)}
\PYG{g+gp}{\PYGZgt{}\PYGZgt{}\PYGZgt{} }\PYG{n}{xx} \PYG{o}{=} \PYG{n}{np}\PYG{o}{.}\PYG{n}{linspace}\PYG{p}{(}\PYG{l+m+mi}{0}\PYG{p}{,}\PYG{l+m+mi}{1}\PYG{p}{,}\PYG{l+m+mi}{100}\PYG{p}{)}
\PYG{g+gp}{\PYGZgt{}\PYGZgt{}\PYGZgt{} }\PYG{n}{powpdf} \PYG{o}{=} \PYG{n}{stats}\PYG{o}{.}\PYG{n}{powerlaw}\PYG{o}{.}\PYG{n}{pdf}\PYG{p}{(}\PYG{n}{xx}\PYG{p}{,}\PYG{l+m+mi}{5}\PYG{p}{)}
\end{Verbatim}

\begin{Verbatim}[commandchars=\\\{\}]
\PYG{g+gp}{\PYGZgt{}\PYGZgt{}\PYGZgt{} }\PYG{n}{plt}\PYG{o}{.}\PYG{n}{figure}\PYG{p}{(}\PYG{p}{)}
\PYG{g+gp}{\PYGZgt{}\PYGZgt{}\PYGZgt{} }\PYG{n}{plt}\PYG{o}{.}\PYG{n}{hist}\PYG{p}{(}\PYG{n}{rvs}\PYG{p}{,} \PYG{n}{bins}\PYG{o}{=}\PYG{l+m+mi}{50}\PYG{p}{,} \PYG{n}{normed}\PYG{o}{=}\PYG{n+nb+bp}{True}\PYG{p}{)}
\PYG{g+gp}{\PYGZgt{}\PYGZgt{}\PYGZgt{} }\PYG{n}{plt}\PYG{o}{.}\PYG{n}{plot}\PYG{p}{(}\PYG{n}{xx}\PYG{p}{,}\PYG{n}{powpdf}\PYG{p}{,}\PYG{l+s}{\PYGZsq{}}\PYG{l+s}{r\PYGZhy{}}\PYG{l+s}{\PYGZsq{}}\PYG{p}{)}
\PYG{g+gp}{\PYGZgt{}\PYGZgt{}\PYGZgt{} }\PYG{n}{plt}\PYG{o}{.}\PYG{n}{title}\PYG{p}{(}\PYG{l+s}{\PYGZsq{}}\PYG{l+s}{np.random.power(5)}\PYG{l+s}{\PYGZsq{}}\PYG{p}{)}
\end{Verbatim}

\begin{Verbatim}[commandchars=\\\{\}]
\PYG{g+gp}{\PYGZgt{}\PYGZgt{}\PYGZgt{} }\PYG{n}{plt}\PYG{o}{.}\PYG{n}{figure}\PYG{p}{(}\PYG{p}{)}
\PYG{g+gp}{\PYGZgt{}\PYGZgt{}\PYGZgt{} }\PYG{n}{plt}\PYG{o}{.}\PYG{n}{hist}\PYG{p}{(}\PYG{l+m+mf}{1.}\PYG{o}{/}\PYG{p}{(}\PYG{l+m+mf}{1.}\PYG{o}{+}\PYG{n}{rvsp}\PYG{p}{)}\PYG{p}{,} \PYG{n}{bins}\PYG{o}{=}\PYG{l+m+mi}{50}\PYG{p}{,} \PYG{n}{normed}\PYG{o}{=}\PYG{n+nb+bp}{True}\PYG{p}{)}
\PYG{g+gp}{\PYGZgt{}\PYGZgt{}\PYGZgt{} }\PYG{n}{plt}\PYG{o}{.}\PYG{n}{plot}\PYG{p}{(}\PYG{n}{xx}\PYG{p}{,}\PYG{n}{powpdf}\PYG{p}{,}\PYG{l+s}{\PYGZsq{}}\PYG{l+s}{r\PYGZhy{}}\PYG{l+s}{\PYGZsq{}}\PYG{p}{)}
\PYG{g+gp}{\PYGZgt{}\PYGZgt{}\PYGZgt{} }\PYG{n}{plt}\PYG{o}{.}\PYG{n}{title}\PYG{p}{(}\PYG{l+s}{\PYGZsq{}}\PYG{l+s}{inverse of 1 + np.random.pareto(5)}\PYG{l+s}{\PYGZsq{}}\PYG{p}{)}
\end{Verbatim}

\begin{Verbatim}[commandchars=\\\{\}]
\PYG{g+gp}{\PYGZgt{}\PYGZgt{}\PYGZgt{} }\PYG{n}{plt}\PYG{o}{.}\PYG{n}{figure}\PYG{p}{(}\PYG{p}{)}
\PYG{g+gp}{\PYGZgt{}\PYGZgt{}\PYGZgt{} }\PYG{n}{plt}\PYG{o}{.}\PYG{n}{hist}\PYG{p}{(}\PYG{l+m+mf}{1.}\PYG{o}{/}\PYG{p}{(}\PYG{l+m+mf}{1.}\PYG{o}{+}\PYG{n}{rvsp}\PYG{p}{)}\PYG{p}{,} \PYG{n}{bins}\PYG{o}{=}\PYG{l+m+mi}{50}\PYG{p}{,} \PYG{n}{normed}\PYG{o}{=}\PYG{n+nb+bp}{True}\PYG{p}{)}
\PYG{g+gp}{\PYGZgt{}\PYGZgt{}\PYGZgt{} }\PYG{n}{plt}\PYG{o}{.}\PYG{n}{plot}\PYG{p}{(}\PYG{n}{xx}\PYG{p}{,}\PYG{n}{powpdf}\PYG{p}{,}\PYG{l+s}{\PYGZsq{}}\PYG{l+s}{r\PYGZhy{}}\PYG{l+s}{\PYGZsq{}}\PYG{p}{)}
\PYG{g+gp}{\PYGZgt{}\PYGZgt{}\PYGZgt{} }\PYG{n}{plt}\PYG{o}{.}\PYG{n}{title}\PYG{p}{(}\PYG{l+s}{\PYGZsq{}}\PYG{l+s}{inverse of stats.pareto(5)}\PYG{l+s}{\PYGZsq{}}\PYG{p}{)}
\end{Verbatim}

\end{fulllineitems}

\index{rand() (in module graph\_chemistry\_analysis)}

\begin{fulllineitems}
\phantomsection\label{graph_chemistry_analysis:graph_chemistry_analysis.rand}\pysiglinewithargsret{\code{graph\_chemistry\_analysis.}\bfcode{rand}}{\emph{d0}, \emph{d1}, \emph{...}, \emph{dn}}{}
Random values in a given shape.

Create an array of the given shape and propagate it with
random samples from a uniform distribution
over \code{{[}0, 1)}.
\begin{description}
\item[{d0, d1, ..., dn}] \leavevmode{[}int, optional{]}
The dimensions of the returned array, should all be positive.
If no argument is given a single Python float is returned.

\end{description}
\begin{description}
\item[{out}] \leavevmode{[}ndarray, shape \code{(d0, d1, ..., dn)}{]}
Random values.

\end{description}

random

This is a convenience function. If you want an interface that
takes a shape-tuple as the first argument, refer to
np.random.random\_sample .

\begin{Verbatim}[commandchars=\\\{\}]
\PYG{g+gp}{\PYGZgt{}\PYGZgt{}\PYGZgt{} }\PYG{n}{np}\PYG{o}{.}\PYG{n}{random}\PYG{o}{.}\PYG{n}{rand}\PYG{p}{(}\PYG{l+m+mi}{3}\PYG{p}{,}\PYG{l+m+mi}{2}\PYG{p}{)}
\PYG{g+go}{array([[ 0.14022471,  0.96360618],  \PYGZsh{}random}
\PYG{g+go}{       [ 0.37601032,  0.25528411],  \PYGZsh{}random}
\PYG{g+go}{       [ 0.49313049,  0.94909878]]) \PYGZsh{}random}
\end{Verbatim}

\end{fulllineitems}

\index{randint() (in module graph\_chemistry\_analysis)}

\begin{fulllineitems}
\phantomsection\label{graph_chemistry_analysis:graph_chemistry_analysis.randint}\pysiglinewithargsret{\code{graph\_chemistry\_analysis.}\bfcode{randint}}{\emph{low}, \emph{high=None}, \emph{size=None}}{}
Return random integers from \emph{low} (inclusive) to \emph{high} (exclusive).

Return random integers from the ``discrete uniform'' distribution in the
``half-open'' interval {[}\emph{low}, \emph{high}). If \emph{high} is None (the default),
then results are from {[}0, \emph{low}).
\begin{description}
\item[{low}] \leavevmode{[}int{]}
Lowest (signed) integer to be drawn from the distribution (unless
\code{high=None}, in which case this parameter is the \emph{highest} such
integer).

\item[{high}] \leavevmode{[}int, optional{]}
If provided, one above the largest (signed) integer to be drawn
from the distribution (see above for behavior if \code{high=None}).

\item[{size}] \leavevmode{[}int or tuple of ints, optional{]}
Output shape. Default is None, in which case a single int is
returned.

\end{description}
\begin{description}
\item[{out}] \leavevmode{[}int or ndarray of ints{]}
\emph{size}-shaped array of random integers from the appropriate
distribution, or a single such random int if \emph{size} not provided.

\end{description}
\begin{description}
\item[{random.random\_integers}] \leavevmode{[}similar to \emph{randint}, only for the closed{]}
interval {[}\emph{low}, \emph{high}{]}, and 1 is the lowest value if \emph{high} is
omitted. In particular, this other one is the one to use to generate
uniformly distributed discrete non-integers.

\end{description}

\begin{Verbatim}[commandchars=\\\{\}]
\PYG{g+gp}{\PYGZgt{}\PYGZgt{}\PYGZgt{} }\PYG{n}{np}\PYG{o}{.}\PYG{n}{random}\PYG{o}{.}\PYG{n}{randint}\PYG{p}{(}\PYG{l+m+mi}{2}\PYG{p}{,} \PYG{n}{size}\PYG{o}{=}\PYG{l+m+mi}{10}\PYG{p}{)}
\PYG{g+go}{array([1, 0, 0, 0, 1, 1, 0, 0, 1, 0])}
\PYG{g+gp}{\PYGZgt{}\PYGZgt{}\PYGZgt{} }\PYG{n}{np}\PYG{o}{.}\PYG{n}{random}\PYG{o}{.}\PYG{n}{randint}\PYG{p}{(}\PYG{l+m+mi}{1}\PYG{p}{,} \PYG{n}{size}\PYG{o}{=}\PYG{l+m+mi}{10}\PYG{p}{)}
\PYG{g+go}{array([0, 0, 0, 0, 0, 0, 0, 0, 0, 0])}
\end{Verbatim}

Generate a 2 x 4 array of ints between 0 and 4, inclusive:

\begin{Verbatim}[commandchars=\\\{\}]
\PYG{g+gp}{\PYGZgt{}\PYGZgt{}\PYGZgt{} }\PYG{n}{np}\PYG{o}{.}\PYG{n}{random}\PYG{o}{.}\PYG{n}{randint}\PYG{p}{(}\PYG{l+m+mi}{5}\PYG{p}{,} \PYG{n}{size}\PYG{o}{=}\PYG{p}{(}\PYG{l+m+mi}{2}\PYG{p}{,} \PYG{l+m+mi}{4}\PYG{p}{)}\PYG{p}{)}
\PYG{g+go}{array([[4, 0, 2, 1],}
\PYG{g+go}{       [3, 2, 2, 0]])}
\end{Verbatim}

\end{fulllineitems}

\index{randn() (in module graph\_chemistry\_analysis)}

\begin{fulllineitems}
\phantomsection\label{graph_chemistry_analysis:graph_chemistry_analysis.randn}\pysiglinewithargsret{\code{graph\_chemistry\_analysis.}\bfcode{randn}}{\emph{d0}, \emph{d1}, \emph{...}, \emph{dn}}{}
Return a sample (or samples) from the ``standard normal'' distribution.

If positive, int\_like or int-convertible arguments are provided,
\emph{randn} generates an array of shape \code{(d0, d1, ..., dn)}, filled
with random floats sampled from a univariate ``normal'' (Gaussian)
distribution of mean 0 and variance 1 (if any of the \(d_i\) are
floats, they are first converted to integers by truncation). A single
float randomly sampled from the distribution is returned if no
argument is provided.

This is a convenience function.  If you want an interface that takes a
tuple as the first argument, use \emph{numpy.random.standard\_normal} instead.
\begin{description}
\item[{d0, d1, ..., dn}] \leavevmode{[}int, optional{]}
The dimensions of the returned array, should be all positive.
If no argument is given a single Python float is returned.

\end{description}
\begin{description}
\item[{Z}] \leavevmode{[}ndarray or float{]}
A \code{(d0, d1, ..., dn)}-shaped array of floating-point samples from
the standard normal distribution, or a single such float if
no parameters were supplied.

\end{description}

random.standard\_normal : Similar, but takes a tuple as its argument.

For random samples from \(N(\mu, \sigma^2)\), use:

\code{sigma * np.random.randn(...) + mu}

\begin{Verbatim}[commandchars=\\\{\}]
\PYG{g+gp}{\PYGZgt{}\PYGZgt{}\PYGZgt{} }\PYG{n}{np}\PYG{o}{.}\PYG{n}{random}\PYG{o}{.}\PYG{n}{randn}\PYG{p}{(}\PYG{p}{)}
\PYG{g+go}{2.1923875335537315 \PYGZsh{}random}
\end{Verbatim}

Two-by-four array of samples from N(3, 6.25):

\begin{Verbatim}[commandchars=\\\{\}]
\PYG{g+gp}{\PYGZgt{}\PYGZgt{}\PYGZgt{} }\PYG{l+m+mf}{2.5} \PYG{o}{*} \PYG{n}{np}\PYG{o}{.}\PYG{n}{random}\PYG{o}{.}\PYG{n}{randn}\PYG{p}{(}\PYG{l+m+mi}{2}\PYG{p}{,} \PYG{l+m+mi}{4}\PYG{p}{)} \PYG{o}{+} \PYG{l+m+mi}{3}
\PYG{g+go}{array([[\PYGZhy{}4.49401501,  4.00950034, \PYGZhy{}1.81814867,  7.29718677],  \PYGZsh{}random}
\PYG{g+go}{       [ 0.39924804,  4.68456316,  4.99394529,  4.84057254]]) \PYGZsh{}random}
\end{Verbatim}

\end{fulllineitems}

\index{random() (in module graph\_chemistry\_analysis)}

\begin{fulllineitems}
\phantomsection\label{graph_chemistry_analysis:graph_chemistry_analysis.random}\pysiglinewithargsret{\code{graph\_chemistry\_analysis.}\bfcode{random}}{}{}
random\_sample(size=None)

Return random floats in the half-open interval {[}0.0, 1.0).

Results are from the ``continuous uniform'' distribution over the
stated interval.  To sample \(Unif[a, b), b > a\) multiply
the output of \emph{random\_sample} by \emph{(b-a)} and add \emph{a}:

\begin{Verbatim}[commandchars=\\\{\}]
\PYG{p}{(}\PYG{n}{b} \PYG{o}{\PYGZhy{}} \PYG{n}{a}\PYG{p}{)} \PYG{o}{*} \PYG{n}{random\PYGZus{}sample}\PYG{p}{(}\PYG{p}{)} \PYG{o}{+} \PYG{n}{a}
\end{Verbatim}
\begin{description}
\item[{size}] \leavevmode{[}int or tuple of ints, optional{]}
Defines the shape of the returned array of random floats. If None
(the default), returns a single float.

\end{description}
\begin{description}
\item[{out}] \leavevmode{[}float or ndarray of floats{]}
Array of random floats of shape \emph{size} (unless \code{size=None}, in which
case a single float is returned).

\end{description}

\begin{Verbatim}[commandchars=\\\{\}]
\PYG{g+gp}{\PYGZgt{}\PYGZgt{}\PYGZgt{} }\PYG{n}{np}\PYG{o}{.}\PYG{n}{random}\PYG{o}{.}\PYG{n}{random\PYGZus{}sample}\PYG{p}{(}\PYG{p}{)}
\PYG{g+go}{0.47108547995356098}
\PYG{g+gp}{\PYGZgt{}\PYGZgt{}\PYGZgt{} }\PYG{n+nb}{type}\PYG{p}{(}\PYG{n}{np}\PYG{o}{.}\PYG{n}{random}\PYG{o}{.}\PYG{n}{random\PYGZus{}sample}\PYG{p}{(}\PYG{p}{)}\PYG{p}{)}
\PYG{g+go}{\PYGZlt{}type \PYGZsq{}float\PYGZsq{}\PYGZgt{}}
\PYG{g+gp}{\PYGZgt{}\PYGZgt{}\PYGZgt{} }\PYG{n}{np}\PYG{o}{.}\PYG{n}{random}\PYG{o}{.}\PYG{n}{random\PYGZus{}sample}\PYG{p}{(}\PYG{p}{(}\PYG{l+m+mi}{5}\PYG{p}{,}\PYG{p}{)}\PYG{p}{)}
\PYG{g+go}{array([ 0.30220482,  0.86820401,  0.1654503 ,  0.11659149,  0.54323428])}
\end{Verbatim}

Three-by-two array of random numbers from {[}-5, 0):

\begin{Verbatim}[commandchars=\\\{\}]
\PYG{g+gp}{\PYGZgt{}\PYGZgt{}\PYGZgt{} }\PYG{l+m+mi}{5} \PYG{o}{*} \PYG{n}{np}\PYG{o}{.}\PYG{n}{random}\PYG{o}{.}\PYG{n}{random\PYGZus{}sample}\PYG{p}{(}\PYG{p}{(}\PYG{l+m+mi}{3}\PYG{p}{,} \PYG{l+m+mi}{2}\PYG{p}{)}\PYG{p}{)} \PYG{o}{\PYGZhy{}} \PYG{l+m+mi}{5}
\PYG{g+go}{array([[\PYGZhy{}3.99149989, \PYGZhy{}0.52338984],}
\PYG{g+go}{       [\PYGZhy{}2.99091858, \PYGZhy{}0.79479508],}
\PYG{g+go}{       [\PYGZhy{}1.23204345, \PYGZhy{}1.75224494]])}
\end{Verbatim}

\end{fulllineitems}

\index{random\_integers() (in module graph\_chemistry\_analysis)}

\begin{fulllineitems}
\phantomsection\label{graph_chemistry_analysis:graph_chemistry_analysis.random_integers}\pysiglinewithargsret{\code{graph\_chemistry\_analysis.}\bfcode{random\_integers}}{\emph{low}, \emph{high=None}, \emph{size=None}}{}
Return random integers between \emph{low} and \emph{high}, inclusive.

Return random integers from the ``discrete uniform'' distribution in the
closed interval {[}\emph{low}, \emph{high}{]}.  If \emph{high} is None (the default),
then results are from {[}1, \emph{low}{]}.
\begin{description}
\item[{low}] \leavevmode{[}int{]}
Lowest (signed) integer to be drawn from the distribution (unless
\code{high=None}, in which case this parameter is the \emph{highest} such
integer).

\item[{high}] \leavevmode{[}int, optional{]}
If provided, the largest (signed) integer to be drawn from the
distribution (see above for behavior if \code{high=None}).

\item[{size}] \leavevmode{[}int or tuple of ints, optional{]}
Output shape. Default is None, in which case a single int is returned.

\end{description}
\begin{description}
\item[{out}] \leavevmode{[}int or ndarray of ints{]}
\emph{size}-shaped array of random integers from the appropriate
distribution, or a single such random int if \emph{size} not provided.

\end{description}
\begin{description}
\item[{random.randint}] \leavevmode{[}Similar to \emph{random\_integers}, only for the half-open{]}
interval {[}\emph{low}, \emph{high}), and 0 is the lowest value if \emph{high} is
omitted.

\end{description}

To sample from N evenly spaced floating-point numbers between a and b,
use:

\begin{Verbatim}[commandchars=\\\{\}]
\PYG{n}{a} \PYG{o}{+} \PYG{p}{(}\PYG{n}{b} \PYG{o}{\PYGZhy{}} \PYG{n}{a}\PYG{p}{)} \PYG{o}{*} \PYG{p}{(}\PYG{n}{np}\PYG{o}{.}\PYG{n}{random}\PYG{o}{.}\PYG{n}{random\PYGZus{}integers}\PYG{p}{(}\PYG{n}{N}\PYG{p}{)} \PYG{o}{\PYGZhy{}} \PYG{l+m+mi}{1}\PYG{p}{)} \PYG{o}{/} \PYG{p}{(}\PYG{n}{N} \PYG{o}{\PYGZhy{}} \PYG{l+m+mf}{1.}\PYG{p}{)}
\end{Verbatim}

\begin{Verbatim}[commandchars=\\\{\}]
\PYG{g+gp}{\PYGZgt{}\PYGZgt{}\PYGZgt{} }\PYG{n}{np}\PYG{o}{.}\PYG{n}{random}\PYG{o}{.}\PYG{n}{random\PYGZus{}integers}\PYG{p}{(}\PYG{l+m+mi}{5}\PYG{p}{)}
\PYG{g+go}{4}
\PYG{g+gp}{\PYGZgt{}\PYGZgt{}\PYGZgt{} }\PYG{n+nb}{type}\PYG{p}{(}\PYG{n}{np}\PYG{o}{.}\PYG{n}{random}\PYG{o}{.}\PYG{n}{random\PYGZus{}integers}\PYG{p}{(}\PYG{l+m+mi}{5}\PYG{p}{)}\PYG{p}{)}
\PYG{g+go}{\PYGZlt{}type \PYGZsq{}int\PYGZsq{}\PYGZgt{}}
\PYG{g+gp}{\PYGZgt{}\PYGZgt{}\PYGZgt{} }\PYG{n}{np}\PYG{o}{.}\PYG{n}{random}\PYG{o}{.}\PYG{n}{random\PYGZus{}integers}\PYG{p}{(}\PYG{l+m+mi}{5}\PYG{p}{,} \PYG{n}{size}\PYG{o}{=}\PYG{p}{(}\PYG{l+m+mf}{3.}\PYG{p}{,}\PYG{l+m+mf}{2.}\PYG{p}{)}\PYG{p}{)}
\PYG{g+go}{array([[5, 4],}
\PYG{g+go}{       [3, 3],}
\PYG{g+go}{       [4, 5]])}
\end{Verbatim}

Choose five random numbers from the set of five evenly-spaced
numbers between 0 and 2.5, inclusive (\emph{i.e.}, from the set
\({0, 5/8, 10/8, 15/8, 20/8}\)):

\begin{Verbatim}[commandchars=\\\{\}]
\PYG{g+gp}{\PYGZgt{}\PYGZgt{}\PYGZgt{} }\PYG{l+m+mf}{2.5} \PYG{o}{*} \PYG{p}{(}\PYG{n}{np}\PYG{o}{.}\PYG{n}{random}\PYG{o}{.}\PYG{n}{random\PYGZus{}integers}\PYG{p}{(}\PYG{l+m+mi}{5}\PYG{p}{,} \PYG{n}{size}\PYG{o}{=}\PYG{p}{(}\PYG{l+m+mi}{5}\PYG{p}{,}\PYG{p}{)}\PYG{p}{)} \PYG{o}{\PYGZhy{}} \PYG{l+m+mi}{1}\PYG{p}{)} \PYG{o}{/} \PYG{l+m+mf}{4.}
\PYG{g+go}{array([ 0.625,  1.25 ,  0.625,  0.625,  2.5  ])}
\end{Verbatim}

Roll two six sided dice 1000 times and sum the results:

\begin{Verbatim}[commandchars=\\\{\}]
\PYG{g+gp}{\PYGZgt{}\PYGZgt{}\PYGZgt{} }\PYG{n}{d1} \PYG{o}{=} \PYG{n}{np}\PYG{o}{.}\PYG{n}{random}\PYG{o}{.}\PYG{n}{random\PYGZus{}integers}\PYG{p}{(}\PYG{l+m+mi}{1}\PYG{p}{,} \PYG{l+m+mi}{6}\PYG{p}{,} \PYG{l+m+mi}{1000}\PYG{p}{)}
\PYG{g+gp}{\PYGZgt{}\PYGZgt{}\PYGZgt{} }\PYG{n}{d2} \PYG{o}{=} \PYG{n}{np}\PYG{o}{.}\PYG{n}{random}\PYG{o}{.}\PYG{n}{random\PYGZus{}integers}\PYG{p}{(}\PYG{l+m+mi}{1}\PYG{p}{,} \PYG{l+m+mi}{6}\PYG{p}{,} \PYG{l+m+mi}{1000}\PYG{p}{)}
\PYG{g+gp}{\PYGZgt{}\PYGZgt{}\PYGZgt{} }\PYG{n}{dsums} \PYG{o}{=} \PYG{n}{d1} \PYG{o}{+} \PYG{n}{d2}
\end{Verbatim}

Display results as a histogram:

\begin{Verbatim}[commandchars=\\\{\}]
\PYG{g+gp}{\PYGZgt{}\PYGZgt{}\PYGZgt{} }\PYG{k+kn}{import} \PYG{n+nn}{matplotlib.pyplot} \PYG{k+kn}{as} \PYG{n+nn}{plt}
\PYG{g+gp}{\PYGZgt{}\PYGZgt{}\PYGZgt{} }\PYG{n}{count}\PYG{p}{,} \PYG{n}{bins}\PYG{p}{,} \PYG{n}{ignored} \PYG{o}{=} \PYG{n}{plt}\PYG{o}{.}\PYG{n}{hist}\PYG{p}{(}\PYG{n}{dsums}\PYG{p}{,} \PYG{l+m+mi}{11}\PYG{p}{,} \PYG{n}{normed}\PYG{o}{=}\PYG{n+nb+bp}{True}\PYG{p}{)}
\PYG{g+gp}{\PYGZgt{}\PYGZgt{}\PYGZgt{} }\PYG{n}{plt}\PYG{o}{.}\PYG{n}{show}\PYG{p}{(}\PYG{p}{)}
\end{Verbatim}

\end{fulllineitems}

\index{random\_sample() (in module graph\_chemistry\_analysis)}

\begin{fulllineitems}
\phantomsection\label{graph_chemistry_analysis:graph_chemistry_analysis.random_sample}\pysiglinewithargsret{\code{graph\_chemistry\_analysis.}\bfcode{random\_sample}}{\emph{size=None}}{}
Return random floats in the half-open interval {[}0.0, 1.0).

Results are from the ``continuous uniform'' distribution over the
stated interval.  To sample \(Unif[a, b), b > a\) multiply
the output of \emph{random\_sample} by \emph{(b-a)} and add \emph{a}:

\begin{Verbatim}[commandchars=\\\{\}]
\PYG{p}{(}\PYG{n}{b} \PYG{o}{\PYGZhy{}} \PYG{n}{a}\PYG{p}{)} \PYG{o}{*} \PYG{n}{random\PYGZus{}sample}\PYG{p}{(}\PYG{p}{)} \PYG{o}{+} \PYG{n}{a}
\end{Verbatim}
\begin{description}
\item[{size}] \leavevmode{[}int or tuple of ints, optional{]}
Defines the shape of the returned array of random floats. If None
(the default), returns a single float.

\end{description}
\begin{description}
\item[{out}] \leavevmode{[}float or ndarray of floats{]}
Array of random floats of shape \emph{size} (unless \code{size=None}, in which
case a single float is returned).

\end{description}

\begin{Verbatim}[commandchars=\\\{\}]
\PYG{g+gp}{\PYGZgt{}\PYGZgt{}\PYGZgt{} }\PYG{n}{np}\PYG{o}{.}\PYG{n}{random}\PYG{o}{.}\PYG{n}{random\PYGZus{}sample}\PYG{p}{(}\PYG{p}{)}
\PYG{g+go}{0.47108547995356098}
\PYG{g+gp}{\PYGZgt{}\PYGZgt{}\PYGZgt{} }\PYG{n+nb}{type}\PYG{p}{(}\PYG{n}{np}\PYG{o}{.}\PYG{n}{random}\PYG{o}{.}\PYG{n}{random\PYGZus{}sample}\PYG{p}{(}\PYG{p}{)}\PYG{p}{)}
\PYG{g+go}{\PYGZlt{}type \PYGZsq{}float\PYGZsq{}\PYGZgt{}}
\PYG{g+gp}{\PYGZgt{}\PYGZgt{}\PYGZgt{} }\PYG{n}{np}\PYG{o}{.}\PYG{n}{random}\PYG{o}{.}\PYG{n}{random\PYGZus{}sample}\PYG{p}{(}\PYG{p}{(}\PYG{l+m+mi}{5}\PYG{p}{,}\PYG{p}{)}\PYG{p}{)}
\PYG{g+go}{array([ 0.30220482,  0.86820401,  0.1654503 ,  0.11659149,  0.54323428])}
\end{Verbatim}

Three-by-two array of random numbers from {[}-5, 0):

\begin{Verbatim}[commandchars=\\\{\}]
\PYG{g+gp}{\PYGZgt{}\PYGZgt{}\PYGZgt{} }\PYG{l+m+mi}{5} \PYG{o}{*} \PYG{n}{np}\PYG{o}{.}\PYG{n}{random}\PYG{o}{.}\PYG{n}{random\PYGZus{}sample}\PYG{p}{(}\PYG{p}{(}\PYG{l+m+mi}{3}\PYG{p}{,} \PYG{l+m+mi}{2}\PYG{p}{)}\PYG{p}{)} \PYG{o}{\PYGZhy{}} \PYG{l+m+mi}{5}
\PYG{g+go}{array([[\PYGZhy{}3.99149989, \PYGZhy{}0.52338984],}
\PYG{g+go}{       [\PYGZhy{}2.99091858, \PYGZhy{}0.79479508],}
\PYG{g+go}{       [\PYGZhy{}1.23204345, \PYGZhy{}1.75224494]])}
\end{Verbatim}

\end{fulllineitems}

\index{ranf() (in module graph\_chemistry\_analysis)}

\begin{fulllineitems}
\phantomsection\label{graph_chemistry_analysis:graph_chemistry_analysis.ranf}\pysiglinewithargsret{\code{graph\_chemistry\_analysis.}\bfcode{ranf}}{}{}
random\_sample(size=None)

Return random floats in the half-open interval {[}0.0, 1.0).

Results are from the ``continuous uniform'' distribution over the
stated interval.  To sample \(Unif[a, b), b > a\) multiply
the output of \emph{random\_sample} by \emph{(b-a)} and add \emph{a}:

\begin{Verbatim}[commandchars=\\\{\}]
\PYG{p}{(}\PYG{n}{b} \PYG{o}{\PYGZhy{}} \PYG{n}{a}\PYG{p}{)} \PYG{o}{*} \PYG{n}{random\PYGZus{}sample}\PYG{p}{(}\PYG{p}{)} \PYG{o}{+} \PYG{n}{a}
\end{Verbatim}
\begin{description}
\item[{size}] \leavevmode{[}int or tuple of ints, optional{]}
Defines the shape of the returned array of random floats. If None
(the default), returns a single float.

\end{description}
\begin{description}
\item[{out}] \leavevmode{[}float or ndarray of floats{]}
Array of random floats of shape \emph{size} (unless \code{size=None}, in which
case a single float is returned).

\end{description}

\begin{Verbatim}[commandchars=\\\{\}]
\PYG{g+gp}{\PYGZgt{}\PYGZgt{}\PYGZgt{} }\PYG{n}{np}\PYG{o}{.}\PYG{n}{random}\PYG{o}{.}\PYG{n}{random\PYGZus{}sample}\PYG{p}{(}\PYG{p}{)}
\PYG{g+go}{0.47108547995356098}
\PYG{g+gp}{\PYGZgt{}\PYGZgt{}\PYGZgt{} }\PYG{n+nb}{type}\PYG{p}{(}\PYG{n}{np}\PYG{o}{.}\PYG{n}{random}\PYG{o}{.}\PYG{n}{random\PYGZus{}sample}\PYG{p}{(}\PYG{p}{)}\PYG{p}{)}
\PYG{g+go}{\PYGZlt{}type \PYGZsq{}float\PYGZsq{}\PYGZgt{}}
\PYG{g+gp}{\PYGZgt{}\PYGZgt{}\PYGZgt{} }\PYG{n}{np}\PYG{o}{.}\PYG{n}{random}\PYG{o}{.}\PYG{n}{random\PYGZus{}sample}\PYG{p}{(}\PYG{p}{(}\PYG{l+m+mi}{5}\PYG{p}{,}\PYG{p}{)}\PYG{p}{)}
\PYG{g+go}{array([ 0.30220482,  0.86820401,  0.1654503 ,  0.11659149,  0.54323428])}
\end{Verbatim}

Three-by-two array of random numbers from {[}-5, 0):

\begin{Verbatim}[commandchars=\\\{\}]
\PYG{g+gp}{\PYGZgt{}\PYGZgt{}\PYGZgt{} }\PYG{l+m+mi}{5} \PYG{o}{*} \PYG{n}{np}\PYG{o}{.}\PYG{n}{random}\PYG{o}{.}\PYG{n}{random\PYGZus{}sample}\PYG{p}{(}\PYG{p}{(}\PYG{l+m+mi}{3}\PYG{p}{,} \PYG{l+m+mi}{2}\PYG{p}{)}\PYG{p}{)} \PYG{o}{\PYGZhy{}} \PYG{l+m+mi}{5}
\PYG{g+go}{array([[\PYGZhy{}3.99149989, \PYGZhy{}0.52338984],}
\PYG{g+go}{       [\PYGZhy{}2.99091858, \PYGZhy{}0.79479508],}
\PYG{g+go}{       [\PYGZhy{}1.23204345, \PYGZhy{}1.75224494]])}
\end{Verbatim}

\end{fulllineitems}

\index{rayleigh() (in module graph\_chemistry\_analysis)}

\begin{fulllineitems}
\phantomsection\label{graph_chemistry_analysis:graph_chemistry_analysis.rayleigh}\pysiglinewithargsret{\code{graph\_chemistry\_analysis.}\bfcode{rayleigh}}{\emph{scale=1.0}, \emph{size=None}}{}
Draw samples from a Rayleigh distribution.

The \(\chi\) and Weibull distributions are generalizations of the
Rayleigh.
\begin{description}
\item[{scale}] \leavevmode{[}scalar{]}
Scale, also equals the mode. Should be \textgreater{}= 0.

\item[{size}] \leavevmode{[}int or tuple of ints, optional{]}
Shape of the output. Default is None, in which case a single
value is returned.

\end{description}

The probability density function for the Rayleigh distribution is
\begin{gather}
\begin{split}P(x;scale) = \frac{x}{scale^2}e^{\frac{-x^2}{2 \cdotp scale^2}}\end{split}\notag
\end{gather}
The Rayleigh distribution arises if the wind speed and wind direction are
both gaussian variables, then the vector wind velocity forms a Rayleigh
distribution. The Rayleigh distribution is used to model the expected
output from wind turbines.

Draw values from the distribution and plot the histogram

\begin{Verbatim}[commandchars=\\\{\}]
\PYG{g+gp}{\PYGZgt{}\PYGZgt{}\PYGZgt{} }\PYG{n}{values} \PYG{o}{=} \PYG{n}{hist}\PYG{p}{(}\PYG{n}{np}\PYG{o}{.}\PYG{n}{random}\PYG{o}{.}\PYG{n}{rayleigh}\PYG{p}{(}\PYG{l+m+mi}{3}\PYG{p}{,} \PYG{l+m+mi}{100000}\PYG{p}{)}\PYG{p}{,} \PYG{n}{bins}\PYG{o}{=}\PYG{l+m+mi}{200}\PYG{p}{,} \PYG{n}{normed}\PYG{o}{=}\PYG{n+nb+bp}{True}\PYG{p}{)}
\end{Verbatim}

Wave heights tend to follow a Rayleigh distribution. If the mean wave
height is 1 meter, what fraction of waves are likely to be larger than 3
meters?

\begin{Verbatim}[commandchars=\\\{\}]
\PYG{g+gp}{\PYGZgt{}\PYGZgt{}\PYGZgt{} }\PYG{n}{meanvalue} \PYG{o}{=} \PYG{l+m+mi}{1}
\PYG{g+gp}{\PYGZgt{}\PYGZgt{}\PYGZgt{} }\PYG{n}{modevalue} \PYG{o}{=} \PYG{n}{np}\PYG{o}{.}\PYG{n}{sqrt}\PYG{p}{(}\PYG{l+m+mi}{2} \PYG{o}{/} \PYG{n}{np}\PYG{o}{.}\PYG{n}{pi}\PYG{p}{)} \PYG{o}{*} \PYG{n}{meanvalue}
\PYG{g+gp}{\PYGZgt{}\PYGZgt{}\PYGZgt{} }\PYG{n}{s} \PYG{o}{=} \PYG{n}{np}\PYG{o}{.}\PYG{n}{random}\PYG{o}{.}\PYG{n}{rayleigh}\PYG{p}{(}\PYG{n}{modevalue}\PYG{p}{,} \PYG{l+m+mi}{1000000}\PYG{p}{)}
\end{Verbatim}

The percentage of waves larger than 3 meters is:

\begin{Verbatim}[commandchars=\\\{\}]
\PYG{g+gp}{\PYGZgt{}\PYGZgt{}\PYGZgt{} }\PYG{l+m+mf}{100.}\PYG{o}{*}\PYG{n+nb}{sum}\PYG{p}{(}\PYG{n}{s}\PYG{o}{\PYGZgt{}}\PYG{l+m+mi}{3}\PYG{p}{)}\PYG{o}{/}\PYG{l+m+mf}{1000000.}
\PYG{g+go}{0.087300000000000003}
\end{Verbatim}

\end{fulllineitems}

\index{sample() (in module graph\_chemistry\_analysis)}

\begin{fulllineitems}
\phantomsection\label{graph_chemistry_analysis:graph_chemistry_analysis.sample}\pysiglinewithargsret{\code{graph\_chemistry\_analysis.}\bfcode{sample}}{}{}
random\_sample(size=None)

Return random floats in the half-open interval {[}0.0, 1.0).

Results are from the ``continuous uniform'' distribution over the
stated interval.  To sample \(Unif[a, b), b > a\) multiply
the output of \emph{random\_sample} by \emph{(b-a)} and add \emph{a}:

\begin{Verbatim}[commandchars=\\\{\}]
\PYG{p}{(}\PYG{n}{b} \PYG{o}{\PYGZhy{}} \PYG{n}{a}\PYG{p}{)} \PYG{o}{*} \PYG{n}{random\PYGZus{}sample}\PYG{p}{(}\PYG{p}{)} \PYG{o}{+} \PYG{n}{a}
\end{Verbatim}
\begin{description}
\item[{size}] \leavevmode{[}int or tuple of ints, optional{]}
Defines the shape of the returned array of random floats. If None
(the default), returns a single float.

\end{description}
\begin{description}
\item[{out}] \leavevmode{[}float or ndarray of floats{]}
Array of random floats of shape \emph{size} (unless \code{size=None}, in which
case a single float is returned).

\end{description}

\begin{Verbatim}[commandchars=\\\{\}]
\PYG{g+gp}{\PYGZgt{}\PYGZgt{}\PYGZgt{} }\PYG{n}{np}\PYG{o}{.}\PYG{n}{random}\PYG{o}{.}\PYG{n}{random\PYGZus{}sample}\PYG{p}{(}\PYG{p}{)}
\PYG{g+go}{0.47108547995356098}
\PYG{g+gp}{\PYGZgt{}\PYGZgt{}\PYGZgt{} }\PYG{n+nb}{type}\PYG{p}{(}\PYG{n}{np}\PYG{o}{.}\PYG{n}{random}\PYG{o}{.}\PYG{n}{random\PYGZus{}sample}\PYG{p}{(}\PYG{p}{)}\PYG{p}{)}
\PYG{g+go}{\PYGZlt{}type \PYGZsq{}float\PYGZsq{}\PYGZgt{}}
\PYG{g+gp}{\PYGZgt{}\PYGZgt{}\PYGZgt{} }\PYG{n}{np}\PYG{o}{.}\PYG{n}{random}\PYG{o}{.}\PYG{n}{random\PYGZus{}sample}\PYG{p}{(}\PYG{p}{(}\PYG{l+m+mi}{5}\PYG{p}{,}\PYG{p}{)}\PYG{p}{)}
\PYG{g+go}{array([ 0.30220482,  0.86820401,  0.1654503 ,  0.11659149,  0.54323428])}
\end{Verbatim}

Three-by-two array of random numbers from {[}-5, 0):

\begin{Verbatim}[commandchars=\\\{\}]
\PYG{g+gp}{\PYGZgt{}\PYGZgt{}\PYGZgt{} }\PYG{l+m+mi}{5} \PYG{o}{*} \PYG{n}{np}\PYG{o}{.}\PYG{n}{random}\PYG{o}{.}\PYG{n}{random\PYGZus{}sample}\PYG{p}{(}\PYG{p}{(}\PYG{l+m+mi}{3}\PYG{p}{,} \PYG{l+m+mi}{2}\PYG{p}{)}\PYG{p}{)} \PYG{o}{\PYGZhy{}} \PYG{l+m+mi}{5}
\PYG{g+go}{array([[\PYGZhy{}3.99149989, \PYGZhy{}0.52338984],}
\PYG{g+go}{       [\PYGZhy{}2.99091858, \PYGZhy{}0.79479508],}
\PYG{g+go}{       [\PYGZhy{}1.23204345, \PYGZhy{}1.75224494]])}
\end{Verbatim}

\end{fulllineitems}

\index{seed() (in module graph\_chemistry\_analysis)}

\begin{fulllineitems}
\phantomsection\label{graph_chemistry_analysis:graph_chemistry_analysis.seed}\pysiglinewithargsret{\code{graph\_chemistry\_analysis.}\bfcode{seed}}{\emph{seed=None}}{}
Seed the generator.

This method is called when \emph{RandomState} is initialized. It can be
called again to re-seed the generator. For details, see \emph{RandomState}.
\begin{description}
\item[{seed}] \leavevmode{[}int or array\_like, optional{]}
Seed for \emph{RandomState}.

\end{description}

RandomState

\end{fulllineitems}

\index{set\_state() (in module graph\_chemistry\_analysis)}

\begin{fulllineitems}
\phantomsection\label{graph_chemistry_analysis:graph_chemistry_analysis.set_state}\pysiglinewithargsret{\code{graph\_chemistry\_analysis.}\bfcode{set\_state}}{\emph{state}}{}
Set the internal state of the generator from a tuple.

For use if one has reason to manually (re-)set the internal state of the
``Mersenne Twister''{\color{red}\bfseries{}{[}1{]}\_} pseudo-random number generating algorithm.
\begin{description}
\item[{state}] \leavevmode{[}tuple(str, ndarray of 624 uints, int, int, float){]}
The \emph{state} tuple has the following items:
\begin{enumerate}
\item {} 
the string `MT19937', specifying the Mersenne Twister algorithm.

\item {} 
a 1-D array of 624 unsigned integers \code{keys}.

\item {} 
an integer \code{pos}.

\item {} 
an integer \code{has\_gauss}.

\item {} 
a float \code{cached\_gaussian}.

\end{enumerate}

\end{description}
\begin{description}
\item[{out}] \leavevmode{[}None{]}
Returns `None' on success.

\end{description}

get\_state

\emph{set\_state} and \emph{get\_state} are not needed to work with any of the
random distributions in NumPy. If the internal state is manually altered,
the user should know exactly what he/she is doing.

For backwards compatibility, the form (str, array of 624 uints, int) is
also accepted although it is missing some information about the cached
Gaussian value: \code{state = ('MT19937', keys, pos)}.

\end{fulllineitems}

\index{shuffle() (in module graph\_chemistry\_analysis)}

\begin{fulllineitems}
\phantomsection\label{graph_chemistry_analysis:graph_chemistry_analysis.shuffle}\pysiglinewithargsret{\code{graph\_chemistry\_analysis.}\bfcode{shuffle}}{\emph{x}}{}
Modify a sequence in-place by shuffling its contents.
\begin{description}
\item[{x}] \leavevmode{[}array\_like{]}
The array or list to be shuffled.

\end{description}

None

\begin{Verbatim}[commandchars=\\\{\}]
\PYG{g+gp}{\PYGZgt{}\PYGZgt{}\PYGZgt{} }\PYG{n}{arr} \PYG{o}{=} \PYG{n}{np}\PYG{o}{.}\PYG{n}{arange}\PYG{p}{(}\PYG{l+m+mi}{10}\PYG{p}{)}
\PYG{g+gp}{\PYGZgt{}\PYGZgt{}\PYGZgt{} }\PYG{n}{np}\PYG{o}{.}\PYG{n}{random}\PYG{o}{.}\PYG{n}{shuffle}\PYG{p}{(}\PYG{n}{arr}\PYG{p}{)}
\PYG{g+gp}{\PYGZgt{}\PYGZgt{}\PYGZgt{} }\PYG{n}{arr}
\PYG{g+go}{[1 7 5 2 9 4 3 6 0 8]}
\end{Verbatim}

This function only shuffles the array along the first index of a
multi-dimensional array:

\begin{Verbatim}[commandchars=\\\{\}]
\PYG{g+gp}{\PYGZgt{}\PYGZgt{}\PYGZgt{} }\PYG{n}{arr} \PYG{o}{=} \PYG{n}{np}\PYG{o}{.}\PYG{n}{arange}\PYG{p}{(}\PYG{l+m+mi}{9}\PYG{p}{)}\PYG{o}{.}\PYG{n}{reshape}\PYG{p}{(}\PYG{p}{(}\PYG{l+m+mi}{3}\PYG{p}{,} \PYG{l+m+mi}{3}\PYG{p}{)}\PYG{p}{)}
\PYG{g+gp}{\PYGZgt{}\PYGZgt{}\PYGZgt{} }\PYG{n}{np}\PYG{o}{.}\PYG{n}{random}\PYG{o}{.}\PYG{n}{shuffle}\PYG{p}{(}\PYG{n}{arr}\PYG{p}{)}
\PYG{g+gp}{\PYGZgt{}\PYGZgt{}\PYGZgt{} }\PYG{n}{arr}
\PYG{g+go}{array([[3, 4, 5],}
\PYG{g+go}{       [6, 7, 8],}
\PYG{g+go}{       [0, 1, 2]])}
\end{Verbatim}

\end{fulllineitems}

\index{standard\_cauchy() (in module graph\_chemistry\_analysis)}

\begin{fulllineitems}
\phantomsection\label{graph_chemistry_analysis:graph_chemistry_analysis.standard_cauchy}\pysiglinewithargsret{\code{graph\_chemistry\_analysis.}\bfcode{standard\_cauchy}}{\emph{size=None}}{}
Standard Cauchy distribution with mode = 0.

Also known as the Lorentz distribution.
\begin{description}
\item[{size}] \leavevmode{[}int or tuple of ints{]}
Shape of the output.

\end{description}
\begin{description}
\item[{samples}] \leavevmode{[}ndarray or scalar{]}
The drawn samples.

\end{description}

The probability density function for the full Cauchy distribution is
\begin{gather}
\begin{split}P(x; x_0, \gamma) = \frac{1}{\pi \gamma \bigl[ 1+
(\frac{x-x_0}{\gamma})^2 \bigr] }\end{split}\notag
\end{gather}
and the Standard Cauchy distribution just sets \(x_0=0\) and
\(\gamma=1\)

The Cauchy distribution arises in the solution to the driven harmonic
oscillator problem, and also describes spectral line broadening. It
also describes the distribution of values at which a line tilted at
a random angle will cut the x axis.

When studying hypothesis tests that assume normality, seeing how the
tests perform on data from a Cauchy distribution is a good indicator of
their sensitivity to a heavy-tailed distribution, since the Cauchy looks
very much like a Gaussian distribution, but with heavier tails.

Draw samples and plot the distribution:

\begin{Verbatim}[commandchars=\\\{\}]
\PYG{g+gp}{\PYGZgt{}\PYGZgt{}\PYGZgt{} }\PYG{n}{s} \PYG{o}{=} \PYG{n}{np}\PYG{o}{.}\PYG{n}{random}\PYG{o}{.}\PYG{n}{standard\PYGZus{}cauchy}\PYG{p}{(}\PYG{l+m+mi}{1000000}\PYG{p}{)}
\PYG{g+gp}{\PYGZgt{}\PYGZgt{}\PYGZgt{} }\PYG{n}{s} \PYG{o}{=} \PYG{n}{s}\PYG{p}{[}\PYG{p}{(}\PYG{n}{s}\PYG{o}{\PYGZgt{}}\PYG{o}{\PYGZhy{}}\PYG{l+m+mi}{25}\PYG{p}{)} \PYG{o}{\PYGZam{}} \PYG{p}{(}\PYG{n}{s}\PYG{o}{\PYGZlt{}}\PYG{l+m+mi}{25}\PYG{p}{)}\PYG{p}{]}  \PYG{c}{\PYGZsh{} truncate distribution so it plots well}
\PYG{g+gp}{\PYGZgt{}\PYGZgt{}\PYGZgt{} }\PYG{n}{plt}\PYG{o}{.}\PYG{n}{hist}\PYG{p}{(}\PYG{n}{s}\PYG{p}{,} \PYG{n}{bins}\PYG{o}{=}\PYG{l+m+mi}{100}\PYG{p}{)}
\PYG{g+gp}{\PYGZgt{}\PYGZgt{}\PYGZgt{} }\PYG{n}{plt}\PYG{o}{.}\PYG{n}{show}\PYG{p}{(}\PYG{p}{)}
\end{Verbatim}

\end{fulllineitems}

\index{standard\_exponential() (in module graph\_chemistry\_analysis)}

\begin{fulllineitems}
\phantomsection\label{graph_chemistry_analysis:graph_chemistry_analysis.standard_exponential}\pysiglinewithargsret{\code{graph\_chemistry\_analysis.}\bfcode{standard\_exponential}}{\emph{size=None}}{}
Draw samples from the standard exponential distribution.

\emph{standard\_exponential} is identical to the exponential distribution
with a scale parameter of 1.
\begin{description}
\item[{size}] \leavevmode{[}int or tuple of ints{]}
Shape of the output.

\end{description}
\begin{description}
\item[{out}] \leavevmode{[}float or ndarray{]}
Drawn samples.

\end{description}

Output a 3x8000 array:

\begin{Verbatim}[commandchars=\\\{\}]
\PYG{g+gp}{\PYGZgt{}\PYGZgt{}\PYGZgt{} }\PYG{n}{n} \PYG{o}{=} \PYG{n}{np}\PYG{o}{.}\PYG{n}{random}\PYG{o}{.}\PYG{n}{standard\PYGZus{}exponential}\PYG{p}{(}\PYG{p}{(}\PYG{l+m+mi}{3}\PYG{p}{,} \PYG{l+m+mi}{8000}\PYG{p}{)}\PYG{p}{)}
\end{Verbatim}

\end{fulllineitems}

\index{standard\_gamma() (in module graph\_chemistry\_analysis)}

\begin{fulllineitems}
\phantomsection\label{graph_chemistry_analysis:graph_chemistry_analysis.standard_gamma}\pysiglinewithargsret{\code{graph\_chemistry\_analysis.}\bfcode{standard\_gamma}}{\emph{shape}, \emph{size=None}}{}
Draw samples from a Standard Gamma distribution.

Samples are drawn from a Gamma distribution with specified parameters,
shape (sometimes designated ``k'') and scale=1.
\begin{description}
\item[{shape}] \leavevmode{[}float{]}
Parameter, should be \textgreater{} 0.

\item[{size}] \leavevmode{[}int or tuple of ints{]}
Output shape.  If the given shape is, e.g., \code{(m, n, k)}, then
\code{m * n * k} samples are drawn.

\end{description}
\begin{description}
\item[{samples}] \leavevmode{[}ndarray or scalar{]}
The drawn samples.

\end{description}
\begin{description}
\item[{scipy.stats.distributions.gamma}] \leavevmode{[}probability density function,{]}
distribution or cumulative density function, etc.

\end{description}

The probability density for the Gamma distribution is
\begin{gather}
\begin{split}p(x) = x^{k-1}\frac{e^{-x/\theta}}{\theta^k\Gamma(k)},\end{split}\notag
\end{gather}
where \(k\) is the shape and \(\theta\) the scale,
and \(\Gamma\) is the Gamma function.

The Gamma distribution is often used to model the times to failure of
electronic components, and arises naturally in processes for which the
waiting times between Poisson distributed events are relevant.

Draw samples from the distribution:

\begin{Verbatim}[commandchars=\\\{\}]
\PYG{g+gp}{\PYGZgt{}\PYGZgt{}\PYGZgt{} }\PYG{n}{shape}\PYG{p}{,} \PYG{n}{scale} \PYG{o}{=} \PYG{l+m+mf}{2.}\PYG{p}{,} \PYG{l+m+mf}{1.} \PYG{c}{\PYGZsh{} mean and width}
\PYG{g+gp}{\PYGZgt{}\PYGZgt{}\PYGZgt{} }\PYG{n}{s} \PYG{o}{=} \PYG{n}{np}\PYG{o}{.}\PYG{n}{random}\PYG{o}{.}\PYG{n}{standard\PYGZus{}gamma}\PYG{p}{(}\PYG{n}{shape}\PYG{p}{,} \PYG{l+m+mi}{1000000}\PYG{p}{)}
\end{Verbatim}

Display the histogram of the samples, along with
the probability density function:

\begin{Verbatim}[commandchars=\\\{\}]
\PYG{g+gp}{\PYGZgt{}\PYGZgt{}\PYGZgt{} }\PYG{k+kn}{import} \PYG{n+nn}{matplotlib.pyplot} \PYG{k+kn}{as} \PYG{n+nn}{plt}
\PYG{g+gp}{\PYGZgt{}\PYGZgt{}\PYGZgt{} }\PYG{k+kn}{import} \PYG{n+nn}{scipy.special} \PYG{k+kn}{as} \PYG{n+nn}{sps}
\PYG{g+gp}{\PYGZgt{}\PYGZgt{}\PYGZgt{} }\PYG{n}{count}\PYG{p}{,} \PYG{n}{bins}\PYG{p}{,} \PYG{n}{ignored} \PYG{o}{=} \PYG{n}{plt}\PYG{o}{.}\PYG{n}{hist}\PYG{p}{(}\PYG{n}{s}\PYG{p}{,} \PYG{l+m+mi}{50}\PYG{p}{,} \PYG{n}{normed}\PYG{o}{=}\PYG{n+nb+bp}{True}\PYG{p}{)}
\PYG{g+gp}{\PYGZgt{}\PYGZgt{}\PYGZgt{} }\PYG{n}{y} \PYG{o}{=} \PYG{n}{bins}\PYG{o}{*}\PYG{o}{*}\PYG{p}{(}\PYG{n}{shape}\PYG{o}{\PYGZhy{}}\PYG{l+m+mi}{1}\PYG{p}{)} \PYG{o}{*} \PYG{p}{(}\PYG{p}{(}\PYG{n}{np}\PYG{o}{.}\PYG{n}{exp}\PYG{p}{(}\PYG{o}{\PYGZhy{}}\PYG{n}{bins}\PYG{o}{/}\PYG{n}{scale}\PYG{p}{)}\PYG{p}{)}\PYG{o}{/} \PYGZbs{}
\PYG{g+gp}{... }                      \PYG{p}{(}\PYG{n}{sps}\PYG{o}{.}\PYG{n}{gamma}\PYG{p}{(}\PYG{n}{shape}\PYG{p}{)} \PYG{o}{*} \PYG{n}{scale}\PYG{o}{*}\PYG{o}{*}\PYG{n}{shape}\PYG{p}{)}\PYG{p}{)}
\PYG{g+gp}{\PYGZgt{}\PYGZgt{}\PYGZgt{} }\PYG{n}{plt}\PYG{o}{.}\PYG{n}{plot}\PYG{p}{(}\PYG{n}{bins}\PYG{p}{,} \PYG{n}{y}\PYG{p}{,} \PYG{n}{linewidth}\PYG{o}{=}\PYG{l+m+mi}{2}\PYG{p}{,} \PYG{n}{color}\PYG{o}{=}\PYG{l+s}{\PYGZsq{}}\PYG{l+s}{r}\PYG{l+s}{\PYGZsq{}}\PYG{p}{)}
\PYG{g+gp}{\PYGZgt{}\PYGZgt{}\PYGZgt{} }\PYG{n}{plt}\PYG{o}{.}\PYG{n}{show}\PYG{p}{(}\PYG{p}{)}
\end{Verbatim}

\end{fulllineitems}

\index{standard\_normal() (in module graph\_chemistry\_analysis)}

\begin{fulllineitems}
\phantomsection\label{graph_chemistry_analysis:graph_chemistry_analysis.standard_normal}\pysiglinewithargsret{\code{graph\_chemistry\_analysis.}\bfcode{standard\_normal}}{\emph{size=None}}{}
Returns samples from a Standard Normal distribution (mean=0, stdev=1).
\begin{description}
\item[{size}] \leavevmode{[}int or tuple of ints, optional{]}
Output shape. Default is None, in which case a single value is
returned.

\end{description}
\begin{description}
\item[{out}] \leavevmode{[}float or ndarray{]}
Drawn samples.

\end{description}

\begin{Verbatim}[commandchars=\\\{\}]
\PYG{g+gp}{\PYGZgt{}\PYGZgt{}\PYGZgt{} }\PYG{n}{s} \PYG{o}{=} \PYG{n}{np}\PYG{o}{.}\PYG{n}{random}\PYG{o}{.}\PYG{n}{standard\PYGZus{}normal}\PYG{p}{(}\PYG{l+m+mi}{8000}\PYG{p}{)}
\PYG{g+gp}{\PYGZgt{}\PYGZgt{}\PYGZgt{} }\PYG{n}{s}
\PYG{g+go}{array([ 0.6888893 ,  0.78096262, \PYGZhy{}0.89086505, ...,  0.49876311, \PYGZsh{}random}
\PYG{g+go}{       \PYGZhy{}0.38672696, \PYGZhy{}0.4685006 ])                               \PYGZsh{}random}
\PYG{g+gp}{\PYGZgt{}\PYGZgt{}\PYGZgt{} }\PYG{n}{s}\PYG{o}{.}\PYG{n}{shape}
\PYG{g+go}{(8000,)}
\PYG{g+gp}{\PYGZgt{}\PYGZgt{}\PYGZgt{} }\PYG{n}{s} \PYG{o}{=} \PYG{n}{np}\PYG{o}{.}\PYG{n}{random}\PYG{o}{.}\PYG{n}{standard\PYGZus{}normal}\PYG{p}{(}\PYG{n}{size}\PYG{o}{=}\PYG{p}{(}\PYG{l+m+mi}{3}\PYG{p}{,} \PYG{l+m+mi}{4}\PYG{p}{,} \PYG{l+m+mi}{2}\PYG{p}{)}\PYG{p}{)}
\PYG{g+gp}{\PYGZgt{}\PYGZgt{}\PYGZgt{} }\PYG{n}{s}\PYG{o}{.}\PYG{n}{shape}
\PYG{g+go}{(3, 4, 2)}
\end{Verbatim}

\end{fulllineitems}

\index{standard\_t() (in module graph\_chemistry\_analysis)}

\begin{fulllineitems}
\phantomsection\label{graph_chemistry_analysis:graph_chemistry_analysis.standard_t}\pysiglinewithargsret{\code{graph\_chemistry\_analysis.}\bfcode{standard\_t}}{\emph{df}, \emph{size=None}}{}
Standard Student's t distribution with df degrees of freedom.

A special case of the hyperbolic distribution.
As \emph{df} gets large, the result resembles that of the standard normal
distribution (\emph{standard\_normal}).
\begin{description}
\item[{df}] \leavevmode{[}int{]}
Degrees of freedom, should be \textgreater{} 0.

\item[{size}] \leavevmode{[}int or tuple of ints, optional{]}
Output shape. Default is None, in which case a single value is
returned.

\end{description}
\begin{description}
\item[{samples}] \leavevmode{[}ndarray or scalar{]}
Drawn samples.

\end{description}

The probability density function for the t distribution is
\begin{gather}
\begin{split}P(x, df) = \frac{\Gamma(\frac{df+1}{2})}{\sqrt{\pi df}
\Gamma(\frac{df}{2})}\Bigl( 1+\frac{x^2}{df} \Bigr)^{-(df+1)/2}\end{split}\notag
\end{gather}
The t test is based on an assumption that the data come from a Normal
distribution. The t test provides a way to test whether the sample mean
(that is the mean calculated from the data) is a good estimate of the true
mean.

The derivation of the t-distribution was forst published in 1908 by William
Gisset while working for the Guinness Brewery in Dublin. Due to proprietary
issues, he had to publish under a pseudonym, and so he used the name
Student.

From Dalgaard page 83 {\color{red}\bfseries{}{[}1{]}\_}, suppose the daily energy intake for 11
women in Kj is:

\begin{Verbatim}[commandchars=\\\{\}]
\PYG{g+gp}{\PYGZgt{}\PYGZgt{}\PYGZgt{} }\PYG{n}{intake} \PYG{o}{=} \PYG{n}{np}\PYG{o}{.}\PYG{n}{array}\PYG{p}{(}\PYG{p}{[}\PYG{l+m+mf}{5260.}\PYG{p}{,} \PYG{l+m+mi}{5470}\PYG{p}{,} \PYG{l+m+mi}{5640}\PYG{p}{,} \PYG{l+m+mi}{6180}\PYG{p}{,} \PYG{l+m+mi}{6390}\PYG{p}{,} \PYG{l+m+mi}{6515}\PYG{p}{,} \PYG{l+m+mi}{6805}\PYG{p}{,} \PYG{l+m+mi}{7515}\PYG{p}{,} \PYGZbs{}
\PYG{g+gp}{... }                   \PYG{l+m+mi}{7515}\PYG{p}{,} \PYG{l+m+mi}{8230}\PYG{p}{,} \PYG{l+m+mi}{8770}\PYG{p}{]}\PYG{p}{)}
\end{Verbatim}

Does their energy intake deviate systematically from the recommended
value of 7725 kJ?

We have 10 degrees of freedom, so is the sample mean within 95\% of the
recommended value?

\begin{Verbatim}[commandchars=\\\{\}]
\PYG{g+gp}{\PYGZgt{}\PYGZgt{}\PYGZgt{} }\PYG{n}{s} \PYG{o}{=} \PYG{n}{np}\PYG{o}{.}\PYG{n}{random}\PYG{o}{.}\PYG{n}{standard\PYGZus{}t}\PYG{p}{(}\PYG{l+m+mi}{10}\PYG{p}{,} \PYG{n}{size}\PYG{o}{=}\PYG{l+m+mi}{100000}\PYG{p}{)}
\PYG{g+gp}{\PYGZgt{}\PYGZgt{}\PYGZgt{} }\PYG{n}{np}\PYG{o}{.}\PYG{n}{mean}\PYG{p}{(}\PYG{n}{intake}\PYG{p}{)}
\PYG{g+go}{6753.636363636364}
\PYG{g+gp}{\PYGZgt{}\PYGZgt{}\PYGZgt{} }\PYG{n}{intake}\PYG{o}{.}\PYG{n}{std}\PYG{p}{(}\PYG{n}{ddof}\PYG{o}{=}\PYG{l+m+mi}{1}\PYG{p}{)}
\PYG{g+go}{1142.1232221373727}
\end{Verbatim}

Calculate the t statistic, setting the ddof parameter to the unbiased
value so the divisor in the standard deviation will be degrees of
freedom, N-1.

\begin{Verbatim}[commandchars=\\\{\}]
\PYG{g+gp}{\PYGZgt{}\PYGZgt{}\PYGZgt{} }\PYG{n}{t} \PYG{o}{=} \PYG{p}{(}\PYG{n}{np}\PYG{o}{.}\PYG{n}{mean}\PYG{p}{(}\PYG{n}{intake}\PYG{p}{)}\PYG{o}{\PYGZhy{}}\PYG{l+m+mi}{7725}\PYG{p}{)}\PYG{o}{/}\PYG{p}{(}\PYG{n}{intake}\PYG{o}{.}\PYG{n}{std}\PYG{p}{(}\PYG{n}{ddof}\PYG{o}{=}\PYG{l+m+mi}{1}\PYG{p}{)}\PYG{o}{/}\PYG{n}{np}\PYG{o}{.}\PYG{n}{sqrt}\PYG{p}{(}\PYG{n+nb}{len}\PYG{p}{(}\PYG{n}{intake}\PYG{p}{)}\PYG{p}{)}\PYG{p}{)}
\PYG{g+gp}{\PYGZgt{}\PYGZgt{}\PYGZgt{} }\PYG{k+kn}{import} \PYG{n+nn}{matplotlib.pyplot} \PYG{k+kn}{as} \PYG{n+nn}{plt}
\PYG{g+gp}{\PYGZgt{}\PYGZgt{}\PYGZgt{} }\PYG{n}{h} \PYG{o}{=} \PYG{n}{plt}\PYG{o}{.}\PYG{n}{hist}\PYG{p}{(}\PYG{n}{s}\PYG{p}{,} \PYG{n}{bins}\PYG{o}{=}\PYG{l+m+mi}{100}\PYG{p}{,} \PYG{n}{normed}\PYG{o}{=}\PYG{n+nb+bp}{True}\PYG{p}{)}
\end{Verbatim}

For a one-sided t-test, how far out in the distribution does the t
statistic appear?

\begin{Verbatim}[commandchars=\\\{\}]
\PYG{g+gp}{\PYGZgt{}\PYGZgt{}\PYGZgt{} }\PYG{o}{\PYGZgt{}\PYGZgt{}}\PYG{o}{\PYGZgt{}} \PYG{n}{np}\PYG{o}{.}\PYG{n}{sum}\PYG{p}{(}\PYG{n}{s}\PYG{o}{\PYGZlt{}}\PYG{n}{t}\PYG{p}{)} \PYG{o}{/} \PYG{n+nb}{float}\PYG{p}{(}\PYG{n+nb}{len}\PYG{p}{(}\PYG{n}{s}\PYG{p}{)}\PYG{p}{)}
\PYG{g+go}{0.0090699999999999999  \PYGZsh{}random}
\end{Verbatim}

So the p-value is about 0.009, which says the null hypothesis has a
probability of about 99\% of being true.

\end{fulllineitems}

\index{triangular() (in module graph\_chemistry\_analysis)}

\begin{fulllineitems}
\phantomsection\label{graph_chemistry_analysis:graph_chemistry_analysis.triangular}\pysiglinewithargsret{\code{graph\_chemistry\_analysis.}\bfcode{triangular}}{\emph{left}, \emph{mode}, \emph{right}, \emph{size=None}}{}
Draw samples from the triangular distribution.

The triangular distribution is a continuous probability distribution with
lower limit left, peak at mode, and upper limit right. Unlike the other
distributions, these parameters directly define the shape of the pdf.
\begin{description}
\item[{left}] \leavevmode{[}scalar{]}
Lower limit.

\item[{mode}] \leavevmode{[}scalar{]}
The value where the peak of the distribution occurs.
The value should fulfill the condition \code{left \textless{}= mode \textless{}= right}.

\item[{right}] \leavevmode{[}scalar{]}
Upper limit, should be larger than \emph{left}.

\item[{size}] \leavevmode{[}int or tuple of ints, optional{]}
Output shape. Default is None, in which case a single value is
returned.

\end{description}
\begin{description}
\item[{samples}] \leavevmode{[}ndarray or scalar{]}
The returned samples all lie in the interval {[}left, right{]}.

\end{description}

The probability density function for the Triangular distribution is
\begin{gather}
\begin{split}P(x;l, m, r) = \begin{cases}
\frac{2(x-l)}{(r-l)(m-l)}& \text{for $l \leq x \leq m$},\\
\frac{2(m-x)}{(r-l)(r-m)}& \text{for $m \leq x \leq r$},\\
0& \text{otherwise}.
\end{cases}\end{split}\notag
\end{gather}
The triangular distribution is often used in ill-defined problems where the
underlying distribution is not known, but some knowledge of the limits and
mode exists. Often it is used in simulations.

Draw values from the distribution and plot the histogram:

\begin{Verbatim}[commandchars=\\\{\}]
\PYG{g+gp}{\PYGZgt{}\PYGZgt{}\PYGZgt{} }\PYG{k+kn}{import} \PYG{n+nn}{matplotlib.pyplot} \PYG{k+kn}{as} \PYG{n+nn}{plt}
\PYG{g+gp}{\PYGZgt{}\PYGZgt{}\PYGZgt{} }\PYG{n}{h} \PYG{o}{=} \PYG{n}{plt}\PYG{o}{.}\PYG{n}{hist}\PYG{p}{(}\PYG{n}{np}\PYG{o}{.}\PYG{n}{random}\PYG{o}{.}\PYG{n}{triangular}\PYG{p}{(}\PYG{o}{\PYGZhy{}}\PYG{l+m+mi}{3}\PYG{p}{,} \PYG{l+m+mi}{0}\PYG{p}{,} \PYG{l+m+mi}{8}\PYG{p}{,} \PYG{l+m+mi}{100000}\PYG{p}{)}\PYG{p}{,} \PYG{n}{bins}\PYG{o}{=}\PYG{l+m+mi}{200}\PYG{p}{,}
\PYG{g+gp}{... }             \PYG{n}{normed}\PYG{o}{=}\PYG{n+nb+bp}{True}\PYG{p}{)}
\PYG{g+gp}{\PYGZgt{}\PYGZgt{}\PYGZgt{} }\PYG{n}{plt}\PYG{o}{.}\PYG{n}{show}\PYG{p}{(}\PYG{p}{)}
\end{Verbatim}

\end{fulllineitems}

\index{uniform() (in module graph\_chemistry\_analysis)}

\begin{fulllineitems}
\phantomsection\label{graph_chemistry_analysis:graph_chemistry_analysis.uniform}\pysiglinewithargsret{\code{graph\_chemistry\_analysis.}\bfcode{uniform}}{\emph{low=0.0}, \emph{high=1.0}, \emph{size=1}}{}
Draw samples from a uniform distribution.

Samples are uniformly distributed over the half-open interval
\code{{[}low, high)} (includes low, but excludes high).  In other words,
any value within the given interval is equally likely to be drawn
by \emph{uniform}.
\begin{description}
\item[{low}] \leavevmode{[}float, optional{]}
Lower boundary of the output interval.  All values generated will be
greater than or equal to low.  The default value is 0.

\item[{high}] \leavevmode{[}float{]}
Upper boundary of the output interval.  All values generated will be
less than high.  The default value is 1.0.

\item[{size}] \leavevmode{[}int or tuple of ints, optional{]}
Shape of output.  If the given size is, for example, (m,n,k),
m*n*k samples are generated.  If no shape is specified, a single sample
is returned.

\end{description}
\begin{description}
\item[{out}] \leavevmode{[}ndarray{]}
Drawn samples, with shape \emph{size}.

\end{description}

randint : Discrete uniform distribution, yielding integers.
random\_integers : Discrete uniform distribution over the closed
\begin{quote}

interval \code{{[}low, high{]}}.
\end{quote}

random\_sample : Floats uniformly distributed over \code{{[}0, 1)}.
random : Alias for \emph{random\_sample}.
rand : Convenience function that accepts dimensions as input, e.g.,
\begin{quote}

\code{rand(2,2)} would generate a 2-by-2 array of floats,
uniformly distributed over \code{{[}0, 1)}.
\end{quote}

The probability density function of the uniform distribution is
\begin{gather}
\begin{split}p(x) = \frac{1}{b - a}\end{split}\notag
\end{gather}
anywhere within the interval \code{{[}a, b)}, and zero elsewhere.

Draw samples from the distribution:

\begin{Verbatim}[commandchars=\\\{\}]
\PYG{g+gp}{\PYGZgt{}\PYGZgt{}\PYGZgt{} }\PYG{n}{s} \PYG{o}{=} \PYG{n}{np}\PYG{o}{.}\PYG{n}{random}\PYG{o}{.}\PYG{n}{uniform}\PYG{p}{(}\PYG{o}{\PYGZhy{}}\PYG{l+m+mi}{1}\PYG{p}{,}\PYG{l+m+mi}{0}\PYG{p}{,}\PYG{l+m+mi}{1000}\PYG{p}{)}
\end{Verbatim}

All values are within the given interval:

\begin{Verbatim}[commandchars=\\\{\}]
\PYG{g+gp}{\PYGZgt{}\PYGZgt{}\PYGZgt{} }\PYG{n}{np}\PYG{o}{.}\PYG{n}{all}\PYG{p}{(}\PYG{n}{s} \PYG{o}{\PYGZgt{}}\PYG{o}{=} \PYG{o}{\PYGZhy{}}\PYG{l+m+mi}{1}\PYG{p}{)}
\PYG{g+go}{True}
\PYG{g+gp}{\PYGZgt{}\PYGZgt{}\PYGZgt{} }\PYG{n}{np}\PYG{o}{.}\PYG{n}{all}\PYG{p}{(}\PYG{n}{s} \PYG{o}{\PYGZlt{}} \PYG{l+m+mi}{0}\PYG{p}{)}
\PYG{g+go}{True}
\end{Verbatim}

Display the histogram of the samples, along with the
probability density function:

\begin{Verbatim}[commandchars=\\\{\}]
\PYG{g+gp}{\PYGZgt{}\PYGZgt{}\PYGZgt{} }\PYG{k+kn}{import} \PYG{n+nn}{matplotlib.pyplot} \PYG{k+kn}{as} \PYG{n+nn}{plt}
\PYG{g+gp}{\PYGZgt{}\PYGZgt{}\PYGZgt{} }\PYG{n}{count}\PYG{p}{,} \PYG{n}{bins}\PYG{p}{,} \PYG{n}{ignored} \PYG{o}{=} \PYG{n}{plt}\PYG{o}{.}\PYG{n}{hist}\PYG{p}{(}\PYG{n}{s}\PYG{p}{,} \PYG{l+m+mi}{15}\PYG{p}{,} \PYG{n}{normed}\PYG{o}{=}\PYG{n+nb+bp}{True}\PYG{p}{)}
\PYG{g+gp}{\PYGZgt{}\PYGZgt{}\PYGZgt{} }\PYG{n}{plt}\PYG{o}{.}\PYG{n}{plot}\PYG{p}{(}\PYG{n}{bins}\PYG{p}{,} \PYG{n}{np}\PYG{o}{.}\PYG{n}{ones\PYGZus{}like}\PYG{p}{(}\PYG{n}{bins}\PYG{p}{)}\PYG{p}{,} \PYG{n}{linewidth}\PYG{o}{=}\PYG{l+m+mi}{2}\PYG{p}{,} \PYG{n}{color}\PYG{o}{=}\PYG{l+s}{\PYGZsq{}}\PYG{l+s}{r}\PYG{l+s}{\PYGZsq{}}\PYG{p}{)}
\PYG{g+gp}{\PYGZgt{}\PYGZgt{}\PYGZgt{} }\PYG{n}{plt}\PYG{o}{.}\PYG{n}{show}\PYG{p}{(}\PYG{p}{)}
\end{Verbatim}

\end{fulllineitems}

\index{vonmises() (in module graph\_chemistry\_analysis)}

\begin{fulllineitems}
\phantomsection\label{graph_chemistry_analysis:graph_chemistry_analysis.vonmises}\pysiglinewithargsret{\code{graph\_chemistry\_analysis.}\bfcode{vonmises}}{\emph{mu}, \emph{kappa}, \emph{size=None}}{}
Draw samples from a von Mises distribution.

Samples are drawn from a von Mises distribution with specified mode
(mu) and dispersion (kappa), on the interval {[}-pi, pi{]}.

The von Mises distribution (also known as the circular normal
distribution) is a continuous probability distribution on the unit
circle.  It may be thought of as the circular analogue of the normal
distribution.
\begin{description}
\item[{mu}] \leavevmode{[}float{]}
Mode (``center'') of the distribution.

\item[{kappa}] \leavevmode{[}float{]}
Dispersion of the distribution, has to be \textgreater{}=0.

\item[{size}] \leavevmode{[}int or tuple of int{]}
Output shape.  If the given shape is, e.g., \code{(m, n, k)}, then
\code{m * n * k} samples are drawn.

\end{description}
\begin{description}
\item[{samples}] \leavevmode{[}scalar or ndarray{]}
The returned samples, which are in the interval {[}-pi, pi{]}.

\end{description}
\begin{description}
\item[{scipy.stats.distributions.vonmises}] \leavevmode{[}probability density function,{]}
distribution, or cumulative density function, etc.

\end{description}

The probability density for the von Mises distribution is
\begin{gather}
\begin{split}p(x) = \frac{e^{\kappa cos(x-\mu)}}{2\pi I_0(\kappa)},\end{split}\notag
\end{gather}
where \(\mu\) is the mode and \(\kappa\) the dispersion,
and \(I_0(\kappa)\) is the modified Bessel function of order 0.

The von Mises is named for Richard Edler von Mises, who was born in
Austria-Hungary, in what is now the Ukraine.  He fled to the United
States in 1939 and became a professor at Harvard.  He worked in
probability theory, aerodynamics, fluid mechanics, and philosophy of
science.

Abramowitz, M. and Stegun, I. A. (ed.), \emph{Handbook of Mathematical
Functions}, New York: Dover, 1965.

von Mises, R., \emph{Mathematical Theory of Probability and Statistics},
New York: Academic Press, 1964.

Draw samples from the distribution:

\begin{Verbatim}[commandchars=\\\{\}]
\PYG{g+gp}{\PYGZgt{}\PYGZgt{}\PYGZgt{} }\PYG{n}{mu}\PYG{p}{,} \PYG{n}{kappa} \PYG{o}{=} \PYG{l+m+mf}{0.0}\PYG{p}{,} \PYG{l+m+mf}{4.0} \PYG{c}{\PYGZsh{} mean and dispersion}
\PYG{g+gp}{\PYGZgt{}\PYGZgt{}\PYGZgt{} }\PYG{n}{s} \PYG{o}{=} \PYG{n}{np}\PYG{o}{.}\PYG{n}{random}\PYG{o}{.}\PYG{n}{vonmises}\PYG{p}{(}\PYG{n}{mu}\PYG{p}{,} \PYG{n}{kappa}\PYG{p}{,} \PYG{l+m+mi}{1000}\PYG{p}{)}
\end{Verbatim}

Display the histogram of the samples, along with
the probability density function:

\begin{Verbatim}[commandchars=\\\{\}]
\PYG{g+gp}{\PYGZgt{}\PYGZgt{}\PYGZgt{} }\PYG{k+kn}{import} \PYG{n+nn}{matplotlib.pyplot} \PYG{k+kn}{as} \PYG{n+nn}{plt}
\PYG{g+gp}{\PYGZgt{}\PYGZgt{}\PYGZgt{} }\PYG{k+kn}{import} \PYG{n+nn}{scipy.special} \PYG{k+kn}{as} \PYG{n+nn}{sps}
\PYG{g+gp}{\PYGZgt{}\PYGZgt{}\PYGZgt{} }\PYG{n}{count}\PYG{p}{,} \PYG{n}{bins}\PYG{p}{,} \PYG{n}{ignored} \PYG{o}{=} \PYG{n}{plt}\PYG{o}{.}\PYG{n}{hist}\PYG{p}{(}\PYG{n}{s}\PYG{p}{,} \PYG{l+m+mi}{50}\PYG{p}{,} \PYG{n}{normed}\PYG{o}{=}\PYG{n+nb+bp}{True}\PYG{p}{)}
\PYG{g+gp}{\PYGZgt{}\PYGZgt{}\PYGZgt{} }\PYG{n}{x} \PYG{o}{=} \PYG{n}{np}\PYG{o}{.}\PYG{n}{arange}\PYG{p}{(}\PYG{o}{\PYGZhy{}}\PYG{n}{np}\PYG{o}{.}\PYG{n}{pi}\PYG{p}{,} \PYG{n}{np}\PYG{o}{.}\PYG{n}{pi}\PYG{p}{,} \PYG{l+m+mi}{2}\PYG{o}{*}\PYG{n}{np}\PYG{o}{.}\PYG{n}{pi}\PYG{o}{/}\PYG{l+m+mf}{50.}\PYG{p}{)}
\PYG{g+gp}{\PYGZgt{}\PYGZgt{}\PYGZgt{} }\PYG{n}{y} \PYG{o}{=} \PYG{o}{\PYGZhy{}}\PYG{n}{np}\PYG{o}{.}\PYG{n}{exp}\PYG{p}{(}\PYG{n}{kappa}\PYG{o}{*}\PYG{n}{np}\PYG{o}{.}\PYG{n}{cos}\PYG{p}{(}\PYG{n}{x}\PYG{o}{\PYGZhy{}}\PYG{n}{mu}\PYG{p}{)}\PYG{p}{)}\PYG{o}{/}\PYG{p}{(}\PYG{l+m+mi}{2}\PYG{o}{*}\PYG{n}{np}\PYG{o}{.}\PYG{n}{pi}\PYG{o}{*}\PYG{n}{sps}\PYG{o}{.}\PYG{n}{jn}\PYG{p}{(}\PYG{l+m+mi}{0}\PYG{p}{,}\PYG{n}{kappa}\PYG{p}{)}\PYG{p}{)}
\PYG{g+gp}{\PYGZgt{}\PYGZgt{}\PYGZgt{} }\PYG{n}{plt}\PYG{o}{.}\PYG{n}{plot}\PYG{p}{(}\PYG{n}{x}\PYG{p}{,} \PYG{n}{y}\PYG{o}{/}\PYG{n+nb}{max}\PYG{p}{(}\PYG{n}{y}\PYG{p}{)}\PYG{p}{,} \PYG{n}{linewidth}\PYG{o}{=}\PYG{l+m+mi}{2}\PYG{p}{,} \PYG{n}{color}\PYG{o}{=}\PYG{l+s}{\PYGZsq{}}\PYG{l+s}{r}\PYG{l+s}{\PYGZsq{}}\PYG{p}{)}
\PYG{g+gp}{\PYGZgt{}\PYGZgt{}\PYGZgt{} }\PYG{n}{plt}\PYG{o}{.}\PYG{n}{show}\PYG{p}{(}\PYG{p}{)}
\end{Verbatim}

\end{fulllineitems}

\index{wald() (in module graph\_chemistry\_analysis)}

\begin{fulllineitems}
\phantomsection\label{graph_chemistry_analysis:graph_chemistry_analysis.wald}\pysiglinewithargsret{\code{graph\_chemistry\_analysis.}\bfcode{wald}}{\emph{mean}, \emph{scale}, \emph{size=None}}{}
Draw samples from a Wald, or Inverse Gaussian, distribution.

As the scale approaches infinity, the distribution becomes more like a
Gaussian.

Some references claim that the Wald is an Inverse Gaussian with mean=1, but
this is by no means universal.

The Inverse Gaussian distribution was first studied in relationship to
Brownian motion. In 1956 M.C.K. Tweedie used the name Inverse Gaussian
because there is an inverse relationship between the time to cover a unit
distance and distance covered in unit time.
\begin{description}
\item[{mean}] \leavevmode{[}scalar{]}
Distribution mean, should be \textgreater{} 0.

\item[{scale}] \leavevmode{[}scalar{]}
Scale parameter, should be \textgreater{}= 0.

\item[{size}] \leavevmode{[}int or tuple of ints, optional{]}
Output shape. Default is None, in which case a single value is
returned.

\end{description}
\begin{description}
\item[{samples}] \leavevmode{[}ndarray or scalar{]}
Drawn sample, all greater than zero.

\end{description}

The probability density function for the Wald distribution is
\begin{gather}
\begin{split}P(x;mean,scale) = \sqrt{\frac{scale}{2\pi x^3}}e^
\frac{-scale(x-mean)^2}{2\cdotp mean^2x}\end{split}\notag
\end{gather}
As noted above the Inverse Gaussian distribution first arise from attempts
to model Brownian Motion. It is also a competitor to the Weibull for use in
reliability modeling and modeling stock returns and interest rate
processes.

Draw values from the distribution and plot the histogram:

\begin{Verbatim}[commandchars=\\\{\}]
\PYG{g+gp}{\PYGZgt{}\PYGZgt{}\PYGZgt{} }\PYG{k+kn}{import} \PYG{n+nn}{matplotlib.pyplot} \PYG{k+kn}{as} \PYG{n+nn}{plt}
\PYG{g+gp}{\PYGZgt{}\PYGZgt{}\PYGZgt{} }\PYG{n}{h} \PYG{o}{=} \PYG{n}{plt}\PYG{o}{.}\PYG{n}{hist}\PYG{p}{(}\PYG{n}{np}\PYG{o}{.}\PYG{n}{random}\PYG{o}{.}\PYG{n}{wald}\PYG{p}{(}\PYG{l+m+mi}{3}\PYG{p}{,} \PYG{l+m+mi}{2}\PYG{p}{,} \PYG{l+m+mi}{100000}\PYG{p}{)}\PYG{p}{,} \PYG{n}{bins}\PYG{o}{=}\PYG{l+m+mi}{200}\PYG{p}{,} \PYG{n}{normed}\PYG{o}{=}\PYG{n+nb+bp}{True}\PYG{p}{)}
\PYG{g+gp}{\PYGZgt{}\PYGZgt{}\PYGZgt{} }\PYG{n}{plt}\PYG{o}{.}\PYG{n}{show}\PYG{p}{(}\PYG{p}{)}
\end{Verbatim}

\end{fulllineitems}

\index{weibull() (in module graph\_chemistry\_analysis)}

\begin{fulllineitems}
\phantomsection\label{graph_chemistry_analysis:graph_chemistry_analysis.weibull}\pysiglinewithargsret{\code{graph\_chemistry\_analysis.}\bfcode{weibull}}{\emph{a}, \emph{size=None}}{}
Weibull distribution.

Draw samples from a 1-parameter Weibull distribution with the given
shape parameter \emph{a}.
\begin{gather}
\begin{split}X = (-ln(U))^{1/a}\end{split}\notag
\end{gather}
Here, U is drawn from the uniform distribution over (0,1{]}.

The more common 2-parameter Weibull, including a scale parameter
\(\lambda\) is just \(X = \lambda(-ln(U))^{1/a}\).
\begin{description}
\item[{a}] \leavevmode{[}float{]}
Shape of the distribution.

\item[{size}] \leavevmode{[}tuple of ints{]}
Output shape.  If the given shape is, e.g., \code{(m, n, k)}, then
\code{m * n * k} samples are drawn.

\end{description}

scipy.stats.distributions.weibull\_max
scipy.stats.distributions.weibull\_min
scipy.stats.distributions.genextreme
gumbel

The Weibull (or Type III asymptotic extreme value distribution for smallest
values, SEV Type III, or Rosin-Rammler distribution) is one of a class of
Generalized Extreme Value (GEV) distributions used in modeling extreme
value problems.  This class includes the Gumbel and Frechet distributions.

The probability density for the Weibull distribution is
\begin{gather}
\begin{split}p(x) = \frac{a}
{\lambda}(\frac{x}{\lambda})^{a-1}e^{-(x/\lambda)^a},\end{split}\notag
\end{gather}
where \(a\) is the shape and \(\lambda\) the scale.

The function has its peak (the mode) at
\(\lambda(\frac{a-1}{a})^{1/a}\).

When \code{a = 1}, the Weibull distribution reduces to the exponential
distribution.

Draw samples from the distribution:

\begin{Verbatim}[commandchars=\\\{\}]
\PYG{g+gp}{\PYGZgt{}\PYGZgt{}\PYGZgt{} }\PYG{n}{a} \PYG{o}{=} \PYG{l+m+mf}{5.} \PYG{c}{\PYGZsh{} shape}
\PYG{g+gp}{\PYGZgt{}\PYGZgt{}\PYGZgt{} }\PYG{n}{s} \PYG{o}{=} \PYG{n}{np}\PYG{o}{.}\PYG{n}{random}\PYG{o}{.}\PYG{n}{weibull}\PYG{p}{(}\PYG{n}{a}\PYG{p}{,} \PYG{l+m+mi}{1000}\PYG{p}{)}
\end{Verbatim}

Display the histogram of the samples, along with
the probability density function:

\begin{Verbatim}[commandchars=\\\{\}]
\PYG{g+gp}{\PYGZgt{}\PYGZgt{}\PYGZgt{} }\PYG{k+kn}{import} \PYG{n+nn}{matplotlib.pyplot} \PYG{k+kn}{as} \PYG{n+nn}{plt}
\PYG{g+gp}{\PYGZgt{}\PYGZgt{}\PYGZgt{} }\PYG{n}{x} \PYG{o}{=} \PYG{n}{np}\PYG{o}{.}\PYG{n}{arange}\PYG{p}{(}\PYG{l+m+mi}{1}\PYG{p}{,}\PYG{l+m+mf}{100.}\PYG{p}{)}\PYG{o}{/}\PYG{l+m+mf}{50.}
\PYG{g+gp}{\PYGZgt{}\PYGZgt{}\PYGZgt{} }\PYG{k}{def} \PYG{n+nf}{weib}\PYG{p}{(}\PYG{n}{x}\PYG{p}{,}\PYG{n}{n}\PYG{p}{,}\PYG{n}{a}\PYG{p}{)}\PYG{p}{:}
\PYG{g+gp}{... }    \PYG{k}{return} \PYG{p}{(}\PYG{n}{a} \PYG{o}{/} \PYG{n}{n}\PYG{p}{)} \PYG{o}{*} \PYG{p}{(}\PYG{n}{x} \PYG{o}{/} \PYG{n}{n}\PYG{p}{)}\PYG{o}{*}\PYG{o}{*}\PYG{p}{(}\PYG{n}{a} \PYG{o}{\PYGZhy{}} \PYG{l+m+mi}{1}\PYG{p}{)} \PYG{o}{*} \PYG{n}{np}\PYG{o}{.}\PYG{n}{exp}\PYG{p}{(}\PYG{o}{\PYGZhy{}}\PYG{p}{(}\PYG{n}{x} \PYG{o}{/} \PYG{n}{n}\PYG{p}{)}\PYG{o}{*}\PYG{o}{*}\PYG{n}{a}\PYG{p}{)}
\end{Verbatim}

\begin{Verbatim}[commandchars=\\\{\}]
\PYG{g+gp}{\PYGZgt{}\PYGZgt{}\PYGZgt{} }\PYG{n}{count}\PYG{p}{,} \PYG{n}{bins}\PYG{p}{,} \PYG{n}{ignored} \PYG{o}{=} \PYG{n}{plt}\PYG{o}{.}\PYG{n}{hist}\PYG{p}{(}\PYG{n}{np}\PYG{o}{.}\PYG{n}{random}\PYG{o}{.}\PYG{n}{weibull}\PYG{p}{(}\PYG{l+m+mf}{5.}\PYG{p}{,}\PYG{l+m+mi}{1000}\PYG{p}{)}\PYG{p}{)}
\PYG{g+gp}{\PYGZgt{}\PYGZgt{}\PYGZgt{} }\PYG{n}{x} \PYG{o}{=} \PYG{n}{np}\PYG{o}{.}\PYG{n}{arange}\PYG{p}{(}\PYG{l+m+mi}{1}\PYG{p}{,}\PYG{l+m+mf}{100.}\PYG{p}{)}\PYG{o}{/}\PYG{l+m+mf}{50.}
\PYG{g+gp}{\PYGZgt{}\PYGZgt{}\PYGZgt{} }\PYG{n}{scale} \PYG{o}{=} \PYG{n}{count}\PYG{o}{.}\PYG{n}{max}\PYG{p}{(}\PYG{p}{)}\PYG{o}{/}\PYG{n}{weib}\PYG{p}{(}\PYG{n}{x}\PYG{p}{,} \PYG{l+m+mf}{1.}\PYG{p}{,} \PYG{l+m+mf}{5.}\PYG{p}{)}\PYG{o}{.}\PYG{n}{max}\PYG{p}{(}\PYG{p}{)}
\PYG{g+gp}{\PYGZgt{}\PYGZgt{}\PYGZgt{} }\PYG{n}{plt}\PYG{o}{.}\PYG{n}{plot}\PYG{p}{(}\PYG{n}{x}\PYG{p}{,} \PYG{n}{weib}\PYG{p}{(}\PYG{n}{x}\PYG{p}{,} \PYG{l+m+mf}{1.}\PYG{p}{,} \PYG{l+m+mf}{5.}\PYG{p}{)}\PYG{o}{*}\PYG{n}{scale}\PYG{p}{)}
\PYG{g+gp}{\PYGZgt{}\PYGZgt{}\PYGZgt{} }\PYG{n}{plt}\PYG{o}{.}\PYG{n}{show}\PYG{p}{(}\PYG{p}{)}
\end{Verbatim}

\end{fulllineitems}

\index{zipf() (in module graph\_chemistry\_analysis)}

\begin{fulllineitems}
\phantomsection\label{graph_chemistry_analysis:graph_chemistry_analysis.zipf}\pysiglinewithargsret{\code{graph\_chemistry\_analysis.}\bfcode{zipf}}{\emph{a}, \emph{size=None}}{}
Draw samples from a Zipf distribution.

Samples are drawn from a Zipf distribution with specified parameter
\emph{a} \textgreater{} 1.

The Zipf distribution (also known as the zeta distribution) is a
continuous probability distribution that satisfies Zipf's law: the
frequency of an item is inversely proportional to its rank in a
frequency table.
\begin{description}
\item[{a}] \leavevmode{[}float \textgreater{} 1{]}
Distribution parameter.

\item[{size}] \leavevmode{[}int or tuple of int, optional{]}
Output shape.  If the given shape is, e.g., \code{(m, n, k)}, then
\code{m * n * k} samples are drawn; a single integer is equivalent in
its result to providing a mono-tuple, i.e., a 1-D array of length
\emph{size} is returned.  The default is None, in which case a single
scalar is returned.

\end{description}
\begin{description}
\item[{samples}] \leavevmode{[}scalar or ndarray{]}
The returned samples are greater than or equal to one.

\end{description}
\begin{description}
\item[{scipy.stats.distributions.zipf}] \leavevmode{[}probability density function,{]}
distribution, or cumulative density function, etc.

\end{description}

The probability density for the Zipf distribution is
\begin{gather}
\begin{split}p(x) = \frac{x^{-a}}{\zeta(a)},\end{split}\notag
\end{gather}
where \(\zeta\) is the Riemann Zeta function.

It is named for the American linguist George Kingsley Zipf, who noted
that the frequency of any word in a sample of a language is inversely
proportional to its rank in the frequency table.

Zipf, G. K., \emph{Selected Studies of the Principle of Relative Frequency
in Language}, Cambridge, MA: Harvard Univ. Press, 1932.

Draw samples from the distribution:

\begin{Verbatim}[commandchars=\\\{\}]
\PYG{g+gp}{\PYGZgt{}\PYGZgt{}\PYGZgt{} }\PYG{n}{a} \PYG{o}{=} \PYG{l+m+mf}{2.} \PYG{c}{\PYGZsh{} parameter}
\PYG{g+gp}{\PYGZgt{}\PYGZgt{}\PYGZgt{} }\PYG{n}{s} \PYG{o}{=} \PYG{n}{np}\PYG{o}{.}\PYG{n}{random}\PYG{o}{.}\PYG{n}{zipf}\PYG{p}{(}\PYG{n}{a}\PYG{p}{,} \PYG{l+m+mi}{1000}\PYG{p}{)}
\end{Verbatim}

Display the histogram of the samples, along with
the probability density function:

\begin{Verbatim}[commandchars=\\\{\}]
\PYG{g+gp}{\PYGZgt{}\PYGZgt{}\PYGZgt{} }\PYG{k+kn}{import} \PYG{n+nn}{matplotlib.pyplot} \PYG{k+kn}{as} \PYG{n+nn}{plt}
\PYG{g+gp}{\PYGZgt{}\PYGZgt{}\PYGZgt{} }\PYG{k+kn}{import} \PYG{n+nn}{scipy.special} \PYG{k+kn}{as} \PYG{n+nn}{sps}
\PYG{g+go}{Truncate s values at 50 so plot is interesting}
\PYG{g+gp}{\PYGZgt{}\PYGZgt{}\PYGZgt{} }\PYG{n}{count}\PYG{p}{,} \PYG{n}{bins}\PYG{p}{,} \PYG{n}{ignored} \PYG{o}{=} \PYG{n}{plt}\PYG{o}{.}\PYG{n}{hist}\PYG{p}{(}\PYG{n}{s}\PYG{p}{[}\PYG{n}{s}\PYG{o}{\PYGZlt{}}\PYG{l+m+mi}{50}\PYG{p}{]}\PYG{p}{,} \PYG{l+m+mi}{50}\PYG{p}{,} \PYG{n}{normed}\PYG{o}{=}\PYG{n+nb+bp}{True}\PYG{p}{)}
\PYG{g+gp}{\PYGZgt{}\PYGZgt{}\PYGZgt{} }\PYG{n}{x} \PYG{o}{=} \PYG{n}{np}\PYG{o}{.}\PYG{n}{arange}\PYG{p}{(}\PYG{l+m+mf}{1.}\PYG{p}{,} \PYG{l+m+mf}{50.}\PYG{p}{)}
\PYG{g+gp}{\PYGZgt{}\PYGZgt{}\PYGZgt{} }\PYG{n}{y} \PYG{o}{=} \PYG{n}{x}\PYG{o}{*}\PYG{o}{*}\PYG{p}{(}\PYG{o}{\PYGZhy{}}\PYG{n}{a}\PYG{p}{)}\PYG{o}{/}\PYG{n}{sps}\PYG{o}{.}\PYG{n}{zetac}\PYG{p}{(}\PYG{n}{a}\PYG{p}{)}
\PYG{g+gp}{\PYGZgt{}\PYGZgt{}\PYGZgt{} }\PYG{n}{plt}\PYG{o}{.}\PYG{n}{plot}\PYG{p}{(}\PYG{n}{x}\PYG{p}{,} \PYG{n}{y}\PYG{o}{/}\PYG{n+nb}{max}\PYG{p}{(}\PYG{n}{y}\PYG{p}{)}\PYG{p}{,} \PYG{n}{linewidth}\PYG{o}{=}\PYG{l+m+mi}{2}\PYG{p}{,} \PYG{n}{color}\PYG{o}{=}\PYG{l+s}{\PYGZsq{}}\PYG{l+s}{r}\PYG{l+s}{\PYGZsq{}}\PYG{p}{)}
\PYG{g+gp}{\PYGZgt{}\PYGZgt{}\PYGZgt{} }\PYG{n}{plt}\PYG{o}{.}\PYG{n}{show}\PYG{p}{(}\PYG{p}{)}
\end{Verbatim}

\end{fulllineitems}



\chapter{acsAttractorAnalysis Module}
\label{acsAttractorAnalysis::doc}\label{acsAttractorAnalysis:module-acsAttractorAnalysis}\label{acsAttractorAnalysis:acsattractoranalysis-module}\index{acsAttractorAnalysis (module)}
Created on 4 February 2014

Author: Alessandro Filisetti \textless{}\href{mailto:alessandro.filisetti@gmail.com}{alessandro.filisetti@gmail.com}\textgreater{}

Function to analyze the different final dynamical states from different simulations (final states).
The algorithm compares all the final states (in terms of concentrations) of the simulations contained in StrPath. 
If there are both several generations and simulations the script will process everything.
\index{beta() (in module acsAttractorAnalysis)}

\begin{fulllineitems}
\phantomsection\label{acsAttractorAnalysis:acsAttractorAnalysis.beta}\pysiglinewithargsret{\code{acsAttractorAnalysis.}\bfcode{beta}}{\emph{a}, \emph{b}, \emph{size=None}}{}
The Beta distribution over \code{{[}0, 1{]}}.

The Beta distribution is a special case of the Dirichlet distribution,
and is related to the Gamma distribution.  It has the probability
distribution function
\begin{gather}
\begin{split}f(x; a,b) = \frac{1}{B(\alpha, \beta)} x^{\alpha - 1}
(1 - x)^{\beta - 1},\end{split}\notag
\end{gather}
where the normalisation, B, is the beta function,
\begin{gather}
\begin{split}B(\alpha, \beta) = \int_0^1 t^{\alpha - 1}
(1 - t)^{\beta - 1} dt.\end{split}\notag
\end{gather}
It is often seen in Bayesian inference and order statistics.
\begin{description}
\item[{a}] \leavevmode{[}float{]}
Alpha, non-negative.

\item[{b}] \leavevmode{[}float{]}
Beta, non-negative.

\item[{size}] \leavevmode{[}tuple of ints, optional{]}
The number of samples to draw.  The output is packed according to
the size given.

\end{description}
\begin{description}
\item[{out}] \leavevmode{[}ndarray{]}
Array of the given shape, containing values drawn from a
Beta distribution.

\end{description}

\end{fulllineitems}

\index{binomial() (in module acsAttractorAnalysis)}

\begin{fulllineitems}
\phantomsection\label{acsAttractorAnalysis:acsAttractorAnalysis.binomial}\pysiglinewithargsret{\code{acsAttractorAnalysis.}\bfcode{binomial}}{\emph{n}, \emph{p}, \emph{size=None}}{}
Draw samples from a binomial distribution.

Samples are drawn from a Binomial distribution with specified
parameters, n trials and p probability of success where
n an integer \textgreater{}= 0 and p is in the interval {[}0,1{]}. (n may be
input as a float, but it is truncated to an integer in use)
\begin{description}
\item[{n}] \leavevmode{[}float (but truncated to an integer){]}
parameter, \textgreater{}= 0.

\item[{p}] \leavevmode{[}float{]}
parameter, \textgreater{}= 0 and \textless{}=1.

\item[{size}] \leavevmode{[}\{tuple, int\}{]}
Output shape.  If the given shape is, e.g., \code{(m, n, k)}, then
\code{m * n * k} samples are drawn.

\end{description}
\begin{description}
\item[{samples}] \leavevmode{[}\{ndarray, scalar\}{]}
where the values are all integers in  {[}0, n{]}.

\end{description}
\begin{description}
\item[{scipy.stats.distributions.binom}] \leavevmode{[}probability density function,{]}
distribution or cumulative density function, etc.

\end{description}

The probability density for the Binomial distribution is
\begin{gather}
\begin{split}P(N) = \binom{n}{N}p^N(1-p)^{n-N},\end{split}\notag
\end{gather}
where \(n\) is the number of trials, \(p\) is the probability
of success, and \(N\) is the number of successes.

When estimating the standard error of a proportion in a population by
using a random sample, the normal distribution works well unless the
product p*n \textless{}=5, where p = population proportion estimate, and n =
number of samples, in which case the binomial distribution is used
instead. For example, a sample of 15 people shows 4 who are left
handed, and 11 who are right handed. Then p = 4/15 = 27\%. 0.27*15 = 4,
so the binomial distribution should be used in this case.

Draw samples from the distribution:

\begin{Verbatim}[commandchars=\\\{\}]
\PYG{g+gp}{\PYGZgt{}\PYGZgt{}\PYGZgt{} }\PYG{n}{n}\PYG{p}{,} \PYG{n}{p} \PYG{o}{=} \PYG{l+m+mi}{10}\PYG{p}{,} \PYG{o}{.}\PYG{l+m+mi}{5} \PYG{c}{\PYGZsh{} number of trials, probability of each trial}
\PYG{g+gp}{\PYGZgt{}\PYGZgt{}\PYGZgt{} }\PYG{n}{s} \PYG{o}{=} \PYG{n}{np}\PYG{o}{.}\PYG{n}{random}\PYG{o}{.}\PYG{n}{binomial}\PYG{p}{(}\PYG{n}{n}\PYG{p}{,} \PYG{n}{p}\PYG{p}{,} \PYG{l+m+mi}{1000}\PYG{p}{)}
\PYG{g+go}{\PYGZsh{} result of flipping a coin 10 times, tested 1000 times.}
\end{Verbatim}

A real world example. A company drills 9 wild-cat oil exploration
wells, each with an estimated probability of success of 0.1. All nine
wells fail. What is the probability of that happening?

Let's do 20,000 trials of the model, and count the number that
generate zero positive results.

\begin{Verbatim}[commandchars=\\\{\}]
\PYG{g+gp}{\PYGZgt{}\PYGZgt{}\PYGZgt{} }\PYG{n+nb}{sum}\PYG{p}{(}\PYG{n}{np}\PYG{o}{.}\PYG{n}{random}\PYG{o}{.}\PYG{n}{binomial}\PYG{p}{(}\PYG{l+m+mi}{9}\PYG{p}{,}\PYG{l+m+mf}{0.1}\PYG{p}{,}\PYG{l+m+mi}{20000}\PYG{p}{)}\PYG{o}{==}\PYG{l+m+mi}{0}\PYG{p}{)}\PYG{o}{/}\PYG{l+m+mf}{20000.}
\PYG{g+go}{answer = 0.38885, or 38\PYGZpc{}.}
\end{Verbatim}

\end{fulllineitems}

\index{chisquare() (in module acsAttractorAnalysis)}

\begin{fulllineitems}
\phantomsection\label{acsAttractorAnalysis:acsAttractorAnalysis.chisquare}\pysiglinewithargsret{\code{acsAttractorAnalysis.}\bfcode{chisquare}}{\emph{df}, \emph{size=None}}{}
Draw samples from a chi-square distribution.

When \emph{df} independent random variables, each with standard normal
distributions (mean 0, variance 1), are squared and summed, the
resulting distribution is chi-square (see Notes).  This distribution
is often used in hypothesis testing.
\begin{description}
\item[{df}] \leavevmode{[}int{]}
Number of degrees of freedom.

\item[{size}] \leavevmode{[}tuple of ints, int, optional{]}
Size of the returned array.  By default, a scalar is
returned.

\end{description}
\begin{description}
\item[{output}] \leavevmode{[}ndarray{]}
Samples drawn from the distribution, packed in a \emph{size}-shaped
array.

\end{description}
\begin{description}
\item[{ValueError}] \leavevmode
When \emph{df} \textless{}= 0 or when an inappropriate \emph{size} (e.g. \code{size=-1})
is given.

\end{description}

The variable obtained by summing the squares of \emph{df} independent,
standard normally distributed random variables:
\begin{gather}
\begin{split}Q = \sum_{i=0}^{\mathtt{df}} X^2_i\end{split}\notag
\end{gather}
is chi-square distributed, denoted
\begin{gather}
\begin{split}Q \sim \chi^2_k.\end{split}\notag
\end{gather}
The probability density function of the chi-squared distribution is
\begin{gather}
\begin{split}p(x) = \frac{(1/2)^{k/2}}{\Gamma(k/2)}
x^{k/2 - 1} e^{-x/2},\end{split}\notag
\end{gather}
where \(\Gamma\) is the gamma function,
\begin{gather}
\begin{split}\Gamma(x) = \int_0^{-\infty} t^{x - 1} e^{-t} dt.\end{split}\notag
\end{gather}
\href{http://www.itl.nist.gov/div898/handbook/eda/section3/eda3666.htm}{NIST/SEMATECH e-Handbook of Statistical Methods}

\begin{Verbatim}[commandchars=\\\{\}]
\PYG{g+gp}{\PYGZgt{}\PYGZgt{}\PYGZgt{} }\PYG{n}{np}\PYG{o}{.}\PYG{n}{random}\PYG{o}{.}\PYG{n}{chisquare}\PYG{p}{(}\PYG{l+m+mi}{2}\PYG{p}{,}\PYG{l+m+mi}{4}\PYG{p}{)}
\PYG{g+go}{array([ 1.89920014,  9.00867716,  3.13710533,  5.62318272])}
\end{Verbatim}

\end{fulllineitems}

\index{exponential() (in module acsAttractorAnalysis)}

\begin{fulllineitems}
\phantomsection\label{acsAttractorAnalysis:acsAttractorAnalysis.exponential}\pysiglinewithargsret{\code{acsAttractorAnalysis.}\bfcode{exponential}}{\emph{scale=1.0}, \emph{size=None}}{}
Exponential distribution.

Its probability density function is
\begin{gather}
\begin{split}f(x; \frac{1}{\beta}) = \frac{1}{\beta} \exp(-\frac{x}{\beta}),\end{split}\notag
\end{gather}
for \code{x \textgreater{} 0} and 0 elsewhere. \(\beta\) is the scale parameter,
which is the inverse of the rate parameter \(\lambda = 1/\beta\).
The rate parameter is an alternative, widely used parameterization
of the exponential distribution {\color{red}\bfseries{}{[}3{]}\_}.

The exponential distribution is a continuous analogue of the
geometric distribution.  It describes many common situations, such as
the size of raindrops measured over many rainstorms {\color{red}\bfseries{}{[}1{]}\_}, or the time
between page requests to Wikipedia {\color{red}\bfseries{}{[}2{]}\_}.
\begin{description}
\item[{scale}] \leavevmode{[}float{]}
The scale parameter, \(\beta = 1/\lambda\).

\item[{size}] \leavevmode{[}tuple of ints{]}
Number of samples to draw.  The output is shaped
according to \emph{size}.

\end{description}

\end{fulllineitems}

\index{f() (in module acsAttractorAnalysis)}

\begin{fulllineitems}
\phantomsection\label{acsAttractorAnalysis:acsAttractorAnalysis.f}\pysiglinewithargsret{\code{acsAttractorAnalysis.}\bfcode{f}}{\emph{dfnum}, \emph{dfden}, \emph{size=None}}{}
Draw samples from a F distribution.

Samples are drawn from an F distribution with specified parameters,
\emph{dfnum} (degrees of freedom in numerator) and \emph{dfden} (degrees of freedom
in denominator), where both parameters should be greater than zero.

The random variate of the F distribution (also known as the
Fisher distribution) is a continuous probability distribution
that arises in ANOVA tests, and is the ratio of two chi-square
variates.
\begin{description}
\item[{dfnum}] \leavevmode{[}float{]}
Degrees of freedom in numerator. Should be greater than zero.

\item[{dfden}] \leavevmode{[}float{]}
Degrees of freedom in denominator. Should be greater than zero.

\item[{size}] \leavevmode{[}\{tuple, int\}, optional{]}
Output shape.  If the given shape is, e.g., \code{(m, n, k)},
then \code{m * n * k} samples are drawn. By default only one sample
is returned.

\end{description}
\begin{description}
\item[{samples}] \leavevmode{[}\{ndarray, scalar\}{]}
Samples from the Fisher distribution.

\end{description}
\begin{description}
\item[{scipy.stats.distributions.f}] \leavevmode{[}probability density function,{]}
distribution or cumulative density function, etc.

\end{description}

The F statistic is used to compare in-group variances to between-group
variances. Calculating the distribution depends on the sampling, and
so it is a function of the respective degrees of freedom in the
problem.  The variable \emph{dfnum} is the number of samples minus one, the
between-groups degrees of freedom, while \emph{dfden} is the within-groups
degrees of freedom, the sum of the number of samples in each group
minus the number of groups.

An example from Glantz{[}1{]}, pp 47-40.
Two groups, children of diabetics (25 people) and children from people
without diabetes (25 controls). Fasting blood glucose was measured,
case group had a mean value of 86.1, controls had a mean value of
82.2. Standard deviations were 2.09 and 2.49 respectively. Are these
data consistent with the null hypothesis that the parents diabetic
status does not affect their children's blood glucose levels?
Calculating the F statistic from the data gives a value of 36.01.

Draw samples from the distribution:

\begin{Verbatim}[commandchars=\\\{\}]
\PYG{g+gp}{\PYGZgt{}\PYGZgt{}\PYGZgt{} }\PYG{n}{dfnum} \PYG{o}{=} \PYG{l+m+mf}{1.} \PYG{c}{\PYGZsh{} between group degrees of freedom}
\PYG{g+gp}{\PYGZgt{}\PYGZgt{}\PYGZgt{} }\PYG{n}{dfden} \PYG{o}{=} \PYG{l+m+mf}{48.} \PYG{c}{\PYGZsh{} within groups degrees of freedom}
\PYG{g+gp}{\PYGZgt{}\PYGZgt{}\PYGZgt{} }\PYG{n}{s} \PYG{o}{=} \PYG{n}{np}\PYG{o}{.}\PYG{n}{random}\PYG{o}{.}\PYG{n}{f}\PYG{p}{(}\PYG{n}{dfnum}\PYG{p}{,} \PYG{n}{dfden}\PYG{p}{,} \PYG{l+m+mi}{1000}\PYG{p}{)}
\end{Verbatim}

The lower bound for the top 1\% of the samples is :

\begin{Verbatim}[commandchars=\\\{\}]
\PYG{g+gp}{\PYGZgt{}\PYGZgt{}\PYGZgt{} }\PYG{n}{sort}\PYG{p}{(}\PYG{n}{s}\PYG{p}{)}\PYG{p}{[}\PYG{o}{\PYGZhy{}}\PYG{l+m+mi}{10}\PYG{p}{]}
\PYG{g+go}{7.61988120985}
\end{Verbatim}

So there is about a 1\% chance that the F statistic will exceed 7.62,
the measured value is 36, so the null hypothesis is rejected at the 1\%
level.

\end{fulllineitems}

\index{gamma() (in module acsAttractorAnalysis)}

\begin{fulllineitems}
\phantomsection\label{acsAttractorAnalysis:acsAttractorAnalysis.gamma}\pysiglinewithargsret{\code{acsAttractorAnalysis.}\bfcode{gamma}}{\emph{shape}, \emph{scale=1.0}, \emph{size=None}}{}
Draw samples from a Gamma distribution.

Samples are drawn from a Gamma distribution with specified parameters,
\emph{shape} (sometimes designated ``k'') and \emph{scale} (sometimes designated
``theta''), where both parameters are \textgreater{} 0.
\begin{description}
\item[{shape}] \leavevmode{[}scalar \textgreater{} 0{]}
The shape of the gamma distribution.

\item[{scale}] \leavevmode{[}scalar \textgreater{} 0, optional{]}
The scale of the gamma distribution.  Default is equal to 1.

\item[{size}] \leavevmode{[}shape\_tuple, optional{]}
Output shape.  If the given shape is, e.g., \code{(m, n, k)}, then
\code{m * n * k} samples are drawn.

\end{description}
\begin{description}
\item[{out}] \leavevmode{[}ndarray, float{]}
Returns one sample unless \emph{size} parameter is specified.

\end{description}
\begin{description}
\item[{scipy.stats.distributions.gamma}] \leavevmode{[}probability density function,{]}
distribution or cumulative density function, etc.

\end{description}

The probability density for the Gamma distribution is
\begin{gather}
\begin{split}p(x) = x^{k-1}\frac{e^{-x/\theta}}{\theta^k\Gamma(k)},\end{split}\notag
\end{gather}
where \(k\) is the shape and \(\theta\) the scale,
and \(\Gamma\) is the Gamma function.

The Gamma distribution is often used to model the times to failure of
electronic components, and arises naturally in processes for which the
waiting times between Poisson distributed events are relevant.

Draw samples from the distribution:

\begin{Verbatim}[commandchars=\\\{\}]
\PYG{g+gp}{\PYGZgt{}\PYGZgt{}\PYGZgt{} }\PYG{n}{shape}\PYG{p}{,} \PYG{n}{scale} \PYG{o}{=} \PYG{l+m+mf}{2.}\PYG{p}{,} \PYG{l+m+mf}{2.} \PYG{c}{\PYGZsh{} mean and dispersion}
\PYG{g+gp}{\PYGZgt{}\PYGZgt{}\PYGZgt{} }\PYG{n}{s} \PYG{o}{=} \PYG{n}{np}\PYG{o}{.}\PYG{n}{random}\PYG{o}{.}\PYG{n}{gamma}\PYG{p}{(}\PYG{n}{shape}\PYG{p}{,} \PYG{n}{scale}\PYG{p}{,} \PYG{l+m+mi}{1000}\PYG{p}{)}
\end{Verbatim}

Display the histogram of the samples, along with
the probability density function:

\begin{Verbatim}[commandchars=\\\{\}]
\PYG{g+gp}{\PYGZgt{}\PYGZgt{}\PYGZgt{} }\PYG{k+kn}{import} \PYG{n+nn}{matplotlib.pyplot} \PYG{k+kn}{as} \PYG{n+nn}{plt}
\PYG{g+gp}{\PYGZgt{}\PYGZgt{}\PYGZgt{} }\PYG{k+kn}{import} \PYG{n+nn}{scipy.special} \PYG{k+kn}{as} \PYG{n+nn}{sps}
\PYG{g+gp}{\PYGZgt{}\PYGZgt{}\PYGZgt{} }\PYG{n}{count}\PYG{p}{,} \PYG{n}{bins}\PYG{p}{,} \PYG{n}{ignored} \PYG{o}{=} \PYG{n}{plt}\PYG{o}{.}\PYG{n}{hist}\PYG{p}{(}\PYG{n}{s}\PYG{p}{,} \PYG{l+m+mi}{50}\PYG{p}{,} \PYG{n}{normed}\PYG{o}{=}\PYG{n+nb+bp}{True}\PYG{p}{)}
\PYG{g+gp}{\PYGZgt{}\PYGZgt{}\PYGZgt{} }\PYG{n}{y} \PYG{o}{=} \PYG{n}{bins}\PYG{o}{*}\PYG{o}{*}\PYG{p}{(}\PYG{n}{shape}\PYG{o}{\PYGZhy{}}\PYG{l+m+mi}{1}\PYG{p}{)}\PYG{o}{*}\PYG{p}{(}\PYG{n}{np}\PYG{o}{.}\PYG{n}{exp}\PYG{p}{(}\PYG{o}{\PYGZhy{}}\PYG{n}{bins}\PYG{o}{/}\PYG{n}{scale}\PYG{p}{)} \PYG{o}{/}
\PYG{g+gp}{... }                     \PYG{p}{(}\PYG{n}{sps}\PYG{o}{.}\PYG{n}{gamma}\PYG{p}{(}\PYG{n}{shape}\PYG{p}{)}\PYG{o}{*}\PYG{n}{scale}\PYG{o}{*}\PYG{o}{*}\PYG{n}{shape}\PYG{p}{)}\PYG{p}{)}
\PYG{g+gp}{\PYGZgt{}\PYGZgt{}\PYGZgt{} }\PYG{n}{plt}\PYG{o}{.}\PYG{n}{plot}\PYG{p}{(}\PYG{n}{bins}\PYG{p}{,} \PYG{n}{y}\PYG{p}{,} \PYG{n}{linewidth}\PYG{o}{=}\PYG{l+m+mi}{2}\PYG{p}{,} \PYG{n}{color}\PYG{o}{=}\PYG{l+s}{\PYGZsq{}}\PYG{l+s}{r}\PYG{l+s}{\PYGZsq{}}\PYG{p}{)}
\PYG{g+gp}{\PYGZgt{}\PYGZgt{}\PYGZgt{} }\PYG{n}{plt}\PYG{o}{.}\PYG{n}{show}\PYG{p}{(}\PYG{p}{)}
\end{Verbatim}

\end{fulllineitems}

\index{geometric() (in module acsAttractorAnalysis)}

\begin{fulllineitems}
\phantomsection\label{acsAttractorAnalysis:acsAttractorAnalysis.geometric}\pysiglinewithargsret{\code{acsAttractorAnalysis.}\bfcode{geometric}}{\emph{p}, \emph{size=None}}{}
Draw samples from the geometric distribution.

Bernoulli trials are experiments with one of two outcomes:
success or failure (an example of such an experiment is flipping
a coin).  The geometric distribution models the number of trials
that must be run in order to achieve success.  It is therefore
supported on the positive integers, \code{k = 1, 2, ...}.

The probability mass function of the geometric distribution is
\begin{gather}
\begin{split}f(k) = (1 - p)^{k - 1} p\end{split}\notag
\end{gather}
where \emph{p} is the probability of success of an individual trial.
\begin{description}
\item[{p}] \leavevmode{[}float{]}
The probability of success of an individual trial.

\item[{size}] \leavevmode{[}tuple of ints{]}
Number of values to draw from the distribution.  The output
is shaped according to \emph{size}.

\end{description}
\begin{description}
\item[{out}] \leavevmode{[}ndarray{]}
Samples from the geometric distribution, shaped according to
\emph{size}.

\end{description}

Draw ten thousand values from the geometric distribution,
with the probability of an individual success equal to 0.35:

\begin{Verbatim}[commandchars=\\\{\}]
\PYG{g+gp}{\PYGZgt{}\PYGZgt{}\PYGZgt{} }\PYG{n}{z} \PYG{o}{=} \PYG{n}{np}\PYG{o}{.}\PYG{n}{random}\PYG{o}{.}\PYG{n}{geometric}\PYG{p}{(}\PYG{n}{p}\PYG{o}{=}\PYG{l+m+mf}{0.35}\PYG{p}{,} \PYG{n}{size}\PYG{o}{=}\PYG{l+m+mi}{10000}\PYG{p}{)}
\end{Verbatim}

How many trials succeeded after a single run?

\begin{Verbatim}[commandchars=\\\{\}]
\PYG{g+gp}{\PYGZgt{}\PYGZgt{}\PYGZgt{} }\PYG{p}{(}\PYG{n}{z} \PYG{o}{==} \PYG{l+m+mi}{1}\PYG{p}{)}\PYG{o}{.}\PYG{n}{sum}\PYG{p}{(}\PYG{p}{)} \PYG{o}{/} \PYG{l+m+mf}{10000.}
\PYG{g+go}{0.34889999999999999 \PYGZsh{}random}
\end{Verbatim}

\end{fulllineitems}

\index{get\_state() (in module acsAttractorAnalysis)}

\begin{fulllineitems}
\phantomsection\label{acsAttractorAnalysis:acsAttractorAnalysis.get_state}\pysiglinewithargsret{\code{acsAttractorAnalysis.}\bfcode{get\_state}}{}{}
Return a tuple representing the internal state of the generator.

For more details, see \emph{set\_state}.
\begin{description}
\item[{out}] \leavevmode{[}tuple(str, ndarray of 624 uints, int, int, float){]}
The returned tuple has the following items:
\begin{enumerate}
\item {} 
the string `MT19937'.

\item {} 
a 1-D array of 624 unsigned integer keys.

\item {} 
an integer \code{pos}.

\item {} 
an integer \code{has\_gauss}.

\item {} 
a float \code{cached\_gaussian}.

\end{enumerate}

\end{description}

set\_state

\emph{set\_state} and \emph{get\_state} are not needed to work with any of the
random distributions in NumPy. If the internal state is manually altered,
the user should know exactly what he/she is doing.

\end{fulllineitems}

\index{gumbel() (in module acsAttractorAnalysis)}

\begin{fulllineitems}
\phantomsection\label{acsAttractorAnalysis:acsAttractorAnalysis.gumbel}\pysiglinewithargsret{\code{acsAttractorAnalysis.}\bfcode{gumbel}}{\emph{loc=0.0}, \emph{scale=1.0}, \emph{size=None}}{}
Gumbel distribution.

Draw samples from a Gumbel distribution with specified location and scale.
For more information on the Gumbel distribution, see Notes and References
below.
\begin{description}
\item[{loc}] \leavevmode{[}float{]}
The location of the mode of the distribution.

\item[{scale}] \leavevmode{[}float{]}
The scale parameter of the distribution.

\item[{size}] \leavevmode{[}tuple of ints{]}
Output shape.  If the given shape is, e.g., \code{(m, n, k)}, then
\code{m * n * k} samples are drawn.

\end{description}
\begin{description}
\item[{out}] \leavevmode{[}ndarray{]}
The samples

\end{description}

scipy.stats.gumbel\_l
scipy.stats.gumbel\_r
scipy.stats.genextreme
\begin{quote}

probability density function, distribution, or cumulative density
function, etc. for each of the above
\end{quote}

weibull

The Gumbel (or Smallest Extreme Value (SEV) or the Smallest Extreme Value
Type I) distribution is one of a class of Generalized Extreme Value (GEV)
distributions used in modeling extreme value problems.  The Gumbel is a
special case of the Extreme Value Type I distribution for maximums from
distributions with ``exponential-like'' tails.

The probability density for the Gumbel distribution is
\begin{gather}
\begin{split}p(x) = \frac{e^{-(x - \mu)/ \beta}}{\beta} e^{ -e^{-(x - \mu)/
\beta}},\end{split}\notag
\end{gather}
where \(\mu\) is the mode, a location parameter, and \(\beta\) is
the scale parameter.

The Gumbel (named for German mathematician Emil Julius Gumbel) was used
very early in the hydrology literature, for modeling the occurrence of
flood events. It is also used for modeling maximum wind speed and rainfall
rates.  It is a ``fat-tailed'' distribution - the probability of an event in
the tail of the distribution is larger than if one used a Gaussian, hence
the surprisingly frequent occurrence of 100-year floods. Floods were
initially modeled as a Gaussian process, which underestimated the frequency
of extreme events.

It is one of a class of extreme value distributions, the Generalized
Extreme Value (GEV) distributions, which also includes the Weibull and
Frechet.

The function has a mean of \(\mu + 0.57721\beta\) and a variance of
\(\frac{\pi^2}{6}\beta^2\).

Gumbel, E. J., \emph{Statistics of Extremes}, New York: Columbia University
Press, 1958.

Reiss, R.-D. and Thomas, M., \emph{Statistical Analysis of Extreme Values from
Insurance, Finance, Hydrology and Other Fields}, Basel: Birkhauser Verlag,
2001.

Draw samples from the distribution:

\begin{Verbatim}[commandchars=\\\{\}]
\PYG{g+gp}{\PYGZgt{}\PYGZgt{}\PYGZgt{} }\PYG{n}{mu}\PYG{p}{,} \PYG{n}{beta} \PYG{o}{=} \PYG{l+m+mi}{0}\PYG{p}{,} \PYG{l+m+mf}{0.1} \PYG{c}{\PYGZsh{} location and scale}
\PYG{g+gp}{\PYGZgt{}\PYGZgt{}\PYGZgt{} }\PYG{n}{s} \PYG{o}{=} \PYG{n}{np}\PYG{o}{.}\PYG{n}{random}\PYG{o}{.}\PYG{n}{gumbel}\PYG{p}{(}\PYG{n}{mu}\PYG{p}{,} \PYG{n}{beta}\PYG{p}{,} \PYG{l+m+mi}{1000}\PYG{p}{)}
\end{Verbatim}

Display the histogram of the samples, along with
the probability density function:

\begin{Verbatim}[commandchars=\\\{\}]
\PYG{g+gp}{\PYGZgt{}\PYGZgt{}\PYGZgt{} }\PYG{k+kn}{import} \PYG{n+nn}{matplotlib.pyplot} \PYG{k+kn}{as} \PYG{n+nn}{plt}
\PYG{g+gp}{\PYGZgt{}\PYGZgt{}\PYGZgt{} }\PYG{n}{count}\PYG{p}{,} \PYG{n}{bins}\PYG{p}{,} \PYG{n}{ignored} \PYG{o}{=} \PYG{n}{plt}\PYG{o}{.}\PYG{n}{hist}\PYG{p}{(}\PYG{n}{s}\PYG{p}{,} \PYG{l+m+mi}{30}\PYG{p}{,} \PYG{n}{normed}\PYG{o}{=}\PYG{n+nb+bp}{True}\PYG{p}{)}
\PYG{g+gp}{\PYGZgt{}\PYGZgt{}\PYGZgt{} }\PYG{n}{plt}\PYG{o}{.}\PYG{n}{plot}\PYG{p}{(}\PYG{n}{bins}\PYG{p}{,} \PYG{p}{(}\PYG{l+m+mi}{1}\PYG{o}{/}\PYG{n}{beta}\PYG{p}{)}\PYG{o}{*}\PYG{n}{np}\PYG{o}{.}\PYG{n}{exp}\PYG{p}{(}\PYG{o}{\PYGZhy{}}\PYG{p}{(}\PYG{n}{bins} \PYG{o}{\PYGZhy{}} \PYG{n}{mu}\PYG{p}{)}\PYG{o}{/}\PYG{n}{beta}\PYG{p}{)}
\PYG{g+gp}{... }         \PYG{o}{*} \PYG{n}{np}\PYG{o}{.}\PYG{n}{exp}\PYG{p}{(} \PYG{o}{\PYGZhy{}}\PYG{n}{np}\PYG{o}{.}\PYG{n}{exp}\PYG{p}{(} \PYG{o}{\PYGZhy{}}\PYG{p}{(}\PYG{n}{bins} \PYG{o}{\PYGZhy{}} \PYG{n}{mu}\PYG{p}{)} \PYG{o}{/}\PYG{n}{beta}\PYG{p}{)} \PYG{p}{)}\PYG{p}{,}
\PYG{g+gp}{... }         \PYG{n}{linewidth}\PYG{o}{=}\PYG{l+m+mi}{2}\PYG{p}{,} \PYG{n}{color}\PYG{o}{=}\PYG{l+s}{\PYGZsq{}}\PYG{l+s}{r}\PYG{l+s}{\PYGZsq{}}\PYG{p}{)}
\PYG{g+gp}{\PYGZgt{}\PYGZgt{}\PYGZgt{} }\PYG{n}{plt}\PYG{o}{.}\PYG{n}{show}\PYG{p}{(}\PYG{p}{)}
\end{Verbatim}

Show how an extreme value distribution can arise from a Gaussian process
and compare to a Gaussian:

\begin{Verbatim}[commandchars=\\\{\}]
\PYG{g+gp}{\PYGZgt{}\PYGZgt{}\PYGZgt{} }\PYG{n}{means} \PYG{o}{=} \PYG{p}{[}\PYG{p}{]}
\PYG{g+gp}{\PYGZgt{}\PYGZgt{}\PYGZgt{} }\PYG{n}{maxima} \PYG{o}{=} \PYG{p}{[}\PYG{p}{]}
\PYG{g+gp}{\PYGZgt{}\PYGZgt{}\PYGZgt{} }\PYG{k}{for} \PYG{n}{i} \PYG{o+ow}{in} \PYG{n+nb}{range}\PYG{p}{(}\PYG{l+m+mi}{0}\PYG{p}{,}\PYG{l+m+mi}{1000}\PYG{p}{)} \PYG{p}{:}
\PYG{g+gp}{... }   \PYG{n}{a} \PYG{o}{=} \PYG{n}{np}\PYG{o}{.}\PYG{n}{random}\PYG{o}{.}\PYG{n}{normal}\PYG{p}{(}\PYG{n}{mu}\PYG{p}{,} \PYG{n}{beta}\PYG{p}{,} \PYG{l+m+mi}{1000}\PYG{p}{)}
\PYG{g+gp}{... }   \PYG{n}{means}\PYG{o}{.}\PYG{n}{append}\PYG{p}{(}\PYG{n}{a}\PYG{o}{.}\PYG{n}{mean}\PYG{p}{(}\PYG{p}{)}\PYG{p}{)}
\PYG{g+gp}{... }   \PYG{n}{maxima}\PYG{o}{.}\PYG{n}{append}\PYG{p}{(}\PYG{n}{a}\PYG{o}{.}\PYG{n}{max}\PYG{p}{(}\PYG{p}{)}\PYG{p}{)}
\PYG{g+gp}{\PYGZgt{}\PYGZgt{}\PYGZgt{} }\PYG{n}{count}\PYG{p}{,} \PYG{n}{bins}\PYG{p}{,} \PYG{n}{ignored} \PYG{o}{=} \PYG{n}{plt}\PYG{o}{.}\PYG{n}{hist}\PYG{p}{(}\PYG{n}{maxima}\PYG{p}{,} \PYG{l+m+mi}{30}\PYG{p}{,} \PYG{n}{normed}\PYG{o}{=}\PYG{n+nb+bp}{True}\PYG{p}{)}
\PYG{g+gp}{\PYGZgt{}\PYGZgt{}\PYGZgt{} }\PYG{n}{beta} \PYG{o}{=} \PYG{n}{np}\PYG{o}{.}\PYG{n}{std}\PYG{p}{(}\PYG{n}{maxima}\PYG{p}{)}\PYG{o}{*}\PYG{n}{np}\PYG{o}{.}\PYG{n}{pi}\PYG{o}{/}\PYG{n}{np}\PYG{o}{.}\PYG{n}{sqrt}\PYG{p}{(}\PYG{l+m+mi}{6}\PYG{p}{)}
\PYG{g+gp}{\PYGZgt{}\PYGZgt{}\PYGZgt{} }\PYG{n}{mu} \PYG{o}{=} \PYG{n}{np}\PYG{o}{.}\PYG{n}{mean}\PYG{p}{(}\PYG{n}{maxima}\PYG{p}{)} \PYG{o}{\PYGZhy{}} \PYG{l+m+mf}{0.57721}\PYG{o}{*}\PYG{n}{beta}
\PYG{g+gp}{\PYGZgt{}\PYGZgt{}\PYGZgt{} }\PYG{n}{plt}\PYG{o}{.}\PYG{n}{plot}\PYG{p}{(}\PYG{n}{bins}\PYG{p}{,} \PYG{p}{(}\PYG{l+m+mi}{1}\PYG{o}{/}\PYG{n}{beta}\PYG{p}{)}\PYG{o}{*}\PYG{n}{np}\PYG{o}{.}\PYG{n}{exp}\PYG{p}{(}\PYG{o}{\PYGZhy{}}\PYG{p}{(}\PYG{n}{bins} \PYG{o}{\PYGZhy{}} \PYG{n}{mu}\PYG{p}{)}\PYG{o}{/}\PYG{n}{beta}\PYG{p}{)}
\PYG{g+gp}{... }         \PYG{o}{*} \PYG{n}{np}\PYG{o}{.}\PYG{n}{exp}\PYG{p}{(}\PYG{o}{\PYGZhy{}}\PYG{n}{np}\PYG{o}{.}\PYG{n}{exp}\PYG{p}{(}\PYG{o}{\PYGZhy{}}\PYG{p}{(}\PYG{n}{bins} \PYG{o}{\PYGZhy{}} \PYG{n}{mu}\PYG{p}{)}\PYG{o}{/}\PYG{n}{beta}\PYG{p}{)}\PYG{p}{)}\PYG{p}{,}
\PYG{g+gp}{... }         \PYG{n}{linewidth}\PYG{o}{=}\PYG{l+m+mi}{2}\PYG{p}{,} \PYG{n}{color}\PYG{o}{=}\PYG{l+s}{\PYGZsq{}}\PYG{l+s}{r}\PYG{l+s}{\PYGZsq{}}\PYG{p}{)}
\PYG{g+gp}{\PYGZgt{}\PYGZgt{}\PYGZgt{} }\PYG{n}{plt}\PYG{o}{.}\PYG{n}{plot}\PYG{p}{(}\PYG{n}{bins}\PYG{p}{,} \PYG{l+m+mi}{1}\PYG{o}{/}\PYG{p}{(}\PYG{n}{beta} \PYG{o}{*} \PYG{n}{np}\PYG{o}{.}\PYG{n}{sqrt}\PYG{p}{(}\PYG{l+m+mi}{2} \PYG{o}{*} \PYG{n}{np}\PYG{o}{.}\PYG{n}{pi}\PYG{p}{)}\PYG{p}{)}
\PYG{g+gp}{... }         \PYG{o}{*} \PYG{n}{np}\PYG{o}{.}\PYG{n}{exp}\PYG{p}{(}\PYG{o}{\PYGZhy{}}\PYG{p}{(}\PYG{n}{bins} \PYG{o}{\PYGZhy{}} \PYG{n}{mu}\PYG{p}{)}\PYG{o}{*}\PYG{o}{*}\PYG{l+m+mi}{2} \PYG{o}{/} \PYG{p}{(}\PYG{l+m+mi}{2} \PYG{o}{*} \PYG{n}{beta}\PYG{o}{*}\PYG{o}{*}\PYG{l+m+mi}{2}\PYG{p}{)}\PYG{p}{)}\PYG{p}{,}
\PYG{g+gp}{... }         \PYG{n}{linewidth}\PYG{o}{=}\PYG{l+m+mi}{2}\PYG{p}{,} \PYG{n}{color}\PYG{o}{=}\PYG{l+s}{\PYGZsq{}}\PYG{l+s}{g}\PYG{l+s}{\PYGZsq{}}\PYG{p}{)}
\PYG{g+gp}{\PYGZgt{}\PYGZgt{}\PYGZgt{} }\PYG{n}{plt}\PYG{o}{.}\PYG{n}{show}\PYG{p}{(}\PYG{p}{)}
\end{Verbatim}

\end{fulllineitems}

\index{hypergeometric() (in module acsAttractorAnalysis)}

\begin{fulllineitems}
\phantomsection\label{acsAttractorAnalysis:acsAttractorAnalysis.hypergeometric}\pysiglinewithargsret{\code{acsAttractorAnalysis.}\bfcode{hypergeometric}}{\emph{ngood}, \emph{nbad}, \emph{nsample}, \emph{size=None}}{}
Draw samples from a Hypergeometric distribution.

Samples are drawn from a Hypergeometric distribution with specified
parameters, ngood (ways to make a good selection), nbad (ways to make
a bad selection), and nsample = number of items sampled, which is less
than or equal to the sum ngood + nbad.
\begin{description}
\item[{ngood}] \leavevmode{[}int or array\_like{]}
Number of ways to make a good selection.  Must be nonnegative.

\item[{nbad}] \leavevmode{[}int or array\_like{]}
Number of ways to make a bad selection.  Must be nonnegative.

\item[{nsample}] \leavevmode{[}int or array\_like{]}
Number of items sampled.  Must be at least 1 and at most
\code{ngood + nbad}.

\item[{size}] \leavevmode{[}int or tuple of int{]}
Output shape.  If the given shape is, e.g., \code{(m, n, k)}, then
\code{m * n * k} samples are drawn.

\end{description}
\begin{description}
\item[{samples}] \leavevmode{[}ndarray or scalar{]}
The values are all integers in  {[}0, n{]}.

\end{description}
\begin{description}
\item[{scipy.stats.distributions.hypergeom}] \leavevmode{[}probability density function,{]}
distribution or cumulative density function, etc.

\end{description}

The probability density for the Hypergeometric distribution is
\begin{gather}
\begin{split}P(x) = \frac{\binom{m}{n}\binom{N-m}{n-x}}{\binom{N}{n}},\end{split}\notag
\end{gather}
where \(0 \le x \le m\) and \(n+m-N \le x \le n\)

for P(x) the probability of x successes, n = ngood, m = nbad, and
N = number of samples.

Consider an urn with black and white marbles in it, ngood of them
black and nbad are white. If you draw nsample balls without
replacement, then the Hypergeometric distribution describes the
distribution of black balls in the drawn sample.

Note that this distribution is very similar to the Binomial
distribution, except that in this case, samples are drawn without
replacement, whereas in the Binomial case samples are drawn with
replacement (or the sample space is infinite). As the sample space
becomes large, this distribution approaches the Binomial.

Draw samples from the distribution:

\begin{Verbatim}[commandchars=\\\{\}]
\PYG{g+gp}{\PYGZgt{}\PYGZgt{}\PYGZgt{} }\PYG{n}{ngood}\PYG{p}{,} \PYG{n}{nbad}\PYG{p}{,} \PYG{n}{nsamp} \PYG{o}{=} \PYG{l+m+mi}{100}\PYG{p}{,} \PYG{l+m+mi}{2}\PYG{p}{,} \PYG{l+m+mi}{10}
\PYG{g+go}{\PYGZsh{} number of good, number of bad, and number of samples}
\PYG{g+gp}{\PYGZgt{}\PYGZgt{}\PYGZgt{} }\PYG{n}{s} \PYG{o}{=} \PYG{n}{np}\PYG{o}{.}\PYG{n}{random}\PYG{o}{.}\PYG{n}{hypergeometric}\PYG{p}{(}\PYG{n}{ngood}\PYG{p}{,} \PYG{n}{nbad}\PYG{p}{,} \PYG{n}{nsamp}\PYG{p}{,} \PYG{l+m+mi}{1000}\PYG{p}{)}
\PYG{g+gp}{\PYGZgt{}\PYGZgt{}\PYGZgt{} }\PYG{n}{hist}\PYG{p}{(}\PYG{n}{s}\PYG{p}{)}
\PYG{g+go}{\PYGZsh{}   note that it is very unlikely to grab both bad items}
\end{Verbatim}

Suppose you have an urn with 15 white and 15 black marbles.
If you pull 15 marbles at random, how likely is it that
12 or more of them are one color?

\begin{Verbatim}[commandchars=\\\{\}]
\PYG{g+gp}{\PYGZgt{}\PYGZgt{}\PYGZgt{} }\PYG{n}{s} \PYG{o}{=} \PYG{n}{np}\PYG{o}{.}\PYG{n}{random}\PYG{o}{.}\PYG{n}{hypergeometric}\PYG{p}{(}\PYG{l+m+mi}{15}\PYG{p}{,} \PYG{l+m+mi}{15}\PYG{p}{,} \PYG{l+m+mi}{15}\PYG{p}{,} \PYG{l+m+mi}{100000}\PYG{p}{)}
\PYG{g+gp}{\PYGZgt{}\PYGZgt{}\PYGZgt{} }\PYG{n+nb}{sum}\PYG{p}{(}\PYG{n}{s}\PYG{o}{\PYGZgt{}}\PYG{o}{=}\PYG{l+m+mi}{12}\PYG{p}{)}\PYG{o}{/}\PYG{l+m+mf}{100000.} \PYG{o}{+} \PYG{n+nb}{sum}\PYG{p}{(}\PYG{n}{s}\PYG{o}{\PYGZlt{}}\PYG{o}{=}\PYG{l+m+mi}{3}\PYG{p}{)}\PYG{o}{/}\PYG{l+m+mf}{100000.}
\PYG{g+go}{\PYGZsh{}   answer = 0.003 ... pretty unlikely!}
\end{Verbatim}

\end{fulllineitems}

\index{laplace() (in module acsAttractorAnalysis)}

\begin{fulllineitems}
\phantomsection\label{acsAttractorAnalysis:acsAttractorAnalysis.laplace}\pysiglinewithargsret{\code{acsAttractorAnalysis.}\bfcode{laplace}}{\emph{loc=0.0}, \emph{scale=1.0}, \emph{size=None}}{}
Draw samples from the Laplace or double exponential distribution with
specified location (or mean) and scale (decay).

The Laplace distribution is similar to the Gaussian/normal distribution,
but is sharper at the peak and has fatter tails. It represents the
difference between two independent, identically distributed exponential
random variables.
\begin{description}
\item[{loc}] \leavevmode{[}float{]}
The position, \(\mu\), of the distribution peak.

\item[{scale}] \leavevmode{[}float{]}
\(\lambda\), the exponential decay.

\end{description}

It has the probability density function
\begin{gather}
\begin{split}f(x; \mu, \lambda) = \frac{1}{2\lambda}
\exp\left(-\frac{|x - \mu|}{\lambda}\right).\end{split}\notag
\end{gather}
The first law of Laplace, from 1774, states that the frequency of an error
can be expressed as an exponential function of the absolute magnitude of
the error, which leads to the Laplace distribution. For many problems in
Economics and Health sciences, this distribution seems to model the data
better than the standard Gaussian distribution

Draw samples from the distribution

\begin{Verbatim}[commandchars=\\\{\}]
\PYG{g+gp}{\PYGZgt{}\PYGZgt{}\PYGZgt{} }\PYG{n}{loc}\PYG{p}{,} \PYG{n}{scale} \PYG{o}{=} \PYG{l+m+mf}{0.}\PYG{p}{,} \PYG{l+m+mf}{1.}
\PYG{g+gp}{\PYGZgt{}\PYGZgt{}\PYGZgt{} }\PYG{n}{s} \PYG{o}{=} \PYG{n}{np}\PYG{o}{.}\PYG{n}{random}\PYG{o}{.}\PYG{n}{laplace}\PYG{p}{(}\PYG{n}{loc}\PYG{p}{,} \PYG{n}{scale}\PYG{p}{,} \PYG{l+m+mi}{1000}\PYG{p}{)}
\end{Verbatim}

Display the histogram of the samples, along with
the probability density function:

\begin{Verbatim}[commandchars=\\\{\}]
\PYG{g+gp}{\PYGZgt{}\PYGZgt{}\PYGZgt{} }\PYG{k+kn}{import} \PYG{n+nn}{matplotlib.pyplot} \PYG{k+kn}{as} \PYG{n+nn}{plt}
\PYG{g+gp}{\PYGZgt{}\PYGZgt{}\PYGZgt{} }\PYG{n}{count}\PYG{p}{,} \PYG{n}{bins}\PYG{p}{,} \PYG{n}{ignored} \PYG{o}{=} \PYG{n}{plt}\PYG{o}{.}\PYG{n}{hist}\PYG{p}{(}\PYG{n}{s}\PYG{p}{,} \PYG{l+m+mi}{30}\PYG{p}{,} \PYG{n}{normed}\PYG{o}{=}\PYG{n+nb+bp}{True}\PYG{p}{)}
\PYG{g+gp}{\PYGZgt{}\PYGZgt{}\PYGZgt{} }\PYG{n}{x} \PYG{o}{=} \PYG{n}{np}\PYG{o}{.}\PYG{n}{arange}\PYG{p}{(}\PYG{o}{\PYGZhy{}}\PYG{l+m+mf}{8.}\PYG{p}{,} \PYG{l+m+mf}{8.}\PYG{p}{,} \PYG{o}{.}\PYG{l+m+mo}{01}\PYG{p}{)}
\PYG{g+gp}{\PYGZgt{}\PYGZgt{}\PYGZgt{} }\PYG{n}{pdf} \PYG{o}{=} \PYG{n}{np}\PYG{o}{.}\PYG{n}{exp}\PYG{p}{(}\PYG{o}{\PYGZhy{}}\PYG{n+nb}{abs}\PYG{p}{(}\PYG{n}{x}\PYG{o}{\PYGZhy{}}\PYG{n}{loc}\PYG{o}{/}\PYG{n}{scale}\PYG{p}{)}\PYG{p}{)}\PYG{o}{/}\PYG{p}{(}\PYG{l+m+mf}{2.}\PYG{o}{*}\PYG{n}{scale}\PYG{p}{)}
\PYG{g+gp}{\PYGZgt{}\PYGZgt{}\PYGZgt{} }\PYG{n}{plt}\PYG{o}{.}\PYG{n}{plot}\PYG{p}{(}\PYG{n}{x}\PYG{p}{,} \PYG{n}{pdf}\PYG{p}{)}
\end{Verbatim}

Plot Gaussian for comparison:

\begin{Verbatim}[commandchars=\\\{\}]
\PYG{g+gp}{\PYGZgt{}\PYGZgt{}\PYGZgt{} }\PYG{n}{g} \PYG{o}{=} \PYG{p}{(}\PYG{l+m+mi}{1}\PYG{o}{/}\PYG{p}{(}\PYG{n}{scale} \PYG{o}{*} \PYG{n}{np}\PYG{o}{.}\PYG{n}{sqrt}\PYG{p}{(}\PYG{l+m+mi}{2} \PYG{o}{*} \PYG{n}{np}\PYG{o}{.}\PYG{n}{pi}\PYG{p}{)}\PYG{p}{)} \PYG{o}{*} 
\PYG{g+gp}{... }     \PYG{n}{np}\PYG{o}{.}\PYG{n}{exp}\PYG{p}{(} \PYG{o}{\PYGZhy{}} \PYG{p}{(}\PYG{n}{x} \PYG{o}{\PYGZhy{}} \PYG{n}{loc}\PYG{p}{)}\PYG{o}{*}\PYG{o}{*}\PYG{l+m+mi}{2} \PYG{o}{/} \PYG{p}{(}\PYG{l+m+mi}{2} \PYG{o}{*} \PYG{n}{scale}\PYG{o}{*}\PYG{o}{*}\PYG{l+m+mi}{2}\PYG{p}{)} \PYG{p}{)}\PYG{p}{)}
\PYG{g+gp}{\PYGZgt{}\PYGZgt{}\PYGZgt{} }\PYG{n}{plt}\PYG{o}{.}\PYG{n}{plot}\PYG{p}{(}\PYG{n}{x}\PYG{p}{,}\PYG{n}{g}\PYG{p}{)}
\end{Verbatim}

\end{fulllineitems}

\index{logistic() (in module acsAttractorAnalysis)}

\begin{fulllineitems}
\phantomsection\label{acsAttractorAnalysis:acsAttractorAnalysis.logistic}\pysiglinewithargsret{\code{acsAttractorAnalysis.}\bfcode{logistic}}{\emph{loc=0.0}, \emph{scale=1.0}, \emph{size=None}}{}
Draw samples from a Logistic distribution.

Samples are drawn from a Logistic distribution with specified
parameters, loc (location or mean, also median), and scale (\textgreater{}0).

loc : float

scale : float \textgreater{} 0.
\begin{description}
\item[{size}] \leavevmode{[}\{tuple, int\}{]}
Output shape.  If the given shape is, e.g., \code{(m, n, k)}, then
\code{m * n * k} samples are drawn.

\end{description}
\begin{description}
\item[{samples}] \leavevmode{[}\{ndarray, scalar\}{]}
where the values are all integers in  {[}0, n{]}.

\end{description}
\begin{description}
\item[{scipy.stats.distributions.logistic}] \leavevmode{[}probability density function,{]}
distribution or cumulative density function, etc.

\end{description}

The probability density for the Logistic distribution is
\begin{gather}
\begin{split}P(x) = P(x) = \frac{e^{-(x-\mu)/s}}{s(1+e^{-(x-\mu)/s})^2},\end{split}\notag
\end{gather}
where \(\mu\) = location and \(s\) = scale.

The Logistic distribution is used in Extreme Value problems where it
can act as a mixture of Gumbel distributions, in Epidemiology, and by
the World Chess Federation (FIDE) where it is used in the Elo ranking
system, assuming the performance of each player is a logistically
distributed random variable.

Draw samples from the distribution:

\begin{Verbatim}[commandchars=\\\{\}]
\PYG{g+gp}{\PYGZgt{}\PYGZgt{}\PYGZgt{} }\PYG{n}{loc}\PYG{p}{,} \PYG{n}{scale} \PYG{o}{=} \PYG{l+m+mi}{10}\PYG{p}{,} \PYG{l+m+mi}{1}
\PYG{g+gp}{\PYGZgt{}\PYGZgt{}\PYGZgt{} }\PYG{n}{s} \PYG{o}{=} \PYG{n}{np}\PYG{o}{.}\PYG{n}{random}\PYG{o}{.}\PYG{n}{logistic}\PYG{p}{(}\PYG{n}{loc}\PYG{p}{,} \PYG{n}{scale}\PYG{p}{,} \PYG{l+m+mi}{10000}\PYG{p}{)}
\PYG{g+gp}{\PYGZgt{}\PYGZgt{}\PYGZgt{} }\PYG{n}{count}\PYG{p}{,} \PYG{n}{bins}\PYG{p}{,} \PYG{n}{ignored} \PYG{o}{=} \PYG{n}{plt}\PYG{o}{.}\PYG{n}{hist}\PYG{p}{(}\PYG{n}{s}\PYG{p}{,} \PYG{n}{bins}\PYG{o}{=}\PYG{l+m+mi}{50}\PYG{p}{)}
\end{Verbatim}

\#   plot against distribution

\begin{Verbatim}[commandchars=\\\{\}]
\PYG{g+gp}{\PYGZgt{}\PYGZgt{}\PYGZgt{} }\PYG{k}{def} \PYG{n+nf}{logist}\PYG{p}{(}\PYG{n}{x}\PYG{p}{,} \PYG{n}{loc}\PYG{p}{,} \PYG{n}{scale}\PYG{p}{)}\PYG{p}{:}
\PYG{g+gp}{... }    \PYG{k}{return} \PYG{n}{exp}\PYG{p}{(}\PYG{p}{(}\PYG{n}{loc}\PYG{o}{\PYGZhy{}}\PYG{n}{x}\PYG{p}{)}\PYG{o}{/}\PYG{n}{scale}\PYG{p}{)}\PYG{o}{/}\PYG{p}{(}\PYG{n}{scale}\PYG{o}{*}\PYG{p}{(}\PYG{l+m+mi}{1}\PYG{o}{+}\PYG{n}{exp}\PYG{p}{(}\PYG{p}{(}\PYG{n}{loc}\PYG{o}{\PYGZhy{}}\PYG{n}{x}\PYG{p}{)}\PYG{o}{/}\PYG{n}{scale}\PYG{p}{)}\PYG{p}{)}\PYG{o}{*}\PYG{o}{*}\PYG{l+m+mi}{2}\PYG{p}{)}
\PYG{g+gp}{\PYGZgt{}\PYGZgt{}\PYGZgt{} }\PYG{n}{plt}\PYG{o}{.}\PYG{n}{plot}\PYG{p}{(}\PYG{n}{bins}\PYG{p}{,} \PYG{n}{logist}\PYG{p}{(}\PYG{n}{bins}\PYG{p}{,} \PYG{n}{loc}\PYG{p}{,} \PYG{n}{scale}\PYG{p}{)}\PYG{o}{*}\PYG{n}{count}\PYG{o}{.}\PYG{n}{max}\PYG{p}{(}\PYG{p}{)}\PYG{o}{/}\PYGZbs{}
\PYG{g+gp}{... }\PYG{n}{logist}\PYG{p}{(}\PYG{n}{bins}\PYG{p}{,} \PYG{n}{loc}\PYG{p}{,} \PYG{n}{scale}\PYG{p}{)}\PYG{o}{.}\PYG{n}{max}\PYG{p}{(}\PYG{p}{)}\PYG{p}{)}
\PYG{g+gp}{\PYGZgt{}\PYGZgt{}\PYGZgt{} }\PYG{n}{plt}\PYG{o}{.}\PYG{n}{show}\PYG{p}{(}\PYG{p}{)}
\end{Verbatim}

\end{fulllineitems}

\index{lognormal() (in module acsAttractorAnalysis)}

\begin{fulllineitems}
\phantomsection\label{acsAttractorAnalysis:acsAttractorAnalysis.lognormal}\pysiglinewithargsret{\code{acsAttractorAnalysis.}\bfcode{lognormal}}{\emph{mean=0.0}, \emph{sigma=1.0}, \emph{size=None}}{}
Return samples drawn from a log-normal distribution.

Draw samples from a log-normal distribution with specified mean,
standard deviation, and array shape.  Note that the mean and standard
deviation are not the values for the distribution itself, but of the
underlying normal distribution it is derived from.
\begin{description}
\item[{mean}] \leavevmode{[}float{]}
Mean value of the underlying normal distribution

\item[{sigma}] \leavevmode{[}float, \textgreater{} 0.{]}
Standard deviation of the underlying normal distribution

\item[{size}] \leavevmode{[}tuple of ints{]}
Output shape.  If the given shape is, e.g., \code{(m, n, k)}, then
\code{m * n * k} samples are drawn.

\end{description}
\begin{description}
\item[{samples}] \leavevmode{[}ndarray or float{]}
The desired samples. An array of the same shape as \emph{size} if given,
if \emph{size} is None a float is returned.

\end{description}
\begin{description}
\item[{scipy.stats.lognorm}] \leavevmode{[}probability density function, distribution,{]}
cumulative density function, etc.

\end{description}

A variable \emph{x} has a log-normal distribution if \emph{log(x)} is normally
distributed.  The probability density function for the log-normal
distribution is:
\begin{gather}
\begin{split}p(x) = \frac{1}{\sigma x \sqrt{2\pi}}
e^{(-\frac{(ln(x)-\mu)^2}{2\sigma^2})}\end{split}\notag
\end{gather}
where \(\mu\) is the mean and \(\sigma\) is the standard
deviation of the normally distributed logarithm of the variable.
A log-normal distribution results if a random variable is the \emph{product}
of a large number of independent, identically-distributed variables in
the same way that a normal distribution results if the variable is the
\emph{sum} of a large number of independent, identically-distributed
variables.

Limpert, E., Stahel, W. A., and Abbt, M., ``Log-normal Distributions
across the Sciences: Keys and Clues,'' \emph{BioScience}, Vol. 51, No. 5,
May, 2001.  \href{http://stat.ethz.ch/~stahel/lognormal/bioscience.pdf}{http://stat.ethz.ch/\textasciitilde{}stahel/lognormal/bioscience.pdf}

Reiss, R.D. and Thomas, M., \emph{Statistical Analysis of Extreme Values},
Basel: Birkhauser Verlag, 2001, pp. 31-32.

Draw samples from the distribution:

\begin{Verbatim}[commandchars=\\\{\}]
\PYG{g+gp}{\PYGZgt{}\PYGZgt{}\PYGZgt{} }\PYG{n}{mu}\PYG{p}{,} \PYG{n}{sigma} \PYG{o}{=} \PYG{l+m+mf}{3.}\PYG{p}{,} \PYG{l+m+mf}{1.} \PYG{c}{\PYGZsh{} mean and standard deviation}
\PYG{g+gp}{\PYGZgt{}\PYGZgt{}\PYGZgt{} }\PYG{n}{s} \PYG{o}{=} \PYG{n}{np}\PYG{o}{.}\PYG{n}{random}\PYG{o}{.}\PYG{n}{lognormal}\PYG{p}{(}\PYG{n}{mu}\PYG{p}{,} \PYG{n}{sigma}\PYG{p}{,} \PYG{l+m+mi}{1000}\PYG{p}{)}
\end{Verbatim}

Display the histogram of the samples, along with
the probability density function:

\begin{Verbatim}[commandchars=\\\{\}]
\PYG{g+gp}{\PYGZgt{}\PYGZgt{}\PYGZgt{} }\PYG{k+kn}{import} \PYG{n+nn}{matplotlib.pyplot} \PYG{k+kn}{as} \PYG{n+nn}{plt}
\PYG{g+gp}{\PYGZgt{}\PYGZgt{}\PYGZgt{} }\PYG{n}{count}\PYG{p}{,} \PYG{n}{bins}\PYG{p}{,} \PYG{n}{ignored} \PYG{o}{=} \PYG{n}{plt}\PYG{o}{.}\PYG{n}{hist}\PYG{p}{(}\PYG{n}{s}\PYG{p}{,} \PYG{l+m+mi}{100}\PYG{p}{,} \PYG{n}{normed}\PYG{o}{=}\PYG{n+nb+bp}{True}\PYG{p}{,} \PYG{n}{align}\PYG{o}{=}\PYG{l+s}{\PYGZsq{}}\PYG{l+s}{mid}\PYG{l+s}{\PYGZsq{}}\PYG{p}{)}
\end{Verbatim}

\begin{Verbatim}[commandchars=\\\{\}]
\PYG{g+gp}{\PYGZgt{}\PYGZgt{}\PYGZgt{} }\PYG{n}{x} \PYG{o}{=} \PYG{n}{np}\PYG{o}{.}\PYG{n}{linspace}\PYG{p}{(}\PYG{n+nb}{min}\PYG{p}{(}\PYG{n}{bins}\PYG{p}{)}\PYG{p}{,} \PYG{n+nb}{max}\PYG{p}{(}\PYG{n}{bins}\PYG{p}{)}\PYG{p}{,} \PYG{l+m+mi}{10000}\PYG{p}{)}
\PYG{g+gp}{\PYGZgt{}\PYGZgt{}\PYGZgt{} }\PYG{n}{pdf} \PYG{o}{=} \PYG{p}{(}\PYG{n}{np}\PYG{o}{.}\PYG{n}{exp}\PYG{p}{(}\PYG{o}{\PYGZhy{}}\PYG{p}{(}\PYG{n}{np}\PYG{o}{.}\PYG{n}{log}\PYG{p}{(}\PYG{n}{x}\PYG{p}{)} \PYG{o}{\PYGZhy{}} \PYG{n}{mu}\PYG{p}{)}\PYG{o}{*}\PYG{o}{*}\PYG{l+m+mi}{2} \PYG{o}{/} \PYG{p}{(}\PYG{l+m+mi}{2} \PYG{o}{*} \PYG{n}{sigma}\PYG{o}{*}\PYG{o}{*}\PYG{l+m+mi}{2}\PYG{p}{)}\PYG{p}{)}
\PYG{g+gp}{... }       \PYG{o}{/} \PYG{p}{(}\PYG{n}{x} \PYG{o}{*} \PYG{n}{sigma} \PYG{o}{*} \PYG{n}{np}\PYG{o}{.}\PYG{n}{sqrt}\PYG{p}{(}\PYG{l+m+mi}{2} \PYG{o}{*} \PYG{n}{np}\PYG{o}{.}\PYG{n}{pi}\PYG{p}{)}\PYG{p}{)}\PYG{p}{)}
\end{Verbatim}

\begin{Verbatim}[commandchars=\\\{\}]
\PYG{g+gp}{\PYGZgt{}\PYGZgt{}\PYGZgt{} }\PYG{n}{plt}\PYG{o}{.}\PYG{n}{plot}\PYG{p}{(}\PYG{n}{x}\PYG{p}{,} \PYG{n}{pdf}\PYG{p}{,} \PYG{n}{linewidth}\PYG{o}{=}\PYG{l+m+mi}{2}\PYG{p}{,} \PYG{n}{color}\PYG{o}{=}\PYG{l+s}{\PYGZsq{}}\PYG{l+s}{r}\PYG{l+s}{\PYGZsq{}}\PYG{p}{)}
\PYG{g+gp}{\PYGZgt{}\PYGZgt{}\PYGZgt{} }\PYG{n}{plt}\PYG{o}{.}\PYG{n}{axis}\PYG{p}{(}\PYG{l+s}{\PYGZsq{}}\PYG{l+s}{tight}\PYG{l+s}{\PYGZsq{}}\PYG{p}{)}
\PYG{g+gp}{\PYGZgt{}\PYGZgt{}\PYGZgt{} }\PYG{n}{plt}\PYG{o}{.}\PYG{n}{show}\PYG{p}{(}\PYG{p}{)}
\end{Verbatim}

Demonstrate that taking the products of random samples from a uniform
distribution can be fit well by a log-normal probability density function.

\begin{Verbatim}[commandchars=\\\{\}]
\PYG{g+gp}{\PYGZgt{}\PYGZgt{}\PYGZgt{} }\PYG{c}{\PYGZsh{} Generate a thousand samples: each is the product of 100 random}
\PYG{g+gp}{\PYGZgt{}\PYGZgt{}\PYGZgt{} }\PYG{c}{\PYGZsh{} values, drawn from a normal distribution.}
\PYG{g+gp}{\PYGZgt{}\PYGZgt{}\PYGZgt{} }\PYG{n}{b} \PYG{o}{=} \PYG{p}{[}\PYG{p}{]}
\PYG{g+gp}{\PYGZgt{}\PYGZgt{}\PYGZgt{} }\PYG{k}{for} \PYG{n}{i} \PYG{o+ow}{in} \PYG{n+nb}{range}\PYG{p}{(}\PYG{l+m+mi}{1000}\PYG{p}{)}\PYG{p}{:}
\PYG{g+gp}{... }   \PYG{n}{a} \PYG{o}{=} \PYG{l+m+mf}{10.} \PYG{o}{+} \PYG{n}{np}\PYG{o}{.}\PYG{n}{random}\PYG{o}{.}\PYG{n}{random}\PYG{p}{(}\PYG{l+m+mi}{100}\PYG{p}{)}
\PYG{g+gp}{... }   \PYG{n}{b}\PYG{o}{.}\PYG{n}{append}\PYG{p}{(}\PYG{n}{np}\PYG{o}{.}\PYG{n}{product}\PYG{p}{(}\PYG{n}{a}\PYG{p}{)}\PYG{p}{)}
\end{Verbatim}

\begin{Verbatim}[commandchars=\\\{\}]
\PYG{g+gp}{\PYGZgt{}\PYGZgt{}\PYGZgt{} }\PYG{n}{b} \PYG{o}{=} \PYG{n}{np}\PYG{o}{.}\PYG{n}{array}\PYG{p}{(}\PYG{n}{b}\PYG{p}{)} \PYG{o}{/} \PYG{n}{np}\PYG{o}{.}\PYG{n}{min}\PYG{p}{(}\PYG{n}{b}\PYG{p}{)} \PYG{c}{\PYGZsh{} scale values to be positive}
\PYG{g+gp}{\PYGZgt{}\PYGZgt{}\PYGZgt{} }\PYG{n}{count}\PYG{p}{,} \PYG{n}{bins}\PYG{p}{,} \PYG{n}{ignored} \PYG{o}{=} \PYG{n}{plt}\PYG{o}{.}\PYG{n}{hist}\PYG{p}{(}\PYG{n}{b}\PYG{p}{,} \PYG{l+m+mi}{100}\PYG{p}{,} \PYG{n}{normed}\PYG{o}{=}\PYG{n+nb+bp}{True}\PYG{p}{,} \PYG{n}{align}\PYG{o}{=}\PYG{l+s}{\PYGZsq{}}\PYG{l+s}{center}\PYG{l+s}{\PYGZsq{}}\PYG{p}{)}
\PYG{g+gp}{\PYGZgt{}\PYGZgt{}\PYGZgt{} }\PYG{n}{sigma} \PYG{o}{=} \PYG{n}{np}\PYG{o}{.}\PYG{n}{std}\PYG{p}{(}\PYG{n}{np}\PYG{o}{.}\PYG{n}{log}\PYG{p}{(}\PYG{n}{b}\PYG{p}{)}\PYG{p}{)}
\PYG{g+gp}{\PYGZgt{}\PYGZgt{}\PYGZgt{} }\PYG{n}{mu} \PYG{o}{=} \PYG{n}{np}\PYG{o}{.}\PYG{n}{mean}\PYG{p}{(}\PYG{n}{np}\PYG{o}{.}\PYG{n}{log}\PYG{p}{(}\PYG{n}{b}\PYG{p}{)}\PYG{p}{)}
\end{Verbatim}

\begin{Verbatim}[commandchars=\\\{\}]
\PYG{g+gp}{\PYGZgt{}\PYGZgt{}\PYGZgt{} }\PYG{n}{x} \PYG{o}{=} \PYG{n}{np}\PYG{o}{.}\PYG{n}{linspace}\PYG{p}{(}\PYG{n+nb}{min}\PYG{p}{(}\PYG{n}{bins}\PYG{p}{)}\PYG{p}{,} \PYG{n+nb}{max}\PYG{p}{(}\PYG{n}{bins}\PYG{p}{)}\PYG{p}{,} \PYG{l+m+mi}{10000}\PYG{p}{)}
\PYG{g+gp}{\PYGZgt{}\PYGZgt{}\PYGZgt{} }\PYG{n}{pdf} \PYG{o}{=} \PYG{p}{(}\PYG{n}{np}\PYG{o}{.}\PYG{n}{exp}\PYG{p}{(}\PYG{o}{\PYGZhy{}}\PYG{p}{(}\PYG{n}{np}\PYG{o}{.}\PYG{n}{log}\PYG{p}{(}\PYG{n}{x}\PYG{p}{)} \PYG{o}{\PYGZhy{}} \PYG{n}{mu}\PYG{p}{)}\PYG{o}{*}\PYG{o}{*}\PYG{l+m+mi}{2} \PYG{o}{/} \PYG{p}{(}\PYG{l+m+mi}{2} \PYG{o}{*} \PYG{n}{sigma}\PYG{o}{*}\PYG{o}{*}\PYG{l+m+mi}{2}\PYG{p}{)}\PYG{p}{)}
\PYG{g+gp}{... }       \PYG{o}{/} \PYG{p}{(}\PYG{n}{x} \PYG{o}{*} \PYG{n}{sigma} \PYG{o}{*} \PYG{n}{np}\PYG{o}{.}\PYG{n}{sqrt}\PYG{p}{(}\PYG{l+m+mi}{2} \PYG{o}{*} \PYG{n}{np}\PYG{o}{.}\PYG{n}{pi}\PYG{p}{)}\PYG{p}{)}\PYG{p}{)}
\end{Verbatim}

\begin{Verbatim}[commandchars=\\\{\}]
\PYG{g+gp}{\PYGZgt{}\PYGZgt{}\PYGZgt{} }\PYG{n}{plt}\PYG{o}{.}\PYG{n}{plot}\PYG{p}{(}\PYG{n}{x}\PYG{p}{,} \PYG{n}{pdf}\PYG{p}{,} \PYG{n}{color}\PYG{o}{=}\PYG{l+s}{\PYGZsq{}}\PYG{l+s}{r}\PYG{l+s}{\PYGZsq{}}\PYG{p}{,} \PYG{n}{linewidth}\PYG{o}{=}\PYG{l+m+mi}{2}\PYG{p}{)}
\PYG{g+gp}{\PYGZgt{}\PYGZgt{}\PYGZgt{} }\PYG{n}{plt}\PYG{o}{.}\PYG{n}{show}\PYG{p}{(}\PYG{p}{)}
\end{Verbatim}

\end{fulllineitems}

\index{logseries() (in module acsAttractorAnalysis)}

\begin{fulllineitems}
\phantomsection\label{acsAttractorAnalysis:acsAttractorAnalysis.logseries}\pysiglinewithargsret{\code{acsAttractorAnalysis.}\bfcode{logseries}}{\emph{p}, \emph{size=None}}{}
Draw samples from a Logarithmic Series distribution.

Samples are drawn from a Log Series distribution with specified
parameter, p (probability, 0 \textless{} p \textless{} 1).

loc : float

scale : float \textgreater{} 0.
\begin{description}
\item[{size}] \leavevmode{[}\{tuple, int\}{]}
Output shape.  If the given shape is, e.g., \code{(m, n, k)}, then
\code{m * n * k} samples are drawn.

\end{description}
\begin{description}
\item[{samples}] \leavevmode{[}\{ndarray, scalar\}{]}
where the values are all integers in  {[}0, n{]}.

\end{description}
\begin{description}
\item[{scipy.stats.distributions.logser}] \leavevmode{[}probability density function,{]}
distribution or cumulative density function, etc.

\end{description}

The probability density for the Log Series distribution is
\begin{gather}
\begin{split}P(k) = \frac{-p^k}{k \ln(1-p)},\end{split}\notag
\end{gather}
where p = probability.

The Log Series distribution is frequently used to represent species
richness and occurrence, first proposed by Fisher, Corbet, and
Williams in 1943 {[}2{]}.  It may also be used to model the numbers of
occupants seen in cars {[}3{]}.

Draw samples from the distribution:

\begin{Verbatim}[commandchars=\\\{\}]
\PYG{g+gp}{\PYGZgt{}\PYGZgt{}\PYGZgt{} }\PYG{n}{a} \PYG{o}{=} \PYG{o}{.}\PYG{l+m+mi}{6}
\PYG{g+gp}{\PYGZgt{}\PYGZgt{}\PYGZgt{} }\PYG{n}{s} \PYG{o}{=} \PYG{n}{np}\PYG{o}{.}\PYG{n}{random}\PYG{o}{.}\PYG{n}{logseries}\PYG{p}{(}\PYG{n}{a}\PYG{p}{,} \PYG{l+m+mi}{10000}\PYG{p}{)}
\PYG{g+gp}{\PYGZgt{}\PYGZgt{}\PYGZgt{} }\PYG{n}{count}\PYG{p}{,} \PYG{n}{bins}\PYG{p}{,} \PYG{n}{ignored} \PYG{o}{=} \PYG{n}{plt}\PYG{o}{.}\PYG{n}{hist}\PYG{p}{(}\PYG{n}{s}\PYG{p}{)}
\end{Verbatim}

\#   plot against distribution

\begin{Verbatim}[commandchars=\\\{\}]
\PYG{g+gp}{\PYGZgt{}\PYGZgt{}\PYGZgt{} }\PYG{k}{def} \PYG{n+nf}{logseries}\PYG{p}{(}\PYG{n}{k}\PYG{p}{,} \PYG{n}{p}\PYG{p}{)}\PYG{p}{:}
\PYG{g+gp}{... }    \PYG{k}{return} \PYG{o}{\PYGZhy{}}\PYG{n}{p}\PYG{o}{*}\PYG{o}{*}\PYG{n}{k}\PYG{o}{/}\PYG{p}{(}\PYG{n}{k}\PYG{o}{*}\PYG{n}{log}\PYG{p}{(}\PYG{l+m+mi}{1}\PYG{o}{\PYGZhy{}}\PYG{n}{p}\PYG{p}{)}\PYG{p}{)}
\PYG{g+gp}{\PYGZgt{}\PYGZgt{}\PYGZgt{} }\PYG{n}{plt}\PYG{o}{.}\PYG{n}{plot}\PYG{p}{(}\PYG{n}{bins}\PYG{p}{,} \PYG{n}{logseries}\PYG{p}{(}\PYG{n}{bins}\PYG{p}{,} \PYG{n}{a}\PYG{p}{)}\PYG{o}{*}\PYG{n}{count}\PYG{o}{.}\PYG{n}{max}\PYG{p}{(}\PYG{p}{)}\PYG{o}{/}
\PYG{g+go}{             logseries(bins, a).max(), \PYGZsq{}r\PYGZsq{})}
\PYG{g+gp}{\PYGZgt{}\PYGZgt{}\PYGZgt{} }\PYG{n}{plt}\PYG{o}{.}\PYG{n}{show}\PYG{p}{(}\PYG{p}{)}
\end{Verbatim}

\end{fulllineitems}

\index{multinomial() (in module acsAttractorAnalysis)}

\begin{fulllineitems}
\phantomsection\label{acsAttractorAnalysis:acsAttractorAnalysis.multinomial}\pysiglinewithargsret{\code{acsAttractorAnalysis.}\bfcode{multinomial}}{\emph{n}, \emph{pvals}, \emph{size=None}}{}
Draw samples from a multinomial distribution.

The multinomial distribution is a multivariate generalisation of the
binomial distribution.  Take an experiment with one of \code{p}
possible outcomes.  An example of such an experiment is throwing a dice,
where the outcome can be 1 through 6.  Each sample drawn from the
distribution represents \emph{n} such experiments.  Its values,
\code{X\_i = {[}X\_0, X\_1, ..., X\_p{]}}, represent the number of times the outcome
was \code{i}.
\begin{description}
\item[{n}] \leavevmode{[}int{]}
Number of experiments.

\item[{pvals}] \leavevmode{[}sequence of floats, length p{]}
Probabilities of each of the \code{p} different outcomes.  These
should sum to 1 (however, the last element is always assumed to
account for the remaining probability, as long as
\code{sum(pvals{[}:-1{]}) \textless{}= 1)}.

\item[{size}] \leavevmode{[}tuple of ints{]}
Given a \emph{size} of \code{(M, N, K)}, then \code{M*N*K} samples are drawn,
and the output shape becomes \code{(M, N, K, p)}, since each sample
has shape \code{(p,)}.

\end{description}

Throw a dice 20 times:

\begin{Verbatim}[commandchars=\\\{\}]
\PYG{g+gp}{\PYGZgt{}\PYGZgt{}\PYGZgt{} }\PYG{n}{np}\PYG{o}{.}\PYG{n}{random}\PYG{o}{.}\PYG{n}{multinomial}\PYG{p}{(}\PYG{l+m+mi}{20}\PYG{p}{,} \PYG{p}{[}\PYG{l+m+mi}{1}\PYG{o}{/}\PYG{l+m+mf}{6.}\PYG{p}{]}\PYG{o}{*}\PYG{l+m+mi}{6}\PYG{p}{,} \PYG{n}{size}\PYG{o}{=}\PYG{l+m+mi}{1}\PYG{p}{)}
\PYG{g+go}{array([[4, 1, 7, 5, 2, 1]])}
\end{Verbatim}

It landed 4 times on 1, once on 2, etc.

Now, throw the dice 20 times, and 20 times again:

\begin{Verbatim}[commandchars=\\\{\}]
\PYG{g+gp}{\PYGZgt{}\PYGZgt{}\PYGZgt{} }\PYG{n}{np}\PYG{o}{.}\PYG{n}{random}\PYG{o}{.}\PYG{n}{multinomial}\PYG{p}{(}\PYG{l+m+mi}{20}\PYG{p}{,} \PYG{p}{[}\PYG{l+m+mi}{1}\PYG{o}{/}\PYG{l+m+mf}{6.}\PYG{p}{]}\PYG{o}{*}\PYG{l+m+mi}{6}\PYG{p}{,} \PYG{n}{size}\PYG{o}{=}\PYG{l+m+mi}{2}\PYG{p}{)}
\PYG{g+go}{array([[3, 4, 3, 3, 4, 3],}
\PYG{g+go}{       [2, 4, 3, 4, 0, 7]])}
\end{Verbatim}

For the first run, we threw 3 times 1, 4 times 2, etc.  For the second,
we threw 2 times 1, 4 times 2, etc.

A loaded dice is more likely to land on number 6:

\begin{Verbatim}[commandchars=\\\{\}]
\PYG{g+gp}{\PYGZgt{}\PYGZgt{}\PYGZgt{} }\PYG{n}{np}\PYG{o}{.}\PYG{n}{random}\PYG{o}{.}\PYG{n}{multinomial}\PYG{p}{(}\PYG{l+m+mi}{100}\PYG{p}{,} \PYG{p}{[}\PYG{l+m+mi}{1}\PYG{o}{/}\PYG{l+m+mf}{7.}\PYG{p}{]}\PYG{o}{*}\PYG{l+m+mi}{5}\PYG{p}{)}
\PYG{g+go}{array([13, 16, 13, 16, 42])}
\end{Verbatim}

\end{fulllineitems}

\index{multivariate\_normal() (in module acsAttractorAnalysis)}

\begin{fulllineitems}
\phantomsection\label{acsAttractorAnalysis:acsAttractorAnalysis.multivariate_normal}\pysiglinewithargsret{\code{acsAttractorAnalysis.}\bfcode{multivariate\_normal}}{\emph{mean}, \emph{cov}\optional{, \emph{size}}}{}
Draw random samples from a multivariate normal distribution.

The multivariate normal, multinormal or Gaussian distribution is a
generalization of the one-dimensional normal distribution to higher
dimensions.  Such a distribution is specified by its mean and
covariance matrix.  These parameters are analogous to the mean
(average or ``center'') and variance (standard deviation, or ``width,''
squared) of the one-dimensional normal distribution.
\begin{description}
\item[{mean}] \leavevmode{[}1-D array\_like, of length N{]}
Mean of the N-dimensional distribution.

\item[{cov}] \leavevmode{[}2-D array\_like, of shape (N, N){]}
Covariance matrix of the distribution.  Must be symmetric and
positive semi-definite for ``physically meaningful'' results.

\item[{size}] \leavevmode{[}int or tuple of ints, optional{]}
Given a shape of, for example, \code{(m,n,k)}, \code{m*n*k} samples are
generated, and packed in an \emph{m}-by-\emph{n}-by-\emph{k} arrangement.  Because
each sample is \emph{N}-dimensional, the output shape is \code{(m,n,k,N)}.
If no shape is specified, a single (\emph{N}-D) sample is returned.

\end{description}
\begin{description}
\item[{out}] \leavevmode{[}ndarray{]}
The drawn samples, of shape \emph{size}, if that was provided.  If not,
the shape is \code{(N,)}.

In other words, each entry \code{out{[}i,j,...,:{]}} is an N-dimensional
value drawn from the distribution.

\end{description}

The mean is a coordinate in N-dimensional space, which represents the
location where samples are most likely to be generated.  This is
analogous to the peak of the bell curve for the one-dimensional or
univariate normal distribution.

Covariance indicates the level to which two variables vary together.
From the multivariate normal distribution, we draw N-dimensional
samples, \(X = [x_1, x_2, ... x_N]\).  The covariance matrix
element \(C_{ij}\) is the covariance of \(x_i\) and \(x_j\).
The element \(C_{ii}\) is the variance of \(x_i\) (i.e. its
``spread'').

Instead of specifying the full covariance matrix, popular
approximations include:
\begin{itemize}
\item {} 
Spherical covariance (\emph{cov} is a multiple of the identity matrix)

\item {} 
Diagonal covariance (\emph{cov} has non-negative elements, and only on
the diagonal)

\end{itemize}

This geometrical property can be seen in two dimensions by plotting
generated data-points:

\begin{Verbatim}[commandchars=\\\{\}]
\PYG{g+gp}{\PYGZgt{}\PYGZgt{}\PYGZgt{} }\PYG{n}{mean} \PYG{o}{=} \PYG{p}{[}\PYG{l+m+mi}{0}\PYG{p}{,}\PYG{l+m+mi}{0}\PYG{p}{]}
\PYG{g+gp}{\PYGZgt{}\PYGZgt{}\PYGZgt{} }\PYG{n}{cov} \PYG{o}{=} \PYG{p}{[}\PYG{p}{[}\PYG{l+m+mi}{1}\PYG{p}{,}\PYG{l+m+mi}{0}\PYG{p}{]}\PYG{p}{,}\PYG{p}{[}\PYG{l+m+mi}{0}\PYG{p}{,}\PYG{l+m+mi}{100}\PYG{p}{]}\PYG{p}{]} \PYG{c}{\PYGZsh{} diagonal covariance, points lie on x or y\PYGZhy{}axis}
\end{Verbatim}

\begin{Verbatim}[commandchars=\\\{\}]
\PYG{g+gp}{\PYGZgt{}\PYGZgt{}\PYGZgt{} }\PYG{k+kn}{import} \PYG{n+nn}{matplotlib.pyplot} \PYG{k+kn}{as} \PYG{n+nn}{plt}
\PYG{g+gp}{\PYGZgt{}\PYGZgt{}\PYGZgt{} }\PYG{n}{x}\PYG{p}{,}\PYG{n}{y} \PYG{o}{=} \PYG{n}{np}\PYG{o}{.}\PYG{n}{random}\PYG{o}{.}\PYG{n}{multivariate\PYGZus{}normal}\PYG{p}{(}\PYG{n}{mean}\PYG{p}{,}\PYG{n}{cov}\PYG{p}{,}\PYG{l+m+mi}{5000}\PYG{p}{)}\PYG{o}{.}\PYG{n}{T}
\PYG{g+gp}{\PYGZgt{}\PYGZgt{}\PYGZgt{} }\PYG{n}{plt}\PYG{o}{.}\PYG{n}{plot}\PYG{p}{(}\PYG{n}{x}\PYG{p}{,}\PYG{n}{y}\PYG{p}{,}\PYG{l+s}{\PYGZsq{}}\PYG{l+s}{x}\PYG{l+s}{\PYGZsq{}}\PYG{p}{)}\PYG{p}{;} \PYG{n}{plt}\PYG{o}{.}\PYG{n}{axis}\PYG{p}{(}\PYG{l+s}{\PYGZsq{}}\PYG{l+s}{equal}\PYG{l+s}{\PYGZsq{}}\PYG{p}{)}\PYG{p}{;} \PYG{n}{plt}\PYG{o}{.}\PYG{n}{show}\PYG{p}{(}\PYG{p}{)}
\end{Verbatim}

Note that the covariance matrix must be non-negative definite.

Papoulis, A., \emph{Probability, Random Variables, and Stochastic Processes},
3rd ed., New York: McGraw-Hill, 1991.

Duda, R. O., Hart, P. E., and Stork, D. G., \emph{Pattern Classification},
2nd ed., New York: Wiley, 2001.

\begin{Verbatim}[commandchars=\\\{\}]
\PYG{g+gp}{\PYGZgt{}\PYGZgt{}\PYGZgt{} }\PYG{n}{mean} \PYG{o}{=} \PYG{p}{(}\PYG{l+m+mi}{1}\PYG{p}{,}\PYG{l+m+mi}{2}\PYG{p}{)}
\PYG{g+gp}{\PYGZgt{}\PYGZgt{}\PYGZgt{} }\PYG{n}{cov} \PYG{o}{=} \PYG{p}{[}\PYG{p}{[}\PYG{l+m+mi}{1}\PYG{p}{,}\PYG{l+m+mi}{0}\PYG{p}{]}\PYG{p}{,}\PYG{p}{[}\PYG{l+m+mi}{1}\PYG{p}{,}\PYG{l+m+mi}{0}\PYG{p}{]}\PYG{p}{]}
\PYG{g+gp}{\PYGZgt{}\PYGZgt{}\PYGZgt{} }\PYG{n}{x} \PYG{o}{=} \PYG{n}{np}\PYG{o}{.}\PYG{n}{random}\PYG{o}{.}\PYG{n}{multivariate\PYGZus{}normal}\PYG{p}{(}\PYG{n}{mean}\PYG{p}{,}\PYG{n}{cov}\PYG{p}{,}\PYG{p}{(}\PYG{l+m+mi}{3}\PYG{p}{,}\PYG{l+m+mi}{3}\PYG{p}{)}\PYG{p}{)}
\PYG{g+gp}{\PYGZgt{}\PYGZgt{}\PYGZgt{} }\PYG{n}{x}\PYG{o}{.}\PYG{n}{shape}
\PYG{g+go}{(3, 3, 2)}
\end{Verbatim}

The following is probably true, given that 0.6 is roughly twice the
standard deviation:

\begin{Verbatim}[commandchars=\\\{\}]
\PYG{g+gp}{\PYGZgt{}\PYGZgt{}\PYGZgt{} }\PYG{k}{print} \PYG{n+nb}{list}\PYG{p}{(} \PYG{p}{(}\PYG{n}{x}\PYG{p}{[}\PYG{l+m+mi}{0}\PYG{p}{,}\PYG{l+m+mi}{0}\PYG{p}{,}\PYG{p}{:}\PYG{p}{]} \PYG{o}{\PYGZhy{}} \PYG{n}{mean}\PYG{p}{)} \PYG{o}{\PYGZlt{}} \PYG{l+m+mf}{0.6} \PYG{p}{)}
\PYG{g+go}{[True, True]}
\end{Verbatim}

\end{fulllineitems}

\index{negative\_binomial() (in module acsAttractorAnalysis)}

\begin{fulllineitems}
\phantomsection\label{acsAttractorAnalysis:acsAttractorAnalysis.negative_binomial}\pysiglinewithargsret{\code{acsAttractorAnalysis.}\bfcode{negative\_binomial}}{\emph{n}, \emph{p}, \emph{size=None}}{}
Draw samples from a negative\_binomial distribution.

Samples are drawn from a negative\_Binomial distribution with specified
parameters, \emph{n} trials and \emph{p} probability of success where \emph{n} is an
integer \textgreater{} 0 and \emph{p} is in the interval {[}0, 1{]}.
\begin{description}
\item[{n}] \leavevmode{[}int{]}
Parameter, \textgreater{} 0.

\item[{p}] \leavevmode{[}float{]}
Parameter, \textgreater{}= 0 and \textless{}=1.

\item[{size}] \leavevmode{[}int or tuple of ints{]}
Output shape. If the given shape is, e.g., \code{(m, n, k)}, then
\code{m * n * k} samples are drawn.

\end{description}
\begin{description}
\item[{samples}] \leavevmode{[}int or ndarray of ints{]}
Drawn samples.

\end{description}

The probability density for the Negative Binomial distribution is
\begin{gather}
\begin{split}P(N;n,p) = \binom{N+n-1}{n-1}p^{n}(1-p)^{N},\end{split}\notag
\end{gather}
where \(n-1\) is the number of successes, \(p\) is the probability
of success, and \(N+n-1\) is the number of trials.

The negative binomial distribution gives the probability of n-1 successes
and N failures in N+n-1 trials, and success on the (N+n)th trial.

If one throws a die repeatedly until the third time a ``1'' appears, then the
probability distribution of the number of non-``1''s that appear before the
third ``1'' is a negative binomial distribution.

Draw samples from the distribution:

A real world example. A company drills wild-cat oil exploration wells, each
with an estimated probability of success of 0.1.  What is the probability
of having one success for each successive well, that is what is the
probability of a single success after drilling 5 wells, after 6 wells,
etc.?

\begin{Verbatim}[commandchars=\\\{\}]
\PYG{g+gp}{\PYGZgt{}\PYGZgt{}\PYGZgt{} }\PYG{n}{s} \PYG{o}{=} \PYG{n}{np}\PYG{o}{.}\PYG{n}{random}\PYG{o}{.}\PYG{n}{negative\PYGZus{}binomial}\PYG{p}{(}\PYG{l+m+mi}{1}\PYG{p}{,} \PYG{l+m+mf}{0.1}\PYG{p}{,} \PYG{l+m+mi}{100000}\PYG{p}{)}
\PYG{g+gp}{\PYGZgt{}\PYGZgt{}\PYGZgt{} }\PYG{k}{for} \PYG{n}{i} \PYG{o+ow}{in} \PYG{n+nb}{range}\PYG{p}{(}\PYG{l+m+mi}{1}\PYG{p}{,} \PYG{l+m+mi}{11}\PYG{p}{)}\PYG{p}{:}
\PYG{g+gp}{... }   \PYG{n}{probability} \PYG{o}{=} \PYG{n+nb}{sum}\PYG{p}{(}\PYG{n}{s}\PYG{o}{\PYGZlt{}}\PYG{n}{i}\PYG{p}{)} \PYG{o}{/} \PYG{l+m+mf}{100000.}
\PYG{g+gp}{... }   \PYG{k}{print} \PYG{n}{i}\PYG{p}{,} \PYG{l+s}{\PYGZdq{}}\PYG{l+s}{wells drilled, probability of one success =}\PYG{l+s}{\PYGZdq{}}\PYG{p}{,} \PYG{n}{probability}
\end{Verbatim}

\end{fulllineitems}

\index{noncentral\_chisquare() (in module acsAttractorAnalysis)}

\begin{fulllineitems}
\phantomsection\label{acsAttractorAnalysis:acsAttractorAnalysis.noncentral_chisquare}\pysiglinewithargsret{\code{acsAttractorAnalysis.}\bfcode{noncentral\_chisquare}}{\emph{df}, \emph{nonc}, \emph{size=None}}{}
Draw samples from a noncentral chi-square distribution.

The noncentral \(\chi^2\) distribution is a generalisation of
the \(\chi^2\) distribution.
\begin{description}
\item[{df}] \leavevmode{[}int{]}
Degrees of freedom, should be \textgreater{}= 1.

\item[{nonc}] \leavevmode{[}float{]}
Non-centrality, should be \textgreater{} 0.

\item[{size}] \leavevmode{[}int or tuple of ints{]}
Shape of the output.

\end{description}

The probability density function for the noncentral Chi-square distribution
is
\begin{gather}
\begin{split}P(x;df,nonc) = \sum^{\infty}_{i=0}
\frac{e^{-nonc/2}(nonc/2)^{i}}{i!}P_{Y_{df+2i}}(x),\end{split}\notag
\end{gather}
where \(Y_{q}\) is the Chi-square with q degrees of freedom.

In Delhi (2007), it is noted that the noncentral chi-square is useful in
bombing and coverage problems, the probability of killing the point target
given by the noncentral chi-squared distribution.

Draw values from the distribution and plot the histogram

\begin{Verbatim}[commandchars=\\\{\}]
\PYG{g+gp}{\PYGZgt{}\PYGZgt{}\PYGZgt{} }\PYG{k+kn}{import} \PYG{n+nn}{matplotlib.pyplot} \PYG{k+kn}{as} \PYG{n+nn}{plt}
\PYG{g+gp}{\PYGZgt{}\PYGZgt{}\PYGZgt{} }\PYG{n}{values} \PYG{o}{=} \PYG{n}{plt}\PYG{o}{.}\PYG{n}{hist}\PYG{p}{(}\PYG{n}{np}\PYG{o}{.}\PYG{n}{random}\PYG{o}{.}\PYG{n}{noncentral\PYGZus{}chisquare}\PYG{p}{(}\PYG{l+m+mi}{3}\PYG{p}{,} \PYG{l+m+mi}{20}\PYG{p}{,} \PYG{l+m+mi}{100000}\PYG{p}{)}\PYG{p}{,}
\PYG{g+gp}{... }                  \PYG{n}{bins}\PYG{o}{=}\PYG{l+m+mi}{200}\PYG{p}{,} \PYG{n}{normed}\PYG{o}{=}\PYG{n+nb+bp}{True}\PYG{p}{)}
\PYG{g+gp}{\PYGZgt{}\PYGZgt{}\PYGZgt{} }\PYG{n}{plt}\PYG{o}{.}\PYG{n}{show}\PYG{p}{(}\PYG{p}{)}
\end{Verbatim}

Draw values from a noncentral chisquare with very small noncentrality,
and compare to a chisquare.

\begin{Verbatim}[commandchars=\\\{\}]
\PYG{g+gp}{\PYGZgt{}\PYGZgt{}\PYGZgt{} }\PYG{n}{plt}\PYG{o}{.}\PYG{n}{figure}\PYG{p}{(}\PYG{p}{)}
\PYG{g+gp}{\PYGZgt{}\PYGZgt{}\PYGZgt{} }\PYG{n}{values} \PYG{o}{=} \PYG{n}{plt}\PYG{o}{.}\PYG{n}{hist}\PYG{p}{(}\PYG{n}{np}\PYG{o}{.}\PYG{n}{random}\PYG{o}{.}\PYG{n}{noncentral\PYGZus{}chisquare}\PYG{p}{(}\PYG{l+m+mi}{3}\PYG{p}{,} \PYG{o}{.}\PYG{l+m+mo}{0000001}\PYG{p}{,} \PYG{l+m+mi}{100000}\PYG{p}{)}\PYG{p}{,}
\PYG{g+gp}{... }                  \PYG{n}{bins}\PYG{o}{=}\PYG{n}{np}\PYG{o}{.}\PYG{n}{arange}\PYG{p}{(}\PYG{l+m+mf}{0.}\PYG{p}{,} \PYG{l+m+mi}{25}\PYG{p}{,} \PYG{o}{.}\PYG{l+m+mi}{1}\PYG{p}{)}\PYG{p}{,} \PYG{n}{normed}\PYG{o}{=}\PYG{n+nb+bp}{True}\PYG{p}{)}
\PYG{g+gp}{\PYGZgt{}\PYGZgt{}\PYGZgt{} }\PYG{n}{values2} \PYG{o}{=} \PYG{n}{plt}\PYG{o}{.}\PYG{n}{hist}\PYG{p}{(}\PYG{n}{np}\PYG{o}{.}\PYG{n}{random}\PYG{o}{.}\PYG{n}{chisquare}\PYG{p}{(}\PYG{l+m+mi}{3}\PYG{p}{,} \PYG{l+m+mi}{100000}\PYG{p}{)}\PYG{p}{,}
\PYG{g+gp}{... }                   \PYG{n}{bins}\PYG{o}{=}\PYG{n}{np}\PYG{o}{.}\PYG{n}{arange}\PYG{p}{(}\PYG{l+m+mf}{0.}\PYG{p}{,} \PYG{l+m+mi}{25}\PYG{p}{,} \PYG{o}{.}\PYG{l+m+mi}{1}\PYG{p}{)}\PYG{p}{,} \PYG{n}{normed}\PYG{o}{=}\PYG{n+nb+bp}{True}\PYG{p}{)}
\PYG{g+gp}{\PYGZgt{}\PYGZgt{}\PYGZgt{} }\PYG{n}{plt}\PYG{o}{.}\PYG{n}{plot}\PYG{p}{(}\PYG{n}{values}\PYG{p}{[}\PYG{l+m+mi}{1}\PYG{p}{]}\PYG{p}{[}\PYG{l+m+mi}{0}\PYG{p}{:}\PYG{o}{\PYGZhy{}}\PYG{l+m+mi}{1}\PYG{p}{]}\PYG{p}{,} \PYG{n}{values}\PYG{p}{[}\PYG{l+m+mi}{0}\PYG{p}{]}\PYG{o}{\PYGZhy{}}\PYG{n}{values2}\PYG{p}{[}\PYG{l+m+mi}{0}\PYG{p}{]}\PYG{p}{,} \PYG{l+s}{\PYGZsq{}}\PYG{l+s}{ob}\PYG{l+s}{\PYGZsq{}}\PYG{p}{)}
\PYG{g+gp}{\PYGZgt{}\PYGZgt{}\PYGZgt{} }\PYG{n}{plt}\PYG{o}{.}\PYG{n}{show}\PYG{p}{(}\PYG{p}{)}
\end{Verbatim}

Demonstrate how large values of non-centrality lead to a more symmetric
distribution.

\begin{Verbatim}[commandchars=\\\{\}]
\PYG{g+gp}{\PYGZgt{}\PYGZgt{}\PYGZgt{} }\PYG{n}{plt}\PYG{o}{.}\PYG{n}{figure}\PYG{p}{(}\PYG{p}{)}
\PYG{g+gp}{\PYGZgt{}\PYGZgt{}\PYGZgt{} }\PYG{n}{values} \PYG{o}{=} \PYG{n}{plt}\PYG{o}{.}\PYG{n}{hist}\PYG{p}{(}\PYG{n}{np}\PYG{o}{.}\PYG{n}{random}\PYG{o}{.}\PYG{n}{noncentral\PYGZus{}chisquare}\PYG{p}{(}\PYG{l+m+mi}{3}\PYG{p}{,} \PYG{l+m+mi}{20}\PYG{p}{,} \PYG{l+m+mi}{100000}\PYG{p}{)}\PYG{p}{,}
\PYG{g+gp}{... }                  \PYG{n}{bins}\PYG{o}{=}\PYG{l+m+mi}{200}\PYG{p}{,} \PYG{n}{normed}\PYG{o}{=}\PYG{n+nb+bp}{True}\PYG{p}{)}
\PYG{g+gp}{\PYGZgt{}\PYGZgt{}\PYGZgt{} }\PYG{n}{plt}\PYG{o}{.}\PYG{n}{show}\PYG{p}{(}\PYG{p}{)}
\end{Verbatim}

\end{fulllineitems}

\index{noncentral\_f() (in module acsAttractorAnalysis)}

\begin{fulllineitems}
\phantomsection\label{acsAttractorAnalysis:acsAttractorAnalysis.noncentral_f}\pysiglinewithargsret{\code{acsAttractorAnalysis.}\bfcode{noncentral\_f}}{\emph{dfnum}, \emph{dfden}, \emph{nonc}, \emph{size=None}}{}
Draw samples from the noncentral F distribution.

Samples are drawn from an F distribution with specified parameters,
\emph{dfnum} (degrees of freedom in numerator) and \emph{dfden} (degrees of
freedom in denominator), where both parameters \textgreater{} 1.
\emph{nonc} is the non-centrality parameter.
\begin{description}
\item[{dfnum}] \leavevmode{[}int{]}
Parameter, should be \textgreater{} 1.

\item[{dfden}] \leavevmode{[}int{]}
Parameter, should be \textgreater{} 1.

\item[{nonc}] \leavevmode{[}float{]}
Parameter, should be \textgreater{}= 0.

\item[{size}] \leavevmode{[}int or tuple of ints{]}
Output shape. If the given shape is, e.g., \code{(m, n, k)}, then
\code{m * n * k} samples are drawn.

\end{description}
\begin{description}
\item[{samples}] \leavevmode{[}scalar or ndarray{]}
Drawn samples.

\end{description}

When calculating the power of an experiment (power = probability of
rejecting the null hypothesis when a specific alternative is true) the
non-central F statistic becomes important.  When the null hypothesis is
true, the F statistic follows a central F distribution. When the null
hypothesis is not true, then it follows a non-central F statistic.

Weisstein, Eric W. ``Noncentral F-Distribution.'' From MathWorld--A Wolfram
Web Resource.  \href{http://mathworld.wolfram.com/NoncentralF-Distribution.html}{http://mathworld.wolfram.com/NoncentralF-Distribution.html}

Wikipedia, ``Noncentral F distribution'',
\href{http://en.wikipedia.org/wiki/Noncentral\_F-distribution}{http://en.wikipedia.org/wiki/Noncentral\_F-distribution}

In a study, testing for a specific alternative to the null hypothesis
requires use of the Noncentral F distribution. We need to calculate the
area in the tail of the distribution that exceeds the value of the F
distribution for the null hypothesis.  We'll plot the two probability
distributions for comparison.

\begin{Verbatim}[commandchars=\\\{\}]
\PYG{g+gp}{\PYGZgt{}\PYGZgt{}\PYGZgt{} }\PYG{n}{dfnum} \PYG{o}{=} \PYG{l+m+mi}{3} \PYG{c}{\PYGZsh{} between group deg of freedom}
\PYG{g+gp}{\PYGZgt{}\PYGZgt{}\PYGZgt{} }\PYG{n}{dfden} \PYG{o}{=} \PYG{l+m+mi}{20} \PYG{c}{\PYGZsh{} within groups degrees of freedom}
\PYG{g+gp}{\PYGZgt{}\PYGZgt{}\PYGZgt{} }\PYG{n}{nonc} \PYG{o}{=} \PYG{l+m+mf}{3.0}
\PYG{g+gp}{\PYGZgt{}\PYGZgt{}\PYGZgt{} }\PYG{n}{nc\PYGZus{}vals} \PYG{o}{=} \PYG{n}{np}\PYG{o}{.}\PYG{n}{random}\PYG{o}{.}\PYG{n}{noncentral\PYGZus{}f}\PYG{p}{(}\PYG{n}{dfnum}\PYG{p}{,} \PYG{n}{dfden}\PYG{p}{,} \PYG{n}{nonc}\PYG{p}{,} \PYG{l+m+mi}{1000000}\PYG{p}{)}
\PYG{g+gp}{\PYGZgt{}\PYGZgt{}\PYGZgt{} }\PYG{n}{NF} \PYG{o}{=} \PYG{n}{np}\PYG{o}{.}\PYG{n}{histogram}\PYG{p}{(}\PYG{n}{nc\PYGZus{}vals}\PYG{p}{,} \PYG{n}{bins}\PYG{o}{=}\PYG{l+m+mi}{50}\PYG{p}{,} \PYG{n}{normed}\PYG{o}{=}\PYG{n+nb+bp}{True}\PYG{p}{)}
\PYG{g+gp}{\PYGZgt{}\PYGZgt{}\PYGZgt{} }\PYG{n}{c\PYGZus{}vals} \PYG{o}{=} \PYG{n}{np}\PYG{o}{.}\PYG{n}{random}\PYG{o}{.}\PYG{n}{f}\PYG{p}{(}\PYG{n}{dfnum}\PYG{p}{,} \PYG{n}{dfden}\PYG{p}{,} \PYG{l+m+mi}{1000000}\PYG{p}{)}
\PYG{g+gp}{\PYGZgt{}\PYGZgt{}\PYGZgt{} }\PYG{n}{F} \PYG{o}{=} \PYG{n}{np}\PYG{o}{.}\PYG{n}{histogram}\PYG{p}{(}\PYG{n}{c\PYGZus{}vals}\PYG{p}{,} \PYG{n}{bins}\PYG{o}{=}\PYG{l+m+mi}{50}\PYG{p}{,} \PYG{n}{normed}\PYG{o}{=}\PYG{n+nb+bp}{True}\PYG{p}{)}
\PYG{g+gp}{\PYGZgt{}\PYGZgt{}\PYGZgt{} }\PYG{n}{plt}\PYG{o}{.}\PYG{n}{plot}\PYG{p}{(}\PYG{n}{F}\PYG{p}{[}\PYG{l+m+mi}{1}\PYG{p}{]}\PYG{p}{[}\PYG{l+m+mi}{1}\PYG{p}{:}\PYG{p}{]}\PYG{p}{,} \PYG{n}{F}\PYG{p}{[}\PYG{l+m+mi}{0}\PYG{p}{]}\PYG{p}{)}
\PYG{g+gp}{\PYGZgt{}\PYGZgt{}\PYGZgt{} }\PYG{n}{plt}\PYG{o}{.}\PYG{n}{plot}\PYG{p}{(}\PYG{n}{NF}\PYG{p}{[}\PYG{l+m+mi}{1}\PYG{p}{]}\PYG{p}{[}\PYG{l+m+mi}{1}\PYG{p}{:}\PYG{p}{]}\PYG{p}{,} \PYG{n}{NF}\PYG{p}{[}\PYG{l+m+mi}{0}\PYG{p}{]}\PYG{p}{)}
\PYG{g+gp}{\PYGZgt{}\PYGZgt{}\PYGZgt{} }\PYG{n}{plt}\PYG{o}{.}\PYG{n}{show}\PYG{p}{(}\PYG{p}{)}
\end{Verbatim}

\end{fulllineitems}

\index{normal() (in module acsAttractorAnalysis)}

\begin{fulllineitems}
\phantomsection\label{acsAttractorAnalysis:acsAttractorAnalysis.normal}\pysiglinewithargsret{\code{acsAttractorAnalysis.}\bfcode{normal}}{\emph{loc=0.0}, \emph{scale=1.0}, \emph{size=None}}{}
Draw random samples from a normal (Gaussian) distribution.

The probability density function of the normal distribution, first
derived by De Moivre and 200 years later by both Gauss and Laplace
independently {\color{red}\bfseries{}{[}2{]}\_}, is often called the bell curve because of
its characteristic shape (see the example below).

The normal distributions occurs often in nature.  For example, it
describes the commonly occurring distribution of samples influenced
by a large number of tiny, random disturbances, each with its own
unique distribution {\color{red}\bfseries{}{[}2{]}\_}.
\begin{description}
\item[{loc}] \leavevmode{[}float{]}
Mean (``centre'') of the distribution.

\item[{scale}] \leavevmode{[}float{]}
Standard deviation (spread or ``width'') of the distribution.

\item[{size}] \leavevmode{[}tuple of ints{]}
Output shape.  If the given shape is, e.g., \code{(m, n, k)}, then
\code{m * n * k} samples are drawn.

\end{description}
\begin{description}
\item[{scipy.stats.distributions.norm}] \leavevmode{[}probability density function,{]}
distribution or cumulative density function, etc.

\end{description}

The probability density for the Gaussian distribution is
\begin{gather}
\begin{split}p(x) = \frac{1}{\sqrt{ 2 \pi \sigma^2 }}
e^{ - \frac{ (x - \mu)^2 } {2 \sigma^2} },\end{split}\notag
\end{gather}
where \(\mu\) is the mean and \(\sigma\) the standard deviation.
The square of the standard deviation, \(\sigma^2\), is called the
variance.

The function has its peak at the mean, and its ``spread'' increases with
the standard deviation (the function reaches 0.607 times its maximum at
\(x + \sigma\) and \(x - \sigma\) {\color{red}\bfseries{}{[}2{]}\_}).  This implies that
\emph{numpy.random.normal} is more likely to return samples lying close to the
mean, rather than those far away.

Draw samples from the distribution:

\begin{Verbatim}[commandchars=\\\{\}]
\PYG{g+gp}{\PYGZgt{}\PYGZgt{}\PYGZgt{} }\PYG{n}{mu}\PYG{p}{,} \PYG{n}{sigma} \PYG{o}{=} \PYG{l+m+mi}{0}\PYG{p}{,} \PYG{l+m+mf}{0.1} \PYG{c}{\PYGZsh{} mean and standard deviation}
\PYG{g+gp}{\PYGZgt{}\PYGZgt{}\PYGZgt{} }\PYG{n}{s} \PYG{o}{=} \PYG{n}{np}\PYG{o}{.}\PYG{n}{random}\PYG{o}{.}\PYG{n}{normal}\PYG{p}{(}\PYG{n}{mu}\PYG{p}{,} \PYG{n}{sigma}\PYG{p}{,} \PYG{l+m+mi}{1000}\PYG{p}{)}
\end{Verbatim}

Verify the mean and the variance:

\begin{Verbatim}[commandchars=\\\{\}]
\PYG{g+gp}{\PYGZgt{}\PYGZgt{}\PYGZgt{} }\PYG{n+nb}{abs}\PYG{p}{(}\PYG{n}{mu} \PYG{o}{\PYGZhy{}} \PYG{n}{np}\PYG{o}{.}\PYG{n}{mean}\PYG{p}{(}\PYG{n}{s}\PYG{p}{)}\PYG{p}{)} \PYG{o}{\PYGZlt{}} \PYG{l+m+mf}{0.01}
\PYG{g+go}{True}
\end{Verbatim}

\begin{Verbatim}[commandchars=\\\{\}]
\PYG{g+gp}{\PYGZgt{}\PYGZgt{}\PYGZgt{} }\PYG{n+nb}{abs}\PYG{p}{(}\PYG{n}{sigma} \PYG{o}{\PYGZhy{}} \PYG{n}{np}\PYG{o}{.}\PYG{n}{std}\PYG{p}{(}\PYG{n}{s}\PYG{p}{,} \PYG{n}{ddof}\PYG{o}{=}\PYG{l+m+mi}{1}\PYG{p}{)}\PYG{p}{)} \PYG{o}{\PYGZlt{}} \PYG{l+m+mf}{0.01}
\PYG{g+go}{True}
\end{Verbatim}

Display the histogram of the samples, along with
the probability density function:

\begin{Verbatim}[commandchars=\\\{\}]
\PYG{g+gp}{\PYGZgt{}\PYGZgt{}\PYGZgt{} }\PYG{k+kn}{import} \PYG{n+nn}{matplotlib.pyplot} \PYG{k+kn}{as} \PYG{n+nn}{plt}
\PYG{g+gp}{\PYGZgt{}\PYGZgt{}\PYGZgt{} }\PYG{n}{count}\PYG{p}{,} \PYG{n}{bins}\PYG{p}{,} \PYG{n}{ignored} \PYG{o}{=} \PYG{n}{plt}\PYG{o}{.}\PYG{n}{hist}\PYG{p}{(}\PYG{n}{s}\PYG{p}{,} \PYG{l+m+mi}{30}\PYG{p}{,} \PYG{n}{normed}\PYG{o}{=}\PYG{n+nb+bp}{True}\PYG{p}{)}
\PYG{g+gp}{\PYGZgt{}\PYGZgt{}\PYGZgt{} }\PYG{n}{plt}\PYG{o}{.}\PYG{n}{plot}\PYG{p}{(}\PYG{n}{bins}\PYG{p}{,} \PYG{l+m+mi}{1}\PYG{o}{/}\PYG{p}{(}\PYG{n}{sigma} \PYG{o}{*} \PYG{n}{np}\PYG{o}{.}\PYG{n}{sqrt}\PYG{p}{(}\PYG{l+m+mi}{2} \PYG{o}{*} \PYG{n}{np}\PYG{o}{.}\PYG{n}{pi}\PYG{p}{)}\PYG{p}{)} \PYG{o}{*}
\PYG{g+gp}{... }               \PYG{n}{np}\PYG{o}{.}\PYG{n}{exp}\PYG{p}{(} \PYG{o}{\PYGZhy{}} \PYG{p}{(}\PYG{n}{bins} \PYG{o}{\PYGZhy{}} \PYG{n}{mu}\PYG{p}{)}\PYG{o}{*}\PYG{o}{*}\PYG{l+m+mi}{2} \PYG{o}{/} \PYG{p}{(}\PYG{l+m+mi}{2} \PYG{o}{*} \PYG{n}{sigma}\PYG{o}{*}\PYG{o}{*}\PYG{l+m+mi}{2}\PYG{p}{)} \PYG{p}{)}\PYG{p}{,}
\PYG{g+gp}{... }         \PYG{n}{linewidth}\PYG{o}{=}\PYG{l+m+mi}{2}\PYG{p}{,} \PYG{n}{color}\PYG{o}{=}\PYG{l+s}{\PYGZsq{}}\PYG{l+s}{r}\PYG{l+s}{\PYGZsq{}}\PYG{p}{)}
\PYG{g+gp}{\PYGZgt{}\PYGZgt{}\PYGZgt{} }\PYG{n}{plt}\PYG{o}{.}\PYG{n}{show}\PYG{p}{(}\PYG{p}{)}
\end{Verbatim}

\end{fulllineitems}

\index{pareto() (in module acsAttractorAnalysis)}

\begin{fulllineitems}
\phantomsection\label{acsAttractorAnalysis:acsAttractorAnalysis.pareto}\pysiglinewithargsret{\code{acsAttractorAnalysis.}\bfcode{pareto}}{\emph{a}, \emph{size=None}}{}
Draw samples from a Pareto II or Lomax distribution with specified shape.

The Lomax or Pareto II distribution is a shifted Pareto distribution. The
classical Pareto distribution can be obtained from the Lomax distribution
by adding the location parameter m, see below. The smallest value of the
Lomax distribution is zero while for the classical Pareto distribution it
is m, where the standard Pareto distribution has location m=1.
Lomax can also be considered as a simplified version of the Generalized
Pareto distribution (available in SciPy), with the scale set to one and
the location set to zero.

The Pareto distribution must be greater than zero, and is unbounded above.
It is also known as the ``80-20 rule''.  In this distribution, 80 percent of
the weights are in the lowest 20 percent of the range, while the other 20
percent fill the remaining 80 percent of the range.
\begin{description}
\item[{shape}] \leavevmode{[}float, \textgreater{} 0.{]}
Shape of the distribution.

\item[{size}] \leavevmode{[}tuple of ints{]}
Output shape.  If the given shape is, e.g., \code{(m, n, k)}, then
\code{m * n * k} samples are drawn.

\end{description}
\begin{description}
\item[{scipy.stats.distributions.lomax.pdf}] \leavevmode{[}probability density function,{]}
distribution or cumulative density function, etc.

\item[{scipy.stats.distributions.genpareto.pdf}] \leavevmode{[}probability density function,{]}
distribution or cumulative density function, etc.

\end{description}

The probability density for the Pareto distribution is
\begin{gather}
\begin{split}p(x) = \frac{am^a}{x^{a+1}}\end{split}\notag
\end{gather}
where \(a\) is the shape and \(m\) the location

The Pareto distribution, named after the Italian economist Vilfredo Pareto,
is a power law probability distribution useful in many real world problems.
Outside the field of economics it is generally referred to as the Bradford
distribution. Pareto developed the distribution to describe the
distribution of wealth in an economy.  It has also found use in insurance,
web page access statistics, oil field sizes, and many other problems,
including the download frequency for projects in Sourceforge {[}1{]}.  It is
one of the so-called ``fat-tailed'' distributions.

Draw samples from the distribution:

\begin{Verbatim}[commandchars=\\\{\}]
\PYG{g+gp}{\PYGZgt{}\PYGZgt{}\PYGZgt{} }\PYG{n}{a}\PYG{p}{,} \PYG{n}{m} \PYG{o}{=} \PYG{l+m+mf}{3.}\PYG{p}{,} \PYG{l+m+mf}{1.} \PYG{c}{\PYGZsh{} shape and mode}
\PYG{g+gp}{\PYGZgt{}\PYGZgt{}\PYGZgt{} }\PYG{n}{s} \PYG{o}{=} \PYG{n}{np}\PYG{o}{.}\PYG{n}{random}\PYG{o}{.}\PYG{n}{pareto}\PYG{p}{(}\PYG{n}{a}\PYG{p}{,} \PYG{l+m+mi}{1000}\PYG{p}{)} \PYG{o}{+} \PYG{n}{m}
\end{Verbatim}

Display the histogram of the samples, along with
the probability density function:

\begin{Verbatim}[commandchars=\\\{\}]
\PYG{g+gp}{\PYGZgt{}\PYGZgt{}\PYGZgt{} }\PYG{k+kn}{import} \PYG{n+nn}{matplotlib.pyplot} \PYG{k+kn}{as} \PYG{n+nn}{plt}
\PYG{g+gp}{\PYGZgt{}\PYGZgt{}\PYGZgt{} }\PYG{n}{count}\PYG{p}{,} \PYG{n}{bins}\PYG{p}{,} \PYG{n}{ignored} \PYG{o}{=} \PYG{n}{plt}\PYG{o}{.}\PYG{n}{hist}\PYG{p}{(}\PYG{n}{s}\PYG{p}{,} \PYG{l+m+mi}{100}\PYG{p}{,} \PYG{n}{normed}\PYG{o}{=}\PYG{n+nb+bp}{True}\PYG{p}{,} \PYG{n}{align}\PYG{o}{=}\PYG{l+s}{\PYGZsq{}}\PYG{l+s}{center}\PYG{l+s}{\PYGZsq{}}\PYG{p}{)}
\PYG{g+gp}{\PYGZgt{}\PYGZgt{}\PYGZgt{} }\PYG{n}{fit} \PYG{o}{=} \PYG{n}{a}\PYG{o}{*}\PYG{n}{m}\PYG{o}{*}\PYG{o}{*}\PYG{n}{a}\PYG{o}{/}\PYG{n}{bins}\PYG{o}{*}\PYG{o}{*}\PYG{p}{(}\PYG{n}{a}\PYG{o}{+}\PYG{l+m+mi}{1}\PYG{p}{)}
\PYG{g+gp}{\PYGZgt{}\PYGZgt{}\PYGZgt{} }\PYG{n}{plt}\PYG{o}{.}\PYG{n}{plot}\PYG{p}{(}\PYG{n}{bins}\PYG{p}{,} \PYG{n+nb}{max}\PYG{p}{(}\PYG{n}{count}\PYG{p}{)}\PYG{o}{*}\PYG{n}{fit}\PYG{o}{/}\PYG{n+nb}{max}\PYG{p}{(}\PYG{n}{fit}\PYG{p}{)}\PYG{p}{,}\PYG{n}{linewidth}\PYG{o}{=}\PYG{l+m+mi}{2}\PYG{p}{,} \PYG{n}{color}\PYG{o}{=}\PYG{l+s}{\PYGZsq{}}\PYG{l+s}{r}\PYG{l+s}{\PYGZsq{}}\PYG{p}{)}
\PYG{g+gp}{\PYGZgt{}\PYGZgt{}\PYGZgt{} }\PYG{n}{plt}\PYG{o}{.}\PYG{n}{show}\PYG{p}{(}\PYG{p}{)}
\end{Verbatim}

\end{fulllineitems}

\index{permutation() (in module acsAttractorAnalysis)}

\begin{fulllineitems}
\phantomsection\label{acsAttractorAnalysis:acsAttractorAnalysis.permutation}\pysiglinewithargsret{\code{acsAttractorAnalysis.}\bfcode{permutation}}{\emph{x}}{}
Randomly permute a sequence, or return a permuted range.

If \emph{x} is a multi-dimensional array, it is only shuffled along its
first index.
\begin{description}
\item[{x}] \leavevmode{[}int or array\_like{]}
If \emph{x} is an integer, randomly permute \code{np.arange(x)}.
If \emph{x} is an array, make a copy and shuffle the elements
randomly.

\end{description}
\begin{description}
\item[{out}] \leavevmode{[}ndarray{]}
Permuted sequence or array range.

\end{description}

\begin{Verbatim}[commandchars=\\\{\}]
\PYG{g+gp}{\PYGZgt{}\PYGZgt{}\PYGZgt{} }\PYG{n}{np}\PYG{o}{.}\PYG{n}{random}\PYG{o}{.}\PYG{n}{permutation}\PYG{p}{(}\PYG{l+m+mi}{10}\PYG{p}{)}
\PYG{g+go}{array([1, 7, 4, 3, 0, 9, 2, 5, 8, 6])}
\end{Verbatim}

\begin{Verbatim}[commandchars=\\\{\}]
\PYG{g+gp}{\PYGZgt{}\PYGZgt{}\PYGZgt{} }\PYG{n}{np}\PYG{o}{.}\PYG{n}{random}\PYG{o}{.}\PYG{n}{permutation}\PYG{p}{(}\PYG{p}{[}\PYG{l+m+mi}{1}\PYG{p}{,} \PYG{l+m+mi}{4}\PYG{p}{,} \PYG{l+m+mi}{9}\PYG{p}{,} \PYG{l+m+mi}{12}\PYG{p}{,} \PYG{l+m+mi}{15}\PYG{p}{]}\PYG{p}{)}
\PYG{g+go}{array([15,  1,  9,  4, 12])}
\end{Verbatim}

\begin{Verbatim}[commandchars=\\\{\}]
\PYG{g+gp}{\PYGZgt{}\PYGZgt{}\PYGZgt{} }\PYG{n}{arr} \PYG{o}{=} \PYG{n}{np}\PYG{o}{.}\PYG{n}{arange}\PYG{p}{(}\PYG{l+m+mi}{9}\PYG{p}{)}\PYG{o}{.}\PYG{n}{reshape}\PYG{p}{(}\PYG{p}{(}\PYG{l+m+mi}{3}\PYG{p}{,} \PYG{l+m+mi}{3}\PYG{p}{)}\PYG{p}{)}
\PYG{g+gp}{\PYGZgt{}\PYGZgt{}\PYGZgt{} }\PYG{n}{np}\PYG{o}{.}\PYG{n}{random}\PYG{o}{.}\PYG{n}{permutation}\PYG{p}{(}\PYG{n}{arr}\PYG{p}{)}
\PYG{g+go}{array([[6, 7, 8],}
\PYG{g+go}{       [0, 1, 2],}
\PYG{g+go}{       [3, 4, 5]])}
\end{Verbatim}

\end{fulllineitems}

\index{poisson() (in module acsAttractorAnalysis)}

\begin{fulllineitems}
\phantomsection\label{acsAttractorAnalysis:acsAttractorAnalysis.poisson}\pysiglinewithargsret{\code{acsAttractorAnalysis.}\bfcode{poisson}}{\emph{lam=1.0}, \emph{size=None}}{}
Draw samples from a Poisson distribution.

The Poisson distribution is the limit of the Binomial
distribution for large N.
\begin{description}
\item[{lam}] \leavevmode{[}float{]}
Expectation of interval, should be \textgreater{}= 0.

\item[{size}] \leavevmode{[}int or tuple of ints, optional{]}
Output shape. If the given shape is, e.g., \code{(m, n, k)}, then
\code{m * n * k} samples are drawn.

\end{description}

The Poisson distribution
\begin{gather}
\begin{split}f(k; \lambda)=\frac{\lambda^k e^{-\lambda}}{k!}\end{split}\notag
\end{gather}
For events with an expected separation \(\lambda\) the Poisson
distribution \(f(k; \lambda)\) describes the probability of
\(k\) events occurring within the observed interval \(\lambda\).

Because the output is limited to the range of the C long type, a
ValueError is raised when \emph{lam} is within 10 sigma of the maximum
representable value.

Draw samples from the distribution:

\begin{Verbatim}[commandchars=\\\{\}]
\PYG{g+gp}{\PYGZgt{}\PYGZgt{}\PYGZgt{} }\PYG{k+kn}{import} \PYG{n+nn}{numpy} \PYG{k+kn}{as} \PYG{n+nn}{np}
\PYG{g+gp}{\PYGZgt{}\PYGZgt{}\PYGZgt{} }\PYG{n}{s} \PYG{o}{=} \PYG{n}{np}\PYG{o}{.}\PYG{n}{random}\PYG{o}{.}\PYG{n}{poisson}\PYG{p}{(}\PYG{l+m+mi}{5}\PYG{p}{,} \PYG{l+m+mi}{10000}\PYG{p}{)}
\end{Verbatim}

Display histogram of the sample:

\begin{Verbatim}[commandchars=\\\{\}]
\PYG{g+gp}{\PYGZgt{}\PYGZgt{}\PYGZgt{} }\PYG{k+kn}{import} \PYG{n+nn}{matplotlib.pyplot} \PYG{k+kn}{as} \PYG{n+nn}{plt}
\PYG{g+gp}{\PYGZgt{}\PYGZgt{}\PYGZgt{} }\PYG{n}{count}\PYG{p}{,} \PYG{n}{bins}\PYG{p}{,} \PYG{n}{ignored} \PYG{o}{=} \PYG{n}{plt}\PYG{o}{.}\PYG{n}{hist}\PYG{p}{(}\PYG{n}{s}\PYG{p}{,} \PYG{l+m+mi}{14}\PYG{p}{,} \PYG{n}{normed}\PYG{o}{=}\PYG{n+nb+bp}{True}\PYG{p}{)}
\PYG{g+gp}{\PYGZgt{}\PYGZgt{}\PYGZgt{} }\PYG{n}{plt}\PYG{o}{.}\PYG{n}{show}\PYG{p}{(}\PYG{p}{)}
\end{Verbatim}

\end{fulllineitems}

\index{power() (in module acsAttractorAnalysis)}

\begin{fulllineitems}
\phantomsection\label{acsAttractorAnalysis:acsAttractorAnalysis.power}\pysiglinewithargsret{\code{acsAttractorAnalysis.}\bfcode{power}}{\emph{a}, \emph{size=None}}{}
Draws samples in {[}0, 1{]} from a power distribution with positive
exponent a - 1.

Also known as the power function distribution.
\begin{description}
\item[{a}] \leavevmode{[}float{]}
parameter, \textgreater{} 0

\item[{size}] \leavevmode{[}tuple of ints{]}\begin{description}
\item[{Output shape.  If the given shape is, e.g., \code{(m, n, k)}, then}] \leavevmode
\code{m * n * k} samples are drawn.

\end{description}

\end{description}
\begin{description}
\item[{samples}] \leavevmode{[}\{ndarray, scalar\}{]}
The returned samples lie in {[}0, 1{]}.

\end{description}
\begin{description}
\item[{ValueError}] \leavevmode
If a\textless{}1.

\end{description}

The probability density function is
\begin{gather}
\begin{split}P(x; a) = ax^{a-1}, 0 \le x \le 1, a>0.\end{split}\notag
\end{gather}
The power function distribution is just the inverse of the Pareto
distribution. It may also be seen as a special case of the Beta
distribution.

It is used, for example, in modeling the over-reporting of insurance
claims.

Draw samples from the distribution:

\begin{Verbatim}[commandchars=\\\{\}]
\PYG{g+gp}{\PYGZgt{}\PYGZgt{}\PYGZgt{} }\PYG{n}{a} \PYG{o}{=} \PYG{l+m+mf}{5.} \PYG{c}{\PYGZsh{} shape}
\PYG{g+gp}{\PYGZgt{}\PYGZgt{}\PYGZgt{} }\PYG{n}{samples} \PYG{o}{=} \PYG{l+m+mi}{1000}
\PYG{g+gp}{\PYGZgt{}\PYGZgt{}\PYGZgt{} }\PYG{n}{s} \PYG{o}{=} \PYG{n}{np}\PYG{o}{.}\PYG{n}{random}\PYG{o}{.}\PYG{n}{power}\PYG{p}{(}\PYG{n}{a}\PYG{p}{,} \PYG{n}{samples}\PYG{p}{)}
\end{Verbatim}

Display the histogram of the samples, along with
the probability density function:

\begin{Verbatim}[commandchars=\\\{\}]
\PYG{g+gp}{\PYGZgt{}\PYGZgt{}\PYGZgt{} }\PYG{k+kn}{import} \PYG{n+nn}{matplotlib.pyplot} \PYG{k+kn}{as} \PYG{n+nn}{plt}
\PYG{g+gp}{\PYGZgt{}\PYGZgt{}\PYGZgt{} }\PYG{n}{count}\PYG{p}{,} \PYG{n}{bins}\PYG{p}{,} \PYG{n}{ignored} \PYG{o}{=} \PYG{n}{plt}\PYG{o}{.}\PYG{n}{hist}\PYG{p}{(}\PYG{n}{s}\PYG{p}{,} \PYG{n}{bins}\PYG{o}{=}\PYG{l+m+mi}{30}\PYG{p}{)}
\PYG{g+gp}{\PYGZgt{}\PYGZgt{}\PYGZgt{} }\PYG{n}{x} \PYG{o}{=} \PYG{n}{np}\PYG{o}{.}\PYG{n}{linspace}\PYG{p}{(}\PYG{l+m+mi}{0}\PYG{p}{,} \PYG{l+m+mi}{1}\PYG{p}{,} \PYG{l+m+mi}{100}\PYG{p}{)}
\PYG{g+gp}{\PYGZgt{}\PYGZgt{}\PYGZgt{} }\PYG{n}{y} \PYG{o}{=} \PYG{n}{a}\PYG{o}{*}\PYG{n}{x}\PYG{o}{*}\PYG{o}{*}\PYG{p}{(}\PYG{n}{a}\PYG{o}{\PYGZhy{}}\PYG{l+m+mf}{1.}\PYG{p}{)}
\PYG{g+gp}{\PYGZgt{}\PYGZgt{}\PYGZgt{} }\PYG{n}{normed\PYGZus{}y} \PYG{o}{=} \PYG{n}{samples}\PYG{o}{*}\PYG{n}{np}\PYG{o}{.}\PYG{n}{diff}\PYG{p}{(}\PYG{n}{bins}\PYG{p}{)}\PYG{p}{[}\PYG{l+m+mi}{0}\PYG{p}{]}\PYG{o}{*}\PYG{n}{y}
\PYG{g+gp}{\PYGZgt{}\PYGZgt{}\PYGZgt{} }\PYG{n}{plt}\PYG{o}{.}\PYG{n}{plot}\PYG{p}{(}\PYG{n}{x}\PYG{p}{,} \PYG{n}{normed\PYGZus{}y}\PYG{p}{)}
\PYG{g+gp}{\PYGZgt{}\PYGZgt{}\PYGZgt{} }\PYG{n}{plt}\PYG{o}{.}\PYG{n}{show}\PYG{p}{(}\PYG{p}{)}
\end{Verbatim}

Compare the power function distribution to the inverse of the Pareto.

\begin{Verbatim}[commandchars=\\\{\}]
\PYG{g+gp}{\PYGZgt{}\PYGZgt{}\PYGZgt{} }\PYG{k+kn}{from} \PYG{n+nn}{scipy} \PYG{k+kn}{import} \PYG{n}{stats}
\PYG{g+gp}{\PYGZgt{}\PYGZgt{}\PYGZgt{} }\PYG{n}{rvs} \PYG{o}{=} \PYG{n}{np}\PYG{o}{.}\PYG{n}{random}\PYG{o}{.}\PYG{n}{power}\PYG{p}{(}\PYG{l+m+mi}{5}\PYG{p}{,} \PYG{l+m+mi}{1000000}\PYG{p}{)}
\PYG{g+gp}{\PYGZgt{}\PYGZgt{}\PYGZgt{} }\PYG{n}{rvsp} \PYG{o}{=} \PYG{n}{np}\PYG{o}{.}\PYG{n}{random}\PYG{o}{.}\PYG{n}{pareto}\PYG{p}{(}\PYG{l+m+mi}{5}\PYG{p}{,} \PYG{l+m+mi}{1000000}\PYG{p}{)}
\PYG{g+gp}{\PYGZgt{}\PYGZgt{}\PYGZgt{} }\PYG{n}{xx} \PYG{o}{=} \PYG{n}{np}\PYG{o}{.}\PYG{n}{linspace}\PYG{p}{(}\PYG{l+m+mi}{0}\PYG{p}{,}\PYG{l+m+mi}{1}\PYG{p}{,}\PYG{l+m+mi}{100}\PYG{p}{)}
\PYG{g+gp}{\PYGZgt{}\PYGZgt{}\PYGZgt{} }\PYG{n}{powpdf} \PYG{o}{=} \PYG{n}{stats}\PYG{o}{.}\PYG{n}{powerlaw}\PYG{o}{.}\PYG{n}{pdf}\PYG{p}{(}\PYG{n}{xx}\PYG{p}{,}\PYG{l+m+mi}{5}\PYG{p}{)}
\end{Verbatim}

\begin{Verbatim}[commandchars=\\\{\}]
\PYG{g+gp}{\PYGZgt{}\PYGZgt{}\PYGZgt{} }\PYG{n}{plt}\PYG{o}{.}\PYG{n}{figure}\PYG{p}{(}\PYG{p}{)}
\PYG{g+gp}{\PYGZgt{}\PYGZgt{}\PYGZgt{} }\PYG{n}{plt}\PYG{o}{.}\PYG{n}{hist}\PYG{p}{(}\PYG{n}{rvs}\PYG{p}{,} \PYG{n}{bins}\PYG{o}{=}\PYG{l+m+mi}{50}\PYG{p}{,} \PYG{n}{normed}\PYG{o}{=}\PYG{n+nb+bp}{True}\PYG{p}{)}
\PYG{g+gp}{\PYGZgt{}\PYGZgt{}\PYGZgt{} }\PYG{n}{plt}\PYG{o}{.}\PYG{n}{plot}\PYG{p}{(}\PYG{n}{xx}\PYG{p}{,}\PYG{n}{powpdf}\PYG{p}{,}\PYG{l+s}{\PYGZsq{}}\PYG{l+s}{r\PYGZhy{}}\PYG{l+s}{\PYGZsq{}}\PYG{p}{)}
\PYG{g+gp}{\PYGZgt{}\PYGZgt{}\PYGZgt{} }\PYG{n}{plt}\PYG{o}{.}\PYG{n}{title}\PYG{p}{(}\PYG{l+s}{\PYGZsq{}}\PYG{l+s}{np.random.power(5)}\PYG{l+s}{\PYGZsq{}}\PYG{p}{)}
\end{Verbatim}

\begin{Verbatim}[commandchars=\\\{\}]
\PYG{g+gp}{\PYGZgt{}\PYGZgt{}\PYGZgt{} }\PYG{n}{plt}\PYG{o}{.}\PYG{n}{figure}\PYG{p}{(}\PYG{p}{)}
\PYG{g+gp}{\PYGZgt{}\PYGZgt{}\PYGZgt{} }\PYG{n}{plt}\PYG{o}{.}\PYG{n}{hist}\PYG{p}{(}\PYG{l+m+mf}{1.}\PYG{o}{/}\PYG{p}{(}\PYG{l+m+mf}{1.}\PYG{o}{+}\PYG{n}{rvsp}\PYG{p}{)}\PYG{p}{,} \PYG{n}{bins}\PYG{o}{=}\PYG{l+m+mi}{50}\PYG{p}{,} \PYG{n}{normed}\PYG{o}{=}\PYG{n+nb+bp}{True}\PYG{p}{)}
\PYG{g+gp}{\PYGZgt{}\PYGZgt{}\PYGZgt{} }\PYG{n}{plt}\PYG{o}{.}\PYG{n}{plot}\PYG{p}{(}\PYG{n}{xx}\PYG{p}{,}\PYG{n}{powpdf}\PYG{p}{,}\PYG{l+s}{\PYGZsq{}}\PYG{l+s}{r\PYGZhy{}}\PYG{l+s}{\PYGZsq{}}\PYG{p}{)}
\PYG{g+gp}{\PYGZgt{}\PYGZgt{}\PYGZgt{} }\PYG{n}{plt}\PYG{o}{.}\PYG{n}{title}\PYG{p}{(}\PYG{l+s}{\PYGZsq{}}\PYG{l+s}{inverse of 1 + np.random.pareto(5)}\PYG{l+s}{\PYGZsq{}}\PYG{p}{)}
\end{Verbatim}

\begin{Verbatim}[commandchars=\\\{\}]
\PYG{g+gp}{\PYGZgt{}\PYGZgt{}\PYGZgt{} }\PYG{n}{plt}\PYG{o}{.}\PYG{n}{figure}\PYG{p}{(}\PYG{p}{)}
\PYG{g+gp}{\PYGZgt{}\PYGZgt{}\PYGZgt{} }\PYG{n}{plt}\PYG{o}{.}\PYG{n}{hist}\PYG{p}{(}\PYG{l+m+mf}{1.}\PYG{o}{/}\PYG{p}{(}\PYG{l+m+mf}{1.}\PYG{o}{+}\PYG{n}{rvsp}\PYG{p}{)}\PYG{p}{,} \PYG{n}{bins}\PYG{o}{=}\PYG{l+m+mi}{50}\PYG{p}{,} \PYG{n}{normed}\PYG{o}{=}\PYG{n+nb+bp}{True}\PYG{p}{)}
\PYG{g+gp}{\PYGZgt{}\PYGZgt{}\PYGZgt{} }\PYG{n}{plt}\PYG{o}{.}\PYG{n}{plot}\PYG{p}{(}\PYG{n}{xx}\PYG{p}{,}\PYG{n}{powpdf}\PYG{p}{,}\PYG{l+s}{\PYGZsq{}}\PYG{l+s}{r\PYGZhy{}}\PYG{l+s}{\PYGZsq{}}\PYG{p}{)}
\PYG{g+gp}{\PYGZgt{}\PYGZgt{}\PYGZgt{} }\PYG{n}{plt}\PYG{o}{.}\PYG{n}{title}\PYG{p}{(}\PYG{l+s}{\PYGZsq{}}\PYG{l+s}{inverse of stats.pareto(5)}\PYG{l+s}{\PYGZsq{}}\PYG{p}{)}
\end{Verbatim}

\end{fulllineitems}

\index{rand() (in module acsAttractorAnalysis)}

\begin{fulllineitems}
\phantomsection\label{acsAttractorAnalysis:acsAttractorAnalysis.rand}\pysiglinewithargsret{\code{acsAttractorAnalysis.}\bfcode{rand}}{\emph{d0}, \emph{d1}, \emph{...}, \emph{dn}}{}
Random values in a given shape.

Create an array of the given shape and propagate it with
random samples from a uniform distribution
over \code{{[}0, 1)}.
\begin{description}
\item[{d0, d1, ..., dn}] \leavevmode{[}int, optional{]}
The dimensions of the returned array, should all be positive.
If no argument is given a single Python float is returned.

\end{description}
\begin{description}
\item[{out}] \leavevmode{[}ndarray, shape \code{(d0, d1, ..., dn)}{]}
Random values.

\end{description}

random

This is a convenience function. If you want an interface that
takes a shape-tuple as the first argument, refer to
np.random.random\_sample .

\begin{Verbatim}[commandchars=\\\{\}]
\PYG{g+gp}{\PYGZgt{}\PYGZgt{}\PYGZgt{} }\PYG{n}{np}\PYG{o}{.}\PYG{n}{random}\PYG{o}{.}\PYG{n}{rand}\PYG{p}{(}\PYG{l+m+mi}{3}\PYG{p}{,}\PYG{l+m+mi}{2}\PYG{p}{)}
\PYG{g+go}{array([[ 0.14022471,  0.96360618],  \PYGZsh{}random}
\PYG{g+go}{       [ 0.37601032,  0.25528411],  \PYGZsh{}random}
\PYG{g+go}{       [ 0.49313049,  0.94909878]]) \PYGZsh{}random}
\end{Verbatim}

\end{fulllineitems}

\index{randint() (in module acsAttractorAnalysis)}

\begin{fulllineitems}
\phantomsection\label{acsAttractorAnalysis:acsAttractorAnalysis.randint}\pysiglinewithargsret{\code{acsAttractorAnalysis.}\bfcode{randint}}{\emph{low}, \emph{high=None}, \emph{size=None}}{}
Return random integers from \emph{low} (inclusive) to \emph{high} (exclusive).

Return random integers from the ``discrete uniform'' distribution in the
``half-open'' interval {[}\emph{low}, \emph{high}). If \emph{high} is None (the default),
then results are from {[}0, \emph{low}).
\begin{description}
\item[{low}] \leavevmode{[}int{]}
Lowest (signed) integer to be drawn from the distribution (unless
\code{high=None}, in which case this parameter is the \emph{highest} such
integer).

\item[{high}] \leavevmode{[}int, optional{]}
If provided, one above the largest (signed) integer to be drawn
from the distribution (see above for behavior if \code{high=None}).

\item[{size}] \leavevmode{[}int or tuple of ints, optional{]}
Output shape. Default is None, in which case a single int is
returned.

\end{description}
\begin{description}
\item[{out}] \leavevmode{[}int or ndarray of ints{]}
\emph{size}-shaped array of random integers from the appropriate
distribution, or a single such random int if \emph{size} not provided.

\end{description}
\begin{description}
\item[{random.random\_integers}] \leavevmode{[}similar to \emph{randint}, only for the closed{]}
interval {[}\emph{low}, \emph{high}{]}, and 1 is the lowest value if \emph{high} is
omitted. In particular, this other one is the one to use to generate
uniformly distributed discrete non-integers.

\end{description}

\begin{Verbatim}[commandchars=\\\{\}]
\PYG{g+gp}{\PYGZgt{}\PYGZgt{}\PYGZgt{} }\PYG{n}{np}\PYG{o}{.}\PYG{n}{random}\PYG{o}{.}\PYG{n}{randint}\PYG{p}{(}\PYG{l+m+mi}{2}\PYG{p}{,} \PYG{n}{size}\PYG{o}{=}\PYG{l+m+mi}{10}\PYG{p}{)}
\PYG{g+go}{array([1, 0, 0, 0, 1, 1, 0, 0, 1, 0])}
\PYG{g+gp}{\PYGZgt{}\PYGZgt{}\PYGZgt{} }\PYG{n}{np}\PYG{o}{.}\PYG{n}{random}\PYG{o}{.}\PYG{n}{randint}\PYG{p}{(}\PYG{l+m+mi}{1}\PYG{p}{,} \PYG{n}{size}\PYG{o}{=}\PYG{l+m+mi}{10}\PYG{p}{)}
\PYG{g+go}{array([0, 0, 0, 0, 0, 0, 0, 0, 0, 0])}
\end{Verbatim}

Generate a 2 x 4 array of ints between 0 and 4, inclusive:

\begin{Verbatim}[commandchars=\\\{\}]
\PYG{g+gp}{\PYGZgt{}\PYGZgt{}\PYGZgt{} }\PYG{n}{np}\PYG{o}{.}\PYG{n}{random}\PYG{o}{.}\PYG{n}{randint}\PYG{p}{(}\PYG{l+m+mi}{5}\PYG{p}{,} \PYG{n}{size}\PYG{o}{=}\PYG{p}{(}\PYG{l+m+mi}{2}\PYG{p}{,} \PYG{l+m+mi}{4}\PYG{p}{)}\PYG{p}{)}
\PYG{g+go}{array([[4, 0, 2, 1],}
\PYG{g+go}{       [3, 2, 2, 0]])}
\end{Verbatim}

\end{fulllineitems}

\index{randn() (in module acsAttractorAnalysis)}

\begin{fulllineitems}
\phantomsection\label{acsAttractorAnalysis:acsAttractorAnalysis.randn}\pysiglinewithargsret{\code{acsAttractorAnalysis.}\bfcode{randn}}{\emph{d0}, \emph{d1}, \emph{...}, \emph{dn}}{}
Return a sample (or samples) from the ``standard normal'' distribution.

If positive, int\_like or int-convertible arguments are provided,
\emph{randn} generates an array of shape \code{(d0, d1, ..., dn)}, filled
with random floats sampled from a univariate ``normal'' (Gaussian)
distribution of mean 0 and variance 1 (if any of the \(d_i\) are
floats, they are first converted to integers by truncation). A single
float randomly sampled from the distribution is returned if no
argument is provided.

This is a convenience function.  If you want an interface that takes a
tuple as the first argument, use \emph{numpy.random.standard\_normal} instead.
\begin{description}
\item[{d0, d1, ..., dn}] \leavevmode{[}int, optional{]}
The dimensions of the returned array, should be all positive.
If no argument is given a single Python float is returned.

\end{description}
\begin{description}
\item[{Z}] \leavevmode{[}ndarray or float{]}
A \code{(d0, d1, ..., dn)}-shaped array of floating-point samples from
the standard normal distribution, or a single such float if
no parameters were supplied.

\end{description}

random.standard\_normal : Similar, but takes a tuple as its argument.

For random samples from \(N(\mu, \sigma^2)\), use:

\code{sigma * np.random.randn(...) + mu}

\begin{Verbatim}[commandchars=\\\{\}]
\PYG{g+gp}{\PYGZgt{}\PYGZgt{}\PYGZgt{} }\PYG{n}{np}\PYG{o}{.}\PYG{n}{random}\PYG{o}{.}\PYG{n}{randn}\PYG{p}{(}\PYG{p}{)}
\PYG{g+go}{2.1923875335537315 \PYGZsh{}random}
\end{Verbatim}

Two-by-four array of samples from N(3, 6.25):

\begin{Verbatim}[commandchars=\\\{\}]
\PYG{g+gp}{\PYGZgt{}\PYGZgt{}\PYGZgt{} }\PYG{l+m+mf}{2.5} \PYG{o}{*} \PYG{n}{np}\PYG{o}{.}\PYG{n}{random}\PYG{o}{.}\PYG{n}{randn}\PYG{p}{(}\PYG{l+m+mi}{2}\PYG{p}{,} \PYG{l+m+mi}{4}\PYG{p}{)} \PYG{o}{+} \PYG{l+m+mi}{3}
\PYG{g+go}{array([[\PYGZhy{}4.49401501,  4.00950034, \PYGZhy{}1.81814867,  7.29718677],  \PYGZsh{}random}
\PYG{g+go}{       [ 0.39924804,  4.68456316,  4.99394529,  4.84057254]]) \PYGZsh{}random}
\end{Verbatim}

\end{fulllineitems}

\index{random() (in module acsAttractorAnalysis)}

\begin{fulllineitems}
\phantomsection\label{acsAttractorAnalysis:acsAttractorAnalysis.random}\pysiglinewithargsret{\code{acsAttractorAnalysis.}\bfcode{random}}{}{}
random\_sample(size=None)

Return random floats in the half-open interval {[}0.0, 1.0).

Results are from the ``continuous uniform'' distribution over the
stated interval.  To sample \(Unif[a, b), b > a\) multiply
the output of \emph{random\_sample} by \emph{(b-a)} and add \emph{a}:

\begin{Verbatim}[commandchars=\\\{\}]
\PYG{p}{(}\PYG{n}{b} \PYG{o}{\PYGZhy{}} \PYG{n}{a}\PYG{p}{)} \PYG{o}{*} \PYG{n}{random\PYGZus{}sample}\PYG{p}{(}\PYG{p}{)} \PYG{o}{+} \PYG{n}{a}
\end{Verbatim}
\begin{description}
\item[{size}] \leavevmode{[}int or tuple of ints, optional{]}
Defines the shape of the returned array of random floats. If None
(the default), returns a single float.

\end{description}
\begin{description}
\item[{out}] \leavevmode{[}float or ndarray of floats{]}
Array of random floats of shape \emph{size} (unless \code{size=None}, in which
case a single float is returned).

\end{description}

\begin{Verbatim}[commandchars=\\\{\}]
\PYG{g+gp}{\PYGZgt{}\PYGZgt{}\PYGZgt{} }\PYG{n}{np}\PYG{o}{.}\PYG{n}{random}\PYG{o}{.}\PYG{n}{random\PYGZus{}sample}\PYG{p}{(}\PYG{p}{)}
\PYG{g+go}{0.47108547995356098}
\PYG{g+gp}{\PYGZgt{}\PYGZgt{}\PYGZgt{} }\PYG{n+nb}{type}\PYG{p}{(}\PYG{n}{np}\PYG{o}{.}\PYG{n}{random}\PYG{o}{.}\PYG{n}{random\PYGZus{}sample}\PYG{p}{(}\PYG{p}{)}\PYG{p}{)}
\PYG{g+go}{\PYGZlt{}type \PYGZsq{}float\PYGZsq{}\PYGZgt{}}
\PYG{g+gp}{\PYGZgt{}\PYGZgt{}\PYGZgt{} }\PYG{n}{np}\PYG{o}{.}\PYG{n}{random}\PYG{o}{.}\PYG{n}{random\PYGZus{}sample}\PYG{p}{(}\PYG{p}{(}\PYG{l+m+mi}{5}\PYG{p}{,}\PYG{p}{)}\PYG{p}{)}
\PYG{g+go}{array([ 0.30220482,  0.86820401,  0.1654503 ,  0.11659149,  0.54323428])}
\end{Verbatim}

Three-by-two array of random numbers from {[}-5, 0):

\begin{Verbatim}[commandchars=\\\{\}]
\PYG{g+gp}{\PYGZgt{}\PYGZgt{}\PYGZgt{} }\PYG{l+m+mi}{5} \PYG{o}{*} \PYG{n}{np}\PYG{o}{.}\PYG{n}{random}\PYG{o}{.}\PYG{n}{random\PYGZus{}sample}\PYG{p}{(}\PYG{p}{(}\PYG{l+m+mi}{3}\PYG{p}{,} \PYG{l+m+mi}{2}\PYG{p}{)}\PYG{p}{)} \PYG{o}{\PYGZhy{}} \PYG{l+m+mi}{5}
\PYG{g+go}{array([[\PYGZhy{}3.99149989, \PYGZhy{}0.52338984],}
\PYG{g+go}{       [\PYGZhy{}2.99091858, \PYGZhy{}0.79479508],}
\PYG{g+go}{       [\PYGZhy{}1.23204345, \PYGZhy{}1.75224494]])}
\end{Verbatim}

\end{fulllineitems}

\index{random\_integers() (in module acsAttractorAnalysis)}

\begin{fulllineitems}
\phantomsection\label{acsAttractorAnalysis:acsAttractorAnalysis.random_integers}\pysiglinewithargsret{\code{acsAttractorAnalysis.}\bfcode{random\_integers}}{\emph{low}, \emph{high=None}, \emph{size=None}}{}
Return random integers between \emph{low} and \emph{high}, inclusive.

Return random integers from the ``discrete uniform'' distribution in the
closed interval {[}\emph{low}, \emph{high}{]}.  If \emph{high} is None (the default),
then results are from {[}1, \emph{low}{]}.
\begin{description}
\item[{low}] \leavevmode{[}int{]}
Lowest (signed) integer to be drawn from the distribution (unless
\code{high=None}, in which case this parameter is the \emph{highest} such
integer).

\item[{high}] \leavevmode{[}int, optional{]}
If provided, the largest (signed) integer to be drawn from the
distribution (see above for behavior if \code{high=None}).

\item[{size}] \leavevmode{[}int or tuple of ints, optional{]}
Output shape. Default is None, in which case a single int is returned.

\end{description}
\begin{description}
\item[{out}] \leavevmode{[}int or ndarray of ints{]}
\emph{size}-shaped array of random integers from the appropriate
distribution, or a single such random int if \emph{size} not provided.

\end{description}
\begin{description}
\item[{random.randint}] \leavevmode{[}Similar to \emph{random\_integers}, only for the half-open{]}
interval {[}\emph{low}, \emph{high}), and 0 is the lowest value if \emph{high} is
omitted.

\end{description}

To sample from N evenly spaced floating-point numbers between a and b,
use:

\begin{Verbatim}[commandchars=\\\{\}]
\PYG{n}{a} \PYG{o}{+} \PYG{p}{(}\PYG{n}{b} \PYG{o}{\PYGZhy{}} \PYG{n}{a}\PYG{p}{)} \PYG{o}{*} \PYG{p}{(}\PYG{n}{np}\PYG{o}{.}\PYG{n}{random}\PYG{o}{.}\PYG{n}{random\PYGZus{}integers}\PYG{p}{(}\PYG{n}{N}\PYG{p}{)} \PYG{o}{\PYGZhy{}} \PYG{l+m+mi}{1}\PYG{p}{)} \PYG{o}{/} \PYG{p}{(}\PYG{n}{N} \PYG{o}{\PYGZhy{}} \PYG{l+m+mf}{1.}\PYG{p}{)}
\end{Verbatim}

\begin{Verbatim}[commandchars=\\\{\}]
\PYG{g+gp}{\PYGZgt{}\PYGZgt{}\PYGZgt{} }\PYG{n}{np}\PYG{o}{.}\PYG{n}{random}\PYG{o}{.}\PYG{n}{random\PYGZus{}integers}\PYG{p}{(}\PYG{l+m+mi}{5}\PYG{p}{)}
\PYG{g+go}{4}
\PYG{g+gp}{\PYGZgt{}\PYGZgt{}\PYGZgt{} }\PYG{n+nb}{type}\PYG{p}{(}\PYG{n}{np}\PYG{o}{.}\PYG{n}{random}\PYG{o}{.}\PYG{n}{random\PYGZus{}integers}\PYG{p}{(}\PYG{l+m+mi}{5}\PYG{p}{)}\PYG{p}{)}
\PYG{g+go}{\PYGZlt{}type \PYGZsq{}int\PYGZsq{}\PYGZgt{}}
\PYG{g+gp}{\PYGZgt{}\PYGZgt{}\PYGZgt{} }\PYG{n}{np}\PYG{o}{.}\PYG{n}{random}\PYG{o}{.}\PYG{n}{random\PYGZus{}integers}\PYG{p}{(}\PYG{l+m+mi}{5}\PYG{p}{,} \PYG{n}{size}\PYG{o}{=}\PYG{p}{(}\PYG{l+m+mf}{3.}\PYG{p}{,}\PYG{l+m+mf}{2.}\PYG{p}{)}\PYG{p}{)}
\PYG{g+go}{array([[5, 4],}
\PYG{g+go}{       [3, 3],}
\PYG{g+go}{       [4, 5]])}
\end{Verbatim}

Choose five random numbers from the set of five evenly-spaced
numbers between 0 and 2.5, inclusive (\emph{i.e.}, from the set
\({0, 5/8, 10/8, 15/8, 20/8}\)):

\begin{Verbatim}[commandchars=\\\{\}]
\PYG{g+gp}{\PYGZgt{}\PYGZgt{}\PYGZgt{} }\PYG{l+m+mf}{2.5} \PYG{o}{*} \PYG{p}{(}\PYG{n}{np}\PYG{o}{.}\PYG{n}{random}\PYG{o}{.}\PYG{n}{random\PYGZus{}integers}\PYG{p}{(}\PYG{l+m+mi}{5}\PYG{p}{,} \PYG{n}{size}\PYG{o}{=}\PYG{p}{(}\PYG{l+m+mi}{5}\PYG{p}{,}\PYG{p}{)}\PYG{p}{)} \PYG{o}{\PYGZhy{}} \PYG{l+m+mi}{1}\PYG{p}{)} \PYG{o}{/} \PYG{l+m+mf}{4.}
\PYG{g+go}{array([ 0.625,  1.25 ,  0.625,  0.625,  2.5  ])}
\end{Verbatim}

Roll two six sided dice 1000 times and sum the results:

\begin{Verbatim}[commandchars=\\\{\}]
\PYG{g+gp}{\PYGZgt{}\PYGZgt{}\PYGZgt{} }\PYG{n}{d1} \PYG{o}{=} \PYG{n}{np}\PYG{o}{.}\PYG{n}{random}\PYG{o}{.}\PYG{n}{random\PYGZus{}integers}\PYG{p}{(}\PYG{l+m+mi}{1}\PYG{p}{,} \PYG{l+m+mi}{6}\PYG{p}{,} \PYG{l+m+mi}{1000}\PYG{p}{)}
\PYG{g+gp}{\PYGZgt{}\PYGZgt{}\PYGZgt{} }\PYG{n}{d2} \PYG{o}{=} \PYG{n}{np}\PYG{o}{.}\PYG{n}{random}\PYG{o}{.}\PYG{n}{random\PYGZus{}integers}\PYG{p}{(}\PYG{l+m+mi}{1}\PYG{p}{,} \PYG{l+m+mi}{6}\PYG{p}{,} \PYG{l+m+mi}{1000}\PYG{p}{)}
\PYG{g+gp}{\PYGZgt{}\PYGZgt{}\PYGZgt{} }\PYG{n}{dsums} \PYG{o}{=} \PYG{n}{d1} \PYG{o}{+} \PYG{n}{d2}
\end{Verbatim}

Display results as a histogram:

\begin{Verbatim}[commandchars=\\\{\}]
\PYG{g+gp}{\PYGZgt{}\PYGZgt{}\PYGZgt{} }\PYG{k+kn}{import} \PYG{n+nn}{matplotlib.pyplot} \PYG{k+kn}{as} \PYG{n+nn}{plt}
\PYG{g+gp}{\PYGZgt{}\PYGZgt{}\PYGZgt{} }\PYG{n}{count}\PYG{p}{,} \PYG{n}{bins}\PYG{p}{,} \PYG{n}{ignored} \PYG{o}{=} \PYG{n}{plt}\PYG{o}{.}\PYG{n}{hist}\PYG{p}{(}\PYG{n}{dsums}\PYG{p}{,} \PYG{l+m+mi}{11}\PYG{p}{,} \PYG{n}{normed}\PYG{o}{=}\PYG{n+nb+bp}{True}\PYG{p}{)}
\PYG{g+gp}{\PYGZgt{}\PYGZgt{}\PYGZgt{} }\PYG{n}{plt}\PYG{o}{.}\PYG{n}{show}\PYG{p}{(}\PYG{p}{)}
\end{Verbatim}

\end{fulllineitems}

\index{random\_sample() (in module acsAttractorAnalysis)}

\begin{fulllineitems}
\phantomsection\label{acsAttractorAnalysis:acsAttractorAnalysis.random_sample}\pysiglinewithargsret{\code{acsAttractorAnalysis.}\bfcode{random\_sample}}{\emph{size=None}}{}
Return random floats in the half-open interval {[}0.0, 1.0).

Results are from the ``continuous uniform'' distribution over the
stated interval.  To sample \(Unif[a, b), b > a\) multiply
the output of \emph{random\_sample} by \emph{(b-a)} and add \emph{a}:

\begin{Verbatim}[commandchars=\\\{\}]
\PYG{p}{(}\PYG{n}{b} \PYG{o}{\PYGZhy{}} \PYG{n}{a}\PYG{p}{)} \PYG{o}{*} \PYG{n}{random\PYGZus{}sample}\PYG{p}{(}\PYG{p}{)} \PYG{o}{+} \PYG{n}{a}
\end{Verbatim}
\begin{description}
\item[{size}] \leavevmode{[}int or tuple of ints, optional{]}
Defines the shape of the returned array of random floats. If None
(the default), returns a single float.

\end{description}
\begin{description}
\item[{out}] \leavevmode{[}float or ndarray of floats{]}
Array of random floats of shape \emph{size} (unless \code{size=None}, in which
case a single float is returned).

\end{description}

\begin{Verbatim}[commandchars=\\\{\}]
\PYG{g+gp}{\PYGZgt{}\PYGZgt{}\PYGZgt{} }\PYG{n}{np}\PYG{o}{.}\PYG{n}{random}\PYG{o}{.}\PYG{n}{random\PYGZus{}sample}\PYG{p}{(}\PYG{p}{)}
\PYG{g+go}{0.47108547995356098}
\PYG{g+gp}{\PYGZgt{}\PYGZgt{}\PYGZgt{} }\PYG{n+nb}{type}\PYG{p}{(}\PYG{n}{np}\PYG{o}{.}\PYG{n}{random}\PYG{o}{.}\PYG{n}{random\PYGZus{}sample}\PYG{p}{(}\PYG{p}{)}\PYG{p}{)}
\PYG{g+go}{\PYGZlt{}type \PYGZsq{}float\PYGZsq{}\PYGZgt{}}
\PYG{g+gp}{\PYGZgt{}\PYGZgt{}\PYGZgt{} }\PYG{n}{np}\PYG{o}{.}\PYG{n}{random}\PYG{o}{.}\PYG{n}{random\PYGZus{}sample}\PYG{p}{(}\PYG{p}{(}\PYG{l+m+mi}{5}\PYG{p}{,}\PYG{p}{)}\PYG{p}{)}
\PYG{g+go}{array([ 0.30220482,  0.86820401,  0.1654503 ,  0.11659149,  0.54323428])}
\end{Verbatim}

Three-by-two array of random numbers from {[}-5, 0):

\begin{Verbatim}[commandchars=\\\{\}]
\PYG{g+gp}{\PYGZgt{}\PYGZgt{}\PYGZgt{} }\PYG{l+m+mi}{5} \PYG{o}{*} \PYG{n}{np}\PYG{o}{.}\PYG{n}{random}\PYG{o}{.}\PYG{n}{random\PYGZus{}sample}\PYG{p}{(}\PYG{p}{(}\PYG{l+m+mi}{3}\PYG{p}{,} \PYG{l+m+mi}{2}\PYG{p}{)}\PYG{p}{)} \PYG{o}{\PYGZhy{}} \PYG{l+m+mi}{5}
\PYG{g+go}{array([[\PYGZhy{}3.99149989, \PYGZhy{}0.52338984],}
\PYG{g+go}{       [\PYGZhy{}2.99091858, \PYGZhy{}0.79479508],}
\PYG{g+go}{       [\PYGZhy{}1.23204345, \PYGZhy{}1.75224494]])}
\end{Verbatim}

\end{fulllineitems}

\index{ranf() (in module acsAttractorAnalysis)}

\begin{fulllineitems}
\phantomsection\label{acsAttractorAnalysis:acsAttractorAnalysis.ranf}\pysiglinewithargsret{\code{acsAttractorAnalysis.}\bfcode{ranf}}{}{}
random\_sample(size=None)

Return random floats in the half-open interval {[}0.0, 1.0).

Results are from the ``continuous uniform'' distribution over the
stated interval.  To sample \(Unif[a, b), b > a\) multiply
the output of \emph{random\_sample} by \emph{(b-a)} and add \emph{a}:

\begin{Verbatim}[commandchars=\\\{\}]
\PYG{p}{(}\PYG{n}{b} \PYG{o}{\PYGZhy{}} \PYG{n}{a}\PYG{p}{)} \PYG{o}{*} \PYG{n}{random\PYGZus{}sample}\PYG{p}{(}\PYG{p}{)} \PYG{o}{+} \PYG{n}{a}
\end{Verbatim}
\begin{description}
\item[{size}] \leavevmode{[}int or tuple of ints, optional{]}
Defines the shape of the returned array of random floats. If None
(the default), returns a single float.

\end{description}
\begin{description}
\item[{out}] \leavevmode{[}float or ndarray of floats{]}
Array of random floats of shape \emph{size} (unless \code{size=None}, in which
case a single float is returned).

\end{description}

\begin{Verbatim}[commandchars=\\\{\}]
\PYG{g+gp}{\PYGZgt{}\PYGZgt{}\PYGZgt{} }\PYG{n}{np}\PYG{o}{.}\PYG{n}{random}\PYG{o}{.}\PYG{n}{random\PYGZus{}sample}\PYG{p}{(}\PYG{p}{)}
\PYG{g+go}{0.47108547995356098}
\PYG{g+gp}{\PYGZgt{}\PYGZgt{}\PYGZgt{} }\PYG{n+nb}{type}\PYG{p}{(}\PYG{n}{np}\PYG{o}{.}\PYG{n}{random}\PYG{o}{.}\PYG{n}{random\PYGZus{}sample}\PYG{p}{(}\PYG{p}{)}\PYG{p}{)}
\PYG{g+go}{\PYGZlt{}type \PYGZsq{}float\PYGZsq{}\PYGZgt{}}
\PYG{g+gp}{\PYGZgt{}\PYGZgt{}\PYGZgt{} }\PYG{n}{np}\PYG{o}{.}\PYG{n}{random}\PYG{o}{.}\PYG{n}{random\PYGZus{}sample}\PYG{p}{(}\PYG{p}{(}\PYG{l+m+mi}{5}\PYG{p}{,}\PYG{p}{)}\PYG{p}{)}
\PYG{g+go}{array([ 0.30220482,  0.86820401,  0.1654503 ,  0.11659149,  0.54323428])}
\end{Verbatim}

Three-by-two array of random numbers from {[}-5, 0):

\begin{Verbatim}[commandchars=\\\{\}]
\PYG{g+gp}{\PYGZgt{}\PYGZgt{}\PYGZgt{} }\PYG{l+m+mi}{5} \PYG{o}{*} \PYG{n}{np}\PYG{o}{.}\PYG{n}{random}\PYG{o}{.}\PYG{n}{random\PYGZus{}sample}\PYG{p}{(}\PYG{p}{(}\PYG{l+m+mi}{3}\PYG{p}{,} \PYG{l+m+mi}{2}\PYG{p}{)}\PYG{p}{)} \PYG{o}{\PYGZhy{}} \PYG{l+m+mi}{5}
\PYG{g+go}{array([[\PYGZhy{}3.99149989, \PYGZhy{}0.52338984],}
\PYG{g+go}{       [\PYGZhy{}2.99091858, \PYGZhy{}0.79479508],}
\PYG{g+go}{       [\PYGZhy{}1.23204345, \PYGZhy{}1.75224494]])}
\end{Verbatim}

\end{fulllineitems}

\index{rayleigh() (in module acsAttractorAnalysis)}

\begin{fulllineitems}
\phantomsection\label{acsAttractorAnalysis:acsAttractorAnalysis.rayleigh}\pysiglinewithargsret{\code{acsAttractorAnalysis.}\bfcode{rayleigh}}{\emph{scale=1.0}, \emph{size=None}}{}
Draw samples from a Rayleigh distribution.

The \(\chi\) and Weibull distributions are generalizations of the
Rayleigh.
\begin{description}
\item[{scale}] \leavevmode{[}scalar{]}
Scale, also equals the mode. Should be \textgreater{}= 0.

\item[{size}] \leavevmode{[}int or tuple of ints, optional{]}
Shape of the output. Default is None, in which case a single
value is returned.

\end{description}

The probability density function for the Rayleigh distribution is
\begin{gather}
\begin{split}P(x;scale) = \frac{x}{scale^2}e^{\frac{-x^2}{2 \cdotp scale^2}}\end{split}\notag
\end{gather}
The Rayleigh distribution arises if the wind speed and wind direction are
both gaussian variables, then the vector wind velocity forms a Rayleigh
distribution. The Rayleigh distribution is used to model the expected
output from wind turbines.

Draw values from the distribution and plot the histogram

\begin{Verbatim}[commandchars=\\\{\}]
\PYG{g+gp}{\PYGZgt{}\PYGZgt{}\PYGZgt{} }\PYG{n}{values} \PYG{o}{=} \PYG{n}{hist}\PYG{p}{(}\PYG{n}{np}\PYG{o}{.}\PYG{n}{random}\PYG{o}{.}\PYG{n}{rayleigh}\PYG{p}{(}\PYG{l+m+mi}{3}\PYG{p}{,} \PYG{l+m+mi}{100000}\PYG{p}{)}\PYG{p}{,} \PYG{n}{bins}\PYG{o}{=}\PYG{l+m+mi}{200}\PYG{p}{,} \PYG{n}{normed}\PYG{o}{=}\PYG{n+nb+bp}{True}\PYG{p}{)}
\end{Verbatim}

Wave heights tend to follow a Rayleigh distribution. If the mean wave
height is 1 meter, what fraction of waves are likely to be larger than 3
meters?

\begin{Verbatim}[commandchars=\\\{\}]
\PYG{g+gp}{\PYGZgt{}\PYGZgt{}\PYGZgt{} }\PYG{n}{meanvalue} \PYG{o}{=} \PYG{l+m+mi}{1}
\PYG{g+gp}{\PYGZgt{}\PYGZgt{}\PYGZgt{} }\PYG{n}{modevalue} \PYG{o}{=} \PYG{n}{np}\PYG{o}{.}\PYG{n}{sqrt}\PYG{p}{(}\PYG{l+m+mi}{2} \PYG{o}{/} \PYG{n}{np}\PYG{o}{.}\PYG{n}{pi}\PYG{p}{)} \PYG{o}{*} \PYG{n}{meanvalue}
\PYG{g+gp}{\PYGZgt{}\PYGZgt{}\PYGZgt{} }\PYG{n}{s} \PYG{o}{=} \PYG{n}{np}\PYG{o}{.}\PYG{n}{random}\PYG{o}{.}\PYG{n}{rayleigh}\PYG{p}{(}\PYG{n}{modevalue}\PYG{p}{,} \PYG{l+m+mi}{1000000}\PYG{p}{)}
\end{Verbatim}

The percentage of waves larger than 3 meters is:

\begin{Verbatim}[commandchars=\\\{\}]
\PYG{g+gp}{\PYGZgt{}\PYGZgt{}\PYGZgt{} }\PYG{l+m+mf}{100.}\PYG{o}{*}\PYG{n+nb}{sum}\PYG{p}{(}\PYG{n}{s}\PYG{o}{\PYGZgt{}}\PYG{l+m+mi}{3}\PYG{p}{)}\PYG{o}{/}\PYG{l+m+mf}{1000000.}
\PYG{g+go}{0.087300000000000003}
\end{Verbatim}

\end{fulllineitems}

\index{sample() (in module acsAttractorAnalysis)}

\begin{fulllineitems}
\phantomsection\label{acsAttractorAnalysis:acsAttractorAnalysis.sample}\pysiglinewithargsret{\code{acsAttractorAnalysis.}\bfcode{sample}}{}{}
random\_sample(size=None)

Return random floats in the half-open interval {[}0.0, 1.0).

Results are from the ``continuous uniform'' distribution over the
stated interval.  To sample \(Unif[a, b), b > a\) multiply
the output of \emph{random\_sample} by \emph{(b-a)} and add \emph{a}:

\begin{Verbatim}[commandchars=\\\{\}]
\PYG{p}{(}\PYG{n}{b} \PYG{o}{\PYGZhy{}} \PYG{n}{a}\PYG{p}{)} \PYG{o}{*} \PYG{n}{random\PYGZus{}sample}\PYG{p}{(}\PYG{p}{)} \PYG{o}{+} \PYG{n}{a}
\end{Verbatim}
\begin{description}
\item[{size}] \leavevmode{[}int or tuple of ints, optional{]}
Defines the shape of the returned array of random floats. If None
(the default), returns a single float.

\end{description}
\begin{description}
\item[{out}] \leavevmode{[}float or ndarray of floats{]}
Array of random floats of shape \emph{size} (unless \code{size=None}, in which
case a single float is returned).

\end{description}

\begin{Verbatim}[commandchars=\\\{\}]
\PYG{g+gp}{\PYGZgt{}\PYGZgt{}\PYGZgt{} }\PYG{n}{np}\PYG{o}{.}\PYG{n}{random}\PYG{o}{.}\PYG{n}{random\PYGZus{}sample}\PYG{p}{(}\PYG{p}{)}
\PYG{g+go}{0.47108547995356098}
\PYG{g+gp}{\PYGZgt{}\PYGZgt{}\PYGZgt{} }\PYG{n+nb}{type}\PYG{p}{(}\PYG{n}{np}\PYG{o}{.}\PYG{n}{random}\PYG{o}{.}\PYG{n}{random\PYGZus{}sample}\PYG{p}{(}\PYG{p}{)}\PYG{p}{)}
\PYG{g+go}{\PYGZlt{}type \PYGZsq{}float\PYGZsq{}\PYGZgt{}}
\PYG{g+gp}{\PYGZgt{}\PYGZgt{}\PYGZgt{} }\PYG{n}{np}\PYG{o}{.}\PYG{n}{random}\PYG{o}{.}\PYG{n}{random\PYGZus{}sample}\PYG{p}{(}\PYG{p}{(}\PYG{l+m+mi}{5}\PYG{p}{,}\PYG{p}{)}\PYG{p}{)}
\PYG{g+go}{array([ 0.30220482,  0.86820401,  0.1654503 ,  0.11659149,  0.54323428])}
\end{Verbatim}

Three-by-two array of random numbers from {[}-5, 0):

\begin{Verbatim}[commandchars=\\\{\}]
\PYG{g+gp}{\PYGZgt{}\PYGZgt{}\PYGZgt{} }\PYG{l+m+mi}{5} \PYG{o}{*} \PYG{n}{np}\PYG{o}{.}\PYG{n}{random}\PYG{o}{.}\PYG{n}{random\PYGZus{}sample}\PYG{p}{(}\PYG{p}{(}\PYG{l+m+mi}{3}\PYG{p}{,} \PYG{l+m+mi}{2}\PYG{p}{)}\PYG{p}{)} \PYG{o}{\PYGZhy{}} \PYG{l+m+mi}{5}
\PYG{g+go}{array([[\PYGZhy{}3.99149989, \PYGZhy{}0.52338984],}
\PYG{g+go}{       [\PYGZhy{}2.99091858, \PYGZhy{}0.79479508],}
\PYG{g+go}{       [\PYGZhy{}1.23204345, \PYGZhy{}1.75224494]])}
\end{Verbatim}

\end{fulllineitems}

\index{seed() (in module acsAttractorAnalysis)}

\begin{fulllineitems}
\phantomsection\label{acsAttractorAnalysis:acsAttractorAnalysis.seed}\pysiglinewithargsret{\code{acsAttractorAnalysis.}\bfcode{seed}}{\emph{seed=None}}{}
Seed the generator.

This method is called when \emph{RandomState} is initialized. It can be
called again to re-seed the generator. For details, see \emph{RandomState}.
\begin{description}
\item[{seed}] \leavevmode{[}int or array\_like, optional{]}
Seed for \emph{RandomState}.

\end{description}

RandomState

\end{fulllineitems}

\index{set\_state() (in module acsAttractorAnalysis)}

\begin{fulllineitems}
\phantomsection\label{acsAttractorAnalysis:acsAttractorAnalysis.set_state}\pysiglinewithargsret{\code{acsAttractorAnalysis.}\bfcode{set\_state}}{\emph{state}}{}
Set the internal state of the generator from a tuple.

For use if one has reason to manually (re-)set the internal state of the
``Mersenne Twister''{\color{red}\bfseries{}{[}1{]}\_} pseudo-random number generating algorithm.
\begin{description}
\item[{state}] \leavevmode{[}tuple(str, ndarray of 624 uints, int, int, float){]}
The \emph{state} tuple has the following items:
\begin{enumerate}
\item {} 
the string `MT19937', specifying the Mersenne Twister algorithm.

\item {} 
a 1-D array of 624 unsigned integers \code{keys}.

\item {} 
an integer \code{pos}.

\item {} 
an integer \code{has\_gauss}.

\item {} 
a float \code{cached\_gaussian}.

\end{enumerate}

\end{description}
\begin{description}
\item[{out}] \leavevmode{[}None{]}
Returns `None' on success.

\end{description}

get\_state

\emph{set\_state} and \emph{get\_state} are not needed to work with any of the
random distributions in NumPy. If the internal state is manually altered,
the user should know exactly what he/she is doing.

For backwards compatibility, the form (str, array of 624 uints, int) is
also accepted although it is missing some information about the cached
Gaussian value: \code{state = ('MT19937', keys, pos)}.

\end{fulllineitems}

\index{shuffle() (in module acsAttractorAnalysis)}

\begin{fulllineitems}
\phantomsection\label{acsAttractorAnalysis:acsAttractorAnalysis.shuffle}\pysiglinewithargsret{\code{acsAttractorAnalysis.}\bfcode{shuffle}}{\emph{x}}{}
Modify a sequence in-place by shuffling its contents.
\begin{description}
\item[{x}] \leavevmode{[}array\_like{]}
The array or list to be shuffled.

\end{description}

None

\begin{Verbatim}[commandchars=\\\{\}]
\PYG{g+gp}{\PYGZgt{}\PYGZgt{}\PYGZgt{} }\PYG{n}{arr} \PYG{o}{=} \PYG{n}{np}\PYG{o}{.}\PYG{n}{arange}\PYG{p}{(}\PYG{l+m+mi}{10}\PYG{p}{)}
\PYG{g+gp}{\PYGZgt{}\PYGZgt{}\PYGZgt{} }\PYG{n}{np}\PYG{o}{.}\PYG{n}{random}\PYG{o}{.}\PYG{n}{shuffle}\PYG{p}{(}\PYG{n}{arr}\PYG{p}{)}
\PYG{g+gp}{\PYGZgt{}\PYGZgt{}\PYGZgt{} }\PYG{n}{arr}
\PYG{g+go}{[1 7 5 2 9 4 3 6 0 8]}
\end{Verbatim}

This function only shuffles the array along the first index of a
multi-dimensional array:

\begin{Verbatim}[commandchars=\\\{\}]
\PYG{g+gp}{\PYGZgt{}\PYGZgt{}\PYGZgt{} }\PYG{n}{arr} \PYG{o}{=} \PYG{n}{np}\PYG{o}{.}\PYG{n}{arange}\PYG{p}{(}\PYG{l+m+mi}{9}\PYG{p}{)}\PYG{o}{.}\PYG{n}{reshape}\PYG{p}{(}\PYG{p}{(}\PYG{l+m+mi}{3}\PYG{p}{,} \PYG{l+m+mi}{3}\PYG{p}{)}\PYG{p}{)}
\PYG{g+gp}{\PYGZgt{}\PYGZgt{}\PYGZgt{} }\PYG{n}{np}\PYG{o}{.}\PYG{n}{random}\PYG{o}{.}\PYG{n}{shuffle}\PYG{p}{(}\PYG{n}{arr}\PYG{p}{)}
\PYG{g+gp}{\PYGZgt{}\PYGZgt{}\PYGZgt{} }\PYG{n}{arr}
\PYG{g+go}{array([[3, 4, 5],}
\PYG{g+go}{       [6, 7, 8],}
\PYG{g+go}{       [0, 1, 2]])}
\end{Verbatim}

\end{fulllineitems}

\index{standard\_cauchy() (in module acsAttractorAnalysis)}

\begin{fulllineitems}
\phantomsection\label{acsAttractorAnalysis:acsAttractorAnalysis.standard_cauchy}\pysiglinewithargsret{\code{acsAttractorAnalysis.}\bfcode{standard\_cauchy}}{\emph{size=None}}{}
Standard Cauchy distribution with mode = 0.

Also known as the Lorentz distribution.
\begin{description}
\item[{size}] \leavevmode{[}int or tuple of ints{]}
Shape of the output.

\end{description}
\begin{description}
\item[{samples}] \leavevmode{[}ndarray or scalar{]}
The drawn samples.

\end{description}

The probability density function for the full Cauchy distribution is
\begin{gather}
\begin{split}P(x; x_0, \gamma) = \frac{1}{\pi \gamma \bigl[ 1+
(\frac{x-x_0}{\gamma})^2 \bigr] }\end{split}\notag
\end{gather}
and the Standard Cauchy distribution just sets \(x_0=0\) and
\(\gamma=1\)

The Cauchy distribution arises in the solution to the driven harmonic
oscillator problem, and also describes spectral line broadening. It
also describes the distribution of values at which a line tilted at
a random angle will cut the x axis.

When studying hypothesis tests that assume normality, seeing how the
tests perform on data from a Cauchy distribution is a good indicator of
their sensitivity to a heavy-tailed distribution, since the Cauchy looks
very much like a Gaussian distribution, but with heavier tails.

Draw samples and plot the distribution:

\begin{Verbatim}[commandchars=\\\{\}]
\PYG{g+gp}{\PYGZgt{}\PYGZgt{}\PYGZgt{} }\PYG{n}{s} \PYG{o}{=} \PYG{n}{np}\PYG{o}{.}\PYG{n}{random}\PYG{o}{.}\PYG{n}{standard\PYGZus{}cauchy}\PYG{p}{(}\PYG{l+m+mi}{1000000}\PYG{p}{)}
\PYG{g+gp}{\PYGZgt{}\PYGZgt{}\PYGZgt{} }\PYG{n}{s} \PYG{o}{=} \PYG{n}{s}\PYG{p}{[}\PYG{p}{(}\PYG{n}{s}\PYG{o}{\PYGZgt{}}\PYG{o}{\PYGZhy{}}\PYG{l+m+mi}{25}\PYG{p}{)} \PYG{o}{\PYGZam{}} \PYG{p}{(}\PYG{n}{s}\PYG{o}{\PYGZlt{}}\PYG{l+m+mi}{25}\PYG{p}{)}\PYG{p}{]}  \PYG{c}{\PYGZsh{} truncate distribution so it plots well}
\PYG{g+gp}{\PYGZgt{}\PYGZgt{}\PYGZgt{} }\PYG{n}{plt}\PYG{o}{.}\PYG{n}{hist}\PYG{p}{(}\PYG{n}{s}\PYG{p}{,} \PYG{n}{bins}\PYG{o}{=}\PYG{l+m+mi}{100}\PYG{p}{)}
\PYG{g+gp}{\PYGZgt{}\PYGZgt{}\PYGZgt{} }\PYG{n}{plt}\PYG{o}{.}\PYG{n}{show}\PYG{p}{(}\PYG{p}{)}
\end{Verbatim}

\end{fulllineitems}

\index{standard\_exponential() (in module acsAttractorAnalysis)}

\begin{fulllineitems}
\phantomsection\label{acsAttractorAnalysis:acsAttractorAnalysis.standard_exponential}\pysiglinewithargsret{\code{acsAttractorAnalysis.}\bfcode{standard\_exponential}}{\emph{size=None}}{}
Draw samples from the standard exponential distribution.

\emph{standard\_exponential} is identical to the exponential distribution
with a scale parameter of 1.
\begin{description}
\item[{size}] \leavevmode{[}int or tuple of ints{]}
Shape of the output.

\end{description}
\begin{description}
\item[{out}] \leavevmode{[}float or ndarray{]}
Drawn samples.

\end{description}

Output a 3x8000 array:

\begin{Verbatim}[commandchars=\\\{\}]
\PYG{g+gp}{\PYGZgt{}\PYGZgt{}\PYGZgt{} }\PYG{n}{n} \PYG{o}{=} \PYG{n}{np}\PYG{o}{.}\PYG{n}{random}\PYG{o}{.}\PYG{n}{standard\PYGZus{}exponential}\PYG{p}{(}\PYG{p}{(}\PYG{l+m+mi}{3}\PYG{p}{,} \PYG{l+m+mi}{8000}\PYG{p}{)}\PYG{p}{)}
\end{Verbatim}

\end{fulllineitems}

\index{standard\_gamma() (in module acsAttractorAnalysis)}

\begin{fulllineitems}
\phantomsection\label{acsAttractorAnalysis:acsAttractorAnalysis.standard_gamma}\pysiglinewithargsret{\code{acsAttractorAnalysis.}\bfcode{standard\_gamma}}{\emph{shape}, \emph{size=None}}{}
Draw samples from a Standard Gamma distribution.

Samples are drawn from a Gamma distribution with specified parameters,
shape (sometimes designated ``k'') and scale=1.
\begin{description}
\item[{shape}] \leavevmode{[}float{]}
Parameter, should be \textgreater{} 0.

\item[{size}] \leavevmode{[}int or tuple of ints{]}
Output shape.  If the given shape is, e.g., \code{(m, n, k)}, then
\code{m * n * k} samples are drawn.

\end{description}
\begin{description}
\item[{samples}] \leavevmode{[}ndarray or scalar{]}
The drawn samples.

\end{description}
\begin{description}
\item[{scipy.stats.distributions.gamma}] \leavevmode{[}probability density function,{]}
distribution or cumulative density function, etc.

\end{description}

The probability density for the Gamma distribution is
\begin{gather}
\begin{split}p(x) = x^{k-1}\frac{e^{-x/\theta}}{\theta^k\Gamma(k)},\end{split}\notag
\end{gather}
where \(k\) is the shape and \(\theta\) the scale,
and \(\Gamma\) is the Gamma function.

The Gamma distribution is often used to model the times to failure of
electronic components, and arises naturally in processes for which the
waiting times between Poisson distributed events are relevant.

Draw samples from the distribution:

\begin{Verbatim}[commandchars=\\\{\}]
\PYG{g+gp}{\PYGZgt{}\PYGZgt{}\PYGZgt{} }\PYG{n}{shape}\PYG{p}{,} \PYG{n}{scale} \PYG{o}{=} \PYG{l+m+mf}{2.}\PYG{p}{,} \PYG{l+m+mf}{1.} \PYG{c}{\PYGZsh{} mean and width}
\PYG{g+gp}{\PYGZgt{}\PYGZgt{}\PYGZgt{} }\PYG{n}{s} \PYG{o}{=} \PYG{n}{np}\PYG{o}{.}\PYG{n}{random}\PYG{o}{.}\PYG{n}{standard\PYGZus{}gamma}\PYG{p}{(}\PYG{n}{shape}\PYG{p}{,} \PYG{l+m+mi}{1000000}\PYG{p}{)}
\end{Verbatim}

Display the histogram of the samples, along with
the probability density function:

\begin{Verbatim}[commandchars=\\\{\}]
\PYG{g+gp}{\PYGZgt{}\PYGZgt{}\PYGZgt{} }\PYG{k+kn}{import} \PYG{n+nn}{matplotlib.pyplot} \PYG{k+kn}{as} \PYG{n+nn}{plt}
\PYG{g+gp}{\PYGZgt{}\PYGZgt{}\PYGZgt{} }\PYG{k+kn}{import} \PYG{n+nn}{scipy.special} \PYG{k+kn}{as} \PYG{n+nn}{sps}
\PYG{g+gp}{\PYGZgt{}\PYGZgt{}\PYGZgt{} }\PYG{n}{count}\PYG{p}{,} \PYG{n}{bins}\PYG{p}{,} \PYG{n}{ignored} \PYG{o}{=} \PYG{n}{plt}\PYG{o}{.}\PYG{n}{hist}\PYG{p}{(}\PYG{n}{s}\PYG{p}{,} \PYG{l+m+mi}{50}\PYG{p}{,} \PYG{n}{normed}\PYG{o}{=}\PYG{n+nb+bp}{True}\PYG{p}{)}
\PYG{g+gp}{\PYGZgt{}\PYGZgt{}\PYGZgt{} }\PYG{n}{y} \PYG{o}{=} \PYG{n}{bins}\PYG{o}{*}\PYG{o}{*}\PYG{p}{(}\PYG{n}{shape}\PYG{o}{\PYGZhy{}}\PYG{l+m+mi}{1}\PYG{p}{)} \PYG{o}{*} \PYG{p}{(}\PYG{p}{(}\PYG{n}{np}\PYG{o}{.}\PYG{n}{exp}\PYG{p}{(}\PYG{o}{\PYGZhy{}}\PYG{n}{bins}\PYG{o}{/}\PYG{n}{scale}\PYG{p}{)}\PYG{p}{)}\PYG{o}{/} \PYGZbs{}
\PYG{g+gp}{... }                      \PYG{p}{(}\PYG{n}{sps}\PYG{o}{.}\PYG{n}{gamma}\PYG{p}{(}\PYG{n}{shape}\PYG{p}{)} \PYG{o}{*} \PYG{n}{scale}\PYG{o}{*}\PYG{o}{*}\PYG{n}{shape}\PYG{p}{)}\PYG{p}{)}
\PYG{g+gp}{\PYGZgt{}\PYGZgt{}\PYGZgt{} }\PYG{n}{plt}\PYG{o}{.}\PYG{n}{plot}\PYG{p}{(}\PYG{n}{bins}\PYG{p}{,} \PYG{n}{y}\PYG{p}{,} \PYG{n}{linewidth}\PYG{o}{=}\PYG{l+m+mi}{2}\PYG{p}{,} \PYG{n}{color}\PYG{o}{=}\PYG{l+s}{\PYGZsq{}}\PYG{l+s}{r}\PYG{l+s}{\PYGZsq{}}\PYG{p}{)}
\PYG{g+gp}{\PYGZgt{}\PYGZgt{}\PYGZgt{} }\PYG{n}{plt}\PYG{o}{.}\PYG{n}{show}\PYG{p}{(}\PYG{p}{)}
\end{Verbatim}

\end{fulllineitems}

\index{standard\_normal() (in module acsAttractorAnalysis)}

\begin{fulllineitems}
\phantomsection\label{acsAttractorAnalysis:acsAttractorAnalysis.standard_normal}\pysiglinewithargsret{\code{acsAttractorAnalysis.}\bfcode{standard\_normal}}{\emph{size=None}}{}
Returns samples from a Standard Normal distribution (mean=0, stdev=1).
\begin{description}
\item[{size}] \leavevmode{[}int or tuple of ints, optional{]}
Output shape. Default is None, in which case a single value is
returned.

\end{description}
\begin{description}
\item[{out}] \leavevmode{[}float or ndarray{]}
Drawn samples.

\end{description}

\begin{Verbatim}[commandchars=\\\{\}]
\PYG{g+gp}{\PYGZgt{}\PYGZgt{}\PYGZgt{} }\PYG{n}{s} \PYG{o}{=} \PYG{n}{np}\PYG{o}{.}\PYG{n}{random}\PYG{o}{.}\PYG{n}{standard\PYGZus{}normal}\PYG{p}{(}\PYG{l+m+mi}{8000}\PYG{p}{)}
\PYG{g+gp}{\PYGZgt{}\PYGZgt{}\PYGZgt{} }\PYG{n}{s}
\PYG{g+go}{array([ 0.6888893 ,  0.78096262, \PYGZhy{}0.89086505, ...,  0.49876311, \PYGZsh{}random}
\PYG{g+go}{       \PYGZhy{}0.38672696, \PYGZhy{}0.4685006 ])                               \PYGZsh{}random}
\PYG{g+gp}{\PYGZgt{}\PYGZgt{}\PYGZgt{} }\PYG{n}{s}\PYG{o}{.}\PYG{n}{shape}
\PYG{g+go}{(8000,)}
\PYG{g+gp}{\PYGZgt{}\PYGZgt{}\PYGZgt{} }\PYG{n}{s} \PYG{o}{=} \PYG{n}{np}\PYG{o}{.}\PYG{n}{random}\PYG{o}{.}\PYG{n}{standard\PYGZus{}normal}\PYG{p}{(}\PYG{n}{size}\PYG{o}{=}\PYG{p}{(}\PYG{l+m+mi}{3}\PYG{p}{,} \PYG{l+m+mi}{4}\PYG{p}{,} \PYG{l+m+mi}{2}\PYG{p}{)}\PYG{p}{)}
\PYG{g+gp}{\PYGZgt{}\PYGZgt{}\PYGZgt{} }\PYG{n}{s}\PYG{o}{.}\PYG{n}{shape}
\PYG{g+go}{(3, 4, 2)}
\end{Verbatim}

\end{fulllineitems}

\index{standard\_t() (in module acsAttractorAnalysis)}

\begin{fulllineitems}
\phantomsection\label{acsAttractorAnalysis:acsAttractorAnalysis.standard_t}\pysiglinewithargsret{\code{acsAttractorAnalysis.}\bfcode{standard\_t}}{\emph{df}, \emph{size=None}}{}
Standard Student's t distribution with df degrees of freedom.

A special case of the hyperbolic distribution.
As \emph{df} gets large, the result resembles that of the standard normal
distribution (\emph{standard\_normal}).
\begin{description}
\item[{df}] \leavevmode{[}int{]}
Degrees of freedom, should be \textgreater{} 0.

\item[{size}] \leavevmode{[}int or tuple of ints, optional{]}
Output shape. Default is None, in which case a single value is
returned.

\end{description}
\begin{description}
\item[{samples}] \leavevmode{[}ndarray or scalar{]}
Drawn samples.

\end{description}

The probability density function for the t distribution is
\begin{gather}
\begin{split}P(x, df) = \frac{\Gamma(\frac{df+1}{2})}{\sqrt{\pi df}
\Gamma(\frac{df}{2})}\Bigl( 1+\frac{x^2}{df} \Bigr)^{-(df+1)/2}\end{split}\notag
\end{gather}
The t test is based on an assumption that the data come from a Normal
distribution. The t test provides a way to test whether the sample mean
(that is the mean calculated from the data) is a good estimate of the true
mean.

The derivation of the t-distribution was forst published in 1908 by William
Gisset while working for the Guinness Brewery in Dublin. Due to proprietary
issues, he had to publish under a pseudonym, and so he used the name
Student.

From Dalgaard page 83 {\color{red}\bfseries{}{[}1{]}\_}, suppose the daily energy intake for 11
women in Kj is:

\begin{Verbatim}[commandchars=\\\{\}]
\PYG{g+gp}{\PYGZgt{}\PYGZgt{}\PYGZgt{} }\PYG{n}{intake} \PYG{o}{=} \PYG{n}{np}\PYG{o}{.}\PYG{n}{array}\PYG{p}{(}\PYG{p}{[}\PYG{l+m+mf}{5260.}\PYG{p}{,} \PYG{l+m+mi}{5470}\PYG{p}{,} \PYG{l+m+mi}{5640}\PYG{p}{,} \PYG{l+m+mi}{6180}\PYG{p}{,} \PYG{l+m+mi}{6390}\PYG{p}{,} \PYG{l+m+mi}{6515}\PYG{p}{,} \PYG{l+m+mi}{6805}\PYG{p}{,} \PYG{l+m+mi}{7515}\PYG{p}{,} \PYGZbs{}
\PYG{g+gp}{... }                   \PYG{l+m+mi}{7515}\PYG{p}{,} \PYG{l+m+mi}{8230}\PYG{p}{,} \PYG{l+m+mi}{8770}\PYG{p}{]}\PYG{p}{)}
\end{Verbatim}

Does their energy intake deviate systematically from the recommended
value of 7725 kJ?

We have 10 degrees of freedom, so is the sample mean within 95\% of the
recommended value?

\begin{Verbatim}[commandchars=\\\{\}]
\PYG{g+gp}{\PYGZgt{}\PYGZgt{}\PYGZgt{} }\PYG{n}{s} \PYG{o}{=} \PYG{n}{np}\PYG{o}{.}\PYG{n}{random}\PYG{o}{.}\PYG{n}{standard\PYGZus{}t}\PYG{p}{(}\PYG{l+m+mi}{10}\PYG{p}{,} \PYG{n}{size}\PYG{o}{=}\PYG{l+m+mi}{100000}\PYG{p}{)}
\PYG{g+gp}{\PYGZgt{}\PYGZgt{}\PYGZgt{} }\PYG{n}{np}\PYG{o}{.}\PYG{n}{mean}\PYG{p}{(}\PYG{n}{intake}\PYG{p}{)}
\PYG{g+go}{6753.636363636364}
\PYG{g+gp}{\PYGZgt{}\PYGZgt{}\PYGZgt{} }\PYG{n}{intake}\PYG{o}{.}\PYG{n}{std}\PYG{p}{(}\PYG{n}{ddof}\PYG{o}{=}\PYG{l+m+mi}{1}\PYG{p}{)}
\PYG{g+go}{1142.1232221373727}
\end{Verbatim}

Calculate the t statistic, setting the ddof parameter to the unbiased
value so the divisor in the standard deviation will be degrees of
freedom, N-1.

\begin{Verbatim}[commandchars=\\\{\}]
\PYG{g+gp}{\PYGZgt{}\PYGZgt{}\PYGZgt{} }\PYG{n}{t} \PYG{o}{=} \PYG{p}{(}\PYG{n}{np}\PYG{o}{.}\PYG{n}{mean}\PYG{p}{(}\PYG{n}{intake}\PYG{p}{)}\PYG{o}{\PYGZhy{}}\PYG{l+m+mi}{7725}\PYG{p}{)}\PYG{o}{/}\PYG{p}{(}\PYG{n}{intake}\PYG{o}{.}\PYG{n}{std}\PYG{p}{(}\PYG{n}{ddof}\PYG{o}{=}\PYG{l+m+mi}{1}\PYG{p}{)}\PYG{o}{/}\PYG{n}{np}\PYG{o}{.}\PYG{n}{sqrt}\PYG{p}{(}\PYG{n+nb}{len}\PYG{p}{(}\PYG{n}{intake}\PYG{p}{)}\PYG{p}{)}\PYG{p}{)}
\PYG{g+gp}{\PYGZgt{}\PYGZgt{}\PYGZgt{} }\PYG{k+kn}{import} \PYG{n+nn}{matplotlib.pyplot} \PYG{k+kn}{as} \PYG{n+nn}{plt}
\PYG{g+gp}{\PYGZgt{}\PYGZgt{}\PYGZgt{} }\PYG{n}{h} \PYG{o}{=} \PYG{n}{plt}\PYG{o}{.}\PYG{n}{hist}\PYG{p}{(}\PYG{n}{s}\PYG{p}{,} \PYG{n}{bins}\PYG{o}{=}\PYG{l+m+mi}{100}\PYG{p}{,} \PYG{n}{normed}\PYG{o}{=}\PYG{n+nb+bp}{True}\PYG{p}{)}
\end{Verbatim}

For a one-sided t-test, how far out in the distribution does the t
statistic appear?

\begin{Verbatim}[commandchars=\\\{\}]
\PYG{g+gp}{\PYGZgt{}\PYGZgt{}\PYGZgt{} }\PYG{o}{\PYGZgt{}\PYGZgt{}}\PYG{o}{\PYGZgt{}} \PYG{n}{np}\PYG{o}{.}\PYG{n}{sum}\PYG{p}{(}\PYG{n}{s}\PYG{o}{\PYGZlt{}}\PYG{n}{t}\PYG{p}{)} \PYG{o}{/} \PYG{n+nb}{float}\PYG{p}{(}\PYG{n+nb}{len}\PYG{p}{(}\PYG{n}{s}\PYG{p}{)}\PYG{p}{)}
\PYG{g+go}{0.0090699999999999999  \PYGZsh{}random}
\end{Verbatim}

So the p-value is about 0.009, which says the null hypothesis has a
probability of about 99\% of being true.

\end{fulllineitems}

\index{triangular() (in module acsAttractorAnalysis)}

\begin{fulllineitems}
\phantomsection\label{acsAttractorAnalysis:acsAttractorAnalysis.triangular}\pysiglinewithargsret{\code{acsAttractorAnalysis.}\bfcode{triangular}}{\emph{left}, \emph{mode}, \emph{right}, \emph{size=None}}{}
Draw samples from the triangular distribution.

The triangular distribution is a continuous probability distribution with
lower limit left, peak at mode, and upper limit right. Unlike the other
distributions, these parameters directly define the shape of the pdf.
\begin{description}
\item[{left}] \leavevmode{[}scalar{]}
Lower limit.

\item[{mode}] \leavevmode{[}scalar{]}
The value where the peak of the distribution occurs.
The value should fulfill the condition \code{left \textless{}= mode \textless{}= right}.

\item[{right}] \leavevmode{[}scalar{]}
Upper limit, should be larger than \emph{left}.

\item[{size}] \leavevmode{[}int or tuple of ints, optional{]}
Output shape. Default is None, in which case a single value is
returned.

\end{description}
\begin{description}
\item[{samples}] \leavevmode{[}ndarray or scalar{]}
The returned samples all lie in the interval {[}left, right{]}.

\end{description}

The probability density function for the Triangular distribution is
\begin{gather}
\begin{split}P(x;l, m, r) = \begin{cases}
\frac{2(x-l)}{(r-l)(m-l)}& \text{for $l \leq x \leq m$},\\
\frac{2(m-x)}{(r-l)(r-m)}& \text{for $m \leq x \leq r$},\\
0& \text{otherwise}.
\end{cases}\end{split}\notag
\end{gather}
The triangular distribution is often used in ill-defined problems where the
underlying distribution is not known, but some knowledge of the limits and
mode exists. Often it is used in simulations.

Draw values from the distribution and plot the histogram:

\begin{Verbatim}[commandchars=\\\{\}]
\PYG{g+gp}{\PYGZgt{}\PYGZgt{}\PYGZgt{} }\PYG{k+kn}{import} \PYG{n+nn}{matplotlib.pyplot} \PYG{k+kn}{as} \PYG{n+nn}{plt}
\PYG{g+gp}{\PYGZgt{}\PYGZgt{}\PYGZgt{} }\PYG{n}{h} \PYG{o}{=} \PYG{n}{plt}\PYG{o}{.}\PYG{n}{hist}\PYG{p}{(}\PYG{n}{np}\PYG{o}{.}\PYG{n}{random}\PYG{o}{.}\PYG{n}{triangular}\PYG{p}{(}\PYG{o}{\PYGZhy{}}\PYG{l+m+mi}{3}\PYG{p}{,} \PYG{l+m+mi}{0}\PYG{p}{,} \PYG{l+m+mi}{8}\PYG{p}{,} \PYG{l+m+mi}{100000}\PYG{p}{)}\PYG{p}{,} \PYG{n}{bins}\PYG{o}{=}\PYG{l+m+mi}{200}\PYG{p}{,}
\PYG{g+gp}{... }             \PYG{n}{normed}\PYG{o}{=}\PYG{n+nb+bp}{True}\PYG{p}{)}
\PYG{g+gp}{\PYGZgt{}\PYGZgt{}\PYGZgt{} }\PYG{n}{plt}\PYG{o}{.}\PYG{n}{show}\PYG{p}{(}\PYG{p}{)}
\end{Verbatim}

\end{fulllineitems}

\index{uniform() (in module acsAttractorAnalysis)}

\begin{fulllineitems}
\phantomsection\label{acsAttractorAnalysis:acsAttractorAnalysis.uniform}\pysiglinewithargsret{\code{acsAttractorAnalysis.}\bfcode{uniform}}{\emph{low=0.0}, \emph{high=1.0}, \emph{size=1}}{}
Draw samples from a uniform distribution.

Samples are uniformly distributed over the half-open interval
\code{{[}low, high)} (includes low, but excludes high).  In other words,
any value within the given interval is equally likely to be drawn
by \emph{uniform}.
\begin{description}
\item[{low}] \leavevmode{[}float, optional{]}
Lower boundary of the output interval.  All values generated will be
greater than or equal to low.  The default value is 0.

\item[{high}] \leavevmode{[}float{]}
Upper boundary of the output interval.  All values generated will be
less than high.  The default value is 1.0.

\item[{size}] \leavevmode{[}int or tuple of ints, optional{]}
Shape of output.  If the given size is, for example, (m,n,k),
m*n*k samples are generated.  If no shape is specified, a single sample
is returned.

\end{description}
\begin{description}
\item[{out}] \leavevmode{[}ndarray{]}
Drawn samples, with shape \emph{size}.

\end{description}

randint : Discrete uniform distribution, yielding integers.
random\_integers : Discrete uniform distribution over the closed
\begin{quote}

interval \code{{[}low, high{]}}.
\end{quote}

random\_sample : Floats uniformly distributed over \code{{[}0, 1)}.
random : Alias for \emph{random\_sample}.
rand : Convenience function that accepts dimensions as input, e.g.,
\begin{quote}

\code{rand(2,2)} would generate a 2-by-2 array of floats,
uniformly distributed over \code{{[}0, 1)}.
\end{quote}

The probability density function of the uniform distribution is
\begin{gather}
\begin{split}p(x) = \frac{1}{b - a}\end{split}\notag
\end{gather}
anywhere within the interval \code{{[}a, b)}, and zero elsewhere.

Draw samples from the distribution:

\begin{Verbatim}[commandchars=\\\{\}]
\PYG{g+gp}{\PYGZgt{}\PYGZgt{}\PYGZgt{} }\PYG{n}{s} \PYG{o}{=} \PYG{n}{np}\PYG{o}{.}\PYG{n}{random}\PYG{o}{.}\PYG{n}{uniform}\PYG{p}{(}\PYG{o}{\PYGZhy{}}\PYG{l+m+mi}{1}\PYG{p}{,}\PYG{l+m+mi}{0}\PYG{p}{,}\PYG{l+m+mi}{1000}\PYG{p}{)}
\end{Verbatim}

All values are within the given interval:

\begin{Verbatim}[commandchars=\\\{\}]
\PYG{g+gp}{\PYGZgt{}\PYGZgt{}\PYGZgt{} }\PYG{n}{np}\PYG{o}{.}\PYG{n}{all}\PYG{p}{(}\PYG{n}{s} \PYG{o}{\PYGZgt{}}\PYG{o}{=} \PYG{o}{\PYGZhy{}}\PYG{l+m+mi}{1}\PYG{p}{)}
\PYG{g+go}{True}
\PYG{g+gp}{\PYGZgt{}\PYGZgt{}\PYGZgt{} }\PYG{n}{np}\PYG{o}{.}\PYG{n}{all}\PYG{p}{(}\PYG{n}{s} \PYG{o}{\PYGZlt{}} \PYG{l+m+mi}{0}\PYG{p}{)}
\PYG{g+go}{True}
\end{Verbatim}

Display the histogram of the samples, along with the
probability density function:

\begin{Verbatim}[commandchars=\\\{\}]
\PYG{g+gp}{\PYGZgt{}\PYGZgt{}\PYGZgt{} }\PYG{k+kn}{import} \PYG{n+nn}{matplotlib.pyplot} \PYG{k+kn}{as} \PYG{n+nn}{plt}
\PYG{g+gp}{\PYGZgt{}\PYGZgt{}\PYGZgt{} }\PYG{n}{count}\PYG{p}{,} \PYG{n}{bins}\PYG{p}{,} \PYG{n}{ignored} \PYG{o}{=} \PYG{n}{plt}\PYG{o}{.}\PYG{n}{hist}\PYG{p}{(}\PYG{n}{s}\PYG{p}{,} \PYG{l+m+mi}{15}\PYG{p}{,} \PYG{n}{normed}\PYG{o}{=}\PYG{n+nb+bp}{True}\PYG{p}{)}
\PYG{g+gp}{\PYGZgt{}\PYGZgt{}\PYGZgt{} }\PYG{n}{plt}\PYG{o}{.}\PYG{n}{plot}\PYG{p}{(}\PYG{n}{bins}\PYG{p}{,} \PYG{n}{np}\PYG{o}{.}\PYG{n}{ones\PYGZus{}like}\PYG{p}{(}\PYG{n}{bins}\PYG{p}{)}\PYG{p}{,} \PYG{n}{linewidth}\PYG{o}{=}\PYG{l+m+mi}{2}\PYG{p}{,} \PYG{n}{color}\PYG{o}{=}\PYG{l+s}{\PYGZsq{}}\PYG{l+s}{r}\PYG{l+s}{\PYGZsq{}}\PYG{p}{)}
\PYG{g+gp}{\PYGZgt{}\PYGZgt{}\PYGZgt{} }\PYG{n}{plt}\PYG{o}{.}\PYG{n}{show}\PYG{p}{(}\PYG{p}{)}
\end{Verbatim}

\end{fulllineitems}

\index{vonmises() (in module acsAttractorAnalysis)}

\begin{fulllineitems}
\phantomsection\label{acsAttractorAnalysis:acsAttractorAnalysis.vonmises}\pysiglinewithargsret{\code{acsAttractorAnalysis.}\bfcode{vonmises}}{\emph{mu}, \emph{kappa}, \emph{size=None}}{}
Draw samples from a von Mises distribution.

Samples are drawn from a von Mises distribution with specified mode
(mu) and dispersion (kappa), on the interval {[}-pi, pi{]}.

The von Mises distribution (also known as the circular normal
distribution) is a continuous probability distribution on the unit
circle.  It may be thought of as the circular analogue of the normal
distribution.
\begin{description}
\item[{mu}] \leavevmode{[}float{]}
Mode (``center'') of the distribution.

\item[{kappa}] \leavevmode{[}float{]}
Dispersion of the distribution, has to be \textgreater{}=0.

\item[{size}] \leavevmode{[}int or tuple of int{]}
Output shape.  If the given shape is, e.g., \code{(m, n, k)}, then
\code{m * n * k} samples are drawn.

\end{description}
\begin{description}
\item[{samples}] \leavevmode{[}scalar or ndarray{]}
The returned samples, which are in the interval {[}-pi, pi{]}.

\end{description}
\begin{description}
\item[{scipy.stats.distributions.vonmises}] \leavevmode{[}probability density function,{]}
distribution, or cumulative density function, etc.

\end{description}

The probability density for the von Mises distribution is
\begin{gather}
\begin{split}p(x) = \frac{e^{\kappa cos(x-\mu)}}{2\pi I_0(\kappa)},\end{split}\notag
\end{gather}
where \(\mu\) is the mode and \(\kappa\) the dispersion,
and \(I_0(\kappa)\) is the modified Bessel function of order 0.

The von Mises is named for Richard Edler von Mises, who was born in
Austria-Hungary, in what is now the Ukraine.  He fled to the United
States in 1939 and became a professor at Harvard.  He worked in
probability theory, aerodynamics, fluid mechanics, and philosophy of
science.

Abramowitz, M. and Stegun, I. A. (ed.), \emph{Handbook of Mathematical
Functions}, New York: Dover, 1965.

von Mises, R., \emph{Mathematical Theory of Probability and Statistics},
New York: Academic Press, 1964.

Draw samples from the distribution:

\begin{Verbatim}[commandchars=\\\{\}]
\PYG{g+gp}{\PYGZgt{}\PYGZgt{}\PYGZgt{} }\PYG{n}{mu}\PYG{p}{,} \PYG{n}{kappa} \PYG{o}{=} \PYG{l+m+mf}{0.0}\PYG{p}{,} \PYG{l+m+mf}{4.0} \PYG{c}{\PYGZsh{} mean and dispersion}
\PYG{g+gp}{\PYGZgt{}\PYGZgt{}\PYGZgt{} }\PYG{n}{s} \PYG{o}{=} \PYG{n}{np}\PYG{o}{.}\PYG{n}{random}\PYG{o}{.}\PYG{n}{vonmises}\PYG{p}{(}\PYG{n}{mu}\PYG{p}{,} \PYG{n}{kappa}\PYG{p}{,} \PYG{l+m+mi}{1000}\PYG{p}{)}
\end{Verbatim}

Display the histogram of the samples, along with
the probability density function:

\begin{Verbatim}[commandchars=\\\{\}]
\PYG{g+gp}{\PYGZgt{}\PYGZgt{}\PYGZgt{} }\PYG{k+kn}{import} \PYG{n+nn}{matplotlib.pyplot} \PYG{k+kn}{as} \PYG{n+nn}{plt}
\PYG{g+gp}{\PYGZgt{}\PYGZgt{}\PYGZgt{} }\PYG{k+kn}{import} \PYG{n+nn}{scipy.special} \PYG{k+kn}{as} \PYG{n+nn}{sps}
\PYG{g+gp}{\PYGZgt{}\PYGZgt{}\PYGZgt{} }\PYG{n}{count}\PYG{p}{,} \PYG{n}{bins}\PYG{p}{,} \PYG{n}{ignored} \PYG{o}{=} \PYG{n}{plt}\PYG{o}{.}\PYG{n}{hist}\PYG{p}{(}\PYG{n}{s}\PYG{p}{,} \PYG{l+m+mi}{50}\PYG{p}{,} \PYG{n}{normed}\PYG{o}{=}\PYG{n+nb+bp}{True}\PYG{p}{)}
\PYG{g+gp}{\PYGZgt{}\PYGZgt{}\PYGZgt{} }\PYG{n}{x} \PYG{o}{=} \PYG{n}{np}\PYG{o}{.}\PYG{n}{arange}\PYG{p}{(}\PYG{o}{\PYGZhy{}}\PYG{n}{np}\PYG{o}{.}\PYG{n}{pi}\PYG{p}{,} \PYG{n}{np}\PYG{o}{.}\PYG{n}{pi}\PYG{p}{,} \PYG{l+m+mi}{2}\PYG{o}{*}\PYG{n}{np}\PYG{o}{.}\PYG{n}{pi}\PYG{o}{/}\PYG{l+m+mf}{50.}\PYG{p}{)}
\PYG{g+gp}{\PYGZgt{}\PYGZgt{}\PYGZgt{} }\PYG{n}{y} \PYG{o}{=} \PYG{o}{\PYGZhy{}}\PYG{n}{np}\PYG{o}{.}\PYG{n}{exp}\PYG{p}{(}\PYG{n}{kappa}\PYG{o}{*}\PYG{n}{np}\PYG{o}{.}\PYG{n}{cos}\PYG{p}{(}\PYG{n}{x}\PYG{o}{\PYGZhy{}}\PYG{n}{mu}\PYG{p}{)}\PYG{p}{)}\PYG{o}{/}\PYG{p}{(}\PYG{l+m+mi}{2}\PYG{o}{*}\PYG{n}{np}\PYG{o}{.}\PYG{n}{pi}\PYG{o}{*}\PYG{n}{sps}\PYG{o}{.}\PYG{n}{jn}\PYG{p}{(}\PYG{l+m+mi}{0}\PYG{p}{,}\PYG{n}{kappa}\PYG{p}{)}\PYG{p}{)}
\PYG{g+gp}{\PYGZgt{}\PYGZgt{}\PYGZgt{} }\PYG{n}{plt}\PYG{o}{.}\PYG{n}{plot}\PYG{p}{(}\PYG{n}{x}\PYG{p}{,} \PYG{n}{y}\PYG{o}{/}\PYG{n+nb}{max}\PYG{p}{(}\PYG{n}{y}\PYG{p}{)}\PYG{p}{,} \PYG{n}{linewidth}\PYG{o}{=}\PYG{l+m+mi}{2}\PYG{p}{,} \PYG{n}{color}\PYG{o}{=}\PYG{l+s}{\PYGZsq{}}\PYG{l+s}{r}\PYG{l+s}{\PYGZsq{}}\PYG{p}{)}
\PYG{g+gp}{\PYGZgt{}\PYGZgt{}\PYGZgt{} }\PYG{n}{plt}\PYG{o}{.}\PYG{n}{show}\PYG{p}{(}\PYG{p}{)}
\end{Verbatim}

\end{fulllineitems}

\index{wald() (in module acsAttractorAnalysis)}

\begin{fulllineitems}
\phantomsection\label{acsAttractorAnalysis:acsAttractorAnalysis.wald}\pysiglinewithargsret{\code{acsAttractorAnalysis.}\bfcode{wald}}{\emph{mean}, \emph{scale}, \emph{size=None}}{}
Draw samples from a Wald, or Inverse Gaussian, distribution.

As the scale approaches infinity, the distribution becomes more like a
Gaussian.

Some references claim that the Wald is an Inverse Gaussian with mean=1, but
this is by no means universal.

The Inverse Gaussian distribution was first studied in relationship to
Brownian motion. In 1956 M.C.K. Tweedie used the name Inverse Gaussian
because there is an inverse relationship between the time to cover a unit
distance and distance covered in unit time.
\begin{description}
\item[{mean}] \leavevmode{[}scalar{]}
Distribution mean, should be \textgreater{} 0.

\item[{scale}] \leavevmode{[}scalar{]}
Scale parameter, should be \textgreater{}= 0.

\item[{size}] \leavevmode{[}int or tuple of ints, optional{]}
Output shape. Default is None, in which case a single value is
returned.

\end{description}
\begin{description}
\item[{samples}] \leavevmode{[}ndarray or scalar{]}
Drawn sample, all greater than zero.

\end{description}

The probability density function for the Wald distribution is
\begin{gather}
\begin{split}P(x;mean,scale) = \sqrt{\frac{scale}{2\pi x^3}}e^
\frac{-scale(x-mean)^2}{2\cdotp mean^2x}\end{split}\notag
\end{gather}
As noted above the Inverse Gaussian distribution first arise from attempts
to model Brownian Motion. It is also a competitor to the Weibull for use in
reliability modeling and modeling stock returns and interest rate
processes.

Draw values from the distribution and plot the histogram:

\begin{Verbatim}[commandchars=\\\{\}]
\PYG{g+gp}{\PYGZgt{}\PYGZgt{}\PYGZgt{} }\PYG{k+kn}{import} \PYG{n+nn}{matplotlib.pyplot} \PYG{k+kn}{as} \PYG{n+nn}{plt}
\PYG{g+gp}{\PYGZgt{}\PYGZgt{}\PYGZgt{} }\PYG{n}{h} \PYG{o}{=} \PYG{n}{plt}\PYG{o}{.}\PYG{n}{hist}\PYG{p}{(}\PYG{n}{np}\PYG{o}{.}\PYG{n}{random}\PYG{o}{.}\PYG{n}{wald}\PYG{p}{(}\PYG{l+m+mi}{3}\PYG{p}{,} \PYG{l+m+mi}{2}\PYG{p}{,} \PYG{l+m+mi}{100000}\PYG{p}{)}\PYG{p}{,} \PYG{n}{bins}\PYG{o}{=}\PYG{l+m+mi}{200}\PYG{p}{,} \PYG{n}{normed}\PYG{o}{=}\PYG{n+nb+bp}{True}\PYG{p}{)}
\PYG{g+gp}{\PYGZgt{}\PYGZgt{}\PYGZgt{} }\PYG{n}{plt}\PYG{o}{.}\PYG{n}{show}\PYG{p}{(}\PYG{p}{)}
\end{Verbatim}

\end{fulllineitems}

\index{weibull() (in module acsAttractorAnalysis)}

\begin{fulllineitems}
\phantomsection\label{acsAttractorAnalysis:acsAttractorAnalysis.weibull}\pysiglinewithargsret{\code{acsAttractorAnalysis.}\bfcode{weibull}}{\emph{a}, \emph{size=None}}{}
Weibull distribution.

Draw samples from a 1-parameter Weibull distribution with the given
shape parameter \emph{a}.
\begin{gather}
\begin{split}X = (-ln(U))^{1/a}\end{split}\notag
\end{gather}
Here, U is drawn from the uniform distribution over (0,1{]}.

The more common 2-parameter Weibull, including a scale parameter
\(\lambda\) is just \(X = \lambda(-ln(U))^{1/a}\).
\begin{description}
\item[{a}] \leavevmode{[}float{]}
Shape of the distribution.

\item[{size}] \leavevmode{[}tuple of ints{]}
Output shape.  If the given shape is, e.g., \code{(m, n, k)}, then
\code{m * n * k} samples are drawn.

\end{description}

scipy.stats.distributions.weibull\_max
scipy.stats.distributions.weibull\_min
scipy.stats.distributions.genextreme
gumbel

The Weibull (or Type III asymptotic extreme value distribution for smallest
values, SEV Type III, or Rosin-Rammler distribution) is one of a class of
Generalized Extreme Value (GEV) distributions used in modeling extreme
value problems.  This class includes the Gumbel and Frechet distributions.

The probability density for the Weibull distribution is
\begin{gather}
\begin{split}p(x) = \frac{a}
{\lambda}(\frac{x}{\lambda})^{a-1}e^{-(x/\lambda)^a},\end{split}\notag
\end{gather}
where \(a\) is the shape and \(\lambda\) the scale.

The function has its peak (the mode) at
\(\lambda(\frac{a-1}{a})^{1/a}\).

When \code{a = 1}, the Weibull distribution reduces to the exponential
distribution.

Draw samples from the distribution:

\begin{Verbatim}[commandchars=\\\{\}]
\PYG{g+gp}{\PYGZgt{}\PYGZgt{}\PYGZgt{} }\PYG{n}{a} \PYG{o}{=} \PYG{l+m+mf}{5.} \PYG{c}{\PYGZsh{} shape}
\PYG{g+gp}{\PYGZgt{}\PYGZgt{}\PYGZgt{} }\PYG{n}{s} \PYG{o}{=} \PYG{n}{np}\PYG{o}{.}\PYG{n}{random}\PYG{o}{.}\PYG{n}{weibull}\PYG{p}{(}\PYG{n}{a}\PYG{p}{,} \PYG{l+m+mi}{1000}\PYG{p}{)}
\end{Verbatim}

Display the histogram of the samples, along with
the probability density function:

\begin{Verbatim}[commandchars=\\\{\}]
\PYG{g+gp}{\PYGZgt{}\PYGZgt{}\PYGZgt{} }\PYG{k+kn}{import} \PYG{n+nn}{matplotlib.pyplot} \PYG{k+kn}{as} \PYG{n+nn}{plt}
\PYG{g+gp}{\PYGZgt{}\PYGZgt{}\PYGZgt{} }\PYG{n}{x} \PYG{o}{=} \PYG{n}{np}\PYG{o}{.}\PYG{n}{arange}\PYG{p}{(}\PYG{l+m+mi}{1}\PYG{p}{,}\PYG{l+m+mf}{100.}\PYG{p}{)}\PYG{o}{/}\PYG{l+m+mf}{50.}
\PYG{g+gp}{\PYGZgt{}\PYGZgt{}\PYGZgt{} }\PYG{k}{def} \PYG{n+nf}{weib}\PYG{p}{(}\PYG{n}{x}\PYG{p}{,}\PYG{n}{n}\PYG{p}{,}\PYG{n}{a}\PYG{p}{)}\PYG{p}{:}
\PYG{g+gp}{... }    \PYG{k}{return} \PYG{p}{(}\PYG{n}{a} \PYG{o}{/} \PYG{n}{n}\PYG{p}{)} \PYG{o}{*} \PYG{p}{(}\PYG{n}{x} \PYG{o}{/} \PYG{n}{n}\PYG{p}{)}\PYG{o}{*}\PYG{o}{*}\PYG{p}{(}\PYG{n}{a} \PYG{o}{\PYGZhy{}} \PYG{l+m+mi}{1}\PYG{p}{)} \PYG{o}{*} \PYG{n}{np}\PYG{o}{.}\PYG{n}{exp}\PYG{p}{(}\PYG{o}{\PYGZhy{}}\PYG{p}{(}\PYG{n}{x} \PYG{o}{/} \PYG{n}{n}\PYG{p}{)}\PYG{o}{*}\PYG{o}{*}\PYG{n}{a}\PYG{p}{)}
\end{Verbatim}

\begin{Verbatim}[commandchars=\\\{\}]
\PYG{g+gp}{\PYGZgt{}\PYGZgt{}\PYGZgt{} }\PYG{n}{count}\PYG{p}{,} \PYG{n}{bins}\PYG{p}{,} \PYG{n}{ignored} \PYG{o}{=} \PYG{n}{plt}\PYG{o}{.}\PYG{n}{hist}\PYG{p}{(}\PYG{n}{np}\PYG{o}{.}\PYG{n}{random}\PYG{o}{.}\PYG{n}{weibull}\PYG{p}{(}\PYG{l+m+mf}{5.}\PYG{p}{,}\PYG{l+m+mi}{1000}\PYG{p}{)}\PYG{p}{)}
\PYG{g+gp}{\PYGZgt{}\PYGZgt{}\PYGZgt{} }\PYG{n}{x} \PYG{o}{=} \PYG{n}{np}\PYG{o}{.}\PYG{n}{arange}\PYG{p}{(}\PYG{l+m+mi}{1}\PYG{p}{,}\PYG{l+m+mf}{100.}\PYG{p}{)}\PYG{o}{/}\PYG{l+m+mf}{50.}
\PYG{g+gp}{\PYGZgt{}\PYGZgt{}\PYGZgt{} }\PYG{n}{scale} \PYG{o}{=} \PYG{n}{count}\PYG{o}{.}\PYG{n}{max}\PYG{p}{(}\PYG{p}{)}\PYG{o}{/}\PYG{n}{weib}\PYG{p}{(}\PYG{n}{x}\PYG{p}{,} \PYG{l+m+mf}{1.}\PYG{p}{,} \PYG{l+m+mf}{5.}\PYG{p}{)}\PYG{o}{.}\PYG{n}{max}\PYG{p}{(}\PYG{p}{)}
\PYG{g+gp}{\PYGZgt{}\PYGZgt{}\PYGZgt{} }\PYG{n}{plt}\PYG{o}{.}\PYG{n}{plot}\PYG{p}{(}\PYG{n}{x}\PYG{p}{,} \PYG{n}{weib}\PYG{p}{(}\PYG{n}{x}\PYG{p}{,} \PYG{l+m+mf}{1.}\PYG{p}{,} \PYG{l+m+mf}{5.}\PYG{p}{)}\PYG{o}{*}\PYG{n}{scale}\PYG{p}{)}
\PYG{g+gp}{\PYGZgt{}\PYGZgt{}\PYGZgt{} }\PYG{n}{plt}\PYG{o}{.}\PYG{n}{show}\PYG{p}{(}\PYG{p}{)}
\end{Verbatim}

\end{fulllineitems}

\index{zeroBeforeStrNum() (in module acsAttractorAnalysis)}

\begin{fulllineitems}
\phantomsection\label{acsAttractorAnalysis:acsAttractorAnalysis.zeroBeforeStrNum}\pysiglinewithargsret{\code{acsAttractorAnalysis.}\bfcode{zeroBeforeStrNum}}{\emph{tmpl}, \emph{tmpL}}{}
Function to create string zero string vector before graph filename. 
According to the total number of reactions N zeros will be add before the instant reaction number 
(e.g. reaction 130 of 10000 the string became `00130')
:param tmpl: length (e.g. 1 = 1, 10 = 2, 100 = 3, ... ) of the current folder
:param tmpL: total length last folder 
:returns: A zero string numbers to complete the length of tmpL

\end{fulllineitems}

\index{zipf() (in module acsAttractorAnalysis)}

\begin{fulllineitems}
\phantomsection\label{acsAttractorAnalysis:acsAttractorAnalysis.zipf}\pysiglinewithargsret{\code{acsAttractorAnalysis.}\bfcode{zipf}}{\emph{a}, \emph{size=None}}{}
Draw samples from a Zipf distribution.

Samples are drawn from a Zipf distribution with specified parameter
\emph{a} \textgreater{} 1.

The Zipf distribution (also known as the zeta distribution) is a
continuous probability distribution that satisfies Zipf's law: the
frequency of an item is inversely proportional to its rank in a
frequency table.
\begin{description}
\item[{a}] \leavevmode{[}float \textgreater{} 1{]}
Distribution parameter.

\item[{size}] \leavevmode{[}int or tuple of int, optional{]}
Output shape.  If the given shape is, e.g., \code{(m, n, k)}, then
\code{m * n * k} samples are drawn; a single integer is equivalent in
its result to providing a mono-tuple, i.e., a 1-D array of length
\emph{size} is returned.  The default is None, in which case a single
scalar is returned.

\end{description}
\begin{description}
\item[{samples}] \leavevmode{[}scalar or ndarray{]}
The returned samples are greater than or equal to one.

\end{description}
\begin{description}
\item[{scipy.stats.distributions.zipf}] \leavevmode{[}probability density function,{]}
distribution, or cumulative density function, etc.

\end{description}

The probability density for the Zipf distribution is
\begin{gather}
\begin{split}p(x) = \frac{x^{-a}}{\zeta(a)},\end{split}\notag
\end{gather}
where \(\zeta\) is the Riemann Zeta function.

It is named for the American linguist George Kingsley Zipf, who noted
that the frequency of any word in a sample of a language is inversely
proportional to its rank in the frequency table.

Zipf, G. K., \emph{Selected Studies of the Principle of Relative Frequency
in Language}, Cambridge, MA: Harvard Univ. Press, 1932.

Draw samples from the distribution:

\begin{Verbatim}[commandchars=\\\{\}]
\PYG{g+gp}{\PYGZgt{}\PYGZgt{}\PYGZgt{} }\PYG{n}{a} \PYG{o}{=} \PYG{l+m+mf}{2.} \PYG{c}{\PYGZsh{} parameter}
\PYG{g+gp}{\PYGZgt{}\PYGZgt{}\PYGZgt{} }\PYG{n}{s} \PYG{o}{=} \PYG{n}{np}\PYG{o}{.}\PYG{n}{random}\PYG{o}{.}\PYG{n}{zipf}\PYG{p}{(}\PYG{n}{a}\PYG{p}{,} \PYG{l+m+mi}{1000}\PYG{p}{)}
\end{Verbatim}

Display the histogram of the samples, along with
the probability density function:

\begin{Verbatim}[commandchars=\\\{\}]
\PYG{g+gp}{\PYGZgt{}\PYGZgt{}\PYGZgt{} }\PYG{k+kn}{import} \PYG{n+nn}{matplotlib.pyplot} \PYG{k+kn}{as} \PYG{n+nn}{plt}
\PYG{g+gp}{\PYGZgt{}\PYGZgt{}\PYGZgt{} }\PYG{k+kn}{import} \PYG{n+nn}{scipy.special} \PYG{k+kn}{as} \PYG{n+nn}{sps}
\PYG{g+go}{Truncate s values at 50 so plot is interesting}
\PYG{g+gp}{\PYGZgt{}\PYGZgt{}\PYGZgt{} }\PYG{n}{count}\PYG{p}{,} \PYG{n}{bins}\PYG{p}{,} \PYG{n}{ignored} \PYG{o}{=} \PYG{n}{plt}\PYG{o}{.}\PYG{n}{hist}\PYG{p}{(}\PYG{n}{s}\PYG{p}{[}\PYG{n}{s}\PYG{o}{\PYGZlt{}}\PYG{l+m+mi}{50}\PYG{p}{]}\PYG{p}{,} \PYG{l+m+mi}{50}\PYG{p}{,} \PYG{n}{normed}\PYG{o}{=}\PYG{n+nb+bp}{True}\PYG{p}{)}
\PYG{g+gp}{\PYGZgt{}\PYGZgt{}\PYGZgt{} }\PYG{n}{x} \PYG{o}{=} \PYG{n}{np}\PYG{o}{.}\PYG{n}{arange}\PYG{p}{(}\PYG{l+m+mf}{1.}\PYG{p}{,} \PYG{l+m+mf}{50.}\PYG{p}{)}
\PYG{g+gp}{\PYGZgt{}\PYGZgt{}\PYGZgt{} }\PYG{n}{y} \PYG{o}{=} \PYG{n}{x}\PYG{o}{*}\PYG{o}{*}\PYG{p}{(}\PYG{o}{\PYGZhy{}}\PYG{n}{a}\PYG{p}{)}\PYG{o}{/}\PYG{n}{sps}\PYG{o}{.}\PYG{n}{zetac}\PYG{p}{(}\PYG{n}{a}\PYG{p}{)}
\PYG{g+gp}{\PYGZgt{}\PYGZgt{}\PYGZgt{} }\PYG{n}{plt}\PYG{o}{.}\PYG{n}{plot}\PYG{p}{(}\PYG{n}{x}\PYG{p}{,} \PYG{n}{y}\PYG{o}{/}\PYG{n+nb}{max}\PYG{p}{(}\PYG{n}{y}\PYG{p}{)}\PYG{p}{,} \PYG{n}{linewidth}\PYG{o}{=}\PYG{l+m+mi}{2}\PYG{p}{,} \PYG{n}{color}\PYG{o}{=}\PYG{l+s}{\PYGZsq{}}\PYG{l+s}{r}\PYG{l+s}{\PYGZsq{}}\PYG{p}{)}
\PYG{g+gp}{\PYGZgt{}\PYGZgt{}\PYGZgt{} }\PYG{n}{plt}\PYG{o}{.}\PYG{n}{show}\PYG{p}{(}\PYG{p}{)}
\end{Verbatim}

\end{fulllineitems}



\chapter{acsAttractorAnalysisInTime Module}
\label{acsAttractorAnalysisInTime:acsattractoranalysisintime-module}\label{acsAttractorAnalysisInTime::doc}\label{acsAttractorAnalysisInTime:module-acsAttractorAnalysisInTime}\index{acsAttractorAnalysisInTime (module)}
Function to analyse the different attractors emerging from different simulations in time. 
python \textasciitilde{}/Dropbox/python/GIT/ACS\_analysis/initializator.py -t2 -k3 -K-1 -f2 -s6 -m6 -p5 -I ./acsm2s.conf -H1 -v3 -c0.5 -F PROTO\_ac3\_f2\_s6\_m6\_p5\_RAF -N5 -x0 -i 100
\index{beta() (in module acsAttractorAnalysisInTime)}

\begin{fulllineitems}
\phantomsection\label{acsAttractorAnalysisInTime:acsAttractorAnalysisInTime.beta}\pysiglinewithargsret{\code{acsAttractorAnalysisInTime.}\bfcode{beta}}{\emph{a}, \emph{b}, \emph{size=None}}{}
The Beta distribution over \code{{[}0, 1{]}}.

The Beta distribution is a special case of the Dirichlet distribution,
and is related to the Gamma distribution.  It has the probability
distribution function
\begin{gather}
\begin{split}f(x; a,b) = \frac{1}{B(\alpha, \beta)} x^{\alpha - 1}
(1 - x)^{\beta - 1},\end{split}\notag
\end{gather}
where the normalisation, B, is the beta function,
\begin{gather}
\begin{split}B(\alpha, \beta) = \int_0^1 t^{\alpha - 1}
(1 - t)^{\beta - 1} dt.\end{split}\notag
\end{gather}
It is often seen in Bayesian inference and order statistics.
\begin{description}
\item[{a}] \leavevmode{[}float{]}
Alpha, non-negative.

\item[{b}] \leavevmode{[}float{]}
Beta, non-negative.

\item[{size}] \leavevmode{[}tuple of ints, optional{]}
The number of samples to draw.  The output is packed according to
the size given.

\end{description}
\begin{description}
\item[{out}] \leavevmode{[}ndarray{]}
Array of the given shape, containing values drawn from a
Beta distribution.

\end{description}

\end{fulllineitems}

\index{binomial() (in module acsAttractorAnalysisInTime)}

\begin{fulllineitems}
\phantomsection\label{acsAttractorAnalysisInTime:acsAttractorAnalysisInTime.binomial}\pysiglinewithargsret{\code{acsAttractorAnalysisInTime.}\bfcode{binomial}}{\emph{n}, \emph{p}, \emph{size=None}}{}
Draw samples from a binomial distribution.

Samples are drawn from a Binomial distribution with specified
parameters, n trials and p probability of success where
n an integer \textgreater{}= 0 and p is in the interval {[}0,1{]}. (n may be
input as a float, but it is truncated to an integer in use)
\begin{description}
\item[{n}] \leavevmode{[}float (but truncated to an integer){]}
parameter, \textgreater{}= 0.

\item[{p}] \leavevmode{[}float{]}
parameter, \textgreater{}= 0 and \textless{}=1.

\item[{size}] \leavevmode{[}\{tuple, int\}{]}
Output shape.  If the given shape is, e.g., \code{(m, n, k)}, then
\code{m * n * k} samples are drawn.

\end{description}
\begin{description}
\item[{samples}] \leavevmode{[}\{ndarray, scalar\}{]}
where the values are all integers in  {[}0, n{]}.

\end{description}
\begin{description}
\item[{scipy.stats.distributions.binom}] \leavevmode{[}probability density function,{]}
distribution or cumulative density function, etc.

\end{description}

The probability density for the Binomial distribution is
\begin{gather}
\begin{split}P(N) = \binom{n}{N}p^N(1-p)^{n-N},\end{split}\notag
\end{gather}
where \(n\) is the number of trials, \(p\) is the probability
of success, and \(N\) is the number of successes.

When estimating the standard error of a proportion in a population by
using a random sample, the normal distribution works well unless the
product p*n \textless{}=5, where p = population proportion estimate, and n =
number of samples, in which case the binomial distribution is used
instead. For example, a sample of 15 people shows 4 who are left
handed, and 11 who are right handed. Then p = 4/15 = 27\%. 0.27*15 = 4,
so the binomial distribution should be used in this case.

Draw samples from the distribution:

\begin{Verbatim}[commandchars=\\\{\}]
\PYG{g+gp}{\PYGZgt{}\PYGZgt{}\PYGZgt{} }\PYG{n}{n}\PYG{p}{,} \PYG{n}{p} \PYG{o}{=} \PYG{l+m+mi}{10}\PYG{p}{,} \PYG{o}{.}\PYG{l+m+mi}{5} \PYG{c}{\PYGZsh{} number of trials, probability of each trial}
\PYG{g+gp}{\PYGZgt{}\PYGZgt{}\PYGZgt{} }\PYG{n}{s} \PYG{o}{=} \PYG{n}{np}\PYG{o}{.}\PYG{n}{random}\PYG{o}{.}\PYG{n}{binomial}\PYG{p}{(}\PYG{n}{n}\PYG{p}{,} \PYG{n}{p}\PYG{p}{,} \PYG{l+m+mi}{1000}\PYG{p}{)}
\PYG{g+go}{\PYGZsh{} result of flipping a coin 10 times, tested 1000 times.}
\end{Verbatim}

A real world example. A company drills 9 wild-cat oil exploration
wells, each with an estimated probability of success of 0.1. All nine
wells fail. What is the probability of that happening?

Let's do 20,000 trials of the model, and count the number that
generate zero positive results.

\begin{Verbatim}[commandchars=\\\{\}]
\PYG{g+gp}{\PYGZgt{}\PYGZgt{}\PYGZgt{} }\PYG{n+nb}{sum}\PYG{p}{(}\PYG{n}{np}\PYG{o}{.}\PYG{n}{random}\PYG{o}{.}\PYG{n}{binomial}\PYG{p}{(}\PYG{l+m+mi}{9}\PYG{p}{,}\PYG{l+m+mf}{0.1}\PYG{p}{,}\PYG{l+m+mi}{20000}\PYG{p}{)}\PYG{o}{==}\PYG{l+m+mi}{0}\PYG{p}{)}\PYG{o}{/}\PYG{l+m+mf}{20000.}
\PYG{g+go}{answer = 0.38885, or 38\PYGZpc{}.}
\end{Verbatim}

\end{fulllineitems}

\index{chisquare() (in module acsAttractorAnalysisInTime)}

\begin{fulllineitems}
\phantomsection\label{acsAttractorAnalysisInTime:acsAttractorAnalysisInTime.chisquare}\pysiglinewithargsret{\code{acsAttractorAnalysisInTime.}\bfcode{chisquare}}{\emph{df}, \emph{size=None}}{}
Draw samples from a chi-square distribution.

When \emph{df} independent random variables, each with standard normal
distributions (mean 0, variance 1), are squared and summed, the
resulting distribution is chi-square (see Notes).  This distribution
is often used in hypothesis testing.
\begin{description}
\item[{df}] \leavevmode{[}int{]}
Number of degrees of freedom.

\item[{size}] \leavevmode{[}tuple of ints, int, optional{]}
Size of the returned array.  By default, a scalar is
returned.

\end{description}
\begin{description}
\item[{output}] \leavevmode{[}ndarray{]}
Samples drawn from the distribution, packed in a \emph{size}-shaped
array.

\end{description}
\begin{description}
\item[{ValueError}] \leavevmode
When \emph{df} \textless{}= 0 or when an inappropriate \emph{size} (e.g. \code{size=-1})
is given.

\end{description}

The variable obtained by summing the squares of \emph{df} independent,
standard normally distributed random variables:
\begin{gather}
\begin{split}Q = \sum_{i=0}^{\mathtt{df}} X^2_i\end{split}\notag
\end{gather}
is chi-square distributed, denoted
\begin{gather}
\begin{split}Q \sim \chi^2_k.\end{split}\notag
\end{gather}
The probability density function of the chi-squared distribution is
\begin{gather}
\begin{split}p(x) = \frac{(1/2)^{k/2}}{\Gamma(k/2)}
x^{k/2 - 1} e^{-x/2},\end{split}\notag
\end{gather}
where \(\Gamma\) is the gamma function,
\begin{gather}
\begin{split}\Gamma(x) = \int_0^{-\infty} t^{x - 1} e^{-t} dt.\end{split}\notag
\end{gather}
\href{http://www.itl.nist.gov/div898/handbook/eda/section3/eda3666.htm}{NIST/SEMATECH e-Handbook of Statistical Methods}

\begin{Verbatim}[commandchars=\\\{\}]
\PYG{g+gp}{\PYGZgt{}\PYGZgt{}\PYGZgt{} }\PYG{n}{np}\PYG{o}{.}\PYG{n}{random}\PYG{o}{.}\PYG{n}{chisquare}\PYG{p}{(}\PYG{l+m+mi}{2}\PYG{p}{,}\PYG{l+m+mi}{4}\PYG{p}{)}
\PYG{g+go}{array([ 1.89920014,  9.00867716,  3.13710533,  5.62318272])}
\end{Verbatim}

\end{fulllineitems}

\index{exponential() (in module acsAttractorAnalysisInTime)}

\begin{fulllineitems}
\phantomsection\label{acsAttractorAnalysisInTime:acsAttractorAnalysisInTime.exponential}\pysiglinewithargsret{\code{acsAttractorAnalysisInTime.}\bfcode{exponential}}{\emph{scale=1.0}, \emph{size=None}}{}
Exponential distribution.

Its probability density function is
\begin{gather}
\begin{split}f(x; \frac{1}{\beta}) = \frac{1}{\beta} \exp(-\frac{x}{\beta}),\end{split}\notag
\end{gather}
for \code{x \textgreater{} 0} and 0 elsewhere. \(\beta\) is the scale parameter,
which is the inverse of the rate parameter \(\lambda = 1/\beta\).
The rate parameter is an alternative, widely used parameterization
of the exponential distribution {\color{red}\bfseries{}{[}3{]}\_}.

The exponential distribution is a continuous analogue of the
geometric distribution.  It describes many common situations, such as
the size of raindrops measured over many rainstorms {\color{red}\bfseries{}{[}1{]}\_}, or the time
between page requests to Wikipedia {\color{red}\bfseries{}{[}2{]}\_}.
\begin{description}
\item[{scale}] \leavevmode{[}float{]}
The scale parameter, \(\beta = 1/\lambda\).

\item[{size}] \leavevmode{[}tuple of ints{]}
Number of samples to draw.  The output is shaped
according to \emph{size}.

\end{description}

\end{fulllineitems}

\index{f() (in module acsAttractorAnalysisInTime)}

\begin{fulllineitems}
\phantomsection\label{acsAttractorAnalysisInTime:acsAttractorAnalysisInTime.f}\pysiglinewithargsret{\code{acsAttractorAnalysisInTime.}\bfcode{f}}{\emph{dfnum}, \emph{dfden}, \emph{size=None}}{}
Draw samples from a F distribution.

Samples are drawn from an F distribution with specified parameters,
\emph{dfnum} (degrees of freedom in numerator) and \emph{dfden} (degrees of freedom
in denominator), where both parameters should be greater than zero.

The random variate of the F distribution (also known as the
Fisher distribution) is a continuous probability distribution
that arises in ANOVA tests, and is the ratio of two chi-square
variates.
\begin{description}
\item[{dfnum}] \leavevmode{[}float{]}
Degrees of freedom in numerator. Should be greater than zero.

\item[{dfden}] \leavevmode{[}float{]}
Degrees of freedom in denominator. Should be greater than zero.

\item[{size}] \leavevmode{[}\{tuple, int\}, optional{]}
Output shape.  If the given shape is, e.g., \code{(m, n, k)},
then \code{m * n * k} samples are drawn. By default only one sample
is returned.

\end{description}
\begin{description}
\item[{samples}] \leavevmode{[}\{ndarray, scalar\}{]}
Samples from the Fisher distribution.

\end{description}
\begin{description}
\item[{scipy.stats.distributions.f}] \leavevmode{[}probability density function,{]}
distribution or cumulative density function, etc.

\end{description}

The F statistic is used to compare in-group variances to between-group
variances. Calculating the distribution depends on the sampling, and
so it is a function of the respective degrees of freedom in the
problem.  The variable \emph{dfnum} is the number of samples minus one, the
between-groups degrees of freedom, while \emph{dfden} is the within-groups
degrees of freedom, the sum of the number of samples in each group
minus the number of groups.

An example from Glantz{[}1{]}, pp 47-40.
Two groups, children of diabetics (25 people) and children from people
without diabetes (25 controls). Fasting blood glucose was measured,
case group had a mean value of 86.1, controls had a mean value of
82.2. Standard deviations were 2.09 and 2.49 respectively. Are these
data consistent with the null hypothesis that the parents diabetic
status does not affect their children's blood glucose levels?
Calculating the F statistic from the data gives a value of 36.01.

Draw samples from the distribution:

\begin{Verbatim}[commandchars=\\\{\}]
\PYG{g+gp}{\PYGZgt{}\PYGZgt{}\PYGZgt{} }\PYG{n}{dfnum} \PYG{o}{=} \PYG{l+m+mf}{1.} \PYG{c}{\PYGZsh{} between group degrees of freedom}
\PYG{g+gp}{\PYGZgt{}\PYGZgt{}\PYGZgt{} }\PYG{n}{dfden} \PYG{o}{=} \PYG{l+m+mf}{48.} \PYG{c}{\PYGZsh{} within groups degrees of freedom}
\PYG{g+gp}{\PYGZgt{}\PYGZgt{}\PYGZgt{} }\PYG{n}{s} \PYG{o}{=} \PYG{n}{np}\PYG{o}{.}\PYG{n}{random}\PYG{o}{.}\PYG{n}{f}\PYG{p}{(}\PYG{n}{dfnum}\PYG{p}{,} \PYG{n}{dfden}\PYG{p}{,} \PYG{l+m+mi}{1000}\PYG{p}{)}
\end{Verbatim}

The lower bound for the top 1\% of the samples is :

\begin{Verbatim}[commandchars=\\\{\}]
\PYG{g+gp}{\PYGZgt{}\PYGZgt{}\PYGZgt{} }\PYG{n}{sort}\PYG{p}{(}\PYG{n}{s}\PYG{p}{)}\PYG{p}{[}\PYG{o}{\PYGZhy{}}\PYG{l+m+mi}{10}\PYG{p}{]}
\PYG{g+go}{7.61988120985}
\end{Verbatim}

So there is about a 1\% chance that the F statistic will exceed 7.62,
the measured value is 36, so the null hypothesis is rejected at the 1\%
level.

\end{fulllineitems}

\index{gamma() (in module acsAttractorAnalysisInTime)}

\begin{fulllineitems}
\phantomsection\label{acsAttractorAnalysisInTime:acsAttractorAnalysisInTime.gamma}\pysiglinewithargsret{\code{acsAttractorAnalysisInTime.}\bfcode{gamma}}{\emph{shape}, \emph{scale=1.0}, \emph{size=None}}{}
Draw samples from a Gamma distribution.

Samples are drawn from a Gamma distribution with specified parameters,
\emph{shape} (sometimes designated ``k'') and \emph{scale} (sometimes designated
``theta''), where both parameters are \textgreater{} 0.
\begin{description}
\item[{shape}] \leavevmode{[}scalar \textgreater{} 0{]}
The shape of the gamma distribution.

\item[{scale}] \leavevmode{[}scalar \textgreater{} 0, optional{]}
The scale of the gamma distribution.  Default is equal to 1.

\item[{size}] \leavevmode{[}shape\_tuple, optional{]}
Output shape.  If the given shape is, e.g., \code{(m, n, k)}, then
\code{m * n * k} samples are drawn.

\end{description}
\begin{description}
\item[{out}] \leavevmode{[}ndarray, float{]}
Returns one sample unless \emph{size} parameter is specified.

\end{description}
\begin{description}
\item[{scipy.stats.distributions.gamma}] \leavevmode{[}probability density function,{]}
distribution or cumulative density function, etc.

\end{description}

The probability density for the Gamma distribution is
\begin{gather}
\begin{split}p(x) = x^{k-1}\frac{e^{-x/\theta}}{\theta^k\Gamma(k)},\end{split}\notag
\end{gather}
where \(k\) is the shape and \(\theta\) the scale,
and \(\Gamma\) is the Gamma function.

The Gamma distribution is often used to model the times to failure of
electronic components, and arises naturally in processes for which the
waiting times between Poisson distributed events are relevant.

Draw samples from the distribution:

\begin{Verbatim}[commandchars=\\\{\}]
\PYG{g+gp}{\PYGZgt{}\PYGZgt{}\PYGZgt{} }\PYG{n}{shape}\PYG{p}{,} \PYG{n}{scale} \PYG{o}{=} \PYG{l+m+mf}{2.}\PYG{p}{,} \PYG{l+m+mf}{2.} \PYG{c}{\PYGZsh{} mean and dispersion}
\PYG{g+gp}{\PYGZgt{}\PYGZgt{}\PYGZgt{} }\PYG{n}{s} \PYG{o}{=} \PYG{n}{np}\PYG{o}{.}\PYG{n}{random}\PYG{o}{.}\PYG{n}{gamma}\PYG{p}{(}\PYG{n}{shape}\PYG{p}{,} \PYG{n}{scale}\PYG{p}{,} \PYG{l+m+mi}{1000}\PYG{p}{)}
\end{Verbatim}

Display the histogram of the samples, along with
the probability density function:

\begin{Verbatim}[commandchars=\\\{\}]
\PYG{g+gp}{\PYGZgt{}\PYGZgt{}\PYGZgt{} }\PYG{k+kn}{import} \PYG{n+nn}{matplotlib.pyplot} \PYG{k+kn}{as} \PYG{n+nn}{plt}
\PYG{g+gp}{\PYGZgt{}\PYGZgt{}\PYGZgt{} }\PYG{k+kn}{import} \PYG{n+nn}{scipy.special} \PYG{k+kn}{as} \PYG{n+nn}{sps}
\PYG{g+gp}{\PYGZgt{}\PYGZgt{}\PYGZgt{} }\PYG{n}{count}\PYG{p}{,} \PYG{n}{bins}\PYG{p}{,} \PYG{n}{ignored} \PYG{o}{=} \PYG{n}{plt}\PYG{o}{.}\PYG{n}{hist}\PYG{p}{(}\PYG{n}{s}\PYG{p}{,} \PYG{l+m+mi}{50}\PYG{p}{,} \PYG{n}{normed}\PYG{o}{=}\PYG{n+nb+bp}{True}\PYG{p}{)}
\PYG{g+gp}{\PYGZgt{}\PYGZgt{}\PYGZgt{} }\PYG{n}{y} \PYG{o}{=} \PYG{n}{bins}\PYG{o}{*}\PYG{o}{*}\PYG{p}{(}\PYG{n}{shape}\PYG{o}{\PYGZhy{}}\PYG{l+m+mi}{1}\PYG{p}{)}\PYG{o}{*}\PYG{p}{(}\PYG{n}{np}\PYG{o}{.}\PYG{n}{exp}\PYG{p}{(}\PYG{o}{\PYGZhy{}}\PYG{n}{bins}\PYG{o}{/}\PYG{n}{scale}\PYG{p}{)} \PYG{o}{/}
\PYG{g+gp}{... }                     \PYG{p}{(}\PYG{n}{sps}\PYG{o}{.}\PYG{n}{gamma}\PYG{p}{(}\PYG{n}{shape}\PYG{p}{)}\PYG{o}{*}\PYG{n}{scale}\PYG{o}{*}\PYG{o}{*}\PYG{n}{shape}\PYG{p}{)}\PYG{p}{)}
\PYG{g+gp}{\PYGZgt{}\PYGZgt{}\PYGZgt{} }\PYG{n}{plt}\PYG{o}{.}\PYG{n}{plot}\PYG{p}{(}\PYG{n}{bins}\PYG{p}{,} \PYG{n}{y}\PYG{p}{,} \PYG{n}{linewidth}\PYG{o}{=}\PYG{l+m+mi}{2}\PYG{p}{,} \PYG{n}{color}\PYG{o}{=}\PYG{l+s}{\PYGZsq{}}\PYG{l+s}{r}\PYG{l+s}{\PYGZsq{}}\PYG{p}{)}
\PYG{g+gp}{\PYGZgt{}\PYGZgt{}\PYGZgt{} }\PYG{n}{plt}\PYG{o}{.}\PYG{n}{show}\PYG{p}{(}\PYG{p}{)}
\end{Verbatim}

\end{fulllineitems}

\index{geometric() (in module acsAttractorAnalysisInTime)}

\begin{fulllineitems}
\phantomsection\label{acsAttractorAnalysisInTime:acsAttractorAnalysisInTime.geometric}\pysiglinewithargsret{\code{acsAttractorAnalysisInTime.}\bfcode{geometric}}{\emph{p}, \emph{size=None}}{}
Draw samples from the geometric distribution.

Bernoulli trials are experiments with one of two outcomes:
success or failure (an example of such an experiment is flipping
a coin).  The geometric distribution models the number of trials
that must be run in order to achieve success.  It is therefore
supported on the positive integers, \code{k = 1, 2, ...}.

The probability mass function of the geometric distribution is
\begin{gather}
\begin{split}f(k) = (1 - p)^{k - 1} p\end{split}\notag
\end{gather}
where \emph{p} is the probability of success of an individual trial.
\begin{description}
\item[{p}] \leavevmode{[}float{]}
The probability of success of an individual trial.

\item[{size}] \leavevmode{[}tuple of ints{]}
Number of values to draw from the distribution.  The output
is shaped according to \emph{size}.

\end{description}
\begin{description}
\item[{out}] \leavevmode{[}ndarray{]}
Samples from the geometric distribution, shaped according to
\emph{size}.

\end{description}

Draw ten thousand values from the geometric distribution,
with the probability of an individual success equal to 0.35:

\begin{Verbatim}[commandchars=\\\{\}]
\PYG{g+gp}{\PYGZgt{}\PYGZgt{}\PYGZgt{} }\PYG{n}{z} \PYG{o}{=} \PYG{n}{np}\PYG{o}{.}\PYG{n}{random}\PYG{o}{.}\PYG{n}{geometric}\PYG{p}{(}\PYG{n}{p}\PYG{o}{=}\PYG{l+m+mf}{0.35}\PYG{p}{,} \PYG{n}{size}\PYG{o}{=}\PYG{l+m+mi}{10000}\PYG{p}{)}
\end{Verbatim}

How many trials succeeded after a single run?

\begin{Verbatim}[commandchars=\\\{\}]
\PYG{g+gp}{\PYGZgt{}\PYGZgt{}\PYGZgt{} }\PYG{p}{(}\PYG{n}{z} \PYG{o}{==} \PYG{l+m+mi}{1}\PYG{p}{)}\PYG{o}{.}\PYG{n}{sum}\PYG{p}{(}\PYG{p}{)} \PYG{o}{/} \PYG{l+m+mf}{10000.}
\PYG{g+go}{0.34889999999999999 \PYGZsh{}random}
\end{Verbatim}

\end{fulllineitems}

\index{get\_state() (in module acsAttractorAnalysisInTime)}

\begin{fulllineitems}
\phantomsection\label{acsAttractorAnalysisInTime:acsAttractorAnalysisInTime.get_state}\pysiglinewithargsret{\code{acsAttractorAnalysisInTime.}\bfcode{get\_state}}{}{}
Return a tuple representing the internal state of the generator.

For more details, see \emph{set\_state}.
\begin{description}
\item[{out}] \leavevmode{[}tuple(str, ndarray of 624 uints, int, int, float){]}
The returned tuple has the following items:
\begin{enumerate}
\item {} 
the string `MT19937'.

\item {} 
a 1-D array of 624 unsigned integer keys.

\item {} 
an integer \code{pos}.

\item {} 
an integer \code{has\_gauss}.

\item {} 
a float \code{cached\_gaussian}.

\end{enumerate}

\end{description}

set\_state

\emph{set\_state} and \emph{get\_state} are not needed to work with any of the
random distributions in NumPy. If the internal state is manually altered,
the user should know exactly what he/she is doing.

\end{fulllineitems}

\index{gumbel() (in module acsAttractorAnalysisInTime)}

\begin{fulllineitems}
\phantomsection\label{acsAttractorAnalysisInTime:acsAttractorAnalysisInTime.gumbel}\pysiglinewithargsret{\code{acsAttractorAnalysisInTime.}\bfcode{gumbel}}{\emph{loc=0.0}, \emph{scale=1.0}, \emph{size=None}}{}
Gumbel distribution.

Draw samples from a Gumbel distribution with specified location and scale.
For more information on the Gumbel distribution, see Notes and References
below.
\begin{description}
\item[{loc}] \leavevmode{[}float{]}
The location of the mode of the distribution.

\item[{scale}] \leavevmode{[}float{]}
The scale parameter of the distribution.

\item[{size}] \leavevmode{[}tuple of ints{]}
Output shape.  If the given shape is, e.g., \code{(m, n, k)}, then
\code{m * n * k} samples are drawn.

\end{description}
\begin{description}
\item[{out}] \leavevmode{[}ndarray{]}
The samples

\end{description}

scipy.stats.gumbel\_l
scipy.stats.gumbel\_r
scipy.stats.genextreme
\begin{quote}

probability density function, distribution, or cumulative density
function, etc. for each of the above
\end{quote}

weibull

The Gumbel (or Smallest Extreme Value (SEV) or the Smallest Extreme Value
Type I) distribution is one of a class of Generalized Extreme Value (GEV)
distributions used in modeling extreme value problems.  The Gumbel is a
special case of the Extreme Value Type I distribution for maximums from
distributions with ``exponential-like'' tails.

The probability density for the Gumbel distribution is
\begin{gather}
\begin{split}p(x) = \frac{e^{-(x - \mu)/ \beta}}{\beta} e^{ -e^{-(x - \mu)/
\beta}},\end{split}\notag
\end{gather}
where \(\mu\) is the mode, a location parameter, and \(\beta\) is
the scale parameter.

The Gumbel (named for German mathematician Emil Julius Gumbel) was used
very early in the hydrology literature, for modeling the occurrence of
flood events. It is also used for modeling maximum wind speed and rainfall
rates.  It is a ``fat-tailed'' distribution - the probability of an event in
the tail of the distribution is larger than if one used a Gaussian, hence
the surprisingly frequent occurrence of 100-year floods. Floods were
initially modeled as a Gaussian process, which underestimated the frequency
of extreme events.

It is one of a class of extreme value distributions, the Generalized
Extreme Value (GEV) distributions, which also includes the Weibull and
Frechet.

The function has a mean of \(\mu + 0.57721\beta\) and a variance of
\(\frac{\pi^2}{6}\beta^2\).

Gumbel, E. J., \emph{Statistics of Extremes}, New York: Columbia University
Press, 1958.

Reiss, R.-D. and Thomas, M., \emph{Statistical Analysis of Extreme Values from
Insurance, Finance, Hydrology and Other Fields}, Basel: Birkhauser Verlag,
2001.

Draw samples from the distribution:

\begin{Verbatim}[commandchars=\\\{\}]
\PYG{g+gp}{\PYGZgt{}\PYGZgt{}\PYGZgt{} }\PYG{n}{mu}\PYG{p}{,} \PYG{n}{beta} \PYG{o}{=} \PYG{l+m+mi}{0}\PYG{p}{,} \PYG{l+m+mf}{0.1} \PYG{c}{\PYGZsh{} location and scale}
\PYG{g+gp}{\PYGZgt{}\PYGZgt{}\PYGZgt{} }\PYG{n}{s} \PYG{o}{=} \PYG{n}{np}\PYG{o}{.}\PYG{n}{random}\PYG{o}{.}\PYG{n}{gumbel}\PYG{p}{(}\PYG{n}{mu}\PYG{p}{,} \PYG{n}{beta}\PYG{p}{,} \PYG{l+m+mi}{1000}\PYG{p}{)}
\end{Verbatim}

Display the histogram of the samples, along with
the probability density function:

\begin{Verbatim}[commandchars=\\\{\}]
\PYG{g+gp}{\PYGZgt{}\PYGZgt{}\PYGZgt{} }\PYG{k+kn}{import} \PYG{n+nn}{matplotlib.pyplot} \PYG{k+kn}{as} \PYG{n+nn}{plt}
\PYG{g+gp}{\PYGZgt{}\PYGZgt{}\PYGZgt{} }\PYG{n}{count}\PYG{p}{,} \PYG{n}{bins}\PYG{p}{,} \PYG{n}{ignored} \PYG{o}{=} \PYG{n}{plt}\PYG{o}{.}\PYG{n}{hist}\PYG{p}{(}\PYG{n}{s}\PYG{p}{,} \PYG{l+m+mi}{30}\PYG{p}{,} \PYG{n}{normed}\PYG{o}{=}\PYG{n+nb+bp}{True}\PYG{p}{)}
\PYG{g+gp}{\PYGZgt{}\PYGZgt{}\PYGZgt{} }\PYG{n}{plt}\PYG{o}{.}\PYG{n}{plot}\PYG{p}{(}\PYG{n}{bins}\PYG{p}{,} \PYG{p}{(}\PYG{l+m+mi}{1}\PYG{o}{/}\PYG{n}{beta}\PYG{p}{)}\PYG{o}{*}\PYG{n}{np}\PYG{o}{.}\PYG{n}{exp}\PYG{p}{(}\PYG{o}{\PYGZhy{}}\PYG{p}{(}\PYG{n}{bins} \PYG{o}{\PYGZhy{}} \PYG{n}{mu}\PYG{p}{)}\PYG{o}{/}\PYG{n}{beta}\PYG{p}{)}
\PYG{g+gp}{... }         \PYG{o}{*} \PYG{n}{np}\PYG{o}{.}\PYG{n}{exp}\PYG{p}{(} \PYG{o}{\PYGZhy{}}\PYG{n}{np}\PYG{o}{.}\PYG{n}{exp}\PYG{p}{(} \PYG{o}{\PYGZhy{}}\PYG{p}{(}\PYG{n}{bins} \PYG{o}{\PYGZhy{}} \PYG{n}{mu}\PYG{p}{)} \PYG{o}{/}\PYG{n}{beta}\PYG{p}{)} \PYG{p}{)}\PYG{p}{,}
\PYG{g+gp}{... }         \PYG{n}{linewidth}\PYG{o}{=}\PYG{l+m+mi}{2}\PYG{p}{,} \PYG{n}{color}\PYG{o}{=}\PYG{l+s}{\PYGZsq{}}\PYG{l+s}{r}\PYG{l+s}{\PYGZsq{}}\PYG{p}{)}
\PYG{g+gp}{\PYGZgt{}\PYGZgt{}\PYGZgt{} }\PYG{n}{plt}\PYG{o}{.}\PYG{n}{show}\PYG{p}{(}\PYG{p}{)}
\end{Verbatim}

Show how an extreme value distribution can arise from a Gaussian process
and compare to a Gaussian:

\begin{Verbatim}[commandchars=\\\{\}]
\PYG{g+gp}{\PYGZgt{}\PYGZgt{}\PYGZgt{} }\PYG{n}{means} \PYG{o}{=} \PYG{p}{[}\PYG{p}{]}
\PYG{g+gp}{\PYGZgt{}\PYGZgt{}\PYGZgt{} }\PYG{n}{maxima} \PYG{o}{=} \PYG{p}{[}\PYG{p}{]}
\PYG{g+gp}{\PYGZgt{}\PYGZgt{}\PYGZgt{} }\PYG{k}{for} \PYG{n}{i} \PYG{o+ow}{in} \PYG{n+nb}{range}\PYG{p}{(}\PYG{l+m+mi}{0}\PYG{p}{,}\PYG{l+m+mi}{1000}\PYG{p}{)} \PYG{p}{:}
\PYG{g+gp}{... }   \PYG{n}{a} \PYG{o}{=} \PYG{n}{np}\PYG{o}{.}\PYG{n}{random}\PYG{o}{.}\PYG{n}{normal}\PYG{p}{(}\PYG{n}{mu}\PYG{p}{,} \PYG{n}{beta}\PYG{p}{,} \PYG{l+m+mi}{1000}\PYG{p}{)}
\PYG{g+gp}{... }   \PYG{n}{means}\PYG{o}{.}\PYG{n}{append}\PYG{p}{(}\PYG{n}{a}\PYG{o}{.}\PYG{n}{mean}\PYG{p}{(}\PYG{p}{)}\PYG{p}{)}
\PYG{g+gp}{... }   \PYG{n}{maxima}\PYG{o}{.}\PYG{n}{append}\PYG{p}{(}\PYG{n}{a}\PYG{o}{.}\PYG{n}{max}\PYG{p}{(}\PYG{p}{)}\PYG{p}{)}
\PYG{g+gp}{\PYGZgt{}\PYGZgt{}\PYGZgt{} }\PYG{n}{count}\PYG{p}{,} \PYG{n}{bins}\PYG{p}{,} \PYG{n}{ignored} \PYG{o}{=} \PYG{n}{plt}\PYG{o}{.}\PYG{n}{hist}\PYG{p}{(}\PYG{n}{maxima}\PYG{p}{,} \PYG{l+m+mi}{30}\PYG{p}{,} \PYG{n}{normed}\PYG{o}{=}\PYG{n+nb+bp}{True}\PYG{p}{)}
\PYG{g+gp}{\PYGZgt{}\PYGZgt{}\PYGZgt{} }\PYG{n}{beta} \PYG{o}{=} \PYG{n}{np}\PYG{o}{.}\PYG{n}{std}\PYG{p}{(}\PYG{n}{maxima}\PYG{p}{)}\PYG{o}{*}\PYG{n}{np}\PYG{o}{.}\PYG{n}{pi}\PYG{o}{/}\PYG{n}{np}\PYG{o}{.}\PYG{n}{sqrt}\PYG{p}{(}\PYG{l+m+mi}{6}\PYG{p}{)}
\PYG{g+gp}{\PYGZgt{}\PYGZgt{}\PYGZgt{} }\PYG{n}{mu} \PYG{o}{=} \PYG{n}{np}\PYG{o}{.}\PYG{n}{mean}\PYG{p}{(}\PYG{n}{maxima}\PYG{p}{)} \PYG{o}{\PYGZhy{}} \PYG{l+m+mf}{0.57721}\PYG{o}{*}\PYG{n}{beta}
\PYG{g+gp}{\PYGZgt{}\PYGZgt{}\PYGZgt{} }\PYG{n}{plt}\PYG{o}{.}\PYG{n}{plot}\PYG{p}{(}\PYG{n}{bins}\PYG{p}{,} \PYG{p}{(}\PYG{l+m+mi}{1}\PYG{o}{/}\PYG{n}{beta}\PYG{p}{)}\PYG{o}{*}\PYG{n}{np}\PYG{o}{.}\PYG{n}{exp}\PYG{p}{(}\PYG{o}{\PYGZhy{}}\PYG{p}{(}\PYG{n}{bins} \PYG{o}{\PYGZhy{}} \PYG{n}{mu}\PYG{p}{)}\PYG{o}{/}\PYG{n}{beta}\PYG{p}{)}
\PYG{g+gp}{... }         \PYG{o}{*} \PYG{n}{np}\PYG{o}{.}\PYG{n}{exp}\PYG{p}{(}\PYG{o}{\PYGZhy{}}\PYG{n}{np}\PYG{o}{.}\PYG{n}{exp}\PYG{p}{(}\PYG{o}{\PYGZhy{}}\PYG{p}{(}\PYG{n}{bins} \PYG{o}{\PYGZhy{}} \PYG{n}{mu}\PYG{p}{)}\PYG{o}{/}\PYG{n}{beta}\PYG{p}{)}\PYG{p}{)}\PYG{p}{,}
\PYG{g+gp}{... }         \PYG{n}{linewidth}\PYG{o}{=}\PYG{l+m+mi}{2}\PYG{p}{,} \PYG{n}{color}\PYG{o}{=}\PYG{l+s}{\PYGZsq{}}\PYG{l+s}{r}\PYG{l+s}{\PYGZsq{}}\PYG{p}{)}
\PYG{g+gp}{\PYGZgt{}\PYGZgt{}\PYGZgt{} }\PYG{n}{plt}\PYG{o}{.}\PYG{n}{plot}\PYG{p}{(}\PYG{n}{bins}\PYG{p}{,} \PYG{l+m+mi}{1}\PYG{o}{/}\PYG{p}{(}\PYG{n}{beta} \PYG{o}{*} \PYG{n}{np}\PYG{o}{.}\PYG{n}{sqrt}\PYG{p}{(}\PYG{l+m+mi}{2} \PYG{o}{*} \PYG{n}{np}\PYG{o}{.}\PYG{n}{pi}\PYG{p}{)}\PYG{p}{)}
\PYG{g+gp}{... }         \PYG{o}{*} \PYG{n}{np}\PYG{o}{.}\PYG{n}{exp}\PYG{p}{(}\PYG{o}{\PYGZhy{}}\PYG{p}{(}\PYG{n}{bins} \PYG{o}{\PYGZhy{}} \PYG{n}{mu}\PYG{p}{)}\PYG{o}{*}\PYG{o}{*}\PYG{l+m+mi}{2} \PYG{o}{/} \PYG{p}{(}\PYG{l+m+mi}{2} \PYG{o}{*} \PYG{n}{beta}\PYG{o}{*}\PYG{o}{*}\PYG{l+m+mi}{2}\PYG{p}{)}\PYG{p}{)}\PYG{p}{,}
\PYG{g+gp}{... }         \PYG{n}{linewidth}\PYG{o}{=}\PYG{l+m+mi}{2}\PYG{p}{,} \PYG{n}{color}\PYG{o}{=}\PYG{l+s}{\PYGZsq{}}\PYG{l+s}{g}\PYG{l+s}{\PYGZsq{}}\PYG{p}{)}
\PYG{g+gp}{\PYGZgt{}\PYGZgt{}\PYGZgt{} }\PYG{n}{plt}\PYG{o}{.}\PYG{n}{show}\PYG{p}{(}\PYG{p}{)}
\end{Verbatim}

\end{fulllineitems}

\index{hypergeometric() (in module acsAttractorAnalysisInTime)}

\begin{fulllineitems}
\phantomsection\label{acsAttractorAnalysisInTime:acsAttractorAnalysisInTime.hypergeometric}\pysiglinewithargsret{\code{acsAttractorAnalysisInTime.}\bfcode{hypergeometric}}{\emph{ngood}, \emph{nbad}, \emph{nsample}, \emph{size=None}}{}
Draw samples from a Hypergeometric distribution.

Samples are drawn from a Hypergeometric distribution with specified
parameters, ngood (ways to make a good selection), nbad (ways to make
a bad selection), and nsample = number of items sampled, which is less
than or equal to the sum ngood + nbad.
\begin{description}
\item[{ngood}] \leavevmode{[}int or array\_like{]}
Number of ways to make a good selection.  Must be nonnegative.

\item[{nbad}] \leavevmode{[}int or array\_like{]}
Number of ways to make a bad selection.  Must be nonnegative.

\item[{nsample}] \leavevmode{[}int or array\_like{]}
Number of items sampled.  Must be at least 1 and at most
\code{ngood + nbad}.

\item[{size}] \leavevmode{[}int or tuple of int{]}
Output shape.  If the given shape is, e.g., \code{(m, n, k)}, then
\code{m * n * k} samples are drawn.

\end{description}
\begin{description}
\item[{samples}] \leavevmode{[}ndarray or scalar{]}
The values are all integers in  {[}0, n{]}.

\end{description}
\begin{description}
\item[{scipy.stats.distributions.hypergeom}] \leavevmode{[}probability density function,{]}
distribution or cumulative density function, etc.

\end{description}

The probability density for the Hypergeometric distribution is
\begin{gather}
\begin{split}P(x) = \frac{\binom{m}{n}\binom{N-m}{n-x}}{\binom{N}{n}},\end{split}\notag
\end{gather}
where \(0 \le x \le m\) and \(n+m-N \le x \le n\)

for P(x) the probability of x successes, n = ngood, m = nbad, and
N = number of samples.

Consider an urn with black and white marbles in it, ngood of them
black and nbad are white. If you draw nsample balls without
replacement, then the Hypergeometric distribution describes the
distribution of black balls in the drawn sample.

Note that this distribution is very similar to the Binomial
distribution, except that in this case, samples are drawn without
replacement, whereas in the Binomial case samples are drawn with
replacement (or the sample space is infinite). As the sample space
becomes large, this distribution approaches the Binomial.

Draw samples from the distribution:

\begin{Verbatim}[commandchars=\\\{\}]
\PYG{g+gp}{\PYGZgt{}\PYGZgt{}\PYGZgt{} }\PYG{n}{ngood}\PYG{p}{,} \PYG{n}{nbad}\PYG{p}{,} \PYG{n}{nsamp} \PYG{o}{=} \PYG{l+m+mi}{100}\PYG{p}{,} \PYG{l+m+mi}{2}\PYG{p}{,} \PYG{l+m+mi}{10}
\PYG{g+go}{\PYGZsh{} number of good, number of bad, and number of samples}
\PYG{g+gp}{\PYGZgt{}\PYGZgt{}\PYGZgt{} }\PYG{n}{s} \PYG{o}{=} \PYG{n}{np}\PYG{o}{.}\PYG{n}{random}\PYG{o}{.}\PYG{n}{hypergeometric}\PYG{p}{(}\PYG{n}{ngood}\PYG{p}{,} \PYG{n}{nbad}\PYG{p}{,} \PYG{n}{nsamp}\PYG{p}{,} \PYG{l+m+mi}{1000}\PYG{p}{)}
\PYG{g+gp}{\PYGZgt{}\PYGZgt{}\PYGZgt{} }\PYG{n}{hist}\PYG{p}{(}\PYG{n}{s}\PYG{p}{)}
\PYG{g+go}{\PYGZsh{}   note that it is very unlikely to grab both bad items}
\end{Verbatim}

Suppose you have an urn with 15 white and 15 black marbles.
If you pull 15 marbles at random, how likely is it that
12 or more of them are one color?

\begin{Verbatim}[commandchars=\\\{\}]
\PYG{g+gp}{\PYGZgt{}\PYGZgt{}\PYGZgt{} }\PYG{n}{s} \PYG{o}{=} \PYG{n}{np}\PYG{o}{.}\PYG{n}{random}\PYG{o}{.}\PYG{n}{hypergeometric}\PYG{p}{(}\PYG{l+m+mi}{15}\PYG{p}{,} \PYG{l+m+mi}{15}\PYG{p}{,} \PYG{l+m+mi}{15}\PYG{p}{,} \PYG{l+m+mi}{100000}\PYG{p}{)}
\PYG{g+gp}{\PYGZgt{}\PYGZgt{}\PYGZgt{} }\PYG{n+nb}{sum}\PYG{p}{(}\PYG{n}{s}\PYG{o}{\PYGZgt{}}\PYG{o}{=}\PYG{l+m+mi}{12}\PYG{p}{)}\PYG{o}{/}\PYG{l+m+mf}{100000.} \PYG{o}{+} \PYG{n+nb}{sum}\PYG{p}{(}\PYG{n}{s}\PYG{o}{\PYGZlt{}}\PYG{o}{=}\PYG{l+m+mi}{3}\PYG{p}{)}\PYG{o}{/}\PYG{l+m+mf}{100000.}
\PYG{g+go}{\PYGZsh{}   answer = 0.003 ... pretty unlikely!}
\end{Verbatim}

\end{fulllineitems}

\index{laplace() (in module acsAttractorAnalysisInTime)}

\begin{fulllineitems}
\phantomsection\label{acsAttractorAnalysisInTime:acsAttractorAnalysisInTime.laplace}\pysiglinewithargsret{\code{acsAttractorAnalysisInTime.}\bfcode{laplace}}{\emph{loc=0.0}, \emph{scale=1.0}, \emph{size=None}}{}
Draw samples from the Laplace or double exponential distribution with
specified location (or mean) and scale (decay).

The Laplace distribution is similar to the Gaussian/normal distribution,
but is sharper at the peak and has fatter tails. It represents the
difference between two independent, identically distributed exponential
random variables.
\begin{description}
\item[{loc}] \leavevmode{[}float{]}
The position, \(\mu\), of the distribution peak.

\item[{scale}] \leavevmode{[}float{]}
\(\lambda\), the exponential decay.

\end{description}

It has the probability density function
\begin{gather}
\begin{split}f(x; \mu, \lambda) = \frac{1}{2\lambda}
\exp\left(-\frac{|x - \mu|}{\lambda}\right).\end{split}\notag
\end{gather}
The first law of Laplace, from 1774, states that the frequency of an error
can be expressed as an exponential function of the absolute magnitude of
the error, which leads to the Laplace distribution. For many problems in
Economics and Health sciences, this distribution seems to model the data
better than the standard Gaussian distribution

Draw samples from the distribution

\begin{Verbatim}[commandchars=\\\{\}]
\PYG{g+gp}{\PYGZgt{}\PYGZgt{}\PYGZgt{} }\PYG{n}{loc}\PYG{p}{,} \PYG{n}{scale} \PYG{o}{=} \PYG{l+m+mf}{0.}\PYG{p}{,} \PYG{l+m+mf}{1.}
\PYG{g+gp}{\PYGZgt{}\PYGZgt{}\PYGZgt{} }\PYG{n}{s} \PYG{o}{=} \PYG{n}{np}\PYG{o}{.}\PYG{n}{random}\PYG{o}{.}\PYG{n}{laplace}\PYG{p}{(}\PYG{n}{loc}\PYG{p}{,} \PYG{n}{scale}\PYG{p}{,} \PYG{l+m+mi}{1000}\PYG{p}{)}
\end{Verbatim}

Display the histogram of the samples, along with
the probability density function:

\begin{Verbatim}[commandchars=\\\{\}]
\PYG{g+gp}{\PYGZgt{}\PYGZgt{}\PYGZgt{} }\PYG{k+kn}{import} \PYG{n+nn}{matplotlib.pyplot} \PYG{k+kn}{as} \PYG{n+nn}{plt}
\PYG{g+gp}{\PYGZgt{}\PYGZgt{}\PYGZgt{} }\PYG{n}{count}\PYG{p}{,} \PYG{n}{bins}\PYG{p}{,} \PYG{n}{ignored} \PYG{o}{=} \PYG{n}{plt}\PYG{o}{.}\PYG{n}{hist}\PYG{p}{(}\PYG{n}{s}\PYG{p}{,} \PYG{l+m+mi}{30}\PYG{p}{,} \PYG{n}{normed}\PYG{o}{=}\PYG{n+nb+bp}{True}\PYG{p}{)}
\PYG{g+gp}{\PYGZgt{}\PYGZgt{}\PYGZgt{} }\PYG{n}{x} \PYG{o}{=} \PYG{n}{np}\PYG{o}{.}\PYG{n}{arange}\PYG{p}{(}\PYG{o}{\PYGZhy{}}\PYG{l+m+mf}{8.}\PYG{p}{,} \PYG{l+m+mf}{8.}\PYG{p}{,} \PYG{o}{.}\PYG{l+m+mo}{01}\PYG{p}{)}
\PYG{g+gp}{\PYGZgt{}\PYGZgt{}\PYGZgt{} }\PYG{n}{pdf} \PYG{o}{=} \PYG{n}{np}\PYG{o}{.}\PYG{n}{exp}\PYG{p}{(}\PYG{o}{\PYGZhy{}}\PYG{n+nb}{abs}\PYG{p}{(}\PYG{n}{x}\PYG{o}{\PYGZhy{}}\PYG{n}{loc}\PYG{o}{/}\PYG{n}{scale}\PYG{p}{)}\PYG{p}{)}\PYG{o}{/}\PYG{p}{(}\PYG{l+m+mf}{2.}\PYG{o}{*}\PYG{n}{scale}\PYG{p}{)}
\PYG{g+gp}{\PYGZgt{}\PYGZgt{}\PYGZgt{} }\PYG{n}{plt}\PYG{o}{.}\PYG{n}{plot}\PYG{p}{(}\PYG{n}{x}\PYG{p}{,} \PYG{n}{pdf}\PYG{p}{)}
\end{Verbatim}

Plot Gaussian for comparison:

\begin{Verbatim}[commandchars=\\\{\}]
\PYG{g+gp}{\PYGZgt{}\PYGZgt{}\PYGZgt{} }\PYG{n}{g} \PYG{o}{=} \PYG{p}{(}\PYG{l+m+mi}{1}\PYG{o}{/}\PYG{p}{(}\PYG{n}{scale} \PYG{o}{*} \PYG{n}{np}\PYG{o}{.}\PYG{n}{sqrt}\PYG{p}{(}\PYG{l+m+mi}{2} \PYG{o}{*} \PYG{n}{np}\PYG{o}{.}\PYG{n}{pi}\PYG{p}{)}\PYG{p}{)} \PYG{o}{*} 
\PYG{g+gp}{... }     \PYG{n}{np}\PYG{o}{.}\PYG{n}{exp}\PYG{p}{(} \PYG{o}{\PYGZhy{}} \PYG{p}{(}\PYG{n}{x} \PYG{o}{\PYGZhy{}} \PYG{n}{loc}\PYG{p}{)}\PYG{o}{*}\PYG{o}{*}\PYG{l+m+mi}{2} \PYG{o}{/} \PYG{p}{(}\PYG{l+m+mi}{2} \PYG{o}{*} \PYG{n}{scale}\PYG{o}{*}\PYG{o}{*}\PYG{l+m+mi}{2}\PYG{p}{)} \PYG{p}{)}\PYG{p}{)}
\PYG{g+gp}{\PYGZgt{}\PYGZgt{}\PYGZgt{} }\PYG{n}{plt}\PYG{o}{.}\PYG{n}{plot}\PYG{p}{(}\PYG{n}{x}\PYG{p}{,}\PYG{n}{g}\PYG{p}{)}
\end{Verbatim}

\end{fulllineitems}

\index{logistic() (in module acsAttractorAnalysisInTime)}

\begin{fulllineitems}
\phantomsection\label{acsAttractorAnalysisInTime:acsAttractorAnalysisInTime.logistic}\pysiglinewithargsret{\code{acsAttractorAnalysisInTime.}\bfcode{logistic}}{\emph{loc=0.0}, \emph{scale=1.0}, \emph{size=None}}{}
Draw samples from a Logistic distribution.

Samples are drawn from a Logistic distribution with specified
parameters, loc (location or mean, also median), and scale (\textgreater{}0).

loc : float

scale : float \textgreater{} 0.
\begin{description}
\item[{size}] \leavevmode{[}\{tuple, int\}{]}
Output shape.  If the given shape is, e.g., \code{(m, n, k)}, then
\code{m * n * k} samples are drawn.

\end{description}
\begin{description}
\item[{samples}] \leavevmode{[}\{ndarray, scalar\}{]}
where the values are all integers in  {[}0, n{]}.

\end{description}
\begin{description}
\item[{scipy.stats.distributions.logistic}] \leavevmode{[}probability density function,{]}
distribution or cumulative density function, etc.

\end{description}

The probability density for the Logistic distribution is
\begin{gather}
\begin{split}P(x) = P(x) = \frac{e^{-(x-\mu)/s}}{s(1+e^{-(x-\mu)/s})^2},\end{split}\notag
\end{gather}
where \(\mu\) = location and \(s\) = scale.

The Logistic distribution is used in Extreme Value problems where it
can act as a mixture of Gumbel distributions, in Epidemiology, and by
the World Chess Federation (FIDE) where it is used in the Elo ranking
system, assuming the performance of each player is a logistically
distributed random variable.

Draw samples from the distribution:

\begin{Verbatim}[commandchars=\\\{\}]
\PYG{g+gp}{\PYGZgt{}\PYGZgt{}\PYGZgt{} }\PYG{n}{loc}\PYG{p}{,} \PYG{n}{scale} \PYG{o}{=} \PYG{l+m+mi}{10}\PYG{p}{,} \PYG{l+m+mi}{1}
\PYG{g+gp}{\PYGZgt{}\PYGZgt{}\PYGZgt{} }\PYG{n}{s} \PYG{o}{=} \PYG{n}{np}\PYG{o}{.}\PYG{n}{random}\PYG{o}{.}\PYG{n}{logistic}\PYG{p}{(}\PYG{n}{loc}\PYG{p}{,} \PYG{n}{scale}\PYG{p}{,} \PYG{l+m+mi}{10000}\PYG{p}{)}
\PYG{g+gp}{\PYGZgt{}\PYGZgt{}\PYGZgt{} }\PYG{n}{count}\PYG{p}{,} \PYG{n}{bins}\PYG{p}{,} \PYG{n}{ignored} \PYG{o}{=} \PYG{n}{plt}\PYG{o}{.}\PYG{n}{hist}\PYG{p}{(}\PYG{n}{s}\PYG{p}{,} \PYG{n}{bins}\PYG{o}{=}\PYG{l+m+mi}{50}\PYG{p}{)}
\end{Verbatim}

\#   plot against distribution

\begin{Verbatim}[commandchars=\\\{\}]
\PYG{g+gp}{\PYGZgt{}\PYGZgt{}\PYGZgt{} }\PYG{k}{def} \PYG{n+nf}{logist}\PYG{p}{(}\PYG{n}{x}\PYG{p}{,} \PYG{n}{loc}\PYG{p}{,} \PYG{n}{scale}\PYG{p}{)}\PYG{p}{:}
\PYG{g+gp}{... }    \PYG{k}{return} \PYG{n}{exp}\PYG{p}{(}\PYG{p}{(}\PYG{n}{loc}\PYG{o}{\PYGZhy{}}\PYG{n}{x}\PYG{p}{)}\PYG{o}{/}\PYG{n}{scale}\PYG{p}{)}\PYG{o}{/}\PYG{p}{(}\PYG{n}{scale}\PYG{o}{*}\PYG{p}{(}\PYG{l+m+mi}{1}\PYG{o}{+}\PYG{n}{exp}\PYG{p}{(}\PYG{p}{(}\PYG{n}{loc}\PYG{o}{\PYGZhy{}}\PYG{n}{x}\PYG{p}{)}\PYG{o}{/}\PYG{n}{scale}\PYG{p}{)}\PYG{p}{)}\PYG{o}{*}\PYG{o}{*}\PYG{l+m+mi}{2}\PYG{p}{)}
\PYG{g+gp}{\PYGZgt{}\PYGZgt{}\PYGZgt{} }\PYG{n}{plt}\PYG{o}{.}\PYG{n}{plot}\PYG{p}{(}\PYG{n}{bins}\PYG{p}{,} \PYG{n}{logist}\PYG{p}{(}\PYG{n}{bins}\PYG{p}{,} \PYG{n}{loc}\PYG{p}{,} \PYG{n}{scale}\PYG{p}{)}\PYG{o}{*}\PYG{n}{count}\PYG{o}{.}\PYG{n}{max}\PYG{p}{(}\PYG{p}{)}\PYG{o}{/}\PYGZbs{}
\PYG{g+gp}{... }\PYG{n}{logist}\PYG{p}{(}\PYG{n}{bins}\PYG{p}{,} \PYG{n}{loc}\PYG{p}{,} \PYG{n}{scale}\PYG{p}{)}\PYG{o}{.}\PYG{n}{max}\PYG{p}{(}\PYG{p}{)}\PYG{p}{)}
\PYG{g+gp}{\PYGZgt{}\PYGZgt{}\PYGZgt{} }\PYG{n}{plt}\PYG{o}{.}\PYG{n}{show}\PYG{p}{(}\PYG{p}{)}
\end{Verbatim}

\end{fulllineitems}

\index{lognormal() (in module acsAttractorAnalysisInTime)}

\begin{fulllineitems}
\phantomsection\label{acsAttractorAnalysisInTime:acsAttractorAnalysisInTime.lognormal}\pysiglinewithargsret{\code{acsAttractorAnalysisInTime.}\bfcode{lognormal}}{\emph{mean=0.0}, \emph{sigma=1.0}, \emph{size=None}}{}
Return samples drawn from a log-normal distribution.

Draw samples from a log-normal distribution with specified mean,
standard deviation, and array shape.  Note that the mean and standard
deviation are not the values for the distribution itself, but of the
underlying normal distribution it is derived from.
\begin{description}
\item[{mean}] \leavevmode{[}float{]}
Mean value of the underlying normal distribution

\item[{sigma}] \leavevmode{[}float, \textgreater{} 0.{]}
Standard deviation of the underlying normal distribution

\item[{size}] \leavevmode{[}tuple of ints{]}
Output shape.  If the given shape is, e.g., \code{(m, n, k)}, then
\code{m * n * k} samples are drawn.

\end{description}
\begin{description}
\item[{samples}] \leavevmode{[}ndarray or float{]}
The desired samples. An array of the same shape as \emph{size} if given,
if \emph{size} is None a float is returned.

\end{description}
\begin{description}
\item[{scipy.stats.lognorm}] \leavevmode{[}probability density function, distribution,{]}
cumulative density function, etc.

\end{description}

A variable \emph{x} has a log-normal distribution if \emph{log(x)} is normally
distributed.  The probability density function for the log-normal
distribution is:
\begin{gather}
\begin{split}p(x) = \frac{1}{\sigma x \sqrt{2\pi}}
e^{(-\frac{(ln(x)-\mu)^2}{2\sigma^2})}\end{split}\notag
\end{gather}
where \(\mu\) is the mean and \(\sigma\) is the standard
deviation of the normally distributed logarithm of the variable.
A log-normal distribution results if a random variable is the \emph{product}
of a large number of independent, identically-distributed variables in
the same way that a normal distribution results if the variable is the
\emph{sum} of a large number of independent, identically-distributed
variables.

Limpert, E., Stahel, W. A., and Abbt, M., ``Log-normal Distributions
across the Sciences: Keys and Clues,'' \emph{BioScience}, Vol. 51, No. 5,
May, 2001.  \href{http://stat.ethz.ch/~stahel/lognormal/bioscience.pdf}{http://stat.ethz.ch/\textasciitilde{}stahel/lognormal/bioscience.pdf}

Reiss, R.D. and Thomas, M., \emph{Statistical Analysis of Extreme Values},
Basel: Birkhauser Verlag, 2001, pp. 31-32.

Draw samples from the distribution:

\begin{Verbatim}[commandchars=\\\{\}]
\PYG{g+gp}{\PYGZgt{}\PYGZgt{}\PYGZgt{} }\PYG{n}{mu}\PYG{p}{,} \PYG{n}{sigma} \PYG{o}{=} \PYG{l+m+mf}{3.}\PYG{p}{,} \PYG{l+m+mf}{1.} \PYG{c}{\PYGZsh{} mean and standard deviation}
\PYG{g+gp}{\PYGZgt{}\PYGZgt{}\PYGZgt{} }\PYG{n}{s} \PYG{o}{=} \PYG{n}{np}\PYG{o}{.}\PYG{n}{random}\PYG{o}{.}\PYG{n}{lognormal}\PYG{p}{(}\PYG{n}{mu}\PYG{p}{,} \PYG{n}{sigma}\PYG{p}{,} \PYG{l+m+mi}{1000}\PYG{p}{)}
\end{Verbatim}

Display the histogram of the samples, along with
the probability density function:

\begin{Verbatim}[commandchars=\\\{\}]
\PYG{g+gp}{\PYGZgt{}\PYGZgt{}\PYGZgt{} }\PYG{k+kn}{import} \PYG{n+nn}{matplotlib.pyplot} \PYG{k+kn}{as} \PYG{n+nn}{plt}
\PYG{g+gp}{\PYGZgt{}\PYGZgt{}\PYGZgt{} }\PYG{n}{count}\PYG{p}{,} \PYG{n}{bins}\PYG{p}{,} \PYG{n}{ignored} \PYG{o}{=} \PYG{n}{plt}\PYG{o}{.}\PYG{n}{hist}\PYG{p}{(}\PYG{n}{s}\PYG{p}{,} \PYG{l+m+mi}{100}\PYG{p}{,} \PYG{n}{normed}\PYG{o}{=}\PYG{n+nb+bp}{True}\PYG{p}{,} \PYG{n}{align}\PYG{o}{=}\PYG{l+s}{\PYGZsq{}}\PYG{l+s}{mid}\PYG{l+s}{\PYGZsq{}}\PYG{p}{)}
\end{Verbatim}

\begin{Verbatim}[commandchars=\\\{\}]
\PYG{g+gp}{\PYGZgt{}\PYGZgt{}\PYGZgt{} }\PYG{n}{x} \PYG{o}{=} \PYG{n}{np}\PYG{o}{.}\PYG{n}{linspace}\PYG{p}{(}\PYG{n+nb}{min}\PYG{p}{(}\PYG{n}{bins}\PYG{p}{)}\PYG{p}{,} \PYG{n+nb}{max}\PYG{p}{(}\PYG{n}{bins}\PYG{p}{)}\PYG{p}{,} \PYG{l+m+mi}{10000}\PYG{p}{)}
\PYG{g+gp}{\PYGZgt{}\PYGZgt{}\PYGZgt{} }\PYG{n}{pdf} \PYG{o}{=} \PYG{p}{(}\PYG{n}{np}\PYG{o}{.}\PYG{n}{exp}\PYG{p}{(}\PYG{o}{\PYGZhy{}}\PYG{p}{(}\PYG{n}{np}\PYG{o}{.}\PYG{n}{log}\PYG{p}{(}\PYG{n}{x}\PYG{p}{)} \PYG{o}{\PYGZhy{}} \PYG{n}{mu}\PYG{p}{)}\PYG{o}{*}\PYG{o}{*}\PYG{l+m+mi}{2} \PYG{o}{/} \PYG{p}{(}\PYG{l+m+mi}{2} \PYG{o}{*} \PYG{n}{sigma}\PYG{o}{*}\PYG{o}{*}\PYG{l+m+mi}{2}\PYG{p}{)}\PYG{p}{)}
\PYG{g+gp}{... }       \PYG{o}{/} \PYG{p}{(}\PYG{n}{x} \PYG{o}{*} \PYG{n}{sigma} \PYG{o}{*} \PYG{n}{np}\PYG{o}{.}\PYG{n}{sqrt}\PYG{p}{(}\PYG{l+m+mi}{2} \PYG{o}{*} \PYG{n}{np}\PYG{o}{.}\PYG{n}{pi}\PYG{p}{)}\PYG{p}{)}\PYG{p}{)}
\end{Verbatim}

\begin{Verbatim}[commandchars=\\\{\}]
\PYG{g+gp}{\PYGZgt{}\PYGZgt{}\PYGZgt{} }\PYG{n}{plt}\PYG{o}{.}\PYG{n}{plot}\PYG{p}{(}\PYG{n}{x}\PYG{p}{,} \PYG{n}{pdf}\PYG{p}{,} \PYG{n}{linewidth}\PYG{o}{=}\PYG{l+m+mi}{2}\PYG{p}{,} \PYG{n}{color}\PYG{o}{=}\PYG{l+s}{\PYGZsq{}}\PYG{l+s}{r}\PYG{l+s}{\PYGZsq{}}\PYG{p}{)}
\PYG{g+gp}{\PYGZgt{}\PYGZgt{}\PYGZgt{} }\PYG{n}{plt}\PYG{o}{.}\PYG{n}{axis}\PYG{p}{(}\PYG{l+s}{\PYGZsq{}}\PYG{l+s}{tight}\PYG{l+s}{\PYGZsq{}}\PYG{p}{)}
\PYG{g+gp}{\PYGZgt{}\PYGZgt{}\PYGZgt{} }\PYG{n}{plt}\PYG{o}{.}\PYG{n}{show}\PYG{p}{(}\PYG{p}{)}
\end{Verbatim}

Demonstrate that taking the products of random samples from a uniform
distribution can be fit well by a log-normal probability density function.

\begin{Verbatim}[commandchars=\\\{\}]
\PYG{g+gp}{\PYGZgt{}\PYGZgt{}\PYGZgt{} }\PYG{c}{\PYGZsh{} Generate a thousand samples: each is the product of 100 random}
\PYG{g+gp}{\PYGZgt{}\PYGZgt{}\PYGZgt{} }\PYG{c}{\PYGZsh{} values, drawn from a normal distribution.}
\PYG{g+gp}{\PYGZgt{}\PYGZgt{}\PYGZgt{} }\PYG{n}{b} \PYG{o}{=} \PYG{p}{[}\PYG{p}{]}
\PYG{g+gp}{\PYGZgt{}\PYGZgt{}\PYGZgt{} }\PYG{k}{for} \PYG{n}{i} \PYG{o+ow}{in} \PYG{n+nb}{range}\PYG{p}{(}\PYG{l+m+mi}{1000}\PYG{p}{)}\PYG{p}{:}
\PYG{g+gp}{... }   \PYG{n}{a} \PYG{o}{=} \PYG{l+m+mf}{10.} \PYG{o}{+} \PYG{n}{np}\PYG{o}{.}\PYG{n}{random}\PYG{o}{.}\PYG{n}{random}\PYG{p}{(}\PYG{l+m+mi}{100}\PYG{p}{)}
\PYG{g+gp}{... }   \PYG{n}{b}\PYG{o}{.}\PYG{n}{append}\PYG{p}{(}\PYG{n}{np}\PYG{o}{.}\PYG{n}{product}\PYG{p}{(}\PYG{n}{a}\PYG{p}{)}\PYG{p}{)}
\end{Verbatim}

\begin{Verbatim}[commandchars=\\\{\}]
\PYG{g+gp}{\PYGZgt{}\PYGZgt{}\PYGZgt{} }\PYG{n}{b} \PYG{o}{=} \PYG{n}{np}\PYG{o}{.}\PYG{n}{array}\PYG{p}{(}\PYG{n}{b}\PYG{p}{)} \PYG{o}{/} \PYG{n}{np}\PYG{o}{.}\PYG{n}{min}\PYG{p}{(}\PYG{n}{b}\PYG{p}{)} \PYG{c}{\PYGZsh{} scale values to be positive}
\PYG{g+gp}{\PYGZgt{}\PYGZgt{}\PYGZgt{} }\PYG{n}{count}\PYG{p}{,} \PYG{n}{bins}\PYG{p}{,} \PYG{n}{ignored} \PYG{o}{=} \PYG{n}{plt}\PYG{o}{.}\PYG{n}{hist}\PYG{p}{(}\PYG{n}{b}\PYG{p}{,} \PYG{l+m+mi}{100}\PYG{p}{,} \PYG{n}{normed}\PYG{o}{=}\PYG{n+nb+bp}{True}\PYG{p}{,} \PYG{n}{align}\PYG{o}{=}\PYG{l+s}{\PYGZsq{}}\PYG{l+s}{center}\PYG{l+s}{\PYGZsq{}}\PYG{p}{)}
\PYG{g+gp}{\PYGZgt{}\PYGZgt{}\PYGZgt{} }\PYG{n}{sigma} \PYG{o}{=} \PYG{n}{np}\PYG{o}{.}\PYG{n}{std}\PYG{p}{(}\PYG{n}{np}\PYG{o}{.}\PYG{n}{log}\PYG{p}{(}\PYG{n}{b}\PYG{p}{)}\PYG{p}{)}
\PYG{g+gp}{\PYGZgt{}\PYGZgt{}\PYGZgt{} }\PYG{n}{mu} \PYG{o}{=} \PYG{n}{np}\PYG{o}{.}\PYG{n}{mean}\PYG{p}{(}\PYG{n}{np}\PYG{o}{.}\PYG{n}{log}\PYG{p}{(}\PYG{n}{b}\PYG{p}{)}\PYG{p}{)}
\end{Verbatim}

\begin{Verbatim}[commandchars=\\\{\}]
\PYG{g+gp}{\PYGZgt{}\PYGZgt{}\PYGZgt{} }\PYG{n}{x} \PYG{o}{=} \PYG{n}{np}\PYG{o}{.}\PYG{n}{linspace}\PYG{p}{(}\PYG{n+nb}{min}\PYG{p}{(}\PYG{n}{bins}\PYG{p}{)}\PYG{p}{,} \PYG{n+nb}{max}\PYG{p}{(}\PYG{n}{bins}\PYG{p}{)}\PYG{p}{,} \PYG{l+m+mi}{10000}\PYG{p}{)}
\PYG{g+gp}{\PYGZgt{}\PYGZgt{}\PYGZgt{} }\PYG{n}{pdf} \PYG{o}{=} \PYG{p}{(}\PYG{n}{np}\PYG{o}{.}\PYG{n}{exp}\PYG{p}{(}\PYG{o}{\PYGZhy{}}\PYG{p}{(}\PYG{n}{np}\PYG{o}{.}\PYG{n}{log}\PYG{p}{(}\PYG{n}{x}\PYG{p}{)} \PYG{o}{\PYGZhy{}} \PYG{n}{mu}\PYG{p}{)}\PYG{o}{*}\PYG{o}{*}\PYG{l+m+mi}{2} \PYG{o}{/} \PYG{p}{(}\PYG{l+m+mi}{2} \PYG{o}{*} \PYG{n}{sigma}\PYG{o}{*}\PYG{o}{*}\PYG{l+m+mi}{2}\PYG{p}{)}\PYG{p}{)}
\PYG{g+gp}{... }       \PYG{o}{/} \PYG{p}{(}\PYG{n}{x} \PYG{o}{*} \PYG{n}{sigma} \PYG{o}{*} \PYG{n}{np}\PYG{o}{.}\PYG{n}{sqrt}\PYG{p}{(}\PYG{l+m+mi}{2} \PYG{o}{*} \PYG{n}{np}\PYG{o}{.}\PYG{n}{pi}\PYG{p}{)}\PYG{p}{)}\PYG{p}{)}
\end{Verbatim}

\begin{Verbatim}[commandchars=\\\{\}]
\PYG{g+gp}{\PYGZgt{}\PYGZgt{}\PYGZgt{} }\PYG{n}{plt}\PYG{o}{.}\PYG{n}{plot}\PYG{p}{(}\PYG{n}{x}\PYG{p}{,} \PYG{n}{pdf}\PYG{p}{,} \PYG{n}{color}\PYG{o}{=}\PYG{l+s}{\PYGZsq{}}\PYG{l+s}{r}\PYG{l+s}{\PYGZsq{}}\PYG{p}{,} \PYG{n}{linewidth}\PYG{o}{=}\PYG{l+m+mi}{2}\PYG{p}{)}
\PYG{g+gp}{\PYGZgt{}\PYGZgt{}\PYGZgt{} }\PYG{n}{plt}\PYG{o}{.}\PYG{n}{show}\PYG{p}{(}\PYG{p}{)}
\end{Verbatim}

\end{fulllineitems}

\index{logseries() (in module acsAttractorAnalysisInTime)}

\begin{fulllineitems}
\phantomsection\label{acsAttractorAnalysisInTime:acsAttractorAnalysisInTime.logseries}\pysiglinewithargsret{\code{acsAttractorAnalysisInTime.}\bfcode{logseries}}{\emph{p}, \emph{size=None}}{}
Draw samples from a Logarithmic Series distribution.

Samples are drawn from a Log Series distribution with specified
parameter, p (probability, 0 \textless{} p \textless{} 1).

loc : float

scale : float \textgreater{} 0.
\begin{description}
\item[{size}] \leavevmode{[}\{tuple, int\}{]}
Output shape.  If the given shape is, e.g., \code{(m, n, k)}, then
\code{m * n * k} samples are drawn.

\end{description}
\begin{description}
\item[{samples}] \leavevmode{[}\{ndarray, scalar\}{]}
where the values are all integers in  {[}0, n{]}.

\end{description}
\begin{description}
\item[{scipy.stats.distributions.logser}] \leavevmode{[}probability density function,{]}
distribution or cumulative density function, etc.

\end{description}

The probability density for the Log Series distribution is
\begin{gather}
\begin{split}P(k) = \frac{-p^k}{k \ln(1-p)},\end{split}\notag
\end{gather}
where p = probability.

The Log Series distribution is frequently used to represent species
richness and occurrence, first proposed by Fisher, Corbet, and
Williams in 1943 {[}2{]}.  It may also be used to model the numbers of
occupants seen in cars {[}3{]}.

Draw samples from the distribution:

\begin{Verbatim}[commandchars=\\\{\}]
\PYG{g+gp}{\PYGZgt{}\PYGZgt{}\PYGZgt{} }\PYG{n}{a} \PYG{o}{=} \PYG{o}{.}\PYG{l+m+mi}{6}
\PYG{g+gp}{\PYGZgt{}\PYGZgt{}\PYGZgt{} }\PYG{n}{s} \PYG{o}{=} \PYG{n}{np}\PYG{o}{.}\PYG{n}{random}\PYG{o}{.}\PYG{n}{logseries}\PYG{p}{(}\PYG{n}{a}\PYG{p}{,} \PYG{l+m+mi}{10000}\PYG{p}{)}
\PYG{g+gp}{\PYGZgt{}\PYGZgt{}\PYGZgt{} }\PYG{n}{count}\PYG{p}{,} \PYG{n}{bins}\PYG{p}{,} \PYG{n}{ignored} \PYG{o}{=} \PYG{n}{plt}\PYG{o}{.}\PYG{n}{hist}\PYG{p}{(}\PYG{n}{s}\PYG{p}{)}
\end{Verbatim}

\#   plot against distribution

\begin{Verbatim}[commandchars=\\\{\}]
\PYG{g+gp}{\PYGZgt{}\PYGZgt{}\PYGZgt{} }\PYG{k}{def} \PYG{n+nf}{logseries}\PYG{p}{(}\PYG{n}{k}\PYG{p}{,} \PYG{n}{p}\PYG{p}{)}\PYG{p}{:}
\PYG{g+gp}{... }    \PYG{k}{return} \PYG{o}{\PYGZhy{}}\PYG{n}{p}\PYG{o}{*}\PYG{o}{*}\PYG{n}{k}\PYG{o}{/}\PYG{p}{(}\PYG{n}{k}\PYG{o}{*}\PYG{n}{log}\PYG{p}{(}\PYG{l+m+mi}{1}\PYG{o}{\PYGZhy{}}\PYG{n}{p}\PYG{p}{)}\PYG{p}{)}
\PYG{g+gp}{\PYGZgt{}\PYGZgt{}\PYGZgt{} }\PYG{n}{plt}\PYG{o}{.}\PYG{n}{plot}\PYG{p}{(}\PYG{n}{bins}\PYG{p}{,} \PYG{n}{logseries}\PYG{p}{(}\PYG{n}{bins}\PYG{p}{,} \PYG{n}{a}\PYG{p}{)}\PYG{o}{*}\PYG{n}{count}\PYG{o}{.}\PYG{n}{max}\PYG{p}{(}\PYG{p}{)}\PYG{o}{/}
\PYG{g+go}{             logseries(bins, a).max(), \PYGZsq{}r\PYGZsq{})}
\PYG{g+gp}{\PYGZgt{}\PYGZgt{}\PYGZgt{} }\PYG{n}{plt}\PYG{o}{.}\PYG{n}{show}\PYG{p}{(}\PYG{p}{)}
\end{Verbatim}

\end{fulllineitems}

\index{multinomial() (in module acsAttractorAnalysisInTime)}

\begin{fulllineitems}
\phantomsection\label{acsAttractorAnalysisInTime:acsAttractorAnalysisInTime.multinomial}\pysiglinewithargsret{\code{acsAttractorAnalysisInTime.}\bfcode{multinomial}}{\emph{n}, \emph{pvals}, \emph{size=None}}{}
Draw samples from a multinomial distribution.

The multinomial distribution is a multivariate generalisation of the
binomial distribution.  Take an experiment with one of \code{p}
possible outcomes.  An example of such an experiment is throwing a dice,
where the outcome can be 1 through 6.  Each sample drawn from the
distribution represents \emph{n} such experiments.  Its values,
\code{X\_i = {[}X\_0, X\_1, ..., X\_p{]}}, represent the number of times the outcome
was \code{i}.
\begin{description}
\item[{n}] \leavevmode{[}int{]}
Number of experiments.

\item[{pvals}] \leavevmode{[}sequence of floats, length p{]}
Probabilities of each of the \code{p} different outcomes.  These
should sum to 1 (however, the last element is always assumed to
account for the remaining probability, as long as
\code{sum(pvals{[}:-1{]}) \textless{}= 1)}.

\item[{size}] \leavevmode{[}tuple of ints{]}
Given a \emph{size} of \code{(M, N, K)}, then \code{M*N*K} samples are drawn,
and the output shape becomes \code{(M, N, K, p)}, since each sample
has shape \code{(p,)}.

\end{description}

Throw a dice 20 times:

\begin{Verbatim}[commandchars=\\\{\}]
\PYG{g+gp}{\PYGZgt{}\PYGZgt{}\PYGZgt{} }\PYG{n}{np}\PYG{o}{.}\PYG{n}{random}\PYG{o}{.}\PYG{n}{multinomial}\PYG{p}{(}\PYG{l+m+mi}{20}\PYG{p}{,} \PYG{p}{[}\PYG{l+m+mi}{1}\PYG{o}{/}\PYG{l+m+mf}{6.}\PYG{p}{]}\PYG{o}{*}\PYG{l+m+mi}{6}\PYG{p}{,} \PYG{n}{size}\PYG{o}{=}\PYG{l+m+mi}{1}\PYG{p}{)}
\PYG{g+go}{array([[4, 1, 7, 5, 2, 1]])}
\end{Verbatim}

It landed 4 times on 1, once on 2, etc.

Now, throw the dice 20 times, and 20 times again:

\begin{Verbatim}[commandchars=\\\{\}]
\PYG{g+gp}{\PYGZgt{}\PYGZgt{}\PYGZgt{} }\PYG{n}{np}\PYG{o}{.}\PYG{n}{random}\PYG{o}{.}\PYG{n}{multinomial}\PYG{p}{(}\PYG{l+m+mi}{20}\PYG{p}{,} \PYG{p}{[}\PYG{l+m+mi}{1}\PYG{o}{/}\PYG{l+m+mf}{6.}\PYG{p}{]}\PYG{o}{*}\PYG{l+m+mi}{6}\PYG{p}{,} \PYG{n}{size}\PYG{o}{=}\PYG{l+m+mi}{2}\PYG{p}{)}
\PYG{g+go}{array([[3, 4, 3, 3, 4, 3],}
\PYG{g+go}{       [2, 4, 3, 4, 0, 7]])}
\end{Verbatim}

For the first run, we threw 3 times 1, 4 times 2, etc.  For the second,
we threw 2 times 1, 4 times 2, etc.

A loaded dice is more likely to land on number 6:

\begin{Verbatim}[commandchars=\\\{\}]
\PYG{g+gp}{\PYGZgt{}\PYGZgt{}\PYGZgt{} }\PYG{n}{np}\PYG{o}{.}\PYG{n}{random}\PYG{o}{.}\PYG{n}{multinomial}\PYG{p}{(}\PYG{l+m+mi}{100}\PYG{p}{,} \PYG{p}{[}\PYG{l+m+mi}{1}\PYG{o}{/}\PYG{l+m+mf}{7.}\PYG{p}{]}\PYG{o}{*}\PYG{l+m+mi}{5}\PYG{p}{)}
\PYG{g+go}{array([13, 16, 13, 16, 42])}
\end{Verbatim}

\end{fulllineitems}

\index{multivariate\_normal() (in module acsAttractorAnalysisInTime)}

\begin{fulllineitems}
\phantomsection\label{acsAttractorAnalysisInTime:acsAttractorAnalysisInTime.multivariate_normal}\pysiglinewithargsret{\code{acsAttractorAnalysisInTime.}\bfcode{multivariate\_normal}}{\emph{mean}, \emph{cov}\optional{, \emph{size}}}{}
Draw random samples from a multivariate normal distribution.

The multivariate normal, multinormal or Gaussian distribution is a
generalization of the one-dimensional normal distribution to higher
dimensions.  Such a distribution is specified by its mean and
covariance matrix.  These parameters are analogous to the mean
(average or ``center'') and variance (standard deviation, or ``width,''
squared) of the one-dimensional normal distribution.
\begin{description}
\item[{mean}] \leavevmode{[}1-D array\_like, of length N{]}
Mean of the N-dimensional distribution.

\item[{cov}] \leavevmode{[}2-D array\_like, of shape (N, N){]}
Covariance matrix of the distribution.  Must be symmetric and
positive semi-definite for ``physically meaningful'' results.

\item[{size}] \leavevmode{[}int or tuple of ints, optional{]}
Given a shape of, for example, \code{(m,n,k)}, \code{m*n*k} samples are
generated, and packed in an \emph{m}-by-\emph{n}-by-\emph{k} arrangement.  Because
each sample is \emph{N}-dimensional, the output shape is \code{(m,n,k,N)}.
If no shape is specified, a single (\emph{N}-D) sample is returned.

\end{description}
\begin{description}
\item[{out}] \leavevmode{[}ndarray{]}
The drawn samples, of shape \emph{size}, if that was provided.  If not,
the shape is \code{(N,)}.

In other words, each entry \code{out{[}i,j,...,:{]}} is an N-dimensional
value drawn from the distribution.

\end{description}

The mean is a coordinate in N-dimensional space, which represents the
location where samples are most likely to be generated.  This is
analogous to the peak of the bell curve for the one-dimensional or
univariate normal distribution.

Covariance indicates the level to which two variables vary together.
From the multivariate normal distribution, we draw N-dimensional
samples, \(X = [x_1, x_2, ... x_N]\).  The covariance matrix
element \(C_{ij}\) is the covariance of \(x_i\) and \(x_j\).
The element \(C_{ii}\) is the variance of \(x_i\) (i.e. its
``spread'').

Instead of specifying the full covariance matrix, popular
approximations include:
\begin{itemize}
\item {} 
Spherical covariance (\emph{cov} is a multiple of the identity matrix)

\item {} 
Diagonal covariance (\emph{cov} has non-negative elements, and only on
the diagonal)

\end{itemize}

This geometrical property can be seen in two dimensions by plotting
generated data-points:

\begin{Verbatim}[commandchars=\\\{\}]
\PYG{g+gp}{\PYGZgt{}\PYGZgt{}\PYGZgt{} }\PYG{n}{mean} \PYG{o}{=} \PYG{p}{[}\PYG{l+m+mi}{0}\PYG{p}{,}\PYG{l+m+mi}{0}\PYG{p}{]}
\PYG{g+gp}{\PYGZgt{}\PYGZgt{}\PYGZgt{} }\PYG{n}{cov} \PYG{o}{=} \PYG{p}{[}\PYG{p}{[}\PYG{l+m+mi}{1}\PYG{p}{,}\PYG{l+m+mi}{0}\PYG{p}{]}\PYG{p}{,}\PYG{p}{[}\PYG{l+m+mi}{0}\PYG{p}{,}\PYG{l+m+mi}{100}\PYG{p}{]}\PYG{p}{]} \PYG{c}{\PYGZsh{} diagonal covariance, points lie on x or y\PYGZhy{}axis}
\end{Verbatim}

\begin{Verbatim}[commandchars=\\\{\}]
\PYG{g+gp}{\PYGZgt{}\PYGZgt{}\PYGZgt{} }\PYG{k+kn}{import} \PYG{n+nn}{matplotlib.pyplot} \PYG{k+kn}{as} \PYG{n+nn}{plt}
\PYG{g+gp}{\PYGZgt{}\PYGZgt{}\PYGZgt{} }\PYG{n}{x}\PYG{p}{,}\PYG{n}{y} \PYG{o}{=} \PYG{n}{np}\PYG{o}{.}\PYG{n}{random}\PYG{o}{.}\PYG{n}{multivariate\PYGZus{}normal}\PYG{p}{(}\PYG{n}{mean}\PYG{p}{,}\PYG{n}{cov}\PYG{p}{,}\PYG{l+m+mi}{5000}\PYG{p}{)}\PYG{o}{.}\PYG{n}{T}
\PYG{g+gp}{\PYGZgt{}\PYGZgt{}\PYGZgt{} }\PYG{n}{plt}\PYG{o}{.}\PYG{n}{plot}\PYG{p}{(}\PYG{n}{x}\PYG{p}{,}\PYG{n}{y}\PYG{p}{,}\PYG{l+s}{\PYGZsq{}}\PYG{l+s}{x}\PYG{l+s}{\PYGZsq{}}\PYG{p}{)}\PYG{p}{;} \PYG{n}{plt}\PYG{o}{.}\PYG{n}{axis}\PYG{p}{(}\PYG{l+s}{\PYGZsq{}}\PYG{l+s}{equal}\PYG{l+s}{\PYGZsq{}}\PYG{p}{)}\PYG{p}{;} \PYG{n}{plt}\PYG{o}{.}\PYG{n}{show}\PYG{p}{(}\PYG{p}{)}
\end{Verbatim}

Note that the covariance matrix must be non-negative definite.

Papoulis, A., \emph{Probability, Random Variables, and Stochastic Processes},
3rd ed., New York: McGraw-Hill, 1991.

Duda, R. O., Hart, P. E., and Stork, D. G., \emph{Pattern Classification},
2nd ed., New York: Wiley, 2001.

\begin{Verbatim}[commandchars=\\\{\}]
\PYG{g+gp}{\PYGZgt{}\PYGZgt{}\PYGZgt{} }\PYG{n}{mean} \PYG{o}{=} \PYG{p}{(}\PYG{l+m+mi}{1}\PYG{p}{,}\PYG{l+m+mi}{2}\PYG{p}{)}
\PYG{g+gp}{\PYGZgt{}\PYGZgt{}\PYGZgt{} }\PYG{n}{cov} \PYG{o}{=} \PYG{p}{[}\PYG{p}{[}\PYG{l+m+mi}{1}\PYG{p}{,}\PYG{l+m+mi}{0}\PYG{p}{]}\PYG{p}{,}\PYG{p}{[}\PYG{l+m+mi}{1}\PYG{p}{,}\PYG{l+m+mi}{0}\PYG{p}{]}\PYG{p}{]}
\PYG{g+gp}{\PYGZgt{}\PYGZgt{}\PYGZgt{} }\PYG{n}{x} \PYG{o}{=} \PYG{n}{np}\PYG{o}{.}\PYG{n}{random}\PYG{o}{.}\PYG{n}{multivariate\PYGZus{}normal}\PYG{p}{(}\PYG{n}{mean}\PYG{p}{,}\PYG{n}{cov}\PYG{p}{,}\PYG{p}{(}\PYG{l+m+mi}{3}\PYG{p}{,}\PYG{l+m+mi}{3}\PYG{p}{)}\PYG{p}{)}
\PYG{g+gp}{\PYGZgt{}\PYGZgt{}\PYGZgt{} }\PYG{n}{x}\PYG{o}{.}\PYG{n}{shape}
\PYG{g+go}{(3, 3, 2)}
\end{Verbatim}

The following is probably true, given that 0.6 is roughly twice the
standard deviation:

\begin{Verbatim}[commandchars=\\\{\}]
\PYG{g+gp}{\PYGZgt{}\PYGZgt{}\PYGZgt{} }\PYG{k}{print} \PYG{n+nb}{list}\PYG{p}{(} \PYG{p}{(}\PYG{n}{x}\PYG{p}{[}\PYG{l+m+mi}{0}\PYG{p}{,}\PYG{l+m+mi}{0}\PYG{p}{,}\PYG{p}{:}\PYG{p}{]} \PYG{o}{\PYGZhy{}} \PYG{n}{mean}\PYG{p}{)} \PYG{o}{\PYGZlt{}} \PYG{l+m+mf}{0.6} \PYG{p}{)}
\PYG{g+go}{[True, True]}
\end{Verbatim}

\end{fulllineitems}

\index{negative\_binomial() (in module acsAttractorAnalysisInTime)}

\begin{fulllineitems}
\phantomsection\label{acsAttractorAnalysisInTime:acsAttractorAnalysisInTime.negative_binomial}\pysiglinewithargsret{\code{acsAttractorAnalysisInTime.}\bfcode{negative\_binomial}}{\emph{n}, \emph{p}, \emph{size=None}}{}
Draw samples from a negative\_binomial distribution.

Samples are drawn from a negative\_Binomial distribution with specified
parameters, \emph{n} trials and \emph{p} probability of success where \emph{n} is an
integer \textgreater{} 0 and \emph{p} is in the interval {[}0, 1{]}.
\begin{description}
\item[{n}] \leavevmode{[}int{]}
Parameter, \textgreater{} 0.

\item[{p}] \leavevmode{[}float{]}
Parameter, \textgreater{}= 0 and \textless{}=1.

\item[{size}] \leavevmode{[}int or tuple of ints{]}
Output shape. If the given shape is, e.g., \code{(m, n, k)}, then
\code{m * n * k} samples are drawn.

\end{description}
\begin{description}
\item[{samples}] \leavevmode{[}int or ndarray of ints{]}
Drawn samples.

\end{description}

The probability density for the Negative Binomial distribution is
\begin{gather}
\begin{split}P(N;n,p) = \binom{N+n-1}{n-1}p^{n}(1-p)^{N},\end{split}\notag
\end{gather}
where \(n-1\) is the number of successes, \(p\) is the probability
of success, and \(N+n-1\) is the number of trials.

The negative binomial distribution gives the probability of n-1 successes
and N failures in N+n-1 trials, and success on the (N+n)th trial.

If one throws a die repeatedly until the third time a ``1'' appears, then the
probability distribution of the number of non-``1''s that appear before the
third ``1'' is a negative binomial distribution.

Draw samples from the distribution:

A real world example. A company drills wild-cat oil exploration wells, each
with an estimated probability of success of 0.1.  What is the probability
of having one success for each successive well, that is what is the
probability of a single success after drilling 5 wells, after 6 wells,
etc.?

\begin{Verbatim}[commandchars=\\\{\}]
\PYG{g+gp}{\PYGZgt{}\PYGZgt{}\PYGZgt{} }\PYG{n}{s} \PYG{o}{=} \PYG{n}{np}\PYG{o}{.}\PYG{n}{random}\PYG{o}{.}\PYG{n}{negative\PYGZus{}binomial}\PYG{p}{(}\PYG{l+m+mi}{1}\PYG{p}{,} \PYG{l+m+mf}{0.1}\PYG{p}{,} \PYG{l+m+mi}{100000}\PYG{p}{)}
\PYG{g+gp}{\PYGZgt{}\PYGZgt{}\PYGZgt{} }\PYG{k}{for} \PYG{n}{i} \PYG{o+ow}{in} \PYG{n+nb}{range}\PYG{p}{(}\PYG{l+m+mi}{1}\PYG{p}{,} \PYG{l+m+mi}{11}\PYG{p}{)}\PYG{p}{:}
\PYG{g+gp}{... }   \PYG{n}{probability} \PYG{o}{=} \PYG{n+nb}{sum}\PYG{p}{(}\PYG{n}{s}\PYG{o}{\PYGZlt{}}\PYG{n}{i}\PYG{p}{)} \PYG{o}{/} \PYG{l+m+mf}{100000.}
\PYG{g+gp}{... }   \PYG{k}{print} \PYG{n}{i}\PYG{p}{,} \PYG{l+s}{\PYGZdq{}}\PYG{l+s}{wells drilled, probability of one success =}\PYG{l+s}{\PYGZdq{}}\PYG{p}{,} \PYG{n}{probability}
\end{Verbatim}

\end{fulllineitems}

\index{noncentral\_chisquare() (in module acsAttractorAnalysisInTime)}

\begin{fulllineitems}
\phantomsection\label{acsAttractorAnalysisInTime:acsAttractorAnalysisInTime.noncentral_chisquare}\pysiglinewithargsret{\code{acsAttractorAnalysisInTime.}\bfcode{noncentral\_chisquare}}{\emph{df}, \emph{nonc}, \emph{size=None}}{}
Draw samples from a noncentral chi-square distribution.

The noncentral \(\chi^2\) distribution is a generalisation of
the \(\chi^2\) distribution.
\begin{description}
\item[{df}] \leavevmode{[}int{]}
Degrees of freedom, should be \textgreater{}= 1.

\item[{nonc}] \leavevmode{[}float{]}
Non-centrality, should be \textgreater{} 0.

\item[{size}] \leavevmode{[}int or tuple of ints{]}
Shape of the output.

\end{description}

The probability density function for the noncentral Chi-square distribution
is
\begin{gather}
\begin{split}P(x;df,nonc) = \sum^{\infty}_{i=0}
\frac{e^{-nonc/2}(nonc/2)^{i}}{i!}P_{Y_{df+2i}}(x),\end{split}\notag
\end{gather}
where \(Y_{q}\) is the Chi-square with q degrees of freedom.

In Delhi (2007), it is noted that the noncentral chi-square is useful in
bombing and coverage problems, the probability of killing the point target
given by the noncentral chi-squared distribution.

Draw values from the distribution and plot the histogram

\begin{Verbatim}[commandchars=\\\{\}]
\PYG{g+gp}{\PYGZgt{}\PYGZgt{}\PYGZgt{} }\PYG{k+kn}{import} \PYG{n+nn}{matplotlib.pyplot} \PYG{k+kn}{as} \PYG{n+nn}{plt}
\PYG{g+gp}{\PYGZgt{}\PYGZgt{}\PYGZgt{} }\PYG{n}{values} \PYG{o}{=} \PYG{n}{plt}\PYG{o}{.}\PYG{n}{hist}\PYG{p}{(}\PYG{n}{np}\PYG{o}{.}\PYG{n}{random}\PYG{o}{.}\PYG{n}{noncentral\PYGZus{}chisquare}\PYG{p}{(}\PYG{l+m+mi}{3}\PYG{p}{,} \PYG{l+m+mi}{20}\PYG{p}{,} \PYG{l+m+mi}{100000}\PYG{p}{)}\PYG{p}{,}
\PYG{g+gp}{... }                  \PYG{n}{bins}\PYG{o}{=}\PYG{l+m+mi}{200}\PYG{p}{,} \PYG{n}{normed}\PYG{o}{=}\PYG{n+nb+bp}{True}\PYG{p}{)}
\PYG{g+gp}{\PYGZgt{}\PYGZgt{}\PYGZgt{} }\PYG{n}{plt}\PYG{o}{.}\PYG{n}{show}\PYG{p}{(}\PYG{p}{)}
\end{Verbatim}

Draw values from a noncentral chisquare with very small noncentrality,
and compare to a chisquare.

\begin{Verbatim}[commandchars=\\\{\}]
\PYG{g+gp}{\PYGZgt{}\PYGZgt{}\PYGZgt{} }\PYG{n}{plt}\PYG{o}{.}\PYG{n}{figure}\PYG{p}{(}\PYG{p}{)}
\PYG{g+gp}{\PYGZgt{}\PYGZgt{}\PYGZgt{} }\PYG{n}{values} \PYG{o}{=} \PYG{n}{plt}\PYG{o}{.}\PYG{n}{hist}\PYG{p}{(}\PYG{n}{np}\PYG{o}{.}\PYG{n}{random}\PYG{o}{.}\PYG{n}{noncentral\PYGZus{}chisquare}\PYG{p}{(}\PYG{l+m+mi}{3}\PYG{p}{,} \PYG{o}{.}\PYG{l+m+mo}{0000001}\PYG{p}{,} \PYG{l+m+mi}{100000}\PYG{p}{)}\PYG{p}{,}
\PYG{g+gp}{... }                  \PYG{n}{bins}\PYG{o}{=}\PYG{n}{np}\PYG{o}{.}\PYG{n}{arange}\PYG{p}{(}\PYG{l+m+mf}{0.}\PYG{p}{,} \PYG{l+m+mi}{25}\PYG{p}{,} \PYG{o}{.}\PYG{l+m+mi}{1}\PYG{p}{)}\PYG{p}{,} \PYG{n}{normed}\PYG{o}{=}\PYG{n+nb+bp}{True}\PYG{p}{)}
\PYG{g+gp}{\PYGZgt{}\PYGZgt{}\PYGZgt{} }\PYG{n}{values2} \PYG{o}{=} \PYG{n}{plt}\PYG{o}{.}\PYG{n}{hist}\PYG{p}{(}\PYG{n}{np}\PYG{o}{.}\PYG{n}{random}\PYG{o}{.}\PYG{n}{chisquare}\PYG{p}{(}\PYG{l+m+mi}{3}\PYG{p}{,} \PYG{l+m+mi}{100000}\PYG{p}{)}\PYG{p}{,}
\PYG{g+gp}{... }                   \PYG{n}{bins}\PYG{o}{=}\PYG{n}{np}\PYG{o}{.}\PYG{n}{arange}\PYG{p}{(}\PYG{l+m+mf}{0.}\PYG{p}{,} \PYG{l+m+mi}{25}\PYG{p}{,} \PYG{o}{.}\PYG{l+m+mi}{1}\PYG{p}{)}\PYG{p}{,} \PYG{n}{normed}\PYG{o}{=}\PYG{n+nb+bp}{True}\PYG{p}{)}
\PYG{g+gp}{\PYGZgt{}\PYGZgt{}\PYGZgt{} }\PYG{n}{plt}\PYG{o}{.}\PYG{n}{plot}\PYG{p}{(}\PYG{n}{values}\PYG{p}{[}\PYG{l+m+mi}{1}\PYG{p}{]}\PYG{p}{[}\PYG{l+m+mi}{0}\PYG{p}{:}\PYG{o}{\PYGZhy{}}\PYG{l+m+mi}{1}\PYG{p}{]}\PYG{p}{,} \PYG{n}{values}\PYG{p}{[}\PYG{l+m+mi}{0}\PYG{p}{]}\PYG{o}{\PYGZhy{}}\PYG{n}{values2}\PYG{p}{[}\PYG{l+m+mi}{0}\PYG{p}{]}\PYG{p}{,} \PYG{l+s}{\PYGZsq{}}\PYG{l+s}{ob}\PYG{l+s}{\PYGZsq{}}\PYG{p}{)}
\PYG{g+gp}{\PYGZgt{}\PYGZgt{}\PYGZgt{} }\PYG{n}{plt}\PYG{o}{.}\PYG{n}{show}\PYG{p}{(}\PYG{p}{)}
\end{Verbatim}

Demonstrate how large values of non-centrality lead to a more symmetric
distribution.

\begin{Verbatim}[commandchars=\\\{\}]
\PYG{g+gp}{\PYGZgt{}\PYGZgt{}\PYGZgt{} }\PYG{n}{plt}\PYG{o}{.}\PYG{n}{figure}\PYG{p}{(}\PYG{p}{)}
\PYG{g+gp}{\PYGZgt{}\PYGZgt{}\PYGZgt{} }\PYG{n}{values} \PYG{o}{=} \PYG{n}{plt}\PYG{o}{.}\PYG{n}{hist}\PYG{p}{(}\PYG{n}{np}\PYG{o}{.}\PYG{n}{random}\PYG{o}{.}\PYG{n}{noncentral\PYGZus{}chisquare}\PYG{p}{(}\PYG{l+m+mi}{3}\PYG{p}{,} \PYG{l+m+mi}{20}\PYG{p}{,} \PYG{l+m+mi}{100000}\PYG{p}{)}\PYG{p}{,}
\PYG{g+gp}{... }                  \PYG{n}{bins}\PYG{o}{=}\PYG{l+m+mi}{200}\PYG{p}{,} \PYG{n}{normed}\PYG{o}{=}\PYG{n+nb+bp}{True}\PYG{p}{)}
\PYG{g+gp}{\PYGZgt{}\PYGZgt{}\PYGZgt{} }\PYG{n}{plt}\PYG{o}{.}\PYG{n}{show}\PYG{p}{(}\PYG{p}{)}
\end{Verbatim}

\end{fulllineitems}

\index{noncentral\_f() (in module acsAttractorAnalysisInTime)}

\begin{fulllineitems}
\phantomsection\label{acsAttractorAnalysisInTime:acsAttractorAnalysisInTime.noncentral_f}\pysiglinewithargsret{\code{acsAttractorAnalysisInTime.}\bfcode{noncentral\_f}}{\emph{dfnum}, \emph{dfden}, \emph{nonc}, \emph{size=None}}{}
Draw samples from the noncentral F distribution.

Samples are drawn from an F distribution with specified parameters,
\emph{dfnum} (degrees of freedom in numerator) and \emph{dfden} (degrees of
freedom in denominator), where both parameters \textgreater{} 1.
\emph{nonc} is the non-centrality parameter.
\begin{description}
\item[{dfnum}] \leavevmode{[}int{]}
Parameter, should be \textgreater{} 1.

\item[{dfden}] \leavevmode{[}int{]}
Parameter, should be \textgreater{} 1.

\item[{nonc}] \leavevmode{[}float{]}
Parameter, should be \textgreater{}= 0.

\item[{size}] \leavevmode{[}int or tuple of ints{]}
Output shape. If the given shape is, e.g., \code{(m, n, k)}, then
\code{m * n * k} samples are drawn.

\end{description}
\begin{description}
\item[{samples}] \leavevmode{[}scalar or ndarray{]}
Drawn samples.

\end{description}

When calculating the power of an experiment (power = probability of
rejecting the null hypothesis when a specific alternative is true) the
non-central F statistic becomes important.  When the null hypothesis is
true, the F statistic follows a central F distribution. When the null
hypothesis is not true, then it follows a non-central F statistic.

Weisstein, Eric W. ``Noncentral F-Distribution.'' From MathWorld--A Wolfram
Web Resource.  \href{http://mathworld.wolfram.com/NoncentralF-Distribution.html}{http://mathworld.wolfram.com/NoncentralF-Distribution.html}

Wikipedia, ``Noncentral F distribution'',
\href{http://en.wikipedia.org/wiki/Noncentral\_F-distribution}{http://en.wikipedia.org/wiki/Noncentral\_F-distribution}

In a study, testing for a specific alternative to the null hypothesis
requires use of the Noncentral F distribution. We need to calculate the
area in the tail of the distribution that exceeds the value of the F
distribution for the null hypothesis.  We'll plot the two probability
distributions for comparison.

\begin{Verbatim}[commandchars=\\\{\}]
\PYG{g+gp}{\PYGZgt{}\PYGZgt{}\PYGZgt{} }\PYG{n}{dfnum} \PYG{o}{=} \PYG{l+m+mi}{3} \PYG{c}{\PYGZsh{} between group deg of freedom}
\PYG{g+gp}{\PYGZgt{}\PYGZgt{}\PYGZgt{} }\PYG{n}{dfden} \PYG{o}{=} \PYG{l+m+mi}{20} \PYG{c}{\PYGZsh{} within groups degrees of freedom}
\PYG{g+gp}{\PYGZgt{}\PYGZgt{}\PYGZgt{} }\PYG{n}{nonc} \PYG{o}{=} \PYG{l+m+mf}{3.0}
\PYG{g+gp}{\PYGZgt{}\PYGZgt{}\PYGZgt{} }\PYG{n}{nc\PYGZus{}vals} \PYG{o}{=} \PYG{n}{np}\PYG{o}{.}\PYG{n}{random}\PYG{o}{.}\PYG{n}{noncentral\PYGZus{}f}\PYG{p}{(}\PYG{n}{dfnum}\PYG{p}{,} \PYG{n}{dfden}\PYG{p}{,} \PYG{n}{nonc}\PYG{p}{,} \PYG{l+m+mi}{1000000}\PYG{p}{)}
\PYG{g+gp}{\PYGZgt{}\PYGZgt{}\PYGZgt{} }\PYG{n}{NF} \PYG{o}{=} \PYG{n}{np}\PYG{o}{.}\PYG{n}{histogram}\PYG{p}{(}\PYG{n}{nc\PYGZus{}vals}\PYG{p}{,} \PYG{n}{bins}\PYG{o}{=}\PYG{l+m+mi}{50}\PYG{p}{,} \PYG{n}{normed}\PYG{o}{=}\PYG{n+nb+bp}{True}\PYG{p}{)}
\PYG{g+gp}{\PYGZgt{}\PYGZgt{}\PYGZgt{} }\PYG{n}{c\PYGZus{}vals} \PYG{o}{=} \PYG{n}{np}\PYG{o}{.}\PYG{n}{random}\PYG{o}{.}\PYG{n}{f}\PYG{p}{(}\PYG{n}{dfnum}\PYG{p}{,} \PYG{n}{dfden}\PYG{p}{,} \PYG{l+m+mi}{1000000}\PYG{p}{)}
\PYG{g+gp}{\PYGZgt{}\PYGZgt{}\PYGZgt{} }\PYG{n}{F} \PYG{o}{=} \PYG{n}{np}\PYG{o}{.}\PYG{n}{histogram}\PYG{p}{(}\PYG{n}{c\PYGZus{}vals}\PYG{p}{,} \PYG{n}{bins}\PYG{o}{=}\PYG{l+m+mi}{50}\PYG{p}{,} \PYG{n}{normed}\PYG{o}{=}\PYG{n+nb+bp}{True}\PYG{p}{)}
\PYG{g+gp}{\PYGZgt{}\PYGZgt{}\PYGZgt{} }\PYG{n}{plt}\PYG{o}{.}\PYG{n}{plot}\PYG{p}{(}\PYG{n}{F}\PYG{p}{[}\PYG{l+m+mi}{1}\PYG{p}{]}\PYG{p}{[}\PYG{l+m+mi}{1}\PYG{p}{:}\PYG{p}{]}\PYG{p}{,} \PYG{n}{F}\PYG{p}{[}\PYG{l+m+mi}{0}\PYG{p}{]}\PYG{p}{)}
\PYG{g+gp}{\PYGZgt{}\PYGZgt{}\PYGZgt{} }\PYG{n}{plt}\PYG{o}{.}\PYG{n}{plot}\PYG{p}{(}\PYG{n}{NF}\PYG{p}{[}\PYG{l+m+mi}{1}\PYG{p}{]}\PYG{p}{[}\PYG{l+m+mi}{1}\PYG{p}{:}\PYG{p}{]}\PYG{p}{,} \PYG{n}{NF}\PYG{p}{[}\PYG{l+m+mi}{0}\PYG{p}{]}\PYG{p}{)}
\PYG{g+gp}{\PYGZgt{}\PYGZgt{}\PYGZgt{} }\PYG{n}{plt}\PYG{o}{.}\PYG{n}{show}\PYG{p}{(}\PYG{p}{)}
\end{Verbatim}

\end{fulllineitems}

\index{normal() (in module acsAttractorAnalysisInTime)}

\begin{fulllineitems}
\phantomsection\label{acsAttractorAnalysisInTime:acsAttractorAnalysisInTime.normal}\pysiglinewithargsret{\code{acsAttractorAnalysisInTime.}\bfcode{normal}}{\emph{loc=0.0}, \emph{scale=1.0}, \emph{size=None}}{}
Draw random samples from a normal (Gaussian) distribution.

The probability density function of the normal distribution, first
derived by De Moivre and 200 years later by both Gauss and Laplace
independently {\color{red}\bfseries{}{[}2{]}\_}, is often called the bell curve because of
its characteristic shape (see the example below).

The normal distributions occurs often in nature.  For example, it
describes the commonly occurring distribution of samples influenced
by a large number of tiny, random disturbances, each with its own
unique distribution {\color{red}\bfseries{}{[}2{]}\_}.
\begin{description}
\item[{loc}] \leavevmode{[}float{]}
Mean (``centre'') of the distribution.

\item[{scale}] \leavevmode{[}float{]}
Standard deviation (spread or ``width'') of the distribution.

\item[{size}] \leavevmode{[}tuple of ints{]}
Output shape.  If the given shape is, e.g., \code{(m, n, k)}, then
\code{m * n * k} samples are drawn.

\end{description}
\begin{description}
\item[{scipy.stats.distributions.norm}] \leavevmode{[}probability density function,{]}
distribution or cumulative density function, etc.

\end{description}

The probability density for the Gaussian distribution is
\begin{gather}
\begin{split}p(x) = \frac{1}{\sqrt{ 2 \pi \sigma^2 }}
e^{ - \frac{ (x - \mu)^2 } {2 \sigma^2} },\end{split}\notag
\end{gather}
where \(\mu\) is the mean and \(\sigma\) the standard deviation.
The square of the standard deviation, \(\sigma^2\), is called the
variance.

The function has its peak at the mean, and its ``spread'' increases with
the standard deviation (the function reaches 0.607 times its maximum at
\(x + \sigma\) and \(x - \sigma\) {\color{red}\bfseries{}{[}2{]}\_}).  This implies that
\emph{numpy.random.normal} is more likely to return samples lying close to the
mean, rather than those far away.

Draw samples from the distribution:

\begin{Verbatim}[commandchars=\\\{\}]
\PYG{g+gp}{\PYGZgt{}\PYGZgt{}\PYGZgt{} }\PYG{n}{mu}\PYG{p}{,} \PYG{n}{sigma} \PYG{o}{=} \PYG{l+m+mi}{0}\PYG{p}{,} \PYG{l+m+mf}{0.1} \PYG{c}{\PYGZsh{} mean and standard deviation}
\PYG{g+gp}{\PYGZgt{}\PYGZgt{}\PYGZgt{} }\PYG{n}{s} \PYG{o}{=} \PYG{n}{np}\PYG{o}{.}\PYG{n}{random}\PYG{o}{.}\PYG{n}{normal}\PYG{p}{(}\PYG{n}{mu}\PYG{p}{,} \PYG{n}{sigma}\PYG{p}{,} \PYG{l+m+mi}{1000}\PYG{p}{)}
\end{Verbatim}

Verify the mean and the variance:

\begin{Verbatim}[commandchars=\\\{\}]
\PYG{g+gp}{\PYGZgt{}\PYGZgt{}\PYGZgt{} }\PYG{n+nb}{abs}\PYG{p}{(}\PYG{n}{mu} \PYG{o}{\PYGZhy{}} \PYG{n}{np}\PYG{o}{.}\PYG{n}{mean}\PYG{p}{(}\PYG{n}{s}\PYG{p}{)}\PYG{p}{)} \PYG{o}{\PYGZlt{}} \PYG{l+m+mf}{0.01}
\PYG{g+go}{True}
\end{Verbatim}

\begin{Verbatim}[commandchars=\\\{\}]
\PYG{g+gp}{\PYGZgt{}\PYGZgt{}\PYGZgt{} }\PYG{n+nb}{abs}\PYG{p}{(}\PYG{n}{sigma} \PYG{o}{\PYGZhy{}} \PYG{n}{np}\PYG{o}{.}\PYG{n}{std}\PYG{p}{(}\PYG{n}{s}\PYG{p}{,} \PYG{n}{ddof}\PYG{o}{=}\PYG{l+m+mi}{1}\PYG{p}{)}\PYG{p}{)} \PYG{o}{\PYGZlt{}} \PYG{l+m+mf}{0.01}
\PYG{g+go}{True}
\end{Verbatim}

Display the histogram of the samples, along with
the probability density function:

\begin{Verbatim}[commandchars=\\\{\}]
\PYG{g+gp}{\PYGZgt{}\PYGZgt{}\PYGZgt{} }\PYG{k+kn}{import} \PYG{n+nn}{matplotlib.pyplot} \PYG{k+kn}{as} \PYG{n+nn}{plt}
\PYG{g+gp}{\PYGZgt{}\PYGZgt{}\PYGZgt{} }\PYG{n}{count}\PYG{p}{,} \PYG{n}{bins}\PYG{p}{,} \PYG{n}{ignored} \PYG{o}{=} \PYG{n}{plt}\PYG{o}{.}\PYG{n}{hist}\PYG{p}{(}\PYG{n}{s}\PYG{p}{,} \PYG{l+m+mi}{30}\PYG{p}{,} \PYG{n}{normed}\PYG{o}{=}\PYG{n+nb+bp}{True}\PYG{p}{)}
\PYG{g+gp}{\PYGZgt{}\PYGZgt{}\PYGZgt{} }\PYG{n}{plt}\PYG{o}{.}\PYG{n}{plot}\PYG{p}{(}\PYG{n}{bins}\PYG{p}{,} \PYG{l+m+mi}{1}\PYG{o}{/}\PYG{p}{(}\PYG{n}{sigma} \PYG{o}{*} \PYG{n}{np}\PYG{o}{.}\PYG{n}{sqrt}\PYG{p}{(}\PYG{l+m+mi}{2} \PYG{o}{*} \PYG{n}{np}\PYG{o}{.}\PYG{n}{pi}\PYG{p}{)}\PYG{p}{)} \PYG{o}{*}
\PYG{g+gp}{... }               \PYG{n}{np}\PYG{o}{.}\PYG{n}{exp}\PYG{p}{(} \PYG{o}{\PYGZhy{}} \PYG{p}{(}\PYG{n}{bins} \PYG{o}{\PYGZhy{}} \PYG{n}{mu}\PYG{p}{)}\PYG{o}{*}\PYG{o}{*}\PYG{l+m+mi}{2} \PYG{o}{/} \PYG{p}{(}\PYG{l+m+mi}{2} \PYG{o}{*} \PYG{n}{sigma}\PYG{o}{*}\PYG{o}{*}\PYG{l+m+mi}{2}\PYG{p}{)} \PYG{p}{)}\PYG{p}{,}
\PYG{g+gp}{... }         \PYG{n}{linewidth}\PYG{o}{=}\PYG{l+m+mi}{2}\PYG{p}{,} \PYG{n}{color}\PYG{o}{=}\PYG{l+s}{\PYGZsq{}}\PYG{l+s}{r}\PYG{l+s}{\PYGZsq{}}\PYG{p}{)}
\PYG{g+gp}{\PYGZgt{}\PYGZgt{}\PYGZgt{} }\PYG{n}{plt}\PYG{o}{.}\PYG{n}{show}\PYG{p}{(}\PYG{p}{)}
\end{Verbatim}

\end{fulllineitems}

\index{pareto() (in module acsAttractorAnalysisInTime)}

\begin{fulllineitems}
\phantomsection\label{acsAttractorAnalysisInTime:acsAttractorAnalysisInTime.pareto}\pysiglinewithargsret{\code{acsAttractorAnalysisInTime.}\bfcode{pareto}}{\emph{a}, \emph{size=None}}{}
Draw samples from a Pareto II or Lomax distribution with specified shape.

The Lomax or Pareto II distribution is a shifted Pareto distribution. The
classical Pareto distribution can be obtained from the Lomax distribution
by adding the location parameter m, see below. The smallest value of the
Lomax distribution is zero while for the classical Pareto distribution it
is m, where the standard Pareto distribution has location m=1.
Lomax can also be considered as a simplified version of the Generalized
Pareto distribution (available in SciPy), with the scale set to one and
the location set to zero.

The Pareto distribution must be greater than zero, and is unbounded above.
It is also known as the ``80-20 rule''.  In this distribution, 80 percent of
the weights are in the lowest 20 percent of the range, while the other 20
percent fill the remaining 80 percent of the range.
\begin{description}
\item[{shape}] \leavevmode{[}float, \textgreater{} 0.{]}
Shape of the distribution.

\item[{size}] \leavevmode{[}tuple of ints{]}
Output shape.  If the given shape is, e.g., \code{(m, n, k)}, then
\code{m * n * k} samples are drawn.

\end{description}
\begin{description}
\item[{scipy.stats.distributions.lomax.pdf}] \leavevmode{[}probability density function,{]}
distribution or cumulative density function, etc.

\item[{scipy.stats.distributions.genpareto.pdf}] \leavevmode{[}probability density function,{]}
distribution or cumulative density function, etc.

\end{description}

The probability density for the Pareto distribution is
\begin{gather}
\begin{split}p(x) = \frac{am^a}{x^{a+1}}\end{split}\notag
\end{gather}
where \(a\) is the shape and \(m\) the location

The Pareto distribution, named after the Italian economist Vilfredo Pareto,
is a power law probability distribution useful in many real world problems.
Outside the field of economics it is generally referred to as the Bradford
distribution. Pareto developed the distribution to describe the
distribution of wealth in an economy.  It has also found use in insurance,
web page access statistics, oil field sizes, and many other problems,
including the download frequency for projects in Sourceforge {[}1{]}.  It is
one of the so-called ``fat-tailed'' distributions.

Draw samples from the distribution:

\begin{Verbatim}[commandchars=\\\{\}]
\PYG{g+gp}{\PYGZgt{}\PYGZgt{}\PYGZgt{} }\PYG{n}{a}\PYG{p}{,} \PYG{n}{m} \PYG{o}{=} \PYG{l+m+mf}{3.}\PYG{p}{,} \PYG{l+m+mf}{1.} \PYG{c}{\PYGZsh{} shape and mode}
\PYG{g+gp}{\PYGZgt{}\PYGZgt{}\PYGZgt{} }\PYG{n}{s} \PYG{o}{=} \PYG{n}{np}\PYG{o}{.}\PYG{n}{random}\PYG{o}{.}\PYG{n}{pareto}\PYG{p}{(}\PYG{n}{a}\PYG{p}{,} \PYG{l+m+mi}{1000}\PYG{p}{)} \PYG{o}{+} \PYG{n}{m}
\end{Verbatim}

Display the histogram of the samples, along with
the probability density function:

\begin{Verbatim}[commandchars=\\\{\}]
\PYG{g+gp}{\PYGZgt{}\PYGZgt{}\PYGZgt{} }\PYG{k+kn}{import} \PYG{n+nn}{matplotlib.pyplot} \PYG{k+kn}{as} \PYG{n+nn}{plt}
\PYG{g+gp}{\PYGZgt{}\PYGZgt{}\PYGZgt{} }\PYG{n}{count}\PYG{p}{,} \PYG{n}{bins}\PYG{p}{,} \PYG{n}{ignored} \PYG{o}{=} \PYG{n}{plt}\PYG{o}{.}\PYG{n}{hist}\PYG{p}{(}\PYG{n}{s}\PYG{p}{,} \PYG{l+m+mi}{100}\PYG{p}{,} \PYG{n}{normed}\PYG{o}{=}\PYG{n+nb+bp}{True}\PYG{p}{,} \PYG{n}{align}\PYG{o}{=}\PYG{l+s}{\PYGZsq{}}\PYG{l+s}{center}\PYG{l+s}{\PYGZsq{}}\PYG{p}{)}
\PYG{g+gp}{\PYGZgt{}\PYGZgt{}\PYGZgt{} }\PYG{n}{fit} \PYG{o}{=} \PYG{n}{a}\PYG{o}{*}\PYG{n}{m}\PYG{o}{*}\PYG{o}{*}\PYG{n}{a}\PYG{o}{/}\PYG{n}{bins}\PYG{o}{*}\PYG{o}{*}\PYG{p}{(}\PYG{n}{a}\PYG{o}{+}\PYG{l+m+mi}{1}\PYG{p}{)}
\PYG{g+gp}{\PYGZgt{}\PYGZgt{}\PYGZgt{} }\PYG{n}{plt}\PYG{o}{.}\PYG{n}{plot}\PYG{p}{(}\PYG{n}{bins}\PYG{p}{,} \PYG{n+nb}{max}\PYG{p}{(}\PYG{n}{count}\PYG{p}{)}\PYG{o}{*}\PYG{n}{fit}\PYG{o}{/}\PYG{n+nb}{max}\PYG{p}{(}\PYG{n}{fit}\PYG{p}{)}\PYG{p}{,}\PYG{n}{linewidth}\PYG{o}{=}\PYG{l+m+mi}{2}\PYG{p}{,} \PYG{n}{color}\PYG{o}{=}\PYG{l+s}{\PYGZsq{}}\PYG{l+s}{r}\PYG{l+s}{\PYGZsq{}}\PYG{p}{)}
\PYG{g+gp}{\PYGZgt{}\PYGZgt{}\PYGZgt{} }\PYG{n}{plt}\PYG{o}{.}\PYG{n}{show}\PYG{p}{(}\PYG{p}{)}
\end{Verbatim}

\end{fulllineitems}

\index{permutation() (in module acsAttractorAnalysisInTime)}

\begin{fulllineitems}
\phantomsection\label{acsAttractorAnalysisInTime:acsAttractorAnalysisInTime.permutation}\pysiglinewithargsret{\code{acsAttractorAnalysisInTime.}\bfcode{permutation}}{\emph{x}}{}
Randomly permute a sequence, or return a permuted range.

If \emph{x} is a multi-dimensional array, it is only shuffled along its
first index.
\begin{description}
\item[{x}] \leavevmode{[}int or array\_like{]}
If \emph{x} is an integer, randomly permute \code{np.arange(x)}.
If \emph{x} is an array, make a copy and shuffle the elements
randomly.

\end{description}
\begin{description}
\item[{out}] \leavevmode{[}ndarray{]}
Permuted sequence or array range.

\end{description}

\begin{Verbatim}[commandchars=\\\{\}]
\PYG{g+gp}{\PYGZgt{}\PYGZgt{}\PYGZgt{} }\PYG{n}{np}\PYG{o}{.}\PYG{n}{random}\PYG{o}{.}\PYG{n}{permutation}\PYG{p}{(}\PYG{l+m+mi}{10}\PYG{p}{)}
\PYG{g+go}{array([1, 7, 4, 3, 0, 9, 2, 5, 8, 6])}
\end{Verbatim}

\begin{Verbatim}[commandchars=\\\{\}]
\PYG{g+gp}{\PYGZgt{}\PYGZgt{}\PYGZgt{} }\PYG{n}{np}\PYG{o}{.}\PYG{n}{random}\PYG{o}{.}\PYG{n}{permutation}\PYG{p}{(}\PYG{p}{[}\PYG{l+m+mi}{1}\PYG{p}{,} \PYG{l+m+mi}{4}\PYG{p}{,} \PYG{l+m+mi}{9}\PYG{p}{,} \PYG{l+m+mi}{12}\PYG{p}{,} \PYG{l+m+mi}{15}\PYG{p}{]}\PYG{p}{)}
\PYG{g+go}{array([15,  1,  9,  4, 12])}
\end{Verbatim}

\begin{Verbatim}[commandchars=\\\{\}]
\PYG{g+gp}{\PYGZgt{}\PYGZgt{}\PYGZgt{} }\PYG{n}{arr} \PYG{o}{=} \PYG{n}{np}\PYG{o}{.}\PYG{n}{arange}\PYG{p}{(}\PYG{l+m+mi}{9}\PYG{p}{)}\PYG{o}{.}\PYG{n}{reshape}\PYG{p}{(}\PYG{p}{(}\PYG{l+m+mi}{3}\PYG{p}{,} \PYG{l+m+mi}{3}\PYG{p}{)}\PYG{p}{)}
\PYG{g+gp}{\PYGZgt{}\PYGZgt{}\PYGZgt{} }\PYG{n}{np}\PYG{o}{.}\PYG{n}{random}\PYG{o}{.}\PYG{n}{permutation}\PYG{p}{(}\PYG{n}{arr}\PYG{p}{)}
\PYG{g+go}{array([[6, 7, 8],}
\PYG{g+go}{       [0, 1, 2],}
\PYG{g+go}{       [3, 4, 5]])}
\end{Verbatim}

\end{fulllineitems}

\index{poisson() (in module acsAttractorAnalysisInTime)}

\begin{fulllineitems}
\phantomsection\label{acsAttractorAnalysisInTime:acsAttractorAnalysisInTime.poisson}\pysiglinewithargsret{\code{acsAttractorAnalysisInTime.}\bfcode{poisson}}{\emph{lam=1.0}, \emph{size=None}}{}
Draw samples from a Poisson distribution.

The Poisson distribution is the limit of the Binomial
distribution for large N.
\begin{description}
\item[{lam}] \leavevmode{[}float{]}
Expectation of interval, should be \textgreater{}= 0.

\item[{size}] \leavevmode{[}int or tuple of ints, optional{]}
Output shape. If the given shape is, e.g., \code{(m, n, k)}, then
\code{m * n * k} samples are drawn.

\end{description}

The Poisson distribution
\begin{gather}
\begin{split}f(k; \lambda)=\frac{\lambda^k e^{-\lambda}}{k!}\end{split}\notag
\end{gather}
For events with an expected separation \(\lambda\) the Poisson
distribution \(f(k; \lambda)\) describes the probability of
\(k\) events occurring within the observed interval \(\lambda\).

Because the output is limited to the range of the C long type, a
ValueError is raised when \emph{lam} is within 10 sigma of the maximum
representable value.

Draw samples from the distribution:

\begin{Verbatim}[commandchars=\\\{\}]
\PYG{g+gp}{\PYGZgt{}\PYGZgt{}\PYGZgt{} }\PYG{k+kn}{import} \PYG{n+nn}{numpy} \PYG{k+kn}{as} \PYG{n+nn}{np}
\PYG{g+gp}{\PYGZgt{}\PYGZgt{}\PYGZgt{} }\PYG{n}{s} \PYG{o}{=} \PYG{n}{np}\PYG{o}{.}\PYG{n}{random}\PYG{o}{.}\PYG{n}{poisson}\PYG{p}{(}\PYG{l+m+mi}{5}\PYG{p}{,} \PYG{l+m+mi}{10000}\PYG{p}{)}
\end{Verbatim}

Display histogram of the sample:

\begin{Verbatim}[commandchars=\\\{\}]
\PYG{g+gp}{\PYGZgt{}\PYGZgt{}\PYGZgt{} }\PYG{k+kn}{import} \PYG{n+nn}{matplotlib.pyplot} \PYG{k+kn}{as} \PYG{n+nn}{plt}
\PYG{g+gp}{\PYGZgt{}\PYGZgt{}\PYGZgt{} }\PYG{n}{count}\PYG{p}{,} \PYG{n}{bins}\PYG{p}{,} \PYG{n}{ignored} \PYG{o}{=} \PYG{n}{plt}\PYG{o}{.}\PYG{n}{hist}\PYG{p}{(}\PYG{n}{s}\PYG{p}{,} \PYG{l+m+mi}{14}\PYG{p}{,} \PYG{n}{normed}\PYG{o}{=}\PYG{n+nb+bp}{True}\PYG{p}{)}
\PYG{g+gp}{\PYGZgt{}\PYGZgt{}\PYGZgt{} }\PYG{n}{plt}\PYG{o}{.}\PYG{n}{show}\PYG{p}{(}\PYG{p}{)}
\end{Verbatim}

\end{fulllineitems}

\index{power() (in module acsAttractorAnalysisInTime)}

\begin{fulllineitems}
\phantomsection\label{acsAttractorAnalysisInTime:acsAttractorAnalysisInTime.power}\pysiglinewithargsret{\code{acsAttractorAnalysisInTime.}\bfcode{power}}{\emph{a}, \emph{size=None}}{}
Draws samples in {[}0, 1{]} from a power distribution with positive
exponent a - 1.

Also known as the power function distribution.
\begin{description}
\item[{a}] \leavevmode{[}float{]}
parameter, \textgreater{} 0

\item[{size}] \leavevmode{[}tuple of ints{]}\begin{description}
\item[{Output shape.  If the given shape is, e.g., \code{(m, n, k)}, then}] \leavevmode
\code{m * n * k} samples are drawn.

\end{description}

\end{description}
\begin{description}
\item[{samples}] \leavevmode{[}\{ndarray, scalar\}{]}
The returned samples lie in {[}0, 1{]}.

\end{description}
\begin{description}
\item[{ValueError}] \leavevmode
If a\textless{}1.

\end{description}

The probability density function is
\begin{gather}
\begin{split}P(x; a) = ax^{a-1}, 0 \le x \le 1, a>0.\end{split}\notag
\end{gather}
The power function distribution is just the inverse of the Pareto
distribution. It may also be seen as a special case of the Beta
distribution.

It is used, for example, in modeling the over-reporting of insurance
claims.

Draw samples from the distribution:

\begin{Verbatim}[commandchars=\\\{\}]
\PYG{g+gp}{\PYGZgt{}\PYGZgt{}\PYGZgt{} }\PYG{n}{a} \PYG{o}{=} \PYG{l+m+mf}{5.} \PYG{c}{\PYGZsh{} shape}
\PYG{g+gp}{\PYGZgt{}\PYGZgt{}\PYGZgt{} }\PYG{n}{samples} \PYG{o}{=} \PYG{l+m+mi}{1000}
\PYG{g+gp}{\PYGZgt{}\PYGZgt{}\PYGZgt{} }\PYG{n}{s} \PYG{o}{=} \PYG{n}{np}\PYG{o}{.}\PYG{n}{random}\PYG{o}{.}\PYG{n}{power}\PYG{p}{(}\PYG{n}{a}\PYG{p}{,} \PYG{n}{samples}\PYG{p}{)}
\end{Verbatim}

Display the histogram of the samples, along with
the probability density function:

\begin{Verbatim}[commandchars=\\\{\}]
\PYG{g+gp}{\PYGZgt{}\PYGZgt{}\PYGZgt{} }\PYG{k+kn}{import} \PYG{n+nn}{matplotlib.pyplot} \PYG{k+kn}{as} \PYG{n+nn}{plt}
\PYG{g+gp}{\PYGZgt{}\PYGZgt{}\PYGZgt{} }\PYG{n}{count}\PYG{p}{,} \PYG{n}{bins}\PYG{p}{,} \PYG{n}{ignored} \PYG{o}{=} \PYG{n}{plt}\PYG{o}{.}\PYG{n}{hist}\PYG{p}{(}\PYG{n}{s}\PYG{p}{,} \PYG{n}{bins}\PYG{o}{=}\PYG{l+m+mi}{30}\PYG{p}{)}
\PYG{g+gp}{\PYGZgt{}\PYGZgt{}\PYGZgt{} }\PYG{n}{x} \PYG{o}{=} \PYG{n}{np}\PYG{o}{.}\PYG{n}{linspace}\PYG{p}{(}\PYG{l+m+mi}{0}\PYG{p}{,} \PYG{l+m+mi}{1}\PYG{p}{,} \PYG{l+m+mi}{100}\PYG{p}{)}
\PYG{g+gp}{\PYGZgt{}\PYGZgt{}\PYGZgt{} }\PYG{n}{y} \PYG{o}{=} \PYG{n}{a}\PYG{o}{*}\PYG{n}{x}\PYG{o}{*}\PYG{o}{*}\PYG{p}{(}\PYG{n}{a}\PYG{o}{\PYGZhy{}}\PYG{l+m+mf}{1.}\PYG{p}{)}
\PYG{g+gp}{\PYGZgt{}\PYGZgt{}\PYGZgt{} }\PYG{n}{normed\PYGZus{}y} \PYG{o}{=} \PYG{n}{samples}\PYG{o}{*}\PYG{n}{np}\PYG{o}{.}\PYG{n}{diff}\PYG{p}{(}\PYG{n}{bins}\PYG{p}{)}\PYG{p}{[}\PYG{l+m+mi}{0}\PYG{p}{]}\PYG{o}{*}\PYG{n}{y}
\PYG{g+gp}{\PYGZgt{}\PYGZgt{}\PYGZgt{} }\PYG{n}{plt}\PYG{o}{.}\PYG{n}{plot}\PYG{p}{(}\PYG{n}{x}\PYG{p}{,} \PYG{n}{normed\PYGZus{}y}\PYG{p}{)}
\PYG{g+gp}{\PYGZgt{}\PYGZgt{}\PYGZgt{} }\PYG{n}{plt}\PYG{o}{.}\PYG{n}{show}\PYG{p}{(}\PYG{p}{)}
\end{Verbatim}

Compare the power function distribution to the inverse of the Pareto.

\begin{Verbatim}[commandchars=\\\{\}]
\PYG{g+gp}{\PYGZgt{}\PYGZgt{}\PYGZgt{} }\PYG{k+kn}{from} \PYG{n+nn}{scipy} \PYG{k+kn}{import} \PYG{n}{stats}
\PYG{g+gp}{\PYGZgt{}\PYGZgt{}\PYGZgt{} }\PYG{n}{rvs} \PYG{o}{=} \PYG{n}{np}\PYG{o}{.}\PYG{n}{random}\PYG{o}{.}\PYG{n}{power}\PYG{p}{(}\PYG{l+m+mi}{5}\PYG{p}{,} \PYG{l+m+mi}{1000000}\PYG{p}{)}
\PYG{g+gp}{\PYGZgt{}\PYGZgt{}\PYGZgt{} }\PYG{n}{rvsp} \PYG{o}{=} \PYG{n}{np}\PYG{o}{.}\PYG{n}{random}\PYG{o}{.}\PYG{n}{pareto}\PYG{p}{(}\PYG{l+m+mi}{5}\PYG{p}{,} \PYG{l+m+mi}{1000000}\PYG{p}{)}
\PYG{g+gp}{\PYGZgt{}\PYGZgt{}\PYGZgt{} }\PYG{n}{xx} \PYG{o}{=} \PYG{n}{np}\PYG{o}{.}\PYG{n}{linspace}\PYG{p}{(}\PYG{l+m+mi}{0}\PYG{p}{,}\PYG{l+m+mi}{1}\PYG{p}{,}\PYG{l+m+mi}{100}\PYG{p}{)}
\PYG{g+gp}{\PYGZgt{}\PYGZgt{}\PYGZgt{} }\PYG{n}{powpdf} \PYG{o}{=} \PYG{n}{stats}\PYG{o}{.}\PYG{n}{powerlaw}\PYG{o}{.}\PYG{n}{pdf}\PYG{p}{(}\PYG{n}{xx}\PYG{p}{,}\PYG{l+m+mi}{5}\PYG{p}{)}
\end{Verbatim}

\begin{Verbatim}[commandchars=\\\{\}]
\PYG{g+gp}{\PYGZgt{}\PYGZgt{}\PYGZgt{} }\PYG{n}{plt}\PYG{o}{.}\PYG{n}{figure}\PYG{p}{(}\PYG{p}{)}
\PYG{g+gp}{\PYGZgt{}\PYGZgt{}\PYGZgt{} }\PYG{n}{plt}\PYG{o}{.}\PYG{n}{hist}\PYG{p}{(}\PYG{n}{rvs}\PYG{p}{,} \PYG{n}{bins}\PYG{o}{=}\PYG{l+m+mi}{50}\PYG{p}{,} \PYG{n}{normed}\PYG{o}{=}\PYG{n+nb+bp}{True}\PYG{p}{)}
\PYG{g+gp}{\PYGZgt{}\PYGZgt{}\PYGZgt{} }\PYG{n}{plt}\PYG{o}{.}\PYG{n}{plot}\PYG{p}{(}\PYG{n}{xx}\PYG{p}{,}\PYG{n}{powpdf}\PYG{p}{,}\PYG{l+s}{\PYGZsq{}}\PYG{l+s}{r\PYGZhy{}}\PYG{l+s}{\PYGZsq{}}\PYG{p}{)}
\PYG{g+gp}{\PYGZgt{}\PYGZgt{}\PYGZgt{} }\PYG{n}{plt}\PYG{o}{.}\PYG{n}{title}\PYG{p}{(}\PYG{l+s}{\PYGZsq{}}\PYG{l+s}{np.random.power(5)}\PYG{l+s}{\PYGZsq{}}\PYG{p}{)}
\end{Verbatim}

\begin{Verbatim}[commandchars=\\\{\}]
\PYG{g+gp}{\PYGZgt{}\PYGZgt{}\PYGZgt{} }\PYG{n}{plt}\PYG{o}{.}\PYG{n}{figure}\PYG{p}{(}\PYG{p}{)}
\PYG{g+gp}{\PYGZgt{}\PYGZgt{}\PYGZgt{} }\PYG{n}{plt}\PYG{o}{.}\PYG{n}{hist}\PYG{p}{(}\PYG{l+m+mf}{1.}\PYG{o}{/}\PYG{p}{(}\PYG{l+m+mf}{1.}\PYG{o}{+}\PYG{n}{rvsp}\PYG{p}{)}\PYG{p}{,} \PYG{n}{bins}\PYG{o}{=}\PYG{l+m+mi}{50}\PYG{p}{,} \PYG{n}{normed}\PYG{o}{=}\PYG{n+nb+bp}{True}\PYG{p}{)}
\PYG{g+gp}{\PYGZgt{}\PYGZgt{}\PYGZgt{} }\PYG{n}{plt}\PYG{o}{.}\PYG{n}{plot}\PYG{p}{(}\PYG{n}{xx}\PYG{p}{,}\PYG{n}{powpdf}\PYG{p}{,}\PYG{l+s}{\PYGZsq{}}\PYG{l+s}{r\PYGZhy{}}\PYG{l+s}{\PYGZsq{}}\PYG{p}{)}
\PYG{g+gp}{\PYGZgt{}\PYGZgt{}\PYGZgt{} }\PYG{n}{plt}\PYG{o}{.}\PYG{n}{title}\PYG{p}{(}\PYG{l+s}{\PYGZsq{}}\PYG{l+s}{inverse of 1 + np.random.pareto(5)}\PYG{l+s}{\PYGZsq{}}\PYG{p}{)}
\end{Verbatim}

\begin{Verbatim}[commandchars=\\\{\}]
\PYG{g+gp}{\PYGZgt{}\PYGZgt{}\PYGZgt{} }\PYG{n}{plt}\PYG{o}{.}\PYG{n}{figure}\PYG{p}{(}\PYG{p}{)}
\PYG{g+gp}{\PYGZgt{}\PYGZgt{}\PYGZgt{} }\PYG{n}{plt}\PYG{o}{.}\PYG{n}{hist}\PYG{p}{(}\PYG{l+m+mf}{1.}\PYG{o}{/}\PYG{p}{(}\PYG{l+m+mf}{1.}\PYG{o}{+}\PYG{n}{rvsp}\PYG{p}{)}\PYG{p}{,} \PYG{n}{bins}\PYG{o}{=}\PYG{l+m+mi}{50}\PYG{p}{,} \PYG{n}{normed}\PYG{o}{=}\PYG{n+nb+bp}{True}\PYG{p}{)}
\PYG{g+gp}{\PYGZgt{}\PYGZgt{}\PYGZgt{} }\PYG{n}{plt}\PYG{o}{.}\PYG{n}{plot}\PYG{p}{(}\PYG{n}{xx}\PYG{p}{,}\PYG{n}{powpdf}\PYG{p}{,}\PYG{l+s}{\PYGZsq{}}\PYG{l+s}{r\PYGZhy{}}\PYG{l+s}{\PYGZsq{}}\PYG{p}{)}
\PYG{g+gp}{\PYGZgt{}\PYGZgt{}\PYGZgt{} }\PYG{n}{plt}\PYG{o}{.}\PYG{n}{title}\PYG{p}{(}\PYG{l+s}{\PYGZsq{}}\PYG{l+s}{inverse of stats.pareto(5)}\PYG{l+s}{\PYGZsq{}}\PYG{p}{)}
\end{Verbatim}

\end{fulllineitems}

\index{rand() (in module acsAttractorAnalysisInTime)}

\begin{fulllineitems}
\phantomsection\label{acsAttractorAnalysisInTime:acsAttractorAnalysisInTime.rand}\pysiglinewithargsret{\code{acsAttractorAnalysisInTime.}\bfcode{rand}}{\emph{d0}, \emph{d1}, \emph{...}, \emph{dn}}{}
Random values in a given shape.

Create an array of the given shape and propagate it with
random samples from a uniform distribution
over \code{{[}0, 1)}.
\begin{description}
\item[{d0, d1, ..., dn}] \leavevmode{[}int, optional{]}
The dimensions of the returned array, should all be positive.
If no argument is given a single Python float is returned.

\end{description}
\begin{description}
\item[{out}] \leavevmode{[}ndarray, shape \code{(d0, d1, ..., dn)}{]}
Random values.

\end{description}

random

This is a convenience function. If you want an interface that
takes a shape-tuple as the first argument, refer to
np.random.random\_sample .

\begin{Verbatim}[commandchars=\\\{\}]
\PYG{g+gp}{\PYGZgt{}\PYGZgt{}\PYGZgt{} }\PYG{n}{np}\PYG{o}{.}\PYG{n}{random}\PYG{o}{.}\PYG{n}{rand}\PYG{p}{(}\PYG{l+m+mi}{3}\PYG{p}{,}\PYG{l+m+mi}{2}\PYG{p}{)}
\PYG{g+go}{array([[ 0.14022471,  0.96360618],  \PYGZsh{}random}
\PYG{g+go}{       [ 0.37601032,  0.25528411],  \PYGZsh{}random}
\PYG{g+go}{       [ 0.49313049,  0.94909878]]) \PYGZsh{}random}
\end{Verbatim}

\end{fulllineitems}

\index{randint() (in module acsAttractorAnalysisInTime)}

\begin{fulllineitems}
\phantomsection\label{acsAttractorAnalysisInTime:acsAttractorAnalysisInTime.randint}\pysiglinewithargsret{\code{acsAttractorAnalysisInTime.}\bfcode{randint}}{\emph{low}, \emph{high=None}, \emph{size=None}}{}
Return random integers from \emph{low} (inclusive) to \emph{high} (exclusive).

Return random integers from the ``discrete uniform'' distribution in the
``half-open'' interval {[}\emph{low}, \emph{high}). If \emph{high} is None (the default),
then results are from {[}0, \emph{low}).
\begin{description}
\item[{low}] \leavevmode{[}int{]}
Lowest (signed) integer to be drawn from the distribution (unless
\code{high=None}, in which case this parameter is the \emph{highest} such
integer).

\item[{high}] \leavevmode{[}int, optional{]}
If provided, one above the largest (signed) integer to be drawn
from the distribution (see above for behavior if \code{high=None}).

\item[{size}] \leavevmode{[}int or tuple of ints, optional{]}
Output shape. Default is None, in which case a single int is
returned.

\end{description}
\begin{description}
\item[{out}] \leavevmode{[}int or ndarray of ints{]}
\emph{size}-shaped array of random integers from the appropriate
distribution, or a single such random int if \emph{size} not provided.

\end{description}
\begin{description}
\item[{random.random\_integers}] \leavevmode{[}similar to \emph{randint}, only for the closed{]}
interval {[}\emph{low}, \emph{high}{]}, and 1 is the lowest value if \emph{high} is
omitted. In particular, this other one is the one to use to generate
uniformly distributed discrete non-integers.

\end{description}

\begin{Verbatim}[commandchars=\\\{\}]
\PYG{g+gp}{\PYGZgt{}\PYGZgt{}\PYGZgt{} }\PYG{n}{np}\PYG{o}{.}\PYG{n}{random}\PYG{o}{.}\PYG{n}{randint}\PYG{p}{(}\PYG{l+m+mi}{2}\PYG{p}{,} \PYG{n}{size}\PYG{o}{=}\PYG{l+m+mi}{10}\PYG{p}{)}
\PYG{g+go}{array([1, 0, 0, 0, 1, 1, 0, 0, 1, 0])}
\PYG{g+gp}{\PYGZgt{}\PYGZgt{}\PYGZgt{} }\PYG{n}{np}\PYG{o}{.}\PYG{n}{random}\PYG{o}{.}\PYG{n}{randint}\PYG{p}{(}\PYG{l+m+mi}{1}\PYG{p}{,} \PYG{n}{size}\PYG{o}{=}\PYG{l+m+mi}{10}\PYG{p}{)}
\PYG{g+go}{array([0, 0, 0, 0, 0, 0, 0, 0, 0, 0])}
\end{Verbatim}

Generate a 2 x 4 array of ints between 0 and 4, inclusive:

\begin{Verbatim}[commandchars=\\\{\}]
\PYG{g+gp}{\PYGZgt{}\PYGZgt{}\PYGZgt{} }\PYG{n}{np}\PYG{o}{.}\PYG{n}{random}\PYG{o}{.}\PYG{n}{randint}\PYG{p}{(}\PYG{l+m+mi}{5}\PYG{p}{,} \PYG{n}{size}\PYG{o}{=}\PYG{p}{(}\PYG{l+m+mi}{2}\PYG{p}{,} \PYG{l+m+mi}{4}\PYG{p}{)}\PYG{p}{)}
\PYG{g+go}{array([[4, 0, 2, 1],}
\PYG{g+go}{       [3, 2, 2, 0]])}
\end{Verbatim}

\end{fulllineitems}

\index{randn() (in module acsAttractorAnalysisInTime)}

\begin{fulllineitems}
\phantomsection\label{acsAttractorAnalysisInTime:acsAttractorAnalysisInTime.randn}\pysiglinewithargsret{\code{acsAttractorAnalysisInTime.}\bfcode{randn}}{\emph{d0}, \emph{d1}, \emph{...}, \emph{dn}}{}
Return a sample (or samples) from the ``standard normal'' distribution.

If positive, int\_like or int-convertible arguments are provided,
\emph{randn} generates an array of shape \code{(d0, d1, ..., dn)}, filled
with random floats sampled from a univariate ``normal'' (Gaussian)
distribution of mean 0 and variance 1 (if any of the \(d_i\) are
floats, they are first converted to integers by truncation). A single
float randomly sampled from the distribution is returned if no
argument is provided.

This is a convenience function.  If you want an interface that takes a
tuple as the first argument, use \emph{numpy.random.standard\_normal} instead.
\begin{description}
\item[{d0, d1, ..., dn}] \leavevmode{[}int, optional{]}
The dimensions of the returned array, should be all positive.
If no argument is given a single Python float is returned.

\end{description}
\begin{description}
\item[{Z}] \leavevmode{[}ndarray or float{]}
A \code{(d0, d1, ..., dn)}-shaped array of floating-point samples from
the standard normal distribution, or a single such float if
no parameters were supplied.

\end{description}

random.standard\_normal : Similar, but takes a tuple as its argument.

For random samples from \(N(\mu, \sigma^2)\), use:

\code{sigma * np.random.randn(...) + mu}

\begin{Verbatim}[commandchars=\\\{\}]
\PYG{g+gp}{\PYGZgt{}\PYGZgt{}\PYGZgt{} }\PYG{n}{np}\PYG{o}{.}\PYG{n}{random}\PYG{o}{.}\PYG{n}{randn}\PYG{p}{(}\PYG{p}{)}
\PYG{g+go}{2.1923875335537315 \PYGZsh{}random}
\end{Verbatim}

Two-by-four array of samples from N(3, 6.25):

\begin{Verbatim}[commandchars=\\\{\}]
\PYG{g+gp}{\PYGZgt{}\PYGZgt{}\PYGZgt{} }\PYG{l+m+mf}{2.5} \PYG{o}{*} \PYG{n}{np}\PYG{o}{.}\PYG{n}{random}\PYG{o}{.}\PYG{n}{randn}\PYG{p}{(}\PYG{l+m+mi}{2}\PYG{p}{,} \PYG{l+m+mi}{4}\PYG{p}{)} \PYG{o}{+} \PYG{l+m+mi}{3}
\PYG{g+go}{array([[\PYGZhy{}4.49401501,  4.00950034, \PYGZhy{}1.81814867,  7.29718677],  \PYGZsh{}random}
\PYG{g+go}{       [ 0.39924804,  4.68456316,  4.99394529,  4.84057254]]) \PYGZsh{}random}
\end{Verbatim}

\end{fulllineitems}

\index{random() (in module acsAttractorAnalysisInTime)}

\begin{fulllineitems}
\phantomsection\label{acsAttractorAnalysisInTime:acsAttractorAnalysisInTime.random}\pysiglinewithargsret{\code{acsAttractorAnalysisInTime.}\bfcode{random}}{}{}
random\_sample(size=None)

Return random floats in the half-open interval {[}0.0, 1.0).

Results are from the ``continuous uniform'' distribution over the
stated interval.  To sample \(Unif[a, b), b > a\) multiply
the output of \emph{random\_sample} by \emph{(b-a)} and add \emph{a}:

\begin{Verbatim}[commandchars=\\\{\}]
\PYG{p}{(}\PYG{n}{b} \PYG{o}{\PYGZhy{}} \PYG{n}{a}\PYG{p}{)} \PYG{o}{*} \PYG{n}{random\PYGZus{}sample}\PYG{p}{(}\PYG{p}{)} \PYG{o}{+} \PYG{n}{a}
\end{Verbatim}
\begin{description}
\item[{size}] \leavevmode{[}int or tuple of ints, optional{]}
Defines the shape of the returned array of random floats. If None
(the default), returns a single float.

\end{description}
\begin{description}
\item[{out}] \leavevmode{[}float or ndarray of floats{]}
Array of random floats of shape \emph{size} (unless \code{size=None}, in which
case a single float is returned).

\end{description}

\begin{Verbatim}[commandchars=\\\{\}]
\PYG{g+gp}{\PYGZgt{}\PYGZgt{}\PYGZgt{} }\PYG{n}{np}\PYG{o}{.}\PYG{n}{random}\PYG{o}{.}\PYG{n}{random\PYGZus{}sample}\PYG{p}{(}\PYG{p}{)}
\PYG{g+go}{0.47108547995356098}
\PYG{g+gp}{\PYGZgt{}\PYGZgt{}\PYGZgt{} }\PYG{n+nb}{type}\PYG{p}{(}\PYG{n}{np}\PYG{o}{.}\PYG{n}{random}\PYG{o}{.}\PYG{n}{random\PYGZus{}sample}\PYG{p}{(}\PYG{p}{)}\PYG{p}{)}
\PYG{g+go}{\PYGZlt{}type \PYGZsq{}float\PYGZsq{}\PYGZgt{}}
\PYG{g+gp}{\PYGZgt{}\PYGZgt{}\PYGZgt{} }\PYG{n}{np}\PYG{o}{.}\PYG{n}{random}\PYG{o}{.}\PYG{n}{random\PYGZus{}sample}\PYG{p}{(}\PYG{p}{(}\PYG{l+m+mi}{5}\PYG{p}{,}\PYG{p}{)}\PYG{p}{)}
\PYG{g+go}{array([ 0.30220482,  0.86820401,  0.1654503 ,  0.11659149,  0.54323428])}
\end{Verbatim}

Three-by-two array of random numbers from {[}-5, 0):

\begin{Verbatim}[commandchars=\\\{\}]
\PYG{g+gp}{\PYGZgt{}\PYGZgt{}\PYGZgt{} }\PYG{l+m+mi}{5} \PYG{o}{*} \PYG{n}{np}\PYG{o}{.}\PYG{n}{random}\PYG{o}{.}\PYG{n}{random\PYGZus{}sample}\PYG{p}{(}\PYG{p}{(}\PYG{l+m+mi}{3}\PYG{p}{,} \PYG{l+m+mi}{2}\PYG{p}{)}\PYG{p}{)} \PYG{o}{\PYGZhy{}} \PYG{l+m+mi}{5}
\PYG{g+go}{array([[\PYGZhy{}3.99149989, \PYGZhy{}0.52338984],}
\PYG{g+go}{       [\PYGZhy{}2.99091858, \PYGZhy{}0.79479508],}
\PYG{g+go}{       [\PYGZhy{}1.23204345, \PYGZhy{}1.75224494]])}
\end{Verbatim}

\end{fulllineitems}

\index{random\_integers() (in module acsAttractorAnalysisInTime)}

\begin{fulllineitems}
\phantomsection\label{acsAttractorAnalysisInTime:acsAttractorAnalysisInTime.random_integers}\pysiglinewithargsret{\code{acsAttractorAnalysisInTime.}\bfcode{random\_integers}}{\emph{low}, \emph{high=None}, \emph{size=None}}{}
Return random integers between \emph{low} and \emph{high}, inclusive.

Return random integers from the ``discrete uniform'' distribution in the
closed interval {[}\emph{low}, \emph{high}{]}.  If \emph{high} is None (the default),
then results are from {[}1, \emph{low}{]}.
\begin{description}
\item[{low}] \leavevmode{[}int{]}
Lowest (signed) integer to be drawn from the distribution (unless
\code{high=None}, in which case this parameter is the \emph{highest} such
integer).

\item[{high}] \leavevmode{[}int, optional{]}
If provided, the largest (signed) integer to be drawn from the
distribution (see above for behavior if \code{high=None}).

\item[{size}] \leavevmode{[}int or tuple of ints, optional{]}
Output shape. Default is None, in which case a single int is returned.

\end{description}
\begin{description}
\item[{out}] \leavevmode{[}int or ndarray of ints{]}
\emph{size}-shaped array of random integers from the appropriate
distribution, or a single such random int if \emph{size} not provided.

\end{description}
\begin{description}
\item[{random.randint}] \leavevmode{[}Similar to \emph{random\_integers}, only for the half-open{]}
interval {[}\emph{low}, \emph{high}), and 0 is the lowest value if \emph{high} is
omitted.

\end{description}

To sample from N evenly spaced floating-point numbers between a and b,
use:

\begin{Verbatim}[commandchars=\\\{\}]
\PYG{n}{a} \PYG{o}{+} \PYG{p}{(}\PYG{n}{b} \PYG{o}{\PYGZhy{}} \PYG{n}{a}\PYG{p}{)} \PYG{o}{*} \PYG{p}{(}\PYG{n}{np}\PYG{o}{.}\PYG{n}{random}\PYG{o}{.}\PYG{n}{random\PYGZus{}integers}\PYG{p}{(}\PYG{n}{N}\PYG{p}{)} \PYG{o}{\PYGZhy{}} \PYG{l+m+mi}{1}\PYG{p}{)} \PYG{o}{/} \PYG{p}{(}\PYG{n}{N} \PYG{o}{\PYGZhy{}} \PYG{l+m+mf}{1.}\PYG{p}{)}
\end{Verbatim}

\begin{Verbatim}[commandchars=\\\{\}]
\PYG{g+gp}{\PYGZgt{}\PYGZgt{}\PYGZgt{} }\PYG{n}{np}\PYG{o}{.}\PYG{n}{random}\PYG{o}{.}\PYG{n}{random\PYGZus{}integers}\PYG{p}{(}\PYG{l+m+mi}{5}\PYG{p}{)}
\PYG{g+go}{4}
\PYG{g+gp}{\PYGZgt{}\PYGZgt{}\PYGZgt{} }\PYG{n+nb}{type}\PYG{p}{(}\PYG{n}{np}\PYG{o}{.}\PYG{n}{random}\PYG{o}{.}\PYG{n}{random\PYGZus{}integers}\PYG{p}{(}\PYG{l+m+mi}{5}\PYG{p}{)}\PYG{p}{)}
\PYG{g+go}{\PYGZlt{}type \PYGZsq{}int\PYGZsq{}\PYGZgt{}}
\PYG{g+gp}{\PYGZgt{}\PYGZgt{}\PYGZgt{} }\PYG{n}{np}\PYG{o}{.}\PYG{n}{random}\PYG{o}{.}\PYG{n}{random\PYGZus{}integers}\PYG{p}{(}\PYG{l+m+mi}{5}\PYG{p}{,} \PYG{n}{size}\PYG{o}{=}\PYG{p}{(}\PYG{l+m+mf}{3.}\PYG{p}{,}\PYG{l+m+mf}{2.}\PYG{p}{)}\PYG{p}{)}
\PYG{g+go}{array([[5, 4],}
\PYG{g+go}{       [3, 3],}
\PYG{g+go}{       [4, 5]])}
\end{Verbatim}

Choose five random numbers from the set of five evenly-spaced
numbers between 0 and 2.5, inclusive (\emph{i.e.}, from the set
\({0, 5/8, 10/8, 15/8, 20/8}\)):

\begin{Verbatim}[commandchars=\\\{\}]
\PYG{g+gp}{\PYGZgt{}\PYGZgt{}\PYGZgt{} }\PYG{l+m+mf}{2.5} \PYG{o}{*} \PYG{p}{(}\PYG{n}{np}\PYG{o}{.}\PYG{n}{random}\PYG{o}{.}\PYG{n}{random\PYGZus{}integers}\PYG{p}{(}\PYG{l+m+mi}{5}\PYG{p}{,} \PYG{n}{size}\PYG{o}{=}\PYG{p}{(}\PYG{l+m+mi}{5}\PYG{p}{,}\PYG{p}{)}\PYG{p}{)} \PYG{o}{\PYGZhy{}} \PYG{l+m+mi}{1}\PYG{p}{)} \PYG{o}{/} \PYG{l+m+mf}{4.}
\PYG{g+go}{array([ 0.625,  1.25 ,  0.625,  0.625,  2.5  ])}
\end{Verbatim}

Roll two six sided dice 1000 times and sum the results:

\begin{Verbatim}[commandchars=\\\{\}]
\PYG{g+gp}{\PYGZgt{}\PYGZgt{}\PYGZgt{} }\PYG{n}{d1} \PYG{o}{=} \PYG{n}{np}\PYG{o}{.}\PYG{n}{random}\PYG{o}{.}\PYG{n}{random\PYGZus{}integers}\PYG{p}{(}\PYG{l+m+mi}{1}\PYG{p}{,} \PYG{l+m+mi}{6}\PYG{p}{,} \PYG{l+m+mi}{1000}\PYG{p}{)}
\PYG{g+gp}{\PYGZgt{}\PYGZgt{}\PYGZgt{} }\PYG{n}{d2} \PYG{o}{=} \PYG{n}{np}\PYG{o}{.}\PYG{n}{random}\PYG{o}{.}\PYG{n}{random\PYGZus{}integers}\PYG{p}{(}\PYG{l+m+mi}{1}\PYG{p}{,} \PYG{l+m+mi}{6}\PYG{p}{,} \PYG{l+m+mi}{1000}\PYG{p}{)}
\PYG{g+gp}{\PYGZgt{}\PYGZgt{}\PYGZgt{} }\PYG{n}{dsums} \PYG{o}{=} \PYG{n}{d1} \PYG{o}{+} \PYG{n}{d2}
\end{Verbatim}

Display results as a histogram:

\begin{Verbatim}[commandchars=\\\{\}]
\PYG{g+gp}{\PYGZgt{}\PYGZgt{}\PYGZgt{} }\PYG{k+kn}{import} \PYG{n+nn}{matplotlib.pyplot} \PYG{k+kn}{as} \PYG{n+nn}{plt}
\PYG{g+gp}{\PYGZgt{}\PYGZgt{}\PYGZgt{} }\PYG{n}{count}\PYG{p}{,} \PYG{n}{bins}\PYG{p}{,} \PYG{n}{ignored} \PYG{o}{=} \PYG{n}{plt}\PYG{o}{.}\PYG{n}{hist}\PYG{p}{(}\PYG{n}{dsums}\PYG{p}{,} \PYG{l+m+mi}{11}\PYG{p}{,} \PYG{n}{normed}\PYG{o}{=}\PYG{n+nb+bp}{True}\PYG{p}{)}
\PYG{g+gp}{\PYGZgt{}\PYGZgt{}\PYGZgt{} }\PYG{n}{plt}\PYG{o}{.}\PYG{n}{show}\PYG{p}{(}\PYG{p}{)}
\end{Verbatim}

\end{fulllineitems}

\index{random\_sample() (in module acsAttractorAnalysisInTime)}

\begin{fulllineitems}
\phantomsection\label{acsAttractorAnalysisInTime:acsAttractorAnalysisInTime.random_sample}\pysiglinewithargsret{\code{acsAttractorAnalysisInTime.}\bfcode{random\_sample}}{\emph{size=None}}{}
Return random floats in the half-open interval {[}0.0, 1.0).

Results are from the ``continuous uniform'' distribution over the
stated interval.  To sample \(Unif[a, b), b > a\) multiply
the output of \emph{random\_sample} by \emph{(b-a)} and add \emph{a}:

\begin{Verbatim}[commandchars=\\\{\}]
\PYG{p}{(}\PYG{n}{b} \PYG{o}{\PYGZhy{}} \PYG{n}{a}\PYG{p}{)} \PYG{o}{*} \PYG{n}{random\PYGZus{}sample}\PYG{p}{(}\PYG{p}{)} \PYG{o}{+} \PYG{n}{a}
\end{Verbatim}
\begin{description}
\item[{size}] \leavevmode{[}int or tuple of ints, optional{]}
Defines the shape of the returned array of random floats. If None
(the default), returns a single float.

\end{description}
\begin{description}
\item[{out}] \leavevmode{[}float or ndarray of floats{]}
Array of random floats of shape \emph{size} (unless \code{size=None}, in which
case a single float is returned).

\end{description}

\begin{Verbatim}[commandchars=\\\{\}]
\PYG{g+gp}{\PYGZgt{}\PYGZgt{}\PYGZgt{} }\PYG{n}{np}\PYG{o}{.}\PYG{n}{random}\PYG{o}{.}\PYG{n}{random\PYGZus{}sample}\PYG{p}{(}\PYG{p}{)}
\PYG{g+go}{0.47108547995356098}
\PYG{g+gp}{\PYGZgt{}\PYGZgt{}\PYGZgt{} }\PYG{n+nb}{type}\PYG{p}{(}\PYG{n}{np}\PYG{o}{.}\PYG{n}{random}\PYG{o}{.}\PYG{n}{random\PYGZus{}sample}\PYG{p}{(}\PYG{p}{)}\PYG{p}{)}
\PYG{g+go}{\PYGZlt{}type \PYGZsq{}float\PYGZsq{}\PYGZgt{}}
\PYG{g+gp}{\PYGZgt{}\PYGZgt{}\PYGZgt{} }\PYG{n}{np}\PYG{o}{.}\PYG{n}{random}\PYG{o}{.}\PYG{n}{random\PYGZus{}sample}\PYG{p}{(}\PYG{p}{(}\PYG{l+m+mi}{5}\PYG{p}{,}\PYG{p}{)}\PYG{p}{)}
\PYG{g+go}{array([ 0.30220482,  0.86820401,  0.1654503 ,  0.11659149,  0.54323428])}
\end{Verbatim}

Three-by-two array of random numbers from {[}-5, 0):

\begin{Verbatim}[commandchars=\\\{\}]
\PYG{g+gp}{\PYGZgt{}\PYGZgt{}\PYGZgt{} }\PYG{l+m+mi}{5} \PYG{o}{*} \PYG{n}{np}\PYG{o}{.}\PYG{n}{random}\PYG{o}{.}\PYG{n}{random\PYGZus{}sample}\PYG{p}{(}\PYG{p}{(}\PYG{l+m+mi}{3}\PYG{p}{,} \PYG{l+m+mi}{2}\PYG{p}{)}\PYG{p}{)} \PYG{o}{\PYGZhy{}} \PYG{l+m+mi}{5}
\PYG{g+go}{array([[\PYGZhy{}3.99149989, \PYGZhy{}0.52338984],}
\PYG{g+go}{       [\PYGZhy{}2.99091858, \PYGZhy{}0.79479508],}
\PYG{g+go}{       [\PYGZhy{}1.23204345, \PYGZhy{}1.75224494]])}
\end{Verbatim}

\end{fulllineitems}

\index{ranf() (in module acsAttractorAnalysisInTime)}

\begin{fulllineitems}
\phantomsection\label{acsAttractorAnalysisInTime:acsAttractorAnalysisInTime.ranf}\pysiglinewithargsret{\code{acsAttractorAnalysisInTime.}\bfcode{ranf}}{}{}
random\_sample(size=None)

Return random floats in the half-open interval {[}0.0, 1.0).

Results are from the ``continuous uniform'' distribution over the
stated interval.  To sample \(Unif[a, b), b > a\) multiply
the output of \emph{random\_sample} by \emph{(b-a)} and add \emph{a}:

\begin{Verbatim}[commandchars=\\\{\}]
\PYG{p}{(}\PYG{n}{b} \PYG{o}{\PYGZhy{}} \PYG{n}{a}\PYG{p}{)} \PYG{o}{*} \PYG{n}{random\PYGZus{}sample}\PYG{p}{(}\PYG{p}{)} \PYG{o}{+} \PYG{n}{a}
\end{Verbatim}
\begin{description}
\item[{size}] \leavevmode{[}int or tuple of ints, optional{]}
Defines the shape of the returned array of random floats. If None
(the default), returns a single float.

\end{description}
\begin{description}
\item[{out}] \leavevmode{[}float or ndarray of floats{]}
Array of random floats of shape \emph{size} (unless \code{size=None}, in which
case a single float is returned).

\end{description}

\begin{Verbatim}[commandchars=\\\{\}]
\PYG{g+gp}{\PYGZgt{}\PYGZgt{}\PYGZgt{} }\PYG{n}{np}\PYG{o}{.}\PYG{n}{random}\PYG{o}{.}\PYG{n}{random\PYGZus{}sample}\PYG{p}{(}\PYG{p}{)}
\PYG{g+go}{0.47108547995356098}
\PYG{g+gp}{\PYGZgt{}\PYGZgt{}\PYGZgt{} }\PYG{n+nb}{type}\PYG{p}{(}\PYG{n}{np}\PYG{o}{.}\PYG{n}{random}\PYG{o}{.}\PYG{n}{random\PYGZus{}sample}\PYG{p}{(}\PYG{p}{)}\PYG{p}{)}
\PYG{g+go}{\PYGZlt{}type \PYGZsq{}float\PYGZsq{}\PYGZgt{}}
\PYG{g+gp}{\PYGZgt{}\PYGZgt{}\PYGZgt{} }\PYG{n}{np}\PYG{o}{.}\PYG{n}{random}\PYG{o}{.}\PYG{n}{random\PYGZus{}sample}\PYG{p}{(}\PYG{p}{(}\PYG{l+m+mi}{5}\PYG{p}{,}\PYG{p}{)}\PYG{p}{)}
\PYG{g+go}{array([ 0.30220482,  0.86820401,  0.1654503 ,  0.11659149,  0.54323428])}
\end{Verbatim}

Three-by-two array of random numbers from {[}-5, 0):

\begin{Verbatim}[commandchars=\\\{\}]
\PYG{g+gp}{\PYGZgt{}\PYGZgt{}\PYGZgt{} }\PYG{l+m+mi}{5} \PYG{o}{*} \PYG{n}{np}\PYG{o}{.}\PYG{n}{random}\PYG{o}{.}\PYG{n}{random\PYGZus{}sample}\PYG{p}{(}\PYG{p}{(}\PYG{l+m+mi}{3}\PYG{p}{,} \PYG{l+m+mi}{2}\PYG{p}{)}\PYG{p}{)} \PYG{o}{\PYGZhy{}} \PYG{l+m+mi}{5}
\PYG{g+go}{array([[\PYGZhy{}3.99149989, \PYGZhy{}0.52338984],}
\PYG{g+go}{       [\PYGZhy{}2.99091858, \PYGZhy{}0.79479508],}
\PYG{g+go}{       [\PYGZhy{}1.23204345, \PYGZhy{}1.75224494]])}
\end{Verbatim}

\end{fulllineitems}

\index{rayleigh() (in module acsAttractorAnalysisInTime)}

\begin{fulllineitems}
\phantomsection\label{acsAttractorAnalysisInTime:acsAttractorAnalysisInTime.rayleigh}\pysiglinewithargsret{\code{acsAttractorAnalysisInTime.}\bfcode{rayleigh}}{\emph{scale=1.0}, \emph{size=None}}{}
Draw samples from a Rayleigh distribution.

The \(\chi\) and Weibull distributions are generalizations of the
Rayleigh.
\begin{description}
\item[{scale}] \leavevmode{[}scalar{]}
Scale, also equals the mode. Should be \textgreater{}= 0.

\item[{size}] \leavevmode{[}int or tuple of ints, optional{]}
Shape of the output. Default is None, in which case a single
value is returned.

\end{description}

The probability density function for the Rayleigh distribution is
\begin{gather}
\begin{split}P(x;scale) = \frac{x}{scale^2}e^{\frac{-x^2}{2 \cdotp scale^2}}\end{split}\notag
\end{gather}
The Rayleigh distribution arises if the wind speed and wind direction are
both gaussian variables, then the vector wind velocity forms a Rayleigh
distribution. The Rayleigh distribution is used to model the expected
output from wind turbines.

Draw values from the distribution and plot the histogram

\begin{Verbatim}[commandchars=\\\{\}]
\PYG{g+gp}{\PYGZgt{}\PYGZgt{}\PYGZgt{} }\PYG{n}{values} \PYG{o}{=} \PYG{n}{hist}\PYG{p}{(}\PYG{n}{np}\PYG{o}{.}\PYG{n}{random}\PYG{o}{.}\PYG{n}{rayleigh}\PYG{p}{(}\PYG{l+m+mi}{3}\PYG{p}{,} \PYG{l+m+mi}{100000}\PYG{p}{)}\PYG{p}{,} \PYG{n}{bins}\PYG{o}{=}\PYG{l+m+mi}{200}\PYG{p}{,} \PYG{n}{normed}\PYG{o}{=}\PYG{n+nb+bp}{True}\PYG{p}{)}
\end{Verbatim}

Wave heights tend to follow a Rayleigh distribution. If the mean wave
height is 1 meter, what fraction of waves are likely to be larger than 3
meters?

\begin{Verbatim}[commandchars=\\\{\}]
\PYG{g+gp}{\PYGZgt{}\PYGZgt{}\PYGZgt{} }\PYG{n}{meanvalue} \PYG{o}{=} \PYG{l+m+mi}{1}
\PYG{g+gp}{\PYGZgt{}\PYGZgt{}\PYGZgt{} }\PYG{n}{modevalue} \PYG{o}{=} \PYG{n}{np}\PYG{o}{.}\PYG{n}{sqrt}\PYG{p}{(}\PYG{l+m+mi}{2} \PYG{o}{/} \PYG{n}{np}\PYG{o}{.}\PYG{n}{pi}\PYG{p}{)} \PYG{o}{*} \PYG{n}{meanvalue}
\PYG{g+gp}{\PYGZgt{}\PYGZgt{}\PYGZgt{} }\PYG{n}{s} \PYG{o}{=} \PYG{n}{np}\PYG{o}{.}\PYG{n}{random}\PYG{o}{.}\PYG{n}{rayleigh}\PYG{p}{(}\PYG{n}{modevalue}\PYG{p}{,} \PYG{l+m+mi}{1000000}\PYG{p}{)}
\end{Verbatim}

The percentage of waves larger than 3 meters is:

\begin{Verbatim}[commandchars=\\\{\}]
\PYG{g+gp}{\PYGZgt{}\PYGZgt{}\PYGZgt{} }\PYG{l+m+mf}{100.}\PYG{o}{*}\PYG{n+nb}{sum}\PYG{p}{(}\PYG{n}{s}\PYG{o}{\PYGZgt{}}\PYG{l+m+mi}{3}\PYG{p}{)}\PYG{o}{/}\PYG{l+m+mf}{1000000.}
\PYG{g+go}{0.087300000000000003}
\end{Verbatim}

\end{fulllineitems}

\index{sample() (in module acsAttractorAnalysisInTime)}

\begin{fulllineitems}
\phantomsection\label{acsAttractorAnalysisInTime:acsAttractorAnalysisInTime.sample}\pysiglinewithargsret{\code{acsAttractorAnalysisInTime.}\bfcode{sample}}{}{}
random\_sample(size=None)

Return random floats in the half-open interval {[}0.0, 1.0).

Results are from the ``continuous uniform'' distribution over the
stated interval.  To sample \(Unif[a, b), b > a\) multiply
the output of \emph{random\_sample} by \emph{(b-a)} and add \emph{a}:

\begin{Verbatim}[commandchars=\\\{\}]
\PYG{p}{(}\PYG{n}{b} \PYG{o}{\PYGZhy{}} \PYG{n}{a}\PYG{p}{)} \PYG{o}{*} \PYG{n}{random\PYGZus{}sample}\PYG{p}{(}\PYG{p}{)} \PYG{o}{+} \PYG{n}{a}
\end{Verbatim}
\begin{description}
\item[{size}] \leavevmode{[}int or tuple of ints, optional{]}
Defines the shape of the returned array of random floats. If None
(the default), returns a single float.

\end{description}
\begin{description}
\item[{out}] \leavevmode{[}float or ndarray of floats{]}
Array of random floats of shape \emph{size} (unless \code{size=None}, in which
case a single float is returned).

\end{description}

\begin{Verbatim}[commandchars=\\\{\}]
\PYG{g+gp}{\PYGZgt{}\PYGZgt{}\PYGZgt{} }\PYG{n}{np}\PYG{o}{.}\PYG{n}{random}\PYG{o}{.}\PYG{n}{random\PYGZus{}sample}\PYG{p}{(}\PYG{p}{)}
\PYG{g+go}{0.47108547995356098}
\PYG{g+gp}{\PYGZgt{}\PYGZgt{}\PYGZgt{} }\PYG{n+nb}{type}\PYG{p}{(}\PYG{n}{np}\PYG{o}{.}\PYG{n}{random}\PYG{o}{.}\PYG{n}{random\PYGZus{}sample}\PYG{p}{(}\PYG{p}{)}\PYG{p}{)}
\PYG{g+go}{\PYGZlt{}type \PYGZsq{}float\PYGZsq{}\PYGZgt{}}
\PYG{g+gp}{\PYGZgt{}\PYGZgt{}\PYGZgt{} }\PYG{n}{np}\PYG{o}{.}\PYG{n}{random}\PYG{o}{.}\PYG{n}{random\PYGZus{}sample}\PYG{p}{(}\PYG{p}{(}\PYG{l+m+mi}{5}\PYG{p}{,}\PYG{p}{)}\PYG{p}{)}
\PYG{g+go}{array([ 0.30220482,  0.86820401,  0.1654503 ,  0.11659149,  0.54323428])}
\end{Verbatim}

Three-by-two array of random numbers from {[}-5, 0):

\begin{Verbatim}[commandchars=\\\{\}]
\PYG{g+gp}{\PYGZgt{}\PYGZgt{}\PYGZgt{} }\PYG{l+m+mi}{5} \PYG{o}{*} \PYG{n}{np}\PYG{o}{.}\PYG{n}{random}\PYG{o}{.}\PYG{n}{random\PYGZus{}sample}\PYG{p}{(}\PYG{p}{(}\PYG{l+m+mi}{3}\PYG{p}{,} \PYG{l+m+mi}{2}\PYG{p}{)}\PYG{p}{)} \PYG{o}{\PYGZhy{}} \PYG{l+m+mi}{5}
\PYG{g+go}{array([[\PYGZhy{}3.99149989, \PYGZhy{}0.52338984],}
\PYG{g+go}{       [\PYGZhy{}2.99091858, \PYGZhy{}0.79479508],}
\PYG{g+go}{       [\PYGZhy{}1.23204345, \PYGZhy{}1.75224494]])}
\end{Verbatim}

\end{fulllineitems}

\index{seed() (in module acsAttractorAnalysisInTime)}

\begin{fulllineitems}
\phantomsection\label{acsAttractorAnalysisInTime:acsAttractorAnalysisInTime.seed}\pysiglinewithargsret{\code{acsAttractorAnalysisInTime.}\bfcode{seed}}{\emph{seed=None}}{}
Seed the generator.

This method is called when \emph{RandomState} is initialized. It can be
called again to re-seed the generator. For details, see \emph{RandomState}.
\begin{description}
\item[{seed}] \leavevmode{[}int or array\_like, optional{]}
Seed for \emph{RandomState}.

\end{description}

RandomState

\end{fulllineitems}

\index{set\_state() (in module acsAttractorAnalysisInTime)}

\begin{fulllineitems}
\phantomsection\label{acsAttractorAnalysisInTime:acsAttractorAnalysisInTime.set_state}\pysiglinewithargsret{\code{acsAttractorAnalysisInTime.}\bfcode{set\_state}}{\emph{state}}{}
Set the internal state of the generator from a tuple.

For use if one has reason to manually (re-)set the internal state of the
``Mersenne Twister''{\color{red}\bfseries{}{[}1{]}\_} pseudo-random number generating algorithm.
\begin{description}
\item[{state}] \leavevmode{[}tuple(str, ndarray of 624 uints, int, int, float){]}
The \emph{state} tuple has the following items:
\begin{enumerate}
\item {} 
the string `MT19937', specifying the Mersenne Twister algorithm.

\item {} 
a 1-D array of 624 unsigned integers \code{keys}.

\item {} 
an integer \code{pos}.

\item {} 
an integer \code{has\_gauss}.

\item {} 
a float \code{cached\_gaussian}.

\end{enumerate}

\end{description}
\begin{description}
\item[{out}] \leavevmode{[}None{]}
Returns `None' on success.

\end{description}

get\_state

\emph{set\_state} and \emph{get\_state} are not needed to work with any of the
random distributions in NumPy. If the internal state is manually altered,
the user should know exactly what he/she is doing.

For backwards compatibility, the form (str, array of 624 uints, int) is
also accepted although it is missing some information about the cached
Gaussian value: \code{state = ('MT19937', keys, pos)}.

\end{fulllineitems}

\index{shuffle() (in module acsAttractorAnalysisInTime)}

\begin{fulllineitems}
\phantomsection\label{acsAttractorAnalysisInTime:acsAttractorAnalysisInTime.shuffle}\pysiglinewithargsret{\code{acsAttractorAnalysisInTime.}\bfcode{shuffle}}{\emph{x}}{}
Modify a sequence in-place by shuffling its contents.
\begin{description}
\item[{x}] \leavevmode{[}array\_like{]}
The array or list to be shuffled.

\end{description}

None

\begin{Verbatim}[commandchars=\\\{\}]
\PYG{g+gp}{\PYGZgt{}\PYGZgt{}\PYGZgt{} }\PYG{n}{arr} \PYG{o}{=} \PYG{n}{np}\PYG{o}{.}\PYG{n}{arange}\PYG{p}{(}\PYG{l+m+mi}{10}\PYG{p}{)}
\PYG{g+gp}{\PYGZgt{}\PYGZgt{}\PYGZgt{} }\PYG{n}{np}\PYG{o}{.}\PYG{n}{random}\PYG{o}{.}\PYG{n}{shuffle}\PYG{p}{(}\PYG{n}{arr}\PYG{p}{)}
\PYG{g+gp}{\PYGZgt{}\PYGZgt{}\PYGZgt{} }\PYG{n}{arr}
\PYG{g+go}{[1 7 5 2 9 4 3 6 0 8]}
\end{Verbatim}

This function only shuffles the array along the first index of a
multi-dimensional array:

\begin{Verbatim}[commandchars=\\\{\}]
\PYG{g+gp}{\PYGZgt{}\PYGZgt{}\PYGZgt{} }\PYG{n}{arr} \PYG{o}{=} \PYG{n}{np}\PYG{o}{.}\PYG{n}{arange}\PYG{p}{(}\PYG{l+m+mi}{9}\PYG{p}{)}\PYG{o}{.}\PYG{n}{reshape}\PYG{p}{(}\PYG{p}{(}\PYG{l+m+mi}{3}\PYG{p}{,} \PYG{l+m+mi}{3}\PYG{p}{)}\PYG{p}{)}
\PYG{g+gp}{\PYGZgt{}\PYGZgt{}\PYGZgt{} }\PYG{n}{np}\PYG{o}{.}\PYG{n}{random}\PYG{o}{.}\PYG{n}{shuffle}\PYG{p}{(}\PYG{n}{arr}\PYG{p}{)}
\PYG{g+gp}{\PYGZgt{}\PYGZgt{}\PYGZgt{} }\PYG{n}{arr}
\PYG{g+go}{array([[3, 4, 5],}
\PYG{g+go}{       [6, 7, 8],}
\PYG{g+go}{       [0, 1, 2]])}
\end{Verbatim}

\end{fulllineitems}

\index{standard\_cauchy() (in module acsAttractorAnalysisInTime)}

\begin{fulllineitems}
\phantomsection\label{acsAttractorAnalysisInTime:acsAttractorAnalysisInTime.standard_cauchy}\pysiglinewithargsret{\code{acsAttractorAnalysisInTime.}\bfcode{standard\_cauchy}}{\emph{size=None}}{}
Standard Cauchy distribution with mode = 0.

Also known as the Lorentz distribution.
\begin{description}
\item[{size}] \leavevmode{[}int or tuple of ints{]}
Shape of the output.

\end{description}
\begin{description}
\item[{samples}] \leavevmode{[}ndarray or scalar{]}
The drawn samples.

\end{description}

The probability density function for the full Cauchy distribution is
\begin{gather}
\begin{split}P(x; x_0, \gamma) = \frac{1}{\pi \gamma \bigl[ 1+
(\frac{x-x_0}{\gamma})^2 \bigr] }\end{split}\notag
\end{gather}
and the Standard Cauchy distribution just sets \(x_0=0\) and
\(\gamma=1\)

The Cauchy distribution arises in the solution to the driven harmonic
oscillator problem, and also describes spectral line broadening. It
also describes the distribution of values at which a line tilted at
a random angle will cut the x axis.

When studying hypothesis tests that assume normality, seeing how the
tests perform on data from a Cauchy distribution is a good indicator of
their sensitivity to a heavy-tailed distribution, since the Cauchy looks
very much like a Gaussian distribution, but with heavier tails.

Draw samples and plot the distribution:

\begin{Verbatim}[commandchars=\\\{\}]
\PYG{g+gp}{\PYGZgt{}\PYGZgt{}\PYGZgt{} }\PYG{n}{s} \PYG{o}{=} \PYG{n}{np}\PYG{o}{.}\PYG{n}{random}\PYG{o}{.}\PYG{n}{standard\PYGZus{}cauchy}\PYG{p}{(}\PYG{l+m+mi}{1000000}\PYG{p}{)}
\PYG{g+gp}{\PYGZgt{}\PYGZgt{}\PYGZgt{} }\PYG{n}{s} \PYG{o}{=} \PYG{n}{s}\PYG{p}{[}\PYG{p}{(}\PYG{n}{s}\PYG{o}{\PYGZgt{}}\PYG{o}{\PYGZhy{}}\PYG{l+m+mi}{25}\PYG{p}{)} \PYG{o}{\PYGZam{}} \PYG{p}{(}\PYG{n}{s}\PYG{o}{\PYGZlt{}}\PYG{l+m+mi}{25}\PYG{p}{)}\PYG{p}{]}  \PYG{c}{\PYGZsh{} truncate distribution so it plots well}
\PYG{g+gp}{\PYGZgt{}\PYGZgt{}\PYGZgt{} }\PYG{n}{plt}\PYG{o}{.}\PYG{n}{hist}\PYG{p}{(}\PYG{n}{s}\PYG{p}{,} \PYG{n}{bins}\PYG{o}{=}\PYG{l+m+mi}{100}\PYG{p}{)}
\PYG{g+gp}{\PYGZgt{}\PYGZgt{}\PYGZgt{} }\PYG{n}{plt}\PYG{o}{.}\PYG{n}{show}\PYG{p}{(}\PYG{p}{)}
\end{Verbatim}

\end{fulllineitems}

\index{standard\_exponential() (in module acsAttractorAnalysisInTime)}

\begin{fulllineitems}
\phantomsection\label{acsAttractorAnalysisInTime:acsAttractorAnalysisInTime.standard_exponential}\pysiglinewithargsret{\code{acsAttractorAnalysisInTime.}\bfcode{standard\_exponential}}{\emph{size=None}}{}
Draw samples from the standard exponential distribution.

\emph{standard\_exponential} is identical to the exponential distribution
with a scale parameter of 1.
\begin{description}
\item[{size}] \leavevmode{[}int or tuple of ints{]}
Shape of the output.

\end{description}
\begin{description}
\item[{out}] \leavevmode{[}float or ndarray{]}
Drawn samples.

\end{description}

Output a 3x8000 array:

\begin{Verbatim}[commandchars=\\\{\}]
\PYG{g+gp}{\PYGZgt{}\PYGZgt{}\PYGZgt{} }\PYG{n}{n} \PYG{o}{=} \PYG{n}{np}\PYG{o}{.}\PYG{n}{random}\PYG{o}{.}\PYG{n}{standard\PYGZus{}exponential}\PYG{p}{(}\PYG{p}{(}\PYG{l+m+mi}{3}\PYG{p}{,} \PYG{l+m+mi}{8000}\PYG{p}{)}\PYG{p}{)}
\end{Verbatim}

\end{fulllineitems}

\index{standard\_gamma() (in module acsAttractorAnalysisInTime)}

\begin{fulllineitems}
\phantomsection\label{acsAttractorAnalysisInTime:acsAttractorAnalysisInTime.standard_gamma}\pysiglinewithargsret{\code{acsAttractorAnalysisInTime.}\bfcode{standard\_gamma}}{\emph{shape}, \emph{size=None}}{}
Draw samples from a Standard Gamma distribution.

Samples are drawn from a Gamma distribution with specified parameters,
shape (sometimes designated ``k'') and scale=1.
\begin{description}
\item[{shape}] \leavevmode{[}float{]}
Parameter, should be \textgreater{} 0.

\item[{size}] \leavevmode{[}int or tuple of ints{]}
Output shape.  If the given shape is, e.g., \code{(m, n, k)}, then
\code{m * n * k} samples are drawn.

\end{description}
\begin{description}
\item[{samples}] \leavevmode{[}ndarray or scalar{]}
The drawn samples.

\end{description}
\begin{description}
\item[{scipy.stats.distributions.gamma}] \leavevmode{[}probability density function,{]}
distribution or cumulative density function, etc.

\end{description}

The probability density for the Gamma distribution is
\begin{gather}
\begin{split}p(x) = x^{k-1}\frac{e^{-x/\theta}}{\theta^k\Gamma(k)},\end{split}\notag
\end{gather}
where \(k\) is the shape and \(\theta\) the scale,
and \(\Gamma\) is the Gamma function.

The Gamma distribution is often used to model the times to failure of
electronic components, and arises naturally in processes for which the
waiting times between Poisson distributed events are relevant.

Draw samples from the distribution:

\begin{Verbatim}[commandchars=\\\{\}]
\PYG{g+gp}{\PYGZgt{}\PYGZgt{}\PYGZgt{} }\PYG{n}{shape}\PYG{p}{,} \PYG{n}{scale} \PYG{o}{=} \PYG{l+m+mf}{2.}\PYG{p}{,} \PYG{l+m+mf}{1.} \PYG{c}{\PYGZsh{} mean and width}
\PYG{g+gp}{\PYGZgt{}\PYGZgt{}\PYGZgt{} }\PYG{n}{s} \PYG{o}{=} \PYG{n}{np}\PYG{o}{.}\PYG{n}{random}\PYG{o}{.}\PYG{n}{standard\PYGZus{}gamma}\PYG{p}{(}\PYG{n}{shape}\PYG{p}{,} \PYG{l+m+mi}{1000000}\PYG{p}{)}
\end{Verbatim}

Display the histogram of the samples, along with
the probability density function:

\begin{Verbatim}[commandchars=\\\{\}]
\PYG{g+gp}{\PYGZgt{}\PYGZgt{}\PYGZgt{} }\PYG{k+kn}{import} \PYG{n+nn}{matplotlib.pyplot} \PYG{k+kn}{as} \PYG{n+nn}{plt}
\PYG{g+gp}{\PYGZgt{}\PYGZgt{}\PYGZgt{} }\PYG{k+kn}{import} \PYG{n+nn}{scipy.special} \PYG{k+kn}{as} \PYG{n+nn}{sps}
\PYG{g+gp}{\PYGZgt{}\PYGZgt{}\PYGZgt{} }\PYG{n}{count}\PYG{p}{,} \PYG{n}{bins}\PYG{p}{,} \PYG{n}{ignored} \PYG{o}{=} \PYG{n}{plt}\PYG{o}{.}\PYG{n}{hist}\PYG{p}{(}\PYG{n}{s}\PYG{p}{,} \PYG{l+m+mi}{50}\PYG{p}{,} \PYG{n}{normed}\PYG{o}{=}\PYG{n+nb+bp}{True}\PYG{p}{)}
\PYG{g+gp}{\PYGZgt{}\PYGZgt{}\PYGZgt{} }\PYG{n}{y} \PYG{o}{=} \PYG{n}{bins}\PYG{o}{*}\PYG{o}{*}\PYG{p}{(}\PYG{n}{shape}\PYG{o}{\PYGZhy{}}\PYG{l+m+mi}{1}\PYG{p}{)} \PYG{o}{*} \PYG{p}{(}\PYG{p}{(}\PYG{n}{np}\PYG{o}{.}\PYG{n}{exp}\PYG{p}{(}\PYG{o}{\PYGZhy{}}\PYG{n}{bins}\PYG{o}{/}\PYG{n}{scale}\PYG{p}{)}\PYG{p}{)}\PYG{o}{/} \PYGZbs{}
\PYG{g+gp}{... }                      \PYG{p}{(}\PYG{n}{sps}\PYG{o}{.}\PYG{n}{gamma}\PYG{p}{(}\PYG{n}{shape}\PYG{p}{)} \PYG{o}{*} \PYG{n}{scale}\PYG{o}{*}\PYG{o}{*}\PYG{n}{shape}\PYG{p}{)}\PYG{p}{)}
\PYG{g+gp}{\PYGZgt{}\PYGZgt{}\PYGZgt{} }\PYG{n}{plt}\PYG{o}{.}\PYG{n}{plot}\PYG{p}{(}\PYG{n}{bins}\PYG{p}{,} \PYG{n}{y}\PYG{p}{,} \PYG{n}{linewidth}\PYG{o}{=}\PYG{l+m+mi}{2}\PYG{p}{,} \PYG{n}{color}\PYG{o}{=}\PYG{l+s}{\PYGZsq{}}\PYG{l+s}{r}\PYG{l+s}{\PYGZsq{}}\PYG{p}{)}
\PYG{g+gp}{\PYGZgt{}\PYGZgt{}\PYGZgt{} }\PYG{n}{plt}\PYG{o}{.}\PYG{n}{show}\PYG{p}{(}\PYG{p}{)}
\end{Verbatim}

\end{fulllineitems}

\index{standard\_normal() (in module acsAttractorAnalysisInTime)}

\begin{fulllineitems}
\phantomsection\label{acsAttractorAnalysisInTime:acsAttractorAnalysisInTime.standard_normal}\pysiglinewithargsret{\code{acsAttractorAnalysisInTime.}\bfcode{standard\_normal}}{\emph{size=None}}{}
Returns samples from a Standard Normal distribution (mean=0, stdev=1).
\begin{description}
\item[{size}] \leavevmode{[}int or tuple of ints, optional{]}
Output shape. Default is None, in which case a single value is
returned.

\end{description}
\begin{description}
\item[{out}] \leavevmode{[}float or ndarray{]}
Drawn samples.

\end{description}

\begin{Verbatim}[commandchars=\\\{\}]
\PYG{g+gp}{\PYGZgt{}\PYGZgt{}\PYGZgt{} }\PYG{n}{s} \PYG{o}{=} \PYG{n}{np}\PYG{o}{.}\PYG{n}{random}\PYG{o}{.}\PYG{n}{standard\PYGZus{}normal}\PYG{p}{(}\PYG{l+m+mi}{8000}\PYG{p}{)}
\PYG{g+gp}{\PYGZgt{}\PYGZgt{}\PYGZgt{} }\PYG{n}{s}
\PYG{g+go}{array([ 0.6888893 ,  0.78096262, \PYGZhy{}0.89086505, ...,  0.49876311, \PYGZsh{}random}
\PYG{g+go}{       \PYGZhy{}0.38672696, \PYGZhy{}0.4685006 ])                               \PYGZsh{}random}
\PYG{g+gp}{\PYGZgt{}\PYGZgt{}\PYGZgt{} }\PYG{n}{s}\PYG{o}{.}\PYG{n}{shape}
\PYG{g+go}{(8000,)}
\PYG{g+gp}{\PYGZgt{}\PYGZgt{}\PYGZgt{} }\PYG{n}{s} \PYG{o}{=} \PYG{n}{np}\PYG{o}{.}\PYG{n}{random}\PYG{o}{.}\PYG{n}{standard\PYGZus{}normal}\PYG{p}{(}\PYG{n}{size}\PYG{o}{=}\PYG{p}{(}\PYG{l+m+mi}{3}\PYG{p}{,} \PYG{l+m+mi}{4}\PYG{p}{,} \PYG{l+m+mi}{2}\PYG{p}{)}\PYG{p}{)}
\PYG{g+gp}{\PYGZgt{}\PYGZgt{}\PYGZgt{} }\PYG{n}{s}\PYG{o}{.}\PYG{n}{shape}
\PYG{g+go}{(3, 4, 2)}
\end{Verbatim}

\end{fulllineitems}

\index{standard\_t() (in module acsAttractorAnalysisInTime)}

\begin{fulllineitems}
\phantomsection\label{acsAttractorAnalysisInTime:acsAttractorAnalysisInTime.standard_t}\pysiglinewithargsret{\code{acsAttractorAnalysisInTime.}\bfcode{standard\_t}}{\emph{df}, \emph{size=None}}{}
Standard Student's t distribution with df degrees of freedom.

A special case of the hyperbolic distribution.
As \emph{df} gets large, the result resembles that of the standard normal
distribution (\emph{standard\_normal}).
\begin{description}
\item[{df}] \leavevmode{[}int{]}
Degrees of freedom, should be \textgreater{} 0.

\item[{size}] \leavevmode{[}int or tuple of ints, optional{]}
Output shape. Default is None, in which case a single value is
returned.

\end{description}
\begin{description}
\item[{samples}] \leavevmode{[}ndarray or scalar{]}
Drawn samples.

\end{description}

The probability density function for the t distribution is
\begin{gather}
\begin{split}P(x, df) = \frac{\Gamma(\frac{df+1}{2})}{\sqrt{\pi df}
\Gamma(\frac{df}{2})}\Bigl( 1+\frac{x^2}{df} \Bigr)^{-(df+1)/2}\end{split}\notag
\end{gather}
The t test is based on an assumption that the data come from a Normal
distribution. The t test provides a way to test whether the sample mean
(that is the mean calculated from the data) is a good estimate of the true
mean.

The derivation of the t-distribution was forst published in 1908 by William
Gisset while working for the Guinness Brewery in Dublin. Due to proprietary
issues, he had to publish under a pseudonym, and so he used the name
Student.

From Dalgaard page 83 {\color{red}\bfseries{}{[}1{]}\_}, suppose the daily energy intake for 11
women in Kj is:

\begin{Verbatim}[commandchars=\\\{\}]
\PYG{g+gp}{\PYGZgt{}\PYGZgt{}\PYGZgt{} }\PYG{n}{intake} \PYG{o}{=} \PYG{n}{np}\PYG{o}{.}\PYG{n}{array}\PYG{p}{(}\PYG{p}{[}\PYG{l+m+mf}{5260.}\PYG{p}{,} \PYG{l+m+mi}{5470}\PYG{p}{,} \PYG{l+m+mi}{5640}\PYG{p}{,} \PYG{l+m+mi}{6180}\PYG{p}{,} \PYG{l+m+mi}{6390}\PYG{p}{,} \PYG{l+m+mi}{6515}\PYG{p}{,} \PYG{l+m+mi}{6805}\PYG{p}{,} \PYG{l+m+mi}{7515}\PYG{p}{,} \PYGZbs{}
\PYG{g+gp}{... }                   \PYG{l+m+mi}{7515}\PYG{p}{,} \PYG{l+m+mi}{8230}\PYG{p}{,} \PYG{l+m+mi}{8770}\PYG{p}{]}\PYG{p}{)}
\end{Verbatim}

Does their energy intake deviate systematically from the recommended
value of 7725 kJ?

We have 10 degrees of freedom, so is the sample mean within 95\% of the
recommended value?

\begin{Verbatim}[commandchars=\\\{\}]
\PYG{g+gp}{\PYGZgt{}\PYGZgt{}\PYGZgt{} }\PYG{n}{s} \PYG{o}{=} \PYG{n}{np}\PYG{o}{.}\PYG{n}{random}\PYG{o}{.}\PYG{n}{standard\PYGZus{}t}\PYG{p}{(}\PYG{l+m+mi}{10}\PYG{p}{,} \PYG{n}{size}\PYG{o}{=}\PYG{l+m+mi}{100000}\PYG{p}{)}
\PYG{g+gp}{\PYGZgt{}\PYGZgt{}\PYGZgt{} }\PYG{n}{np}\PYG{o}{.}\PYG{n}{mean}\PYG{p}{(}\PYG{n}{intake}\PYG{p}{)}
\PYG{g+go}{6753.636363636364}
\PYG{g+gp}{\PYGZgt{}\PYGZgt{}\PYGZgt{} }\PYG{n}{intake}\PYG{o}{.}\PYG{n}{std}\PYG{p}{(}\PYG{n}{ddof}\PYG{o}{=}\PYG{l+m+mi}{1}\PYG{p}{)}
\PYG{g+go}{1142.1232221373727}
\end{Verbatim}

Calculate the t statistic, setting the ddof parameter to the unbiased
value so the divisor in the standard deviation will be degrees of
freedom, N-1.

\begin{Verbatim}[commandchars=\\\{\}]
\PYG{g+gp}{\PYGZgt{}\PYGZgt{}\PYGZgt{} }\PYG{n}{t} \PYG{o}{=} \PYG{p}{(}\PYG{n}{np}\PYG{o}{.}\PYG{n}{mean}\PYG{p}{(}\PYG{n}{intake}\PYG{p}{)}\PYG{o}{\PYGZhy{}}\PYG{l+m+mi}{7725}\PYG{p}{)}\PYG{o}{/}\PYG{p}{(}\PYG{n}{intake}\PYG{o}{.}\PYG{n}{std}\PYG{p}{(}\PYG{n}{ddof}\PYG{o}{=}\PYG{l+m+mi}{1}\PYG{p}{)}\PYG{o}{/}\PYG{n}{np}\PYG{o}{.}\PYG{n}{sqrt}\PYG{p}{(}\PYG{n+nb}{len}\PYG{p}{(}\PYG{n}{intake}\PYG{p}{)}\PYG{p}{)}\PYG{p}{)}
\PYG{g+gp}{\PYGZgt{}\PYGZgt{}\PYGZgt{} }\PYG{k+kn}{import} \PYG{n+nn}{matplotlib.pyplot} \PYG{k+kn}{as} \PYG{n+nn}{plt}
\PYG{g+gp}{\PYGZgt{}\PYGZgt{}\PYGZgt{} }\PYG{n}{h} \PYG{o}{=} \PYG{n}{plt}\PYG{o}{.}\PYG{n}{hist}\PYG{p}{(}\PYG{n}{s}\PYG{p}{,} \PYG{n}{bins}\PYG{o}{=}\PYG{l+m+mi}{100}\PYG{p}{,} \PYG{n}{normed}\PYG{o}{=}\PYG{n+nb+bp}{True}\PYG{p}{)}
\end{Verbatim}

For a one-sided t-test, how far out in the distribution does the t
statistic appear?

\begin{Verbatim}[commandchars=\\\{\}]
\PYG{g+gp}{\PYGZgt{}\PYGZgt{}\PYGZgt{} }\PYG{o}{\PYGZgt{}\PYGZgt{}}\PYG{o}{\PYGZgt{}} \PYG{n}{np}\PYG{o}{.}\PYG{n}{sum}\PYG{p}{(}\PYG{n}{s}\PYG{o}{\PYGZlt{}}\PYG{n}{t}\PYG{p}{)} \PYG{o}{/} \PYG{n+nb}{float}\PYG{p}{(}\PYG{n+nb}{len}\PYG{p}{(}\PYG{n}{s}\PYG{p}{)}\PYG{p}{)}
\PYG{g+go}{0.0090699999999999999  \PYGZsh{}random}
\end{Verbatim}

So the p-value is about 0.009, which says the null hypothesis has a
probability of about 99\% of being true.

\end{fulllineitems}

\index{triangular() (in module acsAttractorAnalysisInTime)}

\begin{fulllineitems}
\phantomsection\label{acsAttractorAnalysisInTime:acsAttractorAnalysisInTime.triangular}\pysiglinewithargsret{\code{acsAttractorAnalysisInTime.}\bfcode{triangular}}{\emph{left}, \emph{mode}, \emph{right}, \emph{size=None}}{}
Draw samples from the triangular distribution.

The triangular distribution is a continuous probability distribution with
lower limit left, peak at mode, and upper limit right. Unlike the other
distributions, these parameters directly define the shape of the pdf.
\begin{description}
\item[{left}] \leavevmode{[}scalar{]}
Lower limit.

\item[{mode}] \leavevmode{[}scalar{]}
The value where the peak of the distribution occurs.
The value should fulfill the condition \code{left \textless{}= mode \textless{}= right}.

\item[{right}] \leavevmode{[}scalar{]}
Upper limit, should be larger than \emph{left}.

\item[{size}] \leavevmode{[}int or tuple of ints, optional{]}
Output shape. Default is None, in which case a single value is
returned.

\end{description}
\begin{description}
\item[{samples}] \leavevmode{[}ndarray or scalar{]}
The returned samples all lie in the interval {[}left, right{]}.

\end{description}

The probability density function for the Triangular distribution is
\begin{gather}
\begin{split}P(x;l, m, r) = \begin{cases}
\frac{2(x-l)}{(r-l)(m-l)}& \text{for $l \leq x \leq m$},\\
\frac{2(m-x)}{(r-l)(r-m)}& \text{for $m \leq x \leq r$},\\
0& \text{otherwise}.
\end{cases}\end{split}\notag
\end{gather}
The triangular distribution is often used in ill-defined problems where the
underlying distribution is not known, but some knowledge of the limits and
mode exists. Often it is used in simulations.

Draw values from the distribution and plot the histogram:

\begin{Verbatim}[commandchars=\\\{\}]
\PYG{g+gp}{\PYGZgt{}\PYGZgt{}\PYGZgt{} }\PYG{k+kn}{import} \PYG{n+nn}{matplotlib.pyplot} \PYG{k+kn}{as} \PYG{n+nn}{plt}
\PYG{g+gp}{\PYGZgt{}\PYGZgt{}\PYGZgt{} }\PYG{n}{h} \PYG{o}{=} \PYG{n}{plt}\PYG{o}{.}\PYG{n}{hist}\PYG{p}{(}\PYG{n}{np}\PYG{o}{.}\PYG{n}{random}\PYG{o}{.}\PYG{n}{triangular}\PYG{p}{(}\PYG{o}{\PYGZhy{}}\PYG{l+m+mi}{3}\PYG{p}{,} \PYG{l+m+mi}{0}\PYG{p}{,} \PYG{l+m+mi}{8}\PYG{p}{,} \PYG{l+m+mi}{100000}\PYG{p}{)}\PYG{p}{,} \PYG{n}{bins}\PYG{o}{=}\PYG{l+m+mi}{200}\PYG{p}{,}
\PYG{g+gp}{... }             \PYG{n}{normed}\PYG{o}{=}\PYG{n+nb+bp}{True}\PYG{p}{)}
\PYG{g+gp}{\PYGZgt{}\PYGZgt{}\PYGZgt{} }\PYG{n}{plt}\PYG{o}{.}\PYG{n}{show}\PYG{p}{(}\PYG{p}{)}
\end{Verbatim}

\end{fulllineitems}

\index{uniform() (in module acsAttractorAnalysisInTime)}

\begin{fulllineitems}
\phantomsection\label{acsAttractorAnalysisInTime:acsAttractorAnalysisInTime.uniform}\pysiglinewithargsret{\code{acsAttractorAnalysisInTime.}\bfcode{uniform}}{\emph{low=0.0}, \emph{high=1.0}, \emph{size=1}}{}
Draw samples from a uniform distribution.

Samples are uniformly distributed over the half-open interval
\code{{[}low, high)} (includes low, but excludes high).  In other words,
any value within the given interval is equally likely to be drawn
by \emph{uniform}.
\begin{description}
\item[{low}] \leavevmode{[}float, optional{]}
Lower boundary of the output interval.  All values generated will be
greater than or equal to low.  The default value is 0.

\item[{high}] \leavevmode{[}float{]}
Upper boundary of the output interval.  All values generated will be
less than high.  The default value is 1.0.

\item[{size}] \leavevmode{[}int or tuple of ints, optional{]}
Shape of output.  If the given size is, for example, (m,n,k),
m*n*k samples are generated.  If no shape is specified, a single sample
is returned.

\end{description}
\begin{description}
\item[{out}] \leavevmode{[}ndarray{]}
Drawn samples, with shape \emph{size}.

\end{description}

randint : Discrete uniform distribution, yielding integers.
random\_integers : Discrete uniform distribution over the closed
\begin{quote}

interval \code{{[}low, high{]}}.
\end{quote}

random\_sample : Floats uniformly distributed over \code{{[}0, 1)}.
random : Alias for \emph{random\_sample}.
rand : Convenience function that accepts dimensions as input, e.g.,
\begin{quote}

\code{rand(2,2)} would generate a 2-by-2 array of floats,
uniformly distributed over \code{{[}0, 1)}.
\end{quote}

The probability density function of the uniform distribution is
\begin{gather}
\begin{split}p(x) = \frac{1}{b - a}\end{split}\notag
\end{gather}
anywhere within the interval \code{{[}a, b)}, and zero elsewhere.

Draw samples from the distribution:

\begin{Verbatim}[commandchars=\\\{\}]
\PYG{g+gp}{\PYGZgt{}\PYGZgt{}\PYGZgt{} }\PYG{n}{s} \PYG{o}{=} \PYG{n}{np}\PYG{o}{.}\PYG{n}{random}\PYG{o}{.}\PYG{n}{uniform}\PYG{p}{(}\PYG{o}{\PYGZhy{}}\PYG{l+m+mi}{1}\PYG{p}{,}\PYG{l+m+mi}{0}\PYG{p}{,}\PYG{l+m+mi}{1000}\PYG{p}{)}
\end{Verbatim}

All values are within the given interval:

\begin{Verbatim}[commandchars=\\\{\}]
\PYG{g+gp}{\PYGZgt{}\PYGZgt{}\PYGZgt{} }\PYG{n}{np}\PYG{o}{.}\PYG{n}{all}\PYG{p}{(}\PYG{n}{s} \PYG{o}{\PYGZgt{}}\PYG{o}{=} \PYG{o}{\PYGZhy{}}\PYG{l+m+mi}{1}\PYG{p}{)}
\PYG{g+go}{True}
\PYG{g+gp}{\PYGZgt{}\PYGZgt{}\PYGZgt{} }\PYG{n}{np}\PYG{o}{.}\PYG{n}{all}\PYG{p}{(}\PYG{n}{s} \PYG{o}{\PYGZlt{}} \PYG{l+m+mi}{0}\PYG{p}{)}
\PYG{g+go}{True}
\end{Verbatim}

Display the histogram of the samples, along with the
probability density function:

\begin{Verbatim}[commandchars=\\\{\}]
\PYG{g+gp}{\PYGZgt{}\PYGZgt{}\PYGZgt{} }\PYG{k+kn}{import} \PYG{n+nn}{matplotlib.pyplot} \PYG{k+kn}{as} \PYG{n+nn}{plt}
\PYG{g+gp}{\PYGZgt{}\PYGZgt{}\PYGZgt{} }\PYG{n}{count}\PYG{p}{,} \PYG{n}{bins}\PYG{p}{,} \PYG{n}{ignored} \PYG{o}{=} \PYG{n}{plt}\PYG{o}{.}\PYG{n}{hist}\PYG{p}{(}\PYG{n}{s}\PYG{p}{,} \PYG{l+m+mi}{15}\PYG{p}{,} \PYG{n}{normed}\PYG{o}{=}\PYG{n+nb+bp}{True}\PYG{p}{)}
\PYG{g+gp}{\PYGZgt{}\PYGZgt{}\PYGZgt{} }\PYG{n}{plt}\PYG{o}{.}\PYG{n}{plot}\PYG{p}{(}\PYG{n}{bins}\PYG{p}{,} \PYG{n}{np}\PYG{o}{.}\PYG{n}{ones\PYGZus{}like}\PYG{p}{(}\PYG{n}{bins}\PYG{p}{)}\PYG{p}{,} \PYG{n}{linewidth}\PYG{o}{=}\PYG{l+m+mi}{2}\PYG{p}{,} \PYG{n}{color}\PYG{o}{=}\PYG{l+s}{\PYGZsq{}}\PYG{l+s}{r}\PYG{l+s}{\PYGZsq{}}\PYG{p}{)}
\PYG{g+gp}{\PYGZgt{}\PYGZgt{}\PYGZgt{} }\PYG{n}{plt}\PYG{o}{.}\PYG{n}{show}\PYG{p}{(}\PYG{p}{)}
\end{Verbatim}

\end{fulllineitems}

\index{vonmises() (in module acsAttractorAnalysisInTime)}

\begin{fulllineitems}
\phantomsection\label{acsAttractorAnalysisInTime:acsAttractorAnalysisInTime.vonmises}\pysiglinewithargsret{\code{acsAttractorAnalysisInTime.}\bfcode{vonmises}}{\emph{mu}, \emph{kappa}, \emph{size=None}}{}
Draw samples from a von Mises distribution.

Samples are drawn from a von Mises distribution with specified mode
(mu) and dispersion (kappa), on the interval {[}-pi, pi{]}.

The von Mises distribution (also known as the circular normal
distribution) is a continuous probability distribution on the unit
circle.  It may be thought of as the circular analogue of the normal
distribution.
\begin{description}
\item[{mu}] \leavevmode{[}float{]}
Mode (``center'') of the distribution.

\item[{kappa}] \leavevmode{[}float{]}
Dispersion of the distribution, has to be \textgreater{}=0.

\item[{size}] \leavevmode{[}int or tuple of int{]}
Output shape.  If the given shape is, e.g., \code{(m, n, k)}, then
\code{m * n * k} samples are drawn.

\end{description}
\begin{description}
\item[{samples}] \leavevmode{[}scalar or ndarray{]}
The returned samples, which are in the interval {[}-pi, pi{]}.

\end{description}
\begin{description}
\item[{scipy.stats.distributions.vonmises}] \leavevmode{[}probability density function,{]}
distribution, or cumulative density function, etc.

\end{description}

The probability density for the von Mises distribution is
\begin{gather}
\begin{split}p(x) = \frac{e^{\kappa cos(x-\mu)}}{2\pi I_0(\kappa)},\end{split}\notag
\end{gather}
where \(\mu\) is the mode and \(\kappa\) the dispersion,
and \(I_0(\kappa)\) is the modified Bessel function of order 0.

The von Mises is named for Richard Edler von Mises, who was born in
Austria-Hungary, in what is now the Ukraine.  He fled to the United
States in 1939 and became a professor at Harvard.  He worked in
probability theory, aerodynamics, fluid mechanics, and philosophy of
science.

Abramowitz, M. and Stegun, I. A. (ed.), \emph{Handbook of Mathematical
Functions}, New York: Dover, 1965.

von Mises, R., \emph{Mathematical Theory of Probability and Statistics},
New York: Academic Press, 1964.

Draw samples from the distribution:

\begin{Verbatim}[commandchars=\\\{\}]
\PYG{g+gp}{\PYGZgt{}\PYGZgt{}\PYGZgt{} }\PYG{n}{mu}\PYG{p}{,} \PYG{n}{kappa} \PYG{o}{=} \PYG{l+m+mf}{0.0}\PYG{p}{,} \PYG{l+m+mf}{4.0} \PYG{c}{\PYGZsh{} mean and dispersion}
\PYG{g+gp}{\PYGZgt{}\PYGZgt{}\PYGZgt{} }\PYG{n}{s} \PYG{o}{=} \PYG{n}{np}\PYG{o}{.}\PYG{n}{random}\PYG{o}{.}\PYG{n}{vonmises}\PYG{p}{(}\PYG{n}{mu}\PYG{p}{,} \PYG{n}{kappa}\PYG{p}{,} \PYG{l+m+mi}{1000}\PYG{p}{)}
\end{Verbatim}

Display the histogram of the samples, along with
the probability density function:

\begin{Verbatim}[commandchars=\\\{\}]
\PYG{g+gp}{\PYGZgt{}\PYGZgt{}\PYGZgt{} }\PYG{k+kn}{import} \PYG{n+nn}{matplotlib.pyplot} \PYG{k+kn}{as} \PYG{n+nn}{plt}
\PYG{g+gp}{\PYGZgt{}\PYGZgt{}\PYGZgt{} }\PYG{k+kn}{import} \PYG{n+nn}{scipy.special} \PYG{k+kn}{as} \PYG{n+nn}{sps}
\PYG{g+gp}{\PYGZgt{}\PYGZgt{}\PYGZgt{} }\PYG{n}{count}\PYG{p}{,} \PYG{n}{bins}\PYG{p}{,} \PYG{n}{ignored} \PYG{o}{=} \PYG{n}{plt}\PYG{o}{.}\PYG{n}{hist}\PYG{p}{(}\PYG{n}{s}\PYG{p}{,} \PYG{l+m+mi}{50}\PYG{p}{,} \PYG{n}{normed}\PYG{o}{=}\PYG{n+nb+bp}{True}\PYG{p}{)}
\PYG{g+gp}{\PYGZgt{}\PYGZgt{}\PYGZgt{} }\PYG{n}{x} \PYG{o}{=} \PYG{n}{np}\PYG{o}{.}\PYG{n}{arange}\PYG{p}{(}\PYG{o}{\PYGZhy{}}\PYG{n}{np}\PYG{o}{.}\PYG{n}{pi}\PYG{p}{,} \PYG{n}{np}\PYG{o}{.}\PYG{n}{pi}\PYG{p}{,} \PYG{l+m+mi}{2}\PYG{o}{*}\PYG{n}{np}\PYG{o}{.}\PYG{n}{pi}\PYG{o}{/}\PYG{l+m+mf}{50.}\PYG{p}{)}
\PYG{g+gp}{\PYGZgt{}\PYGZgt{}\PYGZgt{} }\PYG{n}{y} \PYG{o}{=} \PYG{o}{\PYGZhy{}}\PYG{n}{np}\PYG{o}{.}\PYG{n}{exp}\PYG{p}{(}\PYG{n}{kappa}\PYG{o}{*}\PYG{n}{np}\PYG{o}{.}\PYG{n}{cos}\PYG{p}{(}\PYG{n}{x}\PYG{o}{\PYGZhy{}}\PYG{n}{mu}\PYG{p}{)}\PYG{p}{)}\PYG{o}{/}\PYG{p}{(}\PYG{l+m+mi}{2}\PYG{o}{*}\PYG{n}{np}\PYG{o}{.}\PYG{n}{pi}\PYG{o}{*}\PYG{n}{sps}\PYG{o}{.}\PYG{n}{jn}\PYG{p}{(}\PYG{l+m+mi}{0}\PYG{p}{,}\PYG{n}{kappa}\PYG{p}{)}\PYG{p}{)}
\PYG{g+gp}{\PYGZgt{}\PYGZgt{}\PYGZgt{} }\PYG{n}{plt}\PYG{o}{.}\PYG{n}{plot}\PYG{p}{(}\PYG{n}{x}\PYG{p}{,} \PYG{n}{y}\PYG{o}{/}\PYG{n+nb}{max}\PYG{p}{(}\PYG{n}{y}\PYG{p}{)}\PYG{p}{,} \PYG{n}{linewidth}\PYG{o}{=}\PYG{l+m+mi}{2}\PYG{p}{,} \PYG{n}{color}\PYG{o}{=}\PYG{l+s}{\PYGZsq{}}\PYG{l+s}{r}\PYG{l+s}{\PYGZsq{}}\PYG{p}{)}
\PYG{g+gp}{\PYGZgt{}\PYGZgt{}\PYGZgt{} }\PYG{n}{plt}\PYG{o}{.}\PYG{n}{show}\PYG{p}{(}\PYG{p}{)}
\end{Verbatim}

\end{fulllineitems}

\index{wald() (in module acsAttractorAnalysisInTime)}

\begin{fulllineitems}
\phantomsection\label{acsAttractorAnalysisInTime:acsAttractorAnalysisInTime.wald}\pysiglinewithargsret{\code{acsAttractorAnalysisInTime.}\bfcode{wald}}{\emph{mean}, \emph{scale}, \emph{size=None}}{}
Draw samples from a Wald, or Inverse Gaussian, distribution.

As the scale approaches infinity, the distribution becomes more like a
Gaussian.

Some references claim that the Wald is an Inverse Gaussian with mean=1, but
this is by no means universal.

The Inverse Gaussian distribution was first studied in relationship to
Brownian motion. In 1956 M.C.K. Tweedie used the name Inverse Gaussian
because there is an inverse relationship between the time to cover a unit
distance and distance covered in unit time.
\begin{description}
\item[{mean}] \leavevmode{[}scalar{]}
Distribution mean, should be \textgreater{} 0.

\item[{scale}] \leavevmode{[}scalar{]}
Scale parameter, should be \textgreater{}= 0.

\item[{size}] \leavevmode{[}int or tuple of ints, optional{]}
Output shape. Default is None, in which case a single value is
returned.

\end{description}
\begin{description}
\item[{samples}] \leavevmode{[}ndarray or scalar{]}
Drawn sample, all greater than zero.

\end{description}

The probability density function for the Wald distribution is
\begin{gather}
\begin{split}P(x;mean,scale) = \sqrt{\frac{scale}{2\pi x^3}}e^
\frac{-scale(x-mean)^2}{2\cdotp mean^2x}\end{split}\notag
\end{gather}
As noted above the Inverse Gaussian distribution first arise from attempts
to model Brownian Motion. It is also a competitor to the Weibull for use in
reliability modeling and modeling stock returns and interest rate
processes.

Draw values from the distribution and plot the histogram:

\begin{Verbatim}[commandchars=\\\{\}]
\PYG{g+gp}{\PYGZgt{}\PYGZgt{}\PYGZgt{} }\PYG{k+kn}{import} \PYG{n+nn}{matplotlib.pyplot} \PYG{k+kn}{as} \PYG{n+nn}{plt}
\PYG{g+gp}{\PYGZgt{}\PYGZgt{}\PYGZgt{} }\PYG{n}{h} \PYG{o}{=} \PYG{n}{plt}\PYG{o}{.}\PYG{n}{hist}\PYG{p}{(}\PYG{n}{np}\PYG{o}{.}\PYG{n}{random}\PYG{o}{.}\PYG{n}{wald}\PYG{p}{(}\PYG{l+m+mi}{3}\PYG{p}{,} \PYG{l+m+mi}{2}\PYG{p}{,} \PYG{l+m+mi}{100000}\PYG{p}{)}\PYG{p}{,} \PYG{n}{bins}\PYG{o}{=}\PYG{l+m+mi}{200}\PYG{p}{,} \PYG{n}{normed}\PYG{o}{=}\PYG{n+nb+bp}{True}\PYG{p}{)}
\PYG{g+gp}{\PYGZgt{}\PYGZgt{}\PYGZgt{} }\PYG{n}{plt}\PYG{o}{.}\PYG{n}{show}\PYG{p}{(}\PYG{p}{)}
\end{Verbatim}

\end{fulllineitems}

\index{weibull() (in module acsAttractorAnalysisInTime)}

\begin{fulllineitems}
\phantomsection\label{acsAttractorAnalysisInTime:acsAttractorAnalysisInTime.weibull}\pysiglinewithargsret{\code{acsAttractorAnalysisInTime.}\bfcode{weibull}}{\emph{a}, \emph{size=None}}{}
Weibull distribution.

Draw samples from a 1-parameter Weibull distribution with the given
shape parameter \emph{a}.
\begin{gather}
\begin{split}X = (-ln(U))^{1/a}\end{split}\notag
\end{gather}
Here, U is drawn from the uniform distribution over (0,1{]}.

The more common 2-parameter Weibull, including a scale parameter
\(\lambda\) is just \(X = \lambda(-ln(U))^{1/a}\).
\begin{description}
\item[{a}] \leavevmode{[}float{]}
Shape of the distribution.

\item[{size}] \leavevmode{[}tuple of ints{]}
Output shape.  If the given shape is, e.g., \code{(m, n, k)}, then
\code{m * n * k} samples are drawn.

\end{description}

scipy.stats.distributions.weibull\_max
scipy.stats.distributions.weibull\_min
scipy.stats.distributions.genextreme
gumbel

The Weibull (or Type III asymptotic extreme value distribution for smallest
values, SEV Type III, or Rosin-Rammler distribution) is one of a class of
Generalized Extreme Value (GEV) distributions used in modeling extreme
value problems.  This class includes the Gumbel and Frechet distributions.

The probability density for the Weibull distribution is
\begin{gather}
\begin{split}p(x) = \frac{a}
{\lambda}(\frac{x}{\lambda})^{a-1}e^{-(x/\lambda)^a},\end{split}\notag
\end{gather}
where \(a\) is the shape and \(\lambda\) the scale.

The function has its peak (the mode) at
\(\lambda(\frac{a-1}{a})^{1/a}\).

When \code{a = 1}, the Weibull distribution reduces to the exponential
distribution.

Draw samples from the distribution:

\begin{Verbatim}[commandchars=\\\{\}]
\PYG{g+gp}{\PYGZgt{}\PYGZgt{}\PYGZgt{} }\PYG{n}{a} \PYG{o}{=} \PYG{l+m+mf}{5.} \PYG{c}{\PYGZsh{} shape}
\PYG{g+gp}{\PYGZgt{}\PYGZgt{}\PYGZgt{} }\PYG{n}{s} \PYG{o}{=} \PYG{n}{np}\PYG{o}{.}\PYG{n}{random}\PYG{o}{.}\PYG{n}{weibull}\PYG{p}{(}\PYG{n}{a}\PYG{p}{,} \PYG{l+m+mi}{1000}\PYG{p}{)}
\end{Verbatim}

Display the histogram of the samples, along with
the probability density function:

\begin{Verbatim}[commandchars=\\\{\}]
\PYG{g+gp}{\PYGZgt{}\PYGZgt{}\PYGZgt{} }\PYG{k+kn}{import} \PYG{n+nn}{matplotlib.pyplot} \PYG{k+kn}{as} \PYG{n+nn}{plt}
\PYG{g+gp}{\PYGZgt{}\PYGZgt{}\PYGZgt{} }\PYG{n}{x} \PYG{o}{=} \PYG{n}{np}\PYG{o}{.}\PYG{n}{arange}\PYG{p}{(}\PYG{l+m+mi}{1}\PYG{p}{,}\PYG{l+m+mf}{100.}\PYG{p}{)}\PYG{o}{/}\PYG{l+m+mf}{50.}
\PYG{g+gp}{\PYGZgt{}\PYGZgt{}\PYGZgt{} }\PYG{k}{def} \PYG{n+nf}{weib}\PYG{p}{(}\PYG{n}{x}\PYG{p}{,}\PYG{n}{n}\PYG{p}{,}\PYG{n}{a}\PYG{p}{)}\PYG{p}{:}
\PYG{g+gp}{... }    \PYG{k}{return} \PYG{p}{(}\PYG{n}{a} \PYG{o}{/} \PYG{n}{n}\PYG{p}{)} \PYG{o}{*} \PYG{p}{(}\PYG{n}{x} \PYG{o}{/} \PYG{n}{n}\PYG{p}{)}\PYG{o}{*}\PYG{o}{*}\PYG{p}{(}\PYG{n}{a} \PYG{o}{\PYGZhy{}} \PYG{l+m+mi}{1}\PYG{p}{)} \PYG{o}{*} \PYG{n}{np}\PYG{o}{.}\PYG{n}{exp}\PYG{p}{(}\PYG{o}{\PYGZhy{}}\PYG{p}{(}\PYG{n}{x} \PYG{o}{/} \PYG{n}{n}\PYG{p}{)}\PYG{o}{*}\PYG{o}{*}\PYG{n}{a}\PYG{p}{)}
\end{Verbatim}

\begin{Verbatim}[commandchars=\\\{\}]
\PYG{g+gp}{\PYGZgt{}\PYGZgt{}\PYGZgt{} }\PYG{n}{count}\PYG{p}{,} \PYG{n}{bins}\PYG{p}{,} \PYG{n}{ignored} \PYG{o}{=} \PYG{n}{plt}\PYG{o}{.}\PYG{n}{hist}\PYG{p}{(}\PYG{n}{np}\PYG{o}{.}\PYG{n}{random}\PYG{o}{.}\PYG{n}{weibull}\PYG{p}{(}\PYG{l+m+mf}{5.}\PYG{p}{,}\PYG{l+m+mi}{1000}\PYG{p}{)}\PYG{p}{)}
\PYG{g+gp}{\PYGZgt{}\PYGZgt{}\PYGZgt{} }\PYG{n}{x} \PYG{o}{=} \PYG{n}{np}\PYG{o}{.}\PYG{n}{arange}\PYG{p}{(}\PYG{l+m+mi}{1}\PYG{p}{,}\PYG{l+m+mf}{100.}\PYG{p}{)}\PYG{o}{/}\PYG{l+m+mf}{50.}
\PYG{g+gp}{\PYGZgt{}\PYGZgt{}\PYGZgt{} }\PYG{n}{scale} \PYG{o}{=} \PYG{n}{count}\PYG{o}{.}\PYG{n}{max}\PYG{p}{(}\PYG{p}{)}\PYG{o}{/}\PYG{n}{weib}\PYG{p}{(}\PYG{n}{x}\PYG{p}{,} \PYG{l+m+mf}{1.}\PYG{p}{,} \PYG{l+m+mf}{5.}\PYG{p}{)}\PYG{o}{.}\PYG{n}{max}\PYG{p}{(}\PYG{p}{)}
\PYG{g+gp}{\PYGZgt{}\PYGZgt{}\PYGZgt{} }\PYG{n}{plt}\PYG{o}{.}\PYG{n}{plot}\PYG{p}{(}\PYG{n}{x}\PYG{p}{,} \PYG{n}{weib}\PYG{p}{(}\PYG{n}{x}\PYG{p}{,} \PYG{l+m+mf}{1.}\PYG{p}{,} \PYG{l+m+mf}{5.}\PYG{p}{)}\PYG{o}{*}\PYG{n}{scale}\PYG{p}{)}
\PYG{g+gp}{\PYGZgt{}\PYGZgt{}\PYGZgt{} }\PYG{n}{plt}\PYG{o}{.}\PYG{n}{show}\PYG{p}{(}\PYG{p}{)}
\end{Verbatim}

\end{fulllineitems}

\index{zeroBeforeStrNum() (in module acsAttractorAnalysisInTime)}

\begin{fulllineitems}
\phantomsection\label{acsAttractorAnalysisInTime:acsAttractorAnalysisInTime.zeroBeforeStrNum}\pysiglinewithargsret{\code{acsAttractorAnalysisInTime.}\bfcode{zeroBeforeStrNum}}{\emph{tmpl}, \emph{tmpL}}{}
\end{fulllineitems}

\index{zipf() (in module acsAttractorAnalysisInTime)}

\begin{fulllineitems}
\phantomsection\label{acsAttractorAnalysisInTime:acsAttractorAnalysisInTime.zipf}\pysiglinewithargsret{\code{acsAttractorAnalysisInTime.}\bfcode{zipf}}{\emph{a}, \emph{size=None}}{}
Draw samples from a Zipf distribution.

Samples are drawn from a Zipf distribution with specified parameter
\emph{a} \textgreater{} 1.

The Zipf distribution (also known as the zeta distribution) is a
continuous probability distribution that satisfies Zipf's law: the
frequency of an item is inversely proportional to its rank in a
frequency table.
\begin{description}
\item[{a}] \leavevmode{[}float \textgreater{} 1{]}
Distribution parameter.

\item[{size}] \leavevmode{[}int or tuple of int, optional{]}
Output shape.  If the given shape is, e.g., \code{(m, n, k)}, then
\code{m * n * k} samples are drawn; a single integer is equivalent in
its result to providing a mono-tuple, i.e., a 1-D array of length
\emph{size} is returned.  The default is None, in which case a single
scalar is returned.

\end{description}
\begin{description}
\item[{samples}] \leavevmode{[}scalar or ndarray{]}
The returned samples are greater than or equal to one.

\end{description}
\begin{description}
\item[{scipy.stats.distributions.zipf}] \leavevmode{[}probability density function,{]}
distribution, or cumulative density function, etc.

\end{description}

The probability density for the Zipf distribution is
\begin{gather}
\begin{split}p(x) = \frac{x^{-a}}{\zeta(a)},\end{split}\notag
\end{gather}
where \(\zeta\) is the Riemann Zeta function.

It is named for the American linguist George Kingsley Zipf, who noted
that the frequency of any word in a sample of a language is inversely
proportional to its rank in the frequency table.

Zipf, G. K., \emph{Selected Studies of the Principle of Relative Frequency
in Language}, Cambridge, MA: Harvard Univ. Press, 1932.

Draw samples from the distribution:

\begin{Verbatim}[commandchars=\\\{\}]
\PYG{g+gp}{\PYGZgt{}\PYGZgt{}\PYGZgt{} }\PYG{n}{a} \PYG{o}{=} \PYG{l+m+mf}{2.} \PYG{c}{\PYGZsh{} parameter}
\PYG{g+gp}{\PYGZgt{}\PYGZgt{}\PYGZgt{} }\PYG{n}{s} \PYG{o}{=} \PYG{n}{np}\PYG{o}{.}\PYG{n}{random}\PYG{o}{.}\PYG{n}{zipf}\PYG{p}{(}\PYG{n}{a}\PYG{p}{,} \PYG{l+m+mi}{1000}\PYG{p}{)}
\end{Verbatim}

Display the histogram of the samples, along with
the probability density function:

\begin{Verbatim}[commandchars=\\\{\}]
\PYG{g+gp}{\PYGZgt{}\PYGZgt{}\PYGZgt{} }\PYG{k+kn}{import} \PYG{n+nn}{matplotlib.pyplot} \PYG{k+kn}{as} \PYG{n+nn}{plt}
\PYG{g+gp}{\PYGZgt{}\PYGZgt{}\PYGZgt{} }\PYG{k+kn}{import} \PYG{n+nn}{scipy.special} \PYG{k+kn}{as} \PYG{n+nn}{sps}
\PYG{g+go}{Truncate s values at 50 so plot is interesting}
\PYG{g+gp}{\PYGZgt{}\PYGZgt{}\PYGZgt{} }\PYG{n}{count}\PYG{p}{,} \PYG{n}{bins}\PYG{p}{,} \PYG{n}{ignored} \PYG{o}{=} \PYG{n}{plt}\PYG{o}{.}\PYG{n}{hist}\PYG{p}{(}\PYG{n}{s}\PYG{p}{[}\PYG{n}{s}\PYG{o}{\PYGZlt{}}\PYG{l+m+mi}{50}\PYG{p}{]}\PYG{p}{,} \PYG{l+m+mi}{50}\PYG{p}{,} \PYG{n}{normed}\PYG{o}{=}\PYG{n+nb+bp}{True}\PYG{p}{)}
\PYG{g+gp}{\PYGZgt{}\PYGZgt{}\PYGZgt{} }\PYG{n}{x} \PYG{o}{=} \PYG{n}{np}\PYG{o}{.}\PYG{n}{arange}\PYG{p}{(}\PYG{l+m+mf}{1.}\PYG{p}{,} \PYG{l+m+mf}{50.}\PYG{p}{)}
\PYG{g+gp}{\PYGZgt{}\PYGZgt{}\PYGZgt{} }\PYG{n}{y} \PYG{o}{=} \PYG{n}{x}\PYG{o}{*}\PYG{o}{*}\PYG{p}{(}\PYG{o}{\PYGZhy{}}\PYG{n}{a}\PYG{p}{)}\PYG{o}{/}\PYG{n}{sps}\PYG{o}{.}\PYG{n}{zetac}\PYG{p}{(}\PYG{n}{a}\PYG{p}{)}
\PYG{g+gp}{\PYGZgt{}\PYGZgt{}\PYGZgt{} }\PYG{n}{plt}\PYG{o}{.}\PYG{n}{plot}\PYG{p}{(}\PYG{n}{x}\PYG{p}{,} \PYG{n}{y}\PYG{o}{/}\PYG{n+nb}{max}\PYG{p}{(}\PYG{n}{y}\PYG{p}{)}\PYG{p}{,} \PYG{n}{linewidth}\PYG{o}{=}\PYG{l+m+mi}{2}\PYG{p}{,} \PYG{n}{color}\PYG{o}{=}\PYG{l+s}{\PYGZsq{}}\PYG{l+s}{r}\PYG{l+s}{\PYGZsq{}}\PYG{p}{)}
\PYG{g+gp}{\PYGZgt{}\PYGZgt{}\PYGZgt{} }\PYG{n}{plt}\PYG{o}{.}\PYG{n}{show}\PYG{p}{(}\PYG{p}{)}
\end{Verbatim}

\end{fulllineitems}



\chapter{acsBufferedFluxes Module}
\label{acsBufferedFluxes:module-acsBufferedFluxes}\label{acsBufferedFluxes:acsbufferedfluxes-module}\label{acsBufferedFluxes::doc}\index{acsBufferedFluxes (module)}
Function to evaluate the activity of each species during the simulation, 
catalyst substrate product or nothing
\index{zeroBeforeStrNum() (in module acsBufferedFluxes)}

\begin{fulllineitems}
\phantomsection\label{acsBufferedFluxes:acsBufferedFluxes.zeroBeforeStrNum}\pysiglinewithargsret{\code{acsBufferedFluxes.}\bfcode{zeroBeforeStrNum}}{\emph{tmpl}, \emph{tmpL}}{}
\end{fulllineitems}



\chapter{acsDynStatInTime Module}
\label{acsDynStatInTime::doc}\label{acsDynStatInTime:acsdynstatintime-module}\label{acsDynStatInTime:module-acsDynStatInTime}\index{acsDynStatInTime (module)}
Script to order the analysis of the divergences in time.
\index{beta() (in module acsDynStatInTime)}

\begin{fulllineitems}
\phantomsection\label{acsDynStatInTime:acsDynStatInTime.beta}\pysiglinewithargsret{\code{acsDynStatInTime.}\bfcode{beta}}{\emph{a}, \emph{b}, \emph{size=None}}{}
The Beta distribution over \code{{[}0, 1{]}}.

The Beta distribution is a special case of the Dirichlet distribution,
and is related to the Gamma distribution.  It has the probability
distribution function
\begin{gather}
\begin{split}f(x; a,b) = \frac{1}{B(\alpha, \beta)} x^{\alpha - 1}
(1 - x)^{\beta - 1},\end{split}\notag
\end{gather}
where the normalisation, B, is the beta function,
\begin{gather}
\begin{split}B(\alpha, \beta) = \int_0^1 t^{\alpha - 1}
(1 - t)^{\beta - 1} dt.\end{split}\notag
\end{gather}
It is often seen in Bayesian inference and order statistics.
\begin{description}
\item[{a}] \leavevmode{[}float{]}
Alpha, non-negative.

\item[{b}] \leavevmode{[}float{]}
Beta, non-negative.

\item[{size}] \leavevmode{[}tuple of ints, optional{]}
The number of samples to draw.  The output is packed according to
the size given.

\end{description}
\begin{description}
\item[{out}] \leavevmode{[}ndarray{]}
Array of the given shape, containing values drawn from a
Beta distribution.

\end{description}

\end{fulllineitems}

\index{binomial() (in module acsDynStatInTime)}

\begin{fulllineitems}
\phantomsection\label{acsDynStatInTime:acsDynStatInTime.binomial}\pysiglinewithargsret{\code{acsDynStatInTime.}\bfcode{binomial}}{\emph{n}, \emph{p}, \emph{size=None}}{}
Draw samples from a binomial distribution.

Samples are drawn from a Binomial distribution with specified
parameters, n trials and p probability of success where
n an integer \textgreater{}= 0 and p is in the interval {[}0,1{]}. (n may be
input as a float, but it is truncated to an integer in use)
\begin{description}
\item[{n}] \leavevmode{[}float (but truncated to an integer){]}
parameter, \textgreater{}= 0.

\item[{p}] \leavevmode{[}float{]}
parameter, \textgreater{}= 0 and \textless{}=1.

\item[{size}] \leavevmode{[}\{tuple, int\}{]}
Output shape.  If the given shape is, e.g., \code{(m, n, k)}, then
\code{m * n * k} samples are drawn.

\end{description}
\begin{description}
\item[{samples}] \leavevmode{[}\{ndarray, scalar\}{]}
where the values are all integers in  {[}0, n{]}.

\end{description}
\begin{description}
\item[{scipy.stats.distributions.binom}] \leavevmode{[}probability density function,{]}
distribution or cumulative density function, etc.

\end{description}

The probability density for the Binomial distribution is
\begin{gather}
\begin{split}P(N) = \binom{n}{N}p^N(1-p)^{n-N},\end{split}\notag
\end{gather}
where \(n\) is the number of trials, \(p\) is the probability
of success, and \(N\) is the number of successes.

When estimating the standard error of a proportion in a population by
using a random sample, the normal distribution works well unless the
product p*n \textless{}=5, where p = population proportion estimate, and n =
number of samples, in which case the binomial distribution is used
instead. For example, a sample of 15 people shows 4 who are left
handed, and 11 who are right handed. Then p = 4/15 = 27\%. 0.27*15 = 4,
so the binomial distribution should be used in this case.

Draw samples from the distribution:

\begin{Verbatim}[commandchars=\\\{\}]
\PYG{g+gp}{\PYGZgt{}\PYGZgt{}\PYGZgt{} }\PYG{n}{n}\PYG{p}{,} \PYG{n}{p} \PYG{o}{=} \PYG{l+m+mi}{10}\PYG{p}{,} \PYG{o}{.}\PYG{l+m+mi}{5} \PYG{c}{\PYGZsh{} number of trials, probability of each trial}
\PYG{g+gp}{\PYGZgt{}\PYGZgt{}\PYGZgt{} }\PYG{n}{s} \PYG{o}{=} \PYG{n}{np}\PYG{o}{.}\PYG{n}{random}\PYG{o}{.}\PYG{n}{binomial}\PYG{p}{(}\PYG{n}{n}\PYG{p}{,} \PYG{n}{p}\PYG{p}{,} \PYG{l+m+mi}{1000}\PYG{p}{)}
\PYG{g+go}{\PYGZsh{} result of flipping a coin 10 times, tested 1000 times.}
\end{Verbatim}

A real world example. A company drills 9 wild-cat oil exploration
wells, each with an estimated probability of success of 0.1. All nine
wells fail. What is the probability of that happening?

Let's do 20,000 trials of the model, and count the number that
generate zero positive results.

\begin{Verbatim}[commandchars=\\\{\}]
\PYG{g+gp}{\PYGZgt{}\PYGZgt{}\PYGZgt{} }\PYG{n+nb}{sum}\PYG{p}{(}\PYG{n}{np}\PYG{o}{.}\PYG{n}{random}\PYG{o}{.}\PYG{n}{binomial}\PYG{p}{(}\PYG{l+m+mi}{9}\PYG{p}{,}\PYG{l+m+mf}{0.1}\PYG{p}{,}\PYG{l+m+mi}{20000}\PYG{p}{)}\PYG{o}{==}\PYG{l+m+mi}{0}\PYG{p}{)}\PYG{o}{/}\PYG{l+m+mf}{20000.}
\PYG{g+go}{answer = 0.38885, or 38\PYGZpc{}.}
\end{Verbatim}

\end{fulllineitems}

\index{chisquare() (in module acsDynStatInTime)}

\begin{fulllineitems}
\phantomsection\label{acsDynStatInTime:acsDynStatInTime.chisquare}\pysiglinewithargsret{\code{acsDynStatInTime.}\bfcode{chisquare}}{\emph{df}, \emph{size=None}}{}
Draw samples from a chi-square distribution.

When \emph{df} independent random variables, each with standard normal
distributions (mean 0, variance 1), are squared and summed, the
resulting distribution is chi-square (see Notes).  This distribution
is often used in hypothesis testing.
\begin{description}
\item[{df}] \leavevmode{[}int{]}
Number of degrees of freedom.

\item[{size}] \leavevmode{[}tuple of ints, int, optional{]}
Size of the returned array.  By default, a scalar is
returned.

\end{description}
\begin{description}
\item[{output}] \leavevmode{[}ndarray{]}
Samples drawn from the distribution, packed in a \emph{size}-shaped
array.

\end{description}
\begin{description}
\item[{ValueError}] \leavevmode
When \emph{df} \textless{}= 0 or when an inappropriate \emph{size} (e.g. \code{size=-1})
is given.

\end{description}

The variable obtained by summing the squares of \emph{df} independent,
standard normally distributed random variables:
\begin{gather}
\begin{split}Q = \sum_{i=0}^{\mathtt{df}} X^2_i\end{split}\notag
\end{gather}
is chi-square distributed, denoted
\begin{gather}
\begin{split}Q \sim \chi^2_k.\end{split}\notag
\end{gather}
The probability density function of the chi-squared distribution is
\begin{gather}
\begin{split}p(x) = \frac{(1/2)^{k/2}}{\Gamma(k/2)}
x^{k/2 - 1} e^{-x/2},\end{split}\notag
\end{gather}
where \(\Gamma\) is the gamma function,
\begin{gather}
\begin{split}\Gamma(x) = \int_0^{-\infty} t^{x - 1} e^{-t} dt.\end{split}\notag
\end{gather}
\href{http://www.itl.nist.gov/div898/handbook/eda/section3/eda3666.htm}{NIST/SEMATECH e-Handbook of Statistical Methods}

\begin{Verbatim}[commandchars=\\\{\}]
\PYG{g+gp}{\PYGZgt{}\PYGZgt{}\PYGZgt{} }\PYG{n}{np}\PYG{o}{.}\PYG{n}{random}\PYG{o}{.}\PYG{n}{chisquare}\PYG{p}{(}\PYG{l+m+mi}{2}\PYG{p}{,}\PYG{l+m+mi}{4}\PYG{p}{)}
\PYG{g+go}{array([ 1.89920014,  9.00867716,  3.13710533,  5.62318272])}
\end{Verbatim}

\end{fulllineitems}

\index{exponential() (in module acsDynStatInTime)}

\begin{fulllineitems}
\phantomsection\label{acsDynStatInTime:acsDynStatInTime.exponential}\pysiglinewithargsret{\code{acsDynStatInTime.}\bfcode{exponential}}{\emph{scale=1.0}, \emph{size=None}}{}
Exponential distribution.

Its probability density function is
\begin{gather}
\begin{split}f(x; \frac{1}{\beta}) = \frac{1}{\beta} \exp(-\frac{x}{\beta}),\end{split}\notag
\end{gather}
for \code{x \textgreater{} 0} and 0 elsewhere. \(\beta\) is the scale parameter,
which is the inverse of the rate parameter \(\lambda = 1/\beta\).
The rate parameter is an alternative, widely used parameterization
of the exponential distribution {\color{red}\bfseries{}{[}3{]}\_}.

The exponential distribution is a continuous analogue of the
geometric distribution.  It describes many common situations, such as
the size of raindrops measured over many rainstorms {\color{red}\bfseries{}{[}1{]}\_}, or the time
between page requests to Wikipedia {\color{red}\bfseries{}{[}2{]}\_}.
\begin{description}
\item[{scale}] \leavevmode{[}float{]}
The scale parameter, \(\beta = 1/\lambda\).

\item[{size}] \leavevmode{[}tuple of ints{]}
Number of samples to draw.  The output is shaped
according to \emph{size}.

\end{description}

\end{fulllineitems}

\index{f() (in module acsDynStatInTime)}

\begin{fulllineitems}
\phantomsection\label{acsDynStatInTime:acsDynStatInTime.f}\pysiglinewithargsret{\code{acsDynStatInTime.}\bfcode{f}}{\emph{dfnum}, \emph{dfden}, \emph{size=None}}{}
Draw samples from a F distribution.

Samples are drawn from an F distribution with specified parameters,
\emph{dfnum} (degrees of freedom in numerator) and \emph{dfden} (degrees of freedom
in denominator), where both parameters should be greater than zero.

The random variate of the F distribution (also known as the
Fisher distribution) is a continuous probability distribution
that arises in ANOVA tests, and is the ratio of two chi-square
variates.
\begin{description}
\item[{dfnum}] \leavevmode{[}float{]}
Degrees of freedom in numerator. Should be greater than zero.

\item[{dfden}] \leavevmode{[}float{]}
Degrees of freedom in denominator. Should be greater than zero.

\item[{size}] \leavevmode{[}\{tuple, int\}, optional{]}
Output shape.  If the given shape is, e.g., \code{(m, n, k)},
then \code{m * n * k} samples are drawn. By default only one sample
is returned.

\end{description}
\begin{description}
\item[{samples}] \leavevmode{[}\{ndarray, scalar\}{]}
Samples from the Fisher distribution.

\end{description}
\begin{description}
\item[{scipy.stats.distributions.f}] \leavevmode{[}probability density function,{]}
distribution or cumulative density function, etc.

\end{description}

The F statistic is used to compare in-group variances to between-group
variances. Calculating the distribution depends on the sampling, and
so it is a function of the respective degrees of freedom in the
problem.  The variable \emph{dfnum} is the number of samples minus one, the
between-groups degrees of freedom, while \emph{dfden} is the within-groups
degrees of freedom, the sum of the number of samples in each group
minus the number of groups.

An example from Glantz{[}1{]}, pp 47-40.
Two groups, children of diabetics (25 people) and children from people
without diabetes (25 controls). Fasting blood glucose was measured,
case group had a mean value of 86.1, controls had a mean value of
82.2. Standard deviations were 2.09 and 2.49 respectively. Are these
data consistent with the null hypothesis that the parents diabetic
status does not affect their children's blood glucose levels?
Calculating the F statistic from the data gives a value of 36.01.

Draw samples from the distribution:

\begin{Verbatim}[commandchars=\\\{\}]
\PYG{g+gp}{\PYGZgt{}\PYGZgt{}\PYGZgt{} }\PYG{n}{dfnum} \PYG{o}{=} \PYG{l+m+mf}{1.} \PYG{c}{\PYGZsh{} between group degrees of freedom}
\PYG{g+gp}{\PYGZgt{}\PYGZgt{}\PYGZgt{} }\PYG{n}{dfden} \PYG{o}{=} \PYG{l+m+mf}{48.} \PYG{c}{\PYGZsh{} within groups degrees of freedom}
\PYG{g+gp}{\PYGZgt{}\PYGZgt{}\PYGZgt{} }\PYG{n}{s} \PYG{o}{=} \PYG{n}{np}\PYG{o}{.}\PYG{n}{random}\PYG{o}{.}\PYG{n}{f}\PYG{p}{(}\PYG{n}{dfnum}\PYG{p}{,} \PYG{n}{dfden}\PYG{p}{,} \PYG{l+m+mi}{1000}\PYG{p}{)}
\end{Verbatim}

The lower bound for the top 1\% of the samples is :

\begin{Verbatim}[commandchars=\\\{\}]
\PYG{g+gp}{\PYGZgt{}\PYGZgt{}\PYGZgt{} }\PYG{n}{sort}\PYG{p}{(}\PYG{n}{s}\PYG{p}{)}\PYG{p}{[}\PYG{o}{\PYGZhy{}}\PYG{l+m+mi}{10}\PYG{p}{]}
\PYG{g+go}{7.61988120985}
\end{Verbatim}

So there is about a 1\% chance that the F statistic will exceed 7.62,
the measured value is 36, so the null hypothesis is rejected at the 1\%
level.

\end{fulllineitems}

\index{gamma() (in module acsDynStatInTime)}

\begin{fulllineitems}
\phantomsection\label{acsDynStatInTime:acsDynStatInTime.gamma}\pysiglinewithargsret{\code{acsDynStatInTime.}\bfcode{gamma}}{\emph{shape}, \emph{scale=1.0}, \emph{size=None}}{}
Draw samples from a Gamma distribution.

Samples are drawn from a Gamma distribution with specified parameters,
\emph{shape} (sometimes designated ``k'') and \emph{scale} (sometimes designated
``theta''), where both parameters are \textgreater{} 0.
\begin{description}
\item[{shape}] \leavevmode{[}scalar \textgreater{} 0{]}
The shape of the gamma distribution.

\item[{scale}] \leavevmode{[}scalar \textgreater{} 0, optional{]}
The scale of the gamma distribution.  Default is equal to 1.

\item[{size}] \leavevmode{[}shape\_tuple, optional{]}
Output shape.  If the given shape is, e.g., \code{(m, n, k)}, then
\code{m * n * k} samples are drawn.

\end{description}
\begin{description}
\item[{out}] \leavevmode{[}ndarray, float{]}
Returns one sample unless \emph{size} parameter is specified.

\end{description}
\begin{description}
\item[{scipy.stats.distributions.gamma}] \leavevmode{[}probability density function,{]}
distribution or cumulative density function, etc.

\end{description}

The probability density for the Gamma distribution is
\begin{gather}
\begin{split}p(x) = x^{k-1}\frac{e^{-x/\theta}}{\theta^k\Gamma(k)},\end{split}\notag
\end{gather}
where \(k\) is the shape and \(\theta\) the scale,
and \(\Gamma\) is the Gamma function.

The Gamma distribution is often used to model the times to failure of
electronic components, and arises naturally in processes for which the
waiting times between Poisson distributed events are relevant.

Draw samples from the distribution:

\begin{Verbatim}[commandchars=\\\{\}]
\PYG{g+gp}{\PYGZgt{}\PYGZgt{}\PYGZgt{} }\PYG{n}{shape}\PYG{p}{,} \PYG{n}{scale} \PYG{o}{=} \PYG{l+m+mf}{2.}\PYG{p}{,} \PYG{l+m+mf}{2.} \PYG{c}{\PYGZsh{} mean and dispersion}
\PYG{g+gp}{\PYGZgt{}\PYGZgt{}\PYGZgt{} }\PYG{n}{s} \PYG{o}{=} \PYG{n}{np}\PYG{o}{.}\PYG{n}{random}\PYG{o}{.}\PYG{n}{gamma}\PYG{p}{(}\PYG{n}{shape}\PYG{p}{,} \PYG{n}{scale}\PYG{p}{,} \PYG{l+m+mi}{1000}\PYG{p}{)}
\end{Verbatim}

Display the histogram of the samples, along with
the probability density function:

\begin{Verbatim}[commandchars=\\\{\}]
\PYG{g+gp}{\PYGZgt{}\PYGZgt{}\PYGZgt{} }\PYG{k+kn}{import} \PYG{n+nn}{matplotlib.pyplot} \PYG{k+kn}{as} \PYG{n+nn}{plt}
\PYG{g+gp}{\PYGZgt{}\PYGZgt{}\PYGZgt{} }\PYG{k+kn}{import} \PYG{n+nn}{scipy.special} \PYG{k+kn}{as} \PYG{n+nn}{sps}
\PYG{g+gp}{\PYGZgt{}\PYGZgt{}\PYGZgt{} }\PYG{n}{count}\PYG{p}{,} \PYG{n}{bins}\PYG{p}{,} \PYG{n}{ignored} \PYG{o}{=} \PYG{n}{plt}\PYG{o}{.}\PYG{n}{hist}\PYG{p}{(}\PYG{n}{s}\PYG{p}{,} \PYG{l+m+mi}{50}\PYG{p}{,} \PYG{n}{normed}\PYG{o}{=}\PYG{n+nb+bp}{True}\PYG{p}{)}
\PYG{g+gp}{\PYGZgt{}\PYGZgt{}\PYGZgt{} }\PYG{n}{y} \PYG{o}{=} \PYG{n}{bins}\PYG{o}{*}\PYG{o}{*}\PYG{p}{(}\PYG{n}{shape}\PYG{o}{\PYGZhy{}}\PYG{l+m+mi}{1}\PYG{p}{)}\PYG{o}{*}\PYG{p}{(}\PYG{n}{np}\PYG{o}{.}\PYG{n}{exp}\PYG{p}{(}\PYG{o}{\PYGZhy{}}\PYG{n}{bins}\PYG{o}{/}\PYG{n}{scale}\PYG{p}{)} \PYG{o}{/}
\PYG{g+gp}{... }                     \PYG{p}{(}\PYG{n}{sps}\PYG{o}{.}\PYG{n}{gamma}\PYG{p}{(}\PYG{n}{shape}\PYG{p}{)}\PYG{o}{*}\PYG{n}{scale}\PYG{o}{*}\PYG{o}{*}\PYG{n}{shape}\PYG{p}{)}\PYG{p}{)}
\PYG{g+gp}{\PYGZgt{}\PYGZgt{}\PYGZgt{} }\PYG{n}{plt}\PYG{o}{.}\PYG{n}{plot}\PYG{p}{(}\PYG{n}{bins}\PYG{p}{,} \PYG{n}{y}\PYG{p}{,} \PYG{n}{linewidth}\PYG{o}{=}\PYG{l+m+mi}{2}\PYG{p}{,} \PYG{n}{color}\PYG{o}{=}\PYG{l+s}{\PYGZsq{}}\PYG{l+s}{r}\PYG{l+s}{\PYGZsq{}}\PYG{p}{)}
\PYG{g+gp}{\PYGZgt{}\PYGZgt{}\PYGZgt{} }\PYG{n}{plt}\PYG{o}{.}\PYG{n}{show}\PYG{p}{(}\PYG{p}{)}
\end{Verbatim}

\end{fulllineitems}

\index{geometric() (in module acsDynStatInTime)}

\begin{fulllineitems}
\phantomsection\label{acsDynStatInTime:acsDynStatInTime.geometric}\pysiglinewithargsret{\code{acsDynStatInTime.}\bfcode{geometric}}{\emph{p}, \emph{size=None}}{}
Draw samples from the geometric distribution.

Bernoulli trials are experiments with one of two outcomes:
success or failure (an example of such an experiment is flipping
a coin).  The geometric distribution models the number of trials
that must be run in order to achieve success.  It is therefore
supported on the positive integers, \code{k = 1, 2, ...}.

The probability mass function of the geometric distribution is
\begin{gather}
\begin{split}f(k) = (1 - p)^{k - 1} p\end{split}\notag
\end{gather}
where \emph{p} is the probability of success of an individual trial.
\begin{description}
\item[{p}] \leavevmode{[}float{]}
The probability of success of an individual trial.

\item[{size}] \leavevmode{[}tuple of ints{]}
Number of values to draw from the distribution.  The output
is shaped according to \emph{size}.

\end{description}
\begin{description}
\item[{out}] \leavevmode{[}ndarray{]}
Samples from the geometric distribution, shaped according to
\emph{size}.

\end{description}

Draw ten thousand values from the geometric distribution,
with the probability of an individual success equal to 0.35:

\begin{Verbatim}[commandchars=\\\{\}]
\PYG{g+gp}{\PYGZgt{}\PYGZgt{}\PYGZgt{} }\PYG{n}{z} \PYG{o}{=} \PYG{n}{np}\PYG{o}{.}\PYG{n}{random}\PYG{o}{.}\PYG{n}{geometric}\PYG{p}{(}\PYG{n}{p}\PYG{o}{=}\PYG{l+m+mf}{0.35}\PYG{p}{,} \PYG{n}{size}\PYG{o}{=}\PYG{l+m+mi}{10000}\PYG{p}{)}
\end{Verbatim}

How many trials succeeded after a single run?

\begin{Verbatim}[commandchars=\\\{\}]
\PYG{g+gp}{\PYGZgt{}\PYGZgt{}\PYGZgt{} }\PYG{p}{(}\PYG{n}{z} \PYG{o}{==} \PYG{l+m+mi}{1}\PYG{p}{)}\PYG{o}{.}\PYG{n}{sum}\PYG{p}{(}\PYG{p}{)} \PYG{o}{/} \PYG{l+m+mf}{10000.}
\PYG{g+go}{0.34889999999999999 \PYGZsh{}random}
\end{Verbatim}

\end{fulllineitems}

\index{get\_state() (in module acsDynStatInTime)}

\begin{fulllineitems}
\phantomsection\label{acsDynStatInTime:acsDynStatInTime.get_state}\pysiglinewithargsret{\code{acsDynStatInTime.}\bfcode{get\_state}}{}{}
Return a tuple representing the internal state of the generator.

For more details, see \emph{set\_state}.
\begin{description}
\item[{out}] \leavevmode{[}tuple(str, ndarray of 624 uints, int, int, float){]}
The returned tuple has the following items:
\begin{enumerate}
\item {} 
the string `MT19937'.

\item {} 
a 1-D array of 624 unsigned integer keys.

\item {} 
an integer \code{pos}.

\item {} 
an integer \code{has\_gauss}.

\item {} 
a float \code{cached\_gaussian}.

\end{enumerate}

\end{description}

set\_state

\emph{set\_state} and \emph{get\_state} are not needed to work with any of the
random distributions in NumPy. If the internal state is manually altered,
the user should know exactly what he/she is doing.

\end{fulllineitems}

\index{gumbel() (in module acsDynStatInTime)}

\begin{fulllineitems}
\phantomsection\label{acsDynStatInTime:acsDynStatInTime.gumbel}\pysiglinewithargsret{\code{acsDynStatInTime.}\bfcode{gumbel}}{\emph{loc=0.0}, \emph{scale=1.0}, \emph{size=None}}{}
Gumbel distribution.

Draw samples from a Gumbel distribution with specified location and scale.
For more information on the Gumbel distribution, see Notes and References
below.
\begin{description}
\item[{loc}] \leavevmode{[}float{]}
The location of the mode of the distribution.

\item[{scale}] \leavevmode{[}float{]}
The scale parameter of the distribution.

\item[{size}] \leavevmode{[}tuple of ints{]}
Output shape.  If the given shape is, e.g., \code{(m, n, k)}, then
\code{m * n * k} samples are drawn.

\end{description}
\begin{description}
\item[{out}] \leavevmode{[}ndarray{]}
The samples

\end{description}

scipy.stats.gumbel\_l
scipy.stats.gumbel\_r
scipy.stats.genextreme
\begin{quote}

probability density function, distribution, or cumulative density
function, etc. for each of the above
\end{quote}

weibull

The Gumbel (or Smallest Extreme Value (SEV) or the Smallest Extreme Value
Type I) distribution is one of a class of Generalized Extreme Value (GEV)
distributions used in modeling extreme value problems.  The Gumbel is a
special case of the Extreme Value Type I distribution for maximums from
distributions with ``exponential-like'' tails.

The probability density for the Gumbel distribution is
\begin{gather}
\begin{split}p(x) = \frac{e^{-(x - \mu)/ \beta}}{\beta} e^{ -e^{-(x - \mu)/
\beta}},\end{split}\notag
\end{gather}
where \(\mu\) is the mode, a location parameter, and \(\beta\) is
the scale parameter.

The Gumbel (named for German mathematician Emil Julius Gumbel) was used
very early in the hydrology literature, for modeling the occurrence of
flood events. It is also used for modeling maximum wind speed and rainfall
rates.  It is a ``fat-tailed'' distribution - the probability of an event in
the tail of the distribution is larger than if one used a Gaussian, hence
the surprisingly frequent occurrence of 100-year floods. Floods were
initially modeled as a Gaussian process, which underestimated the frequency
of extreme events.

It is one of a class of extreme value distributions, the Generalized
Extreme Value (GEV) distributions, which also includes the Weibull and
Frechet.

The function has a mean of \(\mu + 0.57721\beta\) and a variance of
\(\frac{\pi^2}{6}\beta^2\).

Gumbel, E. J., \emph{Statistics of Extremes}, New York: Columbia University
Press, 1958.

Reiss, R.-D. and Thomas, M., \emph{Statistical Analysis of Extreme Values from
Insurance, Finance, Hydrology and Other Fields}, Basel: Birkhauser Verlag,
2001.

Draw samples from the distribution:

\begin{Verbatim}[commandchars=\\\{\}]
\PYG{g+gp}{\PYGZgt{}\PYGZgt{}\PYGZgt{} }\PYG{n}{mu}\PYG{p}{,} \PYG{n}{beta} \PYG{o}{=} \PYG{l+m+mi}{0}\PYG{p}{,} \PYG{l+m+mf}{0.1} \PYG{c}{\PYGZsh{} location and scale}
\PYG{g+gp}{\PYGZgt{}\PYGZgt{}\PYGZgt{} }\PYG{n}{s} \PYG{o}{=} \PYG{n}{np}\PYG{o}{.}\PYG{n}{random}\PYG{o}{.}\PYG{n}{gumbel}\PYG{p}{(}\PYG{n}{mu}\PYG{p}{,} \PYG{n}{beta}\PYG{p}{,} \PYG{l+m+mi}{1000}\PYG{p}{)}
\end{Verbatim}

Display the histogram of the samples, along with
the probability density function:

\begin{Verbatim}[commandchars=\\\{\}]
\PYG{g+gp}{\PYGZgt{}\PYGZgt{}\PYGZgt{} }\PYG{k+kn}{import} \PYG{n+nn}{matplotlib.pyplot} \PYG{k+kn}{as} \PYG{n+nn}{plt}
\PYG{g+gp}{\PYGZgt{}\PYGZgt{}\PYGZgt{} }\PYG{n}{count}\PYG{p}{,} \PYG{n}{bins}\PYG{p}{,} \PYG{n}{ignored} \PYG{o}{=} \PYG{n}{plt}\PYG{o}{.}\PYG{n}{hist}\PYG{p}{(}\PYG{n}{s}\PYG{p}{,} \PYG{l+m+mi}{30}\PYG{p}{,} \PYG{n}{normed}\PYG{o}{=}\PYG{n+nb+bp}{True}\PYG{p}{)}
\PYG{g+gp}{\PYGZgt{}\PYGZgt{}\PYGZgt{} }\PYG{n}{plt}\PYG{o}{.}\PYG{n}{plot}\PYG{p}{(}\PYG{n}{bins}\PYG{p}{,} \PYG{p}{(}\PYG{l+m+mi}{1}\PYG{o}{/}\PYG{n}{beta}\PYG{p}{)}\PYG{o}{*}\PYG{n}{np}\PYG{o}{.}\PYG{n}{exp}\PYG{p}{(}\PYG{o}{\PYGZhy{}}\PYG{p}{(}\PYG{n}{bins} \PYG{o}{\PYGZhy{}} \PYG{n}{mu}\PYG{p}{)}\PYG{o}{/}\PYG{n}{beta}\PYG{p}{)}
\PYG{g+gp}{... }         \PYG{o}{*} \PYG{n}{np}\PYG{o}{.}\PYG{n}{exp}\PYG{p}{(} \PYG{o}{\PYGZhy{}}\PYG{n}{np}\PYG{o}{.}\PYG{n}{exp}\PYG{p}{(} \PYG{o}{\PYGZhy{}}\PYG{p}{(}\PYG{n}{bins} \PYG{o}{\PYGZhy{}} \PYG{n}{mu}\PYG{p}{)} \PYG{o}{/}\PYG{n}{beta}\PYG{p}{)} \PYG{p}{)}\PYG{p}{,}
\PYG{g+gp}{... }         \PYG{n}{linewidth}\PYG{o}{=}\PYG{l+m+mi}{2}\PYG{p}{,} \PYG{n}{color}\PYG{o}{=}\PYG{l+s}{\PYGZsq{}}\PYG{l+s}{r}\PYG{l+s}{\PYGZsq{}}\PYG{p}{)}
\PYG{g+gp}{\PYGZgt{}\PYGZgt{}\PYGZgt{} }\PYG{n}{plt}\PYG{o}{.}\PYG{n}{show}\PYG{p}{(}\PYG{p}{)}
\end{Verbatim}

Show how an extreme value distribution can arise from a Gaussian process
and compare to a Gaussian:

\begin{Verbatim}[commandchars=\\\{\}]
\PYG{g+gp}{\PYGZgt{}\PYGZgt{}\PYGZgt{} }\PYG{n}{means} \PYG{o}{=} \PYG{p}{[}\PYG{p}{]}
\PYG{g+gp}{\PYGZgt{}\PYGZgt{}\PYGZgt{} }\PYG{n}{maxima} \PYG{o}{=} \PYG{p}{[}\PYG{p}{]}
\PYG{g+gp}{\PYGZgt{}\PYGZgt{}\PYGZgt{} }\PYG{k}{for} \PYG{n}{i} \PYG{o+ow}{in} \PYG{n+nb}{range}\PYG{p}{(}\PYG{l+m+mi}{0}\PYG{p}{,}\PYG{l+m+mi}{1000}\PYG{p}{)} \PYG{p}{:}
\PYG{g+gp}{... }   \PYG{n}{a} \PYG{o}{=} \PYG{n}{np}\PYG{o}{.}\PYG{n}{random}\PYG{o}{.}\PYG{n}{normal}\PYG{p}{(}\PYG{n}{mu}\PYG{p}{,} \PYG{n}{beta}\PYG{p}{,} \PYG{l+m+mi}{1000}\PYG{p}{)}
\PYG{g+gp}{... }   \PYG{n}{means}\PYG{o}{.}\PYG{n}{append}\PYG{p}{(}\PYG{n}{a}\PYG{o}{.}\PYG{n}{mean}\PYG{p}{(}\PYG{p}{)}\PYG{p}{)}
\PYG{g+gp}{... }   \PYG{n}{maxima}\PYG{o}{.}\PYG{n}{append}\PYG{p}{(}\PYG{n}{a}\PYG{o}{.}\PYG{n}{max}\PYG{p}{(}\PYG{p}{)}\PYG{p}{)}
\PYG{g+gp}{\PYGZgt{}\PYGZgt{}\PYGZgt{} }\PYG{n}{count}\PYG{p}{,} \PYG{n}{bins}\PYG{p}{,} \PYG{n}{ignored} \PYG{o}{=} \PYG{n}{plt}\PYG{o}{.}\PYG{n}{hist}\PYG{p}{(}\PYG{n}{maxima}\PYG{p}{,} \PYG{l+m+mi}{30}\PYG{p}{,} \PYG{n}{normed}\PYG{o}{=}\PYG{n+nb+bp}{True}\PYG{p}{)}
\PYG{g+gp}{\PYGZgt{}\PYGZgt{}\PYGZgt{} }\PYG{n}{beta} \PYG{o}{=} \PYG{n}{np}\PYG{o}{.}\PYG{n}{std}\PYG{p}{(}\PYG{n}{maxima}\PYG{p}{)}\PYG{o}{*}\PYG{n}{np}\PYG{o}{.}\PYG{n}{pi}\PYG{o}{/}\PYG{n}{np}\PYG{o}{.}\PYG{n}{sqrt}\PYG{p}{(}\PYG{l+m+mi}{6}\PYG{p}{)}
\PYG{g+gp}{\PYGZgt{}\PYGZgt{}\PYGZgt{} }\PYG{n}{mu} \PYG{o}{=} \PYG{n}{np}\PYG{o}{.}\PYG{n}{mean}\PYG{p}{(}\PYG{n}{maxima}\PYG{p}{)} \PYG{o}{\PYGZhy{}} \PYG{l+m+mf}{0.57721}\PYG{o}{*}\PYG{n}{beta}
\PYG{g+gp}{\PYGZgt{}\PYGZgt{}\PYGZgt{} }\PYG{n}{plt}\PYG{o}{.}\PYG{n}{plot}\PYG{p}{(}\PYG{n}{bins}\PYG{p}{,} \PYG{p}{(}\PYG{l+m+mi}{1}\PYG{o}{/}\PYG{n}{beta}\PYG{p}{)}\PYG{o}{*}\PYG{n}{np}\PYG{o}{.}\PYG{n}{exp}\PYG{p}{(}\PYG{o}{\PYGZhy{}}\PYG{p}{(}\PYG{n}{bins} \PYG{o}{\PYGZhy{}} \PYG{n}{mu}\PYG{p}{)}\PYG{o}{/}\PYG{n}{beta}\PYG{p}{)}
\PYG{g+gp}{... }         \PYG{o}{*} \PYG{n}{np}\PYG{o}{.}\PYG{n}{exp}\PYG{p}{(}\PYG{o}{\PYGZhy{}}\PYG{n}{np}\PYG{o}{.}\PYG{n}{exp}\PYG{p}{(}\PYG{o}{\PYGZhy{}}\PYG{p}{(}\PYG{n}{bins} \PYG{o}{\PYGZhy{}} \PYG{n}{mu}\PYG{p}{)}\PYG{o}{/}\PYG{n}{beta}\PYG{p}{)}\PYG{p}{)}\PYG{p}{,}
\PYG{g+gp}{... }         \PYG{n}{linewidth}\PYG{o}{=}\PYG{l+m+mi}{2}\PYG{p}{,} \PYG{n}{color}\PYG{o}{=}\PYG{l+s}{\PYGZsq{}}\PYG{l+s}{r}\PYG{l+s}{\PYGZsq{}}\PYG{p}{)}
\PYG{g+gp}{\PYGZgt{}\PYGZgt{}\PYGZgt{} }\PYG{n}{plt}\PYG{o}{.}\PYG{n}{plot}\PYG{p}{(}\PYG{n}{bins}\PYG{p}{,} \PYG{l+m+mi}{1}\PYG{o}{/}\PYG{p}{(}\PYG{n}{beta} \PYG{o}{*} \PYG{n}{np}\PYG{o}{.}\PYG{n}{sqrt}\PYG{p}{(}\PYG{l+m+mi}{2} \PYG{o}{*} \PYG{n}{np}\PYG{o}{.}\PYG{n}{pi}\PYG{p}{)}\PYG{p}{)}
\PYG{g+gp}{... }         \PYG{o}{*} \PYG{n}{np}\PYG{o}{.}\PYG{n}{exp}\PYG{p}{(}\PYG{o}{\PYGZhy{}}\PYG{p}{(}\PYG{n}{bins} \PYG{o}{\PYGZhy{}} \PYG{n}{mu}\PYG{p}{)}\PYG{o}{*}\PYG{o}{*}\PYG{l+m+mi}{2} \PYG{o}{/} \PYG{p}{(}\PYG{l+m+mi}{2} \PYG{o}{*} \PYG{n}{beta}\PYG{o}{*}\PYG{o}{*}\PYG{l+m+mi}{2}\PYG{p}{)}\PYG{p}{)}\PYG{p}{,}
\PYG{g+gp}{... }         \PYG{n}{linewidth}\PYG{o}{=}\PYG{l+m+mi}{2}\PYG{p}{,} \PYG{n}{color}\PYG{o}{=}\PYG{l+s}{\PYGZsq{}}\PYG{l+s}{g}\PYG{l+s}{\PYGZsq{}}\PYG{p}{)}
\PYG{g+gp}{\PYGZgt{}\PYGZgt{}\PYGZgt{} }\PYG{n}{plt}\PYG{o}{.}\PYG{n}{show}\PYG{p}{(}\PYG{p}{)}
\end{Verbatim}

\end{fulllineitems}

\index{hypergeometric() (in module acsDynStatInTime)}

\begin{fulllineitems}
\phantomsection\label{acsDynStatInTime:acsDynStatInTime.hypergeometric}\pysiglinewithargsret{\code{acsDynStatInTime.}\bfcode{hypergeometric}}{\emph{ngood}, \emph{nbad}, \emph{nsample}, \emph{size=None}}{}
Draw samples from a Hypergeometric distribution.

Samples are drawn from a Hypergeometric distribution with specified
parameters, ngood (ways to make a good selection), nbad (ways to make
a bad selection), and nsample = number of items sampled, which is less
than or equal to the sum ngood + nbad.
\begin{description}
\item[{ngood}] \leavevmode{[}int or array\_like{]}
Number of ways to make a good selection.  Must be nonnegative.

\item[{nbad}] \leavevmode{[}int or array\_like{]}
Number of ways to make a bad selection.  Must be nonnegative.

\item[{nsample}] \leavevmode{[}int or array\_like{]}
Number of items sampled.  Must be at least 1 and at most
\code{ngood + nbad}.

\item[{size}] \leavevmode{[}int or tuple of int{]}
Output shape.  If the given shape is, e.g., \code{(m, n, k)}, then
\code{m * n * k} samples are drawn.

\end{description}
\begin{description}
\item[{samples}] \leavevmode{[}ndarray or scalar{]}
The values are all integers in  {[}0, n{]}.

\end{description}
\begin{description}
\item[{scipy.stats.distributions.hypergeom}] \leavevmode{[}probability density function,{]}
distribution or cumulative density function, etc.

\end{description}

The probability density for the Hypergeometric distribution is
\begin{gather}
\begin{split}P(x) = \frac{\binom{m}{n}\binom{N-m}{n-x}}{\binom{N}{n}},\end{split}\notag
\end{gather}
where \(0 \le x \le m\) and \(n+m-N \le x \le n\)

for P(x) the probability of x successes, n = ngood, m = nbad, and
N = number of samples.

Consider an urn with black and white marbles in it, ngood of them
black and nbad are white. If you draw nsample balls without
replacement, then the Hypergeometric distribution describes the
distribution of black balls in the drawn sample.

Note that this distribution is very similar to the Binomial
distribution, except that in this case, samples are drawn without
replacement, whereas in the Binomial case samples are drawn with
replacement (or the sample space is infinite). As the sample space
becomes large, this distribution approaches the Binomial.

Draw samples from the distribution:

\begin{Verbatim}[commandchars=\\\{\}]
\PYG{g+gp}{\PYGZgt{}\PYGZgt{}\PYGZgt{} }\PYG{n}{ngood}\PYG{p}{,} \PYG{n}{nbad}\PYG{p}{,} \PYG{n}{nsamp} \PYG{o}{=} \PYG{l+m+mi}{100}\PYG{p}{,} \PYG{l+m+mi}{2}\PYG{p}{,} \PYG{l+m+mi}{10}
\PYG{g+go}{\PYGZsh{} number of good, number of bad, and number of samples}
\PYG{g+gp}{\PYGZgt{}\PYGZgt{}\PYGZgt{} }\PYG{n}{s} \PYG{o}{=} \PYG{n}{np}\PYG{o}{.}\PYG{n}{random}\PYG{o}{.}\PYG{n}{hypergeometric}\PYG{p}{(}\PYG{n}{ngood}\PYG{p}{,} \PYG{n}{nbad}\PYG{p}{,} \PYG{n}{nsamp}\PYG{p}{,} \PYG{l+m+mi}{1000}\PYG{p}{)}
\PYG{g+gp}{\PYGZgt{}\PYGZgt{}\PYGZgt{} }\PYG{n}{hist}\PYG{p}{(}\PYG{n}{s}\PYG{p}{)}
\PYG{g+go}{\PYGZsh{}   note that it is very unlikely to grab both bad items}
\end{Verbatim}

Suppose you have an urn with 15 white and 15 black marbles.
If you pull 15 marbles at random, how likely is it that
12 or more of them are one color?

\begin{Verbatim}[commandchars=\\\{\}]
\PYG{g+gp}{\PYGZgt{}\PYGZgt{}\PYGZgt{} }\PYG{n}{s} \PYG{o}{=} \PYG{n}{np}\PYG{o}{.}\PYG{n}{random}\PYG{o}{.}\PYG{n}{hypergeometric}\PYG{p}{(}\PYG{l+m+mi}{15}\PYG{p}{,} \PYG{l+m+mi}{15}\PYG{p}{,} \PYG{l+m+mi}{15}\PYG{p}{,} \PYG{l+m+mi}{100000}\PYG{p}{)}
\PYG{g+gp}{\PYGZgt{}\PYGZgt{}\PYGZgt{} }\PYG{n+nb}{sum}\PYG{p}{(}\PYG{n}{s}\PYG{o}{\PYGZgt{}}\PYG{o}{=}\PYG{l+m+mi}{12}\PYG{p}{)}\PYG{o}{/}\PYG{l+m+mf}{100000.} \PYG{o}{+} \PYG{n+nb}{sum}\PYG{p}{(}\PYG{n}{s}\PYG{o}{\PYGZlt{}}\PYG{o}{=}\PYG{l+m+mi}{3}\PYG{p}{)}\PYG{o}{/}\PYG{l+m+mf}{100000.}
\PYG{g+go}{\PYGZsh{}   answer = 0.003 ... pretty unlikely!}
\end{Verbatim}

\end{fulllineitems}

\index{laplace() (in module acsDynStatInTime)}

\begin{fulllineitems}
\phantomsection\label{acsDynStatInTime:acsDynStatInTime.laplace}\pysiglinewithargsret{\code{acsDynStatInTime.}\bfcode{laplace}}{\emph{loc=0.0}, \emph{scale=1.0}, \emph{size=None}}{}
Draw samples from the Laplace or double exponential distribution with
specified location (or mean) and scale (decay).

The Laplace distribution is similar to the Gaussian/normal distribution,
but is sharper at the peak and has fatter tails. It represents the
difference between two independent, identically distributed exponential
random variables.
\begin{description}
\item[{loc}] \leavevmode{[}float{]}
The position, \(\mu\), of the distribution peak.

\item[{scale}] \leavevmode{[}float{]}
\(\lambda\), the exponential decay.

\end{description}

It has the probability density function
\begin{gather}
\begin{split}f(x; \mu, \lambda) = \frac{1}{2\lambda}
\exp\left(-\frac{|x - \mu|}{\lambda}\right).\end{split}\notag
\end{gather}
The first law of Laplace, from 1774, states that the frequency of an error
can be expressed as an exponential function of the absolute magnitude of
the error, which leads to the Laplace distribution. For many problems in
Economics and Health sciences, this distribution seems to model the data
better than the standard Gaussian distribution

Draw samples from the distribution

\begin{Verbatim}[commandchars=\\\{\}]
\PYG{g+gp}{\PYGZgt{}\PYGZgt{}\PYGZgt{} }\PYG{n}{loc}\PYG{p}{,} \PYG{n}{scale} \PYG{o}{=} \PYG{l+m+mf}{0.}\PYG{p}{,} \PYG{l+m+mf}{1.}
\PYG{g+gp}{\PYGZgt{}\PYGZgt{}\PYGZgt{} }\PYG{n}{s} \PYG{o}{=} \PYG{n}{np}\PYG{o}{.}\PYG{n}{random}\PYG{o}{.}\PYG{n}{laplace}\PYG{p}{(}\PYG{n}{loc}\PYG{p}{,} \PYG{n}{scale}\PYG{p}{,} \PYG{l+m+mi}{1000}\PYG{p}{)}
\end{Verbatim}

Display the histogram of the samples, along with
the probability density function:

\begin{Verbatim}[commandchars=\\\{\}]
\PYG{g+gp}{\PYGZgt{}\PYGZgt{}\PYGZgt{} }\PYG{k+kn}{import} \PYG{n+nn}{matplotlib.pyplot} \PYG{k+kn}{as} \PYG{n+nn}{plt}
\PYG{g+gp}{\PYGZgt{}\PYGZgt{}\PYGZgt{} }\PYG{n}{count}\PYG{p}{,} \PYG{n}{bins}\PYG{p}{,} \PYG{n}{ignored} \PYG{o}{=} \PYG{n}{plt}\PYG{o}{.}\PYG{n}{hist}\PYG{p}{(}\PYG{n}{s}\PYG{p}{,} \PYG{l+m+mi}{30}\PYG{p}{,} \PYG{n}{normed}\PYG{o}{=}\PYG{n+nb+bp}{True}\PYG{p}{)}
\PYG{g+gp}{\PYGZgt{}\PYGZgt{}\PYGZgt{} }\PYG{n}{x} \PYG{o}{=} \PYG{n}{np}\PYG{o}{.}\PYG{n}{arange}\PYG{p}{(}\PYG{o}{\PYGZhy{}}\PYG{l+m+mf}{8.}\PYG{p}{,} \PYG{l+m+mf}{8.}\PYG{p}{,} \PYG{o}{.}\PYG{l+m+mo}{01}\PYG{p}{)}
\PYG{g+gp}{\PYGZgt{}\PYGZgt{}\PYGZgt{} }\PYG{n}{pdf} \PYG{o}{=} \PYG{n}{np}\PYG{o}{.}\PYG{n}{exp}\PYG{p}{(}\PYG{o}{\PYGZhy{}}\PYG{n+nb}{abs}\PYG{p}{(}\PYG{n}{x}\PYG{o}{\PYGZhy{}}\PYG{n}{loc}\PYG{o}{/}\PYG{n}{scale}\PYG{p}{)}\PYG{p}{)}\PYG{o}{/}\PYG{p}{(}\PYG{l+m+mf}{2.}\PYG{o}{*}\PYG{n}{scale}\PYG{p}{)}
\PYG{g+gp}{\PYGZgt{}\PYGZgt{}\PYGZgt{} }\PYG{n}{plt}\PYG{o}{.}\PYG{n}{plot}\PYG{p}{(}\PYG{n}{x}\PYG{p}{,} \PYG{n}{pdf}\PYG{p}{)}
\end{Verbatim}

Plot Gaussian for comparison:

\begin{Verbatim}[commandchars=\\\{\}]
\PYG{g+gp}{\PYGZgt{}\PYGZgt{}\PYGZgt{} }\PYG{n}{g} \PYG{o}{=} \PYG{p}{(}\PYG{l+m+mi}{1}\PYG{o}{/}\PYG{p}{(}\PYG{n}{scale} \PYG{o}{*} \PYG{n}{np}\PYG{o}{.}\PYG{n}{sqrt}\PYG{p}{(}\PYG{l+m+mi}{2} \PYG{o}{*} \PYG{n}{np}\PYG{o}{.}\PYG{n}{pi}\PYG{p}{)}\PYG{p}{)} \PYG{o}{*} 
\PYG{g+gp}{... }     \PYG{n}{np}\PYG{o}{.}\PYG{n}{exp}\PYG{p}{(} \PYG{o}{\PYGZhy{}} \PYG{p}{(}\PYG{n}{x} \PYG{o}{\PYGZhy{}} \PYG{n}{loc}\PYG{p}{)}\PYG{o}{*}\PYG{o}{*}\PYG{l+m+mi}{2} \PYG{o}{/} \PYG{p}{(}\PYG{l+m+mi}{2} \PYG{o}{*} \PYG{n}{scale}\PYG{o}{*}\PYG{o}{*}\PYG{l+m+mi}{2}\PYG{p}{)} \PYG{p}{)}\PYG{p}{)}
\PYG{g+gp}{\PYGZgt{}\PYGZgt{}\PYGZgt{} }\PYG{n}{plt}\PYG{o}{.}\PYG{n}{plot}\PYG{p}{(}\PYG{n}{x}\PYG{p}{,}\PYG{n}{g}\PYG{p}{)}
\end{Verbatim}

\end{fulllineitems}

\index{logistic() (in module acsDynStatInTime)}

\begin{fulllineitems}
\phantomsection\label{acsDynStatInTime:acsDynStatInTime.logistic}\pysiglinewithargsret{\code{acsDynStatInTime.}\bfcode{logistic}}{\emph{loc=0.0}, \emph{scale=1.0}, \emph{size=None}}{}
Draw samples from a Logistic distribution.

Samples are drawn from a Logistic distribution with specified
parameters, loc (location or mean, also median), and scale (\textgreater{}0).

loc : float

scale : float \textgreater{} 0.
\begin{description}
\item[{size}] \leavevmode{[}\{tuple, int\}{]}
Output shape.  If the given shape is, e.g., \code{(m, n, k)}, then
\code{m * n * k} samples are drawn.

\end{description}
\begin{description}
\item[{samples}] \leavevmode{[}\{ndarray, scalar\}{]}
where the values are all integers in  {[}0, n{]}.

\end{description}
\begin{description}
\item[{scipy.stats.distributions.logistic}] \leavevmode{[}probability density function,{]}
distribution or cumulative density function, etc.

\end{description}

The probability density for the Logistic distribution is
\begin{gather}
\begin{split}P(x) = P(x) = \frac{e^{-(x-\mu)/s}}{s(1+e^{-(x-\mu)/s})^2},\end{split}\notag
\end{gather}
where \(\mu\) = location and \(s\) = scale.

The Logistic distribution is used in Extreme Value problems where it
can act as a mixture of Gumbel distributions, in Epidemiology, and by
the World Chess Federation (FIDE) where it is used in the Elo ranking
system, assuming the performance of each player is a logistically
distributed random variable.

Draw samples from the distribution:

\begin{Verbatim}[commandchars=\\\{\}]
\PYG{g+gp}{\PYGZgt{}\PYGZgt{}\PYGZgt{} }\PYG{n}{loc}\PYG{p}{,} \PYG{n}{scale} \PYG{o}{=} \PYG{l+m+mi}{10}\PYG{p}{,} \PYG{l+m+mi}{1}
\PYG{g+gp}{\PYGZgt{}\PYGZgt{}\PYGZgt{} }\PYG{n}{s} \PYG{o}{=} \PYG{n}{np}\PYG{o}{.}\PYG{n}{random}\PYG{o}{.}\PYG{n}{logistic}\PYG{p}{(}\PYG{n}{loc}\PYG{p}{,} \PYG{n}{scale}\PYG{p}{,} \PYG{l+m+mi}{10000}\PYG{p}{)}
\PYG{g+gp}{\PYGZgt{}\PYGZgt{}\PYGZgt{} }\PYG{n}{count}\PYG{p}{,} \PYG{n}{bins}\PYG{p}{,} \PYG{n}{ignored} \PYG{o}{=} \PYG{n}{plt}\PYG{o}{.}\PYG{n}{hist}\PYG{p}{(}\PYG{n}{s}\PYG{p}{,} \PYG{n}{bins}\PYG{o}{=}\PYG{l+m+mi}{50}\PYG{p}{)}
\end{Verbatim}

\#   plot against distribution

\begin{Verbatim}[commandchars=\\\{\}]
\PYG{g+gp}{\PYGZgt{}\PYGZgt{}\PYGZgt{} }\PYG{k}{def} \PYG{n+nf}{logist}\PYG{p}{(}\PYG{n}{x}\PYG{p}{,} \PYG{n}{loc}\PYG{p}{,} \PYG{n}{scale}\PYG{p}{)}\PYG{p}{:}
\PYG{g+gp}{... }    \PYG{k}{return} \PYG{n}{exp}\PYG{p}{(}\PYG{p}{(}\PYG{n}{loc}\PYG{o}{\PYGZhy{}}\PYG{n}{x}\PYG{p}{)}\PYG{o}{/}\PYG{n}{scale}\PYG{p}{)}\PYG{o}{/}\PYG{p}{(}\PYG{n}{scale}\PYG{o}{*}\PYG{p}{(}\PYG{l+m+mi}{1}\PYG{o}{+}\PYG{n}{exp}\PYG{p}{(}\PYG{p}{(}\PYG{n}{loc}\PYG{o}{\PYGZhy{}}\PYG{n}{x}\PYG{p}{)}\PYG{o}{/}\PYG{n}{scale}\PYG{p}{)}\PYG{p}{)}\PYG{o}{*}\PYG{o}{*}\PYG{l+m+mi}{2}\PYG{p}{)}
\PYG{g+gp}{\PYGZgt{}\PYGZgt{}\PYGZgt{} }\PYG{n}{plt}\PYG{o}{.}\PYG{n}{plot}\PYG{p}{(}\PYG{n}{bins}\PYG{p}{,} \PYG{n}{logist}\PYG{p}{(}\PYG{n}{bins}\PYG{p}{,} \PYG{n}{loc}\PYG{p}{,} \PYG{n}{scale}\PYG{p}{)}\PYG{o}{*}\PYG{n}{count}\PYG{o}{.}\PYG{n}{max}\PYG{p}{(}\PYG{p}{)}\PYG{o}{/}\PYGZbs{}
\PYG{g+gp}{... }\PYG{n}{logist}\PYG{p}{(}\PYG{n}{bins}\PYG{p}{,} \PYG{n}{loc}\PYG{p}{,} \PYG{n}{scale}\PYG{p}{)}\PYG{o}{.}\PYG{n}{max}\PYG{p}{(}\PYG{p}{)}\PYG{p}{)}
\PYG{g+gp}{\PYGZgt{}\PYGZgt{}\PYGZgt{} }\PYG{n}{plt}\PYG{o}{.}\PYG{n}{show}\PYG{p}{(}\PYG{p}{)}
\end{Verbatim}

\end{fulllineitems}

\index{lognormal() (in module acsDynStatInTime)}

\begin{fulllineitems}
\phantomsection\label{acsDynStatInTime:acsDynStatInTime.lognormal}\pysiglinewithargsret{\code{acsDynStatInTime.}\bfcode{lognormal}}{\emph{mean=0.0}, \emph{sigma=1.0}, \emph{size=None}}{}
Return samples drawn from a log-normal distribution.

Draw samples from a log-normal distribution with specified mean,
standard deviation, and array shape.  Note that the mean and standard
deviation are not the values for the distribution itself, but of the
underlying normal distribution it is derived from.
\begin{description}
\item[{mean}] \leavevmode{[}float{]}
Mean value of the underlying normal distribution

\item[{sigma}] \leavevmode{[}float, \textgreater{} 0.{]}
Standard deviation of the underlying normal distribution

\item[{size}] \leavevmode{[}tuple of ints{]}
Output shape.  If the given shape is, e.g., \code{(m, n, k)}, then
\code{m * n * k} samples are drawn.

\end{description}
\begin{description}
\item[{samples}] \leavevmode{[}ndarray or float{]}
The desired samples. An array of the same shape as \emph{size} if given,
if \emph{size} is None a float is returned.

\end{description}
\begin{description}
\item[{scipy.stats.lognorm}] \leavevmode{[}probability density function, distribution,{]}
cumulative density function, etc.

\end{description}

A variable \emph{x} has a log-normal distribution if \emph{log(x)} is normally
distributed.  The probability density function for the log-normal
distribution is:
\begin{gather}
\begin{split}p(x) = \frac{1}{\sigma x \sqrt{2\pi}}
e^{(-\frac{(ln(x)-\mu)^2}{2\sigma^2})}\end{split}\notag
\end{gather}
where \(\mu\) is the mean and \(\sigma\) is the standard
deviation of the normally distributed logarithm of the variable.
A log-normal distribution results if a random variable is the \emph{product}
of a large number of independent, identically-distributed variables in
the same way that a normal distribution results if the variable is the
\emph{sum} of a large number of independent, identically-distributed
variables.

Limpert, E., Stahel, W. A., and Abbt, M., ``Log-normal Distributions
across the Sciences: Keys and Clues,'' \emph{BioScience}, Vol. 51, No. 5,
May, 2001.  \href{http://stat.ethz.ch/~stahel/lognormal/bioscience.pdf}{http://stat.ethz.ch/\textasciitilde{}stahel/lognormal/bioscience.pdf}

Reiss, R.D. and Thomas, M., \emph{Statistical Analysis of Extreme Values},
Basel: Birkhauser Verlag, 2001, pp. 31-32.

Draw samples from the distribution:

\begin{Verbatim}[commandchars=\\\{\}]
\PYG{g+gp}{\PYGZgt{}\PYGZgt{}\PYGZgt{} }\PYG{n}{mu}\PYG{p}{,} \PYG{n}{sigma} \PYG{o}{=} \PYG{l+m+mf}{3.}\PYG{p}{,} \PYG{l+m+mf}{1.} \PYG{c}{\PYGZsh{} mean and standard deviation}
\PYG{g+gp}{\PYGZgt{}\PYGZgt{}\PYGZgt{} }\PYG{n}{s} \PYG{o}{=} \PYG{n}{np}\PYG{o}{.}\PYG{n}{random}\PYG{o}{.}\PYG{n}{lognormal}\PYG{p}{(}\PYG{n}{mu}\PYG{p}{,} \PYG{n}{sigma}\PYG{p}{,} \PYG{l+m+mi}{1000}\PYG{p}{)}
\end{Verbatim}

Display the histogram of the samples, along with
the probability density function:

\begin{Verbatim}[commandchars=\\\{\}]
\PYG{g+gp}{\PYGZgt{}\PYGZgt{}\PYGZgt{} }\PYG{k+kn}{import} \PYG{n+nn}{matplotlib.pyplot} \PYG{k+kn}{as} \PYG{n+nn}{plt}
\PYG{g+gp}{\PYGZgt{}\PYGZgt{}\PYGZgt{} }\PYG{n}{count}\PYG{p}{,} \PYG{n}{bins}\PYG{p}{,} \PYG{n}{ignored} \PYG{o}{=} \PYG{n}{plt}\PYG{o}{.}\PYG{n}{hist}\PYG{p}{(}\PYG{n}{s}\PYG{p}{,} \PYG{l+m+mi}{100}\PYG{p}{,} \PYG{n}{normed}\PYG{o}{=}\PYG{n+nb+bp}{True}\PYG{p}{,} \PYG{n}{align}\PYG{o}{=}\PYG{l+s}{\PYGZsq{}}\PYG{l+s}{mid}\PYG{l+s}{\PYGZsq{}}\PYG{p}{)}
\end{Verbatim}

\begin{Verbatim}[commandchars=\\\{\}]
\PYG{g+gp}{\PYGZgt{}\PYGZgt{}\PYGZgt{} }\PYG{n}{x} \PYG{o}{=} \PYG{n}{np}\PYG{o}{.}\PYG{n}{linspace}\PYG{p}{(}\PYG{n+nb}{min}\PYG{p}{(}\PYG{n}{bins}\PYG{p}{)}\PYG{p}{,} \PYG{n+nb}{max}\PYG{p}{(}\PYG{n}{bins}\PYG{p}{)}\PYG{p}{,} \PYG{l+m+mi}{10000}\PYG{p}{)}
\PYG{g+gp}{\PYGZgt{}\PYGZgt{}\PYGZgt{} }\PYG{n}{pdf} \PYG{o}{=} \PYG{p}{(}\PYG{n}{np}\PYG{o}{.}\PYG{n}{exp}\PYG{p}{(}\PYG{o}{\PYGZhy{}}\PYG{p}{(}\PYG{n}{np}\PYG{o}{.}\PYG{n}{log}\PYG{p}{(}\PYG{n}{x}\PYG{p}{)} \PYG{o}{\PYGZhy{}} \PYG{n}{mu}\PYG{p}{)}\PYG{o}{*}\PYG{o}{*}\PYG{l+m+mi}{2} \PYG{o}{/} \PYG{p}{(}\PYG{l+m+mi}{2} \PYG{o}{*} \PYG{n}{sigma}\PYG{o}{*}\PYG{o}{*}\PYG{l+m+mi}{2}\PYG{p}{)}\PYG{p}{)}
\PYG{g+gp}{... }       \PYG{o}{/} \PYG{p}{(}\PYG{n}{x} \PYG{o}{*} \PYG{n}{sigma} \PYG{o}{*} \PYG{n}{np}\PYG{o}{.}\PYG{n}{sqrt}\PYG{p}{(}\PYG{l+m+mi}{2} \PYG{o}{*} \PYG{n}{np}\PYG{o}{.}\PYG{n}{pi}\PYG{p}{)}\PYG{p}{)}\PYG{p}{)}
\end{Verbatim}

\begin{Verbatim}[commandchars=\\\{\}]
\PYG{g+gp}{\PYGZgt{}\PYGZgt{}\PYGZgt{} }\PYG{n}{plt}\PYG{o}{.}\PYG{n}{plot}\PYG{p}{(}\PYG{n}{x}\PYG{p}{,} \PYG{n}{pdf}\PYG{p}{,} \PYG{n}{linewidth}\PYG{o}{=}\PYG{l+m+mi}{2}\PYG{p}{,} \PYG{n}{color}\PYG{o}{=}\PYG{l+s}{\PYGZsq{}}\PYG{l+s}{r}\PYG{l+s}{\PYGZsq{}}\PYG{p}{)}
\PYG{g+gp}{\PYGZgt{}\PYGZgt{}\PYGZgt{} }\PYG{n}{plt}\PYG{o}{.}\PYG{n}{axis}\PYG{p}{(}\PYG{l+s}{\PYGZsq{}}\PYG{l+s}{tight}\PYG{l+s}{\PYGZsq{}}\PYG{p}{)}
\PYG{g+gp}{\PYGZgt{}\PYGZgt{}\PYGZgt{} }\PYG{n}{plt}\PYG{o}{.}\PYG{n}{show}\PYG{p}{(}\PYG{p}{)}
\end{Verbatim}

Demonstrate that taking the products of random samples from a uniform
distribution can be fit well by a log-normal probability density function.

\begin{Verbatim}[commandchars=\\\{\}]
\PYG{g+gp}{\PYGZgt{}\PYGZgt{}\PYGZgt{} }\PYG{c}{\PYGZsh{} Generate a thousand samples: each is the product of 100 random}
\PYG{g+gp}{\PYGZgt{}\PYGZgt{}\PYGZgt{} }\PYG{c}{\PYGZsh{} values, drawn from a normal distribution.}
\PYG{g+gp}{\PYGZgt{}\PYGZgt{}\PYGZgt{} }\PYG{n}{b} \PYG{o}{=} \PYG{p}{[}\PYG{p}{]}
\PYG{g+gp}{\PYGZgt{}\PYGZgt{}\PYGZgt{} }\PYG{k}{for} \PYG{n}{i} \PYG{o+ow}{in} \PYG{n+nb}{range}\PYG{p}{(}\PYG{l+m+mi}{1000}\PYG{p}{)}\PYG{p}{:}
\PYG{g+gp}{... }   \PYG{n}{a} \PYG{o}{=} \PYG{l+m+mf}{10.} \PYG{o}{+} \PYG{n}{np}\PYG{o}{.}\PYG{n}{random}\PYG{o}{.}\PYG{n}{random}\PYG{p}{(}\PYG{l+m+mi}{100}\PYG{p}{)}
\PYG{g+gp}{... }   \PYG{n}{b}\PYG{o}{.}\PYG{n}{append}\PYG{p}{(}\PYG{n}{np}\PYG{o}{.}\PYG{n}{product}\PYG{p}{(}\PYG{n}{a}\PYG{p}{)}\PYG{p}{)}
\end{Verbatim}

\begin{Verbatim}[commandchars=\\\{\}]
\PYG{g+gp}{\PYGZgt{}\PYGZgt{}\PYGZgt{} }\PYG{n}{b} \PYG{o}{=} \PYG{n}{np}\PYG{o}{.}\PYG{n}{array}\PYG{p}{(}\PYG{n}{b}\PYG{p}{)} \PYG{o}{/} \PYG{n}{np}\PYG{o}{.}\PYG{n}{min}\PYG{p}{(}\PYG{n}{b}\PYG{p}{)} \PYG{c}{\PYGZsh{} scale values to be positive}
\PYG{g+gp}{\PYGZgt{}\PYGZgt{}\PYGZgt{} }\PYG{n}{count}\PYG{p}{,} \PYG{n}{bins}\PYG{p}{,} \PYG{n}{ignored} \PYG{o}{=} \PYG{n}{plt}\PYG{o}{.}\PYG{n}{hist}\PYG{p}{(}\PYG{n}{b}\PYG{p}{,} \PYG{l+m+mi}{100}\PYG{p}{,} \PYG{n}{normed}\PYG{o}{=}\PYG{n+nb+bp}{True}\PYG{p}{,} \PYG{n}{align}\PYG{o}{=}\PYG{l+s}{\PYGZsq{}}\PYG{l+s}{center}\PYG{l+s}{\PYGZsq{}}\PYG{p}{)}
\PYG{g+gp}{\PYGZgt{}\PYGZgt{}\PYGZgt{} }\PYG{n}{sigma} \PYG{o}{=} \PYG{n}{np}\PYG{o}{.}\PYG{n}{std}\PYG{p}{(}\PYG{n}{np}\PYG{o}{.}\PYG{n}{log}\PYG{p}{(}\PYG{n}{b}\PYG{p}{)}\PYG{p}{)}
\PYG{g+gp}{\PYGZgt{}\PYGZgt{}\PYGZgt{} }\PYG{n}{mu} \PYG{o}{=} \PYG{n}{np}\PYG{o}{.}\PYG{n}{mean}\PYG{p}{(}\PYG{n}{np}\PYG{o}{.}\PYG{n}{log}\PYG{p}{(}\PYG{n}{b}\PYG{p}{)}\PYG{p}{)}
\end{Verbatim}

\begin{Verbatim}[commandchars=\\\{\}]
\PYG{g+gp}{\PYGZgt{}\PYGZgt{}\PYGZgt{} }\PYG{n}{x} \PYG{o}{=} \PYG{n}{np}\PYG{o}{.}\PYG{n}{linspace}\PYG{p}{(}\PYG{n+nb}{min}\PYG{p}{(}\PYG{n}{bins}\PYG{p}{)}\PYG{p}{,} \PYG{n+nb}{max}\PYG{p}{(}\PYG{n}{bins}\PYG{p}{)}\PYG{p}{,} \PYG{l+m+mi}{10000}\PYG{p}{)}
\PYG{g+gp}{\PYGZgt{}\PYGZgt{}\PYGZgt{} }\PYG{n}{pdf} \PYG{o}{=} \PYG{p}{(}\PYG{n}{np}\PYG{o}{.}\PYG{n}{exp}\PYG{p}{(}\PYG{o}{\PYGZhy{}}\PYG{p}{(}\PYG{n}{np}\PYG{o}{.}\PYG{n}{log}\PYG{p}{(}\PYG{n}{x}\PYG{p}{)} \PYG{o}{\PYGZhy{}} \PYG{n}{mu}\PYG{p}{)}\PYG{o}{*}\PYG{o}{*}\PYG{l+m+mi}{2} \PYG{o}{/} \PYG{p}{(}\PYG{l+m+mi}{2} \PYG{o}{*} \PYG{n}{sigma}\PYG{o}{*}\PYG{o}{*}\PYG{l+m+mi}{2}\PYG{p}{)}\PYG{p}{)}
\PYG{g+gp}{... }       \PYG{o}{/} \PYG{p}{(}\PYG{n}{x} \PYG{o}{*} \PYG{n}{sigma} \PYG{o}{*} \PYG{n}{np}\PYG{o}{.}\PYG{n}{sqrt}\PYG{p}{(}\PYG{l+m+mi}{2} \PYG{o}{*} \PYG{n}{np}\PYG{o}{.}\PYG{n}{pi}\PYG{p}{)}\PYG{p}{)}\PYG{p}{)}
\end{Verbatim}

\begin{Verbatim}[commandchars=\\\{\}]
\PYG{g+gp}{\PYGZgt{}\PYGZgt{}\PYGZgt{} }\PYG{n}{plt}\PYG{o}{.}\PYG{n}{plot}\PYG{p}{(}\PYG{n}{x}\PYG{p}{,} \PYG{n}{pdf}\PYG{p}{,} \PYG{n}{color}\PYG{o}{=}\PYG{l+s}{\PYGZsq{}}\PYG{l+s}{r}\PYG{l+s}{\PYGZsq{}}\PYG{p}{,} \PYG{n}{linewidth}\PYG{o}{=}\PYG{l+m+mi}{2}\PYG{p}{)}
\PYG{g+gp}{\PYGZgt{}\PYGZgt{}\PYGZgt{} }\PYG{n}{plt}\PYG{o}{.}\PYG{n}{show}\PYG{p}{(}\PYG{p}{)}
\end{Verbatim}

\end{fulllineitems}

\index{logseries() (in module acsDynStatInTime)}

\begin{fulllineitems}
\phantomsection\label{acsDynStatInTime:acsDynStatInTime.logseries}\pysiglinewithargsret{\code{acsDynStatInTime.}\bfcode{logseries}}{\emph{p}, \emph{size=None}}{}
Draw samples from a Logarithmic Series distribution.

Samples are drawn from a Log Series distribution with specified
parameter, p (probability, 0 \textless{} p \textless{} 1).

loc : float

scale : float \textgreater{} 0.
\begin{description}
\item[{size}] \leavevmode{[}\{tuple, int\}{]}
Output shape.  If the given shape is, e.g., \code{(m, n, k)}, then
\code{m * n * k} samples are drawn.

\end{description}
\begin{description}
\item[{samples}] \leavevmode{[}\{ndarray, scalar\}{]}
where the values are all integers in  {[}0, n{]}.

\end{description}
\begin{description}
\item[{scipy.stats.distributions.logser}] \leavevmode{[}probability density function,{]}
distribution or cumulative density function, etc.

\end{description}

The probability density for the Log Series distribution is
\begin{gather}
\begin{split}P(k) = \frac{-p^k}{k \ln(1-p)},\end{split}\notag
\end{gather}
where p = probability.

The Log Series distribution is frequently used to represent species
richness and occurrence, first proposed by Fisher, Corbet, and
Williams in 1943 {[}2{]}.  It may also be used to model the numbers of
occupants seen in cars {[}3{]}.

Draw samples from the distribution:

\begin{Verbatim}[commandchars=\\\{\}]
\PYG{g+gp}{\PYGZgt{}\PYGZgt{}\PYGZgt{} }\PYG{n}{a} \PYG{o}{=} \PYG{o}{.}\PYG{l+m+mi}{6}
\PYG{g+gp}{\PYGZgt{}\PYGZgt{}\PYGZgt{} }\PYG{n}{s} \PYG{o}{=} \PYG{n}{np}\PYG{o}{.}\PYG{n}{random}\PYG{o}{.}\PYG{n}{logseries}\PYG{p}{(}\PYG{n}{a}\PYG{p}{,} \PYG{l+m+mi}{10000}\PYG{p}{)}
\PYG{g+gp}{\PYGZgt{}\PYGZgt{}\PYGZgt{} }\PYG{n}{count}\PYG{p}{,} \PYG{n}{bins}\PYG{p}{,} \PYG{n}{ignored} \PYG{o}{=} \PYG{n}{plt}\PYG{o}{.}\PYG{n}{hist}\PYG{p}{(}\PYG{n}{s}\PYG{p}{)}
\end{Verbatim}

\#   plot against distribution

\begin{Verbatim}[commandchars=\\\{\}]
\PYG{g+gp}{\PYGZgt{}\PYGZgt{}\PYGZgt{} }\PYG{k}{def} \PYG{n+nf}{logseries}\PYG{p}{(}\PYG{n}{k}\PYG{p}{,} \PYG{n}{p}\PYG{p}{)}\PYG{p}{:}
\PYG{g+gp}{... }    \PYG{k}{return} \PYG{o}{\PYGZhy{}}\PYG{n}{p}\PYG{o}{*}\PYG{o}{*}\PYG{n}{k}\PYG{o}{/}\PYG{p}{(}\PYG{n}{k}\PYG{o}{*}\PYG{n}{log}\PYG{p}{(}\PYG{l+m+mi}{1}\PYG{o}{\PYGZhy{}}\PYG{n}{p}\PYG{p}{)}\PYG{p}{)}
\PYG{g+gp}{\PYGZgt{}\PYGZgt{}\PYGZgt{} }\PYG{n}{plt}\PYG{o}{.}\PYG{n}{plot}\PYG{p}{(}\PYG{n}{bins}\PYG{p}{,} \PYG{n}{logseries}\PYG{p}{(}\PYG{n}{bins}\PYG{p}{,} \PYG{n}{a}\PYG{p}{)}\PYG{o}{*}\PYG{n}{count}\PYG{o}{.}\PYG{n}{max}\PYG{p}{(}\PYG{p}{)}\PYG{o}{/}
\PYG{g+go}{             logseries(bins, a).max(), \PYGZsq{}r\PYGZsq{})}
\PYG{g+gp}{\PYGZgt{}\PYGZgt{}\PYGZgt{} }\PYG{n}{plt}\PYG{o}{.}\PYG{n}{show}\PYG{p}{(}\PYG{p}{)}
\end{Verbatim}

\end{fulllineitems}

\index{multinomial() (in module acsDynStatInTime)}

\begin{fulllineitems}
\phantomsection\label{acsDynStatInTime:acsDynStatInTime.multinomial}\pysiglinewithargsret{\code{acsDynStatInTime.}\bfcode{multinomial}}{\emph{n}, \emph{pvals}, \emph{size=None}}{}
Draw samples from a multinomial distribution.

The multinomial distribution is a multivariate generalisation of the
binomial distribution.  Take an experiment with one of \code{p}
possible outcomes.  An example of such an experiment is throwing a dice,
where the outcome can be 1 through 6.  Each sample drawn from the
distribution represents \emph{n} such experiments.  Its values,
\code{X\_i = {[}X\_0, X\_1, ..., X\_p{]}}, represent the number of times the outcome
was \code{i}.
\begin{description}
\item[{n}] \leavevmode{[}int{]}
Number of experiments.

\item[{pvals}] \leavevmode{[}sequence of floats, length p{]}
Probabilities of each of the \code{p} different outcomes.  These
should sum to 1 (however, the last element is always assumed to
account for the remaining probability, as long as
\code{sum(pvals{[}:-1{]}) \textless{}= 1)}.

\item[{size}] \leavevmode{[}tuple of ints{]}
Given a \emph{size} of \code{(M, N, K)}, then \code{M*N*K} samples are drawn,
and the output shape becomes \code{(M, N, K, p)}, since each sample
has shape \code{(p,)}.

\end{description}

Throw a dice 20 times:

\begin{Verbatim}[commandchars=\\\{\}]
\PYG{g+gp}{\PYGZgt{}\PYGZgt{}\PYGZgt{} }\PYG{n}{np}\PYG{o}{.}\PYG{n}{random}\PYG{o}{.}\PYG{n}{multinomial}\PYG{p}{(}\PYG{l+m+mi}{20}\PYG{p}{,} \PYG{p}{[}\PYG{l+m+mi}{1}\PYG{o}{/}\PYG{l+m+mf}{6.}\PYG{p}{]}\PYG{o}{*}\PYG{l+m+mi}{6}\PYG{p}{,} \PYG{n}{size}\PYG{o}{=}\PYG{l+m+mi}{1}\PYG{p}{)}
\PYG{g+go}{array([[4, 1, 7, 5, 2, 1]])}
\end{Verbatim}

It landed 4 times on 1, once on 2, etc.

Now, throw the dice 20 times, and 20 times again:

\begin{Verbatim}[commandchars=\\\{\}]
\PYG{g+gp}{\PYGZgt{}\PYGZgt{}\PYGZgt{} }\PYG{n}{np}\PYG{o}{.}\PYG{n}{random}\PYG{o}{.}\PYG{n}{multinomial}\PYG{p}{(}\PYG{l+m+mi}{20}\PYG{p}{,} \PYG{p}{[}\PYG{l+m+mi}{1}\PYG{o}{/}\PYG{l+m+mf}{6.}\PYG{p}{]}\PYG{o}{*}\PYG{l+m+mi}{6}\PYG{p}{,} \PYG{n}{size}\PYG{o}{=}\PYG{l+m+mi}{2}\PYG{p}{)}
\PYG{g+go}{array([[3, 4, 3, 3, 4, 3],}
\PYG{g+go}{       [2, 4, 3, 4, 0, 7]])}
\end{Verbatim}

For the first run, we threw 3 times 1, 4 times 2, etc.  For the second,
we threw 2 times 1, 4 times 2, etc.

A loaded dice is more likely to land on number 6:

\begin{Verbatim}[commandchars=\\\{\}]
\PYG{g+gp}{\PYGZgt{}\PYGZgt{}\PYGZgt{} }\PYG{n}{np}\PYG{o}{.}\PYG{n}{random}\PYG{o}{.}\PYG{n}{multinomial}\PYG{p}{(}\PYG{l+m+mi}{100}\PYG{p}{,} \PYG{p}{[}\PYG{l+m+mi}{1}\PYG{o}{/}\PYG{l+m+mf}{7.}\PYG{p}{]}\PYG{o}{*}\PYG{l+m+mi}{5}\PYG{p}{)}
\PYG{g+go}{array([13, 16, 13, 16, 42])}
\end{Verbatim}

\end{fulllineitems}

\index{multivariate\_normal() (in module acsDynStatInTime)}

\begin{fulllineitems}
\phantomsection\label{acsDynStatInTime:acsDynStatInTime.multivariate_normal}\pysiglinewithargsret{\code{acsDynStatInTime.}\bfcode{multivariate\_normal}}{\emph{mean}, \emph{cov}\optional{, \emph{size}}}{}
Draw random samples from a multivariate normal distribution.

The multivariate normal, multinormal or Gaussian distribution is a
generalization of the one-dimensional normal distribution to higher
dimensions.  Such a distribution is specified by its mean and
covariance matrix.  These parameters are analogous to the mean
(average or ``center'') and variance (standard deviation, or ``width,''
squared) of the one-dimensional normal distribution.
\begin{description}
\item[{mean}] \leavevmode{[}1-D array\_like, of length N{]}
Mean of the N-dimensional distribution.

\item[{cov}] \leavevmode{[}2-D array\_like, of shape (N, N){]}
Covariance matrix of the distribution.  Must be symmetric and
positive semi-definite for ``physically meaningful'' results.

\item[{size}] \leavevmode{[}int or tuple of ints, optional{]}
Given a shape of, for example, \code{(m,n,k)}, \code{m*n*k} samples are
generated, and packed in an \emph{m}-by-\emph{n}-by-\emph{k} arrangement.  Because
each sample is \emph{N}-dimensional, the output shape is \code{(m,n,k,N)}.
If no shape is specified, a single (\emph{N}-D) sample is returned.

\end{description}
\begin{description}
\item[{out}] \leavevmode{[}ndarray{]}
The drawn samples, of shape \emph{size}, if that was provided.  If not,
the shape is \code{(N,)}.

In other words, each entry \code{out{[}i,j,...,:{]}} is an N-dimensional
value drawn from the distribution.

\end{description}

The mean is a coordinate in N-dimensional space, which represents the
location where samples are most likely to be generated.  This is
analogous to the peak of the bell curve for the one-dimensional or
univariate normal distribution.

Covariance indicates the level to which two variables vary together.
From the multivariate normal distribution, we draw N-dimensional
samples, \(X = [x_1, x_2, ... x_N]\).  The covariance matrix
element \(C_{ij}\) is the covariance of \(x_i\) and \(x_j\).
The element \(C_{ii}\) is the variance of \(x_i\) (i.e. its
``spread'').

Instead of specifying the full covariance matrix, popular
approximations include:
\begin{itemize}
\item {} 
Spherical covariance (\emph{cov} is a multiple of the identity matrix)

\item {} 
Diagonal covariance (\emph{cov} has non-negative elements, and only on
the diagonal)

\end{itemize}

This geometrical property can be seen in two dimensions by plotting
generated data-points:

\begin{Verbatim}[commandchars=\\\{\}]
\PYG{g+gp}{\PYGZgt{}\PYGZgt{}\PYGZgt{} }\PYG{n}{mean} \PYG{o}{=} \PYG{p}{[}\PYG{l+m+mi}{0}\PYG{p}{,}\PYG{l+m+mi}{0}\PYG{p}{]}
\PYG{g+gp}{\PYGZgt{}\PYGZgt{}\PYGZgt{} }\PYG{n}{cov} \PYG{o}{=} \PYG{p}{[}\PYG{p}{[}\PYG{l+m+mi}{1}\PYG{p}{,}\PYG{l+m+mi}{0}\PYG{p}{]}\PYG{p}{,}\PYG{p}{[}\PYG{l+m+mi}{0}\PYG{p}{,}\PYG{l+m+mi}{100}\PYG{p}{]}\PYG{p}{]} \PYG{c}{\PYGZsh{} diagonal covariance, points lie on x or y\PYGZhy{}axis}
\end{Verbatim}

\begin{Verbatim}[commandchars=\\\{\}]
\PYG{g+gp}{\PYGZgt{}\PYGZgt{}\PYGZgt{} }\PYG{k+kn}{import} \PYG{n+nn}{matplotlib.pyplot} \PYG{k+kn}{as} \PYG{n+nn}{plt}
\PYG{g+gp}{\PYGZgt{}\PYGZgt{}\PYGZgt{} }\PYG{n}{x}\PYG{p}{,}\PYG{n}{y} \PYG{o}{=} \PYG{n}{np}\PYG{o}{.}\PYG{n}{random}\PYG{o}{.}\PYG{n}{multivariate\PYGZus{}normal}\PYG{p}{(}\PYG{n}{mean}\PYG{p}{,}\PYG{n}{cov}\PYG{p}{,}\PYG{l+m+mi}{5000}\PYG{p}{)}\PYG{o}{.}\PYG{n}{T}
\PYG{g+gp}{\PYGZgt{}\PYGZgt{}\PYGZgt{} }\PYG{n}{plt}\PYG{o}{.}\PYG{n}{plot}\PYG{p}{(}\PYG{n}{x}\PYG{p}{,}\PYG{n}{y}\PYG{p}{,}\PYG{l+s}{\PYGZsq{}}\PYG{l+s}{x}\PYG{l+s}{\PYGZsq{}}\PYG{p}{)}\PYG{p}{;} \PYG{n}{plt}\PYG{o}{.}\PYG{n}{axis}\PYG{p}{(}\PYG{l+s}{\PYGZsq{}}\PYG{l+s}{equal}\PYG{l+s}{\PYGZsq{}}\PYG{p}{)}\PYG{p}{;} \PYG{n}{plt}\PYG{o}{.}\PYG{n}{show}\PYG{p}{(}\PYG{p}{)}
\end{Verbatim}

Note that the covariance matrix must be non-negative definite.

Papoulis, A., \emph{Probability, Random Variables, and Stochastic Processes},
3rd ed., New York: McGraw-Hill, 1991.

Duda, R. O., Hart, P. E., and Stork, D. G., \emph{Pattern Classification},
2nd ed., New York: Wiley, 2001.

\begin{Verbatim}[commandchars=\\\{\}]
\PYG{g+gp}{\PYGZgt{}\PYGZgt{}\PYGZgt{} }\PYG{n}{mean} \PYG{o}{=} \PYG{p}{(}\PYG{l+m+mi}{1}\PYG{p}{,}\PYG{l+m+mi}{2}\PYG{p}{)}
\PYG{g+gp}{\PYGZgt{}\PYGZgt{}\PYGZgt{} }\PYG{n}{cov} \PYG{o}{=} \PYG{p}{[}\PYG{p}{[}\PYG{l+m+mi}{1}\PYG{p}{,}\PYG{l+m+mi}{0}\PYG{p}{]}\PYG{p}{,}\PYG{p}{[}\PYG{l+m+mi}{1}\PYG{p}{,}\PYG{l+m+mi}{0}\PYG{p}{]}\PYG{p}{]}
\PYG{g+gp}{\PYGZgt{}\PYGZgt{}\PYGZgt{} }\PYG{n}{x} \PYG{o}{=} \PYG{n}{np}\PYG{o}{.}\PYG{n}{random}\PYG{o}{.}\PYG{n}{multivariate\PYGZus{}normal}\PYG{p}{(}\PYG{n}{mean}\PYG{p}{,}\PYG{n}{cov}\PYG{p}{,}\PYG{p}{(}\PYG{l+m+mi}{3}\PYG{p}{,}\PYG{l+m+mi}{3}\PYG{p}{)}\PYG{p}{)}
\PYG{g+gp}{\PYGZgt{}\PYGZgt{}\PYGZgt{} }\PYG{n}{x}\PYG{o}{.}\PYG{n}{shape}
\PYG{g+go}{(3, 3, 2)}
\end{Verbatim}

The following is probably true, given that 0.6 is roughly twice the
standard deviation:

\begin{Verbatim}[commandchars=\\\{\}]
\PYG{g+gp}{\PYGZgt{}\PYGZgt{}\PYGZgt{} }\PYG{k}{print} \PYG{n+nb}{list}\PYG{p}{(} \PYG{p}{(}\PYG{n}{x}\PYG{p}{[}\PYG{l+m+mi}{0}\PYG{p}{,}\PYG{l+m+mi}{0}\PYG{p}{,}\PYG{p}{:}\PYG{p}{]} \PYG{o}{\PYGZhy{}} \PYG{n}{mean}\PYG{p}{)} \PYG{o}{\PYGZlt{}} \PYG{l+m+mf}{0.6} \PYG{p}{)}
\PYG{g+go}{[True, True]}
\end{Verbatim}

\end{fulllineitems}

\index{negative\_binomial() (in module acsDynStatInTime)}

\begin{fulllineitems}
\phantomsection\label{acsDynStatInTime:acsDynStatInTime.negative_binomial}\pysiglinewithargsret{\code{acsDynStatInTime.}\bfcode{negative\_binomial}}{\emph{n}, \emph{p}, \emph{size=None}}{}
Draw samples from a negative\_binomial distribution.

Samples are drawn from a negative\_Binomial distribution with specified
parameters, \emph{n} trials and \emph{p} probability of success where \emph{n} is an
integer \textgreater{} 0 and \emph{p} is in the interval {[}0, 1{]}.
\begin{description}
\item[{n}] \leavevmode{[}int{]}
Parameter, \textgreater{} 0.

\item[{p}] \leavevmode{[}float{]}
Parameter, \textgreater{}= 0 and \textless{}=1.

\item[{size}] \leavevmode{[}int or tuple of ints{]}
Output shape. If the given shape is, e.g., \code{(m, n, k)}, then
\code{m * n * k} samples are drawn.

\end{description}
\begin{description}
\item[{samples}] \leavevmode{[}int or ndarray of ints{]}
Drawn samples.

\end{description}

The probability density for the Negative Binomial distribution is
\begin{gather}
\begin{split}P(N;n,p) = \binom{N+n-1}{n-1}p^{n}(1-p)^{N},\end{split}\notag
\end{gather}
where \(n-1\) is the number of successes, \(p\) is the probability
of success, and \(N+n-1\) is the number of trials.

The negative binomial distribution gives the probability of n-1 successes
and N failures in N+n-1 trials, and success on the (N+n)th trial.

If one throws a die repeatedly until the third time a ``1'' appears, then the
probability distribution of the number of non-``1''s that appear before the
third ``1'' is a negative binomial distribution.

Draw samples from the distribution:

A real world example. A company drills wild-cat oil exploration wells, each
with an estimated probability of success of 0.1.  What is the probability
of having one success for each successive well, that is what is the
probability of a single success after drilling 5 wells, after 6 wells,
etc.?

\begin{Verbatim}[commandchars=\\\{\}]
\PYG{g+gp}{\PYGZgt{}\PYGZgt{}\PYGZgt{} }\PYG{n}{s} \PYG{o}{=} \PYG{n}{np}\PYG{o}{.}\PYG{n}{random}\PYG{o}{.}\PYG{n}{negative\PYGZus{}binomial}\PYG{p}{(}\PYG{l+m+mi}{1}\PYG{p}{,} \PYG{l+m+mf}{0.1}\PYG{p}{,} \PYG{l+m+mi}{100000}\PYG{p}{)}
\PYG{g+gp}{\PYGZgt{}\PYGZgt{}\PYGZgt{} }\PYG{k}{for} \PYG{n}{i} \PYG{o+ow}{in} \PYG{n+nb}{range}\PYG{p}{(}\PYG{l+m+mi}{1}\PYG{p}{,} \PYG{l+m+mi}{11}\PYG{p}{)}\PYG{p}{:}
\PYG{g+gp}{... }   \PYG{n}{probability} \PYG{o}{=} \PYG{n+nb}{sum}\PYG{p}{(}\PYG{n}{s}\PYG{o}{\PYGZlt{}}\PYG{n}{i}\PYG{p}{)} \PYG{o}{/} \PYG{l+m+mf}{100000.}
\PYG{g+gp}{... }   \PYG{k}{print} \PYG{n}{i}\PYG{p}{,} \PYG{l+s}{\PYGZdq{}}\PYG{l+s}{wells drilled, probability of one success =}\PYG{l+s}{\PYGZdq{}}\PYG{p}{,} \PYG{n}{probability}
\end{Verbatim}

\end{fulllineitems}

\index{noncentral\_chisquare() (in module acsDynStatInTime)}

\begin{fulllineitems}
\phantomsection\label{acsDynStatInTime:acsDynStatInTime.noncentral_chisquare}\pysiglinewithargsret{\code{acsDynStatInTime.}\bfcode{noncentral\_chisquare}}{\emph{df}, \emph{nonc}, \emph{size=None}}{}
Draw samples from a noncentral chi-square distribution.

The noncentral \(\chi^2\) distribution is a generalisation of
the \(\chi^2\) distribution.
\begin{description}
\item[{df}] \leavevmode{[}int{]}
Degrees of freedom, should be \textgreater{}= 1.

\item[{nonc}] \leavevmode{[}float{]}
Non-centrality, should be \textgreater{} 0.

\item[{size}] \leavevmode{[}int or tuple of ints{]}
Shape of the output.

\end{description}

The probability density function for the noncentral Chi-square distribution
is
\begin{gather}
\begin{split}P(x;df,nonc) = \sum^{\infty}_{i=0}
\frac{e^{-nonc/2}(nonc/2)^{i}}{i!}P_{Y_{df+2i}}(x),\end{split}\notag
\end{gather}
where \(Y_{q}\) is the Chi-square with q degrees of freedom.

In Delhi (2007), it is noted that the noncentral chi-square is useful in
bombing and coverage problems, the probability of killing the point target
given by the noncentral chi-squared distribution.

Draw values from the distribution and plot the histogram

\begin{Verbatim}[commandchars=\\\{\}]
\PYG{g+gp}{\PYGZgt{}\PYGZgt{}\PYGZgt{} }\PYG{k+kn}{import} \PYG{n+nn}{matplotlib.pyplot} \PYG{k+kn}{as} \PYG{n+nn}{plt}
\PYG{g+gp}{\PYGZgt{}\PYGZgt{}\PYGZgt{} }\PYG{n}{values} \PYG{o}{=} \PYG{n}{plt}\PYG{o}{.}\PYG{n}{hist}\PYG{p}{(}\PYG{n}{np}\PYG{o}{.}\PYG{n}{random}\PYG{o}{.}\PYG{n}{noncentral\PYGZus{}chisquare}\PYG{p}{(}\PYG{l+m+mi}{3}\PYG{p}{,} \PYG{l+m+mi}{20}\PYG{p}{,} \PYG{l+m+mi}{100000}\PYG{p}{)}\PYG{p}{,}
\PYG{g+gp}{... }                  \PYG{n}{bins}\PYG{o}{=}\PYG{l+m+mi}{200}\PYG{p}{,} \PYG{n}{normed}\PYG{o}{=}\PYG{n+nb+bp}{True}\PYG{p}{)}
\PYG{g+gp}{\PYGZgt{}\PYGZgt{}\PYGZgt{} }\PYG{n}{plt}\PYG{o}{.}\PYG{n}{show}\PYG{p}{(}\PYG{p}{)}
\end{Verbatim}

Draw values from a noncentral chisquare with very small noncentrality,
and compare to a chisquare.

\begin{Verbatim}[commandchars=\\\{\}]
\PYG{g+gp}{\PYGZgt{}\PYGZgt{}\PYGZgt{} }\PYG{n}{plt}\PYG{o}{.}\PYG{n}{figure}\PYG{p}{(}\PYG{p}{)}
\PYG{g+gp}{\PYGZgt{}\PYGZgt{}\PYGZgt{} }\PYG{n}{values} \PYG{o}{=} \PYG{n}{plt}\PYG{o}{.}\PYG{n}{hist}\PYG{p}{(}\PYG{n}{np}\PYG{o}{.}\PYG{n}{random}\PYG{o}{.}\PYG{n}{noncentral\PYGZus{}chisquare}\PYG{p}{(}\PYG{l+m+mi}{3}\PYG{p}{,} \PYG{o}{.}\PYG{l+m+mo}{0000001}\PYG{p}{,} \PYG{l+m+mi}{100000}\PYG{p}{)}\PYG{p}{,}
\PYG{g+gp}{... }                  \PYG{n}{bins}\PYG{o}{=}\PYG{n}{np}\PYG{o}{.}\PYG{n}{arange}\PYG{p}{(}\PYG{l+m+mf}{0.}\PYG{p}{,} \PYG{l+m+mi}{25}\PYG{p}{,} \PYG{o}{.}\PYG{l+m+mi}{1}\PYG{p}{)}\PYG{p}{,} \PYG{n}{normed}\PYG{o}{=}\PYG{n+nb+bp}{True}\PYG{p}{)}
\PYG{g+gp}{\PYGZgt{}\PYGZgt{}\PYGZgt{} }\PYG{n}{values2} \PYG{o}{=} \PYG{n}{plt}\PYG{o}{.}\PYG{n}{hist}\PYG{p}{(}\PYG{n}{np}\PYG{o}{.}\PYG{n}{random}\PYG{o}{.}\PYG{n}{chisquare}\PYG{p}{(}\PYG{l+m+mi}{3}\PYG{p}{,} \PYG{l+m+mi}{100000}\PYG{p}{)}\PYG{p}{,}
\PYG{g+gp}{... }                   \PYG{n}{bins}\PYG{o}{=}\PYG{n}{np}\PYG{o}{.}\PYG{n}{arange}\PYG{p}{(}\PYG{l+m+mf}{0.}\PYG{p}{,} \PYG{l+m+mi}{25}\PYG{p}{,} \PYG{o}{.}\PYG{l+m+mi}{1}\PYG{p}{)}\PYG{p}{,} \PYG{n}{normed}\PYG{o}{=}\PYG{n+nb+bp}{True}\PYG{p}{)}
\PYG{g+gp}{\PYGZgt{}\PYGZgt{}\PYGZgt{} }\PYG{n}{plt}\PYG{o}{.}\PYG{n}{plot}\PYG{p}{(}\PYG{n}{values}\PYG{p}{[}\PYG{l+m+mi}{1}\PYG{p}{]}\PYG{p}{[}\PYG{l+m+mi}{0}\PYG{p}{:}\PYG{o}{\PYGZhy{}}\PYG{l+m+mi}{1}\PYG{p}{]}\PYG{p}{,} \PYG{n}{values}\PYG{p}{[}\PYG{l+m+mi}{0}\PYG{p}{]}\PYG{o}{\PYGZhy{}}\PYG{n}{values2}\PYG{p}{[}\PYG{l+m+mi}{0}\PYG{p}{]}\PYG{p}{,} \PYG{l+s}{\PYGZsq{}}\PYG{l+s}{ob}\PYG{l+s}{\PYGZsq{}}\PYG{p}{)}
\PYG{g+gp}{\PYGZgt{}\PYGZgt{}\PYGZgt{} }\PYG{n}{plt}\PYG{o}{.}\PYG{n}{show}\PYG{p}{(}\PYG{p}{)}
\end{Verbatim}

Demonstrate how large values of non-centrality lead to a more symmetric
distribution.

\begin{Verbatim}[commandchars=\\\{\}]
\PYG{g+gp}{\PYGZgt{}\PYGZgt{}\PYGZgt{} }\PYG{n}{plt}\PYG{o}{.}\PYG{n}{figure}\PYG{p}{(}\PYG{p}{)}
\PYG{g+gp}{\PYGZgt{}\PYGZgt{}\PYGZgt{} }\PYG{n}{values} \PYG{o}{=} \PYG{n}{plt}\PYG{o}{.}\PYG{n}{hist}\PYG{p}{(}\PYG{n}{np}\PYG{o}{.}\PYG{n}{random}\PYG{o}{.}\PYG{n}{noncentral\PYGZus{}chisquare}\PYG{p}{(}\PYG{l+m+mi}{3}\PYG{p}{,} \PYG{l+m+mi}{20}\PYG{p}{,} \PYG{l+m+mi}{100000}\PYG{p}{)}\PYG{p}{,}
\PYG{g+gp}{... }                  \PYG{n}{bins}\PYG{o}{=}\PYG{l+m+mi}{200}\PYG{p}{,} \PYG{n}{normed}\PYG{o}{=}\PYG{n+nb+bp}{True}\PYG{p}{)}
\PYG{g+gp}{\PYGZgt{}\PYGZgt{}\PYGZgt{} }\PYG{n}{plt}\PYG{o}{.}\PYG{n}{show}\PYG{p}{(}\PYG{p}{)}
\end{Verbatim}

\end{fulllineitems}

\index{noncentral\_f() (in module acsDynStatInTime)}

\begin{fulllineitems}
\phantomsection\label{acsDynStatInTime:acsDynStatInTime.noncentral_f}\pysiglinewithargsret{\code{acsDynStatInTime.}\bfcode{noncentral\_f}}{\emph{dfnum}, \emph{dfden}, \emph{nonc}, \emph{size=None}}{}
Draw samples from the noncentral F distribution.

Samples are drawn from an F distribution with specified parameters,
\emph{dfnum} (degrees of freedom in numerator) and \emph{dfden} (degrees of
freedom in denominator), where both parameters \textgreater{} 1.
\emph{nonc} is the non-centrality parameter.
\begin{description}
\item[{dfnum}] \leavevmode{[}int{]}
Parameter, should be \textgreater{} 1.

\item[{dfden}] \leavevmode{[}int{]}
Parameter, should be \textgreater{} 1.

\item[{nonc}] \leavevmode{[}float{]}
Parameter, should be \textgreater{}= 0.

\item[{size}] \leavevmode{[}int or tuple of ints{]}
Output shape. If the given shape is, e.g., \code{(m, n, k)}, then
\code{m * n * k} samples are drawn.

\end{description}
\begin{description}
\item[{samples}] \leavevmode{[}scalar or ndarray{]}
Drawn samples.

\end{description}

When calculating the power of an experiment (power = probability of
rejecting the null hypothesis when a specific alternative is true) the
non-central F statistic becomes important.  When the null hypothesis is
true, the F statistic follows a central F distribution. When the null
hypothesis is not true, then it follows a non-central F statistic.

Weisstein, Eric W. ``Noncentral F-Distribution.'' From MathWorld--A Wolfram
Web Resource.  \href{http://mathworld.wolfram.com/NoncentralF-Distribution.html}{http://mathworld.wolfram.com/NoncentralF-Distribution.html}

Wikipedia, ``Noncentral F distribution'',
\href{http://en.wikipedia.org/wiki/Noncentral\_F-distribution}{http://en.wikipedia.org/wiki/Noncentral\_F-distribution}

In a study, testing for a specific alternative to the null hypothesis
requires use of the Noncentral F distribution. We need to calculate the
area in the tail of the distribution that exceeds the value of the F
distribution for the null hypothesis.  We'll plot the two probability
distributions for comparison.

\begin{Verbatim}[commandchars=\\\{\}]
\PYG{g+gp}{\PYGZgt{}\PYGZgt{}\PYGZgt{} }\PYG{n}{dfnum} \PYG{o}{=} \PYG{l+m+mi}{3} \PYG{c}{\PYGZsh{} between group deg of freedom}
\PYG{g+gp}{\PYGZgt{}\PYGZgt{}\PYGZgt{} }\PYG{n}{dfden} \PYG{o}{=} \PYG{l+m+mi}{20} \PYG{c}{\PYGZsh{} within groups degrees of freedom}
\PYG{g+gp}{\PYGZgt{}\PYGZgt{}\PYGZgt{} }\PYG{n}{nonc} \PYG{o}{=} \PYG{l+m+mf}{3.0}
\PYG{g+gp}{\PYGZgt{}\PYGZgt{}\PYGZgt{} }\PYG{n}{nc\PYGZus{}vals} \PYG{o}{=} \PYG{n}{np}\PYG{o}{.}\PYG{n}{random}\PYG{o}{.}\PYG{n}{noncentral\PYGZus{}f}\PYG{p}{(}\PYG{n}{dfnum}\PYG{p}{,} \PYG{n}{dfden}\PYG{p}{,} \PYG{n}{nonc}\PYG{p}{,} \PYG{l+m+mi}{1000000}\PYG{p}{)}
\PYG{g+gp}{\PYGZgt{}\PYGZgt{}\PYGZgt{} }\PYG{n}{NF} \PYG{o}{=} \PYG{n}{np}\PYG{o}{.}\PYG{n}{histogram}\PYG{p}{(}\PYG{n}{nc\PYGZus{}vals}\PYG{p}{,} \PYG{n}{bins}\PYG{o}{=}\PYG{l+m+mi}{50}\PYG{p}{,} \PYG{n}{normed}\PYG{o}{=}\PYG{n+nb+bp}{True}\PYG{p}{)}
\PYG{g+gp}{\PYGZgt{}\PYGZgt{}\PYGZgt{} }\PYG{n}{c\PYGZus{}vals} \PYG{o}{=} \PYG{n}{np}\PYG{o}{.}\PYG{n}{random}\PYG{o}{.}\PYG{n}{f}\PYG{p}{(}\PYG{n}{dfnum}\PYG{p}{,} \PYG{n}{dfden}\PYG{p}{,} \PYG{l+m+mi}{1000000}\PYG{p}{)}
\PYG{g+gp}{\PYGZgt{}\PYGZgt{}\PYGZgt{} }\PYG{n}{F} \PYG{o}{=} \PYG{n}{np}\PYG{o}{.}\PYG{n}{histogram}\PYG{p}{(}\PYG{n}{c\PYGZus{}vals}\PYG{p}{,} \PYG{n}{bins}\PYG{o}{=}\PYG{l+m+mi}{50}\PYG{p}{,} \PYG{n}{normed}\PYG{o}{=}\PYG{n+nb+bp}{True}\PYG{p}{)}
\PYG{g+gp}{\PYGZgt{}\PYGZgt{}\PYGZgt{} }\PYG{n}{plt}\PYG{o}{.}\PYG{n}{plot}\PYG{p}{(}\PYG{n}{F}\PYG{p}{[}\PYG{l+m+mi}{1}\PYG{p}{]}\PYG{p}{[}\PYG{l+m+mi}{1}\PYG{p}{:}\PYG{p}{]}\PYG{p}{,} \PYG{n}{F}\PYG{p}{[}\PYG{l+m+mi}{0}\PYG{p}{]}\PYG{p}{)}
\PYG{g+gp}{\PYGZgt{}\PYGZgt{}\PYGZgt{} }\PYG{n}{plt}\PYG{o}{.}\PYG{n}{plot}\PYG{p}{(}\PYG{n}{NF}\PYG{p}{[}\PYG{l+m+mi}{1}\PYG{p}{]}\PYG{p}{[}\PYG{l+m+mi}{1}\PYG{p}{:}\PYG{p}{]}\PYG{p}{,} \PYG{n}{NF}\PYG{p}{[}\PYG{l+m+mi}{0}\PYG{p}{]}\PYG{p}{)}
\PYG{g+gp}{\PYGZgt{}\PYGZgt{}\PYGZgt{} }\PYG{n}{plt}\PYG{o}{.}\PYG{n}{show}\PYG{p}{(}\PYG{p}{)}
\end{Verbatim}

\end{fulllineitems}

\index{normal() (in module acsDynStatInTime)}

\begin{fulllineitems}
\phantomsection\label{acsDynStatInTime:acsDynStatInTime.normal}\pysiglinewithargsret{\code{acsDynStatInTime.}\bfcode{normal}}{\emph{loc=0.0}, \emph{scale=1.0}, \emph{size=None}}{}
Draw random samples from a normal (Gaussian) distribution.

The probability density function of the normal distribution, first
derived by De Moivre and 200 years later by both Gauss and Laplace
independently {\color{red}\bfseries{}{[}2{]}\_}, is often called the bell curve because of
its characteristic shape (see the example below).

The normal distributions occurs often in nature.  For example, it
describes the commonly occurring distribution of samples influenced
by a large number of tiny, random disturbances, each with its own
unique distribution {\color{red}\bfseries{}{[}2{]}\_}.
\begin{description}
\item[{loc}] \leavevmode{[}float{]}
Mean (``centre'') of the distribution.

\item[{scale}] \leavevmode{[}float{]}
Standard deviation (spread or ``width'') of the distribution.

\item[{size}] \leavevmode{[}tuple of ints{]}
Output shape.  If the given shape is, e.g., \code{(m, n, k)}, then
\code{m * n * k} samples are drawn.

\end{description}
\begin{description}
\item[{scipy.stats.distributions.norm}] \leavevmode{[}probability density function,{]}
distribution or cumulative density function, etc.

\end{description}

The probability density for the Gaussian distribution is
\begin{gather}
\begin{split}p(x) = \frac{1}{\sqrt{ 2 \pi \sigma^2 }}
e^{ - \frac{ (x - \mu)^2 } {2 \sigma^2} },\end{split}\notag
\end{gather}
where \(\mu\) is the mean and \(\sigma\) the standard deviation.
The square of the standard deviation, \(\sigma^2\), is called the
variance.

The function has its peak at the mean, and its ``spread'' increases with
the standard deviation (the function reaches 0.607 times its maximum at
\(x + \sigma\) and \(x - \sigma\) {\color{red}\bfseries{}{[}2{]}\_}).  This implies that
\emph{numpy.random.normal} is more likely to return samples lying close to the
mean, rather than those far away.

Draw samples from the distribution:

\begin{Verbatim}[commandchars=\\\{\}]
\PYG{g+gp}{\PYGZgt{}\PYGZgt{}\PYGZgt{} }\PYG{n}{mu}\PYG{p}{,} \PYG{n}{sigma} \PYG{o}{=} \PYG{l+m+mi}{0}\PYG{p}{,} \PYG{l+m+mf}{0.1} \PYG{c}{\PYGZsh{} mean and standard deviation}
\PYG{g+gp}{\PYGZgt{}\PYGZgt{}\PYGZgt{} }\PYG{n}{s} \PYG{o}{=} \PYG{n}{np}\PYG{o}{.}\PYG{n}{random}\PYG{o}{.}\PYG{n}{normal}\PYG{p}{(}\PYG{n}{mu}\PYG{p}{,} \PYG{n}{sigma}\PYG{p}{,} \PYG{l+m+mi}{1000}\PYG{p}{)}
\end{Verbatim}

Verify the mean and the variance:

\begin{Verbatim}[commandchars=\\\{\}]
\PYG{g+gp}{\PYGZgt{}\PYGZgt{}\PYGZgt{} }\PYG{n+nb}{abs}\PYG{p}{(}\PYG{n}{mu} \PYG{o}{\PYGZhy{}} \PYG{n}{np}\PYG{o}{.}\PYG{n}{mean}\PYG{p}{(}\PYG{n}{s}\PYG{p}{)}\PYG{p}{)} \PYG{o}{\PYGZlt{}} \PYG{l+m+mf}{0.01}
\PYG{g+go}{True}
\end{Verbatim}

\begin{Verbatim}[commandchars=\\\{\}]
\PYG{g+gp}{\PYGZgt{}\PYGZgt{}\PYGZgt{} }\PYG{n+nb}{abs}\PYG{p}{(}\PYG{n}{sigma} \PYG{o}{\PYGZhy{}} \PYG{n}{np}\PYG{o}{.}\PYG{n}{std}\PYG{p}{(}\PYG{n}{s}\PYG{p}{,} \PYG{n}{ddof}\PYG{o}{=}\PYG{l+m+mi}{1}\PYG{p}{)}\PYG{p}{)} \PYG{o}{\PYGZlt{}} \PYG{l+m+mf}{0.01}
\PYG{g+go}{True}
\end{Verbatim}

Display the histogram of the samples, along with
the probability density function:

\begin{Verbatim}[commandchars=\\\{\}]
\PYG{g+gp}{\PYGZgt{}\PYGZgt{}\PYGZgt{} }\PYG{k+kn}{import} \PYG{n+nn}{matplotlib.pyplot} \PYG{k+kn}{as} \PYG{n+nn}{plt}
\PYG{g+gp}{\PYGZgt{}\PYGZgt{}\PYGZgt{} }\PYG{n}{count}\PYG{p}{,} \PYG{n}{bins}\PYG{p}{,} \PYG{n}{ignored} \PYG{o}{=} \PYG{n}{plt}\PYG{o}{.}\PYG{n}{hist}\PYG{p}{(}\PYG{n}{s}\PYG{p}{,} \PYG{l+m+mi}{30}\PYG{p}{,} \PYG{n}{normed}\PYG{o}{=}\PYG{n+nb+bp}{True}\PYG{p}{)}
\PYG{g+gp}{\PYGZgt{}\PYGZgt{}\PYGZgt{} }\PYG{n}{plt}\PYG{o}{.}\PYG{n}{plot}\PYG{p}{(}\PYG{n}{bins}\PYG{p}{,} \PYG{l+m+mi}{1}\PYG{o}{/}\PYG{p}{(}\PYG{n}{sigma} \PYG{o}{*} \PYG{n}{np}\PYG{o}{.}\PYG{n}{sqrt}\PYG{p}{(}\PYG{l+m+mi}{2} \PYG{o}{*} \PYG{n}{np}\PYG{o}{.}\PYG{n}{pi}\PYG{p}{)}\PYG{p}{)} \PYG{o}{*}
\PYG{g+gp}{... }               \PYG{n}{np}\PYG{o}{.}\PYG{n}{exp}\PYG{p}{(} \PYG{o}{\PYGZhy{}} \PYG{p}{(}\PYG{n}{bins} \PYG{o}{\PYGZhy{}} \PYG{n}{mu}\PYG{p}{)}\PYG{o}{*}\PYG{o}{*}\PYG{l+m+mi}{2} \PYG{o}{/} \PYG{p}{(}\PYG{l+m+mi}{2} \PYG{o}{*} \PYG{n}{sigma}\PYG{o}{*}\PYG{o}{*}\PYG{l+m+mi}{2}\PYG{p}{)} \PYG{p}{)}\PYG{p}{,}
\PYG{g+gp}{... }         \PYG{n}{linewidth}\PYG{o}{=}\PYG{l+m+mi}{2}\PYG{p}{,} \PYG{n}{color}\PYG{o}{=}\PYG{l+s}{\PYGZsq{}}\PYG{l+s}{r}\PYG{l+s}{\PYGZsq{}}\PYG{p}{)}
\PYG{g+gp}{\PYGZgt{}\PYGZgt{}\PYGZgt{} }\PYG{n}{plt}\PYG{o}{.}\PYG{n}{show}\PYG{p}{(}\PYG{p}{)}
\end{Verbatim}

\end{fulllineitems}

\index{pareto() (in module acsDynStatInTime)}

\begin{fulllineitems}
\phantomsection\label{acsDynStatInTime:acsDynStatInTime.pareto}\pysiglinewithargsret{\code{acsDynStatInTime.}\bfcode{pareto}}{\emph{a}, \emph{size=None}}{}
Draw samples from a Pareto II or Lomax distribution with specified shape.

The Lomax or Pareto II distribution is a shifted Pareto distribution. The
classical Pareto distribution can be obtained from the Lomax distribution
by adding the location parameter m, see below. The smallest value of the
Lomax distribution is zero while for the classical Pareto distribution it
is m, where the standard Pareto distribution has location m=1.
Lomax can also be considered as a simplified version of the Generalized
Pareto distribution (available in SciPy), with the scale set to one and
the location set to zero.

The Pareto distribution must be greater than zero, and is unbounded above.
It is also known as the ``80-20 rule''.  In this distribution, 80 percent of
the weights are in the lowest 20 percent of the range, while the other 20
percent fill the remaining 80 percent of the range.
\begin{description}
\item[{shape}] \leavevmode{[}float, \textgreater{} 0.{]}
Shape of the distribution.

\item[{size}] \leavevmode{[}tuple of ints{]}
Output shape.  If the given shape is, e.g., \code{(m, n, k)}, then
\code{m * n * k} samples are drawn.

\end{description}
\begin{description}
\item[{scipy.stats.distributions.lomax.pdf}] \leavevmode{[}probability density function,{]}
distribution or cumulative density function, etc.

\item[{scipy.stats.distributions.genpareto.pdf}] \leavevmode{[}probability density function,{]}
distribution or cumulative density function, etc.

\end{description}

The probability density for the Pareto distribution is
\begin{gather}
\begin{split}p(x) = \frac{am^a}{x^{a+1}}\end{split}\notag
\end{gather}
where \(a\) is the shape and \(m\) the location

The Pareto distribution, named after the Italian economist Vilfredo Pareto,
is a power law probability distribution useful in many real world problems.
Outside the field of economics it is generally referred to as the Bradford
distribution. Pareto developed the distribution to describe the
distribution of wealth in an economy.  It has also found use in insurance,
web page access statistics, oil field sizes, and many other problems,
including the download frequency for projects in Sourceforge {[}1{]}.  It is
one of the so-called ``fat-tailed'' distributions.

Draw samples from the distribution:

\begin{Verbatim}[commandchars=\\\{\}]
\PYG{g+gp}{\PYGZgt{}\PYGZgt{}\PYGZgt{} }\PYG{n}{a}\PYG{p}{,} \PYG{n}{m} \PYG{o}{=} \PYG{l+m+mf}{3.}\PYG{p}{,} \PYG{l+m+mf}{1.} \PYG{c}{\PYGZsh{} shape and mode}
\PYG{g+gp}{\PYGZgt{}\PYGZgt{}\PYGZgt{} }\PYG{n}{s} \PYG{o}{=} \PYG{n}{np}\PYG{o}{.}\PYG{n}{random}\PYG{o}{.}\PYG{n}{pareto}\PYG{p}{(}\PYG{n}{a}\PYG{p}{,} \PYG{l+m+mi}{1000}\PYG{p}{)} \PYG{o}{+} \PYG{n}{m}
\end{Verbatim}

Display the histogram of the samples, along with
the probability density function:

\begin{Verbatim}[commandchars=\\\{\}]
\PYG{g+gp}{\PYGZgt{}\PYGZgt{}\PYGZgt{} }\PYG{k+kn}{import} \PYG{n+nn}{matplotlib.pyplot} \PYG{k+kn}{as} \PYG{n+nn}{plt}
\PYG{g+gp}{\PYGZgt{}\PYGZgt{}\PYGZgt{} }\PYG{n}{count}\PYG{p}{,} \PYG{n}{bins}\PYG{p}{,} \PYG{n}{ignored} \PYG{o}{=} \PYG{n}{plt}\PYG{o}{.}\PYG{n}{hist}\PYG{p}{(}\PYG{n}{s}\PYG{p}{,} \PYG{l+m+mi}{100}\PYG{p}{,} \PYG{n}{normed}\PYG{o}{=}\PYG{n+nb+bp}{True}\PYG{p}{,} \PYG{n}{align}\PYG{o}{=}\PYG{l+s}{\PYGZsq{}}\PYG{l+s}{center}\PYG{l+s}{\PYGZsq{}}\PYG{p}{)}
\PYG{g+gp}{\PYGZgt{}\PYGZgt{}\PYGZgt{} }\PYG{n}{fit} \PYG{o}{=} \PYG{n}{a}\PYG{o}{*}\PYG{n}{m}\PYG{o}{*}\PYG{o}{*}\PYG{n}{a}\PYG{o}{/}\PYG{n}{bins}\PYG{o}{*}\PYG{o}{*}\PYG{p}{(}\PYG{n}{a}\PYG{o}{+}\PYG{l+m+mi}{1}\PYG{p}{)}
\PYG{g+gp}{\PYGZgt{}\PYGZgt{}\PYGZgt{} }\PYG{n}{plt}\PYG{o}{.}\PYG{n}{plot}\PYG{p}{(}\PYG{n}{bins}\PYG{p}{,} \PYG{n+nb}{max}\PYG{p}{(}\PYG{n}{count}\PYG{p}{)}\PYG{o}{*}\PYG{n}{fit}\PYG{o}{/}\PYG{n+nb}{max}\PYG{p}{(}\PYG{n}{fit}\PYG{p}{)}\PYG{p}{,}\PYG{n}{linewidth}\PYG{o}{=}\PYG{l+m+mi}{2}\PYG{p}{,} \PYG{n}{color}\PYG{o}{=}\PYG{l+s}{\PYGZsq{}}\PYG{l+s}{r}\PYG{l+s}{\PYGZsq{}}\PYG{p}{)}
\PYG{g+gp}{\PYGZgt{}\PYGZgt{}\PYGZgt{} }\PYG{n}{plt}\PYG{o}{.}\PYG{n}{show}\PYG{p}{(}\PYG{p}{)}
\end{Verbatim}

\end{fulllineitems}

\index{permutation() (in module acsDynStatInTime)}

\begin{fulllineitems}
\phantomsection\label{acsDynStatInTime:acsDynStatInTime.permutation}\pysiglinewithargsret{\code{acsDynStatInTime.}\bfcode{permutation}}{\emph{x}}{}
Randomly permute a sequence, or return a permuted range.

If \emph{x} is a multi-dimensional array, it is only shuffled along its
first index.
\begin{description}
\item[{x}] \leavevmode{[}int or array\_like{]}
If \emph{x} is an integer, randomly permute \code{np.arange(x)}.
If \emph{x} is an array, make a copy and shuffle the elements
randomly.

\end{description}
\begin{description}
\item[{out}] \leavevmode{[}ndarray{]}
Permuted sequence or array range.

\end{description}

\begin{Verbatim}[commandchars=\\\{\}]
\PYG{g+gp}{\PYGZgt{}\PYGZgt{}\PYGZgt{} }\PYG{n}{np}\PYG{o}{.}\PYG{n}{random}\PYG{o}{.}\PYG{n}{permutation}\PYG{p}{(}\PYG{l+m+mi}{10}\PYG{p}{)}
\PYG{g+go}{array([1, 7, 4, 3, 0, 9, 2, 5, 8, 6])}
\end{Verbatim}

\begin{Verbatim}[commandchars=\\\{\}]
\PYG{g+gp}{\PYGZgt{}\PYGZgt{}\PYGZgt{} }\PYG{n}{np}\PYG{o}{.}\PYG{n}{random}\PYG{o}{.}\PYG{n}{permutation}\PYG{p}{(}\PYG{p}{[}\PYG{l+m+mi}{1}\PYG{p}{,} \PYG{l+m+mi}{4}\PYG{p}{,} \PYG{l+m+mi}{9}\PYG{p}{,} \PYG{l+m+mi}{12}\PYG{p}{,} \PYG{l+m+mi}{15}\PYG{p}{]}\PYG{p}{)}
\PYG{g+go}{array([15,  1,  9,  4, 12])}
\end{Verbatim}

\begin{Verbatim}[commandchars=\\\{\}]
\PYG{g+gp}{\PYGZgt{}\PYGZgt{}\PYGZgt{} }\PYG{n}{arr} \PYG{o}{=} \PYG{n}{np}\PYG{o}{.}\PYG{n}{arange}\PYG{p}{(}\PYG{l+m+mi}{9}\PYG{p}{)}\PYG{o}{.}\PYG{n}{reshape}\PYG{p}{(}\PYG{p}{(}\PYG{l+m+mi}{3}\PYG{p}{,} \PYG{l+m+mi}{3}\PYG{p}{)}\PYG{p}{)}
\PYG{g+gp}{\PYGZgt{}\PYGZgt{}\PYGZgt{} }\PYG{n}{np}\PYG{o}{.}\PYG{n}{random}\PYG{o}{.}\PYG{n}{permutation}\PYG{p}{(}\PYG{n}{arr}\PYG{p}{)}
\PYG{g+go}{array([[6, 7, 8],}
\PYG{g+go}{       [0, 1, 2],}
\PYG{g+go}{       [3, 4, 5]])}
\end{Verbatim}

\end{fulllineitems}

\index{poisson() (in module acsDynStatInTime)}

\begin{fulllineitems}
\phantomsection\label{acsDynStatInTime:acsDynStatInTime.poisson}\pysiglinewithargsret{\code{acsDynStatInTime.}\bfcode{poisson}}{\emph{lam=1.0}, \emph{size=None}}{}
Draw samples from a Poisson distribution.

The Poisson distribution is the limit of the Binomial
distribution for large N.
\begin{description}
\item[{lam}] \leavevmode{[}float{]}
Expectation of interval, should be \textgreater{}= 0.

\item[{size}] \leavevmode{[}int or tuple of ints, optional{]}
Output shape. If the given shape is, e.g., \code{(m, n, k)}, then
\code{m * n * k} samples are drawn.

\end{description}

The Poisson distribution
\begin{gather}
\begin{split}f(k; \lambda)=\frac{\lambda^k e^{-\lambda}}{k!}\end{split}\notag
\end{gather}
For events with an expected separation \(\lambda\) the Poisson
distribution \(f(k; \lambda)\) describes the probability of
\(k\) events occurring within the observed interval \(\lambda\).

Because the output is limited to the range of the C long type, a
ValueError is raised when \emph{lam} is within 10 sigma of the maximum
representable value.

Draw samples from the distribution:

\begin{Verbatim}[commandchars=\\\{\}]
\PYG{g+gp}{\PYGZgt{}\PYGZgt{}\PYGZgt{} }\PYG{k+kn}{import} \PYG{n+nn}{numpy} \PYG{k+kn}{as} \PYG{n+nn}{np}
\PYG{g+gp}{\PYGZgt{}\PYGZgt{}\PYGZgt{} }\PYG{n}{s} \PYG{o}{=} \PYG{n}{np}\PYG{o}{.}\PYG{n}{random}\PYG{o}{.}\PYG{n}{poisson}\PYG{p}{(}\PYG{l+m+mi}{5}\PYG{p}{,} \PYG{l+m+mi}{10000}\PYG{p}{)}
\end{Verbatim}

Display histogram of the sample:

\begin{Verbatim}[commandchars=\\\{\}]
\PYG{g+gp}{\PYGZgt{}\PYGZgt{}\PYGZgt{} }\PYG{k+kn}{import} \PYG{n+nn}{matplotlib.pyplot} \PYG{k+kn}{as} \PYG{n+nn}{plt}
\PYG{g+gp}{\PYGZgt{}\PYGZgt{}\PYGZgt{} }\PYG{n}{count}\PYG{p}{,} \PYG{n}{bins}\PYG{p}{,} \PYG{n}{ignored} \PYG{o}{=} \PYG{n}{plt}\PYG{o}{.}\PYG{n}{hist}\PYG{p}{(}\PYG{n}{s}\PYG{p}{,} \PYG{l+m+mi}{14}\PYG{p}{,} \PYG{n}{normed}\PYG{o}{=}\PYG{n+nb+bp}{True}\PYG{p}{)}
\PYG{g+gp}{\PYGZgt{}\PYGZgt{}\PYGZgt{} }\PYG{n}{plt}\PYG{o}{.}\PYG{n}{show}\PYG{p}{(}\PYG{p}{)}
\end{Verbatim}

\end{fulllineitems}

\index{power() (in module acsDynStatInTime)}

\begin{fulllineitems}
\phantomsection\label{acsDynStatInTime:acsDynStatInTime.power}\pysiglinewithargsret{\code{acsDynStatInTime.}\bfcode{power}}{\emph{a}, \emph{size=None}}{}
Draws samples in {[}0, 1{]} from a power distribution with positive
exponent a - 1.

Also known as the power function distribution.
\begin{description}
\item[{a}] \leavevmode{[}float{]}
parameter, \textgreater{} 0

\item[{size}] \leavevmode{[}tuple of ints{]}\begin{description}
\item[{Output shape.  If the given shape is, e.g., \code{(m, n, k)}, then}] \leavevmode
\code{m * n * k} samples are drawn.

\end{description}

\end{description}
\begin{description}
\item[{samples}] \leavevmode{[}\{ndarray, scalar\}{]}
The returned samples lie in {[}0, 1{]}.

\end{description}
\begin{description}
\item[{ValueError}] \leavevmode
If a\textless{}1.

\end{description}

The probability density function is
\begin{gather}
\begin{split}P(x; a) = ax^{a-1}, 0 \le x \le 1, a>0.\end{split}\notag
\end{gather}
The power function distribution is just the inverse of the Pareto
distribution. It may also be seen as a special case of the Beta
distribution.

It is used, for example, in modeling the over-reporting of insurance
claims.

Draw samples from the distribution:

\begin{Verbatim}[commandchars=\\\{\}]
\PYG{g+gp}{\PYGZgt{}\PYGZgt{}\PYGZgt{} }\PYG{n}{a} \PYG{o}{=} \PYG{l+m+mf}{5.} \PYG{c}{\PYGZsh{} shape}
\PYG{g+gp}{\PYGZgt{}\PYGZgt{}\PYGZgt{} }\PYG{n}{samples} \PYG{o}{=} \PYG{l+m+mi}{1000}
\PYG{g+gp}{\PYGZgt{}\PYGZgt{}\PYGZgt{} }\PYG{n}{s} \PYG{o}{=} \PYG{n}{np}\PYG{o}{.}\PYG{n}{random}\PYG{o}{.}\PYG{n}{power}\PYG{p}{(}\PYG{n}{a}\PYG{p}{,} \PYG{n}{samples}\PYG{p}{)}
\end{Verbatim}

Display the histogram of the samples, along with
the probability density function:

\begin{Verbatim}[commandchars=\\\{\}]
\PYG{g+gp}{\PYGZgt{}\PYGZgt{}\PYGZgt{} }\PYG{k+kn}{import} \PYG{n+nn}{matplotlib.pyplot} \PYG{k+kn}{as} \PYG{n+nn}{plt}
\PYG{g+gp}{\PYGZgt{}\PYGZgt{}\PYGZgt{} }\PYG{n}{count}\PYG{p}{,} \PYG{n}{bins}\PYG{p}{,} \PYG{n}{ignored} \PYG{o}{=} \PYG{n}{plt}\PYG{o}{.}\PYG{n}{hist}\PYG{p}{(}\PYG{n}{s}\PYG{p}{,} \PYG{n}{bins}\PYG{o}{=}\PYG{l+m+mi}{30}\PYG{p}{)}
\PYG{g+gp}{\PYGZgt{}\PYGZgt{}\PYGZgt{} }\PYG{n}{x} \PYG{o}{=} \PYG{n}{np}\PYG{o}{.}\PYG{n}{linspace}\PYG{p}{(}\PYG{l+m+mi}{0}\PYG{p}{,} \PYG{l+m+mi}{1}\PYG{p}{,} \PYG{l+m+mi}{100}\PYG{p}{)}
\PYG{g+gp}{\PYGZgt{}\PYGZgt{}\PYGZgt{} }\PYG{n}{y} \PYG{o}{=} \PYG{n}{a}\PYG{o}{*}\PYG{n}{x}\PYG{o}{*}\PYG{o}{*}\PYG{p}{(}\PYG{n}{a}\PYG{o}{\PYGZhy{}}\PYG{l+m+mf}{1.}\PYG{p}{)}
\PYG{g+gp}{\PYGZgt{}\PYGZgt{}\PYGZgt{} }\PYG{n}{normed\PYGZus{}y} \PYG{o}{=} \PYG{n}{samples}\PYG{o}{*}\PYG{n}{np}\PYG{o}{.}\PYG{n}{diff}\PYG{p}{(}\PYG{n}{bins}\PYG{p}{)}\PYG{p}{[}\PYG{l+m+mi}{0}\PYG{p}{]}\PYG{o}{*}\PYG{n}{y}
\PYG{g+gp}{\PYGZgt{}\PYGZgt{}\PYGZgt{} }\PYG{n}{plt}\PYG{o}{.}\PYG{n}{plot}\PYG{p}{(}\PYG{n}{x}\PYG{p}{,} \PYG{n}{normed\PYGZus{}y}\PYG{p}{)}
\PYG{g+gp}{\PYGZgt{}\PYGZgt{}\PYGZgt{} }\PYG{n}{plt}\PYG{o}{.}\PYG{n}{show}\PYG{p}{(}\PYG{p}{)}
\end{Verbatim}

Compare the power function distribution to the inverse of the Pareto.

\begin{Verbatim}[commandchars=\\\{\}]
\PYG{g+gp}{\PYGZgt{}\PYGZgt{}\PYGZgt{} }\PYG{k+kn}{from} \PYG{n+nn}{scipy} \PYG{k+kn}{import} \PYG{n}{stats}
\PYG{g+gp}{\PYGZgt{}\PYGZgt{}\PYGZgt{} }\PYG{n}{rvs} \PYG{o}{=} \PYG{n}{np}\PYG{o}{.}\PYG{n}{random}\PYG{o}{.}\PYG{n}{power}\PYG{p}{(}\PYG{l+m+mi}{5}\PYG{p}{,} \PYG{l+m+mi}{1000000}\PYG{p}{)}
\PYG{g+gp}{\PYGZgt{}\PYGZgt{}\PYGZgt{} }\PYG{n}{rvsp} \PYG{o}{=} \PYG{n}{np}\PYG{o}{.}\PYG{n}{random}\PYG{o}{.}\PYG{n}{pareto}\PYG{p}{(}\PYG{l+m+mi}{5}\PYG{p}{,} \PYG{l+m+mi}{1000000}\PYG{p}{)}
\PYG{g+gp}{\PYGZgt{}\PYGZgt{}\PYGZgt{} }\PYG{n}{xx} \PYG{o}{=} \PYG{n}{np}\PYG{o}{.}\PYG{n}{linspace}\PYG{p}{(}\PYG{l+m+mi}{0}\PYG{p}{,}\PYG{l+m+mi}{1}\PYG{p}{,}\PYG{l+m+mi}{100}\PYG{p}{)}
\PYG{g+gp}{\PYGZgt{}\PYGZgt{}\PYGZgt{} }\PYG{n}{powpdf} \PYG{o}{=} \PYG{n}{stats}\PYG{o}{.}\PYG{n}{powerlaw}\PYG{o}{.}\PYG{n}{pdf}\PYG{p}{(}\PYG{n}{xx}\PYG{p}{,}\PYG{l+m+mi}{5}\PYG{p}{)}
\end{Verbatim}

\begin{Verbatim}[commandchars=\\\{\}]
\PYG{g+gp}{\PYGZgt{}\PYGZgt{}\PYGZgt{} }\PYG{n}{plt}\PYG{o}{.}\PYG{n}{figure}\PYG{p}{(}\PYG{p}{)}
\PYG{g+gp}{\PYGZgt{}\PYGZgt{}\PYGZgt{} }\PYG{n}{plt}\PYG{o}{.}\PYG{n}{hist}\PYG{p}{(}\PYG{n}{rvs}\PYG{p}{,} \PYG{n}{bins}\PYG{o}{=}\PYG{l+m+mi}{50}\PYG{p}{,} \PYG{n}{normed}\PYG{o}{=}\PYG{n+nb+bp}{True}\PYG{p}{)}
\PYG{g+gp}{\PYGZgt{}\PYGZgt{}\PYGZgt{} }\PYG{n}{plt}\PYG{o}{.}\PYG{n}{plot}\PYG{p}{(}\PYG{n}{xx}\PYG{p}{,}\PYG{n}{powpdf}\PYG{p}{,}\PYG{l+s}{\PYGZsq{}}\PYG{l+s}{r\PYGZhy{}}\PYG{l+s}{\PYGZsq{}}\PYG{p}{)}
\PYG{g+gp}{\PYGZgt{}\PYGZgt{}\PYGZgt{} }\PYG{n}{plt}\PYG{o}{.}\PYG{n}{title}\PYG{p}{(}\PYG{l+s}{\PYGZsq{}}\PYG{l+s}{np.random.power(5)}\PYG{l+s}{\PYGZsq{}}\PYG{p}{)}
\end{Verbatim}

\begin{Verbatim}[commandchars=\\\{\}]
\PYG{g+gp}{\PYGZgt{}\PYGZgt{}\PYGZgt{} }\PYG{n}{plt}\PYG{o}{.}\PYG{n}{figure}\PYG{p}{(}\PYG{p}{)}
\PYG{g+gp}{\PYGZgt{}\PYGZgt{}\PYGZgt{} }\PYG{n}{plt}\PYG{o}{.}\PYG{n}{hist}\PYG{p}{(}\PYG{l+m+mf}{1.}\PYG{o}{/}\PYG{p}{(}\PYG{l+m+mf}{1.}\PYG{o}{+}\PYG{n}{rvsp}\PYG{p}{)}\PYG{p}{,} \PYG{n}{bins}\PYG{o}{=}\PYG{l+m+mi}{50}\PYG{p}{,} \PYG{n}{normed}\PYG{o}{=}\PYG{n+nb+bp}{True}\PYG{p}{)}
\PYG{g+gp}{\PYGZgt{}\PYGZgt{}\PYGZgt{} }\PYG{n}{plt}\PYG{o}{.}\PYG{n}{plot}\PYG{p}{(}\PYG{n}{xx}\PYG{p}{,}\PYG{n}{powpdf}\PYG{p}{,}\PYG{l+s}{\PYGZsq{}}\PYG{l+s}{r\PYGZhy{}}\PYG{l+s}{\PYGZsq{}}\PYG{p}{)}
\PYG{g+gp}{\PYGZgt{}\PYGZgt{}\PYGZgt{} }\PYG{n}{plt}\PYG{o}{.}\PYG{n}{title}\PYG{p}{(}\PYG{l+s}{\PYGZsq{}}\PYG{l+s}{inverse of 1 + np.random.pareto(5)}\PYG{l+s}{\PYGZsq{}}\PYG{p}{)}
\end{Verbatim}

\begin{Verbatim}[commandchars=\\\{\}]
\PYG{g+gp}{\PYGZgt{}\PYGZgt{}\PYGZgt{} }\PYG{n}{plt}\PYG{o}{.}\PYG{n}{figure}\PYG{p}{(}\PYG{p}{)}
\PYG{g+gp}{\PYGZgt{}\PYGZgt{}\PYGZgt{} }\PYG{n}{plt}\PYG{o}{.}\PYG{n}{hist}\PYG{p}{(}\PYG{l+m+mf}{1.}\PYG{o}{/}\PYG{p}{(}\PYG{l+m+mf}{1.}\PYG{o}{+}\PYG{n}{rvsp}\PYG{p}{)}\PYG{p}{,} \PYG{n}{bins}\PYG{o}{=}\PYG{l+m+mi}{50}\PYG{p}{,} \PYG{n}{normed}\PYG{o}{=}\PYG{n+nb+bp}{True}\PYG{p}{)}
\PYG{g+gp}{\PYGZgt{}\PYGZgt{}\PYGZgt{} }\PYG{n}{plt}\PYG{o}{.}\PYG{n}{plot}\PYG{p}{(}\PYG{n}{xx}\PYG{p}{,}\PYG{n}{powpdf}\PYG{p}{,}\PYG{l+s}{\PYGZsq{}}\PYG{l+s}{r\PYGZhy{}}\PYG{l+s}{\PYGZsq{}}\PYG{p}{)}
\PYG{g+gp}{\PYGZgt{}\PYGZgt{}\PYGZgt{} }\PYG{n}{plt}\PYG{o}{.}\PYG{n}{title}\PYG{p}{(}\PYG{l+s}{\PYGZsq{}}\PYG{l+s}{inverse of stats.pareto(5)}\PYG{l+s}{\PYGZsq{}}\PYG{p}{)}
\end{Verbatim}

\end{fulllineitems}

\index{rand() (in module acsDynStatInTime)}

\begin{fulllineitems}
\phantomsection\label{acsDynStatInTime:acsDynStatInTime.rand}\pysiglinewithargsret{\code{acsDynStatInTime.}\bfcode{rand}}{\emph{d0}, \emph{d1}, \emph{...}, \emph{dn}}{}
Random values in a given shape.

Create an array of the given shape and propagate it with
random samples from a uniform distribution
over \code{{[}0, 1)}.
\begin{description}
\item[{d0, d1, ..., dn}] \leavevmode{[}int, optional{]}
The dimensions of the returned array, should all be positive.
If no argument is given a single Python float is returned.

\end{description}
\begin{description}
\item[{out}] \leavevmode{[}ndarray, shape \code{(d0, d1, ..., dn)}{]}
Random values.

\end{description}

random

This is a convenience function. If you want an interface that
takes a shape-tuple as the first argument, refer to
np.random.random\_sample .

\begin{Verbatim}[commandchars=\\\{\}]
\PYG{g+gp}{\PYGZgt{}\PYGZgt{}\PYGZgt{} }\PYG{n}{np}\PYG{o}{.}\PYG{n}{random}\PYG{o}{.}\PYG{n}{rand}\PYG{p}{(}\PYG{l+m+mi}{3}\PYG{p}{,}\PYG{l+m+mi}{2}\PYG{p}{)}
\PYG{g+go}{array([[ 0.14022471,  0.96360618],  \PYGZsh{}random}
\PYG{g+go}{       [ 0.37601032,  0.25528411],  \PYGZsh{}random}
\PYG{g+go}{       [ 0.49313049,  0.94909878]]) \PYGZsh{}random}
\end{Verbatim}

\end{fulllineitems}

\index{randint() (in module acsDynStatInTime)}

\begin{fulllineitems}
\phantomsection\label{acsDynStatInTime:acsDynStatInTime.randint}\pysiglinewithargsret{\code{acsDynStatInTime.}\bfcode{randint}}{\emph{low}, \emph{high=None}, \emph{size=None}}{}
Return random integers from \emph{low} (inclusive) to \emph{high} (exclusive).

Return random integers from the ``discrete uniform'' distribution in the
``half-open'' interval {[}\emph{low}, \emph{high}). If \emph{high} is None (the default),
then results are from {[}0, \emph{low}).
\begin{description}
\item[{low}] \leavevmode{[}int{]}
Lowest (signed) integer to be drawn from the distribution (unless
\code{high=None}, in which case this parameter is the \emph{highest} such
integer).

\item[{high}] \leavevmode{[}int, optional{]}
If provided, one above the largest (signed) integer to be drawn
from the distribution (see above for behavior if \code{high=None}).

\item[{size}] \leavevmode{[}int or tuple of ints, optional{]}
Output shape. Default is None, in which case a single int is
returned.

\end{description}
\begin{description}
\item[{out}] \leavevmode{[}int or ndarray of ints{]}
\emph{size}-shaped array of random integers from the appropriate
distribution, or a single such random int if \emph{size} not provided.

\end{description}
\begin{description}
\item[{random.random\_integers}] \leavevmode{[}similar to \emph{randint}, only for the closed{]}
interval {[}\emph{low}, \emph{high}{]}, and 1 is the lowest value if \emph{high} is
omitted. In particular, this other one is the one to use to generate
uniformly distributed discrete non-integers.

\end{description}

\begin{Verbatim}[commandchars=\\\{\}]
\PYG{g+gp}{\PYGZgt{}\PYGZgt{}\PYGZgt{} }\PYG{n}{np}\PYG{o}{.}\PYG{n}{random}\PYG{o}{.}\PYG{n}{randint}\PYG{p}{(}\PYG{l+m+mi}{2}\PYG{p}{,} \PYG{n}{size}\PYG{o}{=}\PYG{l+m+mi}{10}\PYG{p}{)}
\PYG{g+go}{array([1, 0, 0, 0, 1, 1, 0, 0, 1, 0])}
\PYG{g+gp}{\PYGZgt{}\PYGZgt{}\PYGZgt{} }\PYG{n}{np}\PYG{o}{.}\PYG{n}{random}\PYG{o}{.}\PYG{n}{randint}\PYG{p}{(}\PYG{l+m+mi}{1}\PYG{p}{,} \PYG{n}{size}\PYG{o}{=}\PYG{l+m+mi}{10}\PYG{p}{)}
\PYG{g+go}{array([0, 0, 0, 0, 0, 0, 0, 0, 0, 0])}
\end{Verbatim}

Generate a 2 x 4 array of ints between 0 and 4, inclusive:

\begin{Verbatim}[commandchars=\\\{\}]
\PYG{g+gp}{\PYGZgt{}\PYGZgt{}\PYGZgt{} }\PYG{n}{np}\PYG{o}{.}\PYG{n}{random}\PYG{o}{.}\PYG{n}{randint}\PYG{p}{(}\PYG{l+m+mi}{5}\PYG{p}{,} \PYG{n}{size}\PYG{o}{=}\PYG{p}{(}\PYG{l+m+mi}{2}\PYG{p}{,} \PYG{l+m+mi}{4}\PYG{p}{)}\PYG{p}{)}
\PYG{g+go}{array([[4, 0, 2, 1],}
\PYG{g+go}{       [3, 2, 2, 0]])}
\end{Verbatim}

\end{fulllineitems}

\index{randn() (in module acsDynStatInTime)}

\begin{fulllineitems}
\phantomsection\label{acsDynStatInTime:acsDynStatInTime.randn}\pysiglinewithargsret{\code{acsDynStatInTime.}\bfcode{randn}}{\emph{d0}, \emph{d1}, \emph{...}, \emph{dn}}{}
Return a sample (or samples) from the ``standard normal'' distribution.

If positive, int\_like or int-convertible arguments are provided,
\emph{randn} generates an array of shape \code{(d0, d1, ..., dn)}, filled
with random floats sampled from a univariate ``normal'' (Gaussian)
distribution of mean 0 and variance 1 (if any of the \(d_i\) are
floats, they are first converted to integers by truncation). A single
float randomly sampled from the distribution is returned if no
argument is provided.

This is a convenience function.  If you want an interface that takes a
tuple as the first argument, use \emph{numpy.random.standard\_normal} instead.
\begin{description}
\item[{d0, d1, ..., dn}] \leavevmode{[}int, optional{]}
The dimensions of the returned array, should be all positive.
If no argument is given a single Python float is returned.

\end{description}
\begin{description}
\item[{Z}] \leavevmode{[}ndarray or float{]}
A \code{(d0, d1, ..., dn)}-shaped array of floating-point samples from
the standard normal distribution, or a single such float if
no parameters were supplied.

\end{description}

random.standard\_normal : Similar, but takes a tuple as its argument.

For random samples from \(N(\mu, \sigma^2)\), use:

\code{sigma * np.random.randn(...) + mu}

\begin{Verbatim}[commandchars=\\\{\}]
\PYG{g+gp}{\PYGZgt{}\PYGZgt{}\PYGZgt{} }\PYG{n}{np}\PYG{o}{.}\PYG{n}{random}\PYG{o}{.}\PYG{n}{randn}\PYG{p}{(}\PYG{p}{)}
\PYG{g+go}{2.1923875335537315 \PYGZsh{}random}
\end{Verbatim}

Two-by-four array of samples from N(3, 6.25):

\begin{Verbatim}[commandchars=\\\{\}]
\PYG{g+gp}{\PYGZgt{}\PYGZgt{}\PYGZgt{} }\PYG{l+m+mf}{2.5} \PYG{o}{*} \PYG{n}{np}\PYG{o}{.}\PYG{n}{random}\PYG{o}{.}\PYG{n}{randn}\PYG{p}{(}\PYG{l+m+mi}{2}\PYG{p}{,} \PYG{l+m+mi}{4}\PYG{p}{)} \PYG{o}{+} \PYG{l+m+mi}{3}
\PYG{g+go}{array([[\PYGZhy{}4.49401501,  4.00950034, \PYGZhy{}1.81814867,  7.29718677],  \PYGZsh{}random}
\PYG{g+go}{       [ 0.39924804,  4.68456316,  4.99394529,  4.84057254]]) \PYGZsh{}random}
\end{Verbatim}

\end{fulllineitems}

\index{random() (in module acsDynStatInTime)}

\begin{fulllineitems}
\phantomsection\label{acsDynStatInTime:acsDynStatInTime.random}\pysiglinewithargsret{\code{acsDynStatInTime.}\bfcode{random}}{}{}
random\_sample(size=None)

Return random floats in the half-open interval {[}0.0, 1.0).

Results are from the ``continuous uniform'' distribution over the
stated interval.  To sample \(Unif[a, b), b > a\) multiply
the output of \emph{random\_sample} by \emph{(b-a)} and add \emph{a}:

\begin{Verbatim}[commandchars=\\\{\}]
\PYG{p}{(}\PYG{n}{b} \PYG{o}{\PYGZhy{}} \PYG{n}{a}\PYG{p}{)} \PYG{o}{*} \PYG{n}{random\PYGZus{}sample}\PYG{p}{(}\PYG{p}{)} \PYG{o}{+} \PYG{n}{a}
\end{Verbatim}
\begin{description}
\item[{size}] \leavevmode{[}int or tuple of ints, optional{]}
Defines the shape of the returned array of random floats. If None
(the default), returns a single float.

\end{description}
\begin{description}
\item[{out}] \leavevmode{[}float or ndarray of floats{]}
Array of random floats of shape \emph{size} (unless \code{size=None}, in which
case a single float is returned).

\end{description}

\begin{Verbatim}[commandchars=\\\{\}]
\PYG{g+gp}{\PYGZgt{}\PYGZgt{}\PYGZgt{} }\PYG{n}{np}\PYG{o}{.}\PYG{n}{random}\PYG{o}{.}\PYG{n}{random\PYGZus{}sample}\PYG{p}{(}\PYG{p}{)}
\PYG{g+go}{0.47108547995356098}
\PYG{g+gp}{\PYGZgt{}\PYGZgt{}\PYGZgt{} }\PYG{n+nb}{type}\PYG{p}{(}\PYG{n}{np}\PYG{o}{.}\PYG{n}{random}\PYG{o}{.}\PYG{n}{random\PYGZus{}sample}\PYG{p}{(}\PYG{p}{)}\PYG{p}{)}
\PYG{g+go}{\PYGZlt{}type \PYGZsq{}float\PYGZsq{}\PYGZgt{}}
\PYG{g+gp}{\PYGZgt{}\PYGZgt{}\PYGZgt{} }\PYG{n}{np}\PYG{o}{.}\PYG{n}{random}\PYG{o}{.}\PYG{n}{random\PYGZus{}sample}\PYG{p}{(}\PYG{p}{(}\PYG{l+m+mi}{5}\PYG{p}{,}\PYG{p}{)}\PYG{p}{)}
\PYG{g+go}{array([ 0.30220482,  0.86820401,  0.1654503 ,  0.11659149,  0.54323428])}
\end{Verbatim}

Three-by-two array of random numbers from {[}-5, 0):

\begin{Verbatim}[commandchars=\\\{\}]
\PYG{g+gp}{\PYGZgt{}\PYGZgt{}\PYGZgt{} }\PYG{l+m+mi}{5} \PYG{o}{*} \PYG{n}{np}\PYG{o}{.}\PYG{n}{random}\PYG{o}{.}\PYG{n}{random\PYGZus{}sample}\PYG{p}{(}\PYG{p}{(}\PYG{l+m+mi}{3}\PYG{p}{,} \PYG{l+m+mi}{2}\PYG{p}{)}\PYG{p}{)} \PYG{o}{\PYGZhy{}} \PYG{l+m+mi}{5}
\PYG{g+go}{array([[\PYGZhy{}3.99149989, \PYGZhy{}0.52338984],}
\PYG{g+go}{       [\PYGZhy{}2.99091858, \PYGZhy{}0.79479508],}
\PYG{g+go}{       [\PYGZhy{}1.23204345, \PYGZhy{}1.75224494]])}
\end{Verbatim}

\end{fulllineitems}

\index{random\_integers() (in module acsDynStatInTime)}

\begin{fulllineitems}
\phantomsection\label{acsDynStatInTime:acsDynStatInTime.random_integers}\pysiglinewithargsret{\code{acsDynStatInTime.}\bfcode{random\_integers}}{\emph{low}, \emph{high=None}, \emph{size=None}}{}
Return random integers between \emph{low} and \emph{high}, inclusive.

Return random integers from the ``discrete uniform'' distribution in the
closed interval {[}\emph{low}, \emph{high}{]}.  If \emph{high} is None (the default),
then results are from {[}1, \emph{low}{]}.
\begin{description}
\item[{low}] \leavevmode{[}int{]}
Lowest (signed) integer to be drawn from the distribution (unless
\code{high=None}, in which case this parameter is the \emph{highest} such
integer).

\item[{high}] \leavevmode{[}int, optional{]}
If provided, the largest (signed) integer to be drawn from the
distribution (see above for behavior if \code{high=None}).

\item[{size}] \leavevmode{[}int or tuple of ints, optional{]}
Output shape. Default is None, in which case a single int is returned.

\end{description}
\begin{description}
\item[{out}] \leavevmode{[}int or ndarray of ints{]}
\emph{size}-shaped array of random integers from the appropriate
distribution, or a single such random int if \emph{size} not provided.

\end{description}
\begin{description}
\item[{random.randint}] \leavevmode{[}Similar to \emph{random\_integers}, only for the half-open{]}
interval {[}\emph{low}, \emph{high}), and 0 is the lowest value if \emph{high} is
omitted.

\end{description}

To sample from N evenly spaced floating-point numbers between a and b,
use:

\begin{Verbatim}[commandchars=\\\{\}]
\PYG{n}{a} \PYG{o}{+} \PYG{p}{(}\PYG{n}{b} \PYG{o}{\PYGZhy{}} \PYG{n}{a}\PYG{p}{)} \PYG{o}{*} \PYG{p}{(}\PYG{n}{np}\PYG{o}{.}\PYG{n}{random}\PYG{o}{.}\PYG{n}{random\PYGZus{}integers}\PYG{p}{(}\PYG{n}{N}\PYG{p}{)} \PYG{o}{\PYGZhy{}} \PYG{l+m+mi}{1}\PYG{p}{)} \PYG{o}{/} \PYG{p}{(}\PYG{n}{N} \PYG{o}{\PYGZhy{}} \PYG{l+m+mf}{1.}\PYG{p}{)}
\end{Verbatim}

\begin{Verbatim}[commandchars=\\\{\}]
\PYG{g+gp}{\PYGZgt{}\PYGZgt{}\PYGZgt{} }\PYG{n}{np}\PYG{o}{.}\PYG{n}{random}\PYG{o}{.}\PYG{n}{random\PYGZus{}integers}\PYG{p}{(}\PYG{l+m+mi}{5}\PYG{p}{)}
\PYG{g+go}{4}
\PYG{g+gp}{\PYGZgt{}\PYGZgt{}\PYGZgt{} }\PYG{n+nb}{type}\PYG{p}{(}\PYG{n}{np}\PYG{o}{.}\PYG{n}{random}\PYG{o}{.}\PYG{n}{random\PYGZus{}integers}\PYG{p}{(}\PYG{l+m+mi}{5}\PYG{p}{)}\PYG{p}{)}
\PYG{g+go}{\PYGZlt{}type \PYGZsq{}int\PYGZsq{}\PYGZgt{}}
\PYG{g+gp}{\PYGZgt{}\PYGZgt{}\PYGZgt{} }\PYG{n}{np}\PYG{o}{.}\PYG{n}{random}\PYG{o}{.}\PYG{n}{random\PYGZus{}integers}\PYG{p}{(}\PYG{l+m+mi}{5}\PYG{p}{,} \PYG{n}{size}\PYG{o}{=}\PYG{p}{(}\PYG{l+m+mf}{3.}\PYG{p}{,}\PYG{l+m+mf}{2.}\PYG{p}{)}\PYG{p}{)}
\PYG{g+go}{array([[5, 4],}
\PYG{g+go}{       [3, 3],}
\PYG{g+go}{       [4, 5]])}
\end{Verbatim}

Choose five random numbers from the set of five evenly-spaced
numbers between 0 and 2.5, inclusive (\emph{i.e.}, from the set
\({0, 5/8, 10/8, 15/8, 20/8}\)):

\begin{Verbatim}[commandchars=\\\{\}]
\PYG{g+gp}{\PYGZgt{}\PYGZgt{}\PYGZgt{} }\PYG{l+m+mf}{2.5} \PYG{o}{*} \PYG{p}{(}\PYG{n}{np}\PYG{o}{.}\PYG{n}{random}\PYG{o}{.}\PYG{n}{random\PYGZus{}integers}\PYG{p}{(}\PYG{l+m+mi}{5}\PYG{p}{,} \PYG{n}{size}\PYG{o}{=}\PYG{p}{(}\PYG{l+m+mi}{5}\PYG{p}{,}\PYG{p}{)}\PYG{p}{)} \PYG{o}{\PYGZhy{}} \PYG{l+m+mi}{1}\PYG{p}{)} \PYG{o}{/} \PYG{l+m+mf}{4.}
\PYG{g+go}{array([ 0.625,  1.25 ,  0.625,  0.625,  2.5  ])}
\end{Verbatim}

Roll two six sided dice 1000 times and sum the results:

\begin{Verbatim}[commandchars=\\\{\}]
\PYG{g+gp}{\PYGZgt{}\PYGZgt{}\PYGZgt{} }\PYG{n}{d1} \PYG{o}{=} \PYG{n}{np}\PYG{o}{.}\PYG{n}{random}\PYG{o}{.}\PYG{n}{random\PYGZus{}integers}\PYG{p}{(}\PYG{l+m+mi}{1}\PYG{p}{,} \PYG{l+m+mi}{6}\PYG{p}{,} \PYG{l+m+mi}{1000}\PYG{p}{)}
\PYG{g+gp}{\PYGZgt{}\PYGZgt{}\PYGZgt{} }\PYG{n}{d2} \PYG{o}{=} \PYG{n}{np}\PYG{o}{.}\PYG{n}{random}\PYG{o}{.}\PYG{n}{random\PYGZus{}integers}\PYG{p}{(}\PYG{l+m+mi}{1}\PYG{p}{,} \PYG{l+m+mi}{6}\PYG{p}{,} \PYG{l+m+mi}{1000}\PYG{p}{)}
\PYG{g+gp}{\PYGZgt{}\PYGZgt{}\PYGZgt{} }\PYG{n}{dsums} \PYG{o}{=} \PYG{n}{d1} \PYG{o}{+} \PYG{n}{d2}
\end{Verbatim}

Display results as a histogram:

\begin{Verbatim}[commandchars=\\\{\}]
\PYG{g+gp}{\PYGZgt{}\PYGZgt{}\PYGZgt{} }\PYG{k+kn}{import} \PYG{n+nn}{matplotlib.pyplot} \PYG{k+kn}{as} \PYG{n+nn}{plt}
\PYG{g+gp}{\PYGZgt{}\PYGZgt{}\PYGZgt{} }\PYG{n}{count}\PYG{p}{,} \PYG{n}{bins}\PYG{p}{,} \PYG{n}{ignored} \PYG{o}{=} \PYG{n}{plt}\PYG{o}{.}\PYG{n}{hist}\PYG{p}{(}\PYG{n}{dsums}\PYG{p}{,} \PYG{l+m+mi}{11}\PYG{p}{,} \PYG{n}{normed}\PYG{o}{=}\PYG{n+nb+bp}{True}\PYG{p}{)}
\PYG{g+gp}{\PYGZgt{}\PYGZgt{}\PYGZgt{} }\PYG{n}{plt}\PYG{o}{.}\PYG{n}{show}\PYG{p}{(}\PYG{p}{)}
\end{Verbatim}

\end{fulllineitems}

\index{random\_sample() (in module acsDynStatInTime)}

\begin{fulllineitems}
\phantomsection\label{acsDynStatInTime:acsDynStatInTime.random_sample}\pysiglinewithargsret{\code{acsDynStatInTime.}\bfcode{random\_sample}}{\emph{size=None}}{}
Return random floats in the half-open interval {[}0.0, 1.0).

Results are from the ``continuous uniform'' distribution over the
stated interval.  To sample \(Unif[a, b), b > a\) multiply
the output of \emph{random\_sample} by \emph{(b-a)} and add \emph{a}:

\begin{Verbatim}[commandchars=\\\{\}]
\PYG{p}{(}\PYG{n}{b} \PYG{o}{\PYGZhy{}} \PYG{n}{a}\PYG{p}{)} \PYG{o}{*} \PYG{n}{random\PYGZus{}sample}\PYG{p}{(}\PYG{p}{)} \PYG{o}{+} \PYG{n}{a}
\end{Verbatim}
\begin{description}
\item[{size}] \leavevmode{[}int or tuple of ints, optional{]}
Defines the shape of the returned array of random floats. If None
(the default), returns a single float.

\end{description}
\begin{description}
\item[{out}] \leavevmode{[}float or ndarray of floats{]}
Array of random floats of shape \emph{size} (unless \code{size=None}, in which
case a single float is returned).

\end{description}

\begin{Verbatim}[commandchars=\\\{\}]
\PYG{g+gp}{\PYGZgt{}\PYGZgt{}\PYGZgt{} }\PYG{n}{np}\PYG{o}{.}\PYG{n}{random}\PYG{o}{.}\PYG{n}{random\PYGZus{}sample}\PYG{p}{(}\PYG{p}{)}
\PYG{g+go}{0.47108547995356098}
\PYG{g+gp}{\PYGZgt{}\PYGZgt{}\PYGZgt{} }\PYG{n+nb}{type}\PYG{p}{(}\PYG{n}{np}\PYG{o}{.}\PYG{n}{random}\PYG{o}{.}\PYG{n}{random\PYGZus{}sample}\PYG{p}{(}\PYG{p}{)}\PYG{p}{)}
\PYG{g+go}{\PYGZlt{}type \PYGZsq{}float\PYGZsq{}\PYGZgt{}}
\PYG{g+gp}{\PYGZgt{}\PYGZgt{}\PYGZgt{} }\PYG{n}{np}\PYG{o}{.}\PYG{n}{random}\PYG{o}{.}\PYG{n}{random\PYGZus{}sample}\PYG{p}{(}\PYG{p}{(}\PYG{l+m+mi}{5}\PYG{p}{,}\PYG{p}{)}\PYG{p}{)}
\PYG{g+go}{array([ 0.30220482,  0.86820401,  0.1654503 ,  0.11659149,  0.54323428])}
\end{Verbatim}

Three-by-two array of random numbers from {[}-5, 0):

\begin{Verbatim}[commandchars=\\\{\}]
\PYG{g+gp}{\PYGZgt{}\PYGZgt{}\PYGZgt{} }\PYG{l+m+mi}{5} \PYG{o}{*} \PYG{n}{np}\PYG{o}{.}\PYG{n}{random}\PYG{o}{.}\PYG{n}{random\PYGZus{}sample}\PYG{p}{(}\PYG{p}{(}\PYG{l+m+mi}{3}\PYG{p}{,} \PYG{l+m+mi}{2}\PYG{p}{)}\PYG{p}{)} \PYG{o}{\PYGZhy{}} \PYG{l+m+mi}{5}
\PYG{g+go}{array([[\PYGZhy{}3.99149989, \PYGZhy{}0.52338984],}
\PYG{g+go}{       [\PYGZhy{}2.99091858, \PYGZhy{}0.79479508],}
\PYG{g+go}{       [\PYGZhy{}1.23204345, \PYGZhy{}1.75224494]])}
\end{Verbatim}

\end{fulllineitems}

\index{ranf() (in module acsDynStatInTime)}

\begin{fulllineitems}
\phantomsection\label{acsDynStatInTime:acsDynStatInTime.ranf}\pysiglinewithargsret{\code{acsDynStatInTime.}\bfcode{ranf}}{}{}
random\_sample(size=None)

Return random floats in the half-open interval {[}0.0, 1.0).

Results are from the ``continuous uniform'' distribution over the
stated interval.  To sample \(Unif[a, b), b > a\) multiply
the output of \emph{random\_sample} by \emph{(b-a)} and add \emph{a}:

\begin{Verbatim}[commandchars=\\\{\}]
\PYG{p}{(}\PYG{n}{b} \PYG{o}{\PYGZhy{}} \PYG{n}{a}\PYG{p}{)} \PYG{o}{*} \PYG{n}{random\PYGZus{}sample}\PYG{p}{(}\PYG{p}{)} \PYG{o}{+} \PYG{n}{a}
\end{Verbatim}
\begin{description}
\item[{size}] \leavevmode{[}int or tuple of ints, optional{]}
Defines the shape of the returned array of random floats. If None
(the default), returns a single float.

\end{description}
\begin{description}
\item[{out}] \leavevmode{[}float or ndarray of floats{]}
Array of random floats of shape \emph{size} (unless \code{size=None}, in which
case a single float is returned).

\end{description}

\begin{Verbatim}[commandchars=\\\{\}]
\PYG{g+gp}{\PYGZgt{}\PYGZgt{}\PYGZgt{} }\PYG{n}{np}\PYG{o}{.}\PYG{n}{random}\PYG{o}{.}\PYG{n}{random\PYGZus{}sample}\PYG{p}{(}\PYG{p}{)}
\PYG{g+go}{0.47108547995356098}
\PYG{g+gp}{\PYGZgt{}\PYGZgt{}\PYGZgt{} }\PYG{n+nb}{type}\PYG{p}{(}\PYG{n}{np}\PYG{o}{.}\PYG{n}{random}\PYG{o}{.}\PYG{n}{random\PYGZus{}sample}\PYG{p}{(}\PYG{p}{)}\PYG{p}{)}
\PYG{g+go}{\PYGZlt{}type \PYGZsq{}float\PYGZsq{}\PYGZgt{}}
\PYG{g+gp}{\PYGZgt{}\PYGZgt{}\PYGZgt{} }\PYG{n}{np}\PYG{o}{.}\PYG{n}{random}\PYG{o}{.}\PYG{n}{random\PYGZus{}sample}\PYG{p}{(}\PYG{p}{(}\PYG{l+m+mi}{5}\PYG{p}{,}\PYG{p}{)}\PYG{p}{)}
\PYG{g+go}{array([ 0.30220482,  0.86820401,  0.1654503 ,  0.11659149,  0.54323428])}
\end{Verbatim}

Three-by-two array of random numbers from {[}-5, 0):

\begin{Verbatim}[commandchars=\\\{\}]
\PYG{g+gp}{\PYGZgt{}\PYGZgt{}\PYGZgt{} }\PYG{l+m+mi}{5} \PYG{o}{*} \PYG{n}{np}\PYG{o}{.}\PYG{n}{random}\PYG{o}{.}\PYG{n}{random\PYGZus{}sample}\PYG{p}{(}\PYG{p}{(}\PYG{l+m+mi}{3}\PYG{p}{,} \PYG{l+m+mi}{2}\PYG{p}{)}\PYG{p}{)} \PYG{o}{\PYGZhy{}} \PYG{l+m+mi}{5}
\PYG{g+go}{array([[\PYGZhy{}3.99149989, \PYGZhy{}0.52338984],}
\PYG{g+go}{       [\PYGZhy{}2.99091858, \PYGZhy{}0.79479508],}
\PYG{g+go}{       [\PYGZhy{}1.23204345, \PYGZhy{}1.75224494]])}
\end{Verbatim}

\end{fulllineitems}

\index{rayleigh() (in module acsDynStatInTime)}

\begin{fulllineitems}
\phantomsection\label{acsDynStatInTime:acsDynStatInTime.rayleigh}\pysiglinewithargsret{\code{acsDynStatInTime.}\bfcode{rayleigh}}{\emph{scale=1.0}, \emph{size=None}}{}
Draw samples from a Rayleigh distribution.

The \(\chi\) and Weibull distributions are generalizations of the
Rayleigh.
\begin{description}
\item[{scale}] \leavevmode{[}scalar{]}
Scale, also equals the mode. Should be \textgreater{}= 0.

\item[{size}] \leavevmode{[}int or tuple of ints, optional{]}
Shape of the output. Default is None, in which case a single
value is returned.

\end{description}

The probability density function for the Rayleigh distribution is
\begin{gather}
\begin{split}P(x;scale) = \frac{x}{scale^2}e^{\frac{-x^2}{2 \cdotp scale^2}}\end{split}\notag
\end{gather}
The Rayleigh distribution arises if the wind speed and wind direction are
both gaussian variables, then the vector wind velocity forms a Rayleigh
distribution. The Rayleigh distribution is used to model the expected
output from wind turbines.

Draw values from the distribution and plot the histogram

\begin{Verbatim}[commandchars=\\\{\}]
\PYG{g+gp}{\PYGZgt{}\PYGZgt{}\PYGZgt{} }\PYG{n}{values} \PYG{o}{=} \PYG{n}{hist}\PYG{p}{(}\PYG{n}{np}\PYG{o}{.}\PYG{n}{random}\PYG{o}{.}\PYG{n}{rayleigh}\PYG{p}{(}\PYG{l+m+mi}{3}\PYG{p}{,} \PYG{l+m+mi}{100000}\PYG{p}{)}\PYG{p}{,} \PYG{n}{bins}\PYG{o}{=}\PYG{l+m+mi}{200}\PYG{p}{,} \PYG{n}{normed}\PYG{o}{=}\PYG{n+nb+bp}{True}\PYG{p}{)}
\end{Verbatim}

Wave heights tend to follow a Rayleigh distribution. If the mean wave
height is 1 meter, what fraction of waves are likely to be larger than 3
meters?

\begin{Verbatim}[commandchars=\\\{\}]
\PYG{g+gp}{\PYGZgt{}\PYGZgt{}\PYGZgt{} }\PYG{n}{meanvalue} \PYG{o}{=} \PYG{l+m+mi}{1}
\PYG{g+gp}{\PYGZgt{}\PYGZgt{}\PYGZgt{} }\PYG{n}{modevalue} \PYG{o}{=} \PYG{n}{np}\PYG{o}{.}\PYG{n}{sqrt}\PYG{p}{(}\PYG{l+m+mi}{2} \PYG{o}{/} \PYG{n}{np}\PYG{o}{.}\PYG{n}{pi}\PYG{p}{)} \PYG{o}{*} \PYG{n}{meanvalue}
\PYG{g+gp}{\PYGZgt{}\PYGZgt{}\PYGZgt{} }\PYG{n}{s} \PYG{o}{=} \PYG{n}{np}\PYG{o}{.}\PYG{n}{random}\PYG{o}{.}\PYG{n}{rayleigh}\PYG{p}{(}\PYG{n}{modevalue}\PYG{p}{,} \PYG{l+m+mi}{1000000}\PYG{p}{)}
\end{Verbatim}

The percentage of waves larger than 3 meters is:

\begin{Verbatim}[commandchars=\\\{\}]
\PYG{g+gp}{\PYGZgt{}\PYGZgt{}\PYGZgt{} }\PYG{l+m+mf}{100.}\PYG{o}{*}\PYG{n+nb}{sum}\PYG{p}{(}\PYG{n}{s}\PYG{o}{\PYGZgt{}}\PYG{l+m+mi}{3}\PYG{p}{)}\PYG{o}{/}\PYG{l+m+mf}{1000000.}
\PYG{g+go}{0.087300000000000003}
\end{Verbatim}

\end{fulllineitems}

\index{sample() (in module acsDynStatInTime)}

\begin{fulllineitems}
\phantomsection\label{acsDynStatInTime:acsDynStatInTime.sample}\pysiglinewithargsret{\code{acsDynStatInTime.}\bfcode{sample}}{}{}
random\_sample(size=None)

Return random floats in the half-open interval {[}0.0, 1.0).

Results are from the ``continuous uniform'' distribution over the
stated interval.  To sample \(Unif[a, b), b > a\) multiply
the output of \emph{random\_sample} by \emph{(b-a)} and add \emph{a}:

\begin{Verbatim}[commandchars=\\\{\}]
\PYG{p}{(}\PYG{n}{b} \PYG{o}{\PYGZhy{}} \PYG{n}{a}\PYG{p}{)} \PYG{o}{*} \PYG{n}{random\PYGZus{}sample}\PYG{p}{(}\PYG{p}{)} \PYG{o}{+} \PYG{n}{a}
\end{Verbatim}
\begin{description}
\item[{size}] \leavevmode{[}int or tuple of ints, optional{]}
Defines the shape of the returned array of random floats. If None
(the default), returns a single float.

\end{description}
\begin{description}
\item[{out}] \leavevmode{[}float or ndarray of floats{]}
Array of random floats of shape \emph{size} (unless \code{size=None}, in which
case a single float is returned).

\end{description}

\begin{Verbatim}[commandchars=\\\{\}]
\PYG{g+gp}{\PYGZgt{}\PYGZgt{}\PYGZgt{} }\PYG{n}{np}\PYG{o}{.}\PYG{n}{random}\PYG{o}{.}\PYG{n}{random\PYGZus{}sample}\PYG{p}{(}\PYG{p}{)}
\PYG{g+go}{0.47108547995356098}
\PYG{g+gp}{\PYGZgt{}\PYGZgt{}\PYGZgt{} }\PYG{n+nb}{type}\PYG{p}{(}\PYG{n}{np}\PYG{o}{.}\PYG{n}{random}\PYG{o}{.}\PYG{n}{random\PYGZus{}sample}\PYG{p}{(}\PYG{p}{)}\PYG{p}{)}
\PYG{g+go}{\PYGZlt{}type \PYGZsq{}float\PYGZsq{}\PYGZgt{}}
\PYG{g+gp}{\PYGZgt{}\PYGZgt{}\PYGZgt{} }\PYG{n}{np}\PYG{o}{.}\PYG{n}{random}\PYG{o}{.}\PYG{n}{random\PYGZus{}sample}\PYG{p}{(}\PYG{p}{(}\PYG{l+m+mi}{5}\PYG{p}{,}\PYG{p}{)}\PYG{p}{)}
\PYG{g+go}{array([ 0.30220482,  0.86820401,  0.1654503 ,  0.11659149,  0.54323428])}
\end{Verbatim}

Three-by-two array of random numbers from {[}-5, 0):

\begin{Verbatim}[commandchars=\\\{\}]
\PYG{g+gp}{\PYGZgt{}\PYGZgt{}\PYGZgt{} }\PYG{l+m+mi}{5} \PYG{o}{*} \PYG{n}{np}\PYG{o}{.}\PYG{n}{random}\PYG{o}{.}\PYG{n}{random\PYGZus{}sample}\PYG{p}{(}\PYG{p}{(}\PYG{l+m+mi}{3}\PYG{p}{,} \PYG{l+m+mi}{2}\PYG{p}{)}\PYG{p}{)} \PYG{o}{\PYGZhy{}} \PYG{l+m+mi}{5}
\PYG{g+go}{array([[\PYGZhy{}3.99149989, \PYGZhy{}0.52338984],}
\PYG{g+go}{       [\PYGZhy{}2.99091858, \PYGZhy{}0.79479508],}
\PYG{g+go}{       [\PYGZhy{}1.23204345, \PYGZhy{}1.75224494]])}
\end{Verbatim}

\end{fulllineitems}

\index{seed() (in module acsDynStatInTime)}

\begin{fulllineitems}
\phantomsection\label{acsDynStatInTime:acsDynStatInTime.seed}\pysiglinewithargsret{\code{acsDynStatInTime.}\bfcode{seed}}{\emph{seed=None}}{}
Seed the generator.

This method is called when \emph{RandomState} is initialized. It can be
called again to re-seed the generator. For details, see \emph{RandomState}.
\begin{description}
\item[{seed}] \leavevmode{[}int or array\_like, optional{]}
Seed for \emph{RandomState}.

\end{description}

RandomState

\end{fulllineitems}

\index{set\_state() (in module acsDynStatInTime)}

\begin{fulllineitems}
\phantomsection\label{acsDynStatInTime:acsDynStatInTime.set_state}\pysiglinewithargsret{\code{acsDynStatInTime.}\bfcode{set\_state}}{\emph{state}}{}
Set the internal state of the generator from a tuple.

For use if one has reason to manually (re-)set the internal state of the
``Mersenne Twister''{\color{red}\bfseries{}{[}1{]}\_} pseudo-random number generating algorithm.
\begin{description}
\item[{state}] \leavevmode{[}tuple(str, ndarray of 624 uints, int, int, float){]}
The \emph{state} tuple has the following items:
\begin{enumerate}
\item {} 
the string `MT19937', specifying the Mersenne Twister algorithm.

\item {} 
a 1-D array of 624 unsigned integers \code{keys}.

\item {} 
an integer \code{pos}.

\item {} 
an integer \code{has\_gauss}.

\item {} 
a float \code{cached\_gaussian}.

\end{enumerate}

\end{description}
\begin{description}
\item[{out}] \leavevmode{[}None{]}
Returns `None' on success.

\end{description}

get\_state

\emph{set\_state} and \emph{get\_state} are not needed to work with any of the
random distributions in NumPy. If the internal state is manually altered,
the user should know exactly what he/she is doing.

For backwards compatibility, the form (str, array of 624 uints, int) is
also accepted although it is missing some information about the cached
Gaussian value: \code{state = ('MT19937', keys, pos)}.

\end{fulllineitems}

\index{shuffle() (in module acsDynStatInTime)}

\begin{fulllineitems}
\phantomsection\label{acsDynStatInTime:acsDynStatInTime.shuffle}\pysiglinewithargsret{\code{acsDynStatInTime.}\bfcode{shuffle}}{\emph{x}}{}
Modify a sequence in-place by shuffling its contents.
\begin{description}
\item[{x}] \leavevmode{[}array\_like{]}
The array or list to be shuffled.

\end{description}

None

\begin{Verbatim}[commandchars=\\\{\}]
\PYG{g+gp}{\PYGZgt{}\PYGZgt{}\PYGZgt{} }\PYG{n}{arr} \PYG{o}{=} \PYG{n}{np}\PYG{o}{.}\PYG{n}{arange}\PYG{p}{(}\PYG{l+m+mi}{10}\PYG{p}{)}
\PYG{g+gp}{\PYGZgt{}\PYGZgt{}\PYGZgt{} }\PYG{n}{np}\PYG{o}{.}\PYG{n}{random}\PYG{o}{.}\PYG{n}{shuffle}\PYG{p}{(}\PYG{n}{arr}\PYG{p}{)}
\PYG{g+gp}{\PYGZgt{}\PYGZgt{}\PYGZgt{} }\PYG{n}{arr}
\PYG{g+go}{[1 7 5 2 9 4 3 6 0 8]}
\end{Verbatim}

This function only shuffles the array along the first index of a
multi-dimensional array:

\begin{Verbatim}[commandchars=\\\{\}]
\PYG{g+gp}{\PYGZgt{}\PYGZgt{}\PYGZgt{} }\PYG{n}{arr} \PYG{o}{=} \PYG{n}{np}\PYG{o}{.}\PYG{n}{arange}\PYG{p}{(}\PYG{l+m+mi}{9}\PYG{p}{)}\PYG{o}{.}\PYG{n}{reshape}\PYG{p}{(}\PYG{p}{(}\PYG{l+m+mi}{3}\PYG{p}{,} \PYG{l+m+mi}{3}\PYG{p}{)}\PYG{p}{)}
\PYG{g+gp}{\PYGZgt{}\PYGZgt{}\PYGZgt{} }\PYG{n}{np}\PYG{o}{.}\PYG{n}{random}\PYG{o}{.}\PYG{n}{shuffle}\PYG{p}{(}\PYG{n}{arr}\PYG{p}{)}
\PYG{g+gp}{\PYGZgt{}\PYGZgt{}\PYGZgt{} }\PYG{n}{arr}
\PYG{g+go}{array([[3, 4, 5],}
\PYG{g+go}{       [6, 7, 8],}
\PYG{g+go}{       [0, 1, 2]])}
\end{Verbatim}

\end{fulllineitems}

\index{standard\_cauchy() (in module acsDynStatInTime)}

\begin{fulllineitems}
\phantomsection\label{acsDynStatInTime:acsDynStatInTime.standard_cauchy}\pysiglinewithargsret{\code{acsDynStatInTime.}\bfcode{standard\_cauchy}}{\emph{size=None}}{}
Standard Cauchy distribution with mode = 0.

Also known as the Lorentz distribution.
\begin{description}
\item[{size}] \leavevmode{[}int or tuple of ints{]}
Shape of the output.

\end{description}
\begin{description}
\item[{samples}] \leavevmode{[}ndarray or scalar{]}
The drawn samples.

\end{description}

The probability density function for the full Cauchy distribution is
\begin{gather}
\begin{split}P(x; x_0, \gamma) = \frac{1}{\pi \gamma \bigl[ 1+
(\frac{x-x_0}{\gamma})^2 \bigr] }\end{split}\notag
\end{gather}
and the Standard Cauchy distribution just sets \(x_0=0\) and
\(\gamma=1\)

The Cauchy distribution arises in the solution to the driven harmonic
oscillator problem, and also describes spectral line broadening. It
also describes the distribution of values at which a line tilted at
a random angle will cut the x axis.

When studying hypothesis tests that assume normality, seeing how the
tests perform on data from a Cauchy distribution is a good indicator of
their sensitivity to a heavy-tailed distribution, since the Cauchy looks
very much like a Gaussian distribution, but with heavier tails.

Draw samples and plot the distribution:

\begin{Verbatim}[commandchars=\\\{\}]
\PYG{g+gp}{\PYGZgt{}\PYGZgt{}\PYGZgt{} }\PYG{n}{s} \PYG{o}{=} \PYG{n}{np}\PYG{o}{.}\PYG{n}{random}\PYG{o}{.}\PYG{n}{standard\PYGZus{}cauchy}\PYG{p}{(}\PYG{l+m+mi}{1000000}\PYG{p}{)}
\PYG{g+gp}{\PYGZgt{}\PYGZgt{}\PYGZgt{} }\PYG{n}{s} \PYG{o}{=} \PYG{n}{s}\PYG{p}{[}\PYG{p}{(}\PYG{n}{s}\PYG{o}{\PYGZgt{}}\PYG{o}{\PYGZhy{}}\PYG{l+m+mi}{25}\PYG{p}{)} \PYG{o}{\PYGZam{}} \PYG{p}{(}\PYG{n}{s}\PYG{o}{\PYGZlt{}}\PYG{l+m+mi}{25}\PYG{p}{)}\PYG{p}{]}  \PYG{c}{\PYGZsh{} truncate distribution so it plots well}
\PYG{g+gp}{\PYGZgt{}\PYGZgt{}\PYGZgt{} }\PYG{n}{plt}\PYG{o}{.}\PYG{n}{hist}\PYG{p}{(}\PYG{n}{s}\PYG{p}{,} \PYG{n}{bins}\PYG{o}{=}\PYG{l+m+mi}{100}\PYG{p}{)}
\PYG{g+gp}{\PYGZgt{}\PYGZgt{}\PYGZgt{} }\PYG{n}{plt}\PYG{o}{.}\PYG{n}{show}\PYG{p}{(}\PYG{p}{)}
\end{Verbatim}

\end{fulllineitems}

\index{standard\_exponential() (in module acsDynStatInTime)}

\begin{fulllineitems}
\phantomsection\label{acsDynStatInTime:acsDynStatInTime.standard_exponential}\pysiglinewithargsret{\code{acsDynStatInTime.}\bfcode{standard\_exponential}}{\emph{size=None}}{}
Draw samples from the standard exponential distribution.

\emph{standard\_exponential} is identical to the exponential distribution
with a scale parameter of 1.
\begin{description}
\item[{size}] \leavevmode{[}int or tuple of ints{]}
Shape of the output.

\end{description}
\begin{description}
\item[{out}] \leavevmode{[}float or ndarray{]}
Drawn samples.

\end{description}

Output a 3x8000 array:

\begin{Verbatim}[commandchars=\\\{\}]
\PYG{g+gp}{\PYGZgt{}\PYGZgt{}\PYGZgt{} }\PYG{n}{n} \PYG{o}{=} \PYG{n}{np}\PYG{o}{.}\PYG{n}{random}\PYG{o}{.}\PYG{n}{standard\PYGZus{}exponential}\PYG{p}{(}\PYG{p}{(}\PYG{l+m+mi}{3}\PYG{p}{,} \PYG{l+m+mi}{8000}\PYG{p}{)}\PYG{p}{)}
\end{Verbatim}

\end{fulllineitems}

\index{standard\_gamma() (in module acsDynStatInTime)}

\begin{fulllineitems}
\phantomsection\label{acsDynStatInTime:acsDynStatInTime.standard_gamma}\pysiglinewithargsret{\code{acsDynStatInTime.}\bfcode{standard\_gamma}}{\emph{shape}, \emph{size=None}}{}
Draw samples from a Standard Gamma distribution.

Samples are drawn from a Gamma distribution with specified parameters,
shape (sometimes designated ``k'') and scale=1.
\begin{description}
\item[{shape}] \leavevmode{[}float{]}
Parameter, should be \textgreater{} 0.

\item[{size}] \leavevmode{[}int or tuple of ints{]}
Output shape.  If the given shape is, e.g., \code{(m, n, k)}, then
\code{m * n * k} samples are drawn.

\end{description}
\begin{description}
\item[{samples}] \leavevmode{[}ndarray or scalar{]}
The drawn samples.

\end{description}
\begin{description}
\item[{scipy.stats.distributions.gamma}] \leavevmode{[}probability density function,{]}
distribution or cumulative density function, etc.

\end{description}

The probability density for the Gamma distribution is
\begin{gather}
\begin{split}p(x) = x^{k-1}\frac{e^{-x/\theta}}{\theta^k\Gamma(k)},\end{split}\notag
\end{gather}
where \(k\) is the shape and \(\theta\) the scale,
and \(\Gamma\) is the Gamma function.

The Gamma distribution is often used to model the times to failure of
electronic components, and arises naturally in processes for which the
waiting times between Poisson distributed events are relevant.

Draw samples from the distribution:

\begin{Verbatim}[commandchars=\\\{\}]
\PYG{g+gp}{\PYGZgt{}\PYGZgt{}\PYGZgt{} }\PYG{n}{shape}\PYG{p}{,} \PYG{n}{scale} \PYG{o}{=} \PYG{l+m+mf}{2.}\PYG{p}{,} \PYG{l+m+mf}{1.} \PYG{c}{\PYGZsh{} mean and width}
\PYG{g+gp}{\PYGZgt{}\PYGZgt{}\PYGZgt{} }\PYG{n}{s} \PYG{o}{=} \PYG{n}{np}\PYG{o}{.}\PYG{n}{random}\PYG{o}{.}\PYG{n}{standard\PYGZus{}gamma}\PYG{p}{(}\PYG{n}{shape}\PYG{p}{,} \PYG{l+m+mi}{1000000}\PYG{p}{)}
\end{Verbatim}

Display the histogram of the samples, along with
the probability density function:

\begin{Verbatim}[commandchars=\\\{\}]
\PYG{g+gp}{\PYGZgt{}\PYGZgt{}\PYGZgt{} }\PYG{k+kn}{import} \PYG{n+nn}{matplotlib.pyplot} \PYG{k+kn}{as} \PYG{n+nn}{plt}
\PYG{g+gp}{\PYGZgt{}\PYGZgt{}\PYGZgt{} }\PYG{k+kn}{import} \PYG{n+nn}{scipy.special} \PYG{k+kn}{as} \PYG{n+nn}{sps}
\PYG{g+gp}{\PYGZgt{}\PYGZgt{}\PYGZgt{} }\PYG{n}{count}\PYG{p}{,} \PYG{n}{bins}\PYG{p}{,} \PYG{n}{ignored} \PYG{o}{=} \PYG{n}{plt}\PYG{o}{.}\PYG{n}{hist}\PYG{p}{(}\PYG{n}{s}\PYG{p}{,} \PYG{l+m+mi}{50}\PYG{p}{,} \PYG{n}{normed}\PYG{o}{=}\PYG{n+nb+bp}{True}\PYG{p}{)}
\PYG{g+gp}{\PYGZgt{}\PYGZgt{}\PYGZgt{} }\PYG{n}{y} \PYG{o}{=} \PYG{n}{bins}\PYG{o}{*}\PYG{o}{*}\PYG{p}{(}\PYG{n}{shape}\PYG{o}{\PYGZhy{}}\PYG{l+m+mi}{1}\PYG{p}{)} \PYG{o}{*} \PYG{p}{(}\PYG{p}{(}\PYG{n}{np}\PYG{o}{.}\PYG{n}{exp}\PYG{p}{(}\PYG{o}{\PYGZhy{}}\PYG{n}{bins}\PYG{o}{/}\PYG{n}{scale}\PYG{p}{)}\PYG{p}{)}\PYG{o}{/} \PYGZbs{}
\PYG{g+gp}{... }                      \PYG{p}{(}\PYG{n}{sps}\PYG{o}{.}\PYG{n}{gamma}\PYG{p}{(}\PYG{n}{shape}\PYG{p}{)} \PYG{o}{*} \PYG{n}{scale}\PYG{o}{*}\PYG{o}{*}\PYG{n}{shape}\PYG{p}{)}\PYG{p}{)}
\PYG{g+gp}{\PYGZgt{}\PYGZgt{}\PYGZgt{} }\PYG{n}{plt}\PYG{o}{.}\PYG{n}{plot}\PYG{p}{(}\PYG{n}{bins}\PYG{p}{,} \PYG{n}{y}\PYG{p}{,} \PYG{n}{linewidth}\PYG{o}{=}\PYG{l+m+mi}{2}\PYG{p}{,} \PYG{n}{color}\PYG{o}{=}\PYG{l+s}{\PYGZsq{}}\PYG{l+s}{r}\PYG{l+s}{\PYGZsq{}}\PYG{p}{)}
\PYG{g+gp}{\PYGZgt{}\PYGZgt{}\PYGZgt{} }\PYG{n}{plt}\PYG{o}{.}\PYG{n}{show}\PYG{p}{(}\PYG{p}{)}
\end{Verbatim}

\end{fulllineitems}

\index{standard\_normal() (in module acsDynStatInTime)}

\begin{fulllineitems}
\phantomsection\label{acsDynStatInTime:acsDynStatInTime.standard_normal}\pysiglinewithargsret{\code{acsDynStatInTime.}\bfcode{standard\_normal}}{\emph{size=None}}{}
Returns samples from a Standard Normal distribution (mean=0, stdev=1).
\begin{description}
\item[{size}] \leavevmode{[}int or tuple of ints, optional{]}
Output shape. Default is None, in which case a single value is
returned.

\end{description}
\begin{description}
\item[{out}] \leavevmode{[}float or ndarray{]}
Drawn samples.

\end{description}

\begin{Verbatim}[commandchars=\\\{\}]
\PYG{g+gp}{\PYGZgt{}\PYGZgt{}\PYGZgt{} }\PYG{n}{s} \PYG{o}{=} \PYG{n}{np}\PYG{o}{.}\PYG{n}{random}\PYG{o}{.}\PYG{n}{standard\PYGZus{}normal}\PYG{p}{(}\PYG{l+m+mi}{8000}\PYG{p}{)}
\PYG{g+gp}{\PYGZgt{}\PYGZgt{}\PYGZgt{} }\PYG{n}{s}
\PYG{g+go}{array([ 0.6888893 ,  0.78096262, \PYGZhy{}0.89086505, ...,  0.49876311, \PYGZsh{}random}
\PYG{g+go}{       \PYGZhy{}0.38672696, \PYGZhy{}0.4685006 ])                               \PYGZsh{}random}
\PYG{g+gp}{\PYGZgt{}\PYGZgt{}\PYGZgt{} }\PYG{n}{s}\PYG{o}{.}\PYG{n}{shape}
\PYG{g+go}{(8000,)}
\PYG{g+gp}{\PYGZgt{}\PYGZgt{}\PYGZgt{} }\PYG{n}{s} \PYG{o}{=} \PYG{n}{np}\PYG{o}{.}\PYG{n}{random}\PYG{o}{.}\PYG{n}{standard\PYGZus{}normal}\PYG{p}{(}\PYG{n}{size}\PYG{o}{=}\PYG{p}{(}\PYG{l+m+mi}{3}\PYG{p}{,} \PYG{l+m+mi}{4}\PYG{p}{,} \PYG{l+m+mi}{2}\PYG{p}{)}\PYG{p}{)}
\PYG{g+gp}{\PYGZgt{}\PYGZgt{}\PYGZgt{} }\PYG{n}{s}\PYG{o}{.}\PYG{n}{shape}
\PYG{g+go}{(3, 4, 2)}
\end{Verbatim}

\end{fulllineitems}

\index{standard\_t() (in module acsDynStatInTime)}

\begin{fulllineitems}
\phantomsection\label{acsDynStatInTime:acsDynStatInTime.standard_t}\pysiglinewithargsret{\code{acsDynStatInTime.}\bfcode{standard\_t}}{\emph{df}, \emph{size=None}}{}
Standard Student's t distribution with df degrees of freedom.

A special case of the hyperbolic distribution.
As \emph{df} gets large, the result resembles that of the standard normal
distribution (\emph{standard\_normal}).
\begin{description}
\item[{df}] \leavevmode{[}int{]}
Degrees of freedom, should be \textgreater{} 0.

\item[{size}] \leavevmode{[}int or tuple of ints, optional{]}
Output shape. Default is None, in which case a single value is
returned.

\end{description}
\begin{description}
\item[{samples}] \leavevmode{[}ndarray or scalar{]}
Drawn samples.

\end{description}

The probability density function for the t distribution is
\begin{gather}
\begin{split}P(x, df) = \frac{\Gamma(\frac{df+1}{2})}{\sqrt{\pi df}
\Gamma(\frac{df}{2})}\Bigl( 1+\frac{x^2}{df} \Bigr)^{-(df+1)/2}\end{split}\notag
\end{gather}
The t test is based on an assumption that the data come from a Normal
distribution. The t test provides a way to test whether the sample mean
(that is the mean calculated from the data) is a good estimate of the true
mean.

The derivation of the t-distribution was forst published in 1908 by William
Gisset while working for the Guinness Brewery in Dublin. Due to proprietary
issues, he had to publish under a pseudonym, and so he used the name
Student.

From Dalgaard page 83 {\color{red}\bfseries{}{[}1{]}\_}, suppose the daily energy intake for 11
women in Kj is:

\begin{Verbatim}[commandchars=\\\{\}]
\PYG{g+gp}{\PYGZgt{}\PYGZgt{}\PYGZgt{} }\PYG{n}{intake} \PYG{o}{=} \PYG{n}{np}\PYG{o}{.}\PYG{n}{array}\PYG{p}{(}\PYG{p}{[}\PYG{l+m+mf}{5260.}\PYG{p}{,} \PYG{l+m+mi}{5470}\PYG{p}{,} \PYG{l+m+mi}{5640}\PYG{p}{,} \PYG{l+m+mi}{6180}\PYG{p}{,} \PYG{l+m+mi}{6390}\PYG{p}{,} \PYG{l+m+mi}{6515}\PYG{p}{,} \PYG{l+m+mi}{6805}\PYG{p}{,} \PYG{l+m+mi}{7515}\PYG{p}{,} \PYGZbs{}
\PYG{g+gp}{... }                   \PYG{l+m+mi}{7515}\PYG{p}{,} \PYG{l+m+mi}{8230}\PYG{p}{,} \PYG{l+m+mi}{8770}\PYG{p}{]}\PYG{p}{)}
\end{Verbatim}

Does their energy intake deviate systematically from the recommended
value of 7725 kJ?

We have 10 degrees of freedom, so is the sample mean within 95\% of the
recommended value?

\begin{Verbatim}[commandchars=\\\{\}]
\PYG{g+gp}{\PYGZgt{}\PYGZgt{}\PYGZgt{} }\PYG{n}{s} \PYG{o}{=} \PYG{n}{np}\PYG{o}{.}\PYG{n}{random}\PYG{o}{.}\PYG{n}{standard\PYGZus{}t}\PYG{p}{(}\PYG{l+m+mi}{10}\PYG{p}{,} \PYG{n}{size}\PYG{o}{=}\PYG{l+m+mi}{100000}\PYG{p}{)}
\PYG{g+gp}{\PYGZgt{}\PYGZgt{}\PYGZgt{} }\PYG{n}{np}\PYG{o}{.}\PYG{n}{mean}\PYG{p}{(}\PYG{n}{intake}\PYG{p}{)}
\PYG{g+go}{6753.636363636364}
\PYG{g+gp}{\PYGZgt{}\PYGZgt{}\PYGZgt{} }\PYG{n}{intake}\PYG{o}{.}\PYG{n}{std}\PYG{p}{(}\PYG{n}{ddof}\PYG{o}{=}\PYG{l+m+mi}{1}\PYG{p}{)}
\PYG{g+go}{1142.1232221373727}
\end{Verbatim}

Calculate the t statistic, setting the ddof parameter to the unbiased
value so the divisor in the standard deviation will be degrees of
freedom, N-1.

\begin{Verbatim}[commandchars=\\\{\}]
\PYG{g+gp}{\PYGZgt{}\PYGZgt{}\PYGZgt{} }\PYG{n}{t} \PYG{o}{=} \PYG{p}{(}\PYG{n}{np}\PYG{o}{.}\PYG{n}{mean}\PYG{p}{(}\PYG{n}{intake}\PYG{p}{)}\PYG{o}{\PYGZhy{}}\PYG{l+m+mi}{7725}\PYG{p}{)}\PYG{o}{/}\PYG{p}{(}\PYG{n}{intake}\PYG{o}{.}\PYG{n}{std}\PYG{p}{(}\PYG{n}{ddof}\PYG{o}{=}\PYG{l+m+mi}{1}\PYG{p}{)}\PYG{o}{/}\PYG{n}{np}\PYG{o}{.}\PYG{n}{sqrt}\PYG{p}{(}\PYG{n+nb}{len}\PYG{p}{(}\PYG{n}{intake}\PYG{p}{)}\PYG{p}{)}\PYG{p}{)}
\PYG{g+gp}{\PYGZgt{}\PYGZgt{}\PYGZgt{} }\PYG{k+kn}{import} \PYG{n+nn}{matplotlib.pyplot} \PYG{k+kn}{as} \PYG{n+nn}{plt}
\PYG{g+gp}{\PYGZgt{}\PYGZgt{}\PYGZgt{} }\PYG{n}{h} \PYG{o}{=} \PYG{n}{plt}\PYG{o}{.}\PYG{n}{hist}\PYG{p}{(}\PYG{n}{s}\PYG{p}{,} \PYG{n}{bins}\PYG{o}{=}\PYG{l+m+mi}{100}\PYG{p}{,} \PYG{n}{normed}\PYG{o}{=}\PYG{n+nb+bp}{True}\PYG{p}{)}
\end{Verbatim}

For a one-sided t-test, how far out in the distribution does the t
statistic appear?

\begin{Verbatim}[commandchars=\\\{\}]
\PYG{g+gp}{\PYGZgt{}\PYGZgt{}\PYGZgt{} }\PYG{o}{\PYGZgt{}\PYGZgt{}}\PYG{o}{\PYGZgt{}} \PYG{n}{np}\PYG{o}{.}\PYG{n}{sum}\PYG{p}{(}\PYG{n}{s}\PYG{o}{\PYGZlt{}}\PYG{n}{t}\PYG{p}{)} \PYG{o}{/} \PYG{n+nb}{float}\PYG{p}{(}\PYG{n+nb}{len}\PYG{p}{(}\PYG{n}{s}\PYG{p}{)}\PYG{p}{)}
\PYG{g+go}{0.0090699999999999999  \PYGZsh{}random}
\end{Verbatim}

So the p-value is about 0.009, which says the null hypothesis has a
probability of about 99\% of being true.

\end{fulllineitems}

\index{triangular() (in module acsDynStatInTime)}

\begin{fulllineitems}
\phantomsection\label{acsDynStatInTime:acsDynStatInTime.triangular}\pysiglinewithargsret{\code{acsDynStatInTime.}\bfcode{triangular}}{\emph{left}, \emph{mode}, \emph{right}, \emph{size=None}}{}
Draw samples from the triangular distribution.

The triangular distribution is a continuous probability distribution with
lower limit left, peak at mode, and upper limit right. Unlike the other
distributions, these parameters directly define the shape of the pdf.
\begin{description}
\item[{left}] \leavevmode{[}scalar{]}
Lower limit.

\item[{mode}] \leavevmode{[}scalar{]}
The value where the peak of the distribution occurs.
The value should fulfill the condition \code{left \textless{}= mode \textless{}= right}.

\item[{right}] \leavevmode{[}scalar{]}
Upper limit, should be larger than \emph{left}.

\item[{size}] \leavevmode{[}int or tuple of ints, optional{]}
Output shape. Default is None, in which case a single value is
returned.

\end{description}
\begin{description}
\item[{samples}] \leavevmode{[}ndarray or scalar{]}
The returned samples all lie in the interval {[}left, right{]}.

\end{description}

The probability density function for the Triangular distribution is
\begin{gather}
\begin{split}P(x;l, m, r) = \begin{cases}
\frac{2(x-l)}{(r-l)(m-l)}& \text{for $l \leq x \leq m$},\\
\frac{2(m-x)}{(r-l)(r-m)}& \text{for $m \leq x \leq r$},\\
0& \text{otherwise}.
\end{cases}\end{split}\notag
\end{gather}
The triangular distribution is often used in ill-defined problems where the
underlying distribution is not known, but some knowledge of the limits and
mode exists. Often it is used in simulations.

Draw values from the distribution and plot the histogram:

\begin{Verbatim}[commandchars=\\\{\}]
\PYG{g+gp}{\PYGZgt{}\PYGZgt{}\PYGZgt{} }\PYG{k+kn}{import} \PYG{n+nn}{matplotlib.pyplot} \PYG{k+kn}{as} \PYG{n+nn}{plt}
\PYG{g+gp}{\PYGZgt{}\PYGZgt{}\PYGZgt{} }\PYG{n}{h} \PYG{o}{=} \PYG{n}{plt}\PYG{o}{.}\PYG{n}{hist}\PYG{p}{(}\PYG{n}{np}\PYG{o}{.}\PYG{n}{random}\PYG{o}{.}\PYG{n}{triangular}\PYG{p}{(}\PYG{o}{\PYGZhy{}}\PYG{l+m+mi}{3}\PYG{p}{,} \PYG{l+m+mi}{0}\PYG{p}{,} \PYG{l+m+mi}{8}\PYG{p}{,} \PYG{l+m+mi}{100000}\PYG{p}{)}\PYG{p}{,} \PYG{n}{bins}\PYG{o}{=}\PYG{l+m+mi}{200}\PYG{p}{,}
\PYG{g+gp}{... }             \PYG{n}{normed}\PYG{o}{=}\PYG{n+nb+bp}{True}\PYG{p}{)}
\PYG{g+gp}{\PYGZgt{}\PYGZgt{}\PYGZgt{} }\PYG{n}{plt}\PYG{o}{.}\PYG{n}{show}\PYG{p}{(}\PYG{p}{)}
\end{Verbatim}

\end{fulllineitems}

\index{uniform() (in module acsDynStatInTime)}

\begin{fulllineitems}
\phantomsection\label{acsDynStatInTime:acsDynStatInTime.uniform}\pysiglinewithargsret{\code{acsDynStatInTime.}\bfcode{uniform}}{\emph{low=0.0}, \emph{high=1.0}, \emph{size=1}}{}
Draw samples from a uniform distribution.

Samples are uniformly distributed over the half-open interval
\code{{[}low, high)} (includes low, but excludes high).  In other words,
any value within the given interval is equally likely to be drawn
by \emph{uniform}.
\begin{description}
\item[{low}] \leavevmode{[}float, optional{]}
Lower boundary of the output interval.  All values generated will be
greater than or equal to low.  The default value is 0.

\item[{high}] \leavevmode{[}float{]}
Upper boundary of the output interval.  All values generated will be
less than high.  The default value is 1.0.

\item[{size}] \leavevmode{[}int or tuple of ints, optional{]}
Shape of output.  If the given size is, for example, (m,n,k),
m*n*k samples are generated.  If no shape is specified, a single sample
is returned.

\end{description}
\begin{description}
\item[{out}] \leavevmode{[}ndarray{]}
Drawn samples, with shape \emph{size}.

\end{description}

randint : Discrete uniform distribution, yielding integers.
random\_integers : Discrete uniform distribution over the closed
\begin{quote}

interval \code{{[}low, high{]}}.
\end{quote}

random\_sample : Floats uniformly distributed over \code{{[}0, 1)}.
random : Alias for \emph{random\_sample}.
rand : Convenience function that accepts dimensions as input, e.g.,
\begin{quote}

\code{rand(2,2)} would generate a 2-by-2 array of floats,
uniformly distributed over \code{{[}0, 1)}.
\end{quote}

The probability density function of the uniform distribution is
\begin{gather}
\begin{split}p(x) = \frac{1}{b - a}\end{split}\notag
\end{gather}
anywhere within the interval \code{{[}a, b)}, and zero elsewhere.

Draw samples from the distribution:

\begin{Verbatim}[commandchars=\\\{\}]
\PYG{g+gp}{\PYGZgt{}\PYGZgt{}\PYGZgt{} }\PYG{n}{s} \PYG{o}{=} \PYG{n}{np}\PYG{o}{.}\PYG{n}{random}\PYG{o}{.}\PYG{n}{uniform}\PYG{p}{(}\PYG{o}{\PYGZhy{}}\PYG{l+m+mi}{1}\PYG{p}{,}\PYG{l+m+mi}{0}\PYG{p}{,}\PYG{l+m+mi}{1000}\PYG{p}{)}
\end{Verbatim}

All values are within the given interval:

\begin{Verbatim}[commandchars=\\\{\}]
\PYG{g+gp}{\PYGZgt{}\PYGZgt{}\PYGZgt{} }\PYG{n}{np}\PYG{o}{.}\PYG{n}{all}\PYG{p}{(}\PYG{n}{s} \PYG{o}{\PYGZgt{}}\PYG{o}{=} \PYG{o}{\PYGZhy{}}\PYG{l+m+mi}{1}\PYG{p}{)}
\PYG{g+go}{True}
\PYG{g+gp}{\PYGZgt{}\PYGZgt{}\PYGZgt{} }\PYG{n}{np}\PYG{o}{.}\PYG{n}{all}\PYG{p}{(}\PYG{n}{s} \PYG{o}{\PYGZlt{}} \PYG{l+m+mi}{0}\PYG{p}{)}
\PYG{g+go}{True}
\end{Verbatim}

Display the histogram of the samples, along with the
probability density function:

\begin{Verbatim}[commandchars=\\\{\}]
\PYG{g+gp}{\PYGZgt{}\PYGZgt{}\PYGZgt{} }\PYG{k+kn}{import} \PYG{n+nn}{matplotlib.pyplot} \PYG{k+kn}{as} \PYG{n+nn}{plt}
\PYG{g+gp}{\PYGZgt{}\PYGZgt{}\PYGZgt{} }\PYG{n}{count}\PYG{p}{,} \PYG{n}{bins}\PYG{p}{,} \PYG{n}{ignored} \PYG{o}{=} \PYG{n}{plt}\PYG{o}{.}\PYG{n}{hist}\PYG{p}{(}\PYG{n}{s}\PYG{p}{,} \PYG{l+m+mi}{15}\PYG{p}{,} \PYG{n}{normed}\PYG{o}{=}\PYG{n+nb+bp}{True}\PYG{p}{)}
\PYG{g+gp}{\PYGZgt{}\PYGZgt{}\PYGZgt{} }\PYG{n}{plt}\PYG{o}{.}\PYG{n}{plot}\PYG{p}{(}\PYG{n}{bins}\PYG{p}{,} \PYG{n}{np}\PYG{o}{.}\PYG{n}{ones\PYGZus{}like}\PYG{p}{(}\PYG{n}{bins}\PYG{p}{)}\PYG{p}{,} \PYG{n}{linewidth}\PYG{o}{=}\PYG{l+m+mi}{2}\PYG{p}{,} \PYG{n}{color}\PYG{o}{=}\PYG{l+s}{\PYGZsq{}}\PYG{l+s}{r}\PYG{l+s}{\PYGZsq{}}\PYG{p}{)}
\PYG{g+gp}{\PYGZgt{}\PYGZgt{}\PYGZgt{} }\PYG{n}{plt}\PYG{o}{.}\PYG{n}{show}\PYG{p}{(}\PYG{p}{)}
\end{Verbatim}

\end{fulllineitems}

\index{vonmises() (in module acsDynStatInTime)}

\begin{fulllineitems}
\phantomsection\label{acsDynStatInTime:acsDynStatInTime.vonmises}\pysiglinewithargsret{\code{acsDynStatInTime.}\bfcode{vonmises}}{\emph{mu}, \emph{kappa}, \emph{size=None}}{}
Draw samples from a von Mises distribution.

Samples are drawn from a von Mises distribution with specified mode
(mu) and dispersion (kappa), on the interval {[}-pi, pi{]}.

The von Mises distribution (also known as the circular normal
distribution) is a continuous probability distribution on the unit
circle.  It may be thought of as the circular analogue of the normal
distribution.
\begin{description}
\item[{mu}] \leavevmode{[}float{]}
Mode (``center'') of the distribution.

\item[{kappa}] \leavevmode{[}float{]}
Dispersion of the distribution, has to be \textgreater{}=0.

\item[{size}] \leavevmode{[}int or tuple of int{]}
Output shape.  If the given shape is, e.g., \code{(m, n, k)}, then
\code{m * n * k} samples are drawn.

\end{description}
\begin{description}
\item[{samples}] \leavevmode{[}scalar or ndarray{]}
The returned samples, which are in the interval {[}-pi, pi{]}.

\end{description}
\begin{description}
\item[{scipy.stats.distributions.vonmises}] \leavevmode{[}probability density function,{]}
distribution, or cumulative density function, etc.

\end{description}

The probability density for the von Mises distribution is
\begin{gather}
\begin{split}p(x) = \frac{e^{\kappa cos(x-\mu)}}{2\pi I_0(\kappa)},\end{split}\notag
\end{gather}
where \(\mu\) is the mode and \(\kappa\) the dispersion,
and \(I_0(\kappa)\) is the modified Bessel function of order 0.

The von Mises is named for Richard Edler von Mises, who was born in
Austria-Hungary, in what is now the Ukraine.  He fled to the United
States in 1939 and became a professor at Harvard.  He worked in
probability theory, aerodynamics, fluid mechanics, and philosophy of
science.

Abramowitz, M. and Stegun, I. A. (ed.), \emph{Handbook of Mathematical
Functions}, New York: Dover, 1965.

von Mises, R., \emph{Mathematical Theory of Probability and Statistics},
New York: Academic Press, 1964.

Draw samples from the distribution:

\begin{Verbatim}[commandchars=\\\{\}]
\PYG{g+gp}{\PYGZgt{}\PYGZgt{}\PYGZgt{} }\PYG{n}{mu}\PYG{p}{,} \PYG{n}{kappa} \PYG{o}{=} \PYG{l+m+mf}{0.0}\PYG{p}{,} \PYG{l+m+mf}{4.0} \PYG{c}{\PYGZsh{} mean and dispersion}
\PYG{g+gp}{\PYGZgt{}\PYGZgt{}\PYGZgt{} }\PYG{n}{s} \PYG{o}{=} \PYG{n}{np}\PYG{o}{.}\PYG{n}{random}\PYG{o}{.}\PYG{n}{vonmises}\PYG{p}{(}\PYG{n}{mu}\PYG{p}{,} \PYG{n}{kappa}\PYG{p}{,} \PYG{l+m+mi}{1000}\PYG{p}{)}
\end{Verbatim}

Display the histogram of the samples, along with
the probability density function:

\begin{Verbatim}[commandchars=\\\{\}]
\PYG{g+gp}{\PYGZgt{}\PYGZgt{}\PYGZgt{} }\PYG{k+kn}{import} \PYG{n+nn}{matplotlib.pyplot} \PYG{k+kn}{as} \PYG{n+nn}{plt}
\PYG{g+gp}{\PYGZgt{}\PYGZgt{}\PYGZgt{} }\PYG{k+kn}{import} \PYG{n+nn}{scipy.special} \PYG{k+kn}{as} \PYG{n+nn}{sps}
\PYG{g+gp}{\PYGZgt{}\PYGZgt{}\PYGZgt{} }\PYG{n}{count}\PYG{p}{,} \PYG{n}{bins}\PYG{p}{,} \PYG{n}{ignored} \PYG{o}{=} \PYG{n}{plt}\PYG{o}{.}\PYG{n}{hist}\PYG{p}{(}\PYG{n}{s}\PYG{p}{,} \PYG{l+m+mi}{50}\PYG{p}{,} \PYG{n}{normed}\PYG{o}{=}\PYG{n+nb+bp}{True}\PYG{p}{)}
\PYG{g+gp}{\PYGZgt{}\PYGZgt{}\PYGZgt{} }\PYG{n}{x} \PYG{o}{=} \PYG{n}{np}\PYG{o}{.}\PYG{n}{arange}\PYG{p}{(}\PYG{o}{\PYGZhy{}}\PYG{n}{np}\PYG{o}{.}\PYG{n}{pi}\PYG{p}{,} \PYG{n}{np}\PYG{o}{.}\PYG{n}{pi}\PYG{p}{,} \PYG{l+m+mi}{2}\PYG{o}{*}\PYG{n}{np}\PYG{o}{.}\PYG{n}{pi}\PYG{o}{/}\PYG{l+m+mf}{50.}\PYG{p}{)}
\PYG{g+gp}{\PYGZgt{}\PYGZgt{}\PYGZgt{} }\PYG{n}{y} \PYG{o}{=} \PYG{o}{\PYGZhy{}}\PYG{n}{np}\PYG{o}{.}\PYG{n}{exp}\PYG{p}{(}\PYG{n}{kappa}\PYG{o}{*}\PYG{n}{np}\PYG{o}{.}\PYG{n}{cos}\PYG{p}{(}\PYG{n}{x}\PYG{o}{\PYGZhy{}}\PYG{n}{mu}\PYG{p}{)}\PYG{p}{)}\PYG{o}{/}\PYG{p}{(}\PYG{l+m+mi}{2}\PYG{o}{*}\PYG{n}{np}\PYG{o}{.}\PYG{n}{pi}\PYG{o}{*}\PYG{n}{sps}\PYG{o}{.}\PYG{n}{jn}\PYG{p}{(}\PYG{l+m+mi}{0}\PYG{p}{,}\PYG{n}{kappa}\PYG{p}{)}\PYG{p}{)}
\PYG{g+gp}{\PYGZgt{}\PYGZgt{}\PYGZgt{} }\PYG{n}{plt}\PYG{o}{.}\PYG{n}{plot}\PYG{p}{(}\PYG{n}{x}\PYG{p}{,} \PYG{n}{y}\PYG{o}{/}\PYG{n+nb}{max}\PYG{p}{(}\PYG{n}{y}\PYG{p}{)}\PYG{p}{,} \PYG{n}{linewidth}\PYG{o}{=}\PYG{l+m+mi}{2}\PYG{p}{,} \PYG{n}{color}\PYG{o}{=}\PYG{l+s}{\PYGZsq{}}\PYG{l+s}{r}\PYG{l+s}{\PYGZsq{}}\PYG{p}{)}
\PYG{g+gp}{\PYGZgt{}\PYGZgt{}\PYGZgt{} }\PYG{n}{plt}\PYG{o}{.}\PYG{n}{show}\PYG{p}{(}\PYG{p}{)}
\end{Verbatim}

\end{fulllineitems}

\index{wald() (in module acsDynStatInTime)}

\begin{fulllineitems}
\phantomsection\label{acsDynStatInTime:acsDynStatInTime.wald}\pysiglinewithargsret{\code{acsDynStatInTime.}\bfcode{wald}}{\emph{mean}, \emph{scale}, \emph{size=None}}{}
Draw samples from a Wald, or Inverse Gaussian, distribution.

As the scale approaches infinity, the distribution becomes more like a
Gaussian.

Some references claim that the Wald is an Inverse Gaussian with mean=1, but
this is by no means universal.

The Inverse Gaussian distribution was first studied in relationship to
Brownian motion. In 1956 M.C.K. Tweedie used the name Inverse Gaussian
because there is an inverse relationship between the time to cover a unit
distance and distance covered in unit time.
\begin{description}
\item[{mean}] \leavevmode{[}scalar{]}
Distribution mean, should be \textgreater{} 0.

\item[{scale}] \leavevmode{[}scalar{]}
Scale parameter, should be \textgreater{}= 0.

\item[{size}] \leavevmode{[}int or tuple of ints, optional{]}
Output shape. Default is None, in which case a single value is
returned.

\end{description}
\begin{description}
\item[{samples}] \leavevmode{[}ndarray or scalar{]}
Drawn sample, all greater than zero.

\end{description}

The probability density function for the Wald distribution is
\begin{gather}
\begin{split}P(x;mean,scale) = \sqrt{\frac{scale}{2\pi x^3}}e^
\frac{-scale(x-mean)^2}{2\cdotp mean^2x}\end{split}\notag
\end{gather}
As noted above the Inverse Gaussian distribution first arise from attempts
to model Brownian Motion. It is also a competitor to the Weibull for use in
reliability modeling and modeling stock returns and interest rate
processes.

Draw values from the distribution and plot the histogram:

\begin{Verbatim}[commandchars=\\\{\}]
\PYG{g+gp}{\PYGZgt{}\PYGZgt{}\PYGZgt{} }\PYG{k+kn}{import} \PYG{n+nn}{matplotlib.pyplot} \PYG{k+kn}{as} \PYG{n+nn}{plt}
\PYG{g+gp}{\PYGZgt{}\PYGZgt{}\PYGZgt{} }\PYG{n}{h} \PYG{o}{=} \PYG{n}{plt}\PYG{o}{.}\PYG{n}{hist}\PYG{p}{(}\PYG{n}{np}\PYG{o}{.}\PYG{n}{random}\PYG{o}{.}\PYG{n}{wald}\PYG{p}{(}\PYG{l+m+mi}{3}\PYG{p}{,} \PYG{l+m+mi}{2}\PYG{p}{,} \PYG{l+m+mi}{100000}\PYG{p}{)}\PYG{p}{,} \PYG{n}{bins}\PYG{o}{=}\PYG{l+m+mi}{200}\PYG{p}{,} \PYG{n}{normed}\PYG{o}{=}\PYG{n+nb+bp}{True}\PYG{p}{)}
\PYG{g+gp}{\PYGZgt{}\PYGZgt{}\PYGZgt{} }\PYG{n}{plt}\PYG{o}{.}\PYG{n}{show}\PYG{p}{(}\PYG{p}{)}
\end{Verbatim}

\end{fulllineitems}

\index{weibull() (in module acsDynStatInTime)}

\begin{fulllineitems}
\phantomsection\label{acsDynStatInTime:acsDynStatInTime.weibull}\pysiglinewithargsret{\code{acsDynStatInTime.}\bfcode{weibull}}{\emph{a}, \emph{size=None}}{}
Weibull distribution.

Draw samples from a 1-parameter Weibull distribution with the given
shape parameter \emph{a}.
\begin{gather}
\begin{split}X = (-ln(U))^{1/a}\end{split}\notag
\end{gather}
Here, U is drawn from the uniform distribution over (0,1{]}.

The more common 2-parameter Weibull, including a scale parameter
\(\lambda\) is just \(X = \lambda(-ln(U))^{1/a}\).
\begin{description}
\item[{a}] \leavevmode{[}float{]}
Shape of the distribution.

\item[{size}] \leavevmode{[}tuple of ints{]}
Output shape.  If the given shape is, e.g., \code{(m, n, k)}, then
\code{m * n * k} samples are drawn.

\end{description}

scipy.stats.distributions.weibull\_max
scipy.stats.distributions.weibull\_min
scipy.stats.distributions.genextreme
gumbel

The Weibull (or Type III asymptotic extreme value distribution for smallest
values, SEV Type III, or Rosin-Rammler distribution) is one of a class of
Generalized Extreme Value (GEV) distributions used in modeling extreme
value problems.  This class includes the Gumbel and Frechet distributions.

The probability density for the Weibull distribution is
\begin{gather}
\begin{split}p(x) = \frac{a}
{\lambda}(\frac{x}{\lambda})^{a-1}e^{-(x/\lambda)^a},\end{split}\notag
\end{gather}
where \(a\) is the shape and \(\lambda\) the scale.

The function has its peak (the mode) at
\(\lambda(\frac{a-1}{a})^{1/a}\).

When \code{a = 1}, the Weibull distribution reduces to the exponential
distribution.

Draw samples from the distribution:

\begin{Verbatim}[commandchars=\\\{\}]
\PYG{g+gp}{\PYGZgt{}\PYGZgt{}\PYGZgt{} }\PYG{n}{a} \PYG{o}{=} \PYG{l+m+mf}{5.} \PYG{c}{\PYGZsh{} shape}
\PYG{g+gp}{\PYGZgt{}\PYGZgt{}\PYGZgt{} }\PYG{n}{s} \PYG{o}{=} \PYG{n}{np}\PYG{o}{.}\PYG{n}{random}\PYG{o}{.}\PYG{n}{weibull}\PYG{p}{(}\PYG{n}{a}\PYG{p}{,} \PYG{l+m+mi}{1000}\PYG{p}{)}
\end{Verbatim}

Display the histogram of the samples, along with
the probability density function:

\begin{Verbatim}[commandchars=\\\{\}]
\PYG{g+gp}{\PYGZgt{}\PYGZgt{}\PYGZgt{} }\PYG{k+kn}{import} \PYG{n+nn}{matplotlib.pyplot} \PYG{k+kn}{as} \PYG{n+nn}{plt}
\PYG{g+gp}{\PYGZgt{}\PYGZgt{}\PYGZgt{} }\PYG{n}{x} \PYG{o}{=} \PYG{n}{np}\PYG{o}{.}\PYG{n}{arange}\PYG{p}{(}\PYG{l+m+mi}{1}\PYG{p}{,}\PYG{l+m+mf}{100.}\PYG{p}{)}\PYG{o}{/}\PYG{l+m+mf}{50.}
\PYG{g+gp}{\PYGZgt{}\PYGZgt{}\PYGZgt{} }\PYG{k}{def} \PYG{n+nf}{weib}\PYG{p}{(}\PYG{n}{x}\PYG{p}{,}\PYG{n}{n}\PYG{p}{,}\PYG{n}{a}\PYG{p}{)}\PYG{p}{:}
\PYG{g+gp}{... }    \PYG{k}{return} \PYG{p}{(}\PYG{n}{a} \PYG{o}{/} \PYG{n}{n}\PYG{p}{)} \PYG{o}{*} \PYG{p}{(}\PYG{n}{x} \PYG{o}{/} \PYG{n}{n}\PYG{p}{)}\PYG{o}{*}\PYG{o}{*}\PYG{p}{(}\PYG{n}{a} \PYG{o}{\PYGZhy{}} \PYG{l+m+mi}{1}\PYG{p}{)} \PYG{o}{*} \PYG{n}{np}\PYG{o}{.}\PYG{n}{exp}\PYG{p}{(}\PYG{o}{\PYGZhy{}}\PYG{p}{(}\PYG{n}{x} \PYG{o}{/} \PYG{n}{n}\PYG{p}{)}\PYG{o}{*}\PYG{o}{*}\PYG{n}{a}\PYG{p}{)}
\end{Verbatim}

\begin{Verbatim}[commandchars=\\\{\}]
\PYG{g+gp}{\PYGZgt{}\PYGZgt{}\PYGZgt{} }\PYG{n}{count}\PYG{p}{,} \PYG{n}{bins}\PYG{p}{,} \PYG{n}{ignored} \PYG{o}{=} \PYG{n}{plt}\PYG{o}{.}\PYG{n}{hist}\PYG{p}{(}\PYG{n}{np}\PYG{o}{.}\PYG{n}{random}\PYG{o}{.}\PYG{n}{weibull}\PYG{p}{(}\PYG{l+m+mf}{5.}\PYG{p}{,}\PYG{l+m+mi}{1000}\PYG{p}{)}\PYG{p}{)}
\PYG{g+gp}{\PYGZgt{}\PYGZgt{}\PYGZgt{} }\PYG{n}{x} \PYG{o}{=} \PYG{n}{np}\PYG{o}{.}\PYG{n}{arange}\PYG{p}{(}\PYG{l+m+mi}{1}\PYG{p}{,}\PYG{l+m+mf}{100.}\PYG{p}{)}\PYG{o}{/}\PYG{l+m+mf}{50.}
\PYG{g+gp}{\PYGZgt{}\PYGZgt{}\PYGZgt{} }\PYG{n}{scale} \PYG{o}{=} \PYG{n}{count}\PYG{o}{.}\PYG{n}{max}\PYG{p}{(}\PYG{p}{)}\PYG{o}{/}\PYG{n}{weib}\PYG{p}{(}\PYG{n}{x}\PYG{p}{,} \PYG{l+m+mf}{1.}\PYG{p}{,} \PYG{l+m+mf}{5.}\PYG{p}{)}\PYG{o}{.}\PYG{n}{max}\PYG{p}{(}\PYG{p}{)}
\PYG{g+gp}{\PYGZgt{}\PYGZgt{}\PYGZgt{} }\PYG{n}{plt}\PYG{o}{.}\PYG{n}{plot}\PYG{p}{(}\PYG{n}{x}\PYG{p}{,} \PYG{n}{weib}\PYG{p}{(}\PYG{n}{x}\PYG{p}{,} \PYG{l+m+mf}{1.}\PYG{p}{,} \PYG{l+m+mf}{5.}\PYG{p}{)}\PYG{o}{*}\PYG{n}{scale}\PYG{p}{)}
\PYG{g+gp}{\PYGZgt{}\PYGZgt{}\PYGZgt{} }\PYG{n}{plt}\PYG{o}{.}\PYG{n}{show}\PYG{p}{(}\PYG{p}{)}
\end{Verbatim}

\end{fulllineitems}

\index{zipf() (in module acsDynStatInTime)}

\begin{fulllineitems}
\phantomsection\label{acsDynStatInTime:acsDynStatInTime.zipf}\pysiglinewithargsret{\code{acsDynStatInTime.}\bfcode{zipf}}{\emph{a}, \emph{size=None}}{}
Draw samples from a Zipf distribution.

Samples are drawn from a Zipf distribution with specified parameter
\emph{a} \textgreater{} 1.

The Zipf distribution (also known as the zeta distribution) is a
continuous probability distribution that satisfies Zipf's law: the
frequency of an item is inversely proportional to its rank in a
frequency table.
\begin{description}
\item[{a}] \leavevmode{[}float \textgreater{} 1{]}
Distribution parameter.

\item[{size}] \leavevmode{[}int or tuple of int, optional{]}
Output shape.  If the given shape is, e.g., \code{(m, n, k)}, then
\code{m * n * k} samples are drawn; a single integer is equivalent in
its result to providing a mono-tuple, i.e., a 1-D array of length
\emph{size} is returned.  The default is None, in which case a single
scalar is returned.

\end{description}
\begin{description}
\item[{samples}] \leavevmode{[}scalar or ndarray{]}
The returned samples are greater than or equal to one.

\end{description}
\begin{description}
\item[{scipy.stats.distributions.zipf}] \leavevmode{[}probability density function,{]}
distribution, or cumulative density function, etc.

\end{description}

The probability density for the Zipf distribution is
\begin{gather}
\begin{split}p(x) = \frac{x^{-a}}{\zeta(a)},\end{split}\notag
\end{gather}
where \(\zeta\) is the Riemann Zeta function.

It is named for the American linguist George Kingsley Zipf, who noted
that the frequency of any word in a sample of a language is inversely
proportional to its rank in the frequency table.

Zipf, G. K., \emph{Selected Studies of the Principle of Relative Frequency
in Language}, Cambridge, MA: Harvard Univ. Press, 1932.

Draw samples from the distribution:

\begin{Verbatim}[commandchars=\\\{\}]
\PYG{g+gp}{\PYGZgt{}\PYGZgt{}\PYGZgt{} }\PYG{n}{a} \PYG{o}{=} \PYG{l+m+mf}{2.} \PYG{c}{\PYGZsh{} parameter}
\PYG{g+gp}{\PYGZgt{}\PYGZgt{}\PYGZgt{} }\PYG{n}{s} \PYG{o}{=} \PYG{n}{np}\PYG{o}{.}\PYG{n}{random}\PYG{o}{.}\PYG{n}{zipf}\PYG{p}{(}\PYG{n}{a}\PYG{p}{,} \PYG{l+m+mi}{1000}\PYG{p}{)}
\end{Verbatim}

Display the histogram of the samples, along with
the probability density function:

\begin{Verbatim}[commandchars=\\\{\}]
\PYG{g+gp}{\PYGZgt{}\PYGZgt{}\PYGZgt{} }\PYG{k+kn}{import} \PYG{n+nn}{matplotlib.pyplot} \PYG{k+kn}{as} \PYG{n+nn}{plt}
\PYG{g+gp}{\PYGZgt{}\PYGZgt{}\PYGZgt{} }\PYG{k+kn}{import} \PYG{n+nn}{scipy.special} \PYG{k+kn}{as} \PYG{n+nn}{sps}
\PYG{g+go}{Truncate s values at 50 so plot is interesting}
\PYG{g+gp}{\PYGZgt{}\PYGZgt{}\PYGZgt{} }\PYG{n}{count}\PYG{p}{,} \PYG{n}{bins}\PYG{p}{,} \PYG{n}{ignored} \PYG{o}{=} \PYG{n}{plt}\PYG{o}{.}\PYG{n}{hist}\PYG{p}{(}\PYG{n}{s}\PYG{p}{[}\PYG{n}{s}\PYG{o}{\PYGZlt{}}\PYG{l+m+mi}{50}\PYG{p}{]}\PYG{p}{,} \PYG{l+m+mi}{50}\PYG{p}{,} \PYG{n}{normed}\PYG{o}{=}\PYG{n+nb+bp}{True}\PYG{p}{)}
\PYG{g+gp}{\PYGZgt{}\PYGZgt{}\PYGZgt{} }\PYG{n}{x} \PYG{o}{=} \PYG{n}{np}\PYG{o}{.}\PYG{n}{arange}\PYG{p}{(}\PYG{l+m+mf}{1.}\PYG{p}{,} \PYG{l+m+mf}{50.}\PYG{p}{)}
\PYG{g+gp}{\PYGZgt{}\PYGZgt{}\PYGZgt{} }\PYG{n}{y} \PYG{o}{=} \PYG{n}{x}\PYG{o}{*}\PYG{o}{*}\PYG{p}{(}\PYG{o}{\PYGZhy{}}\PYG{n}{a}\PYG{p}{)}\PYG{o}{/}\PYG{n}{sps}\PYG{o}{.}\PYG{n}{zetac}\PYG{p}{(}\PYG{n}{a}\PYG{p}{)}
\PYG{g+gp}{\PYGZgt{}\PYGZgt{}\PYGZgt{} }\PYG{n}{plt}\PYG{o}{.}\PYG{n}{plot}\PYG{p}{(}\PYG{n}{x}\PYG{p}{,} \PYG{n}{y}\PYG{o}{/}\PYG{n+nb}{max}\PYG{p}{(}\PYG{n}{y}\PYG{p}{)}\PYG{p}{,} \PYG{n}{linewidth}\PYG{o}{=}\PYG{l+m+mi}{2}\PYG{p}{,} \PYG{n}{color}\PYG{o}{=}\PYG{l+s}{\PYGZsq{}}\PYG{l+s}{r}\PYG{l+s}{\PYGZsq{}}\PYG{p}{)}
\PYG{g+gp}{\PYGZgt{}\PYGZgt{}\PYGZgt{} }\PYG{n}{plt}\PYG{o}{.}\PYG{n}{show}\PYG{p}{(}\PYG{p}{)}
\end{Verbatim}

\end{fulllineitems}



\chapter{acsFromWim2Carness Module}
\label{acsFromWim2Carness::doc}\label{acsFromWim2Carness:module-acsFromWim2Carness}\label{acsFromWim2Carness:acsfromwim2carness-module}\index{acsFromWim2Carness (module)}
File to convert Wim files in my files.
\index{beta() (in module acsFromWim2Carness)}

\begin{fulllineitems}
\phantomsection\label{acsFromWim2Carness:acsFromWim2Carness.beta}\pysiglinewithargsret{\code{acsFromWim2Carness.}\bfcode{beta}}{\emph{a}, \emph{b}, \emph{size=None}}{}
The Beta distribution over \code{{[}0, 1{]}}.

The Beta distribution is a special case of the Dirichlet distribution,
and is related to the Gamma distribution.  It has the probability
distribution function
\begin{gather}
\begin{split}f(x; a,b) = \frac{1}{B(\alpha, \beta)} x^{\alpha - 1}
(1 - x)^{\beta - 1},\end{split}\notag
\end{gather}
where the normalisation, B, is the beta function,
\begin{gather}
\begin{split}B(\alpha, \beta) = \int_0^1 t^{\alpha - 1}
(1 - t)^{\beta - 1} dt.\end{split}\notag
\end{gather}
It is often seen in Bayesian inference and order statistics.
\begin{description}
\item[{a}] \leavevmode{[}float{]}
Alpha, non-negative.

\item[{b}] \leavevmode{[}float{]}
Beta, non-negative.

\item[{size}] \leavevmode{[}tuple of ints, optional{]}
The number of samples to draw.  The output is packed according to
the size given.

\end{description}
\begin{description}
\item[{out}] \leavevmode{[}ndarray{]}
Array of the given shape, containing values drawn from a
Beta distribution.

\end{description}

\end{fulllineitems}

\index{binomial() (in module acsFromWim2Carness)}

\begin{fulllineitems}
\phantomsection\label{acsFromWim2Carness:acsFromWim2Carness.binomial}\pysiglinewithargsret{\code{acsFromWim2Carness.}\bfcode{binomial}}{\emph{n}, \emph{p}, \emph{size=None}}{}
Draw samples from a binomial distribution.

Samples are drawn from a Binomial distribution with specified
parameters, n trials and p probability of success where
n an integer \textgreater{}= 0 and p is in the interval {[}0,1{]}. (n may be
input as a float, but it is truncated to an integer in use)
\begin{description}
\item[{n}] \leavevmode{[}float (but truncated to an integer){]}
parameter, \textgreater{}= 0.

\item[{p}] \leavevmode{[}float{]}
parameter, \textgreater{}= 0 and \textless{}=1.

\item[{size}] \leavevmode{[}\{tuple, int\}{]}
Output shape.  If the given shape is, e.g., \code{(m, n, k)}, then
\code{m * n * k} samples are drawn.

\end{description}
\begin{description}
\item[{samples}] \leavevmode{[}\{ndarray, scalar\}{]}
where the values are all integers in  {[}0, n{]}.

\end{description}
\begin{description}
\item[{scipy.stats.distributions.binom}] \leavevmode{[}probability density function,{]}
distribution or cumulative density function, etc.

\end{description}

The probability density for the Binomial distribution is
\begin{gather}
\begin{split}P(N) = \binom{n}{N}p^N(1-p)^{n-N},\end{split}\notag
\end{gather}
where \(n\) is the number of trials, \(p\) is the probability
of success, and \(N\) is the number of successes.

When estimating the standard error of a proportion in a population by
using a random sample, the normal distribution works well unless the
product p*n \textless{}=5, where p = population proportion estimate, and n =
number of samples, in which case the binomial distribution is used
instead. For example, a sample of 15 people shows 4 who are left
handed, and 11 who are right handed. Then p = 4/15 = 27\%. 0.27*15 = 4,
so the binomial distribution should be used in this case.

Draw samples from the distribution:

\begin{Verbatim}[commandchars=\\\{\}]
\PYG{g+gp}{\PYGZgt{}\PYGZgt{}\PYGZgt{} }\PYG{n}{n}\PYG{p}{,} \PYG{n}{p} \PYG{o}{=} \PYG{l+m+mi}{10}\PYG{p}{,} \PYG{o}{.}\PYG{l+m+mi}{5} \PYG{c}{\PYGZsh{} number of trials, probability of each trial}
\PYG{g+gp}{\PYGZgt{}\PYGZgt{}\PYGZgt{} }\PYG{n}{s} \PYG{o}{=} \PYG{n}{np}\PYG{o}{.}\PYG{n}{random}\PYG{o}{.}\PYG{n}{binomial}\PYG{p}{(}\PYG{n}{n}\PYG{p}{,} \PYG{n}{p}\PYG{p}{,} \PYG{l+m+mi}{1000}\PYG{p}{)}
\PYG{g+go}{\PYGZsh{} result of flipping a coin 10 times, tested 1000 times.}
\end{Verbatim}

A real world example. A company drills 9 wild-cat oil exploration
wells, each with an estimated probability of success of 0.1. All nine
wells fail. What is the probability of that happening?

Let's do 20,000 trials of the model, and count the number that
generate zero positive results.

\begin{Verbatim}[commandchars=\\\{\}]
\PYG{g+gp}{\PYGZgt{}\PYGZgt{}\PYGZgt{} }\PYG{n+nb}{sum}\PYG{p}{(}\PYG{n}{np}\PYG{o}{.}\PYG{n}{random}\PYG{o}{.}\PYG{n}{binomial}\PYG{p}{(}\PYG{l+m+mi}{9}\PYG{p}{,}\PYG{l+m+mf}{0.1}\PYG{p}{,}\PYG{l+m+mi}{20000}\PYG{p}{)}\PYG{o}{==}\PYG{l+m+mi}{0}\PYG{p}{)}\PYG{o}{/}\PYG{l+m+mf}{20000.}
\PYG{g+go}{answer = 0.38885, or 38\PYGZpc{}.}
\end{Verbatim}

\end{fulllineitems}

\index{chisquare() (in module acsFromWim2Carness)}

\begin{fulllineitems}
\phantomsection\label{acsFromWim2Carness:acsFromWim2Carness.chisquare}\pysiglinewithargsret{\code{acsFromWim2Carness.}\bfcode{chisquare}}{\emph{df}, \emph{size=None}}{}
Draw samples from a chi-square distribution.

When \emph{df} independent random variables, each with standard normal
distributions (mean 0, variance 1), are squared and summed, the
resulting distribution is chi-square (see Notes).  This distribution
is often used in hypothesis testing.
\begin{description}
\item[{df}] \leavevmode{[}int{]}
Number of degrees of freedom.

\item[{size}] \leavevmode{[}tuple of ints, int, optional{]}
Size of the returned array.  By default, a scalar is
returned.

\end{description}
\begin{description}
\item[{output}] \leavevmode{[}ndarray{]}
Samples drawn from the distribution, packed in a \emph{size}-shaped
array.

\end{description}
\begin{description}
\item[{ValueError}] \leavevmode
When \emph{df} \textless{}= 0 or when an inappropriate \emph{size} (e.g. \code{size=-1})
is given.

\end{description}

The variable obtained by summing the squares of \emph{df} independent,
standard normally distributed random variables:
\begin{gather}
\begin{split}Q = \sum_{i=0}^{\mathtt{df}} X^2_i\end{split}\notag
\end{gather}
is chi-square distributed, denoted
\begin{gather}
\begin{split}Q \sim \chi^2_k.\end{split}\notag
\end{gather}
The probability density function of the chi-squared distribution is
\begin{gather}
\begin{split}p(x) = \frac{(1/2)^{k/2}}{\Gamma(k/2)}
x^{k/2 - 1} e^{-x/2},\end{split}\notag
\end{gather}
where \(\Gamma\) is the gamma function,
\begin{gather}
\begin{split}\Gamma(x) = \int_0^{-\infty} t^{x - 1} e^{-t} dt.\end{split}\notag
\end{gather}
\href{http://www.itl.nist.gov/div898/handbook/eda/section3/eda3666.htm}{NIST/SEMATECH e-Handbook of Statistical Methods}

\begin{Verbatim}[commandchars=\\\{\}]
\PYG{g+gp}{\PYGZgt{}\PYGZgt{}\PYGZgt{} }\PYG{n}{np}\PYG{o}{.}\PYG{n}{random}\PYG{o}{.}\PYG{n}{chisquare}\PYG{p}{(}\PYG{l+m+mi}{2}\PYG{p}{,}\PYG{l+m+mi}{4}\PYG{p}{)}
\PYG{g+go}{array([ 1.89920014,  9.00867716,  3.13710533,  5.62318272])}
\end{Verbatim}

\end{fulllineitems}

\index{exponential() (in module acsFromWim2Carness)}

\begin{fulllineitems}
\phantomsection\label{acsFromWim2Carness:acsFromWim2Carness.exponential}\pysiglinewithargsret{\code{acsFromWim2Carness.}\bfcode{exponential}}{\emph{scale=1.0}, \emph{size=None}}{}
Exponential distribution.

Its probability density function is
\begin{gather}
\begin{split}f(x; \frac{1}{\beta}) = \frac{1}{\beta} \exp(-\frac{x}{\beta}),\end{split}\notag
\end{gather}
for \code{x \textgreater{} 0} and 0 elsewhere. \(\beta\) is the scale parameter,
which is the inverse of the rate parameter \(\lambda = 1/\beta\).
The rate parameter is an alternative, widely used parameterization
of the exponential distribution {\color{red}\bfseries{}{[}3{]}\_}.

The exponential distribution is a continuous analogue of the
geometric distribution.  It describes many common situations, such as
the size of raindrops measured over many rainstorms {\color{red}\bfseries{}{[}1{]}\_}, or the time
between page requests to Wikipedia {\color{red}\bfseries{}{[}2{]}\_}.
\begin{description}
\item[{scale}] \leavevmode{[}float{]}
The scale parameter, \(\beta = 1/\lambda\).

\item[{size}] \leavevmode{[}tuple of ints{]}
Number of samples to draw.  The output is shaped
according to \emph{size}.

\end{description}

\end{fulllineitems}

\index{f() (in module acsFromWim2Carness)}

\begin{fulllineitems}
\phantomsection\label{acsFromWim2Carness:acsFromWim2Carness.f}\pysiglinewithargsret{\code{acsFromWim2Carness.}\bfcode{f}}{\emph{dfnum}, \emph{dfden}, \emph{size=None}}{}
Draw samples from a F distribution.

Samples are drawn from an F distribution with specified parameters,
\emph{dfnum} (degrees of freedom in numerator) and \emph{dfden} (degrees of freedom
in denominator), where both parameters should be greater than zero.

The random variate of the F distribution (also known as the
Fisher distribution) is a continuous probability distribution
that arises in ANOVA tests, and is the ratio of two chi-square
variates.
\begin{description}
\item[{dfnum}] \leavevmode{[}float{]}
Degrees of freedom in numerator. Should be greater than zero.

\item[{dfden}] \leavevmode{[}float{]}
Degrees of freedom in denominator. Should be greater than zero.

\item[{size}] \leavevmode{[}\{tuple, int\}, optional{]}
Output shape.  If the given shape is, e.g., \code{(m, n, k)},
then \code{m * n * k} samples are drawn. By default only one sample
is returned.

\end{description}
\begin{description}
\item[{samples}] \leavevmode{[}\{ndarray, scalar\}{]}
Samples from the Fisher distribution.

\end{description}
\begin{description}
\item[{scipy.stats.distributions.f}] \leavevmode{[}probability density function,{]}
distribution or cumulative density function, etc.

\end{description}

The F statistic is used to compare in-group variances to between-group
variances. Calculating the distribution depends on the sampling, and
so it is a function of the respective degrees of freedom in the
problem.  The variable \emph{dfnum} is the number of samples minus one, the
between-groups degrees of freedom, while \emph{dfden} is the within-groups
degrees of freedom, the sum of the number of samples in each group
minus the number of groups.

An example from Glantz{[}1{]}, pp 47-40.
Two groups, children of diabetics (25 people) and children from people
without diabetes (25 controls). Fasting blood glucose was measured,
case group had a mean value of 86.1, controls had a mean value of
82.2. Standard deviations were 2.09 and 2.49 respectively. Are these
data consistent with the null hypothesis that the parents diabetic
status does not affect their children's blood glucose levels?
Calculating the F statistic from the data gives a value of 36.01.

Draw samples from the distribution:

\begin{Verbatim}[commandchars=\\\{\}]
\PYG{g+gp}{\PYGZgt{}\PYGZgt{}\PYGZgt{} }\PYG{n}{dfnum} \PYG{o}{=} \PYG{l+m+mf}{1.} \PYG{c}{\PYGZsh{} between group degrees of freedom}
\PYG{g+gp}{\PYGZgt{}\PYGZgt{}\PYGZgt{} }\PYG{n}{dfden} \PYG{o}{=} \PYG{l+m+mf}{48.} \PYG{c}{\PYGZsh{} within groups degrees of freedom}
\PYG{g+gp}{\PYGZgt{}\PYGZgt{}\PYGZgt{} }\PYG{n}{s} \PYG{o}{=} \PYG{n}{np}\PYG{o}{.}\PYG{n}{random}\PYG{o}{.}\PYG{n}{f}\PYG{p}{(}\PYG{n}{dfnum}\PYG{p}{,} \PYG{n}{dfden}\PYG{p}{,} \PYG{l+m+mi}{1000}\PYG{p}{)}
\end{Verbatim}

The lower bound for the top 1\% of the samples is :

\begin{Verbatim}[commandchars=\\\{\}]
\PYG{g+gp}{\PYGZgt{}\PYGZgt{}\PYGZgt{} }\PYG{n}{sort}\PYG{p}{(}\PYG{n}{s}\PYG{p}{)}\PYG{p}{[}\PYG{o}{\PYGZhy{}}\PYG{l+m+mi}{10}\PYG{p}{]}
\PYG{g+go}{7.61988120985}
\end{Verbatim}

So there is about a 1\% chance that the F statistic will exceed 7.62,
the measured value is 36, so the null hypothesis is rejected at the 1\%
level.

\end{fulllineitems}

\index{gamma() (in module acsFromWim2Carness)}

\begin{fulllineitems}
\phantomsection\label{acsFromWim2Carness:acsFromWim2Carness.gamma}\pysiglinewithargsret{\code{acsFromWim2Carness.}\bfcode{gamma}}{\emph{shape}, \emph{scale=1.0}, \emph{size=None}}{}
Draw samples from a Gamma distribution.

Samples are drawn from a Gamma distribution with specified parameters,
\emph{shape} (sometimes designated ``k'') and \emph{scale} (sometimes designated
``theta''), where both parameters are \textgreater{} 0.
\begin{description}
\item[{shape}] \leavevmode{[}scalar \textgreater{} 0{]}
The shape of the gamma distribution.

\item[{scale}] \leavevmode{[}scalar \textgreater{} 0, optional{]}
The scale of the gamma distribution.  Default is equal to 1.

\item[{size}] \leavevmode{[}shape\_tuple, optional{]}
Output shape.  If the given shape is, e.g., \code{(m, n, k)}, then
\code{m * n * k} samples are drawn.

\end{description}
\begin{description}
\item[{out}] \leavevmode{[}ndarray, float{]}
Returns one sample unless \emph{size} parameter is specified.

\end{description}
\begin{description}
\item[{scipy.stats.distributions.gamma}] \leavevmode{[}probability density function,{]}
distribution or cumulative density function, etc.

\end{description}

The probability density for the Gamma distribution is
\begin{gather}
\begin{split}p(x) = x^{k-1}\frac{e^{-x/\theta}}{\theta^k\Gamma(k)},\end{split}\notag
\end{gather}
where \(k\) is the shape and \(\theta\) the scale,
and \(\Gamma\) is the Gamma function.

The Gamma distribution is often used to model the times to failure of
electronic components, and arises naturally in processes for which the
waiting times between Poisson distributed events are relevant.

Draw samples from the distribution:

\begin{Verbatim}[commandchars=\\\{\}]
\PYG{g+gp}{\PYGZgt{}\PYGZgt{}\PYGZgt{} }\PYG{n}{shape}\PYG{p}{,} \PYG{n}{scale} \PYG{o}{=} \PYG{l+m+mf}{2.}\PYG{p}{,} \PYG{l+m+mf}{2.} \PYG{c}{\PYGZsh{} mean and dispersion}
\PYG{g+gp}{\PYGZgt{}\PYGZgt{}\PYGZgt{} }\PYG{n}{s} \PYG{o}{=} \PYG{n}{np}\PYG{o}{.}\PYG{n}{random}\PYG{o}{.}\PYG{n}{gamma}\PYG{p}{(}\PYG{n}{shape}\PYG{p}{,} \PYG{n}{scale}\PYG{p}{,} \PYG{l+m+mi}{1000}\PYG{p}{)}
\end{Verbatim}

Display the histogram of the samples, along with
the probability density function:

\begin{Verbatim}[commandchars=\\\{\}]
\PYG{g+gp}{\PYGZgt{}\PYGZgt{}\PYGZgt{} }\PYG{k+kn}{import} \PYG{n+nn}{matplotlib.pyplot} \PYG{k+kn}{as} \PYG{n+nn}{plt}
\PYG{g+gp}{\PYGZgt{}\PYGZgt{}\PYGZgt{} }\PYG{k+kn}{import} \PYG{n+nn}{scipy.special} \PYG{k+kn}{as} \PYG{n+nn}{sps}
\PYG{g+gp}{\PYGZgt{}\PYGZgt{}\PYGZgt{} }\PYG{n}{count}\PYG{p}{,} \PYG{n}{bins}\PYG{p}{,} \PYG{n}{ignored} \PYG{o}{=} \PYG{n}{plt}\PYG{o}{.}\PYG{n}{hist}\PYG{p}{(}\PYG{n}{s}\PYG{p}{,} \PYG{l+m+mi}{50}\PYG{p}{,} \PYG{n}{normed}\PYG{o}{=}\PYG{n+nb+bp}{True}\PYG{p}{)}
\PYG{g+gp}{\PYGZgt{}\PYGZgt{}\PYGZgt{} }\PYG{n}{y} \PYG{o}{=} \PYG{n}{bins}\PYG{o}{*}\PYG{o}{*}\PYG{p}{(}\PYG{n}{shape}\PYG{o}{\PYGZhy{}}\PYG{l+m+mi}{1}\PYG{p}{)}\PYG{o}{*}\PYG{p}{(}\PYG{n}{np}\PYG{o}{.}\PYG{n}{exp}\PYG{p}{(}\PYG{o}{\PYGZhy{}}\PYG{n}{bins}\PYG{o}{/}\PYG{n}{scale}\PYG{p}{)} \PYG{o}{/}
\PYG{g+gp}{... }                     \PYG{p}{(}\PYG{n}{sps}\PYG{o}{.}\PYG{n}{gamma}\PYG{p}{(}\PYG{n}{shape}\PYG{p}{)}\PYG{o}{*}\PYG{n}{scale}\PYG{o}{*}\PYG{o}{*}\PYG{n}{shape}\PYG{p}{)}\PYG{p}{)}
\PYG{g+gp}{\PYGZgt{}\PYGZgt{}\PYGZgt{} }\PYG{n}{plt}\PYG{o}{.}\PYG{n}{plot}\PYG{p}{(}\PYG{n}{bins}\PYG{p}{,} \PYG{n}{y}\PYG{p}{,} \PYG{n}{linewidth}\PYG{o}{=}\PYG{l+m+mi}{2}\PYG{p}{,} \PYG{n}{color}\PYG{o}{=}\PYG{l+s}{\PYGZsq{}}\PYG{l+s}{r}\PYG{l+s}{\PYGZsq{}}\PYG{p}{)}
\PYG{g+gp}{\PYGZgt{}\PYGZgt{}\PYGZgt{} }\PYG{n}{plt}\PYG{o}{.}\PYG{n}{show}\PYG{p}{(}\PYG{p}{)}
\end{Verbatim}

\end{fulllineitems}

\index{geometric() (in module acsFromWim2Carness)}

\begin{fulllineitems}
\phantomsection\label{acsFromWim2Carness:acsFromWim2Carness.geometric}\pysiglinewithargsret{\code{acsFromWim2Carness.}\bfcode{geometric}}{\emph{p}, \emph{size=None}}{}
Draw samples from the geometric distribution.

Bernoulli trials are experiments with one of two outcomes:
success or failure (an example of such an experiment is flipping
a coin).  The geometric distribution models the number of trials
that must be run in order to achieve success.  It is therefore
supported on the positive integers, \code{k = 1, 2, ...}.

The probability mass function of the geometric distribution is
\begin{gather}
\begin{split}f(k) = (1 - p)^{k - 1} p\end{split}\notag
\end{gather}
where \emph{p} is the probability of success of an individual trial.
\begin{description}
\item[{p}] \leavevmode{[}float{]}
The probability of success of an individual trial.

\item[{size}] \leavevmode{[}tuple of ints{]}
Number of values to draw from the distribution.  The output
is shaped according to \emph{size}.

\end{description}
\begin{description}
\item[{out}] \leavevmode{[}ndarray{]}
Samples from the geometric distribution, shaped according to
\emph{size}.

\end{description}

Draw ten thousand values from the geometric distribution,
with the probability of an individual success equal to 0.35:

\begin{Verbatim}[commandchars=\\\{\}]
\PYG{g+gp}{\PYGZgt{}\PYGZgt{}\PYGZgt{} }\PYG{n}{z} \PYG{o}{=} \PYG{n}{np}\PYG{o}{.}\PYG{n}{random}\PYG{o}{.}\PYG{n}{geometric}\PYG{p}{(}\PYG{n}{p}\PYG{o}{=}\PYG{l+m+mf}{0.35}\PYG{p}{,} \PYG{n}{size}\PYG{o}{=}\PYG{l+m+mi}{10000}\PYG{p}{)}
\end{Verbatim}

How many trials succeeded after a single run?

\begin{Verbatim}[commandchars=\\\{\}]
\PYG{g+gp}{\PYGZgt{}\PYGZgt{}\PYGZgt{} }\PYG{p}{(}\PYG{n}{z} \PYG{o}{==} \PYG{l+m+mi}{1}\PYG{p}{)}\PYG{o}{.}\PYG{n}{sum}\PYG{p}{(}\PYG{p}{)} \PYG{o}{/} \PYG{l+m+mf}{10000.}
\PYG{g+go}{0.34889999999999999 \PYGZsh{}random}
\end{Verbatim}

\end{fulllineitems}

\index{get\_state() (in module acsFromWim2Carness)}

\begin{fulllineitems}
\phantomsection\label{acsFromWim2Carness:acsFromWim2Carness.get_state}\pysiglinewithargsret{\code{acsFromWim2Carness.}\bfcode{get\_state}}{}{}
Return a tuple representing the internal state of the generator.

For more details, see \emph{set\_state}.
\begin{description}
\item[{out}] \leavevmode{[}tuple(str, ndarray of 624 uints, int, int, float){]}
The returned tuple has the following items:
\begin{enumerate}
\item {} 
the string `MT19937'.

\item {} 
a 1-D array of 624 unsigned integer keys.

\item {} 
an integer \code{pos}.

\item {} 
an integer \code{has\_gauss}.

\item {} 
a float \code{cached\_gaussian}.

\end{enumerate}

\end{description}

set\_state

\emph{set\_state} and \emph{get\_state} are not needed to work with any of the
random distributions in NumPy. If the internal state is manually altered,
the user should know exactly what he/she is doing.

\end{fulllineitems}

\index{gumbel() (in module acsFromWim2Carness)}

\begin{fulllineitems}
\phantomsection\label{acsFromWim2Carness:acsFromWim2Carness.gumbel}\pysiglinewithargsret{\code{acsFromWim2Carness.}\bfcode{gumbel}}{\emph{loc=0.0}, \emph{scale=1.0}, \emph{size=None}}{}
Gumbel distribution.

Draw samples from a Gumbel distribution with specified location and scale.
For more information on the Gumbel distribution, see Notes and References
below.
\begin{description}
\item[{loc}] \leavevmode{[}float{]}
The location of the mode of the distribution.

\item[{scale}] \leavevmode{[}float{]}
The scale parameter of the distribution.

\item[{size}] \leavevmode{[}tuple of ints{]}
Output shape.  If the given shape is, e.g., \code{(m, n, k)}, then
\code{m * n * k} samples are drawn.

\end{description}
\begin{description}
\item[{out}] \leavevmode{[}ndarray{]}
The samples

\end{description}

scipy.stats.gumbel\_l
scipy.stats.gumbel\_r
scipy.stats.genextreme
\begin{quote}

probability density function, distribution, or cumulative density
function, etc. for each of the above
\end{quote}

weibull

The Gumbel (or Smallest Extreme Value (SEV) or the Smallest Extreme Value
Type I) distribution is one of a class of Generalized Extreme Value (GEV)
distributions used in modeling extreme value problems.  The Gumbel is a
special case of the Extreme Value Type I distribution for maximums from
distributions with ``exponential-like'' tails.

The probability density for the Gumbel distribution is
\begin{gather}
\begin{split}p(x) = \frac{e^{-(x - \mu)/ \beta}}{\beta} e^{ -e^{-(x - \mu)/
\beta}},\end{split}\notag
\end{gather}
where \(\mu\) is the mode, a location parameter, and \(\beta\) is
the scale parameter.

The Gumbel (named for German mathematician Emil Julius Gumbel) was used
very early in the hydrology literature, for modeling the occurrence of
flood events. It is also used for modeling maximum wind speed and rainfall
rates.  It is a ``fat-tailed'' distribution - the probability of an event in
the tail of the distribution is larger than if one used a Gaussian, hence
the surprisingly frequent occurrence of 100-year floods. Floods were
initially modeled as a Gaussian process, which underestimated the frequency
of extreme events.

It is one of a class of extreme value distributions, the Generalized
Extreme Value (GEV) distributions, which also includes the Weibull and
Frechet.

The function has a mean of \(\mu + 0.57721\beta\) and a variance of
\(\frac{\pi^2}{6}\beta^2\).

Gumbel, E. J., \emph{Statistics of Extremes}, New York: Columbia University
Press, 1958.

Reiss, R.-D. and Thomas, M., \emph{Statistical Analysis of Extreme Values from
Insurance, Finance, Hydrology and Other Fields}, Basel: Birkhauser Verlag,
2001.

Draw samples from the distribution:

\begin{Verbatim}[commandchars=\\\{\}]
\PYG{g+gp}{\PYGZgt{}\PYGZgt{}\PYGZgt{} }\PYG{n}{mu}\PYG{p}{,} \PYG{n}{beta} \PYG{o}{=} \PYG{l+m+mi}{0}\PYG{p}{,} \PYG{l+m+mf}{0.1} \PYG{c}{\PYGZsh{} location and scale}
\PYG{g+gp}{\PYGZgt{}\PYGZgt{}\PYGZgt{} }\PYG{n}{s} \PYG{o}{=} \PYG{n}{np}\PYG{o}{.}\PYG{n}{random}\PYG{o}{.}\PYG{n}{gumbel}\PYG{p}{(}\PYG{n}{mu}\PYG{p}{,} \PYG{n}{beta}\PYG{p}{,} \PYG{l+m+mi}{1000}\PYG{p}{)}
\end{Verbatim}

Display the histogram of the samples, along with
the probability density function:

\begin{Verbatim}[commandchars=\\\{\}]
\PYG{g+gp}{\PYGZgt{}\PYGZgt{}\PYGZgt{} }\PYG{k+kn}{import} \PYG{n+nn}{matplotlib.pyplot} \PYG{k+kn}{as} \PYG{n+nn}{plt}
\PYG{g+gp}{\PYGZgt{}\PYGZgt{}\PYGZgt{} }\PYG{n}{count}\PYG{p}{,} \PYG{n}{bins}\PYG{p}{,} \PYG{n}{ignored} \PYG{o}{=} \PYG{n}{plt}\PYG{o}{.}\PYG{n}{hist}\PYG{p}{(}\PYG{n}{s}\PYG{p}{,} \PYG{l+m+mi}{30}\PYG{p}{,} \PYG{n}{normed}\PYG{o}{=}\PYG{n+nb+bp}{True}\PYG{p}{)}
\PYG{g+gp}{\PYGZgt{}\PYGZgt{}\PYGZgt{} }\PYG{n}{plt}\PYG{o}{.}\PYG{n}{plot}\PYG{p}{(}\PYG{n}{bins}\PYG{p}{,} \PYG{p}{(}\PYG{l+m+mi}{1}\PYG{o}{/}\PYG{n}{beta}\PYG{p}{)}\PYG{o}{*}\PYG{n}{np}\PYG{o}{.}\PYG{n}{exp}\PYG{p}{(}\PYG{o}{\PYGZhy{}}\PYG{p}{(}\PYG{n}{bins} \PYG{o}{\PYGZhy{}} \PYG{n}{mu}\PYG{p}{)}\PYG{o}{/}\PYG{n}{beta}\PYG{p}{)}
\PYG{g+gp}{... }         \PYG{o}{*} \PYG{n}{np}\PYG{o}{.}\PYG{n}{exp}\PYG{p}{(} \PYG{o}{\PYGZhy{}}\PYG{n}{np}\PYG{o}{.}\PYG{n}{exp}\PYG{p}{(} \PYG{o}{\PYGZhy{}}\PYG{p}{(}\PYG{n}{bins} \PYG{o}{\PYGZhy{}} \PYG{n}{mu}\PYG{p}{)} \PYG{o}{/}\PYG{n}{beta}\PYG{p}{)} \PYG{p}{)}\PYG{p}{,}
\PYG{g+gp}{... }         \PYG{n}{linewidth}\PYG{o}{=}\PYG{l+m+mi}{2}\PYG{p}{,} \PYG{n}{color}\PYG{o}{=}\PYG{l+s}{\PYGZsq{}}\PYG{l+s}{r}\PYG{l+s}{\PYGZsq{}}\PYG{p}{)}
\PYG{g+gp}{\PYGZgt{}\PYGZgt{}\PYGZgt{} }\PYG{n}{plt}\PYG{o}{.}\PYG{n}{show}\PYG{p}{(}\PYG{p}{)}
\end{Verbatim}

Show how an extreme value distribution can arise from a Gaussian process
and compare to a Gaussian:

\begin{Verbatim}[commandchars=\\\{\}]
\PYG{g+gp}{\PYGZgt{}\PYGZgt{}\PYGZgt{} }\PYG{n}{means} \PYG{o}{=} \PYG{p}{[}\PYG{p}{]}
\PYG{g+gp}{\PYGZgt{}\PYGZgt{}\PYGZgt{} }\PYG{n}{maxima} \PYG{o}{=} \PYG{p}{[}\PYG{p}{]}
\PYG{g+gp}{\PYGZgt{}\PYGZgt{}\PYGZgt{} }\PYG{k}{for} \PYG{n}{i} \PYG{o+ow}{in} \PYG{n+nb}{range}\PYG{p}{(}\PYG{l+m+mi}{0}\PYG{p}{,}\PYG{l+m+mi}{1000}\PYG{p}{)} \PYG{p}{:}
\PYG{g+gp}{... }   \PYG{n}{a} \PYG{o}{=} \PYG{n}{np}\PYG{o}{.}\PYG{n}{random}\PYG{o}{.}\PYG{n}{normal}\PYG{p}{(}\PYG{n}{mu}\PYG{p}{,} \PYG{n}{beta}\PYG{p}{,} \PYG{l+m+mi}{1000}\PYG{p}{)}
\PYG{g+gp}{... }   \PYG{n}{means}\PYG{o}{.}\PYG{n}{append}\PYG{p}{(}\PYG{n}{a}\PYG{o}{.}\PYG{n}{mean}\PYG{p}{(}\PYG{p}{)}\PYG{p}{)}
\PYG{g+gp}{... }   \PYG{n}{maxima}\PYG{o}{.}\PYG{n}{append}\PYG{p}{(}\PYG{n}{a}\PYG{o}{.}\PYG{n}{max}\PYG{p}{(}\PYG{p}{)}\PYG{p}{)}
\PYG{g+gp}{\PYGZgt{}\PYGZgt{}\PYGZgt{} }\PYG{n}{count}\PYG{p}{,} \PYG{n}{bins}\PYG{p}{,} \PYG{n}{ignored} \PYG{o}{=} \PYG{n}{plt}\PYG{o}{.}\PYG{n}{hist}\PYG{p}{(}\PYG{n}{maxima}\PYG{p}{,} \PYG{l+m+mi}{30}\PYG{p}{,} \PYG{n}{normed}\PYG{o}{=}\PYG{n+nb+bp}{True}\PYG{p}{)}
\PYG{g+gp}{\PYGZgt{}\PYGZgt{}\PYGZgt{} }\PYG{n}{beta} \PYG{o}{=} \PYG{n}{np}\PYG{o}{.}\PYG{n}{std}\PYG{p}{(}\PYG{n}{maxima}\PYG{p}{)}\PYG{o}{*}\PYG{n}{np}\PYG{o}{.}\PYG{n}{pi}\PYG{o}{/}\PYG{n}{np}\PYG{o}{.}\PYG{n}{sqrt}\PYG{p}{(}\PYG{l+m+mi}{6}\PYG{p}{)}
\PYG{g+gp}{\PYGZgt{}\PYGZgt{}\PYGZgt{} }\PYG{n}{mu} \PYG{o}{=} \PYG{n}{np}\PYG{o}{.}\PYG{n}{mean}\PYG{p}{(}\PYG{n}{maxima}\PYG{p}{)} \PYG{o}{\PYGZhy{}} \PYG{l+m+mf}{0.57721}\PYG{o}{*}\PYG{n}{beta}
\PYG{g+gp}{\PYGZgt{}\PYGZgt{}\PYGZgt{} }\PYG{n}{plt}\PYG{o}{.}\PYG{n}{plot}\PYG{p}{(}\PYG{n}{bins}\PYG{p}{,} \PYG{p}{(}\PYG{l+m+mi}{1}\PYG{o}{/}\PYG{n}{beta}\PYG{p}{)}\PYG{o}{*}\PYG{n}{np}\PYG{o}{.}\PYG{n}{exp}\PYG{p}{(}\PYG{o}{\PYGZhy{}}\PYG{p}{(}\PYG{n}{bins} \PYG{o}{\PYGZhy{}} \PYG{n}{mu}\PYG{p}{)}\PYG{o}{/}\PYG{n}{beta}\PYG{p}{)}
\PYG{g+gp}{... }         \PYG{o}{*} \PYG{n}{np}\PYG{o}{.}\PYG{n}{exp}\PYG{p}{(}\PYG{o}{\PYGZhy{}}\PYG{n}{np}\PYG{o}{.}\PYG{n}{exp}\PYG{p}{(}\PYG{o}{\PYGZhy{}}\PYG{p}{(}\PYG{n}{bins} \PYG{o}{\PYGZhy{}} \PYG{n}{mu}\PYG{p}{)}\PYG{o}{/}\PYG{n}{beta}\PYG{p}{)}\PYG{p}{)}\PYG{p}{,}
\PYG{g+gp}{... }         \PYG{n}{linewidth}\PYG{o}{=}\PYG{l+m+mi}{2}\PYG{p}{,} \PYG{n}{color}\PYG{o}{=}\PYG{l+s}{\PYGZsq{}}\PYG{l+s}{r}\PYG{l+s}{\PYGZsq{}}\PYG{p}{)}
\PYG{g+gp}{\PYGZgt{}\PYGZgt{}\PYGZgt{} }\PYG{n}{plt}\PYG{o}{.}\PYG{n}{plot}\PYG{p}{(}\PYG{n}{bins}\PYG{p}{,} \PYG{l+m+mi}{1}\PYG{o}{/}\PYG{p}{(}\PYG{n}{beta} \PYG{o}{*} \PYG{n}{np}\PYG{o}{.}\PYG{n}{sqrt}\PYG{p}{(}\PYG{l+m+mi}{2} \PYG{o}{*} \PYG{n}{np}\PYG{o}{.}\PYG{n}{pi}\PYG{p}{)}\PYG{p}{)}
\PYG{g+gp}{... }         \PYG{o}{*} \PYG{n}{np}\PYG{o}{.}\PYG{n}{exp}\PYG{p}{(}\PYG{o}{\PYGZhy{}}\PYG{p}{(}\PYG{n}{bins} \PYG{o}{\PYGZhy{}} \PYG{n}{mu}\PYG{p}{)}\PYG{o}{*}\PYG{o}{*}\PYG{l+m+mi}{2} \PYG{o}{/} \PYG{p}{(}\PYG{l+m+mi}{2} \PYG{o}{*} \PYG{n}{beta}\PYG{o}{*}\PYG{o}{*}\PYG{l+m+mi}{2}\PYG{p}{)}\PYG{p}{)}\PYG{p}{,}
\PYG{g+gp}{... }         \PYG{n}{linewidth}\PYG{o}{=}\PYG{l+m+mi}{2}\PYG{p}{,} \PYG{n}{color}\PYG{o}{=}\PYG{l+s}{\PYGZsq{}}\PYG{l+s}{g}\PYG{l+s}{\PYGZsq{}}\PYG{p}{)}
\PYG{g+gp}{\PYGZgt{}\PYGZgt{}\PYGZgt{} }\PYG{n}{plt}\PYG{o}{.}\PYG{n}{show}\PYG{p}{(}\PYG{p}{)}
\end{Verbatim}

\end{fulllineitems}

\index{hypergeometric() (in module acsFromWim2Carness)}

\begin{fulllineitems}
\phantomsection\label{acsFromWim2Carness:acsFromWim2Carness.hypergeometric}\pysiglinewithargsret{\code{acsFromWim2Carness.}\bfcode{hypergeometric}}{\emph{ngood}, \emph{nbad}, \emph{nsample}, \emph{size=None}}{}
Draw samples from a Hypergeometric distribution.

Samples are drawn from a Hypergeometric distribution with specified
parameters, ngood (ways to make a good selection), nbad (ways to make
a bad selection), and nsample = number of items sampled, which is less
than or equal to the sum ngood + nbad.
\begin{description}
\item[{ngood}] \leavevmode{[}int or array\_like{]}
Number of ways to make a good selection.  Must be nonnegative.

\item[{nbad}] \leavevmode{[}int or array\_like{]}
Number of ways to make a bad selection.  Must be nonnegative.

\item[{nsample}] \leavevmode{[}int or array\_like{]}
Number of items sampled.  Must be at least 1 and at most
\code{ngood + nbad}.

\item[{size}] \leavevmode{[}int or tuple of int{]}
Output shape.  If the given shape is, e.g., \code{(m, n, k)}, then
\code{m * n * k} samples are drawn.

\end{description}
\begin{description}
\item[{samples}] \leavevmode{[}ndarray or scalar{]}
The values are all integers in  {[}0, n{]}.

\end{description}
\begin{description}
\item[{scipy.stats.distributions.hypergeom}] \leavevmode{[}probability density function,{]}
distribution or cumulative density function, etc.

\end{description}

The probability density for the Hypergeometric distribution is
\begin{gather}
\begin{split}P(x) = \frac{\binom{m}{n}\binom{N-m}{n-x}}{\binom{N}{n}},\end{split}\notag
\end{gather}
where \(0 \le x \le m\) and \(n+m-N \le x \le n\)

for P(x) the probability of x successes, n = ngood, m = nbad, and
N = number of samples.

Consider an urn with black and white marbles in it, ngood of them
black and nbad are white. If you draw nsample balls without
replacement, then the Hypergeometric distribution describes the
distribution of black balls in the drawn sample.

Note that this distribution is very similar to the Binomial
distribution, except that in this case, samples are drawn without
replacement, whereas in the Binomial case samples are drawn with
replacement (or the sample space is infinite). As the sample space
becomes large, this distribution approaches the Binomial.

Draw samples from the distribution:

\begin{Verbatim}[commandchars=\\\{\}]
\PYG{g+gp}{\PYGZgt{}\PYGZgt{}\PYGZgt{} }\PYG{n}{ngood}\PYG{p}{,} \PYG{n}{nbad}\PYG{p}{,} \PYG{n}{nsamp} \PYG{o}{=} \PYG{l+m+mi}{100}\PYG{p}{,} \PYG{l+m+mi}{2}\PYG{p}{,} \PYG{l+m+mi}{10}
\PYG{g+go}{\PYGZsh{} number of good, number of bad, and number of samples}
\PYG{g+gp}{\PYGZgt{}\PYGZgt{}\PYGZgt{} }\PYG{n}{s} \PYG{o}{=} \PYG{n}{np}\PYG{o}{.}\PYG{n}{random}\PYG{o}{.}\PYG{n}{hypergeometric}\PYG{p}{(}\PYG{n}{ngood}\PYG{p}{,} \PYG{n}{nbad}\PYG{p}{,} \PYG{n}{nsamp}\PYG{p}{,} \PYG{l+m+mi}{1000}\PYG{p}{)}
\PYG{g+gp}{\PYGZgt{}\PYGZgt{}\PYGZgt{} }\PYG{n}{hist}\PYG{p}{(}\PYG{n}{s}\PYG{p}{)}
\PYG{g+go}{\PYGZsh{}   note that it is very unlikely to grab both bad items}
\end{Verbatim}

Suppose you have an urn with 15 white and 15 black marbles.
If you pull 15 marbles at random, how likely is it that
12 or more of them are one color?

\begin{Verbatim}[commandchars=\\\{\}]
\PYG{g+gp}{\PYGZgt{}\PYGZgt{}\PYGZgt{} }\PYG{n}{s} \PYG{o}{=} \PYG{n}{np}\PYG{o}{.}\PYG{n}{random}\PYG{o}{.}\PYG{n}{hypergeometric}\PYG{p}{(}\PYG{l+m+mi}{15}\PYG{p}{,} \PYG{l+m+mi}{15}\PYG{p}{,} \PYG{l+m+mi}{15}\PYG{p}{,} \PYG{l+m+mi}{100000}\PYG{p}{)}
\PYG{g+gp}{\PYGZgt{}\PYGZgt{}\PYGZgt{} }\PYG{n+nb}{sum}\PYG{p}{(}\PYG{n}{s}\PYG{o}{\PYGZgt{}}\PYG{o}{=}\PYG{l+m+mi}{12}\PYG{p}{)}\PYG{o}{/}\PYG{l+m+mf}{100000.} \PYG{o}{+} \PYG{n+nb}{sum}\PYG{p}{(}\PYG{n}{s}\PYG{o}{\PYGZlt{}}\PYG{o}{=}\PYG{l+m+mi}{3}\PYG{p}{)}\PYG{o}{/}\PYG{l+m+mf}{100000.}
\PYG{g+go}{\PYGZsh{}   answer = 0.003 ... pretty unlikely!}
\end{Verbatim}

\end{fulllineitems}

\index{laplace() (in module acsFromWim2Carness)}

\begin{fulllineitems}
\phantomsection\label{acsFromWim2Carness:acsFromWim2Carness.laplace}\pysiglinewithargsret{\code{acsFromWim2Carness.}\bfcode{laplace}}{\emph{loc=0.0}, \emph{scale=1.0}, \emph{size=None}}{}
Draw samples from the Laplace or double exponential distribution with
specified location (or mean) and scale (decay).

The Laplace distribution is similar to the Gaussian/normal distribution,
but is sharper at the peak and has fatter tails. It represents the
difference between two independent, identically distributed exponential
random variables.
\begin{description}
\item[{loc}] \leavevmode{[}float{]}
The position, \(\mu\), of the distribution peak.

\item[{scale}] \leavevmode{[}float{]}
\(\lambda\), the exponential decay.

\end{description}

It has the probability density function
\begin{gather}
\begin{split}f(x; \mu, \lambda) = \frac{1}{2\lambda}
\exp\left(-\frac{|x - \mu|}{\lambda}\right).\end{split}\notag
\end{gather}
The first law of Laplace, from 1774, states that the frequency of an error
can be expressed as an exponential function of the absolute magnitude of
the error, which leads to the Laplace distribution. For many problems in
Economics and Health sciences, this distribution seems to model the data
better than the standard Gaussian distribution

Draw samples from the distribution

\begin{Verbatim}[commandchars=\\\{\}]
\PYG{g+gp}{\PYGZgt{}\PYGZgt{}\PYGZgt{} }\PYG{n}{loc}\PYG{p}{,} \PYG{n}{scale} \PYG{o}{=} \PYG{l+m+mf}{0.}\PYG{p}{,} \PYG{l+m+mf}{1.}
\PYG{g+gp}{\PYGZgt{}\PYGZgt{}\PYGZgt{} }\PYG{n}{s} \PYG{o}{=} \PYG{n}{np}\PYG{o}{.}\PYG{n}{random}\PYG{o}{.}\PYG{n}{laplace}\PYG{p}{(}\PYG{n}{loc}\PYG{p}{,} \PYG{n}{scale}\PYG{p}{,} \PYG{l+m+mi}{1000}\PYG{p}{)}
\end{Verbatim}

Display the histogram of the samples, along with
the probability density function:

\begin{Verbatim}[commandchars=\\\{\}]
\PYG{g+gp}{\PYGZgt{}\PYGZgt{}\PYGZgt{} }\PYG{k+kn}{import} \PYG{n+nn}{matplotlib.pyplot} \PYG{k+kn}{as} \PYG{n+nn}{plt}
\PYG{g+gp}{\PYGZgt{}\PYGZgt{}\PYGZgt{} }\PYG{n}{count}\PYG{p}{,} \PYG{n}{bins}\PYG{p}{,} \PYG{n}{ignored} \PYG{o}{=} \PYG{n}{plt}\PYG{o}{.}\PYG{n}{hist}\PYG{p}{(}\PYG{n}{s}\PYG{p}{,} \PYG{l+m+mi}{30}\PYG{p}{,} \PYG{n}{normed}\PYG{o}{=}\PYG{n+nb+bp}{True}\PYG{p}{)}
\PYG{g+gp}{\PYGZgt{}\PYGZgt{}\PYGZgt{} }\PYG{n}{x} \PYG{o}{=} \PYG{n}{np}\PYG{o}{.}\PYG{n}{arange}\PYG{p}{(}\PYG{o}{\PYGZhy{}}\PYG{l+m+mf}{8.}\PYG{p}{,} \PYG{l+m+mf}{8.}\PYG{p}{,} \PYG{o}{.}\PYG{l+m+mo}{01}\PYG{p}{)}
\PYG{g+gp}{\PYGZgt{}\PYGZgt{}\PYGZgt{} }\PYG{n}{pdf} \PYG{o}{=} \PYG{n}{np}\PYG{o}{.}\PYG{n}{exp}\PYG{p}{(}\PYG{o}{\PYGZhy{}}\PYG{n+nb}{abs}\PYG{p}{(}\PYG{n}{x}\PYG{o}{\PYGZhy{}}\PYG{n}{loc}\PYG{o}{/}\PYG{n}{scale}\PYG{p}{)}\PYG{p}{)}\PYG{o}{/}\PYG{p}{(}\PYG{l+m+mf}{2.}\PYG{o}{*}\PYG{n}{scale}\PYG{p}{)}
\PYG{g+gp}{\PYGZgt{}\PYGZgt{}\PYGZgt{} }\PYG{n}{plt}\PYG{o}{.}\PYG{n}{plot}\PYG{p}{(}\PYG{n}{x}\PYG{p}{,} \PYG{n}{pdf}\PYG{p}{)}
\end{Verbatim}

Plot Gaussian for comparison:

\begin{Verbatim}[commandchars=\\\{\}]
\PYG{g+gp}{\PYGZgt{}\PYGZgt{}\PYGZgt{} }\PYG{n}{g} \PYG{o}{=} \PYG{p}{(}\PYG{l+m+mi}{1}\PYG{o}{/}\PYG{p}{(}\PYG{n}{scale} \PYG{o}{*} \PYG{n}{np}\PYG{o}{.}\PYG{n}{sqrt}\PYG{p}{(}\PYG{l+m+mi}{2} \PYG{o}{*} \PYG{n}{np}\PYG{o}{.}\PYG{n}{pi}\PYG{p}{)}\PYG{p}{)} \PYG{o}{*} 
\PYG{g+gp}{... }     \PYG{n}{np}\PYG{o}{.}\PYG{n}{exp}\PYG{p}{(} \PYG{o}{\PYGZhy{}} \PYG{p}{(}\PYG{n}{x} \PYG{o}{\PYGZhy{}} \PYG{n}{loc}\PYG{p}{)}\PYG{o}{*}\PYG{o}{*}\PYG{l+m+mi}{2} \PYG{o}{/} \PYG{p}{(}\PYG{l+m+mi}{2} \PYG{o}{*} \PYG{n}{scale}\PYG{o}{*}\PYG{o}{*}\PYG{l+m+mi}{2}\PYG{p}{)} \PYG{p}{)}\PYG{p}{)}
\PYG{g+gp}{\PYGZgt{}\PYGZgt{}\PYGZgt{} }\PYG{n}{plt}\PYG{o}{.}\PYG{n}{plot}\PYG{p}{(}\PYG{n}{x}\PYG{p}{,}\PYG{n}{g}\PYG{p}{)}
\end{Verbatim}

\end{fulllineitems}

\index{logistic() (in module acsFromWim2Carness)}

\begin{fulllineitems}
\phantomsection\label{acsFromWim2Carness:acsFromWim2Carness.logistic}\pysiglinewithargsret{\code{acsFromWim2Carness.}\bfcode{logistic}}{\emph{loc=0.0}, \emph{scale=1.0}, \emph{size=None}}{}
Draw samples from a Logistic distribution.

Samples are drawn from a Logistic distribution with specified
parameters, loc (location or mean, also median), and scale (\textgreater{}0).

loc : float

scale : float \textgreater{} 0.
\begin{description}
\item[{size}] \leavevmode{[}\{tuple, int\}{]}
Output shape.  If the given shape is, e.g., \code{(m, n, k)}, then
\code{m * n * k} samples are drawn.

\end{description}
\begin{description}
\item[{samples}] \leavevmode{[}\{ndarray, scalar\}{]}
where the values are all integers in  {[}0, n{]}.

\end{description}
\begin{description}
\item[{scipy.stats.distributions.logistic}] \leavevmode{[}probability density function,{]}
distribution or cumulative density function, etc.

\end{description}

The probability density for the Logistic distribution is
\begin{gather}
\begin{split}P(x) = P(x) = \frac{e^{-(x-\mu)/s}}{s(1+e^{-(x-\mu)/s})^2},\end{split}\notag
\end{gather}
where \(\mu\) = location and \(s\) = scale.

The Logistic distribution is used in Extreme Value problems where it
can act as a mixture of Gumbel distributions, in Epidemiology, and by
the World Chess Federation (FIDE) where it is used in the Elo ranking
system, assuming the performance of each player is a logistically
distributed random variable.

Draw samples from the distribution:

\begin{Verbatim}[commandchars=\\\{\}]
\PYG{g+gp}{\PYGZgt{}\PYGZgt{}\PYGZgt{} }\PYG{n}{loc}\PYG{p}{,} \PYG{n}{scale} \PYG{o}{=} \PYG{l+m+mi}{10}\PYG{p}{,} \PYG{l+m+mi}{1}
\PYG{g+gp}{\PYGZgt{}\PYGZgt{}\PYGZgt{} }\PYG{n}{s} \PYG{o}{=} \PYG{n}{np}\PYG{o}{.}\PYG{n}{random}\PYG{o}{.}\PYG{n}{logistic}\PYG{p}{(}\PYG{n}{loc}\PYG{p}{,} \PYG{n}{scale}\PYG{p}{,} \PYG{l+m+mi}{10000}\PYG{p}{)}
\PYG{g+gp}{\PYGZgt{}\PYGZgt{}\PYGZgt{} }\PYG{n}{count}\PYG{p}{,} \PYG{n}{bins}\PYG{p}{,} \PYG{n}{ignored} \PYG{o}{=} \PYG{n}{plt}\PYG{o}{.}\PYG{n}{hist}\PYG{p}{(}\PYG{n}{s}\PYG{p}{,} \PYG{n}{bins}\PYG{o}{=}\PYG{l+m+mi}{50}\PYG{p}{)}
\end{Verbatim}

\#   plot against distribution

\begin{Verbatim}[commandchars=\\\{\}]
\PYG{g+gp}{\PYGZgt{}\PYGZgt{}\PYGZgt{} }\PYG{k}{def} \PYG{n+nf}{logist}\PYG{p}{(}\PYG{n}{x}\PYG{p}{,} \PYG{n}{loc}\PYG{p}{,} \PYG{n}{scale}\PYG{p}{)}\PYG{p}{:}
\PYG{g+gp}{... }    \PYG{k}{return} \PYG{n}{exp}\PYG{p}{(}\PYG{p}{(}\PYG{n}{loc}\PYG{o}{\PYGZhy{}}\PYG{n}{x}\PYG{p}{)}\PYG{o}{/}\PYG{n}{scale}\PYG{p}{)}\PYG{o}{/}\PYG{p}{(}\PYG{n}{scale}\PYG{o}{*}\PYG{p}{(}\PYG{l+m+mi}{1}\PYG{o}{+}\PYG{n}{exp}\PYG{p}{(}\PYG{p}{(}\PYG{n}{loc}\PYG{o}{\PYGZhy{}}\PYG{n}{x}\PYG{p}{)}\PYG{o}{/}\PYG{n}{scale}\PYG{p}{)}\PYG{p}{)}\PYG{o}{*}\PYG{o}{*}\PYG{l+m+mi}{2}\PYG{p}{)}
\PYG{g+gp}{\PYGZgt{}\PYGZgt{}\PYGZgt{} }\PYG{n}{plt}\PYG{o}{.}\PYG{n}{plot}\PYG{p}{(}\PYG{n}{bins}\PYG{p}{,} \PYG{n}{logist}\PYG{p}{(}\PYG{n}{bins}\PYG{p}{,} \PYG{n}{loc}\PYG{p}{,} \PYG{n}{scale}\PYG{p}{)}\PYG{o}{*}\PYG{n}{count}\PYG{o}{.}\PYG{n}{max}\PYG{p}{(}\PYG{p}{)}\PYG{o}{/}\PYGZbs{}
\PYG{g+gp}{... }\PYG{n}{logist}\PYG{p}{(}\PYG{n}{bins}\PYG{p}{,} \PYG{n}{loc}\PYG{p}{,} \PYG{n}{scale}\PYG{p}{)}\PYG{o}{.}\PYG{n}{max}\PYG{p}{(}\PYG{p}{)}\PYG{p}{)}
\PYG{g+gp}{\PYGZgt{}\PYGZgt{}\PYGZgt{} }\PYG{n}{plt}\PYG{o}{.}\PYG{n}{show}\PYG{p}{(}\PYG{p}{)}
\end{Verbatim}

\end{fulllineitems}

\index{lognormal() (in module acsFromWim2Carness)}

\begin{fulllineitems}
\phantomsection\label{acsFromWim2Carness:acsFromWim2Carness.lognormal}\pysiglinewithargsret{\code{acsFromWim2Carness.}\bfcode{lognormal}}{\emph{mean=0.0}, \emph{sigma=1.0}, \emph{size=None}}{}
Return samples drawn from a log-normal distribution.

Draw samples from a log-normal distribution with specified mean,
standard deviation, and array shape.  Note that the mean and standard
deviation are not the values for the distribution itself, but of the
underlying normal distribution it is derived from.
\begin{description}
\item[{mean}] \leavevmode{[}float{]}
Mean value of the underlying normal distribution

\item[{sigma}] \leavevmode{[}float, \textgreater{} 0.{]}
Standard deviation of the underlying normal distribution

\item[{size}] \leavevmode{[}tuple of ints{]}
Output shape.  If the given shape is, e.g., \code{(m, n, k)}, then
\code{m * n * k} samples are drawn.

\end{description}
\begin{description}
\item[{samples}] \leavevmode{[}ndarray or float{]}
The desired samples. An array of the same shape as \emph{size} if given,
if \emph{size} is None a float is returned.

\end{description}
\begin{description}
\item[{scipy.stats.lognorm}] \leavevmode{[}probability density function, distribution,{]}
cumulative density function, etc.

\end{description}

A variable \emph{x} has a log-normal distribution if \emph{log(x)} is normally
distributed.  The probability density function for the log-normal
distribution is:
\begin{gather}
\begin{split}p(x) = \frac{1}{\sigma x \sqrt{2\pi}}
e^{(-\frac{(ln(x)-\mu)^2}{2\sigma^2})}\end{split}\notag
\end{gather}
where \(\mu\) is the mean and \(\sigma\) is the standard
deviation of the normally distributed logarithm of the variable.
A log-normal distribution results if a random variable is the \emph{product}
of a large number of independent, identically-distributed variables in
the same way that a normal distribution results if the variable is the
\emph{sum} of a large number of independent, identically-distributed
variables.

Limpert, E., Stahel, W. A., and Abbt, M., ``Log-normal Distributions
across the Sciences: Keys and Clues,'' \emph{BioScience}, Vol. 51, No. 5,
May, 2001.  \href{http://stat.ethz.ch/~stahel/lognormal/bioscience.pdf}{http://stat.ethz.ch/\textasciitilde{}stahel/lognormal/bioscience.pdf}

Reiss, R.D. and Thomas, M., \emph{Statistical Analysis of Extreme Values},
Basel: Birkhauser Verlag, 2001, pp. 31-32.

Draw samples from the distribution:

\begin{Verbatim}[commandchars=\\\{\}]
\PYG{g+gp}{\PYGZgt{}\PYGZgt{}\PYGZgt{} }\PYG{n}{mu}\PYG{p}{,} \PYG{n}{sigma} \PYG{o}{=} \PYG{l+m+mf}{3.}\PYG{p}{,} \PYG{l+m+mf}{1.} \PYG{c}{\PYGZsh{} mean and standard deviation}
\PYG{g+gp}{\PYGZgt{}\PYGZgt{}\PYGZgt{} }\PYG{n}{s} \PYG{o}{=} \PYG{n}{np}\PYG{o}{.}\PYG{n}{random}\PYG{o}{.}\PYG{n}{lognormal}\PYG{p}{(}\PYG{n}{mu}\PYG{p}{,} \PYG{n}{sigma}\PYG{p}{,} \PYG{l+m+mi}{1000}\PYG{p}{)}
\end{Verbatim}

Display the histogram of the samples, along with
the probability density function:

\begin{Verbatim}[commandchars=\\\{\}]
\PYG{g+gp}{\PYGZgt{}\PYGZgt{}\PYGZgt{} }\PYG{k+kn}{import} \PYG{n+nn}{matplotlib.pyplot} \PYG{k+kn}{as} \PYG{n+nn}{plt}
\PYG{g+gp}{\PYGZgt{}\PYGZgt{}\PYGZgt{} }\PYG{n}{count}\PYG{p}{,} \PYG{n}{bins}\PYG{p}{,} \PYG{n}{ignored} \PYG{o}{=} \PYG{n}{plt}\PYG{o}{.}\PYG{n}{hist}\PYG{p}{(}\PYG{n}{s}\PYG{p}{,} \PYG{l+m+mi}{100}\PYG{p}{,} \PYG{n}{normed}\PYG{o}{=}\PYG{n+nb+bp}{True}\PYG{p}{,} \PYG{n}{align}\PYG{o}{=}\PYG{l+s}{\PYGZsq{}}\PYG{l+s}{mid}\PYG{l+s}{\PYGZsq{}}\PYG{p}{)}
\end{Verbatim}

\begin{Verbatim}[commandchars=\\\{\}]
\PYG{g+gp}{\PYGZgt{}\PYGZgt{}\PYGZgt{} }\PYG{n}{x} \PYG{o}{=} \PYG{n}{np}\PYG{o}{.}\PYG{n}{linspace}\PYG{p}{(}\PYG{n+nb}{min}\PYG{p}{(}\PYG{n}{bins}\PYG{p}{)}\PYG{p}{,} \PYG{n+nb}{max}\PYG{p}{(}\PYG{n}{bins}\PYG{p}{)}\PYG{p}{,} \PYG{l+m+mi}{10000}\PYG{p}{)}
\PYG{g+gp}{\PYGZgt{}\PYGZgt{}\PYGZgt{} }\PYG{n}{pdf} \PYG{o}{=} \PYG{p}{(}\PYG{n}{np}\PYG{o}{.}\PYG{n}{exp}\PYG{p}{(}\PYG{o}{\PYGZhy{}}\PYG{p}{(}\PYG{n}{np}\PYG{o}{.}\PYG{n}{log}\PYG{p}{(}\PYG{n}{x}\PYG{p}{)} \PYG{o}{\PYGZhy{}} \PYG{n}{mu}\PYG{p}{)}\PYG{o}{*}\PYG{o}{*}\PYG{l+m+mi}{2} \PYG{o}{/} \PYG{p}{(}\PYG{l+m+mi}{2} \PYG{o}{*} \PYG{n}{sigma}\PYG{o}{*}\PYG{o}{*}\PYG{l+m+mi}{2}\PYG{p}{)}\PYG{p}{)}
\PYG{g+gp}{... }       \PYG{o}{/} \PYG{p}{(}\PYG{n}{x} \PYG{o}{*} \PYG{n}{sigma} \PYG{o}{*} \PYG{n}{np}\PYG{o}{.}\PYG{n}{sqrt}\PYG{p}{(}\PYG{l+m+mi}{2} \PYG{o}{*} \PYG{n}{np}\PYG{o}{.}\PYG{n}{pi}\PYG{p}{)}\PYG{p}{)}\PYG{p}{)}
\end{Verbatim}

\begin{Verbatim}[commandchars=\\\{\}]
\PYG{g+gp}{\PYGZgt{}\PYGZgt{}\PYGZgt{} }\PYG{n}{plt}\PYG{o}{.}\PYG{n}{plot}\PYG{p}{(}\PYG{n}{x}\PYG{p}{,} \PYG{n}{pdf}\PYG{p}{,} \PYG{n}{linewidth}\PYG{o}{=}\PYG{l+m+mi}{2}\PYG{p}{,} \PYG{n}{color}\PYG{o}{=}\PYG{l+s}{\PYGZsq{}}\PYG{l+s}{r}\PYG{l+s}{\PYGZsq{}}\PYG{p}{)}
\PYG{g+gp}{\PYGZgt{}\PYGZgt{}\PYGZgt{} }\PYG{n}{plt}\PYG{o}{.}\PYG{n}{axis}\PYG{p}{(}\PYG{l+s}{\PYGZsq{}}\PYG{l+s}{tight}\PYG{l+s}{\PYGZsq{}}\PYG{p}{)}
\PYG{g+gp}{\PYGZgt{}\PYGZgt{}\PYGZgt{} }\PYG{n}{plt}\PYG{o}{.}\PYG{n}{show}\PYG{p}{(}\PYG{p}{)}
\end{Verbatim}

Demonstrate that taking the products of random samples from a uniform
distribution can be fit well by a log-normal probability density function.

\begin{Verbatim}[commandchars=\\\{\}]
\PYG{g+gp}{\PYGZgt{}\PYGZgt{}\PYGZgt{} }\PYG{c}{\PYGZsh{} Generate a thousand samples: each is the product of 100 random}
\PYG{g+gp}{\PYGZgt{}\PYGZgt{}\PYGZgt{} }\PYG{c}{\PYGZsh{} values, drawn from a normal distribution.}
\PYG{g+gp}{\PYGZgt{}\PYGZgt{}\PYGZgt{} }\PYG{n}{b} \PYG{o}{=} \PYG{p}{[}\PYG{p}{]}
\PYG{g+gp}{\PYGZgt{}\PYGZgt{}\PYGZgt{} }\PYG{k}{for} \PYG{n}{i} \PYG{o+ow}{in} \PYG{n+nb}{range}\PYG{p}{(}\PYG{l+m+mi}{1000}\PYG{p}{)}\PYG{p}{:}
\PYG{g+gp}{... }   \PYG{n}{a} \PYG{o}{=} \PYG{l+m+mf}{10.} \PYG{o}{+} \PYG{n}{np}\PYG{o}{.}\PYG{n}{random}\PYG{o}{.}\PYG{n}{random}\PYG{p}{(}\PYG{l+m+mi}{100}\PYG{p}{)}
\PYG{g+gp}{... }   \PYG{n}{b}\PYG{o}{.}\PYG{n}{append}\PYG{p}{(}\PYG{n}{np}\PYG{o}{.}\PYG{n}{product}\PYG{p}{(}\PYG{n}{a}\PYG{p}{)}\PYG{p}{)}
\end{Verbatim}

\begin{Verbatim}[commandchars=\\\{\}]
\PYG{g+gp}{\PYGZgt{}\PYGZgt{}\PYGZgt{} }\PYG{n}{b} \PYG{o}{=} \PYG{n}{np}\PYG{o}{.}\PYG{n}{array}\PYG{p}{(}\PYG{n}{b}\PYG{p}{)} \PYG{o}{/} \PYG{n}{np}\PYG{o}{.}\PYG{n}{min}\PYG{p}{(}\PYG{n}{b}\PYG{p}{)} \PYG{c}{\PYGZsh{} scale values to be positive}
\PYG{g+gp}{\PYGZgt{}\PYGZgt{}\PYGZgt{} }\PYG{n}{count}\PYG{p}{,} \PYG{n}{bins}\PYG{p}{,} \PYG{n}{ignored} \PYG{o}{=} \PYG{n}{plt}\PYG{o}{.}\PYG{n}{hist}\PYG{p}{(}\PYG{n}{b}\PYG{p}{,} \PYG{l+m+mi}{100}\PYG{p}{,} \PYG{n}{normed}\PYG{o}{=}\PYG{n+nb+bp}{True}\PYG{p}{,} \PYG{n}{align}\PYG{o}{=}\PYG{l+s}{\PYGZsq{}}\PYG{l+s}{center}\PYG{l+s}{\PYGZsq{}}\PYG{p}{)}
\PYG{g+gp}{\PYGZgt{}\PYGZgt{}\PYGZgt{} }\PYG{n}{sigma} \PYG{o}{=} \PYG{n}{np}\PYG{o}{.}\PYG{n}{std}\PYG{p}{(}\PYG{n}{np}\PYG{o}{.}\PYG{n}{log}\PYG{p}{(}\PYG{n}{b}\PYG{p}{)}\PYG{p}{)}
\PYG{g+gp}{\PYGZgt{}\PYGZgt{}\PYGZgt{} }\PYG{n}{mu} \PYG{o}{=} \PYG{n}{np}\PYG{o}{.}\PYG{n}{mean}\PYG{p}{(}\PYG{n}{np}\PYG{o}{.}\PYG{n}{log}\PYG{p}{(}\PYG{n}{b}\PYG{p}{)}\PYG{p}{)}
\end{Verbatim}

\begin{Verbatim}[commandchars=\\\{\}]
\PYG{g+gp}{\PYGZgt{}\PYGZgt{}\PYGZgt{} }\PYG{n}{x} \PYG{o}{=} \PYG{n}{np}\PYG{o}{.}\PYG{n}{linspace}\PYG{p}{(}\PYG{n+nb}{min}\PYG{p}{(}\PYG{n}{bins}\PYG{p}{)}\PYG{p}{,} \PYG{n+nb}{max}\PYG{p}{(}\PYG{n}{bins}\PYG{p}{)}\PYG{p}{,} \PYG{l+m+mi}{10000}\PYG{p}{)}
\PYG{g+gp}{\PYGZgt{}\PYGZgt{}\PYGZgt{} }\PYG{n}{pdf} \PYG{o}{=} \PYG{p}{(}\PYG{n}{np}\PYG{o}{.}\PYG{n}{exp}\PYG{p}{(}\PYG{o}{\PYGZhy{}}\PYG{p}{(}\PYG{n}{np}\PYG{o}{.}\PYG{n}{log}\PYG{p}{(}\PYG{n}{x}\PYG{p}{)} \PYG{o}{\PYGZhy{}} \PYG{n}{mu}\PYG{p}{)}\PYG{o}{*}\PYG{o}{*}\PYG{l+m+mi}{2} \PYG{o}{/} \PYG{p}{(}\PYG{l+m+mi}{2} \PYG{o}{*} \PYG{n}{sigma}\PYG{o}{*}\PYG{o}{*}\PYG{l+m+mi}{2}\PYG{p}{)}\PYG{p}{)}
\PYG{g+gp}{... }       \PYG{o}{/} \PYG{p}{(}\PYG{n}{x} \PYG{o}{*} \PYG{n}{sigma} \PYG{o}{*} \PYG{n}{np}\PYG{o}{.}\PYG{n}{sqrt}\PYG{p}{(}\PYG{l+m+mi}{2} \PYG{o}{*} \PYG{n}{np}\PYG{o}{.}\PYG{n}{pi}\PYG{p}{)}\PYG{p}{)}\PYG{p}{)}
\end{Verbatim}

\begin{Verbatim}[commandchars=\\\{\}]
\PYG{g+gp}{\PYGZgt{}\PYGZgt{}\PYGZgt{} }\PYG{n}{plt}\PYG{o}{.}\PYG{n}{plot}\PYG{p}{(}\PYG{n}{x}\PYG{p}{,} \PYG{n}{pdf}\PYG{p}{,} \PYG{n}{color}\PYG{o}{=}\PYG{l+s}{\PYGZsq{}}\PYG{l+s}{r}\PYG{l+s}{\PYGZsq{}}\PYG{p}{,} \PYG{n}{linewidth}\PYG{o}{=}\PYG{l+m+mi}{2}\PYG{p}{)}
\PYG{g+gp}{\PYGZgt{}\PYGZgt{}\PYGZgt{} }\PYG{n}{plt}\PYG{o}{.}\PYG{n}{show}\PYG{p}{(}\PYG{p}{)}
\end{Verbatim}

\end{fulllineitems}

\index{logseries() (in module acsFromWim2Carness)}

\begin{fulllineitems}
\phantomsection\label{acsFromWim2Carness:acsFromWim2Carness.logseries}\pysiglinewithargsret{\code{acsFromWim2Carness.}\bfcode{logseries}}{\emph{p}, \emph{size=None}}{}
Draw samples from a Logarithmic Series distribution.

Samples are drawn from a Log Series distribution with specified
parameter, p (probability, 0 \textless{} p \textless{} 1).

loc : float

scale : float \textgreater{} 0.
\begin{description}
\item[{size}] \leavevmode{[}\{tuple, int\}{]}
Output shape.  If the given shape is, e.g., \code{(m, n, k)}, then
\code{m * n * k} samples are drawn.

\end{description}
\begin{description}
\item[{samples}] \leavevmode{[}\{ndarray, scalar\}{]}
where the values are all integers in  {[}0, n{]}.

\end{description}
\begin{description}
\item[{scipy.stats.distributions.logser}] \leavevmode{[}probability density function,{]}
distribution or cumulative density function, etc.

\end{description}

The probability density for the Log Series distribution is
\begin{gather}
\begin{split}P(k) = \frac{-p^k}{k \ln(1-p)},\end{split}\notag
\end{gather}
where p = probability.

The Log Series distribution is frequently used to represent species
richness and occurrence, first proposed by Fisher, Corbet, and
Williams in 1943 {[}2{]}.  It may also be used to model the numbers of
occupants seen in cars {[}3{]}.

Draw samples from the distribution:

\begin{Verbatim}[commandchars=\\\{\}]
\PYG{g+gp}{\PYGZgt{}\PYGZgt{}\PYGZgt{} }\PYG{n}{a} \PYG{o}{=} \PYG{o}{.}\PYG{l+m+mi}{6}
\PYG{g+gp}{\PYGZgt{}\PYGZgt{}\PYGZgt{} }\PYG{n}{s} \PYG{o}{=} \PYG{n}{np}\PYG{o}{.}\PYG{n}{random}\PYG{o}{.}\PYG{n}{logseries}\PYG{p}{(}\PYG{n}{a}\PYG{p}{,} \PYG{l+m+mi}{10000}\PYG{p}{)}
\PYG{g+gp}{\PYGZgt{}\PYGZgt{}\PYGZgt{} }\PYG{n}{count}\PYG{p}{,} \PYG{n}{bins}\PYG{p}{,} \PYG{n}{ignored} \PYG{o}{=} \PYG{n}{plt}\PYG{o}{.}\PYG{n}{hist}\PYG{p}{(}\PYG{n}{s}\PYG{p}{)}
\end{Verbatim}

\#   plot against distribution

\begin{Verbatim}[commandchars=\\\{\}]
\PYG{g+gp}{\PYGZgt{}\PYGZgt{}\PYGZgt{} }\PYG{k}{def} \PYG{n+nf}{logseries}\PYG{p}{(}\PYG{n}{k}\PYG{p}{,} \PYG{n}{p}\PYG{p}{)}\PYG{p}{:}
\PYG{g+gp}{... }    \PYG{k}{return} \PYG{o}{\PYGZhy{}}\PYG{n}{p}\PYG{o}{*}\PYG{o}{*}\PYG{n}{k}\PYG{o}{/}\PYG{p}{(}\PYG{n}{k}\PYG{o}{*}\PYG{n}{log}\PYG{p}{(}\PYG{l+m+mi}{1}\PYG{o}{\PYGZhy{}}\PYG{n}{p}\PYG{p}{)}\PYG{p}{)}
\PYG{g+gp}{\PYGZgt{}\PYGZgt{}\PYGZgt{} }\PYG{n}{plt}\PYG{o}{.}\PYG{n}{plot}\PYG{p}{(}\PYG{n}{bins}\PYG{p}{,} \PYG{n}{logseries}\PYG{p}{(}\PYG{n}{bins}\PYG{p}{,} \PYG{n}{a}\PYG{p}{)}\PYG{o}{*}\PYG{n}{count}\PYG{o}{.}\PYG{n}{max}\PYG{p}{(}\PYG{p}{)}\PYG{o}{/}
\PYG{g+go}{             logseries(bins, a).max(), \PYGZsq{}r\PYGZsq{})}
\PYG{g+gp}{\PYGZgt{}\PYGZgt{}\PYGZgt{} }\PYG{n}{plt}\PYG{o}{.}\PYG{n}{show}\PYG{p}{(}\PYG{p}{)}
\end{Verbatim}

\end{fulllineitems}

\index{multinomial() (in module acsFromWim2Carness)}

\begin{fulllineitems}
\phantomsection\label{acsFromWim2Carness:acsFromWim2Carness.multinomial}\pysiglinewithargsret{\code{acsFromWim2Carness.}\bfcode{multinomial}}{\emph{n}, \emph{pvals}, \emph{size=None}}{}
Draw samples from a multinomial distribution.

The multinomial distribution is a multivariate generalisation of the
binomial distribution.  Take an experiment with one of \code{p}
possible outcomes.  An example of such an experiment is throwing a dice,
where the outcome can be 1 through 6.  Each sample drawn from the
distribution represents \emph{n} such experiments.  Its values,
\code{X\_i = {[}X\_0, X\_1, ..., X\_p{]}}, represent the number of times the outcome
was \code{i}.
\begin{description}
\item[{n}] \leavevmode{[}int{]}
Number of experiments.

\item[{pvals}] \leavevmode{[}sequence of floats, length p{]}
Probabilities of each of the \code{p} different outcomes.  These
should sum to 1 (however, the last element is always assumed to
account for the remaining probability, as long as
\code{sum(pvals{[}:-1{]}) \textless{}= 1)}.

\item[{size}] \leavevmode{[}tuple of ints{]}
Given a \emph{size} of \code{(M, N, K)}, then \code{M*N*K} samples are drawn,
and the output shape becomes \code{(M, N, K, p)}, since each sample
has shape \code{(p,)}.

\end{description}

Throw a dice 20 times:

\begin{Verbatim}[commandchars=\\\{\}]
\PYG{g+gp}{\PYGZgt{}\PYGZgt{}\PYGZgt{} }\PYG{n}{np}\PYG{o}{.}\PYG{n}{random}\PYG{o}{.}\PYG{n}{multinomial}\PYG{p}{(}\PYG{l+m+mi}{20}\PYG{p}{,} \PYG{p}{[}\PYG{l+m+mi}{1}\PYG{o}{/}\PYG{l+m+mf}{6.}\PYG{p}{]}\PYG{o}{*}\PYG{l+m+mi}{6}\PYG{p}{,} \PYG{n}{size}\PYG{o}{=}\PYG{l+m+mi}{1}\PYG{p}{)}
\PYG{g+go}{array([[4, 1, 7, 5, 2, 1]])}
\end{Verbatim}

It landed 4 times on 1, once on 2, etc.

Now, throw the dice 20 times, and 20 times again:

\begin{Verbatim}[commandchars=\\\{\}]
\PYG{g+gp}{\PYGZgt{}\PYGZgt{}\PYGZgt{} }\PYG{n}{np}\PYG{o}{.}\PYG{n}{random}\PYG{o}{.}\PYG{n}{multinomial}\PYG{p}{(}\PYG{l+m+mi}{20}\PYG{p}{,} \PYG{p}{[}\PYG{l+m+mi}{1}\PYG{o}{/}\PYG{l+m+mf}{6.}\PYG{p}{]}\PYG{o}{*}\PYG{l+m+mi}{6}\PYG{p}{,} \PYG{n}{size}\PYG{o}{=}\PYG{l+m+mi}{2}\PYG{p}{)}
\PYG{g+go}{array([[3, 4, 3, 3, 4, 3],}
\PYG{g+go}{       [2, 4, 3, 4, 0, 7]])}
\end{Verbatim}

For the first run, we threw 3 times 1, 4 times 2, etc.  For the second,
we threw 2 times 1, 4 times 2, etc.

A loaded dice is more likely to land on number 6:

\begin{Verbatim}[commandchars=\\\{\}]
\PYG{g+gp}{\PYGZgt{}\PYGZgt{}\PYGZgt{} }\PYG{n}{np}\PYG{o}{.}\PYG{n}{random}\PYG{o}{.}\PYG{n}{multinomial}\PYG{p}{(}\PYG{l+m+mi}{100}\PYG{p}{,} \PYG{p}{[}\PYG{l+m+mi}{1}\PYG{o}{/}\PYG{l+m+mf}{7.}\PYG{p}{]}\PYG{o}{*}\PYG{l+m+mi}{5}\PYG{p}{)}
\PYG{g+go}{array([13, 16, 13, 16, 42])}
\end{Verbatim}

\end{fulllineitems}

\index{multivariate\_normal() (in module acsFromWim2Carness)}

\begin{fulllineitems}
\phantomsection\label{acsFromWim2Carness:acsFromWim2Carness.multivariate_normal}\pysiglinewithargsret{\code{acsFromWim2Carness.}\bfcode{multivariate\_normal}}{\emph{mean}, \emph{cov}\optional{, \emph{size}}}{}
Draw random samples from a multivariate normal distribution.

The multivariate normal, multinormal or Gaussian distribution is a
generalization of the one-dimensional normal distribution to higher
dimensions.  Such a distribution is specified by its mean and
covariance matrix.  These parameters are analogous to the mean
(average or ``center'') and variance (standard deviation, or ``width,''
squared) of the one-dimensional normal distribution.
\begin{description}
\item[{mean}] \leavevmode{[}1-D array\_like, of length N{]}
Mean of the N-dimensional distribution.

\item[{cov}] \leavevmode{[}2-D array\_like, of shape (N, N){]}
Covariance matrix of the distribution.  Must be symmetric and
positive semi-definite for ``physically meaningful'' results.

\item[{size}] \leavevmode{[}int or tuple of ints, optional{]}
Given a shape of, for example, \code{(m,n,k)}, \code{m*n*k} samples are
generated, and packed in an \emph{m}-by-\emph{n}-by-\emph{k} arrangement.  Because
each sample is \emph{N}-dimensional, the output shape is \code{(m,n,k,N)}.
If no shape is specified, a single (\emph{N}-D) sample is returned.

\end{description}
\begin{description}
\item[{out}] \leavevmode{[}ndarray{]}
The drawn samples, of shape \emph{size}, if that was provided.  If not,
the shape is \code{(N,)}.

In other words, each entry \code{out{[}i,j,...,:{]}} is an N-dimensional
value drawn from the distribution.

\end{description}

The mean is a coordinate in N-dimensional space, which represents the
location where samples are most likely to be generated.  This is
analogous to the peak of the bell curve for the one-dimensional or
univariate normal distribution.

Covariance indicates the level to which two variables vary together.
From the multivariate normal distribution, we draw N-dimensional
samples, \(X = [x_1, x_2, ... x_N]\).  The covariance matrix
element \(C_{ij}\) is the covariance of \(x_i\) and \(x_j\).
The element \(C_{ii}\) is the variance of \(x_i\) (i.e. its
``spread'').

Instead of specifying the full covariance matrix, popular
approximations include:
\begin{itemize}
\item {} 
Spherical covariance (\emph{cov} is a multiple of the identity matrix)

\item {} 
Diagonal covariance (\emph{cov} has non-negative elements, and only on
the diagonal)

\end{itemize}

This geometrical property can be seen in two dimensions by plotting
generated data-points:

\begin{Verbatim}[commandchars=\\\{\}]
\PYG{g+gp}{\PYGZgt{}\PYGZgt{}\PYGZgt{} }\PYG{n}{mean} \PYG{o}{=} \PYG{p}{[}\PYG{l+m+mi}{0}\PYG{p}{,}\PYG{l+m+mi}{0}\PYG{p}{]}
\PYG{g+gp}{\PYGZgt{}\PYGZgt{}\PYGZgt{} }\PYG{n}{cov} \PYG{o}{=} \PYG{p}{[}\PYG{p}{[}\PYG{l+m+mi}{1}\PYG{p}{,}\PYG{l+m+mi}{0}\PYG{p}{]}\PYG{p}{,}\PYG{p}{[}\PYG{l+m+mi}{0}\PYG{p}{,}\PYG{l+m+mi}{100}\PYG{p}{]}\PYG{p}{]} \PYG{c}{\PYGZsh{} diagonal covariance, points lie on x or y\PYGZhy{}axis}
\end{Verbatim}

\begin{Verbatim}[commandchars=\\\{\}]
\PYG{g+gp}{\PYGZgt{}\PYGZgt{}\PYGZgt{} }\PYG{k+kn}{import} \PYG{n+nn}{matplotlib.pyplot} \PYG{k+kn}{as} \PYG{n+nn}{plt}
\PYG{g+gp}{\PYGZgt{}\PYGZgt{}\PYGZgt{} }\PYG{n}{x}\PYG{p}{,}\PYG{n}{y} \PYG{o}{=} \PYG{n}{np}\PYG{o}{.}\PYG{n}{random}\PYG{o}{.}\PYG{n}{multivariate\PYGZus{}normal}\PYG{p}{(}\PYG{n}{mean}\PYG{p}{,}\PYG{n}{cov}\PYG{p}{,}\PYG{l+m+mi}{5000}\PYG{p}{)}\PYG{o}{.}\PYG{n}{T}
\PYG{g+gp}{\PYGZgt{}\PYGZgt{}\PYGZgt{} }\PYG{n}{plt}\PYG{o}{.}\PYG{n}{plot}\PYG{p}{(}\PYG{n}{x}\PYG{p}{,}\PYG{n}{y}\PYG{p}{,}\PYG{l+s}{\PYGZsq{}}\PYG{l+s}{x}\PYG{l+s}{\PYGZsq{}}\PYG{p}{)}\PYG{p}{;} \PYG{n}{plt}\PYG{o}{.}\PYG{n}{axis}\PYG{p}{(}\PYG{l+s}{\PYGZsq{}}\PYG{l+s}{equal}\PYG{l+s}{\PYGZsq{}}\PYG{p}{)}\PYG{p}{;} \PYG{n}{plt}\PYG{o}{.}\PYG{n}{show}\PYG{p}{(}\PYG{p}{)}
\end{Verbatim}

Note that the covariance matrix must be non-negative definite.

Papoulis, A., \emph{Probability, Random Variables, and Stochastic Processes},
3rd ed., New York: McGraw-Hill, 1991.

Duda, R. O., Hart, P. E., and Stork, D. G., \emph{Pattern Classification},
2nd ed., New York: Wiley, 2001.

\begin{Verbatim}[commandchars=\\\{\}]
\PYG{g+gp}{\PYGZgt{}\PYGZgt{}\PYGZgt{} }\PYG{n}{mean} \PYG{o}{=} \PYG{p}{(}\PYG{l+m+mi}{1}\PYG{p}{,}\PYG{l+m+mi}{2}\PYG{p}{)}
\PYG{g+gp}{\PYGZgt{}\PYGZgt{}\PYGZgt{} }\PYG{n}{cov} \PYG{o}{=} \PYG{p}{[}\PYG{p}{[}\PYG{l+m+mi}{1}\PYG{p}{,}\PYG{l+m+mi}{0}\PYG{p}{]}\PYG{p}{,}\PYG{p}{[}\PYG{l+m+mi}{1}\PYG{p}{,}\PYG{l+m+mi}{0}\PYG{p}{]}\PYG{p}{]}
\PYG{g+gp}{\PYGZgt{}\PYGZgt{}\PYGZgt{} }\PYG{n}{x} \PYG{o}{=} \PYG{n}{np}\PYG{o}{.}\PYG{n}{random}\PYG{o}{.}\PYG{n}{multivariate\PYGZus{}normal}\PYG{p}{(}\PYG{n}{mean}\PYG{p}{,}\PYG{n}{cov}\PYG{p}{,}\PYG{p}{(}\PYG{l+m+mi}{3}\PYG{p}{,}\PYG{l+m+mi}{3}\PYG{p}{)}\PYG{p}{)}
\PYG{g+gp}{\PYGZgt{}\PYGZgt{}\PYGZgt{} }\PYG{n}{x}\PYG{o}{.}\PYG{n}{shape}
\PYG{g+go}{(3, 3, 2)}
\end{Verbatim}

The following is probably true, given that 0.6 is roughly twice the
standard deviation:

\begin{Verbatim}[commandchars=\\\{\}]
\PYG{g+gp}{\PYGZgt{}\PYGZgt{}\PYGZgt{} }\PYG{k}{print} \PYG{n+nb}{list}\PYG{p}{(} \PYG{p}{(}\PYG{n}{x}\PYG{p}{[}\PYG{l+m+mi}{0}\PYG{p}{,}\PYG{l+m+mi}{0}\PYG{p}{,}\PYG{p}{:}\PYG{p}{]} \PYG{o}{\PYGZhy{}} \PYG{n}{mean}\PYG{p}{)} \PYG{o}{\PYGZlt{}} \PYG{l+m+mf}{0.6} \PYG{p}{)}
\PYG{g+go}{[True, True]}
\end{Verbatim}

\end{fulllineitems}

\index{negative\_binomial() (in module acsFromWim2Carness)}

\begin{fulllineitems}
\phantomsection\label{acsFromWim2Carness:acsFromWim2Carness.negative_binomial}\pysiglinewithargsret{\code{acsFromWim2Carness.}\bfcode{negative\_binomial}}{\emph{n}, \emph{p}, \emph{size=None}}{}
Draw samples from a negative\_binomial distribution.

Samples are drawn from a negative\_Binomial distribution with specified
parameters, \emph{n} trials and \emph{p} probability of success where \emph{n} is an
integer \textgreater{} 0 and \emph{p} is in the interval {[}0, 1{]}.
\begin{description}
\item[{n}] \leavevmode{[}int{]}
Parameter, \textgreater{} 0.

\item[{p}] \leavevmode{[}float{]}
Parameter, \textgreater{}= 0 and \textless{}=1.

\item[{size}] \leavevmode{[}int or tuple of ints{]}
Output shape. If the given shape is, e.g., \code{(m, n, k)}, then
\code{m * n * k} samples are drawn.

\end{description}
\begin{description}
\item[{samples}] \leavevmode{[}int or ndarray of ints{]}
Drawn samples.

\end{description}

The probability density for the Negative Binomial distribution is
\begin{gather}
\begin{split}P(N;n,p) = \binom{N+n-1}{n-1}p^{n}(1-p)^{N},\end{split}\notag
\end{gather}
where \(n-1\) is the number of successes, \(p\) is the probability
of success, and \(N+n-1\) is the number of trials.

The negative binomial distribution gives the probability of n-1 successes
and N failures in N+n-1 trials, and success on the (N+n)th trial.

If one throws a die repeatedly until the third time a ``1'' appears, then the
probability distribution of the number of non-``1''s that appear before the
third ``1'' is a negative binomial distribution.

Draw samples from the distribution:

A real world example. A company drills wild-cat oil exploration wells, each
with an estimated probability of success of 0.1.  What is the probability
of having one success for each successive well, that is what is the
probability of a single success after drilling 5 wells, after 6 wells,
etc.?

\begin{Verbatim}[commandchars=\\\{\}]
\PYG{g+gp}{\PYGZgt{}\PYGZgt{}\PYGZgt{} }\PYG{n}{s} \PYG{o}{=} \PYG{n}{np}\PYG{o}{.}\PYG{n}{random}\PYG{o}{.}\PYG{n}{negative\PYGZus{}binomial}\PYG{p}{(}\PYG{l+m+mi}{1}\PYG{p}{,} \PYG{l+m+mf}{0.1}\PYG{p}{,} \PYG{l+m+mi}{100000}\PYG{p}{)}
\PYG{g+gp}{\PYGZgt{}\PYGZgt{}\PYGZgt{} }\PYG{k}{for} \PYG{n}{i} \PYG{o+ow}{in} \PYG{n+nb}{range}\PYG{p}{(}\PYG{l+m+mi}{1}\PYG{p}{,} \PYG{l+m+mi}{11}\PYG{p}{)}\PYG{p}{:}
\PYG{g+gp}{... }   \PYG{n}{probability} \PYG{o}{=} \PYG{n+nb}{sum}\PYG{p}{(}\PYG{n}{s}\PYG{o}{\PYGZlt{}}\PYG{n}{i}\PYG{p}{)} \PYG{o}{/} \PYG{l+m+mf}{100000.}
\PYG{g+gp}{... }   \PYG{k}{print} \PYG{n}{i}\PYG{p}{,} \PYG{l+s}{\PYGZdq{}}\PYG{l+s}{wells drilled, probability of one success =}\PYG{l+s}{\PYGZdq{}}\PYG{p}{,} \PYG{n}{probability}
\end{Verbatim}

\end{fulllineitems}

\index{noncentral\_chisquare() (in module acsFromWim2Carness)}

\begin{fulllineitems}
\phantomsection\label{acsFromWim2Carness:acsFromWim2Carness.noncentral_chisquare}\pysiglinewithargsret{\code{acsFromWim2Carness.}\bfcode{noncentral\_chisquare}}{\emph{df}, \emph{nonc}, \emph{size=None}}{}
Draw samples from a noncentral chi-square distribution.

The noncentral \(\chi^2\) distribution is a generalisation of
the \(\chi^2\) distribution.
\begin{description}
\item[{df}] \leavevmode{[}int{]}
Degrees of freedom, should be \textgreater{}= 1.

\item[{nonc}] \leavevmode{[}float{]}
Non-centrality, should be \textgreater{} 0.

\item[{size}] \leavevmode{[}int or tuple of ints{]}
Shape of the output.

\end{description}

The probability density function for the noncentral Chi-square distribution
is
\begin{gather}
\begin{split}P(x;df,nonc) = \sum^{\infty}_{i=0}
\frac{e^{-nonc/2}(nonc/2)^{i}}{i!}P_{Y_{df+2i}}(x),\end{split}\notag
\end{gather}
where \(Y_{q}\) is the Chi-square with q degrees of freedom.

In Delhi (2007), it is noted that the noncentral chi-square is useful in
bombing and coverage problems, the probability of killing the point target
given by the noncentral chi-squared distribution.

Draw values from the distribution and plot the histogram

\begin{Verbatim}[commandchars=\\\{\}]
\PYG{g+gp}{\PYGZgt{}\PYGZgt{}\PYGZgt{} }\PYG{k+kn}{import} \PYG{n+nn}{matplotlib.pyplot} \PYG{k+kn}{as} \PYG{n+nn}{plt}
\PYG{g+gp}{\PYGZgt{}\PYGZgt{}\PYGZgt{} }\PYG{n}{values} \PYG{o}{=} \PYG{n}{plt}\PYG{o}{.}\PYG{n}{hist}\PYG{p}{(}\PYG{n}{np}\PYG{o}{.}\PYG{n}{random}\PYG{o}{.}\PYG{n}{noncentral\PYGZus{}chisquare}\PYG{p}{(}\PYG{l+m+mi}{3}\PYG{p}{,} \PYG{l+m+mi}{20}\PYG{p}{,} \PYG{l+m+mi}{100000}\PYG{p}{)}\PYG{p}{,}
\PYG{g+gp}{... }                  \PYG{n}{bins}\PYG{o}{=}\PYG{l+m+mi}{200}\PYG{p}{,} \PYG{n}{normed}\PYG{o}{=}\PYG{n+nb+bp}{True}\PYG{p}{)}
\PYG{g+gp}{\PYGZgt{}\PYGZgt{}\PYGZgt{} }\PYG{n}{plt}\PYG{o}{.}\PYG{n}{show}\PYG{p}{(}\PYG{p}{)}
\end{Verbatim}

Draw values from a noncentral chisquare with very small noncentrality,
and compare to a chisquare.

\begin{Verbatim}[commandchars=\\\{\}]
\PYG{g+gp}{\PYGZgt{}\PYGZgt{}\PYGZgt{} }\PYG{n}{plt}\PYG{o}{.}\PYG{n}{figure}\PYG{p}{(}\PYG{p}{)}
\PYG{g+gp}{\PYGZgt{}\PYGZgt{}\PYGZgt{} }\PYG{n}{values} \PYG{o}{=} \PYG{n}{plt}\PYG{o}{.}\PYG{n}{hist}\PYG{p}{(}\PYG{n}{np}\PYG{o}{.}\PYG{n}{random}\PYG{o}{.}\PYG{n}{noncentral\PYGZus{}chisquare}\PYG{p}{(}\PYG{l+m+mi}{3}\PYG{p}{,} \PYG{o}{.}\PYG{l+m+mo}{0000001}\PYG{p}{,} \PYG{l+m+mi}{100000}\PYG{p}{)}\PYG{p}{,}
\PYG{g+gp}{... }                  \PYG{n}{bins}\PYG{o}{=}\PYG{n}{np}\PYG{o}{.}\PYG{n}{arange}\PYG{p}{(}\PYG{l+m+mf}{0.}\PYG{p}{,} \PYG{l+m+mi}{25}\PYG{p}{,} \PYG{o}{.}\PYG{l+m+mi}{1}\PYG{p}{)}\PYG{p}{,} \PYG{n}{normed}\PYG{o}{=}\PYG{n+nb+bp}{True}\PYG{p}{)}
\PYG{g+gp}{\PYGZgt{}\PYGZgt{}\PYGZgt{} }\PYG{n}{values2} \PYG{o}{=} \PYG{n}{plt}\PYG{o}{.}\PYG{n}{hist}\PYG{p}{(}\PYG{n}{np}\PYG{o}{.}\PYG{n}{random}\PYG{o}{.}\PYG{n}{chisquare}\PYG{p}{(}\PYG{l+m+mi}{3}\PYG{p}{,} \PYG{l+m+mi}{100000}\PYG{p}{)}\PYG{p}{,}
\PYG{g+gp}{... }                   \PYG{n}{bins}\PYG{o}{=}\PYG{n}{np}\PYG{o}{.}\PYG{n}{arange}\PYG{p}{(}\PYG{l+m+mf}{0.}\PYG{p}{,} \PYG{l+m+mi}{25}\PYG{p}{,} \PYG{o}{.}\PYG{l+m+mi}{1}\PYG{p}{)}\PYG{p}{,} \PYG{n}{normed}\PYG{o}{=}\PYG{n+nb+bp}{True}\PYG{p}{)}
\PYG{g+gp}{\PYGZgt{}\PYGZgt{}\PYGZgt{} }\PYG{n}{plt}\PYG{o}{.}\PYG{n}{plot}\PYG{p}{(}\PYG{n}{values}\PYG{p}{[}\PYG{l+m+mi}{1}\PYG{p}{]}\PYG{p}{[}\PYG{l+m+mi}{0}\PYG{p}{:}\PYG{o}{\PYGZhy{}}\PYG{l+m+mi}{1}\PYG{p}{]}\PYG{p}{,} \PYG{n}{values}\PYG{p}{[}\PYG{l+m+mi}{0}\PYG{p}{]}\PYG{o}{\PYGZhy{}}\PYG{n}{values2}\PYG{p}{[}\PYG{l+m+mi}{0}\PYG{p}{]}\PYG{p}{,} \PYG{l+s}{\PYGZsq{}}\PYG{l+s}{ob}\PYG{l+s}{\PYGZsq{}}\PYG{p}{)}
\PYG{g+gp}{\PYGZgt{}\PYGZgt{}\PYGZgt{} }\PYG{n}{plt}\PYG{o}{.}\PYG{n}{show}\PYG{p}{(}\PYG{p}{)}
\end{Verbatim}

Demonstrate how large values of non-centrality lead to a more symmetric
distribution.

\begin{Verbatim}[commandchars=\\\{\}]
\PYG{g+gp}{\PYGZgt{}\PYGZgt{}\PYGZgt{} }\PYG{n}{plt}\PYG{o}{.}\PYG{n}{figure}\PYG{p}{(}\PYG{p}{)}
\PYG{g+gp}{\PYGZgt{}\PYGZgt{}\PYGZgt{} }\PYG{n}{values} \PYG{o}{=} \PYG{n}{plt}\PYG{o}{.}\PYG{n}{hist}\PYG{p}{(}\PYG{n}{np}\PYG{o}{.}\PYG{n}{random}\PYG{o}{.}\PYG{n}{noncentral\PYGZus{}chisquare}\PYG{p}{(}\PYG{l+m+mi}{3}\PYG{p}{,} \PYG{l+m+mi}{20}\PYG{p}{,} \PYG{l+m+mi}{100000}\PYG{p}{)}\PYG{p}{,}
\PYG{g+gp}{... }                  \PYG{n}{bins}\PYG{o}{=}\PYG{l+m+mi}{200}\PYG{p}{,} \PYG{n}{normed}\PYG{o}{=}\PYG{n+nb+bp}{True}\PYG{p}{)}
\PYG{g+gp}{\PYGZgt{}\PYGZgt{}\PYGZgt{} }\PYG{n}{plt}\PYG{o}{.}\PYG{n}{show}\PYG{p}{(}\PYG{p}{)}
\end{Verbatim}

\end{fulllineitems}

\index{noncentral\_f() (in module acsFromWim2Carness)}

\begin{fulllineitems}
\phantomsection\label{acsFromWim2Carness:acsFromWim2Carness.noncentral_f}\pysiglinewithargsret{\code{acsFromWim2Carness.}\bfcode{noncentral\_f}}{\emph{dfnum}, \emph{dfden}, \emph{nonc}, \emph{size=None}}{}
Draw samples from the noncentral F distribution.

Samples are drawn from an F distribution with specified parameters,
\emph{dfnum} (degrees of freedom in numerator) and \emph{dfden} (degrees of
freedom in denominator), where both parameters \textgreater{} 1.
\emph{nonc} is the non-centrality parameter.
\begin{description}
\item[{dfnum}] \leavevmode{[}int{]}
Parameter, should be \textgreater{} 1.

\item[{dfden}] \leavevmode{[}int{]}
Parameter, should be \textgreater{} 1.

\item[{nonc}] \leavevmode{[}float{]}
Parameter, should be \textgreater{}= 0.

\item[{size}] \leavevmode{[}int or tuple of ints{]}
Output shape. If the given shape is, e.g., \code{(m, n, k)}, then
\code{m * n * k} samples are drawn.

\end{description}
\begin{description}
\item[{samples}] \leavevmode{[}scalar or ndarray{]}
Drawn samples.

\end{description}

When calculating the power of an experiment (power = probability of
rejecting the null hypothesis when a specific alternative is true) the
non-central F statistic becomes important.  When the null hypothesis is
true, the F statistic follows a central F distribution. When the null
hypothesis is not true, then it follows a non-central F statistic.

Weisstein, Eric W. ``Noncentral F-Distribution.'' From MathWorld--A Wolfram
Web Resource.  \href{http://mathworld.wolfram.com/NoncentralF-Distribution.html}{http://mathworld.wolfram.com/NoncentralF-Distribution.html}

Wikipedia, ``Noncentral F distribution'',
\href{http://en.wikipedia.org/wiki/Noncentral\_F-distribution}{http://en.wikipedia.org/wiki/Noncentral\_F-distribution}

In a study, testing for a specific alternative to the null hypothesis
requires use of the Noncentral F distribution. We need to calculate the
area in the tail of the distribution that exceeds the value of the F
distribution for the null hypothesis.  We'll plot the two probability
distributions for comparison.

\begin{Verbatim}[commandchars=\\\{\}]
\PYG{g+gp}{\PYGZgt{}\PYGZgt{}\PYGZgt{} }\PYG{n}{dfnum} \PYG{o}{=} \PYG{l+m+mi}{3} \PYG{c}{\PYGZsh{} between group deg of freedom}
\PYG{g+gp}{\PYGZgt{}\PYGZgt{}\PYGZgt{} }\PYG{n}{dfden} \PYG{o}{=} \PYG{l+m+mi}{20} \PYG{c}{\PYGZsh{} within groups degrees of freedom}
\PYG{g+gp}{\PYGZgt{}\PYGZgt{}\PYGZgt{} }\PYG{n}{nonc} \PYG{o}{=} \PYG{l+m+mf}{3.0}
\PYG{g+gp}{\PYGZgt{}\PYGZgt{}\PYGZgt{} }\PYG{n}{nc\PYGZus{}vals} \PYG{o}{=} \PYG{n}{np}\PYG{o}{.}\PYG{n}{random}\PYG{o}{.}\PYG{n}{noncentral\PYGZus{}f}\PYG{p}{(}\PYG{n}{dfnum}\PYG{p}{,} \PYG{n}{dfden}\PYG{p}{,} \PYG{n}{nonc}\PYG{p}{,} \PYG{l+m+mi}{1000000}\PYG{p}{)}
\PYG{g+gp}{\PYGZgt{}\PYGZgt{}\PYGZgt{} }\PYG{n}{NF} \PYG{o}{=} \PYG{n}{np}\PYG{o}{.}\PYG{n}{histogram}\PYG{p}{(}\PYG{n}{nc\PYGZus{}vals}\PYG{p}{,} \PYG{n}{bins}\PYG{o}{=}\PYG{l+m+mi}{50}\PYG{p}{,} \PYG{n}{normed}\PYG{o}{=}\PYG{n+nb+bp}{True}\PYG{p}{)}
\PYG{g+gp}{\PYGZgt{}\PYGZgt{}\PYGZgt{} }\PYG{n}{c\PYGZus{}vals} \PYG{o}{=} \PYG{n}{np}\PYG{o}{.}\PYG{n}{random}\PYG{o}{.}\PYG{n}{f}\PYG{p}{(}\PYG{n}{dfnum}\PYG{p}{,} \PYG{n}{dfden}\PYG{p}{,} \PYG{l+m+mi}{1000000}\PYG{p}{)}
\PYG{g+gp}{\PYGZgt{}\PYGZgt{}\PYGZgt{} }\PYG{n}{F} \PYG{o}{=} \PYG{n}{np}\PYG{o}{.}\PYG{n}{histogram}\PYG{p}{(}\PYG{n}{c\PYGZus{}vals}\PYG{p}{,} \PYG{n}{bins}\PYG{o}{=}\PYG{l+m+mi}{50}\PYG{p}{,} \PYG{n}{normed}\PYG{o}{=}\PYG{n+nb+bp}{True}\PYG{p}{)}
\PYG{g+gp}{\PYGZgt{}\PYGZgt{}\PYGZgt{} }\PYG{n}{plt}\PYG{o}{.}\PYG{n}{plot}\PYG{p}{(}\PYG{n}{F}\PYG{p}{[}\PYG{l+m+mi}{1}\PYG{p}{]}\PYG{p}{[}\PYG{l+m+mi}{1}\PYG{p}{:}\PYG{p}{]}\PYG{p}{,} \PYG{n}{F}\PYG{p}{[}\PYG{l+m+mi}{0}\PYG{p}{]}\PYG{p}{)}
\PYG{g+gp}{\PYGZgt{}\PYGZgt{}\PYGZgt{} }\PYG{n}{plt}\PYG{o}{.}\PYG{n}{plot}\PYG{p}{(}\PYG{n}{NF}\PYG{p}{[}\PYG{l+m+mi}{1}\PYG{p}{]}\PYG{p}{[}\PYG{l+m+mi}{1}\PYG{p}{:}\PYG{p}{]}\PYG{p}{,} \PYG{n}{NF}\PYG{p}{[}\PYG{l+m+mi}{0}\PYG{p}{]}\PYG{p}{)}
\PYG{g+gp}{\PYGZgt{}\PYGZgt{}\PYGZgt{} }\PYG{n}{plt}\PYG{o}{.}\PYG{n}{show}\PYG{p}{(}\PYG{p}{)}
\end{Verbatim}

\end{fulllineitems}

\index{normal() (in module acsFromWim2Carness)}

\begin{fulllineitems}
\phantomsection\label{acsFromWim2Carness:acsFromWim2Carness.normal}\pysiglinewithargsret{\code{acsFromWim2Carness.}\bfcode{normal}}{\emph{loc=0.0}, \emph{scale=1.0}, \emph{size=None}}{}
Draw random samples from a normal (Gaussian) distribution.

The probability density function of the normal distribution, first
derived by De Moivre and 200 years later by both Gauss and Laplace
independently {\color{red}\bfseries{}{[}2{]}\_}, is often called the bell curve because of
its characteristic shape (see the example below).

The normal distributions occurs often in nature.  For example, it
describes the commonly occurring distribution of samples influenced
by a large number of tiny, random disturbances, each with its own
unique distribution {\color{red}\bfseries{}{[}2{]}\_}.
\begin{description}
\item[{loc}] \leavevmode{[}float{]}
Mean (``centre'') of the distribution.

\item[{scale}] \leavevmode{[}float{]}
Standard deviation (spread or ``width'') of the distribution.

\item[{size}] \leavevmode{[}tuple of ints{]}
Output shape.  If the given shape is, e.g., \code{(m, n, k)}, then
\code{m * n * k} samples are drawn.

\end{description}
\begin{description}
\item[{scipy.stats.distributions.norm}] \leavevmode{[}probability density function,{]}
distribution or cumulative density function, etc.

\end{description}

The probability density for the Gaussian distribution is
\begin{gather}
\begin{split}p(x) = \frac{1}{\sqrt{ 2 \pi \sigma^2 }}
e^{ - \frac{ (x - \mu)^2 } {2 \sigma^2} },\end{split}\notag
\end{gather}
where \(\mu\) is the mean and \(\sigma\) the standard deviation.
The square of the standard deviation, \(\sigma^2\), is called the
variance.

The function has its peak at the mean, and its ``spread'' increases with
the standard deviation (the function reaches 0.607 times its maximum at
\(x + \sigma\) and \(x - \sigma\) {\color{red}\bfseries{}{[}2{]}\_}).  This implies that
\emph{numpy.random.normal} is more likely to return samples lying close to the
mean, rather than those far away.

Draw samples from the distribution:

\begin{Verbatim}[commandchars=\\\{\}]
\PYG{g+gp}{\PYGZgt{}\PYGZgt{}\PYGZgt{} }\PYG{n}{mu}\PYG{p}{,} \PYG{n}{sigma} \PYG{o}{=} \PYG{l+m+mi}{0}\PYG{p}{,} \PYG{l+m+mf}{0.1} \PYG{c}{\PYGZsh{} mean and standard deviation}
\PYG{g+gp}{\PYGZgt{}\PYGZgt{}\PYGZgt{} }\PYG{n}{s} \PYG{o}{=} \PYG{n}{np}\PYG{o}{.}\PYG{n}{random}\PYG{o}{.}\PYG{n}{normal}\PYG{p}{(}\PYG{n}{mu}\PYG{p}{,} \PYG{n}{sigma}\PYG{p}{,} \PYG{l+m+mi}{1000}\PYG{p}{)}
\end{Verbatim}

Verify the mean and the variance:

\begin{Verbatim}[commandchars=\\\{\}]
\PYG{g+gp}{\PYGZgt{}\PYGZgt{}\PYGZgt{} }\PYG{n+nb}{abs}\PYG{p}{(}\PYG{n}{mu} \PYG{o}{\PYGZhy{}} \PYG{n}{np}\PYG{o}{.}\PYG{n}{mean}\PYG{p}{(}\PYG{n}{s}\PYG{p}{)}\PYG{p}{)} \PYG{o}{\PYGZlt{}} \PYG{l+m+mf}{0.01}
\PYG{g+go}{True}
\end{Verbatim}

\begin{Verbatim}[commandchars=\\\{\}]
\PYG{g+gp}{\PYGZgt{}\PYGZgt{}\PYGZgt{} }\PYG{n+nb}{abs}\PYG{p}{(}\PYG{n}{sigma} \PYG{o}{\PYGZhy{}} \PYG{n}{np}\PYG{o}{.}\PYG{n}{std}\PYG{p}{(}\PYG{n}{s}\PYG{p}{,} \PYG{n}{ddof}\PYG{o}{=}\PYG{l+m+mi}{1}\PYG{p}{)}\PYG{p}{)} \PYG{o}{\PYGZlt{}} \PYG{l+m+mf}{0.01}
\PYG{g+go}{True}
\end{Verbatim}

Display the histogram of the samples, along with
the probability density function:

\begin{Verbatim}[commandchars=\\\{\}]
\PYG{g+gp}{\PYGZgt{}\PYGZgt{}\PYGZgt{} }\PYG{k+kn}{import} \PYG{n+nn}{matplotlib.pyplot} \PYG{k+kn}{as} \PYG{n+nn}{plt}
\PYG{g+gp}{\PYGZgt{}\PYGZgt{}\PYGZgt{} }\PYG{n}{count}\PYG{p}{,} \PYG{n}{bins}\PYG{p}{,} \PYG{n}{ignored} \PYG{o}{=} \PYG{n}{plt}\PYG{o}{.}\PYG{n}{hist}\PYG{p}{(}\PYG{n}{s}\PYG{p}{,} \PYG{l+m+mi}{30}\PYG{p}{,} \PYG{n}{normed}\PYG{o}{=}\PYG{n+nb+bp}{True}\PYG{p}{)}
\PYG{g+gp}{\PYGZgt{}\PYGZgt{}\PYGZgt{} }\PYG{n}{plt}\PYG{o}{.}\PYG{n}{plot}\PYG{p}{(}\PYG{n}{bins}\PYG{p}{,} \PYG{l+m+mi}{1}\PYG{o}{/}\PYG{p}{(}\PYG{n}{sigma} \PYG{o}{*} \PYG{n}{np}\PYG{o}{.}\PYG{n}{sqrt}\PYG{p}{(}\PYG{l+m+mi}{2} \PYG{o}{*} \PYG{n}{np}\PYG{o}{.}\PYG{n}{pi}\PYG{p}{)}\PYG{p}{)} \PYG{o}{*}
\PYG{g+gp}{... }               \PYG{n}{np}\PYG{o}{.}\PYG{n}{exp}\PYG{p}{(} \PYG{o}{\PYGZhy{}} \PYG{p}{(}\PYG{n}{bins} \PYG{o}{\PYGZhy{}} \PYG{n}{mu}\PYG{p}{)}\PYG{o}{*}\PYG{o}{*}\PYG{l+m+mi}{2} \PYG{o}{/} \PYG{p}{(}\PYG{l+m+mi}{2} \PYG{o}{*} \PYG{n}{sigma}\PYG{o}{*}\PYG{o}{*}\PYG{l+m+mi}{2}\PYG{p}{)} \PYG{p}{)}\PYG{p}{,}
\PYG{g+gp}{... }         \PYG{n}{linewidth}\PYG{o}{=}\PYG{l+m+mi}{2}\PYG{p}{,} \PYG{n}{color}\PYG{o}{=}\PYG{l+s}{\PYGZsq{}}\PYG{l+s}{r}\PYG{l+s}{\PYGZsq{}}\PYG{p}{)}
\PYG{g+gp}{\PYGZgt{}\PYGZgt{}\PYGZgt{} }\PYG{n}{plt}\PYG{o}{.}\PYG{n}{show}\PYG{p}{(}\PYG{p}{)}
\end{Verbatim}

\end{fulllineitems}

\index{pareto() (in module acsFromWim2Carness)}

\begin{fulllineitems}
\phantomsection\label{acsFromWim2Carness:acsFromWim2Carness.pareto}\pysiglinewithargsret{\code{acsFromWim2Carness.}\bfcode{pareto}}{\emph{a}, \emph{size=None}}{}
Draw samples from a Pareto II or Lomax distribution with specified shape.

The Lomax or Pareto II distribution is a shifted Pareto distribution. The
classical Pareto distribution can be obtained from the Lomax distribution
by adding the location parameter m, see below. The smallest value of the
Lomax distribution is zero while for the classical Pareto distribution it
is m, where the standard Pareto distribution has location m=1.
Lomax can also be considered as a simplified version of the Generalized
Pareto distribution (available in SciPy), with the scale set to one and
the location set to zero.

The Pareto distribution must be greater than zero, and is unbounded above.
It is also known as the ``80-20 rule''.  In this distribution, 80 percent of
the weights are in the lowest 20 percent of the range, while the other 20
percent fill the remaining 80 percent of the range.
\begin{description}
\item[{shape}] \leavevmode{[}float, \textgreater{} 0.{]}
Shape of the distribution.

\item[{size}] \leavevmode{[}tuple of ints{]}
Output shape.  If the given shape is, e.g., \code{(m, n, k)}, then
\code{m * n * k} samples are drawn.

\end{description}
\begin{description}
\item[{scipy.stats.distributions.lomax.pdf}] \leavevmode{[}probability density function,{]}
distribution or cumulative density function, etc.

\item[{scipy.stats.distributions.genpareto.pdf}] \leavevmode{[}probability density function,{]}
distribution or cumulative density function, etc.

\end{description}

The probability density for the Pareto distribution is
\begin{gather}
\begin{split}p(x) = \frac{am^a}{x^{a+1}}\end{split}\notag
\end{gather}
where \(a\) is the shape and \(m\) the location

The Pareto distribution, named after the Italian economist Vilfredo Pareto,
is a power law probability distribution useful in many real world problems.
Outside the field of economics it is generally referred to as the Bradford
distribution. Pareto developed the distribution to describe the
distribution of wealth in an economy.  It has also found use in insurance,
web page access statistics, oil field sizes, and many other problems,
including the download frequency for projects in Sourceforge {[}1{]}.  It is
one of the so-called ``fat-tailed'' distributions.

Draw samples from the distribution:

\begin{Verbatim}[commandchars=\\\{\}]
\PYG{g+gp}{\PYGZgt{}\PYGZgt{}\PYGZgt{} }\PYG{n}{a}\PYG{p}{,} \PYG{n}{m} \PYG{o}{=} \PYG{l+m+mf}{3.}\PYG{p}{,} \PYG{l+m+mf}{1.} \PYG{c}{\PYGZsh{} shape and mode}
\PYG{g+gp}{\PYGZgt{}\PYGZgt{}\PYGZgt{} }\PYG{n}{s} \PYG{o}{=} \PYG{n}{np}\PYG{o}{.}\PYG{n}{random}\PYG{o}{.}\PYG{n}{pareto}\PYG{p}{(}\PYG{n}{a}\PYG{p}{,} \PYG{l+m+mi}{1000}\PYG{p}{)} \PYG{o}{+} \PYG{n}{m}
\end{Verbatim}

Display the histogram of the samples, along with
the probability density function:

\begin{Verbatim}[commandchars=\\\{\}]
\PYG{g+gp}{\PYGZgt{}\PYGZgt{}\PYGZgt{} }\PYG{k+kn}{import} \PYG{n+nn}{matplotlib.pyplot} \PYG{k+kn}{as} \PYG{n+nn}{plt}
\PYG{g+gp}{\PYGZgt{}\PYGZgt{}\PYGZgt{} }\PYG{n}{count}\PYG{p}{,} \PYG{n}{bins}\PYG{p}{,} \PYG{n}{ignored} \PYG{o}{=} \PYG{n}{plt}\PYG{o}{.}\PYG{n}{hist}\PYG{p}{(}\PYG{n}{s}\PYG{p}{,} \PYG{l+m+mi}{100}\PYG{p}{,} \PYG{n}{normed}\PYG{o}{=}\PYG{n+nb+bp}{True}\PYG{p}{,} \PYG{n}{align}\PYG{o}{=}\PYG{l+s}{\PYGZsq{}}\PYG{l+s}{center}\PYG{l+s}{\PYGZsq{}}\PYG{p}{)}
\PYG{g+gp}{\PYGZgt{}\PYGZgt{}\PYGZgt{} }\PYG{n}{fit} \PYG{o}{=} \PYG{n}{a}\PYG{o}{*}\PYG{n}{m}\PYG{o}{*}\PYG{o}{*}\PYG{n}{a}\PYG{o}{/}\PYG{n}{bins}\PYG{o}{*}\PYG{o}{*}\PYG{p}{(}\PYG{n}{a}\PYG{o}{+}\PYG{l+m+mi}{1}\PYG{p}{)}
\PYG{g+gp}{\PYGZgt{}\PYGZgt{}\PYGZgt{} }\PYG{n}{plt}\PYG{o}{.}\PYG{n}{plot}\PYG{p}{(}\PYG{n}{bins}\PYG{p}{,} \PYG{n+nb}{max}\PYG{p}{(}\PYG{n}{count}\PYG{p}{)}\PYG{o}{*}\PYG{n}{fit}\PYG{o}{/}\PYG{n+nb}{max}\PYG{p}{(}\PYG{n}{fit}\PYG{p}{)}\PYG{p}{,}\PYG{n}{linewidth}\PYG{o}{=}\PYG{l+m+mi}{2}\PYG{p}{,} \PYG{n}{color}\PYG{o}{=}\PYG{l+s}{\PYGZsq{}}\PYG{l+s}{r}\PYG{l+s}{\PYGZsq{}}\PYG{p}{)}
\PYG{g+gp}{\PYGZgt{}\PYGZgt{}\PYGZgt{} }\PYG{n}{plt}\PYG{o}{.}\PYG{n}{show}\PYG{p}{(}\PYG{p}{)}
\end{Verbatim}

\end{fulllineitems}

\index{permutation() (in module acsFromWim2Carness)}

\begin{fulllineitems}
\phantomsection\label{acsFromWim2Carness:acsFromWim2Carness.permutation}\pysiglinewithargsret{\code{acsFromWim2Carness.}\bfcode{permutation}}{\emph{x}}{}
Randomly permute a sequence, or return a permuted range.

If \emph{x} is a multi-dimensional array, it is only shuffled along its
first index.
\begin{description}
\item[{x}] \leavevmode{[}int or array\_like{]}
If \emph{x} is an integer, randomly permute \code{np.arange(x)}.
If \emph{x} is an array, make a copy and shuffle the elements
randomly.

\end{description}
\begin{description}
\item[{out}] \leavevmode{[}ndarray{]}
Permuted sequence or array range.

\end{description}

\begin{Verbatim}[commandchars=\\\{\}]
\PYG{g+gp}{\PYGZgt{}\PYGZgt{}\PYGZgt{} }\PYG{n}{np}\PYG{o}{.}\PYG{n}{random}\PYG{o}{.}\PYG{n}{permutation}\PYG{p}{(}\PYG{l+m+mi}{10}\PYG{p}{)}
\PYG{g+go}{array([1, 7, 4, 3, 0, 9, 2, 5, 8, 6])}
\end{Verbatim}

\begin{Verbatim}[commandchars=\\\{\}]
\PYG{g+gp}{\PYGZgt{}\PYGZgt{}\PYGZgt{} }\PYG{n}{np}\PYG{o}{.}\PYG{n}{random}\PYG{o}{.}\PYG{n}{permutation}\PYG{p}{(}\PYG{p}{[}\PYG{l+m+mi}{1}\PYG{p}{,} \PYG{l+m+mi}{4}\PYG{p}{,} \PYG{l+m+mi}{9}\PYG{p}{,} \PYG{l+m+mi}{12}\PYG{p}{,} \PYG{l+m+mi}{15}\PYG{p}{]}\PYG{p}{)}
\PYG{g+go}{array([15,  1,  9,  4, 12])}
\end{Verbatim}

\begin{Verbatim}[commandchars=\\\{\}]
\PYG{g+gp}{\PYGZgt{}\PYGZgt{}\PYGZgt{} }\PYG{n}{arr} \PYG{o}{=} \PYG{n}{np}\PYG{o}{.}\PYG{n}{arange}\PYG{p}{(}\PYG{l+m+mi}{9}\PYG{p}{)}\PYG{o}{.}\PYG{n}{reshape}\PYG{p}{(}\PYG{p}{(}\PYG{l+m+mi}{3}\PYG{p}{,} \PYG{l+m+mi}{3}\PYG{p}{)}\PYG{p}{)}
\PYG{g+gp}{\PYGZgt{}\PYGZgt{}\PYGZgt{} }\PYG{n}{np}\PYG{o}{.}\PYG{n}{random}\PYG{o}{.}\PYG{n}{permutation}\PYG{p}{(}\PYG{n}{arr}\PYG{p}{)}
\PYG{g+go}{array([[6, 7, 8],}
\PYG{g+go}{       [0, 1, 2],}
\PYG{g+go}{       [3, 4, 5]])}
\end{Verbatim}

\end{fulllineitems}

\index{poisson() (in module acsFromWim2Carness)}

\begin{fulllineitems}
\phantomsection\label{acsFromWim2Carness:acsFromWim2Carness.poisson}\pysiglinewithargsret{\code{acsFromWim2Carness.}\bfcode{poisson}}{\emph{lam=1.0}, \emph{size=None}}{}
Draw samples from a Poisson distribution.

The Poisson distribution is the limit of the Binomial
distribution for large N.
\begin{description}
\item[{lam}] \leavevmode{[}float{]}
Expectation of interval, should be \textgreater{}= 0.

\item[{size}] \leavevmode{[}int or tuple of ints, optional{]}
Output shape. If the given shape is, e.g., \code{(m, n, k)}, then
\code{m * n * k} samples are drawn.

\end{description}

The Poisson distribution
\begin{gather}
\begin{split}f(k; \lambda)=\frac{\lambda^k e^{-\lambda}}{k!}\end{split}\notag
\end{gather}
For events with an expected separation \(\lambda\) the Poisson
distribution \(f(k; \lambda)\) describes the probability of
\(k\) events occurring within the observed interval \(\lambda\).

Because the output is limited to the range of the C long type, a
ValueError is raised when \emph{lam} is within 10 sigma of the maximum
representable value.

Draw samples from the distribution:

\begin{Verbatim}[commandchars=\\\{\}]
\PYG{g+gp}{\PYGZgt{}\PYGZgt{}\PYGZgt{} }\PYG{k+kn}{import} \PYG{n+nn}{numpy} \PYG{k+kn}{as} \PYG{n+nn}{np}
\PYG{g+gp}{\PYGZgt{}\PYGZgt{}\PYGZgt{} }\PYG{n}{s} \PYG{o}{=} \PYG{n}{np}\PYG{o}{.}\PYG{n}{random}\PYG{o}{.}\PYG{n}{poisson}\PYG{p}{(}\PYG{l+m+mi}{5}\PYG{p}{,} \PYG{l+m+mi}{10000}\PYG{p}{)}
\end{Verbatim}

Display histogram of the sample:

\begin{Verbatim}[commandchars=\\\{\}]
\PYG{g+gp}{\PYGZgt{}\PYGZgt{}\PYGZgt{} }\PYG{k+kn}{import} \PYG{n+nn}{matplotlib.pyplot} \PYG{k+kn}{as} \PYG{n+nn}{plt}
\PYG{g+gp}{\PYGZgt{}\PYGZgt{}\PYGZgt{} }\PYG{n}{count}\PYG{p}{,} \PYG{n}{bins}\PYG{p}{,} \PYG{n}{ignored} \PYG{o}{=} \PYG{n}{plt}\PYG{o}{.}\PYG{n}{hist}\PYG{p}{(}\PYG{n}{s}\PYG{p}{,} \PYG{l+m+mi}{14}\PYG{p}{,} \PYG{n}{normed}\PYG{o}{=}\PYG{n+nb+bp}{True}\PYG{p}{)}
\PYG{g+gp}{\PYGZgt{}\PYGZgt{}\PYGZgt{} }\PYG{n}{plt}\PYG{o}{.}\PYG{n}{show}\PYG{p}{(}\PYG{p}{)}
\end{Verbatim}

\end{fulllineitems}

\index{power() (in module acsFromWim2Carness)}

\begin{fulllineitems}
\phantomsection\label{acsFromWim2Carness:acsFromWim2Carness.power}\pysiglinewithargsret{\code{acsFromWim2Carness.}\bfcode{power}}{\emph{a}, \emph{size=None}}{}
Draws samples in {[}0, 1{]} from a power distribution with positive
exponent a - 1.

Also known as the power function distribution.
\begin{description}
\item[{a}] \leavevmode{[}float{]}
parameter, \textgreater{} 0

\item[{size}] \leavevmode{[}tuple of ints{]}\begin{description}
\item[{Output shape.  If the given shape is, e.g., \code{(m, n, k)}, then}] \leavevmode
\code{m * n * k} samples are drawn.

\end{description}

\end{description}
\begin{description}
\item[{samples}] \leavevmode{[}\{ndarray, scalar\}{]}
The returned samples lie in {[}0, 1{]}.

\end{description}
\begin{description}
\item[{ValueError}] \leavevmode
If a\textless{}1.

\end{description}

The probability density function is
\begin{gather}
\begin{split}P(x; a) = ax^{a-1}, 0 \le x \le 1, a>0.\end{split}\notag
\end{gather}
The power function distribution is just the inverse of the Pareto
distribution. It may also be seen as a special case of the Beta
distribution.

It is used, for example, in modeling the over-reporting of insurance
claims.

Draw samples from the distribution:

\begin{Verbatim}[commandchars=\\\{\}]
\PYG{g+gp}{\PYGZgt{}\PYGZgt{}\PYGZgt{} }\PYG{n}{a} \PYG{o}{=} \PYG{l+m+mf}{5.} \PYG{c}{\PYGZsh{} shape}
\PYG{g+gp}{\PYGZgt{}\PYGZgt{}\PYGZgt{} }\PYG{n}{samples} \PYG{o}{=} \PYG{l+m+mi}{1000}
\PYG{g+gp}{\PYGZgt{}\PYGZgt{}\PYGZgt{} }\PYG{n}{s} \PYG{o}{=} \PYG{n}{np}\PYG{o}{.}\PYG{n}{random}\PYG{o}{.}\PYG{n}{power}\PYG{p}{(}\PYG{n}{a}\PYG{p}{,} \PYG{n}{samples}\PYG{p}{)}
\end{Verbatim}

Display the histogram of the samples, along with
the probability density function:

\begin{Verbatim}[commandchars=\\\{\}]
\PYG{g+gp}{\PYGZgt{}\PYGZgt{}\PYGZgt{} }\PYG{k+kn}{import} \PYG{n+nn}{matplotlib.pyplot} \PYG{k+kn}{as} \PYG{n+nn}{plt}
\PYG{g+gp}{\PYGZgt{}\PYGZgt{}\PYGZgt{} }\PYG{n}{count}\PYG{p}{,} \PYG{n}{bins}\PYG{p}{,} \PYG{n}{ignored} \PYG{o}{=} \PYG{n}{plt}\PYG{o}{.}\PYG{n}{hist}\PYG{p}{(}\PYG{n}{s}\PYG{p}{,} \PYG{n}{bins}\PYG{o}{=}\PYG{l+m+mi}{30}\PYG{p}{)}
\PYG{g+gp}{\PYGZgt{}\PYGZgt{}\PYGZgt{} }\PYG{n}{x} \PYG{o}{=} \PYG{n}{np}\PYG{o}{.}\PYG{n}{linspace}\PYG{p}{(}\PYG{l+m+mi}{0}\PYG{p}{,} \PYG{l+m+mi}{1}\PYG{p}{,} \PYG{l+m+mi}{100}\PYG{p}{)}
\PYG{g+gp}{\PYGZgt{}\PYGZgt{}\PYGZgt{} }\PYG{n}{y} \PYG{o}{=} \PYG{n}{a}\PYG{o}{*}\PYG{n}{x}\PYG{o}{*}\PYG{o}{*}\PYG{p}{(}\PYG{n}{a}\PYG{o}{\PYGZhy{}}\PYG{l+m+mf}{1.}\PYG{p}{)}
\PYG{g+gp}{\PYGZgt{}\PYGZgt{}\PYGZgt{} }\PYG{n}{normed\PYGZus{}y} \PYG{o}{=} \PYG{n}{samples}\PYG{o}{*}\PYG{n}{np}\PYG{o}{.}\PYG{n}{diff}\PYG{p}{(}\PYG{n}{bins}\PYG{p}{)}\PYG{p}{[}\PYG{l+m+mi}{0}\PYG{p}{]}\PYG{o}{*}\PYG{n}{y}
\PYG{g+gp}{\PYGZgt{}\PYGZgt{}\PYGZgt{} }\PYG{n}{plt}\PYG{o}{.}\PYG{n}{plot}\PYG{p}{(}\PYG{n}{x}\PYG{p}{,} \PYG{n}{normed\PYGZus{}y}\PYG{p}{)}
\PYG{g+gp}{\PYGZgt{}\PYGZgt{}\PYGZgt{} }\PYG{n}{plt}\PYG{o}{.}\PYG{n}{show}\PYG{p}{(}\PYG{p}{)}
\end{Verbatim}

Compare the power function distribution to the inverse of the Pareto.

\begin{Verbatim}[commandchars=\\\{\}]
\PYG{g+gp}{\PYGZgt{}\PYGZgt{}\PYGZgt{} }\PYG{k+kn}{from} \PYG{n+nn}{scipy} \PYG{k+kn}{import} \PYG{n}{stats}
\PYG{g+gp}{\PYGZgt{}\PYGZgt{}\PYGZgt{} }\PYG{n}{rvs} \PYG{o}{=} \PYG{n}{np}\PYG{o}{.}\PYG{n}{random}\PYG{o}{.}\PYG{n}{power}\PYG{p}{(}\PYG{l+m+mi}{5}\PYG{p}{,} \PYG{l+m+mi}{1000000}\PYG{p}{)}
\PYG{g+gp}{\PYGZgt{}\PYGZgt{}\PYGZgt{} }\PYG{n}{rvsp} \PYG{o}{=} \PYG{n}{np}\PYG{o}{.}\PYG{n}{random}\PYG{o}{.}\PYG{n}{pareto}\PYG{p}{(}\PYG{l+m+mi}{5}\PYG{p}{,} \PYG{l+m+mi}{1000000}\PYG{p}{)}
\PYG{g+gp}{\PYGZgt{}\PYGZgt{}\PYGZgt{} }\PYG{n}{xx} \PYG{o}{=} \PYG{n}{np}\PYG{o}{.}\PYG{n}{linspace}\PYG{p}{(}\PYG{l+m+mi}{0}\PYG{p}{,}\PYG{l+m+mi}{1}\PYG{p}{,}\PYG{l+m+mi}{100}\PYG{p}{)}
\PYG{g+gp}{\PYGZgt{}\PYGZgt{}\PYGZgt{} }\PYG{n}{powpdf} \PYG{o}{=} \PYG{n}{stats}\PYG{o}{.}\PYG{n}{powerlaw}\PYG{o}{.}\PYG{n}{pdf}\PYG{p}{(}\PYG{n}{xx}\PYG{p}{,}\PYG{l+m+mi}{5}\PYG{p}{)}
\end{Verbatim}

\begin{Verbatim}[commandchars=\\\{\}]
\PYG{g+gp}{\PYGZgt{}\PYGZgt{}\PYGZgt{} }\PYG{n}{plt}\PYG{o}{.}\PYG{n}{figure}\PYG{p}{(}\PYG{p}{)}
\PYG{g+gp}{\PYGZgt{}\PYGZgt{}\PYGZgt{} }\PYG{n}{plt}\PYG{o}{.}\PYG{n}{hist}\PYG{p}{(}\PYG{n}{rvs}\PYG{p}{,} \PYG{n}{bins}\PYG{o}{=}\PYG{l+m+mi}{50}\PYG{p}{,} \PYG{n}{normed}\PYG{o}{=}\PYG{n+nb+bp}{True}\PYG{p}{)}
\PYG{g+gp}{\PYGZgt{}\PYGZgt{}\PYGZgt{} }\PYG{n}{plt}\PYG{o}{.}\PYG{n}{plot}\PYG{p}{(}\PYG{n}{xx}\PYG{p}{,}\PYG{n}{powpdf}\PYG{p}{,}\PYG{l+s}{\PYGZsq{}}\PYG{l+s}{r\PYGZhy{}}\PYG{l+s}{\PYGZsq{}}\PYG{p}{)}
\PYG{g+gp}{\PYGZgt{}\PYGZgt{}\PYGZgt{} }\PYG{n}{plt}\PYG{o}{.}\PYG{n}{title}\PYG{p}{(}\PYG{l+s}{\PYGZsq{}}\PYG{l+s}{np.random.power(5)}\PYG{l+s}{\PYGZsq{}}\PYG{p}{)}
\end{Verbatim}

\begin{Verbatim}[commandchars=\\\{\}]
\PYG{g+gp}{\PYGZgt{}\PYGZgt{}\PYGZgt{} }\PYG{n}{plt}\PYG{o}{.}\PYG{n}{figure}\PYG{p}{(}\PYG{p}{)}
\PYG{g+gp}{\PYGZgt{}\PYGZgt{}\PYGZgt{} }\PYG{n}{plt}\PYG{o}{.}\PYG{n}{hist}\PYG{p}{(}\PYG{l+m+mf}{1.}\PYG{o}{/}\PYG{p}{(}\PYG{l+m+mf}{1.}\PYG{o}{+}\PYG{n}{rvsp}\PYG{p}{)}\PYG{p}{,} \PYG{n}{bins}\PYG{o}{=}\PYG{l+m+mi}{50}\PYG{p}{,} \PYG{n}{normed}\PYG{o}{=}\PYG{n+nb+bp}{True}\PYG{p}{)}
\PYG{g+gp}{\PYGZgt{}\PYGZgt{}\PYGZgt{} }\PYG{n}{plt}\PYG{o}{.}\PYG{n}{plot}\PYG{p}{(}\PYG{n}{xx}\PYG{p}{,}\PYG{n}{powpdf}\PYG{p}{,}\PYG{l+s}{\PYGZsq{}}\PYG{l+s}{r\PYGZhy{}}\PYG{l+s}{\PYGZsq{}}\PYG{p}{)}
\PYG{g+gp}{\PYGZgt{}\PYGZgt{}\PYGZgt{} }\PYG{n}{plt}\PYG{o}{.}\PYG{n}{title}\PYG{p}{(}\PYG{l+s}{\PYGZsq{}}\PYG{l+s}{inverse of 1 + np.random.pareto(5)}\PYG{l+s}{\PYGZsq{}}\PYG{p}{)}
\end{Verbatim}

\begin{Verbatim}[commandchars=\\\{\}]
\PYG{g+gp}{\PYGZgt{}\PYGZgt{}\PYGZgt{} }\PYG{n}{plt}\PYG{o}{.}\PYG{n}{figure}\PYG{p}{(}\PYG{p}{)}
\PYG{g+gp}{\PYGZgt{}\PYGZgt{}\PYGZgt{} }\PYG{n}{plt}\PYG{o}{.}\PYG{n}{hist}\PYG{p}{(}\PYG{l+m+mf}{1.}\PYG{o}{/}\PYG{p}{(}\PYG{l+m+mf}{1.}\PYG{o}{+}\PYG{n}{rvsp}\PYG{p}{)}\PYG{p}{,} \PYG{n}{bins}\PYG{o}{=}\PYG{l+m+mi}{50}\PYG{p}{,} \PYG{n}{normed}\PYG{o}{=}\PYG{n+nb+bp}{True}\PYG{p}{)}
\PYG{g+gp}{\PYGZgt{}\PYGZgt{}\PYGZgt{} }\PYG{n}{plt}\PYG{o}{.}\PYG{n}{plot}\PYG{p}{(}\PYG{n}{xx}\PYG{p}{,}\PYG{n}{powpdf}\PYG{p}{,}\PYG{l+s}{\PYGZsq{}}\PYG{l+s}{r\PYGZhy{}}\PYG{l+s}{\PYGZsq{}}\PYG{p}{)}
\PYG{g+gp}{\PYGZgt{}\PYGZgt{}\PYGZgt{} }\PYG{n}{plt}\PYG{o}{.}\PYG{n}{title}\PYG{p}{(}\PYG{l+s}{\PYGZsq{}}\PYG{l+s}{inverse of stats.pareto(5)}\PYG{l+s}{\PYGZsq{}}\PYG{p}{)}
\end{Verbatim}

\end{fulllineitems}

\index{rand() (in module acsFromWim2Carness)}

\begin{fulllineitems}
\phantomsection\label{acsFromWim2Carness:acsFromWim2Carness.rand}\pysiglinewithargsret{\code{acsFromWim2Carness.}\bfcode{rand}}{\emph{d0}, \emph{d1}, \emph{...}, \emph{dn}}{}
Random values in a given shape.

Create an array of the given shape and propagate it with
random samples from a uniform distribution
over \code{{[}0, 1)}.
\begin{description}
\item[{d0, d1, ..., dn}] \leavevmode{[}int, optional{]}
The dimensions of the returned array, should all be positive.
If no argument is given a single Python float is returned.

\end{description}
\begin{description}
\item[{out}] \leavevmode{[}ndarray, shape \code{(d0, d1, ..., dn)}{]}
Random values.

\end{description}

random

This is a convenience function. If you want an interface that
takes a shape-tuple as the first argument, refer to
np.random.random\_sample .

\begin{Verbatim}[commandchars=\\\{\}]
\PYG{g+gp}{\PYGZgt{}\PYGZgt{}\PYGZgt{} }\PYG{n}{np}\PYG{o}{.}\PYG{n}{random}\PYG{o}{.}\PYG{n}{rand}\PYG{p}{(}\PYG{l+m+mi}{3}\PYG{p}{,}\PYG{l+m+mi}{2}\PYG{p}{)}
\PYG{g+go}{array([[ 0.14022471,  0.96360618],  \PYGZsh{}random}
\PYG{g+go}{       [ 0.37601032,  0.25528411],  \PYGZsh{}random}
\PYG{g+go}{       [ 0.49313049,  0.94909878]]) \PYGZsh{}random}
\end{Verbatim}

\end{fulllineitems}

\index{randint() (in module acsFromWim2Carness)}

\begin{fulllineitems}
\phantomsection\label{acsFromWim2Carness:acsFromWim2Carness.randint}\pysiglinewithargsret{\code{acsFromWim2Carness.}\bfcode{randint}}{\emph{low}, \emph{high=None}, \emph{size=None}}{}
Return random integers from \emph{low} (inclusive) to \emph{high} (exclusive).

Return random integers from the ``discrete uniform'' distribution in the
``half-open'' interval {[}\emph{low}, \emph{high}). If \emph{high} is None (the default),
then results are from {[}0, \emph{low}).
\begin{description}
\item[{low}] \leavevmode{[}int{]}
Lowest (signed) integer to be drawn from the distribution (unless
\code{high=None}, in which case this parameter is the \emph{highest} such
integer).

\item[{high}] \leavevmode{[}int, optional{]}
If provided, one above the largest (signed) integer to be drawn
from the distribution (see above for behavior if \code{high=None}).

\item[{size}] \leavevmode{[}int or tuple of ints, optional{]}
Output shape. Default is None, in which case a single int is
returned.

\end{description}
\begin{description}
\item[{out}] \leavevmode{[}int or ndarray of ints{]}
\emph{size}-shaped array of random integers from the appropriate
distribution, or a single such random int if \emph{size} not provided.

\end{description}
\begin{description}
\item[{random.random\_integers}] \leavevmode{[}similar to \emph{randint}, only for the closed{]}
interval {[}\emph{low}, \emph{high}{]}, and 1 is the lowest value if \emph{high} is
omitted. In particular, this other one is the one to use to generate
uniformly distributed discrete non-integers.

\end{description}

\begin{Verbatim}[commandchars=\\\{\}]
\PYG{g+gp}{\PYGZgt{}\PYGZgt{}\PYGZgt{} }\PYG{n}{np}\PYG{o}{.}\PYG{n}{random}\PYG{o}{.}\PYG{n}{randint}\PYG{p}{(}\PYG{l+m+mi}{2}\PYG{p}{,} \PYG{n}{size}\PYG{o}{=}\PYG{l+m+mi}{10}\PYG{p}{)}
\PYG{g+go}{array([1, 0, 0, 0, 1, 1, 0, 0, 1, 0])}
\PYG{g+gp}{\PYGZgt{}\PYGZgt{}\PYGZgt{} }\PYG{n}{np}\PYG{o}{.}\PYG{n}{random}\PYG{o}{.}\PYG{n}{randint}\PYG{p}{(}\PYG{l+m+mi}{1}\PYG{p}{,} \PYG{n}{size}\PYG{o}{=}\PYG{l+m+mi}{10}\PYG{p}{)}
\PYG{g+go}{array([0, 0, 0, 0, 0, 0, 0, 0, 0, 0])}
\end{Verbatim}

Generate a 2 x 4 array of ints between 0 and 4, inclusive:

\begin{Verbatim}[commandchars=\\\{\}]
\PYG{g+gp}{\PYGZgt{}\PYGZgt{}\PYGZgt{} }\PYG{n}{np}\PYG{o}{.}\PYG{n}{random}\PYG{o}{.}\PYG{n}{randint}\PYG{p}{(}\PYG{l+m+mi}{5}\PYG{p}{,} \PYG{n}{size}\PYG{o}{=}\PYG{p}{(}\PYG{l+m+mi}{2}\PYG{p}{,} \PYG{l+m+mi}{4}\PYG{p}{)}\PYG{p}{)}
\PYG{g+go}{array([[4, 0, 2, 1],}
\PYG{g+go}{       [3, 2, 2, 0]])}
\end{Verbatim}

\end{fulllineitems}

\index{randn() (in module acsFromWim2Carness)}

\begin{fulllineitems}
\phantomsection\label{acsFromWim2Carness:acsFromWim2Carness.randn}\pysiglinewithargsret{\code{acsFromWim2Carness.}\bfcode{randn}}{\emph{d0}, \emph{d1}, \emph{...}, \emph{dn}}{}
Return a sample (or samples) from the ``standard normal'' distribution.

If positive, int\_like or int-convertible arguments are provided,
\emph{randn} generates an array of shape \code{(d0, d1, ..., dn)}, filled
with random floats sampled from a univariate ``normal'' (Gaussian)
distribution of mean 0 and variance 1 (if any of the \(d_i\) are
floats, they are first converted to integers by truncation). A single
float randomly sampled from the distribution is returned if no
argument is provided.

This is a convenience function.  If you want an interface that takes a
tuple as the first argument, use \emph{numpy.random.standard\_normal} instead.
\begin{description}
\item[{d0, d1, ..., dn}] \leavevmode{[}int, optional{]}
The dimensions of the returned array, should be all positive.
If no argument is given a single Python float is returned.

\end{description}
\begin{description}
\item[{Z}] \leavevmode{[}ndarray or float{]}
A \code{(d0, d1, ..., dn)}-shaped array of floating-point samples from
the standard normal distribution, or a single such float if
no parameters were supplied.

\end{description}

random.standard\_normal : Similar, but takes a tuple as its argument.

For random samples from \(N(\mu, \sigma^2)\), use:

\code{sigma * np.random.randn(...) + mu}

\begin{Verbatim}[commandchars=\\\{\}]
\PYG{g+gp}{\PYGZgt{}\PYGZgt{}\PYGZgt{} }\PYG{n}{np}\PYG{o}{.}\PYG{n}{random}\PYG{o}{.}\PYG{n}{randn}\PYG{p}{(}\PYG{p}{)}
\PYG{g+go}{2.1923875335537315 \PYGZsh{}random}
\end{Verbatim}

Two-by-four array of samples from N(3, 6.25):

\begin{Verbatim}[commandchars=\\\{\}]
\PYG{g+gp}{\PYGZgt{}\PYGZgt{}\PYGZgt{} }\PYG{l+m+mf}{2.5} \PYG{o}{*} \PYG{n}{np}\PYG{o}{.}\PYG{n}{random}\PYG{o}{.}\PYG{n}{randn}\PYG{p}{(}\PYG{l+m+mi}{2}\PYG{p}{,} \PYG{l+m+mi}{4}\PYG{p}{)} \PYG{o}{+} \PYG{l+m+mi}{3}
\PYG{g+go}{array([[\PYGZhy{}4.49401501,  4.00950034, \PYGZhy{}1.81814867,  7.29718677],  \PYGZsh{}random}
\PYG{g+go}{       [ 0.39924804,  4.68456316,  4.99394529,  4.84057254]]) \PYGZsh{}random}
\end{Verbatim}

\end{fulllineitems}

\index{random() (in module acsFromWim2Carness)}

\begin{fulllineitems}
\phantomsection\label{acsFromWim2Carness:acsFromWim2Carness.random}\pysiglinewithargsret{\code{acsFromWim2Carness.}\bfcode{random}}{}{}
random\_sample(size=None)

Return random floats in the half-open interval {[}0.0, 1.0).

Results are from the ``continuous uniform'' distribution over the
stated interval.  To sample \(Unif[a, b), b > a\) multiply
the output of \emph{random\_sample} by \emph{(b-a)} and add \emph{a}:

\begin{Verbatim}[commandchars=\\\{\}]
\PYG{p}{(}\PYG{n}{b} \PYG{o}{\PYGZhy{}} \PYG{n}{a}\PYG{p}{)} \PYG{o}{*} \PYG{n}{random\PYGZus{}sample}\PYG{p}{(}\PYG{p}{)} \PYG{o}{+} \PYG{n}{a}
\end{Verbatim}
\begin{description}
\item[{size}] \leavevmode{[}int or tuple of ints, optional{]}
Defines the shape of the returned array of random floats. If None
(the default), returns a single float.

\end{description}
\begin{description}
\item[{out}] \leavevmode{[}float or ndarray of floats{]}
Array of random floats of shape \emph{size} (unless \code{size=None}, in which
case a single float is returned).

\end{description}

\begin{Verbatim}[commandchars=\\\{\}]
\PYG{g+gp}{\PYGZgt{}\PYGZgt{}\PYGZgt{} }\PYG{n}{np}\PYG{o}{.}\PYG{n}{random}\PYG{o}{.}\PYG{n}{random\PYGZus{}sample}\PYG{p}{(}\PYG{p}{)}
\PYG{g+go}{0.47108547995356098}
\PYG{g+gp}{\PYGZgt{}\PYGZgt{}\PYGZgt{} }\PYG{n+nb}{type}\PYG{p}{(}\PYG{n}{np}\PYG{o}{.}\PYG{n}{random}\PYG{o}{.}\PYG{n}{random\PYGZus{}sample}\PYG{p}{(}\PYG{p}{)}\PYG{p}{)}
\PYG{g+go}{\PYGZlt{}type \PYGZsq{}float\PYGZsq{}\PYGZgt{}}
\PYG{g+gp}{\PYGZgt{}\PYGZgt{}\PYGZgt{} }\PYG{n}{np}\PYG{o}{.}\PYG{n}{random}\PYG{o}{.}\PYG{n}{random\PYGZus{}sample}\PYG{p}{(}\PYG{p}{(}\PYG{l+m+mi}{5}\PYG{p}{,}\PYG{p}{)}\PYG{p}{)}
\PYG{g+go}{array([ 0.30220482,  0.86820401,  0.1654503 ,  0.11659149,  0.54323428])}
\end{Verbatim}

Three-by-two array of random numbers from {[}-5, 0):

\begin{Verbatim}[commandchars=\\\{\}]
\PYG{g+gp}{\PYGZgt{}\PYGZgt{}\PYGZgt{} }\PYG{l+m+mi}{5} \PYG{o}{*} \PYG{n}{np}\PYG{o}{.}\PYG{n}{random}\PYG{o}{.}\PYG{n}{random\PYGZus{}sample}\PYG{p}{(}\PYG{p}{(}\PYG{l+m+mi}{3}\PYG{p}{,} \PYG{l+m+mi}{2}\PYG{p}{)}\PYG{p}{)} \PYG{o}{\PYGZhy{}} \PYG{l+m+mi}{5}
\PYG{g+go}{array([[\PYGZhy{}3.99149989, \PYGZhy{}0.52338984],}
\PYG{g+go}{       [\PYGZhy{}2.99091858, \PYGZhy{}0.79479508],}
\PYG{g+go}{       [\PYGZhy{}1.23204345, \PYGZhy{}1.75224494]])}
\end{Verbatim}

\end{fulllineitems}

\index{random\_integers() (in module acsFromWim2Carness)}

\begin{fulllineitems}
\phantomsection\label{acsFromWim2Carness:acsFromWim2Carness.random_integers}\pysiglinewithargsret{\code{acsFromWim2Carness.}\bfcode{random\_integers}}{\emph{low}, \emph{high=None}, \emph{size=None}}{}
Return random integers between \emph{low} and \emph{high}, inclusive.

Return random integers from the ``discrete uniform'' distribution in the
closed interval {[}\emph{low}, \emph{high}{]}.  If \emph{high} is None (the default),
then results are from {[}1, \emph{low}{]}.
\begin{description}
\item[{low}] \leavevmode{[}int{]}
Lowest (signed) integer to be drawn from the distribution (unless
\code{high=None}, in which case this parameter is the \emph{highest} such
integer).

\item[{high}] \leavevmode{[}int, optional{]}
If provided, the largest (signed) integer to be drawn from the
distribution (see above for behavior if \code{high=None}).

\item[{size}] \leavevmode{[}int or tuple of ints, optional{]}
Output shape. Default is None, in which case a single int is returned.

\end{description}
\begin{description}
\item[{out}] \leavevmode{[}int or ndarray of ints{]}
\emph{size}-shaped array of random integers from the appropriate
distribution, or a single such random int if \emph{size} not provided.

\end{description}
\begin{description}
\item[{random.randint}] \leavevmode{[}Similar to \emph{random\_integers}, only for the half-open{]}
interval {[}\emph{low}, \emph{high}), and 0 is the lowest value if \emph{high} is
omitted.

\end{description}

To sample from N evenly spaced floating-point numbers between a and b,
use:

\begin{Verbatim}[commandchars=\\\{\}]
\PYG{n}{a} \PYG{o}{+} \PYG{p}{(}\PYG{n}{b} \PYG{o}{\PYGZhy{}} \PYG{n}{a}\PYG{p}{)} \PYG{o}{*} \PYG{p}{(}\PYG{n}{np}\PYG{o}{.}\PYG{n}{random}\PYG{o}{.}\PYG{n}{random\PYGZus{}integers}\PYG{p}{(}\PYG{n}{N}\PYG{p}{)} \PYG{o}{\PYGZhy{}} \PYG{l+m+mi}{1}\PYG{p}{)} \PYG{o}{/} \PYG{p}{(}\PYG{n}{N} \PYG{o}{\PYGZhy{}} \PYG{l+m+mf}{1.}\PYG{p}{)}
\end{Verbatim}

\begin{Verbatim}[commandchars=\\\{\}]
\PYG{g+gp}{\PYGZgt{}\PYGZgt{}\PYGZgt{} }\PYG{n}{np}\PYG{o}{.}\PYG{n}{random}\PYG{o}{.}\PYG{n}{random\PYGZus{}integers}\PYG{p}{(}\PYG{l+m+mi}{5}\PYG{p}{)}
\PYG{g+go}{4}
\PYG{g+gp}{\PYGZgt{}\PYGZgt{}\PYGZgt{} }\PYG{n+nb}{type}\PYG{p}{(}\PYG{n}{np}\PYG{o}{.}\PYG{n}{random}\PYG{o}{.}\PYG{n}{random\PYGZus{}integers}\PYG{p}{(}\PYG{l+m+mi}{5}\PYG{p}{)}\PYG{p}{)}
\PYG{g+go}{\PYGZlt{}type \PYGZsq{}int\PYGZsq{}\PYGZgt{}}
\PYG{g+gp}{\PYGZgt{}\PYGZgt{}\PYGZgt{} }\PYG{n}{np}\PYG{o}{.}\PYG{n}{random}\PYG{o}{.}\PYG{n}{random\PYGZus{}integers}\PYG{p}{(}\PYG{l+m+mi}{5}\PYG{p}{,} \PYG{n}{size}\PYG{o}{=}\PYG{p}{(}\PYG{l+m+mf}{3.}\PYG{p}{,}\PYG{l+m+mf}{2.}\PYG{p}{)}\PYG{p}{)}
\PYG{g+go}{array([[5, 4],}
\PYG{g+go}{       [3, 3],}
\PYG{g+go}{       [4, 5]])}
\end{Verbatim}

Choose five random numbers from the set of five evenly-spaced
numbers between 0 and 2.5, inclusive (\emph{i.e.}, from the set
\({0, 5/8, 10/8, 15/8, 20/8}\)):

\begin{Verbatim}[commandchars=\\\{\}]
\PYG{g+gp}{\PYGZgt{}\PYGZgt{}\PYGZgt{} }\PYG{l+m+mf}{2.5} \PYG{o}{*} \PYG{p}{(}\PYG{n}{np}\PYG{o}{.}\PYG{n}{random}\PYG{o}{.}\PYG{n}{random\PYGZus{}integers}\PYG{p}{(}\PYG{l+m+mi}{5}\PYG{p}{,} \PYG{n}{size}\PYG{o}{=}\PYG{p}{(}\PYG{l+m+mi}{5}\PYG{p}{,}\PYG{p}{)}\PYG{p}{)} \PYG{o}{\PYGZhy{}} \PYG{l+m+mi}{1}\PYG{p}{)} \PYG{o}{/} \PYG{l+m+mf}{4.}
\PYG{g+go}{array([ 0.625,  1.25 ,  0.625,  0.625,  2.5  ])}
\end{Verbatim}

Roll two six sided dice 1000 times and sum the results:

\begin{Verbatim}[commandchars=\\\{\}]
\PYG{g+gp}{\PYGZgt{}\PYGZgt{}\PYGZgt{} }\PYG{n}{d1} \PYG{o}{=} \PYG{n}{np}\PYG{o}{.}\PYG{n}{random}\PYG{o}{.}\PYG{n}{random\PYGZus{}integers}\PYG{p}{(}\PYG{l+m+mi}{1}\PYG{p}{,} \PYG{l+m+mi}{6}\PYG{p}{,} \PYG{l+m+mi}{1000}\PYG{p}{)}
\PYG{g+gp}{\PYGZgt{}\PYGZgt{}\PYGZgt{} }\PYG{n}{d2} \PYG{o}{=} \PYG{n}{np}\PYG{o}{.}\PYG{n}{random}\PYG{o}{.}\PYG{n}{random\PYGZus{}integers}\PYG{p}{(}\PYG{l+m+mi}{1}\PYG{p}{,} \PYG{l+m+mi}{6}\PYG{p}{,} \PYG{l+m+mi}{1000}\PYG{p}{)}
\PYG{g+gp}{\PYGZgt{}\PYGZgt{}\PYGZgt{} }\PYG{n}{dsums} \PYG{o}{=} \PYG{n}{d1} \PYG{o}{+} \PYG{n}{d2}
\end{Verbatim}

Display results as a histogram:

\begin{Verbatim}[commandchars=\\\{\}]
\PYG{g+gp}{\PYGZgt{}\PYGZgt{}\PYGZgt{} }\PYG{k+kn}{import} \PYG{n+nn}{matplotlib.pyplot} \PYG{k+kn}{as} \PYG{n+nn}{plt}
\PYG{g+gp}{\PYGZgt{}\PYGZgt{}\PYGZgt{} }\PYG{n}{count}\PYG{p}{,} \PYG{n}{bins}\PYG{p}{,} \PYG{n}{ignored} \PYG{o}{=} \PYG{n}{plt}\PYG{o}{.}\PYG{n}{hist}\PYG{p}{(}\PYG{n}{dsums}\PYG{p}{,} \PYG{l+m+mi}{11}\PYG{p}{,} \PYG{n}{normed}\PYG{o}{=}\PYG{n+nb+bp}{True}\PYG{p}{)}
\PYG{g+gp}{\PYGZgt{}\PYGZgt{}\PYGZgt{} }\PYG{n}{plt}\PYG{o}{.}\PYG{n}{show}\PYG{p}{(}\PYG{p}{)}
\end{Verbatim}

\end{fulllineitems}

\index{random\_sample() (in module acsFromWim2Carness)}

\begin{fulllineitems}
\phantomsection\label{acsFromWim2Carness:acsFromWim2Carness.random_sample}\pysiglinewithargsret{\code{acsFromWim2Carness.}\bfcode{random\_sample}}{\emph{size=None}}{}
Return random floats in the half-open interval {[}0.0, 1.0).

Results are from the ``continuous uniform'' distribution over the
stated interval.  To sample \(Unif[a, b), b > a\) multiply
the output of \emph{random\_sample} by \emph{(b-a)} and add \emph{a}:

\begin{Verbatim}[commandchars=\\\{\}]
\PYG{p}{(}\PYG{n}{b} \PYG{o}{\PYGZhy{}} \PYG{n}{a}\PYG{p}{)} \PYG{o}{*} \PYG{n}{random\PYGZus{}sample}\PYG{p}{(}\PYG{p}{)} \PYG{o}{+} \PYG{n}{a}
\end{Verbatim}
\begin{description}
\item[{size}] \leavevmode{[}int or tuple of ints, optional{]}
Defines the shape of the returned array of random floats. If None
(the default), returns a single float.

\end{description}
\begin{description}
\item[{out}] \leavevmode{[}float or ndarray of floats{]}
Array of random floats of shape \emph{size} (unless \code{size=None}, in which
case a single float is returned).

\end{description}

\begin{Verbatim}[commandchars=\\\{\}]
\PYG{g+gp}{\PYGZgt{}\PYGZgt{}\PYGZgt{} }\PYG{n}{np}\PYG{o}{.}\PYG{n}{random}\PYG{o}{.}\PYG{n}{random\PYGZus{}sample}\PYG{p}{(}\PYG{p}{)}
\PYG{g+go}{0.47108547995356098}
\PYG{g+gp}{\PYGZgt{}\PYGZgt{}\PYGZgt{} }\PYG{n+nb}{type}\PYG{p}{(}\PYG{n}{np}\PYG{o}{.}\PYG{n}{random}\PYG{o}{.}\PYG{n}{random\PYGZus{}sample}\PYG{p}{(}\PYG{p}{)}\PYG{p}{)}
\PYG{g+go}{\PYGZlt{}type \PYGZsq{}float\PYGZsq{}\PYGZgt{}}
\PYG{g+gp}{\PYGZgt{}\PYGZgt{}\PYGZgt{} }\PYG{n}{np}\PYG{o}{.}\PYG{n}{random}\PYG{o}{.}\PYG{n}{random\PYGZus{}sample}\PYG{p}{(}\PYG{p}{(}\PYG{l+m+mi}{5}\PYG{p}{,}\PYG{p}{)}\PYG{p}{)}
\PYG{g+go}{array([ 0.30220482,  0.86820401,  0.1654503 ,  0.11659149,  0.54323428])}
\end{Verbatim}

Three-by-two array of random numbers from {[}-5, 0):

\begin{Verbatim}[commandchars=\\\{\}]
\PYG{g+gp}{\PYGZgt{}\PYGZgt{}\PYGZgt{} }\PYG{l+m+mi}{5} \PYG{o}{*} \PYG{n}{np}\PYG{o}{.}\PYG{n}{random}\PYG{o}{.}\PYG{n}{random\PYGZus{}sample}\PYG{p}{(}\PYG{p}{(}\PYG{l+m+mi}{3}\PYG{p}{,} \PYG{l+m+mi}{2}\PYG{p}{)}\PYG{p}{)} \PYG{o}{\PYGZhy{}} \PYG{l+m+mi}{5}
\PYG{g+go}{array([[\PYGZhy{}3.99149989, \PYGZhy{}0.52338984],}
\PYG{g+go}{       [\PYGZhy{}2.99091858, \PYGZhy{}0.79479508],}
\PYG{g+go}{       [\PYGZhy{}1.23204345, \PYGZhy{}1.75224494]])}
\end{Verbatim}

\end{fulllineitems}

\index{ranf() (in module acsFromWim2Carness)}

\begin{fulllineitems}
\phantomsection\label{acsFromWim2Carness:acsFromWim2Carness.ranf}\pysiglinewithargsret{\code{acsFromWim2Carness.}\bfcode{ranf}}{}{}
random\_sample(size=None)

Return random floats in the half-open interval {[}0.0, 1.0).

Results are from the ``continuous uniform'' distribution over the
stated interval.  To sample \(Unif[a, b), b > a\) multiply
the output of \emph{random\_sample} by \emph{(b-a)} and add \emph{a}:

\begin{Verbatim}[commandchars=\\\{\}]
\PYG{p}{(}\PYG{n}{b} \PYG{o}{\PYGZhy{}} \PYG{n}{a}\PYG{p}{)} \PYG{o}{*} \PYG{n}{random\PYGZus{}sample}\PYG{p}{(}\PYG{p}{)} \PYG{o}{+} \PYG{n}{a}
\end{Verbatim}
\begin{description}
\item[{size}] \leavevmode{[}int or tuple of ints, optional{]}
Defines the shape of the returned array of random floats. If None
(the default), returns a single float.

\end{description}
\begin{description}
\item[{out}] \leavevmode{[}float or ndarray of floats{]}
Array of random floats of shape \emph{size} (unless \code{size=None}, in which
case a single float is returned).

\end{description}

\begin{Verbatim}[commandchars=\\\{\}]
\PYG{g+gp}{\PYGZgt{}\PYGZgt{}\PYGZgt{} }\PYG{n}{np}\PYG{o}{.}\PYG{n}{random}\PYG{o}{.}\PYG{n}{random\PYGZus{}sample}\PYG{p}{(}\PYG{p}{)}
\PYG{g+go}{0.47108547995356098}
\PYG{g+gp}{\PYGZgt{}\PYGZgt{}\PYGZgt{} }\PYG{n+nb}{type}\PYG{p}{(}\PYG{n}{np}\PYG{o}{.}\PYG{n}{random}\PYG{o}{.}\PYG{n}{random\PYGZus{}sample}\PYG{p}{(}\PYG{p}{)}\PYG{p}{)}
\PYG{g+go}{\PYGZlt{}type \PYGZsq{}float\PYGZsq{}\PYGZgt{}}
\PYG{g+gp}{\PYGZgt{}\PYGZgt{}\PYGZgt{} }\PYG{n}{np}\PYG{o}{.}\PYG{n}{random}\PYG{o}{.}\PYG{n}{random\PYGZus{}sample}\PYG{p}{(}\PYG{p}{(}\PYG{l+m+mi}{5}\PYG{p}{,}\PYG{p}{)}\PYG{p}{)}
\PYG{g+go}{array([ 0.30220482,  0.86820401,  0.1654503 ,  0.11659149,  0.54323428])}
\end{Verbatim}

Three-by-two array of random numbers from {[}-5, 0):

\begin{Verbatim}[commandchars=\\\{\}]
\PYG{g+gp}{\PYGZgt{}\PYGZgt{}\PYGZgt{} }\PYG{l+m+mi}{5} \PYG{o}{*} \PYG{n}{np}\PYG{o}{.}\PYG{n}{random}\PYG{o}{.}\PYG{n}{random\PYGZus{}sample}\PYG{p}{(}\PYG{p}{(}\PYG{l+m+mi}{3}\PYG{p}{,} \PYG{l+m+mi}{2}\PYG{p}{)}\PYG{p}{)} \PYG{o}{\PYGZhy{}} \PYG{l+m+mi}{5}
\PYG{g+go}{array([[\PYGZhy{}3.99149989, \PYGZhy{}0.52338984],}
\PYG{g+go}{       [\PYGZhy{}2.99091858, \PYGZhy{}0.79479508],}
\PYG{g+go}{       [\PYGZhy{}1.23204345, \PYGZhy{}1.75224494]])}
\end{Verbatim}

\end{fulllineitems}

\index{rayleigh() (in module acsFromWim2Carness)}

\begin{fulllineitems}
\phantomsection\label{acsFromWim2Carness:acsFromWim2Carness.rayleigh}\pysiglinewithargsret{\code{acsFromWim2Carness.}\bfcode{rayleigh}}{\emph{scale=1.0}, \emph{size=None}}{}
Draw samples from a Rayleigh distribution.

The \(\chi\) and Weibull distributions are generalizations of the
Rayleigh.
\begin{description}
\item[{scale}] \leavevmode{[}scalar{]}
Scale, also equals the mode. Should be \textgreater{}= 0.

\item[{size}] \leavevmode{[}int or tuple of ints, optional{]}
Shape of the output. Default is None, in which case a single
value is returned.

\end{description}

The probability density function for the Rayleigh distribution is
\begin{gather}
\begin{split}P(x;scale) = \frac{x}{scale^2}e^{\frac{-x^2}{2 \cdotp scale^2}}\end{split}\notag
\end{gather}
The Rayleigh distribution arises if the wind speed and wind direction are
both gaussian variables, then the vector wind velocity forms a Rayleigh
distribution. The Rayleigh distribution is used to model the expected
output from wind turbines.

Draw values from the distribution and plot the histogram

\begin{Verbatim}[commandchars=\\\{\}]
\PYG{g+gp}{\PYGZgt{}\PYGZgt{}\PYGZgt{} }\PYG{n}{values} \PYG{o}{=} \PYG{n}{hist}\PYG{p}{(}\PYG{n}{np}\PYG{o}{.}\PYG{n}{random}\PYG{o}{.}\PYG{n}{rayleigh}\PYG{p}{(}\PYG{l+m+mi}{3}\PYG{p}{,} \PYG{l+m+mi}{100000}\PYG{p}{)}\PYG{p}{,} \PYG{n}{bins}\PYG{o}{=}\PYG{l+m+mi}{200}\PYG{p}{,} \PYG{n}{normed}\PYG{o}{=}\PYG{n+nb+bp}{True}\PYG{p}{)}
\end{Verbatim}

Wave heights tend to follow a Rayleigh distribution. If the mean wave
height is 1 meter, what fraction of waves are likely to be larger than 3
meters?

\begin{Verbatim}[commandchars=\\\{\}]
\PYG{g+gp}{\PYGZgt{}\PYGZgt{}\PYGZgt{} }\PYG{n}{meanvalue} \PYG{o}{=} \PYG{l+m+mi}{1}
\PYG{g+gp}{\PYGZgt{}\PYGZgt{}\PYGZgt{} }\PYG{n}{modevalue} \PYG{o}{=} \PYG{n}{np}\PYG{o}{.}\PYG{n}{sqrt}\PYG{p}{(}\PYG{l+m+mi}{2} \PYG{o}{/} \PYG{n}{np}\PYG{o}{.}\PYG{n}{pi}\PYG{p}{)} \PYG{o}{*} \PYG{n}{meanvalue}
\PYG{g+gp}{\PYGZgt{}\PYGZgt{}\PYGZgt{} }\PYG{n}{s} \PYG{o}{=} \PYG{n}{np}\PYG{o}{.}\PYG{n}{random}\PYG{o}{.}\PYG{n}{rayleigh}\PYG{p}{(}\PYG{n}{modevalue}\PYG{p}{,} \PYG{l+m+mi}{1000000}\PYG{p}{)}
\end{Verbatim}

The percentage of waves larger than 3 meters is:

\begin{Verbatim}[commandchars=\\\{\}]
\PYG{g+gp}{\PYGZgt{}\PYGZgt{}\PYGZgt{} }\PYG{l+m+mf}{100.}\PYG{o}{*}\PYG{n+nb}{sum}\PYG{p}{(}\PYG{n}{s}\PYG{o}{\PYGZgt{}}\PYG{l+m+mi}{3}\PYG{p}{)}\PYG{o}{/}\PYG{l+m+mf}{1000000.}
\PYG{g+go}{0.087300000000000003}
\end{Verbatim}

\end{fulllineitems}

\index{sample() (in module acsFromWim2Carness)}

\begin{fulllineitems}
\phantomsection\label{acsFromWim2Carness:acsFromWim2Carness.sample}\pysiglinewithargsret{\code{acsFromWim2Carness.}\bfcode{sample}}{}{}
random\_sample(size=None)

Return random floats in the half-open interval {[}0.0, 1.0).

Results are from the ``continuous uniform'' distribution over the
stated interval.  To sample \(Unif[a, b), b > a\) multiply
the output of \emph{random\_sample} by \emph{(b-a)} and add \emph{a}:

\begin{Verbatim}[commandchars=\\\{\}]
\PYG{p}{(}\PYG{n}{b} \PYG{o}{\PYGZhy{}} \PYG{n}{a}\PYG{p}{)} \PYG{o}{*} \PYG{n}{random\PYGZus{}sample}\PYG{p}{(}\PYG{p}{)} \PYG{o}{+} \PYG{n}{a}
\end{Verbatim}
\begin{description}
\item[{size}] \leavevmode{[}int or tuple of ints, optional{]}
Defines the shape of the returned array of random floats. If None
(the default), returns a single float.

\end{description}
\begin{description}
\item[{out}] \leavevmode{[}float or ndarray of floats{]}
Array of random floats of shape \emph{size} (unless \code{size=None}, in which
case a single float is returned).

\end{description}

\begin{Verbatim}[commandchars=\\\{\}]
\PYG{g+gp}{\PYGZgt{}\PYGZgt{}\PYGZgt{} }\PYG{n}{np}\PYG{o}{.}\PYG{n}{random}\PYG{o}{.}\PYG{n}{random\PYGZus{}sample}\PYG{p}{(}\PYG{p}{)}
\PYG{g+go}{0.47108547995356098}
\PYG{g+gp}{\PYGZgt{}\PYGZgt{}\PYGZgt{} }\PYG{n+nb}{type}\PYG{p}{(}\PYG{n}{np}\PYG{o}{.}\PYG{n}{random}\PYG{o}{.}\PYG{n}{random\PYGZus{}sample}\PYG{p}{(}\PYG{p}{)}\PYG{p}{)}
\PYG{g+go}{\PYGZlt{}type \PYGZsq{}float\PYGZsq{}\PYGZgt{}}
\PYG{g+gp}{\PYGZgt{}\PYGZgt{}\PYGZgt{} }\PYG{n}{np}\PYG{o}{.}\PYG{n}{random}\PYG{o}{.}\PYG{n}{random\PYGZus{}sample}\PYG{p}{(}\PYG{p}{(}\PYG{l+m+mi}{5}\PYG{p}{,}\PYG{p}{)}\PYG{p}{)}
\PYG{g+go}{array([ 0.30220482,  0.86820401,  0.1654503 ,  0.11659149,  0.54323428])}
\end{Verbatim}

Three-by-two array of random numbers from {[}-5, 0):

\begin{Verbatim}[commandchars=\\\{\}]
\PYG{g+gp}{\PYGZgt{}\PYGZgt{}\PYGZgt{} }\PYG{l+m+mi}{5} \PYG{o}{*} \PYG{n}{np}\PYG{o}{.}\PYG{n}{random}\PYG{o}{.}\PYG{n}{random\PYGZus{}sample}\PYG{p}{(}\PYG{p}{(}\PYG{l+m+mi}{3}\PYG{p}{,} \PYG{l+m+mi}{2}\PYG{p}{)}\PYG{p}{)} \PYG{o}{\PYGZhy{}} \PYG{l+m+mi}{5}
\PYG{g+go}{array([[\PYGZhy{}3.99149989, \PYGZhy{}0.52338984],}
\PYG{g+go}{       [\PYGZhy{}2.99091858, \PYGZhy{}0.79479508],}
\PYG{g+go}{       [\PYGZhy{}1.23204345, \PYGZhy{}1.75224494]])}
\end{Verbatim}

\end{fulllineitems}

\index{seed() (in module acsFromWim2Carness)}

\begin{fulllineitems}
\phantomsection\label{acsFromWim2Carness:acsFromWim2Carness.seed}\pysiglinewithargsret{\code{acsFromWim2Carness.}\bfcode{seed}}{\emph{seed=None}}{}
Seed the generator.

This method is called when \emph{RandomState} is initialized. It can be
called again to re-seed the generator. For details, see \emph{RandomState}.
\begin{description}
\item[{seed}] \leavevmode{[}int or array\_like, optional{]}
Seed for \emph{RandomState}.

\end{description}

RandomState

\end{fulllineitems}

\index{set\_state() (in module acsFromWim2Carness)}

\begin{fulllineitems}
\phantomsection\label{acsFromWim2Carness:acsFromWim2Carness.set_state}\pysiglinewithargsret{\code{acsFromWim2Carness.}\bfcode{set\_state}}{\emph{state}}{}
Set the internal state of the generator from a tuple.

For use if one has reason to manually (re-)set the internal state of the
``Mersenne Twister''{\color{red}\bfseries{}{[}1{]}\_} pseudo-random number generating algorithm.
\begin{description}
\item[{state}] \leavevmode{[}tuple(str, ndarray of 624 uints, int, int, float){]}
The \emph{state} tuple has the following items:
\begin{enumerate}
\item {} 
the string `MT19937', specifying the Mersenne Twister algorithm.

\item {} 
a 1-D array of 624 unsigned integers \code{keys}.

\item {} 
an integer \code{pos}.

\item {} 
an integer \code{has\_gauss}.

\item {} 
a float \code{cached\_gaussian}.

\end{enumerate}

\end{description}
\begin{description}
\item[{out}] \leavevmode{[}None{]}
Returns `None' on success.

\end{description}

get\_state

\emph{set\_state} and \emph{get\_state} are not needed to work with any of the
random distributions in NumPy. If the internal state is manually altered,
the user should know exactly what he/she is doing.

For backwards compatibility, the form (str, array of 624 uints, int) is
also accepted although it is missing some information about the cached
Gaussian value: \code{state = ('MT19937', keys, pos)}.

\end{fulllineitems}

\index{shuffle() (in module acsFromWim2Carness)}

\begin{fulllineitems}
\phantomsection\label{acsFromWim2Carness:acsFromWim2Carness.shuffle}\pysiglinewithargsret{\code{acsFromWim2Carness.}\bfcode{shuffle}}{\emph{x}}{}
Modify a sequence in-place by shuffling its contents.
\begin{description}
\item[{x}] \leavevmode{[}array\_like{]}
The array or list to be shuffled.

\end{description}

None

\begin{Verbatim}[commandchars=\\\{\}]
\PYG{g+gp}{\PYGZgt{}\PYGZgt{}\PYGZgt{} }\PYG{n}{arr} \PYG{o}{=} \PYG{n}{np}\PYG{o}{.}\PYG{n}{arange}\PYG{p}{(}\PYG{l+m+mi}{10}\PYG{p}{)}
\PYG{g+gp}{\PYGZgt{}\PYGZgt{}\PYGZgt{} }\PYG{n}{np}\PYG{o}{.}\PYG{n}{random}\PYG{o}{.}\PYG{n}{shuffle}\PYG{p}{(}\PYG{n}{arr}\PYG{p}{)}
\PYG{g+gp}{\PYGZgt{}\PYGZgt{}\PYGZgt{} }\PYG{n}{arr}
\PYG{g+go}{[1 7 5 2 9 4 3 6 0 8]}
\end{Verbatim}

This function only shuffles the array along the first index of a
multi-dimensional array:

\begin{Verbatim}[commandchars=\\\{\}]
\PYG{g+gp}{\PYGZgt{}\PYGZgt{}\PYGZgt{} }\PYG{n}{arr} \PYG{o}{=} \PYG{n}{np}\PYG{o}{.}\PYG{n}{arange}\PYG{p}{(}\PYG{l+m+mi}{9}\PYG{p}{)}\PYG{o}{.}\PYG{n}{reshape}\PYG{p}{(}\PYG{p}{(}\PYG{l+m+mi}{3}\PYG{p}{,} \PYG{l+m+mi}{3}\PYG{p}{)}\PYG{p}{)}
\PYG{g+gp}{\PYGZgt{}\PYGZgt{}\PYGZgt{} }\PYG{n}{np}\PYG{o}{.}\PYG{n}{random}\PYG{o}{.}\PYG{n}{shuffle}\PYG{p}{(}\PYG{n}{arr}\PYG{p}{)}
\PYG{g+gp}{\PYGZgt{}\PYGZgt{}\PYGZgt{} }\PYG{n}{arr}
\PYG{g+go}{array([[3, 4, 5],}
\PYG{g+go}{       [6, 7, 8],}
\PYG{g+go}{       [0, 1, 2]])}
\end{Verbatim}

\end{fulllineitems}

\index{standard\_cauchy() (in module acsFromWim2Carness)}

\begin{fulllineitems}
\phantomsection\label{acsFromWim2Carness:acsFromWim2Carness.standard_cauchy}\pysiglinewithargsret{\code{acsFromWim2Carness.}\bfcode{standard\_cauchy}}{\emph{size=None}}{}
Standard Cauchy distribution with mode = 0.

Also known as the Lorentz distribution.
\begin{description}
\item[{size}] \leavevmode{[}int or tuple of ints{]}
Shape of the output.

\end{description}
\begin{description}
\item[{samples}] \leavevmode{[}ndarray or scalar{]}
The drawn samples.

\end{description}

The probability density function for the full Cauchy distribution is
\begin{gather}
\begin{split}P(x; x_0, \gamma) = \frac{1}{\pi \gamma \bigl[ 1+
(\frac{x-x_0}{\gamma})^2 \bigr] }\end{split}\notag
\end{gather}
and the Standard Cauchy distribution just sets \(x_0=0\) and
\(\gamma=1\)

The Cauchy distribution arises in the solution to the driven harmonic
oscillator problem, and also describes spectral line broadening. It
also describes the distribution of values at which a line tilted at
a random angle will cut the x axis.

When studying hypothesis tests that assume normality, seeing how the
tests perform on data from a Cauchy distribution is a good indicator of
their sensitivity to a heavy-tailed distribution, since the Cauchy looks
very much like a Gaussian distribution, but with heavier tails.

Draw samples and plot the distribution:

\begin{Verbatim}[commandchars=\\\{\}]
\PYG{g+gp}{\PYGZgt{}\PYGZgt{}\PYGZgt{} }\PYG{n}{s} \PYG{o}{=} \PYG{n}{np}\PYG{o}{.}\PYG{n}{random}\PYG{o}{.}\PYG{n}{standard\PYGZus{}cauchy}\PYG{p}{(}\PYG{l+m+mi}{1000000}\PYG{p}{)}
\PYG{g+gp}{\PYGZgt{}\PYGZgt{}\PYGZgt{} }\PYG{n}{s} \PYG{o}{=} \PYG{n}{s}\PYG{p}{[}\PYG{p}{(}\PYG{n}{s}\PYG{o}{\PYGZgt{}}\PYG{o}{\PYGZhy{}}\PYG{l+m+mi}{25}\PYG{p}{)} \PYG{o}{\PYGZam{}} \PYG{p}{(}\PYG{n}{s}\PYG{o}{\PYGZlt{}}\PYG{l+m+mi}{25}\PYG{p}{)}\PYG{p}{]}  \PYG{c}{\PYGZsh{} truncate distribution so it plots well}
\PYG{g+gp}{\PYGZgt{}\PYGZgt{}\PYGZgt{} }\PYG{n}{plt}\PYG{o}{.}\PYG{n}{hist}\PYG{p}{(}\PYG{n}{s}\PYG{p}{,} \PYG{n}{bins}\PYG{o}{=}\PYG{l+m+mi}{100}\PYG{p}{)}
\PYG{g+gp}{\PYGZgt{}\PYGZgt{}\PYGZgt{} }\PYG{n}{plt}\PYG{o}{.}\PYG{n}{show}\PYG{p}{(}\PYG{p}{)}
\end{Verbatim}

\end{fulllineitems}

\index{standard\_exponential() (in module acsFromWim2Carness)}

\begin{fulllineitems}
\phantomsection\label{acsFromWim2Carness:acsFromWim2Carness.standard_exponential}\pysiglinewithargsret{\code{acsFromWim2Carness.}\bfcode{standard\_exponential}}{\emph{size=None}}{}
Draw samples from the standard exponential distribution.

\emph{standard\_exponential} is identical to the exponential distribution
with a scale parameter of 1.
\begin{description}
\item[{size}] \leavevmode{[}int or tuple of ints{]}
Shape of the output.

\end{description}
\begin{description}
\item[{out}] \leavevmode{[}float or ndarray{]}
Drawn samples.

\end{description}

Output a 3x8000 array:

\begin{Verbatim}[commandchars=\\\{\}]
\PYG{g+gp}{\PYGZgt{}\PYGZgt{}\PYGZgt{} }\PYG{n}{n} \PYG{o}{=} \PYG{n}{np}\PYG{o}{.}\PYG{n}{random}\PYG{o}{.}\PYG{n}{standard\PYGZus{}exponential}\PYG{p}{(}\PYG{p}{(}\PYG{l+m+mi}{3}\PYG{p}{,} \PYG{l+m+mi}{8000}\PYG{p}{)}\PYG{p}{)}
\end{Verbatim}

\end{fulllineitems}

\index{standard\_gamma() (in module acsFromWim2Carness)}

\begin{fulllineitems}
\phantomsection\label{acsFromWim2Carness:acsFromWim2Carness.standard_gamma}\pysiglinewithargsret{\code{acsFromWim2Carness.}\bfcode{standard\_gamma}}{\emph{shape}, \emph{size=None}}{}
Draw samples from a Standard Gamma distribution.

Samples are drawn from a Gamma distribution with specified parameters,
shape (sometimes designated ``k'') and scale=1.
\begin{description}
\item[{shape}] \leavevmode{[}float{]}
Parameter, should be \textgreater{} 0.

\item[{size}] \leavevmode{[}int or tuple of ints{]}
Output shape.  If the given shape is, e.g., \code{(m, n, k)}, then
\code{m * n * k} samples are drawn.

\end{description}
\begin{description}
\item[{samples}] \leavevmode{[}ndarray or scalar{]}
The drawn samples.

\end{description}
\begin{description}
\item[{scipy.stats.distributions.gamma}] \leavevmode{[}probability density function,{]}
distribution or cumulative density function, etc.

\end{description}

The probability density for the Gamma distribution is
\begin{gather}
\begin{split}p(x) = x^{k-1}\frac{e^{-x/\theta}}{\theta^k\Gamma(k)},\end{split}\notag
\end{gather}
where \(k\) is the shape and \(\theta\) the scale,
and \(\Gamma\) is the Gamma function.

The Gamma distribution is often used to model the times to failure of
electronic components, and arises naturally in processes for which the
waiting times between Poisson distributed events are relevant.

Draw samples from the distribution:

\begin{Verbatim}[commandchars=\\\{\}]
\PYG{g+gp}{\PYGZgt{}\PYGZgt{}\PYGZgt{} }\PYG{n}{shape}\PYG{p}{,} \PYG{n}{scale} \PYG{o}{=} \PYG{l+m+mf}{2.}\PYG{p}{,} \PYG{l+m+mf}{1.} \PYG{c}{\PYGZsh{} mean and width}
\PYG{g+gp}{\PYGZgt{}\PYGZgt{}\PYGZgt{} }\PYG{n}{s} \PYG{o}{=} \PYG{n}{np}\PYG{o}{.}\PYG{n}{random}\PYG{o}{.}\PYG{n}{standard\PYGZus{}gamma}\PYG{p}{(}\PYG{n}{shape}\PYG{p}{,} \PYG{l+m+mi}{1000000}\PYG{p}{)}
\end{Verbatim}

Display the histogram of the samples, along with
the probability density function:

\begin{Verbatim}[commandchars=\\\{\}]
\PYG{g+gp}{\PYGZgt{}\PYGZgt{}\PYGZgt{} }\PYG{k+kn}{import} \PYG{n+nn}{matplotlib.pyplot} \PYG{k+kn}{as} \PYG{n+nn}{plt}
\PYG{g+gp}{\PYGZgt{}\PYGZgt{}\PYGZgt{} }\PYG{k+kn}{import} \PYG{n+nn}{scipy.special} \PYG{k+kn}{as} \PYG{n+nn}{sps}
\PYG{g+gp}{\PYGZgt{}\PYGZgt{}\PYGZgt{} }\PYG{n}{count}\PYG{p}{,} \PYG{n}{bins}\PYG{p}{,} \PYG{n}{ignored} \PYG{o}{=} \PYG{n}{plt}\PYG{o}{.}\PYG{n}{hist}\PYG{p}{(}\PYG{n}{s}\PYG{p}{,} \PYG{l+m+mi}{50}\PYG{p}{,} \PYG{n}{normed}\PYG{o}{=}\PYG{n+nb+bp}{True}\PYG{p}{)}
\PYG{g+gp}{\PYGZgt{}\PYGZgt{}\PYGZgt{} }\PYG{n}{y} \PYG{o}{=} \PYG{n}{bins}\PYG{o}{*}\PYG{o}{*}\PYG{p}{(}\PYG{n}{shape}\PYG{o}{\PYGZhy{}}\PYG{l+m+mi}{1}\PYG{p}{)} \PYG{o}{*} \PYG{p}{(}\PYG{p}{(}\PYG{n}{np}\PYG{o}{.}\PYG{n}{exp}\PYG{p}{(}\PYG{o}{\PYGZhy{}}\PYG{n}{bins}\PYG{o}{/}\PYG{n}{scale}\PYG{p}{)}\PYG{p}{)}\PYG{o}{/} \PYGZbs{}
\PYG{g+gp}{... }                      \PYG{p}{(}\PYG{n}{sps}\PYG{o}{.}\PYG{n}{gamma}\PYG{p}{(}\PYG{n}{shape}\PYG{p}{)} \PYG{o}{*} \PYG{n}{scale}\PYG{o}{*}\PYG{o}{*}\PYG{n}{shape}\PYG{p}{)}\PYG{p}{)}
\PYG{g+gp}{\PYGZgt{}\PYGZgt{}\PYGZgt{} }\PYG{n}{plt}\PYG{o}{.}\PYG{n}{plot}\PYG{p}{(}\PYG{n}{bins}\PYG{p}{,} \PYG{n}{y}\PYG{p}{,} \PYG{n}{linewidth}\PYG{o}{=}\PYG{l+m+mi}{2}\PYG{p}{,} \PYG{n}{color}\PYG{o}{=}\PYG{l+s}{\PYGZsq{}}\PYG{l+s}{r}\PYG{l+s}{\PYGZsq{}}\PYG{p}{)}
\PYG{g+gp}{\PYGZgt{}\PYGZgt{}\PYGZgt{} }\PYG{n}{plt}\PYG{o}{.}\PYG{n}{show}\PYG{p}{(}\PYG{p}{)}
\end{Verbatim}

\end{fulllineitems}

\index{standard\_normal() (in module acsFromWim2Carness)}

\begin{fulllineitems}
\phantomsection\label{acsFromWim2Carness:acsFromWim2Carness.standard_normal}\pysiglinewithargsret{\code{acsFromWim2Carness.}\bfcode{standard\_normal}}{\emph{size=None}}{}
Returns samples from a Standard Normal distribution (mean=0, stdev=1).
\begin{description}
\item[{size}] \leavevmode{[}int or tuple of ints, optional{]}
Output shape. Default is None, in which case a single value is
returned.

\end{description}
\begin{description}
\item[{out}] \leavevmode{[}float or ndarray{]}
Drawn samples.

\end{description}

\begin{Verbatim}[commandchars=\\\{\}]
\PYG{g+gp}{\PYGZgt{}\PYGZgt{}\PYGZgt{} }\PYG{n}{s} \PYG{o}{=} \PYG{n}{np}\PYG{o}{.}\PYG{n}{random}\PYG{o}{.}\PYG{n}{standard\PYGZus{}normal}\PYG{p}{(}\PYG{l+m+mi}{8000}\PYG{p}{)}
\PYG{g+gp}{\PYGZgt{}\PYGZgt{}\PYGZgt{} }\PYG{n}{s}
\PYG{g+go}{array([ 0.6888893 ,  0.78096262, \PYGZhy{}0.89086505, ...,  0.49876311, \PYGZsh{}random}
\PYG{g+go}{       \PYGZhy{}0.38672696, \PYGZhy{}0.4685006 ])                               \PYGZsh{}random}
\PYG{g+gp}{\PYGZgt{}\PYGZgt{}\PYGZgt{} }\PYG{n}{s}\PYG{o}{.}\PYG{n}{shape}
\PYG{g+go}{(8000,)}
\PYG{g+gp}{\PYGZgt{}\PYGZgt{}\PYGZgt{} }\PYG{n}{s} \PYG{o}{=} \PYG{n}{np}\PYG{o}{.}\PYG{n}{random}\PYG{o}{.}\PYG{n}{standard\PYGZus{}normal}\PYG{p}{(}\PYG{n}{size}\PYG{o}{=}\PYG{p}{(}\PYG{l+m+mi}{3}\PYG{p}{,} \PYG{l+m+mi}{4}\PYG{p}{,} \PYG{l+m+mi}{2}\PYG{p}{)}\PYG{p}{)}
\PYG{g+gp}{\PYGZgt{}\PYGZgt{}\PYGZgt{} }\PYG{n}{s}\PYG{o}{.}\PYG{n}{shape}
\PYG{g+go}{(3, 4, 2)}
\end{Verbatim}

\end{fulllineitems}

\index{standard\_t() (in module acsFromWim2Carness)}

\begin{fulllineitems}
\phantomsection\label{acsFromWim2Carness:acsFromWim2Carness.standard_t}\pysiglinewithargsret{\code{acsFromWim2Carness.}\bfcode{standard\_t}}{\emph{df}, \emph{size=None}}{}
Standard Student's t distribution with df degrees of freedom.

A special case of the hyperbolic distribution.
As \emph{df} gets large, the result resembles that of the standard normal
distribution (\emph{standard\_normal}).
\begin{description}
\item[{df}] \leavevmode{[}int{]}
Degrees of freedom, should be \textgreater{} 0.

\item[{size}] \leavevmode{[}int or tuple of ints, optional{]}
Output shape. Default is None, in which case a single value is
returned.

\end{description}
\begin{description}
\item[{samples}] \leavevmode{[}ndarray or scalar{]}
Drawn samples.

\end{description}

The probability density function for the t distribution is
\begin{gather}
\begin{split}P(x, df) = \frac{\Gamma(\frac{df+1}{2})}{\sqrt{\pi df}
\Gamma(\frac{df}{2})}\Bigl( 1+\frac{x^2}{df} \Bigr)^{-(df+1)/2}\end{split}\notag
\end{gather}
The t test is based on an assumption that the data come from a Normal
distribution. The t test provides a way to test whether the sample mean
(that is the mean calculated from the data) is a good estimate of the true
mean.

The derivation of the t-distribution was forst published in 1908 by William
Gisset while working for the Guinness Brewery in Dublin. Due to proprietary
issues, he had to publish under a pseudonym, and so he used the name
Student.

From Dalgaard page 83 {\color{red}\bfseries{}{[}1{]}\_}, suppose the daily energy intake for 11
women in Kj is:

\begin{Verbatim}[commandchars=\\\{\}]
\PYG{g+gp}{\PYGZgt{}\PYGZgt{}\PYGZgt{} }\PYG{n}{intake} \PYG{o}{=} \PYG{n}{np}\PYG{o}{.}\PYG{n}{array}\PYG{p}{(}\PYG{p}{[}\PYG{l+m+mf}{5260.}\PYG{p}{,} \PYG{l+m+mi}{5470}\PYG{p}{,} \PYG{l+m+mi}{5640}\PYG{p}{,} \PYG{l+m+mi}{6180}\PYG{p}{,} \PYG{l+m+mi}{6390}\PYG{p}{,} \PYG{l+m+mi}{6515}\PYG{p}{,} \PYG{l+m+mi}{6805}\PYG{p}{,} \PYG{l+m+mi}{7515}\PYG{p}{,} \PYGZbs{}
\PYG{g+gp}{... }                   \PYG{l+m+mi}{7515}\PYG{p}{,} \PYG{l+m+mi}{8230}\PYG{p}{,} \PYG{l+m+mi}{8770}\PYG{p}{]}\PYG{p}{)}
\end{Verbatim}

Does their energy intake deviate systematically from the recommended
value of 7725 kJ?

We have 10 degrees of freedom, so is the sample mean within 95\% of the
recommended value?

\begin{Verbatim}[commandchars=\\\{\}]
\PYG{g+gp}{\PYGZgt{}\PYGZgt{}\PYGZgt{} }\PYG{n}{s} \PYG{o}{=} \PYG{n}{np}\PYG{o}{.}\PYG{n}{random}\PYG{o}{.}\PYG{n}{standard\PYGZus{}t}\PYG{p}{(}\PYG{l+m+mi}{10}\PYG{p}{,} \PYG{n}{size}\PYG{o}{=}\PYG{l+m+mi}{100000}\PYG{p}{)}
\PYG{g+gp}{\PYGZgt{}\PYGZgt{}\PYGZgt{} }\PYG{n}{np}\PYG{o}{.}\PYG{n}{mean}\PYG{p}{(}\PYG{n}{intake}\PYG{p}{)}
\PYG{g+go}{6753.636363636364}
\PYG{g+gp}{\PYGZgt{}\PYGZgt{}\PYGZgt{} }\PYG{n}{intake}\PYG{o}{.}\PYG{n}{std}\PYG{p}{(}\PYG{n}{ddof}\PYG{o}{=}\PYG{l+m+mi}{1}\PYG{p}{)}
\PYG{g+go}{1142.1232221373727}
\end{Verbatim}

Calculate the t statistic, setting the ddof parameter to the unbiased
value so the divisor in the standard deviation will be degrees of
freedom, N-1.

\begin{Verbatim}[commandchars=\\\{\}]
\PYG{g+gp}{\PYGZgt{}\PYGZgt{}\PYGZgt{} }\PYG{n}{t} \PYG{o}{=} \PYG{p}{(}\PYG{n}{np}\PYG{o}{.}\PYG{n}{mean}\PYG{p}{(}\PYG{n}{intake}\PYG{p}{)}\PYG{o}{\PYGZhy{}}\PYG{l+m+mi}{7725}\PYG{p}{)}\PYG{o}{/}\PYG{p}{(}\PYG{n}{intake}\PYG{o}{.}\PYG{n}{std}\PYG{p}{(}\PYG{n}{ddof}\PYG{o}{=}\PYG{l+m+mi}{1}\PYG{p}{)}\PYG{o}{/}\PYG{n}{np}\PYG{o}{.}\PYG{n}{sqrt}\PYG{p}{(}\PYG{n+nb}{len}\PYG{p}{(}\PYG{n}{intake}\PYG{p}{)}\PYG{p}{)}\PYG{p}{)}
\PYG{g+gp}{\PYGZgt{}\PYGZgt{}\PYGZgt{} }\PYG{k+kn}{import} \PYG{n+nn}{matplotlib.pyplot} \PYG{k+kn}{as} \PYG{n+nn}{plt}
\PYG{g+gp}{\PYGZgt{}\PYGZgt{}\PYGZgt{} }\PYG{n}{h} \PYG{o}{=} \PYG{n}{plt}\PYG{o}{.}\PYG{n}{hist}\PYG{p}{(}\PYG{n}{s}\PYG{p}{,} \PYG{n}{bins}\PYG{o}{=}\PYG{l+m+mi}{100}\PYG{p}{,} \PYG{n}{normed}\PYG{o}{=}\PYG{n+nb+bp}{True}\PYG{p}{)}
\end{Verbatim}

For a one-sided t-test, how far out in the distribution does the t
statistic appear?

\begin{Verbatim}[commandchars=\\\{\}]
\PYG{g+gp}{\PYGZgt{}\PYGZgt{}\PYGZgt{} }\PYG{o}{\PYGZgt{}\PYGZgt{}}\PYG{o}{\PYGZgt{}} \PYG{n}{np}\PYG{o}{.}\PYG{n}{sum}\PYG{p}{(}\PYG{n}{s}\PYG{o}{\PYGZlt{}}\PYG{n}{t}\PYG{p}{)} \PYG{o}{/} \PYG{n+nb}{float}\PYG{p}{(}\PYG{n+nb}{len}\PYG{p}{(}\PYG{n}{s}\PYG{p}{)}\PYG{p}{)}
\PYG{g+go}{0.0090699999999999999  \PYGZsh{}random}
\end{Verbatim}

So the p-value is about 0.009, which says the null hypothesis has a
probability of about 99\% of being true.

\end{fulllineitems}

\index{triangular() (in module acsFromWim2Carness)}

\begin{fulllineitems}
\phantomsection\label{acsFromWim2Carness:acsFromWim2Carness.triangular}\pysiglinewithargsret{\code{acsFromWim2Carness.}\bfcode{triangular}}{\emph{left}, \emph{mode}, \emph{right}, \emph{size=None}}{}
Draw samples from the triangular distribution.

The triangular distribution is a continuous probability distribution with
lower limit left, peak at mode, and upper limit right. Unlike the other
distributions, these parameters directly define the shape of the pdf.
\begin{description}
\item[{left}] \leavevmode{[}scalar{]}
Lower limit.

\item[{mode}] \leavevmode{[}scalar{]}
The value where the peak of the distribution occurs.
The value should fulfill the condition \code{left \textless{}= mode \textless{}= right}.

\item[{right}] \leavevmode{[}scalar{]}
Upper limit, should be larger than \emph{left}.

\item[{size}] \leavevmode{[}int or tuple of ints, optional{]}
Output shape. Default is None, in which case a single value is
returned.

\end{description}
\begin{description}
\item[{samples}] \leavevmode{[}ndarray or scalar{]}
The returned samples all lie in the interval {[}left, right{]}.

\end{description}

The probability density function for the Triangular distribution is
\begin{gather}
\begin{split}P(x;l, m, r) = \begin{cases}
\frac{2(x-l)}{(r-l)(m-l)}& \text{for $l \leq x \leq m$},\\
\frac{2(m-x)}{(r-l)(r-m)}& \text{for $m \leq x \leq r$},\\
0& \text{otherwise}.
\end{cases}\end{split}\notag
\end{gather}
The triangular distribution is often used in ill-defined problems where the
underlying distribution is not known, but some knowledge of the limits and
mode exists. Often it is used in simulations.

Draw values from the distribution and plot the histogram:

\begin{Verbatim}[commandchars=\\\{\}]
\PYG{g+gp}{\PYGZgt{}\PYGZgt{}\PYGZgt{} }\PYG{k+kn}{import} \PYG{n+nn}{matplotlib.pyplot} \PYG{k+kn}{as} \PYG{n+nn}{plt}
\PYG{g+gp}{\PYGZgt{}\PYGZgt{}\PYGZgt{} }\PYG{n}{h} \PYG{o}{=} \PYG{n}{plt}\PYG{o}{.}\PYG{n}{hist}\PYG{p}{(}\PYG{n}{np}\PYG{o}{.}\PYG{n}{random}\PYG{o}{.}\PYG{n}{triangular}\PYG{p}{(}\PYG{o}{\PYGZhy{}}\PYG{l+m+mi}{3}\PYG{p}{,} \PYG{l+m+mi}{0}\PYG{p}{,} \PYG{l+m+mi}{8}\PYG{p}{,} \PYG{l+m+mi}{100000}\PYG{p}{)}\PYG{p}{,} \PYG{n}{bins}\PYG{o}{=}\PYG{l+m+mi}{200}\PYG{p}{,}
\PYG{g+gp}{... }             \PYG{n}{normed}\PYG{o}{=}\PYG{n+nb+bp}{True}\PYG{p}{)}
\PYG{g+gp}{\PYGZgt{}\PYGZgt{}\PYGZgt{} }\PYG{n}{plt}\PYG{o}{.}\PYG{n}{show}\PYG{p}{(}\PYG{p}{)}
\end{Verbatim}

\end{fulllineitems}

\index{uniform() (in module acsFromWim2Carness)}

\begin{fulllineitems}
\phantomsection\label{acsFromWim2Carness:acsFromWim2Carness.uniform}\pysiglinewithargsret{\code{acsFromWim2Carness.}\bfcode{uniform}}{\emph{low=0.0}, \emph{high=1.0}, \emph{size=1}}{}
Draw samples from a uniform distribution.

Samples are uniformly distributed over the half-open interval
\code{{[}low, high)} (includes low, but excludes high).  In other words,
any value within the given interval is equally likely to be drawn
by \emph{uniform}.
\begin{description}
\item[{low}] \leavevmode{[}float, optional{]}
Lower boundary of the output interval.  All values generated will be
greater than or equal to low.  The default value is 0.

\item[{high}] \leavevmode{[}float{]}
Upper boundary of the output interval.  All values generated will be
less than high.  The default value is 1.0.

\item[{size}] \leavevmode{[}int or tuple of ints, optional{]}
Shape of output.  If the given size is, for example, (m,n,k),
m*n*k samples are generated.  If no shape is specified, a single sample
is returned.

\end{description}
\begin{description}
\item[{out}] \leavevmode{[}ndarray{]}
Drawn samples, with shape \emph{size}.

\end{description}

randint : Discrete uniform distribution, yielding integers.
random\_integers : Discrete uniform distribution over the closed
\begin{quote}

interval \code{{[}low, high{]}}.
\end{quote}

random\_sample : Floats uniformly distributed over \code{{[}0, 1)}.
random : Alias for \emph{random\_sample}.
rand : Convenience function that accepts dimensions as input, e.g.,
\begin{quote}

\code{rand(2,2)} would generate a 2-by-2 array of floats,
uniformly distributed over \code{{[}0, 1)}.
\end{quote}

The probability density function of the uniform distribution is
\begin{gather}
\begin{split}p(x) = \frac{1}{b - a}\end{split}\notag
\end{gather}
anywhere within the interval \code{{[}a, b)}, and zero elsewhere.

Draw samples from the distribution:

\begin{Verbatim}[commandchars=\\\{\}]
\PYG{g+gp}{\PYGZgt{}\PYGZgt{}\PYGZgt{} }\PYG{n}{s} \PYG{o}{=} \PYG{n}{np}\PYG{o}{.}\PYG{n}{random}\PYG{o}{.}\PYG{n}{uniform}\PYG{p}{(}\PYG{o}{\PYGZhy{}}\PYG{l+m+mi}{1}\PYG{p}{,}\PYG{l+m+mi}{0}\PYG{p}{,}\PYG{l+m+mi}{1000}\PYG{p}{)}
\end{Verbatim}

All values are within the given interval:

\begin{Verbatim}[commandchars=\\\{\}]
\PYG{g+gp}{\PYGZgt{}\PYGZgt{}\PYGZgt{} }\PYG{n}{np}\PYG{o}{.}\PYG{n}{all}\PYG{p}{(}\PYG{n}{s} \PYG{o}{\PYGZgt{}}\PYG{o}{=} \PYG{o}{\PYGZhy{}}\PYG{l+m+mi}{1}\PYG{p}{)}
\PYG{g+go}{True}
\PYG{g+gp}{\PYGZgt{}\PYGZgt{}\PYGZgt{} }\PYG{n}{np}\PYG{o}{.}\PYG{n}{all}\PYG{p}{(}\PYG{n}{s} \PYG{o}{\PYGZlt{}} \PYG{l+m+mi}{0}\PYG{p}{)}
\PYG{g+go}{True}
\end{Verbatim}

Display the histogram of the samples, along with the
probability density function:

\begin{Verbatim}[commandchars=\\\{\}]
\PYG{g+gp}{\PYGZgt{}\PYGZgt{}\PYGZgt{} }\PYG{k+kn}{import} \PYG{n+nn}{matplotlib.pyplot} \PYG{k+kn}{as} \PYG{n+nn}{plt}
\PYG{g+gp}{\PYGZgt{}\PYGZgt{}\PYGZgt{} }\PYG{n}{count}\PYG{p}{,} \PYG{n}{bins}\PYG{p}{,} \PYG{n}{ignored} \PYG{o}{=} \PYG{n}{plt}\PYG{o}{.}\PYG{n}{hist}\PYG{p}{(}\PYG{n}{s}\PYG{p}{,} \PYG{l+m+mi}{15}\PYG{p}{,} \PYG{n}{normed}\PYG{o}{=}\PYG{n+nb+bp}{True}\PYG{p}{)}
\PYG{g+gp}{\PYGZgt{}\PYGZgt{}\PYGZgt{} }\PYG{n}{plt}\PYG{o}{.}\PYG{n}{plot}\PYG{p}{(}\PYG{n}{bins}\PYG{p}{,} \PYG{n}{np}\PYG{o}{.}\PYG{n}{ones\PYGZus{}like}\PYG{p}{(}\PYG{n}{bins}\PYG{p}{)}\PYG{p}{,} \PYG{n}{linewidth}\PYG{o}{=}\PYG{l+m+mi}{2}\PYG{p}{,} \PYG{n}{color}\PYG{o}{=}\PYG{l+s}{\PYGZsq{}}\PYG{l+s}{r}\PYG{l+s}{\PYGZsq{}}\PYG{p}{)}
\PYG{g+gp}{\PYGZgt{}\PYGZgt{}\PYGZgt{} }\PYG{n}{plt}\PYG{o}{.}\PYG{n}{show}\PYG{p}{(}\PYG{p}{)}
\end{Verbatim}

\end{fulllineitems}

\index{vonmises() (in module acsFromWim2Carness)}

\begin{fulllineitems}
\phantomsection\label{acsFromWim2Carness:acsFromWim2Carness.vonmises}\pysiglinewithargsret{\code{acsFromWim2Carness.}\bfcode{vonmises}}{\emph{mu}, \emph{kappa}, \emph{size=None}}{}
Draw samples from a von Mises distribution.

Samples are drawn from a von Mises distribution with specified mode
(mu) and dispersion (kappa), on the interval {[}-pi, pi{]}.

The von Mises distribution (also known as the circular normal
distribution) is a continuous probability distribution on the unit
circle.  It may be thought of as the circular analogue of the normal
distribution.
\begin{description}
\item[{mu}] \leavevmode{[}float{]}
Mode (``center'') of the distribution.

\item[{kappa}] \leavevmode{[}float{]}
Dispersion of the distribution, has to be \textgreater{}=0.

\item[{size}] \leavevmode{[}int or tuple of int{]}
Output shape.  If the given shape is, e.g., \code{(m, n, k)}, then
\code{m * n * k} samples are drawn.

\end{description}
\begin{description}
\item[{samples}] \leavevmode{[}scalar or ndarray{]}
The returned samples, which are in the interval {[}-pi, pi{]}.

\end{description}
\begin{description}
\item[{scipy.stats.distributions.vonmises}] \leavevmode{[}probability density function,{]}
distribution, or cumulative density function, etc.

\end{description}

The probability density for the von Mises distribution is
\begin{gather}
\begin{split}p(x) = \frac{e^{\kappa cos(x-\mu)}}{2\pi I_0(\kappa)},\end{split}\notag
\end{gather}
where \(\mu\) is the mode and \(\kappa\) the dispersion,
and \(I_0(\kappa)\) is the modified Bessel function of order 0.

The von Mises is named for Richard Edler von Mises, who was born in
Austria-Hungary, in what is now the Ukraine.  He fled to the United
States in 1939 and became a professor at Harvard.  He worked in
probability theory, aerodynamics, fluid mechanics, and philosophy of
science.

Abramowitz, M. and Stegun, I. A. (ed.), \emph{Handbook of Mathematical
Functions}, New York: Dover, 1965.

von Mises, R., \emph{Mathematical Theory of Probability and Statistics},
New York: Academic Press, 1964.

Draw samples from the distribution:

\begin{Verbatim}[commandchars=\\\{\}]
\PYG{g+gp}{\PYGZgt{}\PYGZgt{}\PYGZgt{} }\PYG{n}{mu}\PYG{p}{,} \PYG{n}{kappa} \PYG{o}{=} \PYG{l+m+mf}{0.0}\PYG{p}{,} \PYG{l+m+mf}{4.0} \PYG{c}{\PYGZsh{} mean and dispersion}
\PYG{g+gp}{\PYGZgt{}\PYGZgt{}\PYGZgt{} }\PYG{n}{s} \PYG{o}{=} \PYG{n}{np}\PYG{o}{.}\PYG{n}{random}\PYG{o}{.}\PYG{n}{vonmises}\PYG{p}{(}\PYG{n}{mu}\PYG{p}{,} \PYG{n}{kappa}\PYG{p}{,} \PYG{l+m+mi}{1000}\PYG{p}{)}
\end{Verbatim}

Display the histogram of the samples, along with
the probability density function:

\begin{Verbatim}[commandchars=\\\{\}]
\PYG{g+gp}{\PYGZgt{}\PYGZgt{}\PYGZgt{} }\PYG{k+kn}{import} \PYG{n+nn}{matplotlib.pyplot} \PYG{k+kn}{as} \PYG{n+nn}{plt}
\PYG{g+gp}{\PYGZgt{}\PYGZgt{}\PYGZgt{} }\PYG{k+kn}{import} \PYG{n+nn}{scipy.special} \PYG{k+kn}{as} \PYG{n+nn}{sps}
\PYG{g+gp}{\PYGZgt{}\PYGZgt{}\PYGZgt{} }\PYG{n}{count}\PYG{p}{,} \PYG{n}{bins}\PYG{p}{,} \PYG{n}{ignored} \PYG{o}{=} \PYG{n}{plt}\PYG{o}{.}\PYG{n}{hist}\PYG{p}{(}\PYG{n}{s}\PYG{p}{,} \PYG{l+m+mi}{50}\PYG{p}{,} \PYG{n}{normed}\PYG{o}{=}\PYG{n+nb+bp}{True}\PYG{p}{)}
\PYG{g+gp}{\PYGZgt{}\PYGZgt{}\PYGZgt{} }\PYG{n}{x} \PYG{o}{=} \PYG{n}{np}\PYG{o}{.}\PYG{n}{arange}\PYG{p}{(}\PYG{o}{\PYGZhy{}}\PYG{n}{np}\PYG{o}{.}\PYG{n}{pi}\PYG{p}{,} \PYG{n}{np}\PYG{o}{.}\PYG{n}{pi}\PYG{p}{,} \PYG{l+m+mi}{2}\PYG{o}{*}\PYG{n}{np}\PYG{o}{.}\PYG{n}{pi}\PYG{o}{/}\PYG{l+m+mf}{50.}\PYG{p}{)}
\PYG{g+gp}{\PYGZgt{}\PYGZgt{}\PYGZgt{} }\PYG{n}{y} \PYG{o}{=} \PYG{o}{\PYGZhy{}}\PYG{n}{np}\PYG{o}{.}\PYG{n}{exp}\PYG{p}{(}\PYG{n}{kappa}\PYG{o}{*}\PYG{n}{np}\PYG{o}{.}\PYG{n}{cos}\PYG{p}{(}\PYG{n}{x}\PYG{o}{\PYGZhy{}}\PYG{n}{mu}\PYG{p}{)}\PYG{p}{)}\PYG{o}{/}\PYG{p}{(}\PYG{l+m+mi}{2}\PYG{o}{*}\PYG{n}{np}\PYG{o}{.}\PYG{n}{pi}\PYG{o}{*}\PYG{n}{sps}\PYG{o}{.}\PYG{n}{jn}\PYG{p}{(}\PYG{l+m+mi}{0}\PYG{p}{,}\PYG{n}{kappa}\PYG{p}{)}\PYG{p}{)}
\PYG{g+gp}{\PYGZgt{}\PYGZgt{}\PYGZgt{} }\PYG{n}{plt}\PYG{o}{.}\PYG{n}{plot}\PYG{p}{(}\PYG{n}{x}\PYG{p}{,} \PYG{n}{y}\PYG{o}{/}\PYG{n+nb}{max}\PYG{p}{(}\PYG{n}{y}\PYG{p}{)}\PYG{p}{,} \PYG{n}{linewidth}\PYG{o}{=}\PYG{l+m+mi}{2}\PYG{p}{,} \PYG{n}{color}\PYG{o}{=}\PYG{l+s}{\PYGZsq{}}\PYG{l+s}{r}\PYG{l+s}{\PYGZsq{}}\PYG{p}{)}
\PYG{g+gp}{\PYGZgt{}\PYGZgt{}\PYGZgt{} }\PYG{n}{plt}\PYG{o}{.}\PYG{n}{show}\PYG{p}{(}\PYG{p}{)}
\end{Verbatim}

\end{fulllineitems}

\index{wald() (in module acsFromWim2Carness)}

\begin{fulllineitems}
\phantomsection\label{acsFromWim2Carness:acsFromWim2Carness.wald}\pysiglinewithargsret{\code{acsFromWim2Carness.}\bfcode{wald}}{\emph{mean}, \emph{scale}, \emph{size=None}}{}
Draw samples from a Wald, or Inverse Gaussian, distribution.

As the scale approaches infinity, the distribution becomes more like a
Gaussian.

Some references claim that the Wald is an Inverse Gaussian with mean=1, but
this is by no means universal.

The Inverse Gaussian distribution was first studied in relationship to
Brownian motion. In 1956 M.C.K. Tweedie used the name Inverse Gaussian
because there is an inverse relationship between the time to cover a unit
distance and distance covered in unit time.
\begin{description}
\item[{mean}] \leavevmode{[}scalar{]}
Distribution mean, should be \textgreater{} 0.

\item[{scale}] \leavevmode{[}scalar{]}
Scale parameter, should be \textgreater{}= 0.

\item[{size}] \leavevmode{[}int or tuple of ints, optional{]}
Output shape. Default is None, in which case a single value is
returned.

\end{description}
\begin{description}
\item[{samples}] \leavevmode{[}ndarray or scalar{]}
Drawn sample, all greater than zero.

\end{description}

The probability density function for the Wald distribution is
\begin{gather}
\begin{split}P(x;mean,scale) = \sqrt{\frac{scale}{2\pi x^3}}e^
\frac{-scale(x-mean)^2}{2\cdotp mean^2x}\end{split}\notag
\end{gather}
As noted above the Inverse Gaussian distribution first arise from attempts
to model Brownian Motion. It is also a competitor to the Weibull for use in
reliability modeling and modeling stock returns and interest rate
processes.

Draw values from the distribution and plot the histogram:

\begin{Verbatim}[commandchars=\\\{\}]
\PYG{g+gp}{\PYGZgt{}\PYGZgt{}\PYGZgt{} }\PYG{k+kn}{import} \PYG{n+nn}{matplotlib.pyplot} \PYG{k+kn}{as} \PYG{n+nn}{plt}
\PYG{g+gp}{\PYGZgt{}\PYGZgt{}\PYGZgt{} }\PYG{n}{h} \PYG{o}{=} \PYG{n}{plt}\PYG{o}{.}\PYG{n}{hist}\PYG{p}{(}\PYG{n}{np}\PYG{o}{.}\PYG{n}{random}\PYG{o}{.}\PYG{n}{wald}\PYG{p}{(}\PYG{l+m+mi}{3}\PYG{p}{,} \PYG{l+m+mi}{2}\PYG{p}{,} \PYG{l+m+mi}{100000}\PYG{p}{)}\PYG{p}{,} \PYG{n}{bins}\PYG{o}{=}\PYG{l+m+mi}{200}\PYG{p}{,} \PYG{n}{normed}\PYG{o}{=}\PYG{n+nb+bp}{True}\PYG{p}{)}
\PYG{g+gp}{\PYGZgt{}\PYGZgt{}\PYGZgt{} }\PYG{n}{plt}\PYG{o}{.}\PYG{n}{show}\PYG{p}{(}\PYG{p}{)}
\end{Verbatim}

\end{fulllineitems}

\index{weibull() (in module acsFromWim2Carness)}

\begin{fulllineitems}
\phantomsection\label{acsFromWim2Carness:acsFromWim2Carness.weibull}\pysiglinewithargsret{\code{acsFromWim2Carness.}\bfcode{weibull}}{\emph{a}, \emph{size=None}}{}
Weibull distribution.

Draw samples from a 1-parameter Weibull distribution with the given
shape parameter \emph{a}.
\begin{gather}
\begin{split}X = (-ln(U))^{1/a}\end{split}\notag
\end{gather}
Here, U is drawn from the uniform distribution over (0,1{]}.

The more common 2-parameter Weibull, including a scale parameter
\(\lambda\) is just \(X = \lambda(-ln(U))^{1/a}\).
\begin{description}
\item[{a}] \leavevmode{[}float{]}
Shape of the distribution.

\item[{size}] \leavevmode{[}tuple of ints{]}
Output shape.  If the given shape is, e.g., \code{(m, n, k)}, then
\code{m * n * k} samples are drawn.

\end{description}

scipy.stats.distributions.weibull\_max
scipy.stats.distributions.weibull\_min
scipy.stats.distributions.genextreme
gumbel

The Weibull (or Type III asymptotic extreme value distribution for smallest
values, SEV Type III, or Rosin-Rammler distribution) is one of a class of
Generalized Extreme Value (GEV) distributions used in modeling extreme
value problems.  This class includes the Gumbel and Frechet distributions.

The probability density for the Weibull distribution is
\begin{gather}
\begin{split}p(x) = \frac{a}
{\lambda}(\frac{x}{\lambda})^{a-1}e^{-(x/\lambda)^a},\end{split}\notag
\end{gather}
where \(a\) is the shape and \(\lambda\) the scale.

The function has its peak (the mode) at
\(\lambda(\frac{a-1}{a})^{1/a}\).

When \code{a = 1}, the Weibull distribution reduces to the exponential
distribution.

Draw samples from the distribution:

\begin{Verbatim}[commandchars=\\\{\}]
\PYG{g+gp}{\PYGZgt{}\PYGZgt{}\PYGZgt{} }\PYG{n}{a} \PYG{o}{=} \PYG{l+m+mf}{5.} \PYG{c}{\PYGZsh{} shape}
\PYG{g+gp}{\PYGZgt{}\PYGZgt{}\PYGZgt{} }\PYG{n}{s} \PYG{o}{=} \PYG{n}{np}\PYG{o}{.}\PYG{n}{random}\PYG{o}{.}\PYG{n}{weibull}\PYG{p}{(}\PYG{n}{a}\PYG{p}{,} \PYG{l+m+mi}{1000}\PYG{p}{)}
\end{Verbatim}

Display the histogram of the samples, along with
the probability density function:

\begin{Verbatim}[commandchars=\\\{\}]
\PYG{g+gp}{\PYGZgt{}\PYGZgt{}\PYGZgt{} }\PYG{k+kn}{import} \PYG{n+nn}{matplotlib.pyplot} \PYG{k+kn}{as} \PYG{n+nn}{plt}
\PYG{g+gp}{\PYGZgt{}\PYGZgt{}\PYGZgt{} }\PYG{n}{x} \PYG{o}{=} \PYG{n}{np}\PYG{o}{.}\PYG{n}{arange}\PYG{p}{(}\PYG{l+m+mi}{1}\PYG{p}{,}\PYG{l+m+mf}{100.}\PYG{p}{)}\PYG{o}{/}\PYG{l+m+mf}{50.}
\PYG{g+gp}{\PYGZgt{}\PYGZgt{}\PYGZgt{} }\PYG{k}{def} \PYG{n+nf}{weib}\PYG{p}{(}\PYG{n}{x}\PYG{p}{,}\PYG{n}{n}\PYG{p}{,}\PYG{n}{a}\PYG{p}{)}\PYG{p}{:}
\PYG{g+gp}{... }    \PYG{k}{return} \PYG{p}{(}\PYG{n}{a} \PYG{o}{/} \PYG{n}{n}\PYG{p}{)} \PYG{o}{*} \PYG{p}{(}\PYG{n}{x} \PYG{o}{/} \PYG{n}{n}\PYG{p}{)}\PYG{o}{*}\PYG{o}{*}\PYG{p}{(}\PYG{n}{a} \PYG{o}{\PYGZhy{}} \PYG{l+m+mi}{1}\PYG{p}{)} \PYG{o}{*} \PYG{n}{np}\PYG{o}{.}\PYG{n}{exp}\PYG{p}{(}\PYG{o}{\PYGZhy{}}\PYG{p}{(}\PYG{n}{x} \PYG{o}{/} \PYG{n}{n}\PYG{p}{)}\PYG{o}{*}\PYG{o}{*}\PYG{n}{a}\PYG{p}{)}
\end{Verbatim}

\begin{Verbatim}[commandchars=\\\{\}]
\PYG{g+gp}{\PYGZgt{}\PYGZgt{}\PYGZgt{} }\PYG{n}{count}\PYG{p}{,} \PYG{n}{bins}\PYG{p}{,} \PYG{n}{ignored} \PYG{o}{=} \PYG{n}{plt}\PYG{o}{.}\PYG{n}{hist}\PYG{p}{(}\PYG{n}{np}\PYG{o}{.}\PYG{n}{random}\PYG{o}{.}\PYG{n}{weibull}\PYG{p}{(}\PYG{l+m+mf}{5.}\PYG{p}{,}\PYG{l+m+mi}{1000}\PYG{p}{)}\PYG{p}{)}
\PYG{g+gp}{\PYGZgt{}\PYGZgt{}\PYGZgt{} }\PYG{n}{x} \PYG{o}{=} \PYG{n}{np}\PYG{o}{.}\PYG{n}{arange}\PYG{p}{(}\PYG{l+m+mi}{1}\PYG{p}{,}\PYG{l+m+mf}{100.}\PYG{p}{)}\PYG{o}{/}\PYG{l+m+mf}{50.}
\PYG{g+gp}{\PYGZgt{}\PYGZgt{}\PYGZgt{} }\PYG{n}{scale} \PYG{o}{=} \PYG{n}{count}\PYG{o}{.}\PYG{n}{max}\PYG{p}{(}\PYG{p}{)}\PYG{o}{/}\PYG{n}{weib}\PYG{p}{(}\PYG{n}{x}\PYG{p}{,} \PYG{l+m+mf}{1.}\PYG{p}{,} \PYG{l+m+mf}{5.}\PYG{p}{)}\PYG{o}{.}\PYG{n}{max}\PYG{p}{(}\PYG{p}{)}
\PYG{g+gp}{\PYGZgt{}\PYGZgt{}\PYGZgt{} }\PYG{n}{plt}\PYG{o}{.}\PYG{n}{plot}\PYG{p}{(}\PYG{n}{x}\PYG{p}{,} \PYG{n}{weib}\PYG{p}{(}\PYG{n}{x}\PYG{p}{,} \PYG{l+m+mf}{1.}\PYG{p}{,} \PYG{l+m+mf}{5.}\PYG{p}{)}\PYG{o}{*}\PYG{n}{scale}\PYG{p}{)}
\PYG{g+gp}{\PYGZgt{}\PYGZgt{}\PYGZgt{} }\PYG{n}{plt}\PYG{o}{.}\PYG{n}{show}\PYG{p}{(}\PYG{p}{)}
\end{Verbatim}

\end{fulllineitems}

\index{zipf() (in module acsFromWim2Carness)}

\begin{fulllineitems}
\phantomsection\label{acsFromWim2Carness:acsFromWim2Carness.zipf}\pysiglinewithargsret{\code{acsFromWim2Carness.}\bfcode{zipf}}{\emph{a}, \emph{size=None}}{}
Draw samples from a Zipf distribution.

Samples are drawn from a Zipf distribution with specified parameter
\emph{a} \textgreater{} 1.

The Zipf distribution (also known as the zeta distribution) is a
continuous probability distribution that satisfies Zipf's law: the
frequency of an item is inversely proportional to its rank in a
frequency table.
\begin{description}
\item[{a}] \leavevmode{[}float \textgreater{} 1{]}
Distribution parameter.

\item[{size}] \leavevmode{[}int or tuple of int, optional{]}
Output shape.  If the given shape is, e.g., \code{(m, n, k)}, then
\code{m * n * k} samples are drawn; a single integer is equivalent in
its result to providing a mono-tuple, i.e., a 1-D array of length
\emph{size} is returned.  The default is None, in which case a single
scalar is returned.

\end{description}
\begin{description}
\item[{samples}] \leavevmode{[}scalar or ndarray{]}
The returned samples are greater than or equal to one.

\end{description}
\begin{description}
\item[{scipy.stats.distributions.zipf}] \leavevmode{[}probability density function,{]}
distribution, or cumulative density function, etc.

\end{description}

The probability density for the Zipf distribution is
\begin{gather}
\begin{split}p(x) = \frac{x^{-a}}{\zeta(a)},\end{split}\notag
\end{gather}
where \(\zeta\) is the Riemann Zeta function.

It is named for the American linguist George Kingsley Zipf, who noted
that the frequency of any word in a sample of a language is inversely
proportional to its rank in the frequency table.

Zipf, G. K., \emph{Selected Studies of the Principle of Relative Frequency
in Language}, Cambridge, MA: Harvard Univ. Press, 1932.

Draw samples from the distribution:

\begin{Verbatim}[commandchars=\\\{\}]
\PYG{g+gp}{\PYGZgt{}\PYGZgt{}\PYGZgt{} }\PYG{n}{a} \PYG{o}{=} \PYG{l+m+mf}{2.} \PYG{c}{\PYGZsh{} parameter}
\PYG{g+gp}{\PYGZgt{}\PYGZgt{}\PYGZgt{} }\PYG{n}{s} \PYG{o}{=} \PYG{n}{np}\PYG{o}{.}\PYG{n}{random}\PYG{o}{.}\PYG{n}{zipf}\PYG{p}{(}\PYG{n}{a}\PYG{p}{,} \PYG{l+m+mi}{1000}\PYG{p}{)}
\end{Verbatim}

Display the histogram of the samples, along with
the probability density function:

\begin{Verbatim}[commandchars=\\\{\}]
\PYG{g+gp}{\PYGZgt{}\PYGZgt{}\PYGZgt{} }\PYG{k+kn}{import} \PYG{n+nn}{matplotlib.pyplot} \PYG{k+kn}{as} \PYG{n+nn}{plt}
\PYG{g+gp}{\PYGZgt{}\PYGZgt{}\PYGZgt{} }\PYG{k+kn}{import} \PYG{n+nn}{scipy.special} \PYG{k+kn}{as} \PYG{n+nn}{sps}
\PYG{g+go}{Truncate s values at 50 so plot is interesting}
\PYG{g+gp}{\PYGZgt{}\PYGZgt{}\PYGZgt{} }\PYG{n}{count}\PYG{p}{,} \PYG{n}{bins}\PYG{p}{,} \PYG{n}{ignored} \PYG{o}{=} \PYG{n}{plt}\PYG{o}{.}\PYG{n}{hist}\PYG{p}{(}\PYG{n}{s}\PYG{p}{[}\PYG{n}{s}\PYG{o}{\PYGZlt{}}\PYG{l+m+mi}{50}\PYG{p}{]}\PYG{p}{,} \PYG{l+m+mi}{50}\PYG{p}{,} \PYG{n}{normed}\PYG{o}{=}\PYG{n+nb+bp}{True}\PYG{p}{)}
\PYG{g+gp}{\PYGZgt{}\PYGZgt{}\PYGZgt{} }\PYG{n}{x} \PYG{o}{=} \PYG{n}{np}\PYG{o}{.}\PYG{n}{arange}\PYG{p}{(}\PYG{l+m+mf}{1.}\PYG{p}{,} \PYG{l+m+mf}{50.}\PYG{p}{)}
\PYG{g+gp}{\PYGZgt{}\PYGZgt{}\PYGZgt{} }\PYG{n}{y} \PYG{o}{=} \PYG{n}{x}\PYG{o}{*}\PYG{o}{*}\PYG{p}{(}\PYG{o}{\PYGZhy{}}\PYG{n}{a}\PYG{p}{)}\PYG{o}{/}\PYG{n}{sps}\PYG{o}{.}\PYG{n}{zetac}\PYG{p}{(}\PYG{n}{a}\PYG{p}{)}
\PYG{g+gp}{\PYGZgt{}\PYGZgt{}\PYGZgt{} }\PYG{n}{plt}\PYG{o}{.}\PYG{n}{plot}\PYG{p}{(}\PYG{n}{x}\PYG{p}{,} \PYG{n}{y}\PYG{o}{/}\PYG{n+nb}{max}\PYG{p}{(}\PYG{n}{y}\PYG{p}{)}\PYG{p}{,} \PYG{n}{linewidth}\PYG{o}{=}\PYG{l+m+mi}{2}\PYG{p}{,} \PYG{n}{color}\PYG{o}{=}\PYG{l+s}{\PYGZsq{}}\PYG{l+s}{r}\PYG{l+s}{\PYGZsq{}}\PYG{p}{)}
\PYG{g+gp}{\PYGZgt{}\PYGZgt{}\PYGZgt{} }\PYG{n}{plt}\PYG{o}{.}\PYG{n}{show}\PYG{p}{(}\PYG{p}{)}
\end{Verbatim}

\end{fulllineitems}



\chapter{acsSCCanalysis Module}
\label{acsSCCanalysis::doc}\label{acsSCCanalysis:acssccanalysis-module}\label{acsSCCanalysis:module-acsSCCanalysis}\index{acsSCCanalysis (module)}
script to analyze the emerging strongly connected components.
\index{beta() (in module acsSCCanalysis)}

\begin{fulllineitems}
\phantomsection\label{acsSCCanalysis:acsSCCanalysis.beta}\pysiglinewithargsret{\code{acsSCCanalysis.}\bfcode{beta}}{\emph{a}, \emph{b}, \emph{size=None}}{}
The Beta distribution over \code{{[}0, 1{]}}.

The Beta distribution is a special case of the Dirichlet distribution,
and is related to the Gamma distribution.  It has the probability
distribution function
\begin{gather}
\begin{split}f(x; a,b) = \frac{1}{B(\alpha, \beta)} x^{\alpha - 1}
(1 - x)^{\beta - 1},\end{split}\notag
\end{gather}
where the normalisation, B, is the beta function,
\begin{gather}
\begin{split}B(\alpha, \beta) = \int_0^1 t^{\alpha - 1}
(1 - t)^{\beta - 1} dt.\end{split}\notag
\end{gather}
It is often seen in Bayesian inference and order statistics.
\begin{description}
\item[{a}] \leavevmode{[}float{]}
Alpha, non-negative.

\item[{b}] \leavevmode{[}float{]}
Beta, non-negative.

\item[{size}] \leavevmode{[}tuple of ints, optional{]}
The number of samples to draw.  The output is packed according to
the size given.

\end{description}
\begin{description}
\item[{out}] \leavevmode{[}ndarray{]}
Array of the given shape, containing values drawn from a
Beta distribution.

\end{description}

\end{fulllineitems}

\index{binomial() (in module acsSCCanalysis)}

\begin{fulllineitems}
\phantomsection\label{acsSCCanalysis:acsSCCanalysis.binomial}\pysiglinewithargsret{\code{acsSCCanalysis.}\bfcode{binomial}}{\emph{n}, \emph{p}, \emph{size=None}}{}
Draw samples from a binomial distribution.

Samples are drawn from a Binomial distribution with specified
parameters, n trials and p probability of success where
n an integer \textgreater{}= 0 and p is in the interval {[}0,1{]}. (n may be
input as a float, but it is truncated to an integer in use)
\begin{description}
\item[{n}] \leavevmode{[}float (but truncated to an integer){]}
parameter, \textgreater{}= 0.

\item[{p}] \leavevmode{[}float{]}
parameter, \textgreater{}= 0 and \textless{}=1.

\item[{size}] \leavevmode{[}\{tuple, int\}{]}
Output shape.  If the given shape is, e.g., \code{(m, n, k)}, then
\code{m * n * k} samples are drawn.

\end{description}
\begin{description}
\item[{samples}] \leavevmode{[}\{ndarray, scalar\}{]}
where the values are all integers in  {[}0, n{]}.

\end{description}
\begin{description}
\item[{scipy.stats.distributions.binom}] \leavevmode{[}probability density function,{]}
distribution or cumulative density function, etc.

\end{description}

The probability density for the Binomial distribution is
\begin{gather}
\begin{split}P(N) = \binom{n}{N}p^N(1-p)^{n-N},\end{split}\notag
\end{gather}
where \(n\) is the number of trials, \(p\) is the probability
of success, and \(N\) is the number of successes.

When estimating the standard error of a proportion in a population by
using a random sample, the normal distribution works well unless the
product p*n \textless{}=5, where p = population proportion estimate, and n =
number of samples, in which case the binomial distribution is used
instead. For example, a sample of 15 people shows 4 who are left
handed, and 11 who are right handed. Then p = 4/15 = 27\%. 0.27*15 = 4,
so the binomial distribution should be used in this case.

Draw samples from the distribution:

\begin{Verbatim}[commandchars=\\\{\}]
\PYG{g+gp}{\PYGZgt{}\PYGZgt{}\PYGZgt{} }\PYG{n}{n}\PYG{p}{,} \PYG{n}{p} \PYG{o}{=} \PYG{l+m+mi}{10}\PYG{p}{,} \PYG{o}{.}\PYG{l+m+mi}{5} \PYG{c}{\PYGZsh{} number of trials, probability of each trial}
\PYG{g+gp}{\PYGZgt{}\PYGZgt{}\PYGZgt{} }\PYG{n}{s} \PYG{o}{=} \PYG{n}{np}\PYG{o}{.}\PYG{n}{random}\PYG{o}{.}\PYG{n}{binomial}\PYG{p}{(}\PYG{n}{n}\PYG{p}{,} \PYG{n}{p}\PYG{p}{,} \PYG{l+m+mi}{1000}\PYG{p}{)}
\PYG{g+go}{\PYGZsh{} result of flipping a coin 10 times, tested 1000 times.}
\end{Verbatim}

A real world example. A company drills 9 wild-cat oil exploration
wells, each with an estimated probability of success of 0.1. All nine
wells fail. What is the probability of that happening?

Let's do 20,000 trials of the model, and count the number that
generate zero positive results.

\begin{Verbatim}[commandchars=\\\{\}]
\PYG{g+gp}{\PYGZgt{}\PYGZgt{}\PYGZgt{} }\PYG{n+nb}{sum}\PYG{p}{(}\PYG{n}{np}\PYG{o}{.}\PYG{n}{random}\PYG{o}{.}\PYG{n}{binomial}\PYG{p}{(}\PYG{l+m+mi}{9}\PYG{p}{,}\PYG{l+m+mf}{0.1}\PYG{p}{,}\PYG{l+m+mi}{20000}\PYG{p}{)}\PYG{o}{==}\PYG{l+m+mi}{0}\PYG{p}{)}\PYG{o}{/}\PYG{l+m+mf}{20000.}
\PYG{g+go}{answer = 0.38885, or 38\PYGZpc{}.}
\end{Verbatim}

\end{fulllineitems}

\index{chisquare() (in module acsSCCanalysis)}

\begin{fulllineitems}
\phantomsection\label{acsSCCanalysis:acsSCCanalysis.chisquare}\pysiglinewithargsret{\code{acsSCCanalysis.}\bfcode{chisquare}}{\emph{df}, \emph{size=None}}{}
Draw samples from a chi-square distribution.

When \emph{df} independent random variables, each with standard normal
distributions (mean 0, variance 1), are squared and summed, the
resulting distribution is chi-square (see Notes).  This distribution
is often used in hypothesis testing.
\begin{description}
\item[{df}] \leavevmode{[}int{]}
Number of degrees of freedom.

\item[{size}] \leavevmode{[}tuple of ints, int, optional{]}
Size of the returned array.  By default, a scalar is
returned.

\end{description}
\begin{description}
\item[{output}] \leavevmode{[}ndarray{]}
Samples drawn from the distribution, packed in a \emph{size}-shaped
array.

\end{description}
\begin{description}
\item[{ValueError}] \leavevmode
When \emph{df} \textless{}= 0 or when an inappropriate \emph{size} (e.g. \code{size=-1})
is given.

\end{description}

The variable obtained by summing the squares of \emph{df} independent,
standard normally distributed random variables:
\begin{gather}
\begin{split}Q = \sum_{i=0}^{\mathtt{df}} X^2_i\end{split}\notag
\end{gather}
is chi-square distributed, denoted
\begin{gather}
\begin{split}Q \sim \chi^2_k.\end{split}\notag
\end{gather}
The probability density function of the chi-squared distribution is
\begin{gather}
\begin{split}p(x) = \frac{(1/2)^{k/2}}{\Gamma(k/2)}
x^{k/2 - 1} e^{-x/2},\end{split}\notag
\end{gather}
where \(\Gamma\) is the gamma function,
\begin{gather}
\begin{split}\Gamma(x) = \int_0^{-\infty} t^{x - 1} e^{-t} dt.\end{split}\notag
\end{gather}
\href{http://www.itl.nist.gov/div898/handbook/eda/section3/eda3666.htm}{NIST/SEMATECH e-Handbook of Statistical Methods}

\begin{Verbatim}[commandchars=\\\{\}]
\PYG{g+gp}{\PYGZgt{}\PYGZgt{}\PYGZgt{} }\PYG{n}{np}\PYG{o}{.}\PYG{n}{random}\PYG{o}{.}\PYG{n}{chisquare}\PYG{p}{(}\PYG{l+m+mi}{2}\PYG{p}{,}\PYG{l+m+mi}{4}\PYG{p}{)}
\PYG{g+go}{array([ 1.89920014,  9.00867716,  3.13710533,  5.62318272])}
\end{Verbatim}

\end{fulllineitems}

\index{exponential() (in module acsSCCanalysis)}

\begin{fulllineitems}
\phantomsection\label{acsSCCanalysis:acsSCCanalysis.exponential}\pysiglinewithargsret{\code{acsSCCanalysis.}\bfcode{exponential}}{\emph{scale=1.0}, \emph{size=None}}{}
Exponential distribution.

Its probability density function is
\begin{gather}
\begin{split}f(x; \frac{1}{\beta}) = \frac{1}{\beta} \exp(-\frac{x}{\beta}),\end{split}\notag
\end{gather}
for \code{x \textgreater{} 0} and 0 elsewhere. \(\beta\) is the scale parameter,
which is the inverse of the rate parameter \(\lambda = 1/\beta\).
The rate parameter is an alternative, widely used parameterization
of the exponential distribution {\color{red}\bfseries{}{[}3{]}\_}.

The exponential distribution is a continuous analogue of the
geometric distribution.  It describes many common situations, such as
the size of raindrops measured over many rainstorms {\color{red}\bfseries{}{[}1{]}\_}, or the time
between page requests to Wikipedia {\color{red}\bfseries{}{[}2{]}\_}.
\begin{description}
\item[{scale}] \leavevmode{[}float{]}
The scale parameter, \(\beta = 1/\lambda\).

\item[{size}] \leavevmode{[}tuple of ints{]}
Number of samples to draw.  The output is shaped
according to \emph{size}.

\end{description}

\end{fulllineitems}

\index{f() (in module acsSCCanalysis)}

\begin{fulllineitems}
\phantomsection\label{acsSCCanalysis:acsSCCanalysis.f}\pysiglinewithargsret{\code{acsSCCanalysis.}\bfcode{f}}{\emph{dfnum}, \emph{dfden}, \emph{size=None}}{}
Draw samples from a F distribution.

Samples are drawn from an F distribution with specified parameters,
\emph{dfnum} (degrees of freedom in numerator) and \emph{dfden} (degrees of freedom
in denominator), where both parameters should be greater than zero.

The random variate of the F distribution (also known as the
Fisher distribution) is a continuous probability distribution
that arises in ANOVA tests, and is the ratio of two chi-square
variates.
\begin{description}
\item[{dfnum}] \leavevmode{[}float{]}
Degrees of freedom in numerator. Should be greater than zero.

\item[{dfden}] \leavevmode{[}float{]}
Degrees of freedom in denominator. Should be greater than zero.

\item[{size}] \leavevmode{[}\{tuple, int\}, optional{]}
Output shape.  If the given shape is, e.g., \code{(m, n, k)},
then \code{m * n * k} samples are drawn. By default only one sample
is returned.

\end{description}
\begin{description}
\item[{samples}] \leavevmode{[}\{ndarray, scalar\}{]}
Samples from the Fisher distribution.

\end{description}
\begin{description}
\item[{scipy.stats.distributions.f}] \leavevmode{[}probability density function,{]}
distribution or cumulative density function, etc.

\end{description}

The F statistic is used to compare in-group variances to between-group
variances. Calculating the distribution depends on the sampling, and
so it is a function of the respective degrees of freedom in the
problem.  The variable \emph{dfnum} is the number of samples minus one, the
between-groups degrees of freedom, while \emph{dfden} is the within-groups
degrees of freedom, the sum of the number of samples in each group
minus the number of groups.

An example from Glantz{[}1{]}, pp 47-40.
Two groups, children of diabetics (25 people) and children from people
without diabetes (25 controls). Fasting blood glucose was measured,
case group had a mean value of 86.1, controls had a mean value of
82.2. Standard deviations were 2.09 and 2.49 respectively. Are these
data consistent with the null hypothesis that the parents diabetic
status does not affect their children's blood glucose levels?
Calculating the F statistic from the data gives a value of 36.01.

Draw samples from the distribution:

\begin{Verbatim}[commandchars=\\\{\}]
\PYG{g+gp}{\PYGZgt{}\PYGZgt{}\PYGZgt{} }\PYG{n}{dfnum} \PYG{o}{=} \PYG{l+m+mf}{1.} \PYG{c}{\PYGZsh{} between group degrees of freedom}
\PYG{g+gp}{\PYGZgt{}\PYGZgt{}\PYGZgt{} }\PYG{n}{dfden} \PYG{o}{=} \PYG{l+m+mf}{48.} \PYG{c}{\PYGZsh{} within groups degrees of freedom}
\PYG{g+gp}{\PYGZgt{}\PYGZgt{}\PYGZgt{} }\PYG{n}{s} \PYG{o}{=} \PYG{n}{np}\PYG{o}{.}\PYG{n}{random}\PYG{o}{.}\PYG{n}{f}\PYG{p}{(}\PYG{n}{dfnum}\PYG{p}{,} \PYG{n}{dfden}\PYG{p}{,} \PYG{l+m+mi}{1000}\PYG{p}{)}
\end{Verbatim}

The lower bound for the top 1\% of the samples is :

\begin{Verbatim}[commandchars=\\\{\}]
\PYG{g+gp}{\PYGZgt{}\PYGZgt{}\PYGZgt{} }\PYG{n}{sort}\PYG{p}{(}\PYG{n}{s}\PYG{p}{)}\PYG{p}{[}\PYG{o}{\PYGZhy{}}\PYG{l+m+mi}{10}\PYG{p}{]}
\PYG{g+go}{7.61988120985}
\end{Verbatim}

So there is about a 1\% chance that the F statistic will exceed 7.62,
the measured value is 36, so the null hypothesis is rejected at the 1\%
level.

\end{fulllineitems}

\index{gamma() (in module acsSCCanalysis)}

\begin{fulllineitems}
\phantomsection\label{acsSCCanalysis:acsSCCanalysis.gamma}\pysiglinewithargsret{\code{acsSCCanalysis.}\bfcode{gamma}}{\emph{shape}, \emph{scale=1.0}, \emph{size=None}}{}
Draw samples from a Gamma distribution.

Samples are drawn from a Gamma distribution with specified parameters,
\emph{shape} (sometimes designated ``k'') and \emph{scale} (sometimes designated
``theta''), where both parameters are \textgreater{} 0.
\begin{description}
\item[{shape}] \leavevmode{[}scalar \textgreater{} 0{]}
The shape of the gamma distribution.

\item[{scale}] \leavevmode{[}scalar \textgreater{} 0, optional{]}
The scale of the gamma distribution.  Default is equal to 1.

\item[{size}] \leavevmode{[}shape\_tuple, optional{]}
Output shape.  If the given shape is, e.g., \code{(m, n, k)}, then
\code{m * n * k} samples are drawn.

\end{description}
\begin{description}
\item[{out}] \leavevmode{[}ndarray, float{]}
Returns one sample unless \emph{size} parameter is specified.

\end{description}
\begin{description}
\item[{scipy.stats.distributions.gamma}] \leavevmode{[}probability density function,{]}
distribution or cumulative density function, etc.

\end{description}

The probability density for the Gamma distribution is
\begin{gather}
\begin{split}p(x) = x^{k-1}\frac{e^{-x/\theta}}{\theta^k\Gamma(k)},\end{split}\notag
\end{gather}
where \(k\) is the shape and \(\theta\) the scale,
and \(\Gamma\) is the Gamma function.

The Gamma distribution is often used to model the times to failure of
electronic components, and arises naturally in processes for which the
waiting times between Poisson distributed events are relevant.

Draw samples from the distribution:

\begin{Verbatim}[commandchars=\\\{\}]
\PYG{g+gp}{\PYGZgt{}\PYGZgt{}\PYGZgt{} }\PYG{n}{shape}\PYG{p}{,} \PYG{n}{scale} \PYG{o}{=} \PYG{l+m+mf}{2.}\PYG{p}{,} \PYG{l+m+mf}{2.} \PYG{c}{\PYGZsh{} mean and dispersion}
\PYG{g+gp}{\PYGZgt{}\PYGZgt{}\PYGZgt{} }\PYG{n}{s} \PYG{o}{=} \PYG{n}{np}\PYG{o}{.}\PYG{n}{random}\PYG{o}{.}\PYG{n}{gamma}\PYG{p}{(}\PYG{n}{shape}\PYG{p}{,} \PYG{n}{scale}\PYG{p}{,} \PYG{l+m+mi}{1000}\PYG{p}{)}
\end{Verbatim}

Display the histogram of the samples, along with
the probability density function:

\begin{Verbatim}[commandchars=\\\{\}]
\PYG{g+gp}{\PYGZgt{}\PYGZgt{}\PYGZgt{} }\PYG{k+kn}{import} \PYG{n+nn}{matplotlib.pyplot} \PYG{k+kn}{as} \PYG{n+nn}{plt}
\PYG{g+gp}{\PYGZgt{}\PYGZgt{}\PYGZgt{} }\PYG{k+kn}{import} \PYG{n+nn}{scipy.special} \PYG{k+kn}{as} \PYG{n+nn}{sps}
\PYG{g+gp}{\PYGZgt{}\PYGZgt{}\PYGZgt{} }\PYG{n}{count}\PYG{p}{,} \PYG{n}{bins}\PYG{p}{,} \PYG{n}{ignored} \PYG{o}{=} \PYG{n}{plt}\PYG{o}{.}\PYG{n}{hist}\PYG{p}{(}\PYG{n}{s}\PYG{p}{,} \PYG{l+m+mi}{50}\PYG{p}{,} \PYG{n}{normed}\PYG{o}{=}\PYG{n+nb+bp}{True}\PYG{p}{)}
\PYG{g+gp}{\PYGZgt{}\PYGZgt{}\PYGZgt{} }\PYG{n}{y} \PYG{o}{=} \PYG{n}{bins}\PYG{o}{*}\PYG{o}{*}\PYG{p}{(}\PYG{n}{shape}\PYG{o}{\PYGZhy{}}\PYG{l+m+mi}{1}\PYG{p}{)}\PYG{o}{*}\PYG{p}{(}\PYG{n}{np}\PYG{o}{.}\PYG{n}{exp}\PYG{p}{(}\PYG{o}{\PYGZhy{}}\PYG{n}{bins}\PYG{o}{/}\PYG{n}{scale}\PYG{p}{)} \PYG{o}{/}
\PYG{g+gp}{... }                     \PYG{p}{(}\PYG{n}{sps}\PYG{o}{.}\PYG{n}{gamma}\PYG{p}{(}\PYG{n}{shape}\PYG{p}{)}\PYG{o}{*}\PYG{n}{scale}\PYG{o}{*}\PYG{o}{*}\PYG{n}{shape}\PYG{p}{)}\PYG{p}{)}
\PYG{g+gp}{\PYGZgt{}\PYGZgt{}\PYGZgt{} }\PYG{n}{plt}\PYG{o}{.}\PYG{n}{plot}\PYG{p}{(}\PYG{n}{bins}\PYG{p}{,} \PYG{n}{y}\PYG{p}{,} \PYG{n}{linewidth}\PYG{o}{=}\PYG{l+m+mi}{2}\PYG{p}{,} \PYG{n}{color}\PYG{o}{=}\PYG{l+s}{\PYGZsq{}}\PYG{l+s}{r}\PYG{l+s}{\PYGZsq{}}\PYG{p}{)}
\PYG{g+gp}{\PYGZgt{}\PYGZgt{}\PYGZgt{} }\PYG{n}{plt}\PYG{o}{.}\PYG{n}{show}\PYG{p}{(}\PYG{p}{)}
\end{Verbatim}

\end{fulllineitems}

\index{geometric() (in module acsSCCanalysis)}

\begin{fulllineitems}
\phantomsection\label{acsSCCanalysis:acsSCCanalysis.geometric}\pysiglinewithargsret{\code{acsSCCanalysis.}\bfcode{geometric}}{\emph{p}, \emph{size=None}}{}
Draw samples from the geometric distribution.

Bernoulli trials are experiments with one of two outcomes:
success or failure (an example of such an experiment is flipping
a coin).  The geometric distribution models the number of trials
that must be run in order to achieve success.  It is therefore
supported on the positive integers, \code{k = 1, 2, ...}.

The probability mass function of the geometric distribution is
\begin{gather}
\begin{split}f(k) = (1 - p)^{k - 1} p\end{split}\notag
\end{gather}
where \emph{p} is the probability of success of an individual trial.
\begin{description}
\item[{p}] \leavevmode{[}float{]}
The probability of success of an individual trial.

\item[{size}] \leavevmode{[}tuple of ints{]}
Number of values to draw from the distribution.  The output
is shaped according to \emph{size}.

\end{description}
\begin{description}
\item[{out}] \leavevmode{[}ndarray{]}
Samples from the geometric distribution, shaped according to
\emph{size}.

\end{description}

Draw ten thousand values from the geometric distribution,
with the probability of an individual success equal to 0.35:

\begin{Verbatim}[commandchars=\\\{\}]
\PYG{g+gp}{\PYGZgt{}\PYGZgt{}\PYGZgt{} }\PYG{n}{z} \PYG{o}{=} \PYG{n}{np}\PYG{o}{.}\PYG{n}{random}\PYG{o}{.}\PYG{n}{geometric}\PYG{p}{(}\PYG{n}{p}\PYG{o}{=}\PYG{l+m+mf}{0.35}\PYG{p}{,} \PYG{n}{size}\PYG{o}{=}\PYG{l+m+mi}{10000}\PYG{p}{)}
\end{Verbatim}

How many trials succeeded after a single run?

\begin{Verbatim}[commandchars=\\\{\}]
\PYG{g+gp}{\PYGZgt{}\PYGZgt{}\PYGZgt{} }\PYG{p}{(}\PYG{n}{z} \PYG{o}{==} \PYG{l+m+mi}{1}\PYG{p}{)}\PYG{o}{.}\PYG{n}{sum}\PYG{p}{(}\PYG{p}{)} \PYG{o}{/} \PYG{l+m+mf}{10000.}
\PYG{g+go}{0.34889999999999999 \PYGZsh{}random}
\end{Verbatim}

\end{fulllineitems}

\index{get\_state() (in module acsSCCanalysis)}

\begin{fulllineitems}
\phantomsection\label{acsSCCanalysis:acsSCCanalysis.get_state}\pysiglinewithargsret{\code{acsSCCanalysis.}\bfcode{get\_state}}{}{}
Return a tuple representing the internal state of the generator.

For more details, see \emph{set\_state}.
\begin{description}
\item[{out}] \leavevmode{[}tuple(str, ndarray of 624 uints, int, int, float){]}
The returned tuple has the following items:
\begin{enumerate}
\item {} 
the string `MT19937'.

\item {} 
a 1-D array of 624 unsigned integer keys.

\item {} 
an integer \code{pos}.

\item {} 
an integer \code{has\_gauss}.

\item {} 
a float \code{cached\_gaussian}.

\end{enumerate}

\end{description}

set\_state

\emph{set\_state} and \emph{get\_state} are not needed to work with any of the
random distributions in NumPy. If the internal state is manually altered,
the user should know exactly what he/she is doing.

\end{fulllineitems}

\index{gumbel() (in module acsSCCanalysis)}

\begin{fulllineitems}
\phantomsection\label{acsSCCanalysis:acsSCCanalysis.gumbel}\pysiglinewithargsret{\code{acsSCCanalysis.}\bfcode{gumbel}}{\emph{loc=0.0}, \emph{scale=1.0}, \emph{size=None}}{}
Gumbel distribution.

Draw samples from a Gumbel distribution with specified location and scale.
For more information on the Gumbel distribution, see Notes and References
below.
\begin{description}
\item[{loc}] \leavevmode{[}float{]}
The location of the mode of the distribution.

\item[{scale}] \leavevmode{[}float{]}
The scale parameter of the distribution.

\item[{size}] \leavevmode{[}tuple of ints{]}
Output shape.  If the given shape is, e.g., \code{(m, n, k)}, then
\code{m * n * k} samples are drawn.

\end{description}
\begin{description}
\item[{out}] \leavevmode{[}ndarray{]}
The samples

\end{description}

scipy.stats.gumbel\_l
scipy.stats.gumbel\_r
scipy.stats.genextreme
\begin{quote}

probability density function, distribution, or cumulative density
function, etc. for each of the above
\end{quote}

weibull

The Gumbel (or Smallest Extreme Value (SEV) or the Smallest Extreme Value
Type I) distribution is one of a class of Generalized Extreme Value (GEV)
distributions used in modeling extreme value problems.  The Gumbel is a
special case of the Extreme Value Type I distribution for maximums from
distributions with ``exponential-like'' tails.

The probability density for the Gumbel distribution is
\begin{gather}
\begin{split}p(x) = \frac{e^{-(x - \mu)/ \beta}}{\beta} e^{ -e^{-(x - \mu)/
\beta}},\end{split}\notag
\end{gather}
where \(\mu\) is the mode, a location parameter, and \(\beta\) is
the scale parameter.

The Gumbel (named for German mathematician Emil Julius Gumbel) was used
very early in the hydrology literature, for modeling the occurrence of
flood events. It is also used for modeling maximum wind speed and rainfall
rates.  It is a ``fat-tailed'' distribution - the probability of an event in
the tail of the distribution is larger than if one used a Gaussian, hence
the surprisingly frequent occurrence of 100-year floods. Floods were
initially modeled as a Gaussian process, which underestimated the frequency
of extreme events.

It is one of a class of extreme value distributions, the Generalized
Extreme Value (GEV) distributions, which also includes the Weibull and
Frechet.

The function has a mean of \(\mu + 0.57721\beta\) and a variance of
\(\frac{\pi^2}{6}\beta^2\).

Gumbel, E. J., \emph{Statistics of Extremes}, New York: Columbia University
Press, 1958.

Reiss, R.-D. and Thomas, M., \emph{Statistical Analysis of Extreme Values from
Insurance, Finance, Hydrology and Other Fields}, Basel: Birkhauser Verlag,
2001.

Draw samples from the distribution:

\begin{Verbatim}[commandchars=\\\{\}]
\PYG{g+gp}{\PYGZgt{}\PYGZgt{}\PYGZgt{} }\PYG{n}{mu}\PYG{p}{,} \PYG{n}{beta} \PYG{o}{=} \PYG{l+m+mi}{0}\PYG{p}{,} \PYG{l+m+mf}{0.1} \PYG{c}{\PYGZsh{} location and scale}
\PYG{g+gp}{\PYGZgt{}\PYGZgt{}\PYGZgt{} }\PYG{n}{s} \PYG{o}{=} \PYG{n}{np}\PYG{o}{.}\PYG{n}{random}\PYG{o}{.}\PYG{n}{gumbel}\PYG{p}{(}\PYG{n}{mu}\PYG{p}{,} \PYG{n}{beta}\PYG{p}{,} \PYG{l+m+mi}{1000}\PYG{p}{)}
\end{Verbatim}

Display the histogram of the samples, along with
the probability density function:

\begin{Verbatim}[commandchars=\\\{\}]
\PYG{g+gp}{\PYGZgt{}\PYGZgt{}\PYGZgt{} }\PYG{k+kn}{import} \PYG{n+nn}{matplotlib.pyplot} \PYG{k+kn}{as} \PYG{n+nn}{plt}
\PYG{g+gp}{\PYGZgt{}\PYGZgt{}\PYGZgt{} }\PYG{n}{count}\PYG{p}{,} \PYG{n}{bins}\PYG{p}{,} \PYG{n}{ignored} \PYG{o}{=} \PYG{n}{plt}\PYG{o}{.}\PYG{n}{hist}\PYG{p}{(}\PYG{n}{s}\PYG{p}{,} \PYG{l+m+mi}{30}\PYG{p}{,} \PYG{n}{normed}\PYG{o}{=}\PYG{n+nb+bp}{True}\PYG{p}{)}
\PYG{g+gp}{\PYGZgt{}\PYGZgt{}\PYGZgt{} }\PYG{n}{plt}\PYG{o}{.}\PYG{n}{plot}\PYG{p}{(}\PYG{n}{bins}\PYG{p}{,} \PYG{p}{(}\PYG{l+m+mi}{1}\PYG{o}{/}\PYG{n}{beta}\PYG{p}{)}\PYG{o}{*}\PYG{n}{np}\PYG{o}{.}\PYG{n}{exp}\PYG{p}{(}\PYG{o}{\PYGZhy{}}\PYG{p}{(}\PYG{n}{bins} \PYG{o}{\PYGZhy{}} \PYG{n}{mu}\PYG{p}{)}\PYG{o}{/}\PYG{n}{beta}\PYG{p}{)}
\PYG{g+gp}{... }         \PYG{o}{*} \PYG{n}{np}\PYG{o}{.}\PYG{n}{exp}\PYG{p}{(} \PYG{o}{\PYGZhy{}}\PYG{n}{np}\PYG{o}{.}\PYG{n}{exp}\PYG{p}{(} \PYG{o}{\PYGZhy{}}\PYG{p}{(}\PYG{n}{bins} \PYG{o}{\PYGZhy{}} \PYG{n}{mu}\PYG{p}{)} \PYG{o}{/}\PYG{n}{beta}\PYG{p}{)} \PYG{p}{)}\PYG{p}{,}
\PYG{g+gp}{... }         \PYG{n}{linewidth}\PYG{o}{=}\PYG{l+m+mi}{2}\PYG{p}{,} \PYG{n}{color}\PYG{o}{=}\PYG{l+s}{\PYGZsq{}}\PYG{l+s}{r}\PYG{l+s}{\PYGZsq{}}\PYG{p}{)}
\PYG{g+gp}{\PYGZgt{}\PYGZgt{}\PYGZgt{} }\PYG{n}{plt}\PYG{o}{.}\PYG{n}{show}\PYG{p}{(}\PYG{p}{)}
\end{Verbatim}

Show how an extreme value distribution can arise from a Gaussian process
and compare to a Gaussian:

\begin{Verbatim}[commandchars=\\\{\}]
\PYG{g+gp}{\PYGZgt{}\PYGZgt{}\PYGZgt{} }\PYG{n}{means} \PYG{o}{=} \PYG{p}{[}\PYG{p}{]}
\PYG{g+gp}{\PYGZgt{}\PYGZgt{}\PYGZgt{} }\PYG{n}{maxima} \PYG{o}{=} \PYG{p}{[}\PYG{p}{]}
\PYG{g+gp}{\PYGZgt{}\PYGZgt{}\PYGZgt{} }\PYG{k}{for} \PYG{n}{i} \PYG{o+ow}{in} \PYG{n+nb}{range}\PYG{p}{(}\PYG{l+m+mi}{0}\PYG{p}{,}\PYG{l+m+mi}{1000}\PYG{p}{)} \PYG{p}{:}
\PYG{g+gp}{... }   \PYG{n}{a} \PYG{o}{=} \PYG{n}{np}\PYG{o}{.}\PYG{n}{random}\PYG{o}{.}\PYG{n}{normal}\PYG{p}{(}\PYG{n}{mu}\PYG{p}{,} \PYG{n}{beta}\PYG{p}{,} \PYG{l+m+mi}{1000}\PYG{p}{)}
\PYG{g+gp}{... }   \PYG{n}{means}\PYG{o}{.}\PYG{n}{append}\PYG{p}{(}\PYG{n}{a}\PYG{o}{.}\PYG{n}{mean}\PYG{p}{(}\PYG{p}{)}\PYG{p}{)}
\PYG{g+gp}{... }   \PYG{n}{maxima}\PYG{o}{.}\PYG{n}{append}\PYG{p}{(}\PYG{n}{a}\PYG{o}{.}\PYG{n}{max}\PYG{p}{(}\PYG{p}{)}\PYG{p}{)}
\PYG{g+gp}{\PYGZgt{}\PYGZgt{}\PYGZgt{} }\PYG{n}{count}\PYG{p}{,} \PYG{n}{bins}\PYG{p}{,} \PYG{n}{ignored} \PYG{o}{=} \PYG{n}{plt}\PYG{o}{.}\PYG{n}{hist}\PYG{p}{(}\PYG{n}{maxima}\PYG{p}{,} \PYG{l+m+mi}{30}\PYG{p}{,} \PYG{n}{normed}\PYG{o}{=}\PYG{n+nb+bp}{True}\PYG{p}{)}
\PYG{g+gp}{\PYGZgt{}\PYGZgt{}\PYGZgt{} }\PYG{n}{beta} \PYG{o}{=} \PYG{n}{np}\PYG{o}{.}\PYG{n}{std}\PYG{p}{(}\PYG{n}{maxima}\PYG{p}{)}\PYG{o}{*}\PYG{n}{np}\PYG{o}{.}\PYG{n}{pi}\PYG{o}{/}\PYG{n}{np}\PYG{o}{.}\PYG{n}{sqrt}\PYG{p}{(}\PYG{l+m+mi}{6}\PYG{p}{)}
\PYG{g+gp}{\PYGZgt{}\PYGZgt{}\PYGZgt{} }\PYG{n}{mu} \PYG{o}{=} \PYG{n}{np}\PYG{o}{.}\PYG{n}{mean}\PYG{p}{(}\PYG{n}{maxima}\PYG{p}{)} \PYG{o}{\PYGZhy{}} \PYG{l+m+mf}{0.57721}\PYG{o}{*}\PYG{n}{beta}
\PYG{g+gp}{\PYGZgt{}\PYGZgt{}\PYGZgt{} }\PYG{n}{plt}\PYG{o}{.}\PYG{n}{plot}\PYG{p}{(}\PYG{n}{bins}\PYG{p}{,} \PYG{p}{(}\PYG{l+m+mi}{1}\PYG{o}{/}\PYG{n}{beta}\PYG{p}{)}\PYG{o}{*}\PYG{n}{np}\PYG{o}{.}\PYG{n}{exp}\PYG{p}{(}\PYG{o}{\PYGZhy{}}\PYG{p}{(}\PYG{n}{bins} \PYG{o}{\PYGZhy{}} \PYG{n}{mu}\PYG{p}{)}\PYG{o}{/}\PYG{n}{beta}\PYG{p}{)}
\PYG{g+gp}{... }         \PYG{o}{*} \PYG{n}{np}\PYG{o}{.}\PYG{n}{exp}\PYG{p}{(}\PYG{o}{\PYGZhy{}}\PYG{n}{np}\PYG{o}{.}\PYG{n}{exp}\PYG{p}{(}\PYG{o}{\PYGZhy{}}\PYG{p}{(}\PYG{n}{bins} \PYG{o}{\PYGZhy{}} \PYG{n}{mu}\PYG{p}{)}\PYG{o}{/}\PYG{n}{beta}\PYG{p}{)}\PYG{p}{)}\PYG{p}{,}
\PYG{g+gp}{... }         \PYG{n}{linewidth}\PYG{o}{=}\PYG{l+m+mi}{2}\PYG{p}{,} \PYG{n}{color}\PYG{o}{=}\PYG{l+s}{\PYGZsq{}}\PYG{l+s}{r}\PYG{l+s}{\PYGZsq{}}\PYG{p}{)}
\PYG{g+gp}{\PYGZgt{}\PYGZgt{}\PYGZgt{} }\PYG{n}{plt}\PYG{o}{.}\PYG{n}{plot}\PYG{p}{(}\PYG{n}{bins}\PYG{p}{,} \PYG{l+m+mi}{1}\PYG{o}{/}\PYG{p}{(}\PYG{n}{beta} \PYG{o}{*} \PYG{n}{np}\PYG{o}{.}\PYG{n}{sqrt}\PYG{p}{(}\PYG{l+m+mi}{2} \PYG{o}{*} \PYG{n}{np}\PYG{o}{.}\PYG{n}{pi}\PYG{p}{)}\PYG{p}{)}
\PYG{g+gp}{... }         \PYG{o}{*} \PYG{n}{np}\PYG{o}{.}\PYG{n}{exp}\PYG{p}{(}\PYG{o}{\PYGZhy{}}\PYG{p}{(}\PYG{n}{bins} \PYG{o}{\PYGZhy{}} \PYG{n}{mu}\PYG{p}{)}\PYG{o}{*}\PYG{o}{*}\PYG{l+m+mi}{2} \PYG{o}{/} \PYG{p}{(}\PYG{l+m+mi}{2} \PYG{o}{*} \PYG{n}{beta}\PYG{o}{*}\PYG{o}{*}\PYG{l+m+mi}{2}\PYG{p}{)}\PYG{p}{)}\PYG{p}{,}
\PYG{g+gp}{... }         \PYG{n}{linewidth}\PYG{o}{=}\PYG{l+m+mi}{2}\PYG{p}{,} \PYG{n}{color}\PYG{o}{=}\PYG{l+s}{\PYGZsq{}}\PYG{l+s}{g}\PYG{l+s}{\PYGZsq{}}\PYG{p}{)}
\PYG{g+gp}{\PYGZgt{}\PYGZgt{}\PYGZgt{} }\PYG{n}{plt}\PYG{o}{.}\PYG{n}{show}\PYG{p}{(}\PYG{p}{)}
\end{Verbatim}

\end{fulllineitems}

\index{hypergeometric() (in module acsSCCanalysis)}

\begin{fulllineitems}
\phantomsection\label{acsSCCanalysis:acsSCCanalysis.hypergeometric}\pysiglinewithargsret{\code{acsSCCanalysis.}\bfcode{hypergeometric}}{\emph{ngood}, \emph{nbad}, \emph{nsample}, \emph{size=None}}{}
Draw samples from a Hypergeometric distribution.

Samples are drawn from a Hypergeometric distribution with specified
parameters, ngood (ways to make a good selection), nbad (ways to make
a bad selection), and nsample = number of items sampled, which is less
than or equal to the sum ngood + nbad.
\begin{description}
\item[{ngood}] \leavevmode{[}int or array\_like{]}
Number of ways to make a good selection.  Must be nonnegative.

\item[{nbad}] \leavevmode{[}int or array\_like{]}
Number of ways to make a bad selection.  Must be nonnegative.

\item[{nsample}] \leavevmode{[}int or array\_like{]}
Number of items sampled.  Must be at least 1 and at most
\code{ngood + nbad}.

\item[{size}] \leavevmode{[}int or tuple of int{]}
Output shape.  If the given shape is, e.g., \code{(m, n, k)}, then
\code{m * n * k} samples are drawn.

\end{description}
\begin{description}
\item[{samples}] \leavevmode{[}ndarray or scalar{]}
The values are all integers in  {[}0, n{]}.

\end{description}
\begin{description}
\item[{scipy.stats.distributions.hypergeom}] \leavevmode{[}probability density function,{]}
distribution or cumulative density function, etc.

\end{description}

The probability density for the Hypergeometric distribution is
\begin{gather}
\begin{split}P(x) = \frac{\binom{m}{n}\binom{N-m}{n-x}}{\binom{N}{n}},\end{split}\notag
\end{gather}
where \(0 \le x \le m\) and \(n+m-N \le x \le n\)

for P(x) the probability of x successes, n = ngood, m = nbad, and
N = number of samples.

Consider an urn with black and white marbles in it, ngood of them
black and nbad are white. If you draw nsample balls without
replacement, then the Hypergeometric distribution describes the
distribution of black balls in the drawn sample.

Note that this distribution is very similar to the Binomial
distribution, except that in this case, samples are drawn without
replacement, whereas in the Binomial case samples are drawn with
replacement (or the sample space is infinite). As the sample space
becomes large, this distribution approaches the Binomial.

Draw samples from the distribution:

\begin{Verbatim}[commandchars=\\\{\}]
\PYG{g+gp}{\PYGZgt{}\PYGZgt{}\PYGZgt{} }\PYG{n}{ngood}\PYG{p}{,} \PYG{n}{nbad}\PYG{p}{,} \PYG{n}{nsamp} \PYG{o}{=} \PYG{l+m+mi}{100}\PYG{p}{,} \PYG{l+m+mi}{2}\PYG{p}{,} \PYG{l+m+mi}{10}
\PYG{g+go}{\PYGZsh{} number of good, number of bad, and number of samples}
\PYG{g+gp}{\PYGZgt{}\PYGZgt{}\PYGZgt{} }\PYG{n}{s} \PYG{o}{=} \PYG{n}{np}\PYG{o}{.}\PYG{n}{random}\PYG{o}{.}\PYG{n}{hypergeometric}\PYG{p}{(}\PYG{n}{ngood}\PYG{p}{,} \PYG{n}{nbad}\PYG{p}{,} \PYG{n}{nsamp}\PYG{p}{,} \PYG{l+m+mi}{1000}\PYG{p}{)}
\PYG{g+gp}{\PYGZgt{}\PYGZgt{}\PYGZgt{} }\PYG{n}{hist}\PYG{p}{(}\PYG{n}{s}\PYG{p}{)}
\PYG{g+go}{\PYGZsh{}   note that it is very unlikely to grab both bad items}
\end{Verbatim}

Suppose you have an urn with 15 white and 15 black marbles.
If you pull 15 marbles at random, how likely is it that
12 or more of them are one color?

\begin{Verbatim}[commandchars=\\\{\}]
\PYG{g+gp}{\PYGZgt{}\PYGZgt{}\PYGZgt{} }\PYG{n}{s} \PYG{o}{=} \PYG{n}{np}\PYG{o}{.}\PYG{n}{random}\PYG{o}{.}\PYG{n}{hypergeometric}\PYG{p}{(}\PYG{l+m+mi}{15}\PYG{p}{,} \PYG{l+m+mi}{15}\PYG{p}{,} \PYG{l+m+mi}{15}\PYG{p}{,} \PYG{l+m+mi}{100000}\PYG{p}{)}
\PYG{g+gp}{\PYGZgt{}\PYGZgt{}\PYGZgt{} }\PYG{n+nb}{sum}\PYG{p}{(}\PYG{n}{s}\PYG{o}{\PYGZgt{}}\PYG{o}{=}\PYG{l+m+mi}{12}\PYG{p}{)}\PYG{o}{/}\PYG{l+m+mf}{100000.} \PYG{o}{+} \PYG{n+nb}{sum}\PYG{p}{(}\PYG{n}{s}\PYG{o}{\PYGZlt{}}\PYG{o}{=}\PYG{l+m+mi}{3}\PYG{p}{)}\PYG{o}{/}\PYG{l+m+mf}{100000.}
\PYG{g+go}{\PYGZsh{}   answer = 0.003 ... pretty unlikely!}
\end{Verbatim}

\end{fulllineitems}

\index{laplace() (in module acsSCCanalysis)}

\begin{fulllineitems}
\phantomsection\label{acsSCCanalysis:acsSCCanalysis.laplace}\pysiglinewithargsret{\code{acsSCCanalysis.}\bfcode{laplace}}{\emph{loc=0.0}, \emph{scale=1.0}, \emph{size=None}}{}
Draw samples from the Laplace or double exponential distribution with
specified location (or mean) and scale (decay).

The Laplace distribution is similar to the Gaussian/normal distribution,
but is sharper at the peak and has fatter tails. It represents the
difference between two independent, identically distributed exponential
random variables.
\begin{description}
\item[{loc}] \leavevmode{[}float{]}
The position, \(\mu\), of the distribution peak.

\item[{scale}] \leavevmode{[}float{]}
\(\lambda\), the exponential decay.

\end{description}

It has the probability density function
\begin{gather}
\begin{split}f(x; \mu, \lambda) = \frac{1}{2\lambda}
\exp\left(-\frac{|x - \mu|}{\lambda}\right).\end{split}\notag
\end{gather}
The first law of Laplace, from 1774, states that the frequency of an error
can be expressed as an exponential function of the absolute magnitude of
the error, which leads to the Laplace distribution. For many problems in
Economics and Health sciences, this distribution seems to model the data
better than the standard Gaussian distribution

Draw samples from the distribution

\begin{Verbatim}[commandchars=\\\{\}]
\PYG{g+gp}{\PYGZgt{}\PYGZgt{}\PYGZgt{} }\PYG{n}{loc}\PYG{p}{,} \PYG{n}{scale} \PYG{o}{=} \PYG{l+m+mf}{0.}\PYG{p}{,} \PYG{l+m+mf}{1.}
\PYG{g+gp}{\PYGZgt{}\PYGZgt{}\PYGZgt{} }\PYG{n}{s} \PYG{o}{=} \PYG{n}{np}\PYG{o}{.}\PYG{n}{random}\PYG{o}{.}\PYG{n}{laplace}\PYG{p}{(}\PYG{n}{loc}\PYG{p}{,} \PYG{n}{scale}\PYG{p}{,} \PYG{l+m+mi}{1000}\PYG{p}{)}
\end{Verbatim}

Display the histogram of the samples, along with
the probability density function:

\begin{Verbatim}[commandchars=\\\{\}]
\PYG{g+gp}{\PYGZgt{}\PYGZgt{}\PYGZgt{} }\PYG{k+kn}{import} \PYG{n+nn}{matplotlib.pyplot} \PYG{k+kn}{as} \PYG{n+nn}{plt}
\PYG{g+gp}{\PYGZgt{}\PYGZgt{}\PYGZgt{} }\PYG{n}{count}\PYG{p}{,} \PYG{n}{bins}\PYG{p}{,} \PYG{n}{ignored} \PYG{o}{=} \PYG{n}{plt}\PYG{o}{.}\PYG{n}{hist}\PYG{p}{(}\PYG{n}{s}\PYG{p}{,} \PYG{l+m+mi}{30}\PYG{p}{,} \PYG{n}{normed}\PYG{o}{=}\PYG{n+nb+bp}{True}\PYG{p}{)}
\PYG{g+gp}{\PYGZgt{}\PYGZgt{}\PYGZgt{} }\PYG{n}{x} \PYG{o}{=} \PYG{n}{np}\PYG{o}{.}\PYG{n}{arange}\PYG{p}{(}\PYG{o}{\PYGZhy{}}\PYG{l+m+mf}{8.}\PYG{p}{,} \PYG{l+m+mf}{8.}\PYG{p}{,} \PYG{o}{.}\PYG{l+m+mo}{01}\PYG{p}{)}
\PYG{g+gp}{\PYGZgt{}\PYGZgt{}\PYGZgt{} }\PYG{n}{pdf} \PYG{o}{=} \PYG{n}{np}\PYG{o}{.}\PYG{n}{exp}\PYG{p}{(}\PYG{o}{\PYGZhy{}}\PYG{n+nb}{abs}\PYG{p}{(}\PYG{n}{x}\PYG{o}{\PYGZhy{}}\PYG{n}{loc}\PYG{o}{/}\PYG{n}{scale}\PYG{p}{)}\PYG{p}{)}\PYG{o}{/}\PYG{p}{(}\PYG{l+m+mf}{2.}\PYG{o}{*}\PYG{n}{scale}\PYG{p}{)}
\PYG{g+gp}{\PYGZgt{}\PYGZgt{}\PYGZgt{} }\PYG{n}{plt}\PYG{o}{.}\PYG{n}{plot}\PYG{p}{(}\PYG{n}{x}\PYG{p}{,} \PYG{n}{pdf}\PYG{p}{)}
\end{Verbatim}

Plot Gaussian for comparison:

\begin{Verbatim}[commandchars=\\\{\}]
\PYG{g+gp}{\PYGZgt{}\PYGZgt{}\PYGZgt{} }\PYG{n}{g} \PYG{o}{=} \PYG{p}{(}\PYG{l+m+mi}{1}\PYG{o}{/}\PYG{p}{(}\PYG{n}{scale} \PYG{o}{*} \PYG{n}{np}\PYG{o}{.}\PYG{n}{sqrt}\PYG{p}{(}\PYG{l+m+mi}{2} \PYG{o}{*} \PYG{n}{np}\PYG{o}{.}\PYG{n}{pi}\PYG{p}{)}\PYG{p}{)} \PYG{o}{*} 
\PYG{g+gp}{... }     \PYG{n}{np}\PYG{o}{.}\PYG{n}{exp}\PYG{p}{(} \PYG{o}{\PYGZhy{}} \PYG{p}{(}\PYG{n}{x} \PYG{o}{\PYGZhy{}} \PYG{n}{loc}\PYG{p}{)}\PYG{o}{*}\PYG{o}{*}\PYG{l+m+mi}{2} \PYG{o}{/} \PYG{p}{(}\PYG{l+m+mi}{2} \PYG{o}{*} \PYG{n}{scale}\PYG{o}{*}\PYG{o}{*}\PYG{l+m+mi}{2}\PYG{p}{)} \PYG{p}{)}\PYG{p}{)}
\PYG{g+gp}{\PYGZgt{}\PYGZgt{}\PYGZgt{} }\PYG{n}{plt}\PYG{o}{.}\PYG{n}{plot}\PYG{p}{(}\PYG{n}{x}\PYG{p}{,}\PYG{n}{g}\PYG{p}{)}
\end{Verbatim}

\end{fulllineitems}

\index{loadReactionGraph() (in module acsSCCanalysis)}

\begin{fulllineitems}
\phantomsection\label{acsSCCanalysis:acsSCCanalysis.loadReactionGraph}\pysiglinewithargsret{\code{acsSCCanalysis.}\bfcode{loadReactionGraph}}{}{}
\end{fulllineitems}

\index{loadSpecificReactionGraph() (in module acsSCCanalysis)}

\begin{fulllineitems}
\phantomsection\label{acsSCCanalysis:acsSCCanalysis.loadSpecificReactionGraph}\pysiglinewithargsret{\code{acsSCCanalysis.}\bfcode{loadSpecificReactionGraph}}{}{}
\end{fulllineitems}

\index{loadSpecificReactionSubGraph() (in module acsSCCanalysis)}

\begin{fulllineitems}
\phantomsection\label{acsSCCanalysis:acsSCCanalysis.loadSpecificReactionSubGraph}\pysiglinewithargsret{\code{acsSCCanalysis.}\bfcode{loadSpecificReactionSubGraph}}{}{}
\end{fulllineitems}

\index{logistic() (in module acsSCCanalysis)}

\begin{fulllineitems}
\phantomsection\label{acsSCCanalysis:acsSCCanalysis.logistic}\pysiglinewithargsret{\code{acsSCCanalysis.}\bfcode{logistic}}{\emph{loc=0.0}, \emph{scale=1.0}, \emph{size=None}}{}
Draw samples from a Logistic distribution.

Samples are drawn from a Logistic distribution with specified
parameters, loc (location or mean, also median), and scale (\textgreater{}0).

loc : float

scale : float \textgreater{} 0.
\begin{description}
\item[{size}] \leavevmode{[}\{tuple, int\}{]}
Output shape.  If the given shape is, e.g., \code{(m, n, k)}, then
\code{m * n * k} samples are drawn.

\end{description}
\begin{description}
\item[{samples}] \leavevmode{[}\{ndarray, scalar\}{]}
where the values are all integers in  {[}0, n{]}.

\end{description}
\begin{description}
\item[{scipy.stats.distributions.logistic}] \leavevmode{[}probability density function,{]}
distribution or cumulative density function, etc.

\end{description}

The probability density for the Logistic distribution is
\begin{gather}
\begin{split}P(x) = P(x) = \frac{e^{-(x-\mu)/s}}{s(1+e^{-(x-\mu)/s})^2},\end{split}\notag
\end{gather}
where \(\mu\) = location and \(s\) = scale.

The Logistic distribution is used in Extreme Value problems where it
can act as a mixture of Gumbel distributions, in Epidemiology, and by
the World Chess Federation (FIDE) where it is used in the Elo ranking
system, assuming the performance of each player is a logistically
distributed random variable.

Draw samples from the distribution:

\begin{Verbatim}[commandchars=\\\{\}]
\PYG{g+gp}{\PYGZgt{}\PYGZgt{}\PYGZgt{} }\PYG{n}{loc}\PYG{p}{,} \PYG{n}{scale} \PYG{o}{=} \PYG{l+m+mi}{10}\PYG{p}{,} \PYG{l+m+mi}{1}
\PYG{g+gp}{\PYGZgt{}\PYGZgt{}\PYGZgt{} }\PYG{n}{s} \PYG{o}{=} \PYG{n}{np}\PYG{o}{.}\PYG{n}{random}\PYG{o}{.}\PYG{n}{logistic}\PYG{p}{(}\PYG{n}{loc}\PYG{p}{,} \PYG{n}{scale}\PYG{p}{,} \PYG{l+m+mi}{10000}\PYG{p}{)}
\PYG{g+gp}{\PYGZgt{}\PYGZgt{}\PYGZgt{} }\PYG{n}{count}\PYG{p}{,} \PYG{n}{bins}\PYG{p}{,} \PYG{n}{ignored} \PYG{o}{=} \PYG{n}{plt}\PYG{o}{.}\PYG{n}{hist}\PYG{p}{(}\PYG{n}{s}\PYG{p}{,} \PYG{n}{bins}\PYG{o}{=}\PYG{l+m+mi}{50}\PYG{p}{)}
\end{Verbatim}

\#   plot against distribution

\begin{Verbatim}[commandchars=\\\{\}]
\PYG{g+gp}{\PYGZgt{}\PYGZgt{}\PYGZgt{} }\PYG{k}{def} \PYG{n+nf}{logist}\PYG{p}{(}\PYG{n}{x}\PYG{p}{,} \PYG{n}{loc}\PYG{p}{,} \PYG{n}{scale}\PYG{p}{)}\PYG{p}{:}
\PYG{g+gp}{... }    \PYG{k}{return} \PYG{n}{exp}\PYG{p}{(}\PYG{p}{(}\PYG{n}{loc}\PYG{o}{\PYGZhy{}}\PYG{n}{x}\PYG{p}{)}\PYG{o}{/}\PYG{n}{scale}\PYG{p}{)}\PYG{o}{/}\PYG{p}{(}\PYG{n}{scale}\PYG{o}{*}\PYG{p}{(}\PYG{l+m+mi}{1}\PYG{o}{+}\PYG{n}{exp}\PYG{p}{(}\PYG{p}{(}\PYG{n}{loc}\PYG{o}{\PYGZhy{}}\PYG{n}{x}\PYG{p}{)}\PYG{o}{/}\PYG{n}{scale}\PYG{p}{)}\PYG{p}{)}\PYG{o}{*}\PYG{o}{*}\PYG{l+m+mi}{2}\PYG{p}{)}
\PYG{g+gp}{\PYGZgt{}\PYGZgt{}\PYGZgt{} }\PYG{n}{plt}\PYG{o}{.}\PYG{n}{plot}\PYG{p}{(}\PYG{n}{bins}\PYG{p}{,} \PYG{n}{logist}\PYG{p}{(}\PYG{n}{bins}\PYG{p}{,} \PYG{n}{loc}\PYG{p}{,} \PYG{n}{scale}\PYG{p}{)}\PYG{o}{*}\PYG{n}{count}\PYG{o}{.}\PYG{n}{max}\PYG{p}{(}\PYG{p}{)}\PYG{o}{/}\PYGZbs{}
\PYG{g+gp}{... }\PYG{n}{logist}\PYG{p}{(}\PYG{n}{bins}\PYG{p}{,} \PYG{n}{loc}\PYG{p}{,} \PYG{n}{scale}\PYG{p}{)}\PYG{o}{.}\PYG{n}{max}\PYG{p}{(}\PYG{p}{)}\PYG{p}{)}
\PYG{g+gp}{\PYGZgt{}\PYGZgt{}\PYGZgt{} }\PYG{n}{plt}\PYG{o}{.}\PYG{n}{show}\PYG{p}{(}\PYG{p}{)}
\end{Verbatim}

\end{fulllineitems}

\index{lognormal() (in module acsSCCanalysis)}

\begin{fulllineitems}
\phantomsection\label{acsSCCanalysis:acsSCCanalysis.lognormal}\pysiglinewithargsret{\code{acsSCCanalysis.}\bfcode{lognormal}}{\emph{mean=0.0}, \emph{sigma=1.0}, \emph{size=None}}{}
Return samples drawn from a log-normal distribution.

Draw samples from a log-normal distribution with specified mean,
standard deviation, and array shape.  Note that the mean and standard
deviation are not the values for the distribution itself, but of the
underlying normal distribution it is derived from.
\begin{description}
\item[{mean}] \leavevmode{[}float{]}
Mean value of the underlying normal distribution

\item[{sigma}] \leavevmode{[}float, \textgreater{} 0.{]}
Standard deviation of the underlying normal distribution

\item[{size}] \leavevmode{[}tuple of ints{]}
Output shape.  If the given shape is, e.g., \code{(m, n, k)}, then
\code{m * n * k} samples are drawn.

\end{description}
\begin{description}
\item[{samples}] \leavevmode{[}ndarray or float{]}
The desired samples. An array of the same shape as \emph{size} if given,
if \emph{size} is None a float is returned.

\end{description}
\begin{description}
\item[{scipy.stats.lognorm}] \leavevmode{[}probability density function, distribution,{]}
cumulative density function, etc.

\end{description}

A variable \emph{x} has a log-normal distribution if \emph{log(x)} is normally
distributed.  The probability density function for the log-normal
distribution is:
\begin{gather}
\begin{split}p(x) = \frac{1}{\sigma x \sqrt{2\pi}}
e^{(-\frac{(ln(x)-\mu)^2}{2\sigma^2})}\end{split}\notag
\end{gather}
where \(\mu\) is the mean and \(\sigma\) is the standard
deviation of the normally distributed logarithm of the variable.
A log-normal distribution results if a random variable is the \emph{product}
of a large number of independent, identically-distributed variables in
the same way that a normal distribution results if the variable is the
\emph{sum} of a large number of independent, identically-distributed
variables.

Limpert, E., Stahel, W. A., and Abbt, M., ``Log-normal Distributions
across the Sciences: Keys and Clues,'' \emph{BioScience}, Vol. 51, No. 5,
May, 2001.  \href{http://stat.ethz.ch/~stahel/lognormal/bioscience.pdf}{http://stat.ethz.ch/\textasciitilde{}stahel/lognormal/bioscience.pdf}

Reiss, R.D. and Thomas, M., \emph{Statistical Analysis of Extreme Values},
Basel: Birkhauser Verlag, 2001, pp. 31-32.

Draw samples from the distribution:

\begin{Verbatim}[commandchars=\\\{\}]
\PYG{g+gp}{\PYGZgt{}\PYGZgt{}\PYGZgt{} }\PYG{n}{mu}\PYG{p}{,} \PYG{n}{sigma} \PYG{o}{=} \PYG{l+m+mf}{3.}\PYG{p}{,} \PYG{l+m+mf}{1.} \PYG{c}{\PYGZsh{} mean and standard deviation}
\PYG{g+gp}{\PYGZgt{}\PYGZgt{}\PYGZgt{} }\PYG{n}{s} \PYG{o}{=} \PYG{n}{np}\PYG{o}{.}\PYG{n}{random}\PYG{o}{.}\PYG{n}{lognormal}\PYG{p}{(}\PYG{n}{mu}\PYG{p}{,} \PYG{n}{sigma}\PYG{p}{,} \PYG{l+m+mi}{1000}\PYG{p}{)}
\end{Verbatim}

Display the histogram of the samples, along with
the probability density function:

\begin{Verbatim}[commandchars=\\\{\}]
\PYG{g+gp}{\PYGZgt{}\PYGZgt{}\PYGZgt{} }\PYG{k+kn}{import} \PYG{n+nn}{matplotlib.pyplot} \PYG{k+kn}{as} \PYG{n+nn}{plt}
\PYG{g+gp}{\PYGZgt{}\PYGZgt{}\PYGZgt{} }\PYG{n}{count}\PYG{p}{,} \PYG{n}{bins}\PYG{p}{,} \PYG{n}{ignored} \PYG{o}{=} \PYG{n}{plt}\PYG{o}{.}\PYG{n}{hist}\PYG{p}{(}\PYG{n}{s}\PYG{p}{,} \PYG{l+m+mi}{100}\PYG{p}{,} \PYG{n}{normed}\PYG{o}{=}\PYG{n+nb+bp}{True}\PYG{p}{,} \PYG{n}{align}\PYG{o}{=}\PYG{l+s}{\PYGZsq{}}\PYG{l+s}{mid}\PYG{l+s}{\PYGZsq{}}\PYG{p}{)}
\end{Verbatim}

\begin{Verbatim}[commandchars=\\\{\}]
\PYG{g+gp}{\PYGZgt{}\PYGZgt{}\PYGZgt{} }\PYG{n}{x} \PYG{o}{=} \PYG{n}{np}\PYG{o}{.}\PYG{n}{linspace}\PYG{p}{(}\PYG{n+nb}{min}\PYG{p}{(}\PYG{n}{bins}\PYG{p}{)}\PYG{p}{,} \PYG{n+nb}{max}\PYG{p}{(}\PYG{n}{bins}\PYG{p}{)}\PYG{p}{,} \PYG{l+m+mi}{10000}\PYG{p}{)}
\PYG{g+gp}{\PYGZgt{}\PYGZgt{}\PYGZgt{} }\PYG{n}{pdf} \PYG{o}{=} \PYG{p}{(}\PYG{n}{np}\PYG{o}{.}\PYG{n}{exp}\PYG{p}{(}\PYG{o}{\PYGZhy{}}\PYG{p}{(}\PYG{n}{np}\PYG{o}{.}\PYG{n}{log}\PYG{p}{(}\PYG{n}{x}\PYG{p}{)} \PYG{o}{\PYGZhy{}} \PYG{n}{mu}\PYG{p}{)}\PYG{o}{*}\PYG{o}{*}\PYG{l+m+mi}{2} \PYG{o}{/} \PYG{p}{(}\PYG{l+m+mi}{2} \PYG{o}{*} \PYG{n}{sigma}\PYG{o}{*}\PYG{o}{*}\PYG{l+m+mi}{2}\PYG{p}{)}\PYG{p}{)}
\PYG{g+gp}{... }       \PYG{o}{/} \PYG{p}{(}\PYG{n}{x} \PYG{o}{*} \PYG{n}{sigma} \PYG{o}{*} \PYG{n}{np}\PYG{o}{.}\PYG{n}{sqrt}\PYG{p}{(}\PYG{l+m+mi}{2} \PYG{o}{*} \PYG{n}{np}\PYG{o}{.}\PYG{n}{pi}\PYG{p}{)}\PYG{p}{)}\PYG{p}{)}
\end{Verbatim}

\begin{Verbatim}[commandchars=\\\{\}]
\PYG{g+gp}{\PYGZgt{}\PYGZgt{}\PYGZgt{} }\PYG{n}{plt}\PYG{o}{.}\PYG{n}{plot}\PYG{p}{(}\PYG{n}{x}\PYG{p}{,} \PYG{n}{pdf}\PYG{p}{,} \PYG{n}{linewidth}\PYG{o}{=}\PYG{l+m+mi}{2}\PYG{p}{,} \PYG{n}{color}\PYG{o}{=}\PYG{l+s}{\PYGZsq{}}\PYG{l+s}{r}\PYG{l+s}{\PYGZsq{}}\PYG{p}{)}
\PYG{g+gp}{\PYGZgt{}\PYGZgt{}\PYGZgt{} }\PYG{n}{plt}\PYG{o}{.}\PYG{n}{axis}\PYG{p}{(}\PYG{l+s}{\PYGZsq{}}\PYG{l+s}{tight}\PYG{l+s}{\PYGZsq{}}\PYG{p}{)}
\PYG{g+gp}{\PYGZgt{}\PYGZgt{}\PYGZgt{} }\PYG{n}{plt}\PYG{o}{.}\PYG{n}{show}\PYG{p}{(}\PYG{p}{)}
\end{Verbatim}

Demonstrate that taking the products of random samples from a uniform
distribution can be fit well by a log-normal probability density function.

\begin{Verbatim}[commandchars=\\\{\}]
\PYG{g+gp}{\PYGZgt{}\PYGZgt{}\PYGZgt{} }\PYG{c}{\PYGZsh{} Generate a thousand samples: each is the product of 100 random}
\PYG{g+gp}{\PYGZgt{}\PYGZgt{}\PYGZgt{} }\PYG{c}{\PYGZsh{} values, drawn from a normal distribution.}
\PYG{g+gp}{\PYGZgt{}\PYGZgt{}\PYGZgt{} }\PYG{n}{b} \PYG{o}{=} \PYG{p}{[}\PYG{p}{]}
\PYG{g+gp}{\PYGZgt{}\PYGZgt{}\PYGZgt{} }\PYG{k}{for} \PYG{n}{i} \PYG{o+ow}{in} \PYG{n+nb}{range}\PYG{p}{(}\PYG{l+m+mi}{1000}\PYG{p}{)}\PYG{p}{:}
\PYG{g+gp}{... }   \PYG{n}{a} \PYG{o}{=} \PYG{l+m+mf}{10.} \PYG{o}{+} \PYG{n}{np}\PYG{o}{.}\PYG{n}{random}\PYG{o}{.}\PYG{n}{random}\PYG{p}{(}\PYG{l+m+mi}{100}\PYG{p}{)}
\PYG{g+gp}{... }   \PYG{n}{b}\PYG{o}{.}\PYG{n}{append}\PYG{p}{(}\PYG{n}{np}\PYG{o}{.}\PYG{n}{product}\PYG{p}{(}\PYG{n}{a}\PYG{p}{)}\PYG{p}{)}
\end{Verbatim}

\begin{Verbatim}[commandchars=\\\{\}]
\PYG{g+gp}{\PYGZgt{}\PYGZgt{}\PYGZgt{} }\PYG{n}{b} \PYG{o}{=} \PYG{n}{np}\PYG{o}{.}\PYG{n}{array}\PYG{p}{(}\PYG{n}{b}\PYG{p}{)} \PYG{o}{/} \PYG{n}{np}\PYG{o}{.}\PYG{n}{min}\PYG{p}{(}\PYG{n}{b}\PYG{p}{)} \PYG{c}{\PYGZsh{} scale values to be positive}
\PYG{g+gp}{\PYGZgt{}\PYGZgt{}\PYGZgt{} }\PYG{n}{count}\PYG{p}{,} \PYG{n}{bins}\PYG{p}{,} \PYG{n}{ignored} \PYG{o}{=} \PYG{n}{plt}\PYG{o}{.}\PYG{n}{hist}\PYG{p}{(}\PYG{n}{b}\PYG{p}{,} \PYG{l+m+mi}{100}\PYG{p}{,} \PYG{n}{normed}\PYG{o}{=}\PYG{n+nb+bp}{True}\PYG{p}{,} \PYG{n}{align}\PYG{o}{=}\PYG{l+s}{\PYGZsq{}}\PYG{l+s}{center}\PYG{l+s}{\PYGZsq{}}\PYG{p}{)}
\PYG{g+gp}{\PYGZgt{}\PYGZgt{}\PYGZgt{} }\PYG{n}{sigma} \PYG{o}{=} \PYG{n}{np}\PYG{o}{.}\PYG{n}{std}\PYG{p}{(}\PYG{n}{np}\PYG{o}{.}\PYG{n}{log}\PYG{p}{(}\PYG{n}{b}\PYG{p}{)}\PYG{p}{)}
\PYG{g+gp}{\PYGZgt{}\PYGZgt{}\PYGZgt{} }\PYG{n}{mu} \PYG{o}{=} \PYG{n}{np}\PYG{o}{.}\PYG{n}{mean}\PYG{p}{(}\PYG{n}{np}\PYG{o}{.}\PYG{n}{log}\PYG{p}{(}\PYG{n}{b}\PYG{p}{)}\PYG{p}{)}
\end{Verbatim}

\begin{Verbatim}[commandchars=\\\{\}]
\PYG{g+gp}{\PYGZgt{}\PYGZgt{}\PYGZgt{} }\PYG{n}{x} \PYG{o}{=} \PYG{n}{np}\PYG{o}{.}\PYG{n}{linspace}\PYG{p}{(}\PYG{n+nb}{min}\PYG{p}{(}\PYG{n}{bins}\PYG{p}{)}\PYG{p}{,} \PYG{n+nb}{max}\PYG{p}{(}\PYG{n}{bins}\PYG{p}{)}\PYG{p}{,} \PYG{l+m+mi}{10000}\PYG{p}{)}
\PYG{g+gp}{\PYGZgt{}\PYGZgt{}\PYGZgt{} }\PYG{n}{pdf} \PYG{o}{=} \PYG{p}{(}\PYG{n}{np}\PYG{o}{.}\PYG{n}{exp}\PYG{p}{(}\PYG{o}{\PYGZhy{}}\PYG{p}{(}\PYG{n}{np}\PYG{o}{.}\PYG{n}{log}\PYG{p}{(}\PYG{n}{x}\PYG{p}{)} \PYG{o}{\PYGZhy{}} \PYG{n}{mu}\PYG{p}{)}\PYG{o}{*}\PYG{o}{*}\PYG{l+m+mi}{2} \PYG{o}{/} \PYG{p}{(}\PYG{l+m+mi}{2} \PYG{o}{*} \PYG{n}{sigma}\PYG{o}{*}\PYG{o}{*}\PYG{l+m+mi}{2}\PYG{p}{)}\PYG{p}{)}
\PYG{g+gp}{... }       \PYG{o}{/} \PYG{p}{(}\PYG{n}{x} \PYG{o}{*} \PYG{n}{sigma} \PYG{o}{*} \PYG{n}{np}\PYG{o}{.}\PYG{n}{sqrt}\PYG{p}{(}\PYG{l+m+mi}{2} \PYG{o}{*} \PYG{n}{np}\PYG{o}{.}\PYG{n}{pi}\PYG{p}{)}\PYG{p}{)}\PYG{p}{)}
\end{Verbatim}

\begin{Verbatim}[commandchars=\\\{\}]
\PYG{g+gp}{\PYGZgt{}\PYGZgt{}\PYGZgt{} }\PYG{n}{plt}\PYG{o}{.}\PYG{n}{plot}\PYG{p}{(}\PYG{n}{x}\PYG{p}{,} \PYG{n}{pdf}\PYG{p}{,} \PYG{n}{color}\PYG{o}{=}\PYG{l+s}{\PYGZsq{}}\PYG{l+s}{r}\PYG{l+s}{\PYGZsq{}}\PYG{p}{,} \PYG{n}{linewidth}\PYG{o}{=}\PYG{l+m+mi}{2}\PYG{p}{)}
\PYG{g+gp}{\PYGZgt{}\PYGZgt{}\PYGZgt{} }\PYG{n}{plt}\PYG{o}{.}\PYG{n}{show}\PYG{p}{(}\PYG{p}{)}
\end{Verbatim}

\end{fulllineitems}

\index{logseries() (in module acsSCCanalysis)}

\begin{fulllineitems}
\phantomsection\label{acsSCCanalysis:acsSCCanalysis.logseries}\pysiglinewithargsret{\code{acsSCCanalysis.}\bfcode{logseries}}{\emph{p}, \emph{size=None}}{}
Draw samples from a Logarithmic Series distribution.

Samples are drawn from a Log Series distribution with specified
parameter, p (probability, 0 \textless{} p \textless{} 1).

loc : float

scale : float \textgreater{} 0.
\begin{description}
\item[{size}] \leavevmode{[}\{tuple, int\}{]}
Output shape.  If the given shape is, e.g., \code{(m, n, k)}, then
\code{m * n * k} samples are drawn.

\end{description}
\begin{description}
\item[{samples}] \leavevmode{[}\{ndarray, scalar\}{]}
where the values are all integers in  {[}0, n{]}.

\end{description}
\begin{description}
\item[{scipy.stats.distributions.logser}] \leavevmode{[}probability density function,{]}
distribution or cumulative density function, etc.

\end{description}

The probability density for the Log Series distribution is
\begin{gather}
\begin{split}P(k) = \frac{-p^k}{k \ln(1-p)},\end{split}\notag
\end{gather}
where p = probability.

The Log Series distribution is frequently used to represent species
richness and occurrence, first proposed by Fisher, Corbet, and
Williams in 1943 {[}2{]}.  It may also be used to model the numbers of
occupants seen in cars {[}3{]}.

Draw samples from the distribution:

\begin{Verbatim}[commandchars=\\\{\}]
\PYG{g+gp}{\PYGZgt{}\PYGZgt{}\PYGZgt{} }\PYG{n}{a} \PYG{o}{=} \PYG{o}{.}\PYG{l+m+mi}{6}
\PYG{g+gp}{\PYGZgt{}\PYGZgt{}\PYGZgt{} }\PYG{n}{s} \PYG{o}{=} \PYG{n}{np}\PYG{o}{.}\PYG{n}{random}\PYG{o}{.}\PYG{n}{logseries}\PYG{p}{(}\PYG{n}{a}\PYG{p}{,} \PYG{l+m+mi}{10000}\PYG{p}{)}
\PYG{g+gp}{\PYGZgt{}\PYGZgt{}\PYGZgt{} }\PYG{n}{count}\PYG{p}{,} \PYG{n}{bins}\PYG{p}{,} \PYG{n}{ignored} \PYG{o}{=} \PYG{n}{plt}\PYG{o}{.}\PYG{n}{hist}\PYG{p}{(}\PYG{n}{s}\PYG{p}{)}
\end{Verbatim}

\#   plot against distribution

\begin{Verbatim}[commandchars=\\\{\}]
\PYG{g+gp}{\PYGZgt{}\PYGZgt{}\PYGZgt{} }\PYG{k}{def} \PYG{n+nf}{logseries}\PYG{p}{(}\PYG{n}{k}\PYG{p}{,} \PYG{n}{p}\PYG{p}{)}\PYG{p}{:}
\PYG{g+gp}{... }    \PYG{k}{return} \PYG{o}{\PYGZhy{}}\PYG{n}{p}\PYG{o}{*}\PYG{o}{*}\PYG{n}{k}\PYG{o}{/}\PYG{p}{(}\PYG{n}{k}\PYG{o}{*}\PYG{n}{log}\PYG{p}{(}\PYG{l+m+mi}{1}\PYG{o}{\PYGZhy{}}\PYG{n}{p}\PYG{p}{)}\PYG{p}{)}
\PYG{g+gp}{\PYGZgt{}\PYGZgt{}\PYGZgt{} }\PYG{n}{plt}\PYG{o}{.}\PYG{n}{plot}\PYG{p}{(}\PYG{n}{bins}\PYG{p}{,} \PYG{n}{logseries}\PYG{p}{(}\PYG{n}{bins}\PYG{p}{,} \PYG{n}{a}\PYG{p}{)}\PYG{o}{*}\PYG{n}{count}\PYG{o}{.}\PYG{n}{max}\PYG{p}{(}\PYG{p}{)}\PYG{o}{/}
\PYG{g+go}{             logseries(bins, a).max(), \PYGZsq{}r\PYGZsq{})}
\PYG{g+gp}{\PYGZgt{}\PYGZgt{}\PYGZgt{} }\PYG{n}{plt}\PYG{o}{.}\PYG{n}{show}\PYG{p}{(}\PYG{p}{)}
\end{Verbatim}

\end{fulllineitems}

\index{multinomial() (in module acsSCCanalysis)}

\begin{fulllineitems}
\phantomsection\label{acsSCCanalysis:acsSCCanalysis.multinomial}\pysiglinewithargsret{\code{acsSCCanalysis.}\bfcode{multinomial}}{\emph{n}, \emph{pvals}, \emph{size=None}}{}
Draw samples from a multinomial distribution.

The multinomial distribution is a multivariate generalisation of the
binomial distribution.  Take an experiment with one of \code{p}
possible outcomes.  An example of such an experiment is throwing a dice,
where the outcome can be 1 through 6.  Each sample drawn from the
distribution represents \emph{n} such experiments.  Its values,
\code{X\_i = {[}X\_0, X\_1, ..., X\_p{]}}, represent the number of times the outcome
was \code{i}.
\begin{description}
\item[{n}] \leavevmode{[}int{]}
Number of experiments.

\item[{pvals}] \leavevmode{[}sequence of floats, length p{]}
Probabilities of each of the \code{p} different outcomes.  These
should sum to 1 (however, the last element is always assumed to
account for the remaining probability, as long as
\code{sum(pvals{[}:-1{]}) \textless{}= 1)}.

\item[{size}] \leavevmode{[}tuple of ints{]}
Given a \emph{size} of \code{(M, N, K)}, then \code{M*N*K} samples are drawn,
and the output shape becomes \code{(M, N, K, p)}, since each sample
has shape \code{(p,)}.

\end{description}

Throw a dice 20 times:

\begin{Verbatim}[commandchars=\\\{\}]
\PYG{g+gp}{\PYGZgt{}\PYGZgt{}\PYGZgt{} }\PYG{n}{np}\PYG{o}{.}\PYG{n}{random}\PYG{o}{.}\PYG{n}{multinomial}\PYG{p}{(}\PYG{l+m+mi}{20}\PYG{p}{,} \PYG{p}{[}\PYG{l+m+mi}{1}\PYG{o}{/}\PYG{l+m+mf}{6.}\PYG{p}{]}\PYG{o}{*}\PYG{l+m+mi}{6}\PYG{p}{,} \PYG{n}{size}\PYG{o}{=}\PYG{l+m+mi}{1}\PYG{p}{)}
\PYG{g+go}{array([[4, 1, 7, 5, 2, 1]])}
\end{Verbatim}

It landed 4 times on 1, once on 2, etc.

Now, throw the dice 20 times, and 20 times again:

\begin{Verbatim}[commandchars=\\\{\}]
\PYG{g+gp}{\PYGZgt{}\PYGZgt{}\PYGZgt{} }\PYG{n}{np}\PYG{o}{.}\PYG{n}{random}\PYG{o}{.}\PYG{n}{multinomial}\PYG{p}{(}\PYG{l+m+mi}{20}\PYG{p}{,} \PYG{p}{[}\PYG{l+m+mi}{1}\PYG{o}{/}\PYG{l+m+mf}{6.}\PYG{p}{]}\PYG{o}{*}\PYG{l+m+mi}{6}\PYG{p}{,} \PYG{n}{size}\PYG{o}{=}\PYG{l+m+mi}{2}\PYG{p}{)}
\PYG{g+go}{array([[3, 4, 3, 3, 4, 3],}
\PYG{g+go}{       [2, 4, 3, 4, 0, 7]])}
\end{Verbatim}

For the first run, we threw 3 times 1, 4 times 2, etc.  For the second,
we threw 2 times 1, 4 times 2, etc.

A loaded dice is more likely to land on number 6:

\begin{Verbatim}[commandchars=\\\{\}]
\PYG{g+gp}{\PYGZgt{}\PYGZgt{}\PYGZgt{} }\PYG{n}{np}\PYG{o}{.}\PYG{n}{random}\PYG{o}{.}\PYG{n}{multinomial}\PYG{p}{(}\PYG{l+m+mi}{100}\PYG{p}{,} \PYG{p}{[}\PYG{l+m+mi}{1}\PYG{o}{/}\PYG{l+m+mf}{7.}\PYG{p}{]}\PYG{o}{*}\PYG{l+m+mi}{5}\PYG{p}{)}
\PYG{g+go}{array([13, 16, 13, 16, 42])}
\end{Verbatim}

\end{fulllineitems}

\index{multivariate\_normal() (in module acsSCCanalysis)}

\begin{fulllineitems}
\phantomsection\label{acsSCCanalysis:acsSCCanalysis.multivariate_normal}\pysiglinewithargsret{\code{acsSCCanalysis.}\bfcode{multivariate\_normal}}{\emph{mean}, \emph{cov}\optional{, \emph{size}}}{}
Draw random samples from a multivariate normal distribution.

The multivariate normal, multinormal or Gaussian distribution is a
generalization of the one-dimensional normal distribution to higher
dimensions.  Such a distribution is specified by its mean and
covariance matrix.  These parameters are analogous to the mean
(average or ``center'') and variance (standard deviation, or ``width,''
squared) of the one-dimensional normal distribution.
\begin{description}
\item[{mean}] \leavevmode{[}1-D array\_like, of length N{]}
Mean of the N-dimensional distribution.

\item[{cov}] \leavevmode{[}2-D array\_like, of shape (N, N){]}
Covariance matrix of the distribution.  Must be symmetric and
positive semi-definite for ``physically meaningful'' results.

\item[{size}] \leavevmode{[}int or tuple of ints, optional{]}
Given a shape of, for example, \code{(m,n,k)}, \code{m*n*k} samples are
generated, and packed in an \emph{m}-by-\emph{n}-by-\emph{k} arrangement.  Because
each sample is \emph{N}-dimensional, the output shape is \code{(m,n,k,N)}.
If no shape is specified, a single (\emph{N}-D) sample is returned.

\end{description}
\begin{description}
\item[{out}] \leavevmode{[}ndarray{]}
The drawn samples, of shape \emph{size}, if that was provided.  If not,
the shape is \code{(N,)}.

In other words, each entry \code{out{[}i,j,...,:{]}} is an N-dimensional
value drawn from the distribution.

\end{description}

The mean is a coordinate in N-dimensional space, which represents the
location where samples are most likely to be generated.  This is
analogous to the peak of the bell curve for the one-dimensional or
univariate normal distribution.

Covariance indicates the level to which two variables vary together.
From the multivariate normal distribution, we draw N-dimensional
samples, \(X = [x_1, x_2, ... x_N]\).  The covariance matrix
element \(C_{ij}\) is the covariance of \(x_i\) and \(x_j\).
The element \(C_{ii}\) is the variance of \(x_i\) (i.e. its
``spread'').

Instead of specifying the full covariance matrix, popular
approximations include:
\begin{itemize}
\item {} 
Spherical covariance (\emph{cov} is a multiple of the identity matrix)

\item {} 
Diagonal covariance (\emph{cov} has non-negative elements, and only on
the diagonal)

\end{itemize}

This geometrical property can be seen in two dimensions by plotting
generated data-points:

\begin{Verbatim}[commandchars=\\\{\}]
\PYG{g+gp}{\PYGZgt{}\PYGZgt{}\PYGZgt{} }\PYG{n}{mean} \PYG{o}{=} \PYG{p}{[}\PYG{l+m+mi}{0}\PYG{p}{,}\PYG{l+m+mi}{0}\PYG{p}{]}
\PYG{g+gp}{\PYGZgt{}\PYGZgt{}\PYGZgt{} }\PYG{n}{cov} \PYG{o}{=} \PYG{p}{[}\PYG{p}{[}\PYG{l+m+mi}{1}\PYG{p}{,}\PYG{l+m+mi}{0}\PYG{p}{]}\PYG{p}{,}\PYG{p}{[}\PYG{l+m+mi}{0}\PYG{p}{,}\PYG{l+m+mi}{100}\PYG{p}{]}\PYG{p}{]} \PYG{c}{\PYGZsh{} diagonal covariance, points lie on x or y\PYGZhy{}axis}
\end{Verbatim}

\begin{Verbatim}[commandchars=\\\{\}]
\PYG{g+gp}{\PYGZgt{}\PYGZgt{}\PYGZgt{} }\PYG{k+kn}{import} \PYG{n+nn}{matplotlib.pyplot} \PYG{k+kn}{as} \PYG{n+nn}{plt}
\PYG{g+gp}{\PYGZgt{}\PYGZgt{}\PYGZgt{} }\PYG{n}{x}\PYG{p}{,}\PYG{n}{y} \PYG{o}{=} \PYG{n}{np}\PYG{o}{.}\PYG{n}{random}\PYG{o}{.}\PYG{n}{multivariate\PYGZus{}normal}\PYG{p}{(}\PYG{n}{mean}\PYG{p}{,}\PYG{n}{cov}\PYG{p}{,}\PYG{l+m+mi}{5000}\PYG{p}{)}\PYG{o}{.}\PYG{n}{T}
\PYG{g+gp}{\PYGZgt{}\PYGZgt{}\PYGZgt{} }\PYG{n}{plt}\PYG{o}{.}\PYG{n}{plot}\PYG{p}{(}\PYG{n}{x}\PYG{p}{,}\PYG{n}{y}\PYG{p}{,}\PYG{l+s}{\PYGZsq{}}\PYG{l+s}{x}\PYG{l+s}{\PYGZsq{}}\PYG{p}{)}\PYG{p}{;} \PYG{n}{plt}\PYG{o}{.}\PYG{n}{axis}\PYG{p}{(}\PYG{l+s}{\PYGZsq{}}\PYG{l+s}{equal}\PYG{l+s}{\PYGZsq{}}\PYG{p}{)}\PYG{p}{;} \PYG{n}{plt}\PYG{o}{.}\PYG{n}{show}\PYG{p}{(}\PYG{p}{)}
\end{Verbatim}

Note that the covariance matrix must be non-negative definite.

Papoulis, A., \emph{Probability, Random Variables, and Stochastic Processes},
3rd ed., New York: McGraw-Hill, 1991.

Duda, R. O., Hart, P. E., and Stork, D. G., \emph{Pattern Classification},
2nd ed., New York: Wiley, 2001.

\begin{Verbatim}[commandchars=\\\{\}]
\PYG{g+gp}{\PYGZgt{}\PYGZgt{}\PYGZgt{} }\PYG{n}{mean} \PYG{o}{=} \PYG{p}{(}\PYG{l+m+mi}{1}\PYG{p}{,}\PYG{l+m+mi}{2}\PYG{p}{)}
\PYG{g+gp}{\PYGZgt{}\PYGZgt{}\PYGZgt{} }\PYG{n}{cov} \PYG{o}{=} \PYG{p}{[}\PYG{p}{[}\PYG{l+m+mi}{1}\PYG{p}{,}\PYG{l+m+mi}{0}\PYG{p}{]}\PYG{p}{,}\PYG{p}{[}\PYG{l+m+mi}{1}\PYG{p}{,}\PYG{l+m+mi}{0}\PYG{p}{]}\PYG{p}{]}
\PYG{g+gp}{\PYGZgt{}\PYGZgt{}\PYGZgt{} }\PYG{n}{x} \PYG{o}{=} \PYG{n}{np}\PYG{o}{.}\PYG{n}{random}\PYG{o}{.}\PYG{n}{multivariate\PYGZus{}normal}\PYG{p}{(}\PYG{n}{mean}\PYG{p}{,}\PYG{n}{cov}\PYG{p}{,}\PYG{p}{(}\PYG{l+m+mi}{3}\PYG{p}{,}\PYG{l+m+mi}{3}\PYG{p}{)}\PYG{p}{)}
\PYG{g+gp}{\PYGZgt{}\PYGZgt{}\PYGZgt{} }\PYG{n}{x}\PYG{o}{.}\PYG{n}{shape}
\PYG{g+go}{(3, 3, 2)}
\end{Verbatim}

The following is probably true, given that 0.6 is roughly twice the
standard deviation:

\begin{Verbatim}[commandchars=\\\{\}]
\PYG{g+gp}{\PYGZgt{}\PYGZgt{}\PYGZgt{} }\PYG{k}{print} \PYG{n+nb}{list}\PYG{p}{(} \PYG{p}{(}\PYG{n}{x}\PYG{p}{[}\PYG{l+m+mi}{0}\PYG{p}{,}\PYG{l+m+mi}{0}\PYG{p}{,}\PYG{p}{:}\PYG{p}{]} \PYG{o}{\PYGZhy{}} \PYG{n}{mean}\PYG{p}{)} \PYG{o}{\PYGZlt{}} \PYG{l+m+mf}{0.6} \PYG{p}{)}
\PYG{g+go}{[True, True]}
\end{Verbatim}

\end{fulllineitems}

\index{negative\_binomial() (in module acsSCCanalysis)}

\begin{fulllineitems}
\phantomsection\label{acsSCCanalysis:acsSCCanalysis.negative_binomial}\pysiglinewithargsret{\code{acsSCCanalysis.}\bfcode{negative\_binomial}}{\emph{n}, \emph{p}, \emph{size=None}}{}
Draw samples from a negative\_binomial distribution.

Samples are drawn from a negative\_Binomial distribution with specified
parameters, \emph{n} trials and \emph{p} probability of success where \emph{n} is an
integer \textgreater{} 0 and \emph{p} is in the interval {[}0, 1{]}.
\begin{description}
\item[{n}] \leavevmode{[}int{]}
Parameter, \textgreater{} 0.

\item[{p}] \leavevmode{[}float{]}
Parameter, \textgreater{}= 0 and \textless{}=1.

\item[{size}] \leavevmode{[}int or tuple of ints{]}
Output shape. If the given shape is, e.g., \code{(m, n, k)}, then
\code{m * n * k} samples are drawn.

\end{description}
\begin{description}
\item[{samples}] \leavevmode{[}int or ndarray of ints{]}
Drawn samples.

\end{description}

The probability density for the Negative Binomial distribution is
\begin{gather}
\begin{split}P(N;n,p) = \binom{N+n-1}{n-1}p^{n}(1-p)^{N},\end{split}\notag
\end{gather}
where \(n-1\) is the number of successes, \(p\) is the probability
of success, and \(N+n-1\) is the number of trials.

The negative binomial distribution gives the probability of n-1 successes
and N failures in N+n-1 trials, and success on the (N+n)th trial.

If one throws a die repeatedly until the third time a ``1'' appears, then the
probability distribution of the number of non-``1''s that appear before the
third ``1'' is a negative binomial distribution.

Draw samples from the distribution:

A real world example. A company drills wild-cat oil exploration wells, each
with an estimated probability of success of 0.1.  What is the probability
of having one success for each successive well, that is what is the
probability of a single success after drilling 5 wells, after 6 wells,
etc.?

\begin{Verbatim}[commandchars=\\\{\}]
\PYG{g+gp}{\PYGZgt{}\PYGZgt{}\PYGZgt{} }\PYG{n}{s} \PYG{o}{=} \PYG{n}{np}\PYG{o}{.}\PYG{n}{random}\PYG{o}{.}\PYG{n}{negative\PYGZus{}binomial}\PYG{p}{(}\PYG{l+m+mi}{1}\PYG{p}{,} \PYG{l+m+mf}{0.1}\PYG{p}{,} \PYG{l+m+mi}{100000}\PYG{p}{)}
\PYG{g+gp}{\PYGZgt{}\PYGZgt{}\PYGZgt{} }\PYG{k}{for} \PYG{n}{i} \PYG{o+ow}{in} \PYG{n+nb}{range}\PYG{p}{(}\PYG{l+m+mi}{1}\PYG{p}{,} \PYG{l+m+mi}{11}\PYG{p}{)}\PYG{p}{:}
\PYG{g+gp}{... }   \PYG{n}{probability} \PYG{o}{=} \PYG{n+nb}{sum}\PYG{p}{(}\PYG{n}{s}\PYG{o}{\PYGZlt{}}\PYG{n}{i}\PYG{p}{)} \PYG{o}{/} \PYG{l+m+mf}{100000.}
\PYG{g+gp}{... }   \PYG{k}{print} \PYG{n}{i}\PYG{p}{,} \PYG{l+s}{\PYGZdq{}}\PYG{l+s}{wells drilled, probability of one success =}\PYG{l+s}{\PYGZdq{}}\PYG{p}{,} \PYG{n}{probability}
\end{Verbatim}

\end{fulllineitems}

\index{noncentral\_chisquare() (in module acsSCCanalysis)}

\begin{fulllineitems}
\phantomsection\label{acsSCCanalysis:acsSCCanalysis.noncentral_chisquare}\pysiglinewithargsret{\code{acsSCCanalysis.}\bfcode{noncentral\_chisquare}}{\emph{df}, \emph{nonc}, \emph{size=None}}{}
Draw samples from a noncentral chi-square distribution.

The noncentral \(\chi^2\) distribution is a generalisation of
the \(\chi^2\) distribution.
\begin{description}
\item[{df}] \leavevmode{[}int{]}
Degrees of freedom, should be \textgreater{}= 1.

\item[{nonc}] \leavevmode{[}float{]}
Non-centrality, should be \textgreater{} 0.

\item[{size}] \leavevmode{[}int or tuple of ints{]}
Shape of the output.

\end{description}

The probability density function for the noncentral Chi-square distribution
is
\begin{gather}
\begin{split}P(x;df,nonc) = \sum^{\infty}_{i=0}
\frac{e^{-nonc/2}(nonc/2)^{i}}{i!}P_{Y_{df+2i}}(x),\end{split}\notag
\end{gather}
where \(Y_{q}\) is the Chi-square with q degrees of freedom.

In Delhi (2007), it is noted that the noncentral chi-square is useful in
bombing and coverage problems, the probability of killing the point target
given by the noncentral chi-squared distribution.

Draw values from the distribution and plot the histogram

\begin{Verbatim}[commandchars=\\\{\}]
\PYG{g+gp}{\PYGZgt{}\PYGZgt{}\PYGZgt{} }\PYG{k+kn}{import} \PYG{n+nn}{matplotlib.pyplot} \PYG{k+kn}{as} \PYG{n+nn}{plt}
\PYG{g+gp}{\PYGZgt{}\PYGZgt{}\PYGZgt{} }\PYG{n}{values} \PYG{o}{=} \PYG{n}{plt}\PYG{o}{.}\PYG{n}{hist}\PYG{p}{(}\PYG{n}{np}\PYG{o}{.}\PYG{n}{random}\PYG{o}{.}\PYG{n}{noncentral\PYGZus{}chisquare}\PYG{p}{(}\PYG{l+m+mi}{3}\PYG{p}{,} \PYG{l+m+mi}{20}\PYG{p}{,} \PYG{l+m+mi}{100000}\PYG{p}{)}\PYG{p}{,}
\PYG{g+gp}{... }                  \PYG{n}{bins}\PYG{o}{=}\PYG{l+m+mi}{200}\PYG{p}{,} \PYG{n}{normed}\PYG{o}{=}\PYG{n+nb+bp}{True}\PYG{p}{)}
\PYG{g+gp}{\PYGZgt{}\PYGZgt{}\PYGZgt{} }\PYG{n}{plt}\PYG{o}{.}\PYG{n}{show}\PYG{p}{(}\PYG{p}{)}
\end{Verbatim}

Draw values from a noncentral chisquare with very small noncentrality,
and compare to a chisquare.

\begin{Verbatim}[commandchars=\\\{\}]
\PYG{g+gp}{\PYGZgt{}\PYGZgt{}\PYGZgt{} }\PYG{n}{plt}\PYG{o}{.}\PYG{n}{figure}\PYG{p}{(}\PYG{p}{)}
\PYG{g+gp}{\PYGZgt{}\PYGZgt{}\PYGZgt{} }\PYG{n}{values} \PYG{o}{=} \PYG{n}{plt}\PYG{o}{.}\PYG{n}{hist}\PYG{p}{(}\PYG{n}{np}\PYG{o}{.}\PYG{n}{random}\PYG{o}{.}\PYG{n}{noncentral\PYGZus{}chisquare}\PYG{p}{(}\PYG{l+m+mi}{3}\PYG{p}{,} \PYG{o}{.}\PYG{l+m+mo}{0000001}\PYG{p}{,} \PYG{l+m+mi}{100000}\PYG{p}{)}\PYG{p}{,}
\PYG{g+gp}{... }                  \PYG{n}{bins}\PYG{o}{=}\PYG{n}{np}\PYG{o}{.}\PYG{n}{arange}\PYG{p}{(}\PYG{l+m+mf}{0.}\PYG{p}{,} \PYG{l+m+mi}{25}\PYG{p}{,} \PYG{o}{.}\PYG{l+m+mi}{1}\PYG{p}{)}\PYG{p}{,} \PYG{n}{normed}\PYG{o}{=}\PYG{n+nb+bp}{True}\PYG{p}{)}
\PYG{g+gp}{\PYGZgt{}\PYGZgt{}\PYGZgt{} }\PYG{n}{values2} \PYG{o}{=} \PYG{n}{plt}\PYG{o}{.}\PYG{n}{hist}\PYG{p}{(}\PYG{n}{np}\PYG{o}{.}\PYG{n}{random}\PYG{o}{.}\PYG{n}{chisquare}\PYG{p}{(}\PYG{l+m+mi}{3}\PYG{p}{,} \PYG{l+m+mi}{100000}\PYG{p}{)}\PYG{p}{,}
\PYG{g+gp}{... }                   \PYG{n}{bins}\PYG{o}{=}\PYG{n}{np}\PYG{o}{.}\PYG{n}{arange}\PYG{p}{(}\PYG{l+m+mf}{0.}\PYG{p}{,} \PYG{l+m+mi}{25}\PYG{p}{,} \PYG{o}{.}\PYG{l+m+mi}{1}\PYG{p}{)}\PYG{p}{,} \PYG{n}{normed}\PYG{o}{=}\PYG{n+nb+bp}{True}\PYG{p}{)}
\PYG{g+gp}{\PYGZgt{}\PYGZgt{}\PYGZgt{} }\PYG{n}{plt}\PYG{o}{.}\PYG{n}{plot}\PYG{p}{(}\PYG{n}{values}\PYG{p}{[}\PYG{l+m+mi}{1}\PYG{p}{]}\PYG{p}{[}\PYG{l+m+mi}{0}\PYG{p}{:}\PYG{o}{\PYGZhy{}}\PYG{l+m+mi}{1}\PYG{p}{]}\PYG{p}{,} \PYG{n}{values}\PYG{p}{[}\PYG{l+m+mi}{0}\PYG{p}{]}\PYG{o}{\PYGZhy{}}\PYG{n}{values2}\PYG{p}{[}\PYG{l+m+mi}{0}\PYG{p}{]}\PYG{p}{,} \PYG{l+s}{\PYGZsq{}}\PYG{l+s}{ob}\PYG{l+s}{\PYGZsq{}}\PYG{p}{)}
\PYG{g+gp}{\PYGZgt{}\PYGZgt{}\PYGZgt{} }\PYG{n}{plt}\PYG{o}{.}\PYG{n}{show}\PYG{p}{(}\PYG{p}{)}
\end{Verbatim}

Demonstrate how large values of non-centrality lead to a more symmetric
distribution.

\begin{Verbatim}[commandchars=\\\{\}]
\PYG{g+gp}{\PYGZgt{}\PYGZgt{}\PYGZgt{} }\PYG{n}{plt}\PYG{o}{.}\PYG{n}{figure}\PYG{p}{(}\PYG{p}{)}
\PYG{g+gp}{\PYGZgt{}\PYGZgt{}\PYGZgt{} }\PYG{n}{values} \PYG{o}{=} \PYG{n}{plt}\PYG{o}{.}\PYG{n}{hist}\PYG{p}{(}\PYG{n}{np}\PYG{o}{.}\PYG{n}{random}\PYG{o}{.}\PYG{n}{noncentral\PYGZus{}chisquare}\PYG{p}{(}\PYG{l+m+mi}{3}\PYG{p}{,} \PYG{l+m+mi}{20}\PYG{p}{,} \PYG{l+m+mi}{100000}\PYG{p}{)}\PYG{p}{,}
\PYG{g+gp}{... }                  \PYG{n}{bins}\PYG{o}{=}\PYG{l+m+mi}{200}\PYG{p}{,} \PYG{n}{normed}\PYG{o}{=}\PYG{n+nb+bp}{True}\PYG{p}{)}
\PYG{g+gp}{\PYGZgt{}\PYGZgt{}\PYGZgt{} }\PYG{n}{plt}\PYG{o}{.}\PYG{n}{show}\PYG{p}{(}\PYG{p}{)}
\end{Verbatim}

\end{fulllineitems}

\index{noncentral\_f() (in module acsSCCanalysis)}

\begin{fulllineitems}
\phantomsection\label{acsSCCanalysis:acsSCCanalysis.noncentral_f}\pysiglinewithargsret{\code{acsSCCanalysis.}\bfcode{noncentral\_f}}{\emph{dfnum}, \emph{dfden}, \emph{nonc}, \emph{size=None}}{}
Draw samples from the noncentral F distribution.

Samples are drawn from an F distribution with specified parameters,
\emph{dfnum} (degrees of freedom in numerator) and \emph{dfden} (degrees of
freedom in denominator), where both parameters \textgreater{} 1.
\emph{nonc} is the non-centrality parameter.
\begin{description}
\item[{dfnum}] \leavevmode{[}int{]}
Parameter, should be \textgreater{} 1.

\item[{dfden}] \leavevmode{[}int{]}
Parameter, should be \textgreater{} 1.

\item[{nonc}] \leavevmode{[}float{]}
Parameter, should be \textgreater{}= 0.

\item[{size}] \leavevmode{[}int or tuple of ints{]}
Output shape. If the given shape is, e.g., \code{(m, n, k)}, then
\code{m * n * k} samples are drawn.

\end{description}
\begin{description}
\item[{samples}] \leavevmode{[}scalar or ndarray{]}
Drawn samples.

\end{description}

When calculating the power of an experiment (power = probability of
rejecting the null hypothesis when a specific alternative is true) the
non-central F statistic becomes important.  When the null hypothesis is
true, the F statistic follows a central F distribution. When the null
hypothesis is not true, then it follows a non-central F statistic.

Weisstein, Eric W. ``Noncentral F-Distribution.'' From MathWorld--A Wolfram
Web Resource.  \href{http://mathworld.wolfram.com/NoncentralF-Distribution.html}{http://mathworld.wolfram.com/NoncentralF-Distribution.html}

Wikipedia, ``Noncentral F distribution'',
\href{http://en.wikipedia.org/wiki/Noncentral\_F-distribution}{http://en.wikipedia.org/wiki/Noncentral\_F-distribution}

In a study, testing for a specific alternative to the null hypothesis
requires use of the Noncentral F distribution. We need to calculate the
area in the tail of the distribution that exceeds the value of the F
distribution for the null hypothesis.  We'll plot the two probability
distributions for comparison.

\begin{Verbatim}[commandchars=\\\{\}]
\PYG{g+gp}{\PYGZgt{}\PYGZgt{}\PYGZgt{} }\PYG{n}{dfnum} \PYG{o}{=} \PYG{l+m+mi}{3} \PYG{c}{\PYGZsh{} between group deg of freedom}
\PYG{g+gp}{\PYGZgt{}\PYGZgt{}\PYGZgt{} }\PYG{n}{dfden} \PYG{o}{=} \PYG{l+m+mi}{20} \PYG{c}{\PYGZsh{} within groups degrees of freedom}
\PYG{g+gp}{\PYGZgt{}\PYGZgt{}\PYGZgt{} }\PYG{n}{nonc} \PYG{o}{=} \PYG{l+m+mf}{3.0}
\PYG{g+gp}{\PYGZgt{}\PYGZgt{}\PYGZgt{} }\PYG{n}{nc\PYGZus{}vals} \PYG{o}{=} \PYG{n}{np}\PYG{o}{.}\PYG{n}{random}\PYG{o}{.}\PYG{n}{noncentral\PYGZus{}f}\PYG{p}{(}\PYG{n}{dfnum}\PYG{p}{,} \PYG{n}{dfden}\PYG{p}{,} \PYG{n}{nonc}\PYG{p}{,} \PYG{l+m+mi}{1000000}\PYG{p}{)}
\PYG{g+gp}{\PYGZgt{}\PYGZgt{}\PYGZgt{} }\PYG{n}{NF} \PYG{o}{=} \PYG{n}{np}\PYG{o}{.}\PYG{n}{histogram}\PYG{p}{(}\PYG{n}{nc\PYGZus{}vals}\PYG{p}{,} \PYG{n}{bins}\PYG{o}{=}\PYG{l+m+mi}{50}\PYG{p}{,} \PYG{n}{normed}\PYG{o}{=}\PYG{n+nb+bp}{True}\PYG{p}{)}
\PYG{g+gp}{\PYGZgt{}\PYGZgt{}\PYGZgt{} }\PYG{n}{c\PYGZus{}vals} \PYG{o}{=} \PYG{n}{np}\PYG{o}{.}\PYG{n}{random}\PYG{o}{.}\PYG{n}{f}\PYG{p}{(}\PYG{n}{dfnum}\PYG{p}{,} \PYG{n}{dfden}\PYG{p}{,} \PYG{l+m+mi}{1000000}\PYG{p}{)}
\PYG{g+gp}{\PYGZgt{}\PYGZgt{}\PYGZgt{} }\PYG{n}{F} \PYG{o}{=} \PYG{n}{np}\PYG{o}{.}\PYG{n}{histogram}\PYG{p}{(}\PYG{n}{c\PYGZus{}vals}\PYG{p}{,} \PYG{n}{bins}\PYG{o}{=}\PYG{l+m+mi}{50}\PYG{p}{,} \PYG{n}{normed}\PYG{o}{=}\PYG{n+nb+bp}{True}\PYG{p}{)}
\PYG{g+gp}{\PYGZgt{}\PYGZgt{}\PYGZgt{} }\PYG{n}{plt}\PYG{o}{.}\PYG{n}{plot}\PYG{p}{(}\PYG{n}{F}\PYG{p}{[}\PYG{l+m+mi}{1}\PYG{p}{]}\PYG{p}{[}\PYG{l+m+mi}{1}\PYG{p}{:}\PYG{p}{]}\PYG{p}{,} \PYG{n}{F}\PYG{p}{[}\PYG{l+m+mi}{0}\PYG{p}{]}\PYG{p}{)}
\PYG{g+gp}{\PYGZgt{}\PYGZgt{}\PYGZgt{} }\PYG{n}{plt}\PYG{o}{.}\PYG{n}{plot}\PYG{p}{(}\PYG{n}{NF}\PYG{p}{[}\PYG{l+m+mi}{1}\PYG{p}{]}\PYG{p}{[}\PYG{l+m+mi}{1}\PYG{p}{:}\PYG{p}{]}\PYG{p}{,} \PYG{n}{NF}\PYG{p}{[}\PYG{l+m+mi}{0}\PYG{p}{]}\PYG{p}{)}
\PYG{g+gp}{\PYGZgt{}\PYGZgt{}\PYGZgt{} }\PYG{n}{plt}\PYG{o}{.}\PYG{n}{show}\PYG{p}{(}\PYG{p}{)}
\end{Verbatim}

\end{fulllineitems}

\index{normal() (in module acsSCCanalysis)}

\begin{fulllineitems}
\phantomsection\label{acsSCCanalysis:acsSCCanalysis.normal}\pysiglinewithargsret{\code{acsSCCanalysis.}\bfcode{normal}}{\emph{loc=0.0}, \emph{scale=1.0}, \emph{size=None}}{}
Draw random samples from a normal (Gaussian) distribution.

The probability density function of the normal distribution, first
derived by De Moivre and 200 years later by both Gauss and Laplace
independently {\color{red}\bfseries{}{[}2{]}\_}, is often called the bell curve because of
its characteristic shape (see the example below).

The normal distributions occurs often in nature.  For example, it
describes the commonly occurring distribution of samples influenced
by a large number of tiny, random disturbances, each with its own
unique distribution {\color{red}\bfseries{}{[}2{]}\_}.
\begin{description}
\item[{loc}] \leavevmode{[}float{]}
Mean (``centre'') of the distribution.

\item[{scale}] \leavevmode{[}float{]}
Standard deviation (spread or ``width'') of the distribution.

\item[{size}] \leavevmode{[}tuple of ints{]}
Output shape.  If the given shape is, e.g., \code{(m, n, k)}, then
\code{m * n * k} samples are drawn.

\end{description}
\begin{description}
\item[{scipy.stats.distributions.norm}] \leavevmode{[}probability density function,{]}
distribution or cumulative density function, etc.

\end{description}

The probability density for the Gaussian distribution is
\begin{gather}
\begin{split}p(x) = \frac{1}{\sqrt{ 2 \pi \sigma^2 }}
e^{ - \frac{ (x - \mu)^2 } {2 \sigma^2} },\end{split}\notag
\end{gather}
where \(\mu\) is the mean and \(\sigma\) the standard deviation.
The square of the standard deviation, \(\sigma^2\), is called the
variance.

The function has its peak at the mean, and its ``spread'' increases with
the standard deviation (the function reaches 0.607 times its maximum at
\(x + \sigma\) and \(x - \sigma\) {\color{red}\bfseries{}{[}2{]}\_}).  This implies that
\emph{numpy.random.normal} is more likely to return samples lying close to the
mean, rather than those far away.

Draw samples from the distribution:

\begin{Verbatim}[commandchars=\\\{\}]
\PYG{g+gp}{\PYGZgt{}\PYGZgt{}\PYGZgt{} }\PYG{n}{mu}\PYG{p}{,} \PYG{n}{sigma} \PYG{o}{=} \PYG{l+m+mi}{0}\PYG{p}{,} \PYG{l+m+mf}{0.1} \PYG{c}{\PYGZsh{} mean and standard deviation}
\PYG{g+gp}{\PYGZgt{}\PYGZgt{}\PYGZgt{} }\PYG{n}{s} \PYG{o}{=} \PYG{n}{np}\PYG{o}{.}\PYG{n}{random}\PYG{o}{.}\PYG{n}{normal}\PYG{p}{(}\PYG{n}{mu}\PYG{p}{,} \PYG{n}{sigma}\PYG{p}{,} \PYG{l+m+mi}{1000}\PYG{p}{)}
\end{Verbatim}

Verify the mean and the variance:

\begin{Verbatim}[commandchars=\\\{\}]
\PYG{g+gp}{\PYGZgt{}\PYGZgt{}\PYGZgt{} }\PYG{n+nb}{abs}\PYG{p}{(}\PYG{n}{mu} \PYG{o}{\PYGZhy{}} \PYG{n}{np}\PYG{o}{.}\PYG{n}{mean}\PYG{p}{(}\PYG{n}{s}\PYG{p}{)}\PYG{p}{)} \PYG{o}{\PYGZlt{}} \PYG{l+m+mf}{0.01}
\PYG{g+go}{True}
\end{Verbatim}

\begin{Verbatim}[commandchars=\\\{\}]
\PYG{g+gp}{\PYGZgt{}\PYGZgt{}\PYGZgt{} }\PYG{n+nb}{abs}\PYG{p}{(}\PYG{n}{sigma} \PYG{o}{\PYGZhy{}} \PYG{n}{np}\PYG{o}{.}\PYG{n}{std}\PYG{p}{(}\PYG{n}{s}\PYG{p}{,} \PYG{n}{ddof}\PYG{o}{=}\PYG{l+m+mi}{1}\PYG{p}{)}\PYG{p}{)} \PYG{o}{\PYGZlt{}} \PYG{l+m+mf}{0.01}
\PYG{g+go}{True}
\end{Verbatim}

Display the histogram of the samples, along with
the probability density function:

\begin{Verbatim}[commandchars=\\\{\}]
\PYG{g+gp}{\PYGZgt{}\PYGZgt{}\PYGZgt{} }\PYG{k+kn}{import} \PYG{n+nn}{matplotlib.pyplot} \PYG{k+kn}{as} \PYG{n+nn}{plt}
\PYG{g+gp}{\PYGZgt{}\PYGZgt{}\PYGZgt{} }\PYG{n}{count}\PYG{p}{,} \PYG{n}{bins}\PYG{p}{,} \PYG{n}{ignored} \PYG{o}{=} \PYG{n}{plt}\PYG{o}{.}\PYG{n}{hist}\PYG{p}{(}\PYG{n}{s}\PYG{p}{,} \PYG{l+m+mi}{30}\PYG{p}{,} \PYG{n}{normed}\PYG{o}{=}\PYG{n+nb+bp}{True}\PYG{p}{)}
\PYG{g+gp}{\PYGZgt{}\PYGZgt{}\PYGZgt{} }\PYG{n}{plt}\PYG{o}{.}\PYG{n}{plot}\PYG{p}{(}\PYG{n}{bins}\PYG{p}{,} \PYG{l+m+mi}{1}\PYG{o}{/}\PYG{p}{(}\PYG{n}{sigma} \PYG{o}{*} \PYG{n}{np}\PYG{o}{.}\PYG{n}{sqrt}\PYG{p}{(}\PYG{l+m+mi}{2} \PYG{o}{*} \PYG{n}{np}\PYG{o}{.}\PYG{n}{pi}\PYG{p}{)}\PYG{p}{)} \PYG{o}{*}
\PYG{g+gp}{... }               \PYG{n}{np}\PYG{o}{.}\PYG{n}{exp}\PYG{p}{(} \PYG{o}{\PYGZhy{}} \PYG{p}{(}\PYG{n}{bins} \PYG{o}{\PYGZhy{}} \PYG{n}{mu}\PYG{p}{)}\PYG{o}{*}\PYG{o}{*}\PYG{l+m+mi}{2} \PYG{o}{/} \PYG{p}{(}\PYG{l+m+mi}{2} \PYG{o}{*} \PYG{n}{sigma}\PYG{o}{*}\PYG{o}{*}\PYG{l+m+mi}{2}\PYG{p}{)} \PYG{p}{)}\PYG{p}{,}
\PYG{g+gp}{... }         \PYG{n}{linewidth}\PYG{o}{=}\PYG{l+m+mi}{2}\PYG{p}{,} \PYG{n}{color}\PYG{o}{=}\PYG{l+s}{\PYGZsq{}}\PYG{l+s}{r}\PYG{l+s}{\PYGZsq{}}\PYG{p}{)}
\PYG{g+gp}{\PYGZgt{}\PYGZgt{}\PYGZgt{} }\PYG{n}{plt}\PYG{o}{.}\PYG{n}{show}\PYG{p}{(}\PYG{p}{)}
\end{Verbatim}

\end{fulllineitems}

\index{pareto() (in module acsSCCanalysis)}

\begin{fulllineitems}
\phantomsection\label{acsSCCanalysis:acsSCCanalysis.pareto}\pysiglinewithargsret{\code{acsSCCanalysis.}\bfcode{pareto}}{\emph{a}, \emph{size=None}}{}
Draw samples from a Pareto II or Lomax distribution with specified shape.

The Lomax or Pareto II distribution is a shifted Pareto distribution. The
classical Pareto distribution can be obtained from the Lomax distribution
by adding the location parameter m, see below. The smallest value of the
Lomax distribution is zero while for the classical Pareto distribution it
is m, where the standard Pareto distribution has location m=1.
Lomax can also be considered as a simplified version of the Generalized
Pareto distribution (available in SciPy), with the scale set to one and
the location set to zero.

The Pareto distribution must be greater than zero, and is unbounded above.
It is also known as the ``80-20 rule''.  In this distribution, 80 percent of
the weights are in the lowest 20 percent of the range, while the other 20
percent fill the remaining 80 percent of the range.
\begin{description}
\item[{shape}] \leavevmode{[}float, \textgreater{} 0.{]}
Shape of the distribution.

\item[{size}] \leavevmode{[}tuple of ints{]}
Output shape.  If the given shape is, e.g., \code{(m, n, k)}, then
\code{m * n * k} samples are drawn.

\end{description}
\begin{description}
\item[{scipy.stats.distributions.lomax.pdf}] \leavevmode{[}probability density function,{]}
distribution or cumulative density function, etc.

\item[{scipy.stats.distributions.genpareto.pdf}] \leavevmode{[}probability density function,{]}
distribution or cumulative density function, etc.

\end{description}

The probability density for the Pareto distribution is
\begin{gather}
\begin{split}p(x) = \frac{am^a}{x^{a+1}}\end{split}\notag
\end{gather}
where \(a\) is the shape and \(m\) the location

The Pareto distribution, named after the Italian economist Vilfredo Pareto,
is a power law probability distribution useful in many real world problems.
Outside the field of economics it is generally referred to as the Bradford
distribution. Pareto developed the distribution to describe the
distribution of wealth in an economy.  It has also found use in insurance,
web page access statistics, oil field sizes, and many other problems,
including the download frequency for projects in Sourceforge {[}1{]}.  It is
one of the so-called ``fat-tailed'' distributions.

Draw samples from the distribution:

\begin{Verbatim}[commandchars=\\\{\}]
\PYG{g+gp}{\PYGZgt{}\PYGZgt{}\PYGZgt{} }\PYG{n}{a}\PYG{p}{,} \PYG{n}{m} \PYG{o}{=} \PYG{l+m+mf}{3.}\PYG{p}{,} \PYG{l+m+mf}{1.} \PYG{c}{\PYGZsh{} shape and mode}
\PYG{g+gp}{\PYGZgt{}\PYGZgt{}\PYGZgt{} }\PYG{n}{s} \PYG{o}{=} \PYG{n}{np}\PYG{o}{.}\PYG{n}{random}\PYG{o}{.}\PYG{n}{pareto}\PYG{p}{(}\PYG{n}{a}\PYG{p}{,} \PYG{l+m+mi}{1000}\PYG{p}{)} \PYG{o}{+} \PYG{n}{m}
\end{Verbatim}

Display the histogram of the samples, along with
the probability density function:

\begin{Verbatim}[commandchars=\\\{\}]
\PYG{g+gp}{\PYGZgt{}\PYGZgt{}\PYGZgt{} }\PYG{k+kn}{import} \PYG{n+nn}{matplotlib.pyplot} \PYG{k+kn}{as} \PYG{n+nn}{plt}
\PYG{g+gp}{\PYGZgt{}\PYGZgt{}\PYGZgt{} }\PYG{n}{count}\PYG{p}{,} \PYG{n}{bins}\PYG{p}{,} \PYG{n}{ignored} \PYG{o}{=} \PYG{n}{plt}\PYG{o}{.}\PYG{n}{hist}\PYG{p}{(}\PYG{n}{s}\PYG{p}{,} \PYG{l+m+mi}{100}\PYG{p}{,} \PYG{n}{normed}\PYG{o}{=}\PYG{n+nb+bp}{True}\PYG{p}{,} \PYG{n}{align}\PYG{o}{=}\PYG{l+s}{\PYGZsq{}}\PYG{l+s}{center}\PYG{l+s}{\PYGZsq{}}\PYG{p}{)}
\PYG{g+gp}{\PYGZgt{}\PYGZgt{}\PYGZgt{} }\PYG{n}{fit} \PYG{o}{=} \PYG{n}{a}\PYG{o}{*}\PYG{n}{m}\PYG{o}{*}\PYG{o}{*}\PYG{n}{a}\PYG{o}{/}\PYG{n}{bins}\PYG{o}{*}\PYG{o}{*}\PYG{p}{(}\PYG{n}{a}\PYG{o}{+}\PYG{l+m+mi}{1}\PYG{p}{)}
\PYG{g+gp}{\PYGZgt{}\PYGZgt{}\PYGZgt{} }\PYG{n}{plt}\PYG{o}{.}\PYG{n}{plot}\PYG{p}{(}\PYG{n}{bins}\PYG{p}{,} \PYG{n+nb}{max}\PYG{p}{(}\PYG{n}{count}\PYG{p}{)}\PYG{o}{*}\PYG{n}{fit}\PYG{o}{/}\PYG{n+nb}{max}\PYG{p}{(}\PYG{n}{fit}\PYG{p}{)}\PYG{p}{,}\PYG{n}{linewidth}\PYG{o}{=}\PYG{l+m+mi}{2}\PYG{p}{,} \PYG{n}{color}\PYG{o}{=}\PYG{l+s}{\PYGZsq{}}\PYG{l+s}{r}\PYG{l+s}{\PYGZsq{}}\PYG{p}{)}
\PYG{g+gp}{\PYGZgt{}\PYGZgt{}\PYGZgt{} }\PYG{n}{plt}\PYG{o}{.}\PYG{n}{show}\PYG{p}{(}\PYG{p}{)}
\end{Verbatim}

\end{fulllineitems}

\index{permutation() (in module acsSCCanalysis)}

\begin{fulllineitems}
\phantomsection\label{acsSCCanalysis:acsSCCanalysis.permutation}\pysiglinewithargsret{\code{acsSCCanalysis.}\bfcode{permutation}}{\emph{x}}{}
Randomly permute a sequence, or return a permuted range.

If \emph{x} is a multi-dimensional array, it is only shuffled along its
first index.
\begin{description}
\item[{x}] \leavevmode{[}int or array\_like{]}
If \emph{x} is an integer, randomly permute \code{np.arange(x)}.
If \emph{x} is an array, make a copy and shuffle the elements
randomly.

\end{description}
\begin{description}
\item[{out}] \leavevmode{[}ndarray{]}
Permuted sequence or array range.

\end{description}

\begin{Verbatim}[commandchars=\\\{\}]
\PYG{g+gp}{\PYGZgt{}\PYGZgt{}\PYGZgt{} }\PYG{n}{np}\PYG{o}{.}\PYG{n}{random}\PYG{o}{.}\PYG{n}{permutation}\PYG{p}{(}\PYG{l+m+mi}{10}\PYG{p}{)}
\PYG{g+go}{array([1, 7, 4, 3, 0, 9, 2, 5, 8, 6])}
\end{Verbatim}

\begin{Verbatim}[commandchars=\\\{\}]
\PYG{g+gp}{\PYGZgt{}\PYGZgt{}\PYGZgt{} }\PYG{n}{np}\PYG{o}{.}\PYG{n}{random}\PYG{o}{.}\PYG{n}{permutation}\PYG{p}{(}\PYG{p}{[}\PYG{l+m+mi}{1}\PYG{p}{,} \PYG{l+m+mi}{4}\PYG{p}{,} \PYG{l+m+mi}{9}\PYG{p}{,} \PYG{l+m+mi}{12}\PYG{p}{,} \PYG{l+m+mi}{15}\PYG{p}{]}\PYG{p}{)}
\PYG{g+go}{array([15,  1,  9,  4, 12])}
\end{Verbatim}

\begin{Verbatim}[commandchars=\\\{\}]
\PYG{g+gp}{\PYGZgt{}\PYGZgt{}\PYGZgt{} }\PYG{n}{arr} \PYG{o}{=} \PYG{n}{np}\PYG{o}{.}\PYG{n}{arange}\PYG{p}{(}\PYG{l+m+mi}{9}\PYG{p}{)}\PYG{o}{.}\PYG{n}{reshape}\PYG{p}{(}\PYG{p}{(}\PYG{l+m+mi}{3}\PYG{p}{,} \PYG{l+m+mi}{3}\PYG{p}{)}\PYG{p}{)}
\PYG{g+gp}{\PYGZgt{}\PYGZgt{}\PYGZgt{} }\PYG{n}{np}\PYG{o}{.}\PYG{n}{random}\PYG{o}{.}\PYG{n}{permutation}\PYG{p}{(}\PYG{n}{arr}\PYG{p}{)}
\PYG{g+go}{array([[6, 7, 8],}
\PYG{g+go}{       [0, 1, 2],}
\PYG{g+go}{       [3, 4, 5]])}
\end{Verbatim}

\end{fulllineitems}

\index{poisson() (in module acsSCCanalysis)}

\begin{fulllineitems}
\phantomsection\label{acsSCCanalysis:acsSCCanalysis.poisson}\pysiglinewithargsret{\code{acsSCCanalysis.}\bfcode{poisson}}{\emph{lam=1.0}, \emph{size=None}}{}
Draw samples from a Poisson distribution.

The Poisson distribution is the limit of the Binomial
distribution for large N.
\begin{description}
\item[{lam}] \leavevmode{[}float{]}
Expectation of interval, should be \textgreater{}= 0.

\item[{size}] \leavevmode{[}int or tuple of ints, optional{]}
Output shape. If the given shape is, e.g., \code{(m, n, k)}, then
\code{m * n * k} samples are drawn.

\end{description}

The Poisson distribution
\begin{gather}
\begin{split}f(k; \lambda)=\frac{\lambda^k e^{-\lambda}}{k!}\end{split}\notag
\end{gather}
For events with an expected separation \(\lambda\) the Poisson
distribution \(f(k; \lambda)\) describes the probability of
\(k\) events occurring within the observed interval \(\lambda\).

Because the output is limited to the range of the C long type, a
ValueError is raised when \emph{lam} is within 10 sigma of the maximum
representable value.

Draw samples from the distribution:

\begin{Verbatim}[commandchars=\\\{\}]
\PYG{g+gp}{\PYGZgt{}\PYGZgt{}\PYGZgt{} }\PYG{k+kn}{import} \PYG{n+nn}{numpy} \PYG{k+kn}{as} \PYG{n+nn}{np}
\PYG{g+gp}{\PYGZgt{}\PYGZgt{}\PYGZgt{} }\PYG{n}{s} \PYG{o}{=} \PYG{n}{np}\PYG{o}{.}\PYG{n}{random}\PYG{o}{.}\PYG{n}{poisson}\PYG{p}{(}\PYG{l+m+mi}{5}\PYG{p}{,} \PYG{l+m+mi}{10000}\PYG{p}{)}
\end{Verbatim}

Display histogram of the sample:

\begin{Verbatim}[commandchars=\\\{\}]
\PYG{g+gp}{\PYGZgt{}\PYGZgt{}\PYGZgt{} }\PYG{k+kn}{import} \PYG{n+nn}{matplotlib.pyplot} \PYG{k+kn}{as} \PYG{n+nn}{plt}
\PYG{g+gp}{\PYGZgt{}\PYGZgt{}\PYGZgt{} }\PYG{n}{count}\PYG{p}{,} \PYG{n}{bins}\PYG{p}{,} \PYG{n}{ignored} \PYG{o}{=} \PYG{n}{plt}\PYG{o}{.}\PYG{n}{hist}\PYG{p}{(}\PYG{n}{s}\PYG{p}{,} \PYG{l+m+mi}{14}\PYG{p}{,} \PYG{n}{normed}\PYG{o}{=}\PYG{n+nb+bp}{True}\PYG{p}{)}
\PYG{g+gp}{\PYGZgt{}\PYGZgt{}\PYGZgt{} }\PYG{n}{plt}\PYG{o}{.}\PYG{n}{show}\PYG{p}{(}\PYG{p}{)}
\end{Verbatim}

\end{fulllineitems}

\index{power() (in module acsSCCanalysis)}

\begin{fulllineitems}
\phantomsection\label{acsSCCanalysis:acsSCCanalysis.power}\pysiglinewithargsret{\code{acsSCCanalysis.}\bfcode{power}}{\emph{a}, \emph{size=None}}{}
Draws samples in {[}0, 1{]} from a power distribution with positive
exponent a - 1.

Also known as the power function distribution.
\begin{description}
\item[{a}] \leavevmode{[}float{]}
parameter, \textgreater{} 0

\item[{size}] \leavevmode{[}tuple of ints{]}\begin{description}
\item[{Output shape.  If the given shape is, e.g., \code{(m, n, k)}, then}] \leavevmode
\code{m * n * k} samples are drawn.

\end{description}

\end{description}
\begin{description}
\item[{samples}] \leavevmode{[}\{ndarray, scalar\}{]}
The returned samples lie in {[}0, 1{]}.

\end{description}
\begin{description}
\item[{ValueError}] \leavevmode
If a\textless{}1.

\end{description}

The probability density function is
\begin{gather}
\begin{split}P(x; a) = ax^{a-1}, 0 \le x \le 1, a>0.\end{split}\notag
\end{gather}
The power function distribution is just the inverse of the Pareto
distribution. It may also be seen as a special case of the Beta
distribution.

It is used, for example, in modeling the over-reporting of insurance
claims.

Draw samples from the distribution:

\begin{Verbatim}[commandchars=\\\{\}]
\PYG{g+gp}{\PYGZgt{}\PYGZgt{}\PYGZgt{} }\PYG{n}{a} \PYG{o}{=} \PYG{l+m+mf}{5.} \PYG{c}{\PYGZsh{} shape}
\PYG{g+gp}{\PYGZgt{}\PYGZgt{}\PYGZgt{} }\PYG{n}{samples} \PYG{o}{=} \PYG{l+m+mi}{1000}
\PYG{g+gp}{\PYGZgt{}\PYGZgt{}\PYGZgt{} }\PYG{n}{s} \PYG{o}{=} \PYG{n}{np}\PYG{o}{.}\PYG{n}{random}\PYG{o}{.}\PYG{n}{power}\PYG{p}{(}\PYG{n}{a}\PYG{p}{,} \PYG{n}{samples}\PYG{p}{)}
\end{Verbatim}

Display the histogram of the samples, along with
the probability density function:

\begin{Verbatim}[commandchars=\\\{\}]
\PYG{g+gp}{\PYGZgt{}\PYGZgt{}\PYGZgt{} }\PYG{k+kn}{import} \PYG{n+nn}{matplotlib.pyplot} \PYG{k+kn}{as} \PYG{n+nn}{plt}
\PYG{g+gp}{\PYGZgt{}\PYGZgt{}\PYGZgt{} }\PYG{n}{count}\PYG{p}{,} \PYG{n}{bins}\PYG{p}{,} \PYG{n}{ignored} \PYG{o}{=} \PYG{n}{plt}\PYG{o}{.}\PYG{n}{hist}\PYG{p}{(}\PYG{n}{s}\PYG{p}{,} \PYG{n}{bins}\PYG{o}{=}\PYG{l+m+mi}{30}\PYG{p}{)}
\PYG{g+gp}{\PYGZgt{}\PYGZgt{}\PYGZgt{} }\PYG{n}{x} \PYG{o}{=} \PYG{n}{np}\PYG{o}{.}\PYG{n}{linspace}\PYG{p}{(}\PYG{l+m+mi}{0}\PYG{p}{,} \PYG{l+m+mi}{1}\PYG{p}{,} \PYG{l+m+mi}{100}\PYG{p}{)}
\PYG{g+gp}{\PYGZgt{}\PYGZgt{}\PYGZgt{} }\PYG{n}{y} \PYG{o}{=} \PYG{n}{a}\PYG{o}{*}\PYG{n}{x}\PYG{o}{*}\PYG{o}{*}\PYG{p}{(}\PYG{n}{a}\PYG{o}{\PYGZhy{}}\PYG{l+m+mf}{1.}\PYG{p}{)}
\PYG{g+gp}{\PYGZgt{}\PYGZgt{}\PYGZgt{} }\PYG{n}{normed\PYGZus{}y} \PYG{o}{=} \PYG{n}{samples}\PYG{o}{*}\PYG{n}{np}\PYG{o}{.}\PYG{n}{diff}\PYG{p}{(}\PYG{n}{bins}\PYG{p}{)}\PYG{p}{[}\PYG{l+m+mi}{0}\PYG{p}{]}\PYG{o}{*}\PYG{n}{y}
\PYG{g+gp}{\PYGZgt{}\PYGZgt{}\PYGZgt{} }\PYG{n}{plt}\PYG{o}{.}\PYG{n}{plot}\PYG{p}{(}\PYG{n}{x}\PYG{p}{,} \PYG{n}{normed\PYGZus{}y}\PYG{p}{)}
\PYG{g+gp}{\PYGZgt{}\PYGZgt{}\PYGZgt{} }\PYG{n}{plt}\PYG{o}{.}\PYG{n}{show}\PYG{p}{(}\PYG{p}{)}
\end{Verbatim}

Compare the power function distribution to the inverse of the Pareto.

\begin{Verbatim}[commandchars=\\\{\}]
\PYG{g+gp}{\PYGZgt{}\PYGZgt{}\PYGZgt{} }\PYG{k+kn}{from} \PYG{n+nn}{scipy} \PYG{k+kn}{import} \PYG{n}{stats}
\PYG{g+gp}{\PYGZgt{}\PYGZgt{}\PYGZgt{} }\PYG{n}{rvs} \PYG{o}{=} \PYG{n}{np}\PYG{o}{.}\PYG{n}{random}\PYG{o}{.}\PYG{n}{power}\PYG{p}{(}\PYG{l+m+mi}{5}\PYG{p}{,} \PYG{l+m+mi}{1000000}\PYG{p}{)}
\PYG{g+gp}{\PYGZgt{}\PYGZgt{}\PYGZgt{} }\PYG{n}{rvsp} \PYG{o}{=} \PYG{n}{np}\PYG{o}{.}\PYG{n}{random}\PYG{o}{.}\PYG{n}{pareto}\PYG{p}{(}\PYG{l+m+mi}{5}\PYG{p}{,} \PYG{l+m+mi}{1000000}\PYG{p}{)}
\PYG{g+gp}{\PYGZgt{}\PYGZgt{}\PYGZgt{} }\PYG{n}{xx} \PYG{o}{=} \PYG{n}{np}\PYG{o}{.}\PYG{n}{linspace}\PYG{p}{(}\PYG{l+m+mi}{0}\PYG{p}{,}\PYG{l+m+mi}{1}\PYG{p}{,}\PYG{l+m+mi}{100}\PYG{p}{)}
\PYG{g+gp}{\PYGZgt{}\PYGZgt{}\PYGZgt{} }\PYG{n}{powpdf} \PYG{o}{=} \PYG{n}{stats}\PYG{o}{.}\PYG{n}{powerlaw}\PYG{o}{.}\PYG{n}{pdf}\PYG{p}{(}\PYG{n}{xx}\PYG{p}{,}\PYG{l+m+mi}{5}\PYG{p}{)}
\end{Verbatim}

\begin{Verbatim}[commandchars=\\\{\}]
\PYG{g+gp}{\PYGZgt{}\PYGZgt{}\PYGZgt{} }\PYG{n}{plt}\PYG{o}{.}\PYG{n}{figure}\PYG{p}{(}\PYG{p}{)}
\PYG{g+gp}{\PYGZgt{}\PYGZgt{}\PYGZgt{} }\PYG{n}{plt}\PYG{o}{.}\PYG{n}{hist}\PYG{p}{(}\PYG{n}{rvs}\PYG{p}{,} \PYG{n}{bins}\PYG{o}{=}\PYG{l+m+mi}{50}\PYG{p}{,} \PYG{n}{normed}\PYG{o}{=}\PYG{n+nb+bp}{True}\PYG{p}{)}
\PYG{g+gp}{\PYGZgt{}\PYGZgt{}\PYGZgt{} }\PYG{n}{plt}\PYG{o}{.}\PYG{n}{plot}\PYG{p}{(}\PYG{n}{xx}\PYG{p}{,}\PYG{n}{powpdf}\PYG{p}{,}\PYG{l+s}{\PYGZsq{}}\PYG{l+s}{r\PYGZhy{}}\PYG{l+s}{\PYGZsq{}}\PYG{p}{)}
\PYG{g+gp}{\PYGZgt{}\PYGZgt{}\PYGZgt{} }\PYG{n}{plt}\PYG{o}{.}\PYG{n}{title}\PYG{p}{(}\PYG{l+s}{\PYGZsq{}}\PYG{l+s}{np.random.power(5)}\PYG{l+s}{\PYGZsq{}}\PYG{p}{)}
\end{Verbatim}

\begin{Verbatim}[commandchars=\\\{\}]
\PYG{g+gp}{\PYGZgt{}\PYGZgt{}\PYGZgt{} }\PYG{n}{plt}\PYG{o}{.}\PYG{n}{figure}\PYG{p}{(}\PYG{p}{)}
\PYG{g+gp}{\PYGZgt{}\PYGZgt{}\PYGZgt{} }\PYG{n}{plt}\PYG{o}{.}\PYG{n}{hist}\PYG{p}{(}\PYG{l+m+mf}{1.}\PYG{o}{/}\PYG{p}{(}\PYG{l+m+mf}{1.}\PYG{o}{+}\PYG{n}{rvsp}\PYG{p}{)}\PYG{p}{,} \PYG{n}{bins}\PYG{o}{=}\PYG{l+m+mi}{50}\PYG{p}{,} \PYG{n}{normed}\PYG{o}{=}\PYG{n+nb+bp}{True}\PYG{p}{)}
\PYG{g+gp}{\PYGZgt{}\PYGZgt{}\PYGZgt{} }\PYG{n}{plt}\PYG{o}{.}\PYG{n}{plot}\PYG{p}{(}\PYG{n}{xx}\PYG{p}{,}\PYG{n}{powpdf}\PYG{p}{,}\PYG{l+s}{\PYGZsq{}}\PYG{l+s}{r\PYGZhy{}}\PYG{l+s}{\PYGZsq{}}\PYG{p}{)}
\PYG{g+gp}{\PYGZgt{}\PYGZgt{}\PYGZgt{} }\PYG{n}{plt}\PYG{o}{.}\PYG{n}{title}\PYG{p}{(}\PYG{l+s}{\PYGZsq{}}\PYG{l+s}{inverse of 1 + np.random.pareto(5)}\PYG{l+s}{\PYGZsq{}}\PYG{p}{)}
\end{Verbatim}

\begin{Verbatim}[commandchars=\\\{\}]
\PYG{g+gp}{\PYGZgt{}\PYGZgt{}\PYGZgt{} }\PYG{n}{plt}\PYG{o}{.}\PYG{n}{figure}\PYG{p}{(}\PYG{p}{)}
\PYG{g+gp}{\PYGZgt{}\PYGZgt{}\PYGZgt{} }\PYG{n}{plt}\PYG{o}{.}\PYG{n}{hist}\PYG{p}{(}\PYG{l+m+mf}{1.}\PYG{o}{/}\PYG{p}{(}\PYG{l+m+mf}{1.}\PYG{o}{+}\PYG{n}{rvsp}\PYG{p}{)}\PYG{p}{,} \PYG{n}{bins}\PYG{o}{=}\PYG{l+m+mi}{50}\PYG{p}{,} \PYG{n}{normed}\PYG{o}{=}\PYG{n+nb+bp}{True}\PYG{p}{)}
\PYG{g+gp}{\PYGZgt{}\PYGZgt{}\PYGZgt{} }\PYG{n}{plt}\PYG{o}{.}\PYG{n}{plot}\PYG{p}{(}\PYG{n}{xx}\PYG{p}{,}\PYG{n}{powpdf}\PYG{p}{,}\PYG{l+s}{\PYGZsq{}}\PYG{l+s}{r\PYGZhy{}}\PYG{l+s}{\PYGZsq{}}\PYG{p}{)}
\PYG{g+gp}{\PYGZgt{}\PYGZgt{}\PYGZgt{} }\PYG{n}{plt}\PYG{o}{.}\PYG{n}{title}\PYG{p}{(}\PYG{l+s}{\PYGZsq{}}\PYG{l+s}{inverse of stats.pareto(5)}\PYG{l+s}{\PYGZsq{}}\PYG{p}{)}
\end{Verbatim}

\end{fulllineitems}

\index{rand() (in module acsSCCanalysis)}

\begin{fulllineitems}
\phantomsection\label{acsSCCanalysis:acsSCCanalysis.rand}\pysiglinewithargsret{\code{acsSCCanalysis.}\bfcode{rand}}{\emph{d0}, \emph{d1}, \emph{...}, \emph{dn}}{}
Random values in a given shape.

Create an array of the given shape and propagate it with
random samples from a uniform distribution
over \code{{[}0, 1)}.
\begin{description}
\item[{d0, d1, ..., dn}] \leavevmode{[}int, optional{]}
The dimensions of the returned array, should all be positive.
If no argument is given a single Python float is returned.

\end{description}
\begin{description}
\item[{out}] \leavevmode{[}ndarray, shape \code{(d0, d1, ..., dn)}{]}
Random values.

\end{description}

random

This is a convenience function. If you want an interface that
takes a shape-tuple as the first argument, refer to
np.random.random\_sample .

\begin{Verbatim}[commandchars=\\\{\}]
\PYG{g+gp}{\PYGZgt{}\PYGZgt{}\PYGZgt{} }\PYG{n}{np}\PYG{o}{.}\PYG{n}{random}\PYG{o}{.}\PYG{n}{rand}\PYG{p}{(}\PYG{l+m+mi}{3}\PYG{p}{,}\PYG{l+m+mi}{2}\PYG{p}{)}
\PYG{g+go}{array([[ 0.14022471,  0.96360618],  \PYGZsh{}random}
\PYG{g+go}{       [ 0.37601032,  0.25528411],  \PYGZsh{}random}
\PYG{g+go}{       [ 0.49313049,  0.94909878]]) \PYGZsh{}random}
\end{Verbatim}

\end{fulllineitems}

\index{randint() (in module acsSCCanalysis)}

\begin{fulllineitems}
\phantomsection\label{acsSCCanalysis:acsSCCanalysis.randint}\pysiglinewithargsret{\code{acsSCCanalysis.}\bfcode{randint}}{\emph{low}, \emph{high=None}, \emph{size=None}}{}
Return random integers from \emph{low} (inclusive) to \emph{high} (exclusive).

Return random integers from the ``discrete uniform'' distribution in the
``half-open'' interval {[}\emph{low}, \emph{high}). If \emph{high} is None (the default),
then results are from {[}0, \emph{low}).
\begin{description}
\item[{low}] \leavevmode{[}int{]}
Lowest (signed) integer to be drawn from the distribution (unless
\code{high=None}, in which case this parameter is the \emph{highest} such
integer).

\item[{high}] \leavevmode{[}int, optional{]}
If provided, one above the largest (signed) integer to be drawn
from the distribution (see above for behavior if \code{high=None}).

\item[{size}] \leavevmode{[}int or tuple of ints, optional{]}
Output shape. Default is None, in which case a single int is
returned.

\end{description}
\begin{description}
\item[{out}] \leavevmode{[}int or ndarray of ints{]}
\emph{size}-shaped array of random integers from the appropriate
distribution, or a single such random int if \emph{size} not provided.

\end{description}
\begin{description}
\item[{random.random\_integers}] \leavevmode{[}similar to \emph{randint}, only for the closed{]}
interval {[}\emph{low}, \emph{high}{]}, and 1 is the lowest value if \emph{high} is
omitted. In particular, this other one is the one to use to generate
uniformly distributed discrete non-integers.

\end{description}

\begin{Verbatim}[commandchars=\\\{\}]
\PYG{g+gp}{\PYGZgt{}\PYGZgt{}\PYGZgt{} }\PYG{n}{np}\PYG{o}{.}\PYG{n}{random}\PYG{o}{.}\PYG{n}{randint}\PYG{p}{(}\PYG{l+m+mi}{2}\PYG{p}{,} \PYG{n}{size}\PYG{o}{=}\PYG{l+m+mi}{10}\PYG{p}{)}
\PYG{g+go}{array([1, 0, 0, 0, 1, 1, 0, 0, 1, 0])}
\PYG{g+gp}{\PYGZgt{}\PYGZgt{}\PYGZgt{} }\PYG{n}{np}\PYG{o}{.}\PYG{n}{random}\PYG{o}{.}\PYG{n}{randint}\PYG{p}{(}\PYG{l+m+mi}{1}\PYG{p}{,} \PYG{n}{size}\PYG{o}{=}\PYG{l+m+mi}{10}\PYG{p}{)}
\PYG{g+go}{array([0, 0, 0, 0, 0, 0, 0, 0, 0, 0])}
\end{Verbatim}

Generate a 2 x 4 array of ints between 0 and 4, inclusive:

\begin{Verbatim}[commandchars=\\\{\}]
\PYG{g+gp}{\PYGZgt{}\PYGZgt{}\PYGZgt{} }\PYG{n}{np}\PYG{o}{.}\PYG{n}{random}\PYG{o}{.}\PYG{n}{randint}\PYG{p}{(}\PYG{l+m+mi}{5}\PYG{p}{,} \PYG{n}{size}\PYG{o}{=}\PYG{p}{(}\PYG{l+m+mi}{2}\PYG{p}{,} \PYG{l+m+mi}{4}\PYG{p}{)}\PYG{p}{)}
\PYG{g+go}{array([[4, 0, 2, 1],}
\PYG{g+go}{       [3, 2, 2, 0]])}
\end{Verbatim}

\end{fulllineitems}

\index{randn() (in module acsSCCanalysis)}

\begin{fulllineitems}
\phantomsection\label{acsSCCanalysis:acsSCCanalysis.randn}\pysiglinewithargsret{\code{acsSCCanalysis.}\bfcode{randn}}{\emph{d0}, \emph{d1}, \emph{...}, \emph{dn}}{}
Return a sample (or samples) from the ``standard normal'' distribution.

If positive, int\_like or int-convertible arguments are provided,
\emph{randn} generates an array of shape \code{(d0, d1, ..., dn)}, filled
with random floats sampled from a univariate ``normal'' (Gaussian)
distribution of mean 0 and variance 1 (if any of the \(d_i\) are
floats, they are first converted to integers by truncation). A single
float randomly sampled from the distribution is returned if no
argument is provided.

This is a convenience function.  If you want an interface that takes a
tuple as the first argument, use \emph{numpy.random.standard\_normal} instead.
\begin{description}
\item[{d0, d1, ..., dn}] \leavevmode{[}int, optional{]}
The dimensions of the returned array, should be all positive.
If no argument is given a single Python float is returned.

\end{description}
\begin{description}
\item[{Z}] \leavevmode{[}ndarray or float{]}
A \code{(d0, d1, ..., dn)}-shaped array of floating-point samples from
the standard normal distribution, or a single such float if
no parameters were supplied.

\end{description}

random.standard\_normal : Similar, but takes a tuple as its argument.

For random samples from \(N(\mu, \sigma^2)\), use:

\code{sigma * np.random.randn(...) + mu}

\begin{Verbatim}[commandchars=\\\{\}]
\PYG{g+gp}{\PYGZgt{}\PYGZgt{}\PYGZgt{} }\PYG{n}{np}\PYG{o}{.}\PYG{n}{random}\PYG{o}{.}\PYG{n}{randn}\PYG{p}{(}\PYG{p}{)}
\PYG{g+go}{2.1923875335537315 \PYGZsh{}random}
\end{Verbatim}

Two-by-four array of samples from N(3, 6.25):

\begin{Verbatim}[commandchars=\\\{\}]
\PYG{g+gp}{\PYGZgt{}\PYGZgt{}\PYGZgt{} }\PYG{l+m+mf}{2.5} \PYG{o}{*} \PYG{n}{np}\PYG{o}{.}\PYG{n}{random}\PYG{o}{.}\PYG{n}{randn}\PYG{p}{(}\PYG{l+m+mi}{2}\PYG{p}{,} \PYG{l+m+mi}{4}\PYG{p}{)} \PYG{o}{+} \PYG{l+m+mi}{3}
\PYG{g+go}{array([[\PYGZhy{}4.49401501,  4.00950034, \PYGZhy{}1.81814867,  7.29718677],  \PYGZsh{}random}
\PYG{g+go}{       [ 0.39924804,  4.68456316,  4.99394529,  4.84057254]]) \PYGZsh{}random}
\end{Verbatim}

\end{fulllineitems}

\index{random() (in module acsSCCanalysis)}

\begin{fulllineitems}
\phantomsection\label{acsSCCanalysis:acsSCCanalysis.random}\pysiglinewithargsret{\code{acsSCCanalysis.}\bfcode{random}}{}{}
random\_sample(size=None)

Return random floats in the half-open interval {[}0.0, 1.0).

Results are from the ``continuous uniform'' distribution over the
stated interval.  To sample \(Unif[a, b), b > a\) multiply
the output of \emph{random\_sample} by \emph{(b-a)} and add \emph{a}:

\begin{Verbatim}[commandchars=\\\{\}]
\PYG{p}{(}\PYG{n}{b} \PYG{o}{\PYGZhy{}} \PYG{n}{a}\PYG{p}{)} \PYG{o}{*} \PYG{n}{random\PYGZus{}sample}\PYG{p}{(}\PYG{p}{)} \PYG{o}{+} \PYG{n}{a}
\end{Verbatim}
\begin{description}
\item[{size}] \leavevmode{[}int or tuple of ints, optional{]}
Defines the shape of the returned array of random floats. If None
(the default), returns a single float.

\end{description}
\begin{description}
\item[{out}] \leavevmode{[}float or ndarray of floats{]}
Array of random floats of shape \emph{size} (unless \code{size=None}, in which
case a single float is returned).

\end{description}

\begin{Verbatim}[commandchars=\\\{\}]
\PYG{g+gp}{\PYGZgt{}\PYGZgt{}\PYGZgt{} }\PYG{n}{np}\PYG{o}{.}\PYG{n}{random}\PYG{o}{.}\PYG{n}{random\PYGZus{}sample}\PYG{p}{(}\PYG{p}{)}
\PYG{g+go}{0.47108547995356098}
\PYG{g+gp}{\PYGZgt{}\PYGZgt{}\PYGZgt{} }\PYG{n+nb}{type}\PYG{p}{(}\PYG{n}{np}\PYG{o}{.}\PYG{n}{random}\PYG{o}{.}\PYG{n}{random\PYGZus{}sample}\PYG{p}{(}\PYG{p}{)}\PYG{p}{)}
\PYG{g+go}{\PYGZlt{}type \PYGZsq{}float\PYGZsq{}\PYGZgt{}}
\PYG{g+gp}{\PYGZgt{}\PYGZgt{}\PYGZgt{} }\PYG{n}{np}\PYG{o}{.}\PYG{n}{random}\PYG{o}{.}\PYG{n}{random\PYGZus{}sample}\PYG{p}{(}\PYG{p}{(}\PYG{l+m+mi}{5}\PYG{p}{,}\PYG{p}{)}\PYG{p}{)}
\PYG{g+go}{array([ 0.30220482,  0.86820401,  0.1654503 ,  0.11659149,  0.54323428])}
\end{Verbatim}

Three-by-two array of random numbers from {[}-5, 0):

\begin{Verbatim}[commandchars=\\\{\}]
\PYG{g+gp}{\PYGZgt{}\PYGZgt{}\PYGZgt{} }\PYG{l+m+mi}{5} \PYG{o}{*} \PYG{n}{np}\PYG{o}{.}\PYG{n}{random}\PYG{o}{.}\PYG{n}{random\PYGZus{}sample}\PYG{p}{(}\PYG{p}{(}\PYG{l+m+mi}{3}\PYG{p}{,} \PYG{l+m+mi}{2}\PYG{p}{)}\PYG{p}{)} \PYG{o}{\PYGZhy{}} \PYG{l+m+mi}{5}
\PYG{g+go}{array([[\PYGZhy{}3.99149989, \PYGZhy{}0.52338984],}
\PYG{g+go}{       [\PYGZhy{}2.99091858, \PYGZhy{}0.79479508],}
\PYG{g+go}{       [\PYGZhy{}1.23204345, \PYGZhy{}1.75224494]])}
\end{Verbatim}

\end{fulllineitems}

\index{random\_integers() (in module acsSCCanalysis)}

\begin{fulllineitems}
\phantomsection\label{acsSCCanalysis:acsSCCanalysis.random_integers}\pysiglinewithargsret{\code{acsSCCanalysis.}\bfcode{random\_integers}}{\emph{low}, \emph{high=None}, \emph{size=None}}{}
Return random integers between \emph{low} and \emph{high}, inclusive.

Return random integers from the ``discrete uniform'' distribution in the
closed interval {[}\emph{low}, \emph{high}{]}.  If \emph{high} is None (the default),
then results are from {[}1, \emph{low}{]}.
\begin{description}
\item[{low}] \leavevmode{[}int{]}
Lowest (signed) integer to be drawn from the distribution (unless
\code{high=None}, in which case this parameter is the \emph{highest} such
integer).

\item[{high}] \leavevmode{[}int, optional{]}
If provided, the largest (signed) integer to be drawn from the
distribution (see above for behavior if \code{high=None}).

\item[{size}] \leavevmode{[}int or tuple of ints, optional{]}
Output shape. Default is None, in which case a single int is returned.

\end{description}
\begin{description}
\item[{out}] \leavevmode{[}int or ndarray of ints{]}
\emph{size}-shaped array of random integers from the appropriate
distribution, or a single such random int if \emph{size} not provided.

\end{description}
\begin{description}
\item[{random.randint}] \leavevmode{[}Similar to \emph{random\_integers}, only for the half-open{]}
interval {[}\emph{low}, \emph{high}), and 0 is the lowest value if \emph{high} is
omitted.

\end{description}

To sample from N evenly spaced floating-point numbers between a and b,
use:

\begin{Verbatim}[commandchars=\\\{\}]
\PYG{n}{a} \PYG{o}{+} \PYG{p}{(}\PYG{n}{b} \PYG{o}{\PYGZhy{}} \PYG{n}{a}\PYG{p}{)} \PYG{o}{*} \PYG{p}{(}\PYG{n}{np}\PYG{o}{.}\PYG{n}{random}\PYG{o}{.}\PYG{n}{random\PYGZus{}integers}\PYG{p}{(}\PYG{n}{N}\PYG{p}{)} \PYG{o}{\PYGZhy{}} \PYG{l+m+mi}{1}\PYG{p}{)} \PYG{o}{/} \PYG{p}{(}\PYG{n}{N} \PYG{o}{\PYGZhy{}} \PYG{l+m+mf}{1.}\PYG{p}{)}
\end{Verbatim}

\begin{Verbatim}[commandchars=\\\{\}]
\PYG{g+gp}{\PYGZgt{}\PYGZgt{}\PYGZgt{} }\PYG{n}{np}\PYG{o}{.}\PYG{n}{random}\PYG{o}{.}\PYG{n}{random\PYGZus{}integers}\PYG{p}{(}\PYG{l+m+mi}{5}\PYG{p}{)}
\PYG{g+go}{4}
\PYG{g+gp}{\PYGZgt{}\PYGZgt{}\PYGZgt{} }\PYG{n+nb}{type}\PYG{p}{(}\PYG{n}{np}\PYG{o}{.}\PYG{n}{random}\PYG{o}{.}\PYG{n}{random\PYGZus{}integers}\PYG{p}{(}\PYG{l+m+mi}{5}\PYG{p}{)}\PYG{p}{)}
\PYG{g+go}{\PYGZlt{}type \PYGZsq{}int\PYGZsq{}\PYGZgt{}}
\PYG{g+gp}{\PYGZgt{}\PYGZgt{}\PYGZgt{} }\PYG{n}{np}\PYG{o}{.}\PYG{n}{random}\PYG{o}{.}\PYG{n}{random\PYGZus{}integers}\PYG{p}{(}\PYG{l+m+mi}{5}\PYG{p}{,} \PYG{n}{size}\PYG{o}{=}\PYG{p}{(}\PYG{l+m+mf}{3.}\PYG{p}{,}\PYG{l+m+mf}{2.}\PYG{p}{)}\PYG{p}{)}
\PYG{g+go}{array([[5, 4],}
\PYG{g+go}{       [3, 3],}
\PYG{g+go}{       [4, 5]])}
\end{Verbatim}

Choose five random numbers from the set of five evenly-spaced
numbers between 0 and 2.5, inclusive (\emph{i.e.}, from the set
\({0, 5/8, 10/8, 15/8, 20/8}\)):

\begin{Verbatim}[commandchars=\\\{\}]
\PYG{g+gp}{\PYGZgt{}\PYGZgt{}\PYGZgt{} }\PYG{l+m+mf}{2.5} \PYG{o}{*} \PYG{p}{(}\PYG{n}{np}\PYG{o}{.}\PYG{n}{random}\PYG{o}{.}\PYG{n}{random\PYGZus{}integers}\PYG{p}{(}\PYG{l+m+mi}{5}\PYG{p}{,} \PYG{n}{size}\PYG{o}{=}\PYG{p}{(}\PYG{l+m+mi}{5}\PYG{p}{,}\PYG{p}{)}\PYG{p}{)} \PYG{o}{\PYGZhy{}} \PYG{l+m+mi}{1}\PYG{p}{)} \PYG{o}{/} \PYG{l+m+mf}{4.}
\PYG{g+go}{array([ 0.625,  1.25 ,  0.625,  0.625,  2.5  ])}
\end{Verbatim}

Roll two six sided dice 1000 times and sum the results:

\begin{Verbatim}[commandchars=\\\{\}]
\PYG{g+gp}{\PYGZgt{}\PYGZgt{}\PYGZgt{} }\PYG{n}{d1} \PYG{o}{=} \PYG{n}{np}\PYG{o}{.}\PYG{n}{random}\PYG{o}{.}\PYG{n}{random\PYGZus{}integers}\PYG{p}{(}\PYG{l+m+mi}{1}\PYG{p}{,} \PYG{l+m+mi}{6}\PYG{p}{,} \PYG{l+m+mi}{1000}\PYG{p}{)}
\PYG{g+gp}{\PYGZgt{}\PYGZgt{}\PYGZgt{} }\PYG{n}{d2} \PYG{o}{=} \PYG{n}{np}\PYG{o}{.}\PYG{n}{random}\PYG{o}{.}\PYG{n}{random\PYGZus{}integers}\PYG{p}{(}\PYG{l+m+mi}{1}\PYG{p}{,} \PYG{l+m+mi}{6}\PYG{p}{,} \PYG{l+m+mi}{1000}\PYG{p}{)}
\PYG{g+gp}{\PYGZgt{}\PYGZgt{}\PYGZgt{} }\PYG{n}{dsums} \PYG{o}{=} \PYG{n}{d1} \PYG{o}{+} \PYG{n}{d2}
\end{Verbatim}

Display results as a histogram:

\begin{Verbatim}[commandchars=\\\{\}]
\PYG{g+gp}{\PYGZgt{}\PYGZgt{}\PYGZgt{} }\PYG{k+kn}{import} \PYG{n+nn}{matplotlib.pyplot} \PYG{k+kn}{as} \PYG{n+nn}{plt}
\PYG{g+gp}{\PYGZgt{}\PYGZgt{}\PYGZgt{} }\PYG{n}{count}\PYG{p}{,} \PYG{n}{bins}\PYG{p}{,} \PYG{n}{ignored} \PYG{o}{=} \PYG{n}{plt}\PYG{o}{.}\PYG{n}{hist}\PYG{p}{(}\PYG{n}{dsums}\PYG{p}{,} \PYG{l+m+mi}{11}\PYG{p}{,} \PYG{n}{normed}\PYG{o}{=}\PYG{n+nb+bp}{True}\PYG{p}{)}
\PYG{g+gp}{\PYGZgt{}\PYGZgt{}\PYGZgt{} }\PYG{n}{plt}\PYG{o}{.}\PYG{n}{show}\PYG{p}{(}\PYG{p}{)}
\end{Verbatim}

\end{fulllineitems}

\index{random\_sample() (in module acsSCCanalysis)}

\begin{fulllineitems}
\phantomsection\label{acsSCCanalysis:acsSCCanalysis.random_sample}\pysiglinewithargsret{\code{acsSCCanalysis.}\bfcode{random\_sample}}{\emph{size=None}}{}
Return random floats in the half-open interval {[}0.0, 1.0).

Results are from the ``continuous uniform'' distribution over the
stated interval.  To sample \(Unif[a, b), b > a\) multiply
the output of \emph{random\_sample} by \emph{(b-a)} and add \emph{a}:

\begin{Verbatim}[commandchars=\\\{\}]
\PYG{p}{(}\PYG{n}{b} \PYG{o}{\PYGZhy{}} \PYG{n}{a}\PYG{p}{)} \PYG{o}{*} \PYG{n}{random\PYGZus{}sample}\PYG{p}{(}\PYG{p}{)} \PYG{o}{+} \PYG{n}{a}
\end{Verbatim}
\begin{description}
\item[{size}] \leavevmode{[}int or tuple of ints, optional{]}
Defines the shape of the returned array of random floats. If None
(the default), returns a single float.

\end{description}
\begin{description}
\item[{out}] \leavevmode{[}float or ndarray of floats{]}
Array of random floats of shape \emph{size} (unless \code{size=None}, in which
case a single float is returned).

\end{description}

\begin{Verbatim}[commandchars=\\\{\}]
\PYG{g+gp}{\PYGZgt{}\PYGZgt{}\PYGZgt{} }\PYG{n}{np}\PYG{o}{.}\PYG{n}{random}\PYG{o}{.}\PYG{n}{random\PYGZus{}sample}\PYG{p}{(}\PYG{p}{)}
\PYG{g+go}{0.47108547995356098}
\PYG{g+gp}{\PYGZgt{}\PYGZgt{}\PYGZgt{} }\PYG{n+nb}{type}\PYG{p}{(}\PYG{n}{np}\PYG{o}{.}\PYG{n}{random}\PYG{o}{.}\PYG{n}{random\PYGZus{}sample}\PYG{p}{(}\PYG{p}{)}\PYG{p}{)}
\PYG{g+go}{\PYGZlt{}type \PYGZsq{}float\PYGZsq{}\PYGZgt{}}
\PYG{g+gp}{\PYGZgt{}\PYGZgt{}\PYGZgt{} }\PYG{n}{np}\PYG{o}{.}\PYG{n}{random}\PYG{o}{.}\PYG{n}{random\PYGZus{}sample}\PYG{p}{(}\PYG{p}{(}\PYG{l+m+mi}{5}\PYG{p}{,}\PYG{p}{)}\PYG{p}{)}
\PYG{g+go}{array([ 0.30220482,  0.86820401,  0.1654503 ,  0.11659149,  0.54323428])}
\end{Verbatim}

Three-by-two array of random numbers from {[}-5, 0):

\begin{Verbatim}[commandchars=\\\{\}]
\PYG{g+gp}{\PYGZgt{}\PYGZgt{}\PYGZgt{} }\PYG{l+m+mi}{5} \PYG{o}{*} \PYG{n}{np}\PYG{o}{.}\PYG{n}{random}\PYG{o}{.}\PYG{n}{random\PYGZus{}sample}\PYG{p}{(}\PYG{p}{(}\PYG{l+m+mi}{3}\PYG{p}{,} \PYG{l+m+mi}{2}\PYG{p}{)}\PYG{p}{)} \PYG{o}{\PYGZhy{}} \PYG{l+m+mi}{5}
\PYG{g+go}{array([[\PYGZhy{}3.99149989, \PYGZhy{}0.52338984],}
\PYG{g+go}{       [\PYGZhy{}2.99091858, \PYGZhy{}0.79479508],}
\PYG{g+go}{       [\PYGZhy{}1.23204345, \PYGZhy{}1.75224494]])}
\end{Verbatim}

\end{fulllineitems}

\index{ranf() (in module acsSCCanalysis)}

\begin{fulllineitems}
\phantomsection\label{acsSCCanalysis:acsSCCanalysis.ranf}\pysiglinewithargsret{\code{acsSCCanalysis.}\bfcode{ranf}}{}{}
random\_sample(size=None)

Return random floats in the half-open interval {[}0.0, 1.0).

Results are from the ``continuous uniform'' distribution over the
stated interval.  To sample \(Unif[a, b), b > a\) multiply
the output of \emph{random\_sample} by \emph{(b-a)} and add \emph{a}:

\begin{Verbatim}[commandchars=\\\{\}]
\PYG{p}{(}\PYG{n}{b} \PYG{o}{\PYGZhy{}} \PYG{n}{a}\PYG{p}{)} \PYG{o}{*} \PYG{n}{random\PYGZus{}sample}\PYG{p}{(}\PYG{p}{)} \PYG{o}{+} \PYG{n}{a}
\end{Verbatim}
\begin{description}
\item[{size}] \leavevmode{[}int or tuple of ints, optional{]}
Defines the shape of the returned array of random floats. If None
(the default), returns a single float.

\end{description}
\begin{description}
\item[{out}] \leavevmode{[}float or ndarray of floats{]}
Array of random floats of shape \emph{size} (unless \code{size=None}, in which
case a single float is returned).

\end{description}

\begin{Verbatim}[commandchars=\\\{\}]
\PYG{g+gp}{\PYGZgt{}\PYGZgt{}\PYGZgt{} }\PYG{n}{np}\PYG{o}{.}\PYG{n}{random}\PYG{o}{.}\PYG{n}{random\PYGZus{}sample}\PYG{p}{(}\PYG{p}{)}
\PYG{g+go}{0.47108547995356098}
\PYG{g+gp}{\PYGZgt{}\PYGZgt{}\PYGZgt{} }\PYG{n+nb}{type}\PYG{p}{(}\PYG{n}{np}\PYG{o}{.}\PYG{n}{random}\PYG{o}{.}\PYG{n}{random\PYGZus{}sample}\PYG{p}{(}\PYG{p}{)}\PYG{p}{)}
\PYG{g+go}{\PYGZlt{}type \PYGZsq{}float\PYGZsq{}\PYGZgt{}}
\PYG{g+gp}{\PYGZgt{}\PYGZgt{}\PYGZgt{} }\PYG{n}{np}\PYG{o}{.}\PYG{n}{random}\PYG{o}{.}\PYG{n}{random\PYGZus{}sample}\PYG{p}{(}\PYG{p}{(}\PYG{l+m+mi}{5}\PYG{p}{,}\PYG{p}{)}\PYG{p}{)}
\PYG{g+go}{array([ 0.30220482,  0.86820401,  0.1654503 ,  0.11659149,  0.54323428])}
\end{Verbatim}

Three-by-two array of random numbers from {[}-5, 0):

\begin{Verbatim}[commandchars=\\\{\}]
\PYG{g+gp}{\PYGZgt{}\PYGZgt{}\PYGZgt{} }\PYG{l+m+mi}{5} \PYG{o}{*} \PYG{n}{np}\PYG{o}{.}\PYG{n}{random}\PYG{o}{.}\PYG{n}{random\PYGZus{}sample}\PYG{p}{(}\PYG{p}{(}\PYG{l+m+mi}{3}\PYG{p}{,} \PYG{l+m+mi}{2}\PYG{p}{)}\PYG{p}{)} \PYG{o}{\PYGZhy{}} \PYG{l+m+mi}{5}
\PYG{g+go}{array([[\PYGZhy{}3.99149989, \PYGZhy{}0.52338984],}
\PYG{g+go}{       [\PYGZhy{}2.99091858, \PYGZhy{}0.79479508],}
\PYG{g+go}{       [\PYGZhy{}1.23204345, \PYGZhy{}1.75224494]])}
\end{Verbatim}

\end{fulllineitems}

\index{rayleigh() (in module acsSCCanalysis)}

\begin{fulllineitems}
\phantomsection\label{acsSCCanalysis:acsSCCanalysis.rayleigh}\pysiglinewithargsret{\code{acsSCCanalysis.}\bfcode{rayleigh}}{\emph{scale=1.0}, \emph{size=None}}{}
Draw samples from a Rayleigh distribution.

The \(\chi\) and Weibull distributions are generalizations of the
Rayleigh.
\begin{description}
\item[{scale}] \leavevmode{[}scalar{]}
Scale, also equals the mode. Should be \textgreater{}= 0.

\item[{size}] \leavevmode{[}int or tuple of ints, optional{]}
Shape of the output. Default is None, in which case a single
value is returned.

\end{description}

The probability density function for the Rayleigh distribution is
\begin{gather}
\begin{split}P(x;scale) = \frac{x}{scale^2}e^{\frac{-x^2}{2 \cdotp scale^2}}\end{split}\notag
\end{gather}
The Rayleigh distribution arises if the wind speed and wind direction are
both gaussian variables, then the vector wind velocity forms a Rayleigh
distribution. The Rayleigh distribution is used to model the expected
output from wind turbines.

Draw values from the distribution and plot the histogram

\begin{Verbatim}[commandchars=\\\{\}]
\PYG{g+gp}{\PYGZgt{}\PYGZgt{}\PYGZgt{} }\PYG{n}{values} \PYG{o}{=} \PYG{n}{hist}\PYG{p}{(}\PYG{n}{np}\PYG{o}{.}\PYG{n}{random}\PYG{o}{.}\PYG{n}{rayleigh}\PYG{p}{(}\PYG{l+m+mi}{3}\PYG{p}{,} \PYG{l+m+mi}{100000}\PYG{p}{)}\PYG{p}{,} \PYG{n}{bins}\PYG{o}{=}\PYG{l+m+mi}{200}\PYG{p}{,} \PYG{n}{normed}\PYG{o}{=}\PYG{n+nb+bp}{True}\PYG{p}{)}
\end{Verbatim}

Wave heights tend to follow a Rayleigh distribution. If the mean wave
height is 1 meter, what fraction of waves are likely to be larger than 3
meters?

\begin{Verbatim}[commandchars=\\\{\}]
\PYG{g+gp}{\PYGZgt{}\PYGZgt{}\PYGZgt{} }\PYG{n}{meanvalue} \PYG{o}{=} \PYG{l+m+mi}{1}
\PYG{g+gp}{\PYGZgt{}\PYGZgt{}\PYGZgt{} }\PYG{n}{modevalue} \PYG{o}{=} \PYG{n}{np}\PYG{o}{.}\PYG{n}{sqrt}\PYG{p}{(}\PYG{l+m+mi}{2} \PYG{o}{/} \PYG{n}{np}\PYG{o}{.}\PYG{n}{pi}\PYG{p}{)} \PYG{o}{*} \PYG{n}{meanvalue}
\PYG{g+gp}{\PYGZgt{}\PYGZgt{}\PYGZgt{} }\PYG{n}{s} \PYG{o}{=} \PYG{n}{np}\PYG{o}{.}\PYG{n}{random}\PYG{o}{.}\PYG{n}{rayleigh}\PYG{p}{(}\PYG{n}{modevalue}\PYG{p}{,} \PYG{l+m+mi}{1000000}\PYG{p}{)}
\end{Verbatim}

The percentage of waves larger than 3 meters is:

\begin{Verbatim}[commandchars=\\\{\}]
\PYG{g+gp}{\PYGZgt{}\PYGZgt{}\PYGZgt{} }\PYG{l+m+mf}{100.}\PYG{o}{*}\PYG{n+nb}{sum}\PYG{p}{(}\PYG{n}{s}\PYG{o}{\PYGZgt{}}\PYG{l+m+mi}{3}\PYG{p}{)}\PYG{o}{/}\PYG{l+m+mf}{1000000.}
\PYG{g+go}{0.087300000000000003}
\end{Verbatim}

\end{fulllineitems}

\index{sample() (in module acsSCCanalysis)}

\begin{fulllineitems}
\phantomsection\label{acsSCCanalysis:acsSCCanalysis.sample}\pysiglinewithargsret{\code{acsSCCanalysis.}\bfcode{sample}}{}{}
random\_sample(size=None)

Return random floats in the half-open interval {[}0.0, 1.0).

Results are from the ``continuous uniform'' distribution over the
stated interval.  To sample \(Unif[a, b), b > a\) multiply
the output of \emph{random\_sample} by \emph{(b-a)} and add \emph{a}:

\begin{Verbatim}[commandchars=\\\{\}]
\PYG{p}{(}\PYG{n}{b} \PYG{o}{\PYGZhy{}} \PYG{n}{a}\PYG{p}{)} \PYG{o}{*} \PYG{n}{random\PYGZus{}sample}\PYG{p}{(}\PYG{p}{)} \PYG{o}{+} \PYG{n}{a}
\end{Verbatim}
\begin{description}
\item[{size}] \leavevmode{[}int or tuple of ints, optional{]}
Defines the shape of the returned array of random floats. If None
(the default), returns a single float.

\end{description}
\begin{description}
\item[{out}] \leavevmode{[}float or ndarray of floats{]}
Array of random floats of shape \emph{size} (unless \code{size=None}, in which
case a single float is returned).

\end{description}

\begin{Verbatim}[commandchars=\\\{\}]
\PYG{g+gp}{\PYGZgt{}\PYGZgt{}\PYGZgt{} }\PYG{n}{np}\PYG{o}{.}\PYG{n}{random}\PYG{o}{.}\PYG{n}{random\PYGZus{}sample}\PYG{p}{(}\PYG{p}{)}
\PYG{g+go}{0.47108547995356098}
\PYG{g+gp}{\PYGZgt{}\PYGZgt{}\PYGZgt{} }\PYG{n+nb}{type}\PYG{p}{(}\PYG{n}{np}\PYG{o}{.}\PYG{n}{random}\PYG{o}{.}\PYG{n}{random\PYGZus{}sample}\PYG{p}{(}\PYG{p}{)}\PYG{p}{)}
\PYG{g+go}{\PYGZlt{}type \PYGZsq{}float\PYGZsq{}\PYGZgt{}}
\PYG{g+gp}{\PYGZgt{}\PYGZgt{}\PYGZgt{} }\PYG{n}{np}\PYG{o}{.}\PYG{n}{random}\PYG{o}{.}\PYG{n}{random\PYGZus{}sample}\PYG{p}{(}\PYG{p}{(}\PYG{l+m+mi}{5}\PYG{p}{,}\PYG{p}{)}\PYG{p}{)}
\PYG{g+go}{array([ 0.30220482,  0.86820401,  0.1654503 ,  0.11659149,  0.54323428])}
\end{Verbatim}

Three-by-two array of random numbers from {[}-5, 0):

\begin{Verbatim}[commandchars=\\\{\}]
\PYG{g+gp}{\PYGZgt{}\PYGZgt{}\PYGZgt{} }\PYG{l+m+mi}{5} \PYG{o}{*} \PYG{n}{np}\PYG{o}{.}\PYG{n}{random}\PYG{o}{.}\PYG{n}{random\PYGZus{}sample}\PYG{p}{(}\PYG{p}{(}\PYG{l+m+mi}{3}\PYG{p}{,} \PYG{l+m+mi}{2}\PYG{p}{)}\PYG{p}{)} \PYG{o}{\PYGZhy{}} \PYG{l+m+mi}{5}
\PYG{g+go}{array([[\PYGZhy{}3.99149989, \PYGZhy{}0.52338984],}
\PYG{g+go}{       [\PYGZhy{}2.99091858, \PYGZhy{}0.79479508],}
\PYG{g+go}{       [\PYGZhy{}1.23204345, \PYGZhy{}1.75224494]])}
\end{Verbatim}

\end{fulllineitems}

\index{saveGillToFile() (in module acsSCCanalysis)}

\begin{fulllineitems}
\phantomsection\label{acsSCCanalysis:acsSCCanalysis.saveGillToFile}\pysiglinewithargsret{\code{acsSCCanalysis.}\bfcode{saveGillToFile}}{}{}
\end{fulllineitems}

\index{saveGraphSUBToFile() (in module acsSCCanalysis)}

\begin{fulllineitems}
\phantomsection\label{acsSCCanalysis:acsSCCanalysis.saveGraphSUBToFile}\pysiglinewithargsret{\code{acsSCCanalysis.}\bfcode{saveGraphSUBToFile}}{}{}
\end{fulllineitems}

\index{saveGraphToFile() (in module acsSCCanalysis)}

\begin{fulllineitems}
\phantomsection\label{acsSCCanalysis:acsSCCanalysis.saveGraphToFile}\pysiglinewithargsret{\code{acsSCCanalysis.}\bfcode{saveGraphToFile}}{}{}
\end{fulllineitems}

\index{saveNrgToFile() (in module acsSCCanalysis)}

\begin{fulllineitems}
\phantomsection\label{acsSCCanalysis:acsSCCanalysis.saveNrgToFile}\pysiglinewithargsret{\code{acsSCCanalysis.}\bfcode{saveNrgToFile}}{}{}
\end{fulllineitems}

\index{seed() (in module acsSCCanalysis)}

\begin{fulllineitems}
\phantomsection\label{acsSCCanalysis:acsSCCanalysis.seed}\pysiglinewithargsret{\code{acsSCCanalysis.}\bfcode{seed}}{\emph{seed=None}}{}
Seed the generator.

This method is called when \emph{RandomState} is initialized. It can be
called again to re-seed the generator. For details, see \emph{RandomState}.
\begin{description}
\item[{seed}] \leavevmode{[}int or array\_like, optional{]}
Seed for \emph{RandomState}.

\end{description}

RandomState

\end{fulllineitems}

\index{set\_state() (in module acsSCCanalysis)}

\begin{fulllineitems}
\phantomsection\label{acsSCCanalysis:acsSCCanalysis.set_state}\pysiglinewithargsret{\code{acsSCCanalysis.}\bfcode{set\_state}}{\emph{state}}{}
Set the internal state of the generator from a tuple.

For use if one has reason to manually (re-)set the internal state of the
``Mersenne Twister''{\color{red}\bfseries{}{[}1{]}\_} pseudo-random number generating algorithm.
\begin{description}
\item[{state}] \leavevmode{[}tuple(str, ndarray of 624 uints, int, int, float){]}
The \emph{state} tuple has the following items:
\begin{enumerate}
\item {} 
the string `MT19937', specifying the Mersenne Twister algorithm.

\item {} 
a 1-D array of 624 unsigned integers \code{keys}.

\item {} 
an integer \code{pos}.

\item {} 
an integer \code{has\_gauss}.

\item {} 
a float \code{cached\_gaussian}.

\end{enumerate}

\end{description}
\begin{description}
\item[{out}] \leavevmode{[}None{]}
Returns `None' on success.

\end{description}

get\_state

\emph{set\_state} and \emph{get\_state} are not needed to work with any of the
random distributions in NumPy. If the internal state is manually altered,
the user should know exactly what he/she is doing.

For backwards compatibility, the form (str, array of 624 uints, int) is
also accepted although it is missing some information about the cached
Gaussian value: \code{state = ('MT19937', keys, pos)}.

\end{fulllineitems}

\index{shuffle() (in module acsSCCanalysis)}

\begin{fulllineitems}
\phantomsection\label{acsSCCanalysis:acsSCCanalysis.shuffle}\pysiglinewithargsret{\code{acsSCCanalysis.}\bfcode{shuffle}}{\emph{x}}{}
Modify a sequence in-place by shuffling its contents.
\begin{description}
\item[{x}] \leavevmode{[}array\_like{]}
The array or list to be shuffled.

\end{description}

None

\begin{Verbatim}[commandchars=\\\{\}]
\PYG{g+gp}{\PYGZgt{}\PYGZgt{}\PYGZgt{} }\PYG{n}{arr} \PYG{o}{=} \PYG{n}{np}\PYG{o}{.}\PYG{n}{arange}\PYG{p}{(}\PYG{l+m+mi}{10}\PYG{p}{)}
\PYG{g+gp}{\PYGZgt{}\PYGZgt{}\PYGZgt{} }\PYG{n}{np}\PYG{o}{.}\PYG{n}{random}\PYG{o}{.}\PYG{n}{shuffle}\PYG{p}{(}\PYG{n}{arr}\PYG{p}{)}
\PYG{g+gp}{\PYGZgt{}\PYGZgt{}\PYGZgt{} }\PYG{n}{arr}
\PYG{g+go}{[1 7 5 2 9 4 3 6 0 8]}
\end{Verbatim}

This function only shuffles the array along the first index of a
multi-dimensional array:

\begin{Verbatim}[commandchars=\\\{\}]
\PYG{g+gp}{\PYGZgt{}\PYGZgt{}\PYGZgt{} }\PYG{n}{arr} \PYG{o}{=} \PYG{n}{np}\PYG{o}{.}\PYG{n}{arange}\PYG{p}{(}\PYG{l+m+mi}{9}\PYG{p}{)}\PYG{o}{.}\PYG{n}{reshape}\PYG{p}{(}\PYG{p}{(}\PYG{l+m+mi}{3}\PYG{p}{,} \PYG{l+m+mi}{3}\PYG{p}{)}\PYG{p}{)}
\PYG{g+gp}{\PYGZgt{}\PYGZgt{}\PYGZgt{} }\PYG{n}{np}\PYG{o}{.}\PYG{n}{random}\PYG{o}{.}\PYG{n}{shuffle}\PYG{p}{(}\PYG{n}{arr}\PYG{p}{)}
\PYG{g+gp}{\PYGZgt{}\PYGZgt{}\PYGZgt{} }\PYG{n}{arr}
\PYG{g+go}{array([[3, 4, 5],}
\PYG{g+go}{       [6, 7, 8],}
\PYG{g+go}{       [0, 1, 2]])}
\end{Verbatim}

\end{fulllineitems}

\index{standard\_cauchy() (in module acsSCCanalysis)}

\begin{fulllineitems}
\phantomsection\label{acsSCCanalysis:acsSCCanalysis.standard_cauchy}\pysiglinewithargsret{\code{acsSCCanalysis.}\bfcode{standard\_cauchy}}{\emph{size=None}}{}
Standard Cauchy distribution with mode = 0.

Also known as the Lorentz distribution.
\begin{description}
\item[{size}] \leavevmode{[}int or tuple of ints{]}
Shape of the output.

\end{description}
\begin{description}
\item[{samples}] \leavevmode{[}ndarray or scalar{]}
The drawn samples.

\end{description}

The probability density function for the full Cauchy distribution is
\begin{gather}
\begin{split}P(x; x_0, \gamma) = \frac{1}{\pi \gamma \bigl[ 1+
(\frac{x-x_0}{\gamma})^2 \bigr] }\end{split}\notag
\end{gather}
and the Standard Cauchy distribution just sets \(x_0=0\) and
\(\gamma=1\)

The Cauchy distribution arises in the solution to the driven harmonic
oscillator problem, and also describes spectral line broadening. It
also describes the distribution of values at which a line tilted at
a random angle will cut the x axis.

When studying hypothesis tests that assume normality, seeing how the
tests perform on data from a Cauchy distribution is a good indicator of
their sensitivity to a heavy-tailed distribution, since the Cauchy looks
very much like a Gaussian distribution, but with heavier tails.

Draw samples and plot the distribution:

\begin{Verbatim}[commandchars=\\\{\}]
\PYG{g+gp}{\PYGZgt{}\PYGZgt{}\PYGZgt{} }\PYG{n}{s} \PYG{o}{=} \PYG{n}{np}\PYG{o}{.}\PYG{n}{random}\PYG{o}{.}\PYG{n}{standard\PYGZus{}cauchy}\PYG{p}{(}\PYG{l+m+mi}{1000000}\PYG{p}{)}
\PYG{g+gp}{\PYGZgt{}\PYGZgt{}\PYGZgt{} }\PYG{n}{s} \PYG{o}{=} \PYG{n}{s}\PYG{p}{[}\PYG{p}{(}\PYG{n}{s}\PYG{o}{\PYGZgt{}}\PYG{o}{\PYGZhy{}}\PYG{l+m+mi}{25}\PYG{p}{)} \PYG{o}{\PYGZam{}} \PYG{p}{(}\PYG{n}{s}\PYG{o}{\PYGZlt{}}\PYG{l+m+mi}{25}\PYG{p}{)}\PYG{p}{]}  \PYG{c}{\PYGZsh{} truncate distribution so it plots well}
\PYG{g+gp}{\PYGZgt{}\PYGZgt{}\PYGZgt{} }\PYG{n}{plt}\PYG{o}{.}\PYG{n}{hist}\PYG{p}{(}\PYG{n}{s}\PYG{p}{,} \PYG{n}{bins}\PYG{o}{=}\PYG{l+m+mi}{100}\PYG{p}{)}
\PYG{g+gp}{\PYGZgt{}\PYGZgt{}\PYGZgt{} }\PYG{n}{plt}\PYG{o}{.}\PYG{n}{show}\PYG{p}{(}\PYG{p}{)}
\end{Verbatim}

\end{fulllineitems}

\index{standard\_exponential() (in module acsSCCanalysis)}

\begin{fulllineitems}
\phantomsection\label{acsSCCanalysis:acsSCCanalysis.standard_exponential}\pysiglinewithargsret{\code{acsSCCanalysis.}\bfcode{standard\_exponential}}{\emph{size=None}}{}
Draw samples from the standard exponential distribution.

\emph{standard\_exponential} is identical to the exponential distribution
with a scale parameter of 1.
\begin{description}
\item[{size}] \leavevmode{[}int or tuple of ints{]}
Shape of the output.

\end{description}
\begin{description}
\item[{out}] \leavevmode{[}float or ndarray{]}
Drawn samples.

\end{description}

Output a 3x8000 array:

\begin{Verbatim}[commandchars=\\\{\}]
\PYG{g+gp}{\PYGZgt{}\PYGZgt{}\PYGZgt{} }\PYG{n}{n} \PYG{o}{=} \PYG{n}{np}\PYG{o}{.}\PYG{n}{random}\PYG{o}{.}\PYG{n}{standard\PYGZus{}exponential}\PYG{p}{(}\PYG{p}{(}\PYG{l+m+mi}{3}\PYG{p}{,} \PYG{l+m+mi}{8000}\PYG{p}{)}\PYG{p}{)}
\end{Verbatim}

\end{fulllineitems}

\index{standard\_gamma() (in module acsSCCanalysis)}

\begin{fulllineitems}
\phantomsection\label{acsSCCanalysis:acsSCCanalysis.standard_gamma}\pysiglinewithargsret{\code{acsSCCanalysis.}\bfcode{standard\_gamma}}{\emph{shape}, \emph{size=None}}{}
Draw samples from a Standard Gamma distribution.

Samples are drawn from a Gamma distribution with specified parameters,
shape (sometimes designated ``k'') and scale=1.
\begin{description}
\item[{shape}] \leavevmode{[}float{]}
Parameter, should be \textgreater{} 0.

\item[{size}] \leavevmode{[}int or tuple of ints{]}
Output shape.  If the given shape is, e.g., \code{(m, n, k)}, then
\code{m * n * k} samples are drawn.

\end{description}
\begin{description}
\item[{samples}] \leavevmode{[}ndarray or scalar{]}
The drawn samples.

\end{description}
\begin{description}
\item[{scipy.stats.distributions.gamma}] \leavevmode{[}probability density function,{]}
distribution or cumulative density function, etc.

\end{description}

The probability density for the Gamma distribution is
\begin{gather}
\begin{split}p(x) = x^{k-1}\frac{e^{-x/\theta}}{\theta^k\Gamma(k)},\end{split}\notag
\end{gather}
where \(k\) is the shape and \(\theta\) the scale,
and \(\Gamma\) is the Gamma function.

The Gamma distribution is often used to model the times to failure of
electronic components, and arises naturally in processes for which the
waiting times between Poisson distributed events are relevant.

Draw samples from the distribution:

\begin{Verbatim}[commandchars=\\\{\}]
\PYG{g+gp}{\PYGZgt{}\PYGZgt{}\PYGZgt{} }\PYG{n}{shape}\PYG{p}{,} \PYG{n}{scale} \PYG{o}{=} \PYG{l+m+mf}{2.}\PYG{p}{,} \PYG{l+m+mf}{1.} \PYG{c}{\PYGZsh{} mean and width}
\PYG{g+gp}{\PYGZgt{}\PYGZgt{}\PYGZgt{} }\PYG{n}{s} \PYG{o}{=} \PYG{n}{np}\PYG{o}{.}\PYG{n}{random}\PYG{o}{.}\PYG{n}{standard\PYGZus{}gamma}\PYG{p}{(}\PYG{n}{shape}\PYG{p}{,} \PYG{l+m+mi}{1000000}\PYG{p}{)}
\end{Verbatim}

Display the histogram of the samples, along with
the probability density function:

\begin{Verbatim}[commandchars=\\\{\}]
\PYG{g+gp}{\PYGZgt{}\PYGZgt{}\PYGZgt{} }\PYG{k+kn}{import} \PYG{n+nn}{matplotlib.pyplot} \PYG{k+kn}{as} \PYG{n+nn}{plt}
\PYG{g+gp}{\PYGZgt{}\PYGZgt{}\PYGZgt{} }\PYG{k+kn}{import} \PYG{n+nn}{scipy.special} \PYG{k+kn}{as} \PYG{n+nn}{sps}
\PYG{g+gp}{\PYGZgt{}\PYGZgt{}\PYGZgt{} }\PYG{n}{count}\PYG{p}{,} \PYG{n}{bins}\PYG{p}{,} \PYG{n}{ignored} \PYG{o}{=} \PYG{n}{plt}\PYG{o}{.}\PYG{n}{hist}\PYG{p}{(}\PYG{n}{s}\PYG{p}{,} \PYG{l+m+mi}{50}\PYG{p}{,} \PYG{n}{normed}\PYG{o}{=}\PYG{n+nb+bp}{True}\PYG{p}{)}
\PYG{g+gp}{\PYGZgt{}\PYGZgt{}\PYGZgt{} }\PYG{n}{y} \PYG{o}{=} \PYG{n}{bins}\PYG{o}{*}\PYG{o}{*}\PYG{p}{(}\PYG{n}{shape}\PYG{o}{\PYGZhy{}}\PYG{l+m+mi}{1}\PYG{p}{)} \PYG{o}{*} \PYG{p}{(}\PYG{p}{(}\PYG{n}{np}\PYG{o}{.}\PYG{n}{exp}\PYG{p}{(}\PYG{o}{\PYGZhy{}}\PYG{n}{bins}\PYG{o}{/}\PYG{n}{scale}\PYG{p}{)}\PYG{p}{)}\PYG{o}{/} \PYGZbs{}
\PYG{g+gp}{... }                      \PYG{p}{(}\PYG{n}{sps}\PYG{o}{.}\PYG{n}{gamma}\PYG{p}{(}\PYG{n}{shape}\PYG{p}{)} \PYG{o}{*} \PYG{n}{scale}\PYG{o}{*}\PYG{o}{*}\PYG{n}{shape}\PYG{p}{)}\PYG{p}{)}
\PYG{g+gp}{\PYGZgt{}\PYGZgt{}\PYGZgt{} }\PYG{n}{plt}\PYG{o}{.}\PYG{n}{plot}\PYG{p}{(}\PYG{n}{bins}\PYG{p}{,} \PYG{n}{y}\PYG{p}{,} \PYG{n}{linewidth}\PYG{o}{=}\PYG{l+m+mi}{2}\PYG{p}{,} \PYG{n}{color}\PYG{o}{=}\PYG{l+s}{\PYGZsq{}}\PYG{l+s}{r}\PYG{l+s}{\PYGZsq{}}\PYG{p}{)}
\PYG{g+gp}{\PYGZgt{}\PYGZgt{}\PYGZgt{} }\PYG{n}{plt}\PYG{o}{.}\PYG{n}{show}\PYG{p}{(}\PYG{p}{)}
\end{Verbatim}

\end{fulllineitems}

\index{standard\_normal() (in module acsSCCanalysis)}

\begin{fulllineitems}
\phantomsection\label{acsSCCanalysis:acsSCCanalysis.standard_normal}\pysiglinewithargsret{\code{acsSCCanalysis.}\bfcode{standard\_normal}}{\emph{size=None}}{}
Returns samples from a Standard Normal distribution (mean=0, stdev=1).
\begin{description}
\item[{size}] \leavevmode{[}int or tuple of ints, optional{]}
Output shape. Default is None, in which case a single value is
returned.

\end{description}
\begin{description}
\item[{out}] \leavevmode{[}float or ndarray{]}
Drawn samples.

\end{description}

\begin{Verbatim}[commandchars=\\\{\}]
\PYG{g+gp}{\PYGZgt{}\PYGZgt{}\PYGZgt{} }\PYG{n}{s} \PYG{o}{=} \PYG{n}{np}\PYG{o}{.}\PYG{n}{random}\PYG{o}{.}\PYG{n}{standard\PYGZus{}normal}\PYG{p}{(}\PYG{l+m+mi}{8000}\PYG{p}{)}
\PYG{g+gp}{\PYGZgt{}\PYGZgt{}\PYGZgt{} }\PYG{n}{s}
\PYG{g+go}{array([ 0.6888893 ,  0.78096262, \PYGZhy{}0.89086505, ...,  0.49876311, \PYGZsh{}random}
\PYG{g+go}{       \PYGZhy{}0.38672696, \PYGZhy{}0.4685006 ])                               \PYGZsh{}random}
\PYG{g+gp}{\PYGZgt{}\PYGZgt{}\PYGZgt{} }\PYG{n}{s}\PYG{o}{.}\PYG{n}{shape}
\PYG{g+go}{(8000,)}
\PYG{g+gp}{\PYGZgt{}\PYGZgt{}\PYGZgt{} }\PYG{n}{s} \PYG{o}{=} \PYG{n}{np}\PYG{o}{.}\PYG{n}{random}\PYG{o}{.}\PYG{n}{standard\PYGZus{}normal}\PYG{p}{(}\PYG{n}{size}\PYG{o}{=}\PYG{p}{(}\PYG{l+m+mi}{3}\PYG{p}{,} \PYG{l+m+mi}{4}\PYG{p}{,} \PYG{l+m+mi}{2}\PYG{p}{)}\PYG{p}{)}
\PYG{g+gp}{\PYGZgt{}\PYGZgt{}\PYGZgt{} }\PYG{n}{s}\PYG{o}{.}\PYG{n}{shape}
\PYG{g+go}{(3, 4, 2)}
\end{Verbatim}

\end{fulllineitems}

\index{standard\_t() (in module acsSCCanalysis)}

\begin{fulllineitems}
\phantomsection\label{acsSCCanalysis:acsSCCanalysis.standard_t}\pysiglinewithargsret{\code{acsSCCanalysis.}\bfcode{standard\_t}}{\emph{df}, \emph{size=None}}{}
Standard Student's t distribution with df degrees of freedom.

A special case of the hyperbolic distribution.
As \emph{df} gets large, the result resembles that of the standard normal
distribution (\emph{standard\_normal}).
\begin{description}
\item[{df}] \leavevmode{[}int{]}
Degrees of freedom, should be \textgreater{} 0.

\item[{size}] \leavevmode{[}int or tuple of ints, optional{]}
Output shape. Default is None, in which case a single value is
returned.

\end{description}
\begin{description}
\item[{samples}] \leavevmode{[}ndarray or scalar{]}
Drawn samples.

\end{description}

The probability density function for the t distribution is
\begin{gather}
\begin{split}P(x, df) = \frac{\Gamma(\frac{df+1}{2})}{\sqrt{\pi df}
\Gamma(\frac{df}{2})}\Bigl( 1+\frac{x^2}{df} \Bigr)^{-(df+1)/2}\end{split}\notag
\end{gather}
The t test is based on an assumption that the data come from a Normal
distribution. The t test provides a way to test whether the sample mean
(that is the mean calculated from the data) is a good estimate of the true
mean.

The derivation of the t-distribution was forst published in 1908 by William
Gisset while working for the Guinness Brewery in Dublin. Due to proprietary
issues, he had to publish under a pseudonym, and so he used the name
Student.

From Dalgaard page 83 {\color{red}\bfseries{}{[}1{]}\_}, suppose the daily energy intake for 11
women in Kj is:

\begin{Verbatim}[commandchars=\\\{\}]
\PYG{g+gp}{\PYGZgt{}\PYGZgt{}\PYGZgt{} }\PYG{n}{intake} \PYG{o}{=} \PYG{n}{np}\PYG{o}{.}\PYG{n}{array}\PYG{p}{(}\PYG{p}{[}\PYG{l+m+mf}{5260.}\PYG{p}{,} \PYG{l+m+mi}{5470}\PYG{p}{,} \PYG{l+m+mi}{5640}\PYG{p}{,} \PYG{l+m+mi}{6180}\PYG{p}{,} \PYG{l+m+mi}{6390}\PYG{p}{,} \PYG{l+m+mi}{6515}\PYG{p}{,} \PYG{l+m+mi}{6805}\PYG{p}{,} \PYG{l+m+mi}{7515}\PYG{p}{,} \PYGZbs{}
\PYG{g+gp}{... }                   \PYG{l+m+mi}{7515}\PYG{p}{,} \PYG{l+m+mi}{8230}\PYG{p}{,} \PYG{l+m+mi}{8770}\PYG{p}{]}\PYG{p}{)}
\end{Verbatim}

Does their energy intake deviate systematically from the recommended
value of 7725 kJ?

We have 10 degrees of freedom, so is the sample mean within 95\% of the
recommended value?

\begin{Verbatim}[commandchars=\\\{\}]
\PYG{g+gp}{\PYGZgt{}\PYGZgt{}\PYGZgt{} }\PYG{n}{s} \PYG{o}{=} \PYG{n}{np}\PYG{o}{.}\PYG{n}{random}\PYG{o}{.}\PYG{n}{standard\PYGZus{}t}\PYG{p}{(}\PYG{l+m+mi}{10}\PYG{p}{,} \PYG{n}{size}\PYG{o}{=}\PYG{l+m+mi}{100000}\PYG{p}{)}
\PYG{g+gp}{\PYGZgt{}\PYGZgt{}\PYGZgt{} }\PYG{n}{np}\PYG{o}{.}\PYG{n}{mean}\PYG{p}{(}\PYG{n}{intake}\PYG{p}{)}
\PYG{g+go}{6753.636363636364}
\PYG{g+gp}{\PYGZgt{}\PYGZgt{}\PYGZgt{} }\PYG{n}{intake}\PYG{o}{.}\PYG{n}{std}\PYG{p}{(}\PYG{n}{ddof}\PYG{o}{=}\PYG{l+m+mi}{1}\PYG{p}{)}
\PYG{g+go}{1142.1232221373727}
\end{Verbatim}

Calculate the t statistic, setting the ddof parameter to the unbiased
value so the divisor in the standard deviation will be degrees of
freedom, N-1.

\begin{Verbatim}[commandchars=\\\{\}]
\PYG{g+gp}{\PYGZgt{}\PYGZgt{}\PYGZgt{} }\PYG{n}{t} \PYG{o}{=} \PYG{p}{(}\PYG{n}{np}\PYG{o}{.}\PYG{n}{mean}\PYG{p}{(}\PYG{n}{intake}\PYG{p}{)}\PYG{o}{\PYGZhy{}}\PYG{l+m+mi}{7725}\PYG{p}{)}\PYG{o}{/}\PYG{p}{(}\PYG{n}{intake}\PYG{o}{.}\PYG{n}{std}\PYG{p}{(}\PYG{n}{ddof}\PYG{o}{=}\PYG{l+m+mi}{1}\PYG{p}{)}\PYG{o}{/}\PYG{n}{np}\PYG{o}{.}\PYG{n}{sqrt}\PYG{p}{(}\PYG{n+nb}{len}\PYG{p}{(}\PYG{n}{intake}\PYG{p}{)}\PYG{p}{)}\PYG{p}{)}
\PYG{g+gp}{\PYGZgt{}\PYGZgt{}\PYGZgt{} }\PYG{k+kn}{import} \PYG{n+nn}{matplotlib.pyplot} \PYG{k+kn}{as} \PYG{n+nn}{plt}
\PYG{g+gp}{\PYGZgt{}\PYGZgt{}\PYGZgt{} }\PYG{n}{h} \PYG{o}{=} \PYG{n}{plt}\PYG{o}{.}\PYG{n}{hist}\PYG{p}{(}\PYG{n}{s}\PYG{p}{,} \PYG{n}{bins}\PYG{o}{=}\PYG{l+m+mi}{100}\PYG{p}{,} \PYG{n}{normed}\PYG{o}{=}\PYG{n+nb+bp}{True}\PYG{p}{)}
\end{Verbatim}

For a one-sided t-test, how far out in the distribution does the t
statistic appear?

\begin{Verbatim}[commandchars=\\\{\}]
\PYG{g+gp}{\PYGZgt{}\PYGZgt{}\PYGZgt{} }\PYG{o}{\PYGZgt{}\PYGZgt{}}\PYG{o}{\PYGZgt{}} \PYG{n}{np}\PYG{o}{.}\PYG{n}{sum}\PYG{p}{(}\PYG{n}{s}\PYG{o}{\PYGZlt{}}\PYG{n}{t}\PYG{p}{)} \PYG{o}{/} \PYG{n+nb}{float}\PYG{p}{(}\PYG{n+nb}{len}\PYG{p}{(}\PYG{n}{s}\PYG{p}{)}\PYG{p}{)}
\PYG{g+go}{0.0090699999999999999  \PYGZsh{}random}
\end{Verbatim}

So the p-value is about 0.009, which says the null hypothesis has a
probability of about 99\% of being true.

\end{fulllineitems}

\index{triangular() (in module acsSCCanalysis)}

\begin{fulllineitems}
\phantomsection\label{acsSCCanalysis:acsSCCanalysis.triangular}\pysiglinewithargsret{\code{acsSCCanalysis.}\bfcode{triangular}}{\emph{left}, \emph{mode}, \emph{right}, \emph{size=None}}{}
Draw samples from the triangular distribution.

The triangular distribution is a continuous probability distribution with
lower limit left, peak at mode, and upper limit right. Unlike the other
distributions, these parameters directly define the shape of the pdf.
\begin{description}
\item[{left}] \leavevmode{[}scalar{]}
Lower limit.

\item[{mode}] \leavevmode{[}scalar{]}
The value where the peak of the distribution occurs.
The value should fulfill the condition \code{left \textless{}= mode \textless{}= right}.

\item[{right}] \leavevmode{[}scalar{]}
Upper limit, should be larger than \emph{left}.

\item[{size}] \leavevmode{[}int or tuple of ints, optional{]}
Output shape. Default is None, in which case a single value is
returned.

\end{description}
\begin{description}
\item[{samples}] \leavevmode{[}ndarray or scalar{]}
The returned samples all lie in the interval {[}left, right{]}.

\end{description}

The probability density function for the Triangular distribution is
\begin{gather}
\begin{split}P(x;l, m, r) = \begin{cases}
\frac{2(x-l)}{(r-l)(m-l)}& \text{for $l \leq x \leq m$},\\
\frac{2(m-x)}{(r-l)(r-m)}& \text{for $m \leq x \leq r$},\\
0& \text{otherwise}.
\end{cases}\end{split}\notag
\end{gather}
The triangular distribution is often used in ill-defined problems where the
underlying distribution is not known, but some knowledge of the limits and
mode exists. Often it is used in simulations.

Draw values from the distribution and plot the histogram:

\begin{Verbatim}[commandchars=\\\{\}]
\PYG{g+gp}{\PYGZgt{}\PYGZgt{}\PYGZgt{} }\PYG{k+kn}{import} \PYG{n+nn}{matplotlib.pyplot} \PYG{k+kn}{as} \PYG{n+nn}{plt}
\PYG{g+gp}{\PYGZgt{}\PYGZgt{}\PYGZgt{} }\PYG{n}{h} \PYG{o}{=} \PYG{n}{plt}\PYG{o}{.}\PYG{n}{hist}\PYG{p}{(}\PYG{n}{np}\PYG{o}{.}\PYG{n}{random}\PYG{o}{.}\PYG{n}{triangular}\PYG{p}{(}\PYG{o}{\PYGZhy{}}\PYG{l+m+mi}{3}\PYG{p}{,} \PYG{l+m+mi}{0}\PYG{p}{,} \PYG{l+m+mi}{8}\PYG{p}{,} \PYG{l+m+mi}{100000}\PYG{p}{)}\PYG{p}{,} \PYG{n}{bins}\PYG{o}{=}\PYG{l+m+mi}{200}\PYG{p}{,}
\PYG{g+gp}{... }             \PYG{n}{normed}\PYG{o}{=}\PYG{n+nb+bp}{True}\PYG{p}{)}
\PYG{g+gp}{\PYGZgt{}\PYGZgt{}\PYGZgt{} }\PYG{n}{plt}\PYG{o}{.}\PYG{n}{show}\PYG{p}{(}\PYG{p}{)}
\end{Verbatim}

\end{fulllineitems}

\index{uniform() (in module acsSCCanalysis)}

\begin{fulllineitems}
\phantomsection\label{acsSCCanalysis:acsSCCanalysis.uniform}\pysiglinewithargsret{\code{acsSCCanalysis.}\bfcode{uniform}}{\emph{low=0.0}, \emph{high=1.0}, \emph{size=1}}{}
Draw samples from a uniform distribution.

Samples are uniformly distributed over the half-open interval
\code{{[}low, high)} (includes low, but excludes high).  In other words,
any value within the given interval is equally likely to be drawn
by \emph{uniform}.
\begin{description}
\item[{low}] \leavevmode{[}float, optional{]}
Lower boundary of the output interval.  All values generated will be
greater than or equal to low.  The default value is 0.

\item[{high}] \leavevmode{[}float{]}
Upper boundary of the output interval.  All values generated will be
less than high.  The default value is 1.0.

\item[{size}] \leavevmode{[}int or tuple of ints, optional{]}
Shape of output.  If the given size is, for example, (m,n,k),
m*n*k samples are generated.  If no shape is specified, a single sample
is returned.

\end{description}
\begin{description}
\item[{out}] \leavevmode{[}ndarray{]}
Drawn samples, with shape \emph{size}.

\end{description}

randint : Discrete uniform distribution, yielding integers.
random\_integers : Discrete uniform distribution over the closed
\begin{quote}

interval \code{{[}low, high{]}}.
\end{quote}

random\_sample : Floats uniformly distributed over \code{{[}0, 1)}.
random : Alias for \emph{random\_sample}.
rand : Convenience function that accepts dimensions as input, e.g.,
\begin{quote}

\code{rand(2,2)} would generate a 2-by-2 array of floats,
uniformly distributed over \code{{[}0, 1)}.
\end{quote}

The probability density function of the uniform distribution is
\begin{gather}
\begin{split}p(x) = \frac{1}{b - a}\end{split}\notag
\end{gather}
anywhere within the interval \code{{[}a, b)}, and zero elsewhere.

Draw samples from the distribution:

\begin{Verbatim}[commandchars=\\\{\}]
\PYG{g+gp}{\PYGZgt{}\PYGZgt{}\PYGZgt{} }\PYG{n}{s} \PYG{o}{=} \PYG{n}{np}\PYG{o}{.}\PYG{n}{random}\PYG{o}{.}\PYG{n}{uniform}\PYG{p}{(}\PYG{o}{\PYGZhy{}}\PYG{l+m+mi}{1}\PYG{p}{,}\PYG{l+m+mi}{0}\PYG{p}{,}\PYG{l+m+mi}{1000}\PYG{p}{)}
\end{Verbatim}

All values are within the given interval:

\begin{Verbatim}[commandchars=\\\{\}]
\PYG{g+gp}{\PYGZgt{}\PYGZgt{}\PYGZgt{} }\PYG{n}{np}\PYG{o}{.}\PYG{n}{all}\PYG{p}{(}\PYG{n}{s} \PYG{o}{\PYGZgt{}}\PYG{o}{=} \PYG{o}{\PYGZhy{}}\PYG{l+m+mi}{1}\PYG{p}{)}
\PYG{g+go}{True}
\PYG{g+gp}{\PYGZgt{}\PYGZgt{}\PYGZgt{} }\PYG{n}{np}\PYG{o}{.}\PYG{n}{all}\PYG{p}{(}\PYG{n}{s} \PYG{o}{\PYGZlt{}} \PYG{l+m+mi}{0}\PYG{p}{)}
\PYG{g+go}{True}
\end{Verbatim}

Display the histogram of the samples, along with the
probability density function:

\begin{Verbatim}[commandchars=\\\{\}]
\PYG{g+gp}{\PYGZgt{}\PYGZgt{}\PYGZgt{} }\PYG{k+kn}{import} \PYG{n+nn}{matplotlib.pyplot} \PYG{k+kn}{as} \PYG{n+nn}{plt}
\PYG{g+gp}{\PYGZgt{}\PYGZgt{}\PYGZgt{} }\PYG{n}{count}\PYG{p}{,} \PYG{n}{bins}\PYG{p}{,} \PYG{n}{ignored} \PYG{o}{=} \PYG{n}{plt}\PYG{o}{.}\PYG{n}{hist}\PYG{p}{(}\PYG{n}{s}\PYG{p}{,} \PYG{l+m+mi}{15}\PYG{p}{,} \PYG{n}{normed}\PYG{o}{=}\PYG{n+nb+bp}{True}\PYG{p}{)}
\PYG{g+gp}{\PYGZgt{}\PYGZgt{}\PYGZgt{} }\PYG{n}{plt}\PYG{o}{.}\PYG{n}{plot}\PYG{p}{(}\PYG{n}{bins}\PYG{p}{,} \PYG{n}{np}\PYG{o}{.}\PYG{n}{ones\PYGZus{}like}\PYG{p}{(}\PYG{n}{bins}\PYG{p}{)}\PYG{p}{,} \PYG{n}{linewidth}\PYG{o}{=}\PYG{l+m+mi}{2}\PYG{p}{,} \PYG{n}{color}\PYG{o}{=}\PYG{l+s}{\PYGZsq{}}\PYG{l+s}{r}\PYG{l+s}{\PYGZsq{}}\PYG{p}{)}
\PYG{g+gp}{\PYGZgt{}\PYGZgt{}\PYGZgt{} }\PYG{n}{plt}\PYG{o}{.}\PYG{n}{show}\PYG{p}{(}\PYG{p}{)}
\end{Verbatim}

\end{fulllineitems}

\index{vonmises() (in module acsSCCanalysis)}

\begin{fulllineitems}
\phantomsection\label{acsSCCanalysis:acsSCCanalysis.vonmises}\pysiglinewithargsret{\code{acsSCCanalysis.}\bfcode{vonmises}}{\emph{mu}, \emph{kappa}, \emph{size=None}}{}
Draw samples from a von Mises distribution.

Samples are drawn from a von Mises distribution with specified mode
(mu) and dispersion (kappa), on the interval {[}-pi, pi{]}.

The von Mises distribution (also known as the circular normal
distribution) is a continuous probability distribution on the unit
circle.  It may be thought of as the circular analogue of the normal
distribution.
\begin{description}
\item[{mu}] \leavevmode{[}float{]}
Mode (``center'') of the distribution.

\item[{kappa}] \leavevmode{[}float{]}
Dispersion of the distribution, has to be \textgreater{}=0.

\item[{size}] \leavevmode{[}int or tuple of int{]}
Output shape.  If the given shape is, e.g., \code{(m, n, k)}, then
\code{m * n * k} samples are drawn.

\end{description}
\begin{description}
\item[{samples}] \leavevmode{[}scalar or ndarray{]}
The returned samples, which are in the interval {[}-pi, pi{]}.

\end{description}
\begin{description}
\item[{scipy.stats.distributions.vonmises}] \leavevmode{[}probability density function,{]}
distribution, or cumulative density function, etc.

\end{description}

The probability density for the von Mises distribution is
\begin{gather}
\begin{split}p(x) = \frac{e^{\kappa cos(x-\mu)}}{2\pi I_0(\kappa)},\end{split}\notag
\end{gather}
where \(\mu\) is the mode and \(\kappa\) the dispersion,
and \(I_0(\kappa)\) is the modified Bessel function of order 0.

The von Mises is named for Richard Edler von Mises, who was born in
Austria-Hungary, in what is now the Ukraine.  He fled to the United
States in 1939 and became a professor at Harvard.  He worked in
probability theory, aerodynamics, fluid mechanics, and philosophy of
science.

Abramowitz, M. and Stegun, I. A. (ed.), \emph{Handbook of Mathematical
Functions}, New York: Dover, 1965.

von Mises, R., \emph{Mathematical Theory of Probability and Statistics},
New York: Academic Press, 1964.

Draw samples from the distribution:

\begin{Verbatim}[commandchars=\\\{\}]
\PYG{g+gp}{\PYGZgt{}\PYGZgt{}\PYGZgt{} }\PYG{n}{mu}\PYG{p}{,} \PYG{n}{kappa} \PYG{o}{=} \PYG{l+m+mf}{0.0}\PYG{p}{,} \PYG{l+m+mf}{4.0} \PYG{c}{\PYGZsh{} mean and dispersion}
\PYG{g+gp}{\PYGZgt{}\PYGZgt{}\PYGZgt{} }\PYG{n}{s} \PYG{o}{=} \PYG{n}{np}\PYG{o}{.}\PYG{n}{random}\PYG{o}{.}\PYG{n}{vonmises}\PYG{p}{(}\PYG{n}{mu}\PYG{p}{,} \PYG{n}{kappa}\PYG{p}{,} \PYG{l+m+mi}{1000}\PYG{p}{)}
\end{Verbatim}

Display the histogram of the samples, along with
the probability density function:

\begin{Verbatim}[commandchars=\\\{\}]
\PYG{g+gp}{\PYGZgt{}\PYGZgt{}\PYGZgt{} }\PYG{k+kn}{import} \PYG{n+nn}{matplotlib.pyplot} \PYG{k+kn}{as} \PYG{n+nn}{plt}
\PYG{g+gp}{\PYGZgt{}\PYGZgt{}\PYGZgt{} }\PYG{k+kn}{import} \PYG{n+nn}{scipy.special} \PYG{k+kn}{as} \PYG{n+nn}{sps}
\PYG{g+gp}{\PYGZgt{}\PYGZgt{}\PYGZgt{} }\PYG{n}{count}\PYG{p}{,} \PYG{n}{bins}\PYG{p}{,} \PYG{n}{ignored} \PYG{o}{=} \PYG{n}{plt}\PYG{o}{.}\PYG{n}{hist}\PYG{p}{(}\PYG{n}{s}\PYG{p}{,} \PYG{l+m+mi}{50}\PYG{p}{,} \PYG{n}{normed}\PYG{o}{=}\PYG{n+nb+bp}{True}\PYG{p}{)}
\PYG{g+gp}{\PYGZgt{}\PYGZgt{}\PYGZgt{} }\PYG{n}{x} \PYG{o}{=} \PYG{n}{np}\PYG{o}{.}\PYG{n}{arange}\PYG{p}{(}\PYG{o}{\PYGZhy{}}\PYG{n}{np}\PYG{o}{.}\PYG{n}{pi}\PYG{p}{,} \PYG{n}{np}\PYG{o}{.}\PYG{n}{pi}\PYG{p}{,} \PYG{l+m+mi}{2}\PYG{o}{*}\PYG{n}{np}\PYG{o}{.}\PYG{n}{pi}\PYG{o}{/}\PYG{l+m+mf}{50.}\PYG{p}{)}
\PYG{g+gp}{\PYGZgt{}\PYGZgt{}\PYGZgt{} }\PYG{n}{y} \PYG{o}{=} \PYG{o}{\PYGZhy{}}\PYG{n}{np}\PYG{o}{.}\PYG{n}{exp}\PYG{p}{(}\PYG{n}{kappa}\PYG{o}{*}\PYG{n}{np}\PYG{o}{.}\PYG{n}{cos}\PYG{p}{(}\PYG{n}{x}\PYG{o}{\PYGZhy{}}\PYG{n}{mu}\PYG{p}{)}\PYG{p}{)}\PYG{o}{/}\PYG{p}{(}\PYG{l+m+mi}{2}\PYG{o}{*}\PYG{n}{np}\PYG{o}{.}\PYG{n}{pi}\PYG{o}{*}\PYG{n}{sps}\PYG{o}{.}\PYG{n}{jn}\PYG{p}{(}\PYG{l+m+mi}{0}\PYG{p}{,}\PYG{n}{kappa}\PYG{p}{)}\PYG{p}{)}
\PYG{g+gp}{\PYGZgt{}\PYGZgt{}\PYGZgt{} }\PYG{n}{plt}\PYG{o}{.}\PYG{n}{plot}\PYG{p}{(}\PYG{n}{x}\PYG{p}{,} \PYG{n}{y}\PYG{o}{/}\PYG{n+nb}{max}\PYG{p}{(}\PYG{n}{y}\PYG{p}{)}\PYG{p}{,} \PYG{n}{linewidth}\PYG{o}{=}\PYG{l+m+mi}{2}\PYG{p}{,} \PYG{n}{color}\PYG{o}{=}\PYG{l+s}{\PYGZsq{}}\PYG{l+s}{r}\PYG{l+s}{\PYGZsq{}}\PYG{p}{)}
\PYG{g+gp}{\PYGZgt{}\PYGZgt{}\PYGZgt{} }\PYG{n}{plt}\PYG{o}{.}\PYG{n}{show}\PYG{p}{(}\PYG{p}{)}
\end{Verbatim}

\end{fulllineitems}

\index{wald() (in module acsSCCanalysis)}

\begin{fulllineitems}
\phantomsection\label{acsSCCanalysis:acsSCCanalysis.wald}\pysiglinewithargsret{\code{acsSCCanalysis.}\bfcode{wald}}{\emph{mean}, \emph{scale}, \emph{size=None}}{}
Draw samples from a Wald, or Inverse Gaussian, distribution.

As the scale approaches infinity, the distribution becomes more like a
Gaussian.

Some references claim that the Wald is an Inverse Gaussian with mean=1, but
this is by no means universal.

The Inverse Gaussian distribution was first studied in relationship to
Brownian motion. In 1956 M.C.K. Tweedie used the name Inverse Gaussian
because there is an inverse relationship between the time to cover a unit
distance and distance covered in unit time.
\begin{description}
\item[{mean}] \leavevmode{[}scalar{]}
Distribution mean, should be \textgreater{} 0.

\item[{scale}] \leavevmode{[}scalar{]}
Scale parameter, should be \textgreater{}= 0.

\item[{size}] \leavevmode{[}int or tuple of ints, optional{]}
Output shape. Default is None, in which case a single value is
returned.

\end{description}
\begin{description}
\item[{samples}] \leavevmode{[}ndarray or scalar{]}
Drawn sample, all greater than zero.

\end{description}

The probability density function for the Wald distribution is
\begin{gather}
\begin{split}P(x;mean,scale) = \sqrt{\frac{scale}{2\pi x^3}}e^
\frac{-scale(x-mean)^2}{2\cdotp mean^2x}\end{split}\notag
\end{gather}
As noted above the Inverse Gaussian distribution first arise from attempts
to model Brownian Motion. It is also a competitor to the Weibull for use in
reliability modeling and modeling stock returns and interest rate
processes.

Draw values from the distribution and plot the histogram:

\begin{Verbatim}[commandchars=\\\{\}]
\PYG{g+gp}{\PYGZgt{}\PYGZgt{}\PYGZgt{} }\PYG{k+kn}{import} \PYG{n+nn}{matplotlib.pyplot} \PYG{k+kn}{as} \PYG{n+nn}{plt}
\PYG{g+gp}{\PYGZgt{}\PYGZgt{}\PYGZgt{} }\PYG{n}{h} \PYG{o}{=} \PYG{n}{plt}\PYG{o}{.}\PYG{n}{hist}\PYG{p}{(}\PYG{n}{np}\PYG{o}{.}\PYG{n}{random}\PYG{o}{.}\PYG{n}{wald}\PYG{p}{(}\PYG{l+m+mi}{3}\PYG{p}{,} \PYG{l+m+mi}{2}\PYG{p}{,} \PYG{l+m+mi}{100000}\PYG{p}{)}\PYG{p}{,} \PYG{n}{bins}\PYG{o}{=}\PYG{l+m+mi}{200}\PYG{p}{,} \PYG{n}{normed}\PYG{o}{=}\PYG{n+nb+bp}{True}\PYG{p}{)}
\PYG{g+gp}{\PYGZgt{}\PYGZgt{}\PYGZgt{} }\PYG{n}{plt}\PYG{o}{.}\PYG{n}{show}\PYG{p}{(}\PYG{p}{)}
\end{Verbatim}

\end{fulllineitems}

\index{weibull() (in module acsSCCanalysis)}

\begin{fulllineitems}
\phantomsection\label{acsSCCanalysis:acsSCCanalysis.weibull}\pysiglinewithargsret{\code{acsSCCanalysis.}\bfcode{weibull}}{\emph{a}, \emph{size=None}}{}
Weibull distribution.

Draw samples from a 1-parameter Weibull distribution with the given
shape parameter \emph{a}.
\begin{gather}
\begin{split}X = (-ln(U))^{1/a}\end{split}\notag
\end{gather}
Here, U is drawn from the uniform distribution over (0,1{]}.

The more common 2-parameter Weibull, including a scale parameter
\(\lambda\) is just \(X = \lambda(-ln(U))^{1/a}\).
\begin{description}
\item[{a}] \leavevmode{[}float{]}
Shape of the distribution.

\item[{size}] \leavevmode{[}tuple of ints{]}
Output shape.  If the given shape is, e.g., \code{(m, n, k)}, then
\code{m * n * k} samples are drawn.

\end{description}

scipy.stats.distributions.weibull\_max
scipy.stats.distributions.weibull\_min
scipy.stats.distributions.genextreme
gumbel

The Weibull (or Type III asymptotic extreme value distribution for smallest
values, SEV Type III, or Rosin-Rammler distribution) is one of a class of
Generalized Extreme Value (GEV) distributions used in modeling extreme
value problems.  This class includes the Gumbel and Frechet distributions.

The probability density for the Weibull distribution is
\begin{gather}
\begin{split}p(x) = \frac{a}
{\lambda}(\frac{x}{\lambda})^{a-1}e^{-(x/\lambda)^a},\end{split}\notag
\end{gather}
where \(a\) is the shape and \(\lambda\) the scale.

The function has its peak (the mode) at
\(\lambda(\frac{a-1}{a})^{1/a}\).

When \code{a = 1}, the Weibull distribution reduces to the exponential
distribution.

Draw samples from the distribution:

\begin{Verbatim}[commandchars=\\\{\}]
\PYG{g+gp}{\PYGZgt{}\PYGZgt{}\PYGZgt{} }\PYG{n}{a} \PYG{o}{=} \PYG{l+m+mf}{5.} \PYG{c}{\PYGZsh{} shape}
\PYG{g+gp}{\PYGZgt{}\PYGZgt{}\PYGZgt{} }\PYG{n}{s} \PYG{o}{=} \PYG{n}{np}\PYG{o}{.}\PYG{n}{random}\PYG{o}{.}\PYG{n}{weibull}\PYG{p}{(}\PYG{n}{a}\PYG{p}{,} \PYG{l+m+mi}{1000}\PYG{p}{)}
\end{Verbatim}

Display the histogram of the samples, along with
the probability density function:

\begin{Verbatim}[commandchars=\\\{\}]
\PYG{g+gp}{\PYGZgt{}\PYGZgt{}\PYGZgt{} }\PYG{k+kn}{import} \PYG{n+nn}{matplotlib.pyplot} \PYG{k+kn}{as} \PYG{n+nn}{plt}
\PYG{g+gp}{\PYGZgt{}\PYGZgt{}\PYGZgt{} }\PYG{n}{x} \PYG{o}{=} \PYG{n}{np}\PYG{o}{.}\PYG{n}{arange}\PYG{p}{(}\PYG{l+m+mi}{1}\PYG{p}{,}\PYG{l+m+mf}{100.}\PYG{p}{)}\PYG{o}{/}\PYG{l+m+mf}{50.}
\PYG{g+gp}{\PYGZgt{}\PYGZgt{}\PYGZgt{} }\PYG{k}{def} \PYG{n+nf}{weib}\PYG{p}{(}\PYG{n}{x}\PYG{p}{,}\PYG{n}{n}\PYG{p}{,}\PYG{n}{a}\PYG{p}{)}\PYG{p}{:}
\PYG{g+gp}{... }    \PYG{k}{return} \PYG{p}{(}\PYG{n}{a} \PYG{o}{/} \PYG{n}{n}\PYG{p}{)} \PYG{o}{*} \PYG{p}{(}\PYG{n}{x} \PYG{o}{/} \PYG{n}{n}\PYG{p}{)}\PYG{o}{*}\PYG{o}{*}\PYG{p}{(}\PYG{n}{a} \PYG{o}{\PYGZhy{}} \PYG{l+m+mi}{1}\PYG{p}{)} \PYG{o}{*} \PYG{n}{np}\PYG{o}{.}\PYG{n}{exp}\PYG{p}{(}\PYG{o}{\PYGZhy{}}\PYG{p}{(}\PYG{n}{x} \PYG{o}{/} \PYG{n}{n}\PYG{p}{)}\PYG{o}{*}\PYG{o}{*}\PYG{n}{a}\PYG{p}{)}
\end{Verbatim}

\begin{Verbatim}[commandchars=\\\{\}]
\PYG{g+gp}{\PYGZgt{}\PYGZgt{}\PYGZgt{} }\PYG{n}{count}\PYG{p}{,} \PYG{n}{bins}\PYG{p}{,} \PYG{n}{ignored} \PYG{o}{=} \PYG{n}{plt}\PYG{o}{.}\PYG{n}{hist}\PYG{p}{(}\PYG{n}{np}\PYG{o}{.}\PYG{n}{random}\PYG{o}{.}\PYG{n}{weibull}\PYG{p}{(}\PYG{l+m+mf}{5.}\PYG{p}{,}\PYG{l+m+mi}{1000}\PYG{p}{)}\PYG{p}{)}
\PYG{g+gp}{\PYGZgt{}\PYGZgt{}\PYGZgt{} }\PYG{n}{x} \PYG{o}{=} \PYG{n}{np}\PYG{o}{.}\PYG{n}{arange}\PYG{p}{(}\PYG{l+m+mi}{1}\PYG{p}{,}\PYG{l+m+mf}{100.}\PYG{p}{)}\PYG{o}{/}\PYG{l+m+mf}{50.}
\PYG{g+gp}{\PYGZgt{}\PYGZgt{}\PYGZgt{} }\PYG{n}{scale} \PYG{o}{=} \PYG{n}{count}\PYG{o}{.}\PYG{n}{max}\PYG{p}{(}\PYG{p}{)}\PYG{o}{/}\PYG{n}{weib}\PYG{p}{(}\PYG{n}{x}\PYG{p}{,} \PYG{l+m+mf}{1.}\PYG{p}{,} \PYG{l+m+mf}{5.}\PYG{p}{)}\PYG{o}{.}\PYG{n}{max}\PYG{p}{(}\PYG{p}{)}
\PYG{g+gp}{\PYGZgt{}\PYGZgt{}\PYGZgt{} }\PYG{n}{plt}\PYG{o}{.}\PYG{n}{plot}\PYG{p}{(}\PYG{n}{x}\PYG{p}{,} \PYG{n}{weib}\PYG{p}{(}\PYG{n}{x}\PYG{p}{,} \PYG{l+m+mf}{1.}\PYG{p}{,} \PYG{l+m+mf}{5.}\PYG{p}{)}\PYG{o}{*}\PYG{n}{scale}\PYG{p}{)}
\PYG{g+gp}{\PYGZgt{}\PYGZgt{}\PYGZgt{} }\PYG{n}{plt}\PYG{o}{.}\PYG{n}{show}\PYG{p}{(}\PYG{p}{)}
\end{Verbatim}

\end{fulllineitems}

\index{zeroBeforeStrNum() (in module acsSCCanalysis)}

\begin{fulllineitems}
\phantomsection\label{acsSCCanalysis:acsSCCanalysis.zeroBeforeStrNum}\pysiglinewithargsret{\code{acsSCCanalysis.}\bfcode{zeroBeforeStrNum}}{\emph{tmpl}, \emph{tmpL}}{}
\end{fulllineitems}

\index{zipf() (in module acsSCCanalysis)}

\begin{fulllineitems}
\phantomsection\label{acsSCCanalysis:acsSCCanalysis.zipf}\pysiglinewithargsret{\code{acsSCCanalysis.}\bfcode{zipf}}{\emph{a}, \emph{size=None}}{}
Draw samples from a Zipf distribution.

Samples are drawn from a Zipf distribution with specified parameter
\emph{a} \textgreater{} 1.

The Zipf distribution (also known as the zeta distribution) is a
continuous probability distribution that satisfies Zipf's law: the
frequency of an item is inversely proportional to its rank in a
frequency table.
\begin{description}
\item[{a}] \leavevmode{[}float \textgreater{} 1{]}
Distribution parameter.

\item[{size}] \leavevmode{[}int or tuple of int, optional{]}
Output shape.  If the given shape is, e.g., \code{(m, n, k)}, then
\code{m * n * k} samples are drawn; a single integer is equivalent in
its result to providing a mono-tuple, i.e., a 1-D array of length
\emph{size} is returned.  The default is None, in which case a single
scalar is returned.

\end{description}
\begin{description}
\item[{samples}] \leavevmode{[}scalar or ndarray{]}
The returned samples are greater than or equal to one.

\end{description}
\begin{description}
\item[{scipy.stats.distributions.zipf}] \leavevmode{[}probability density function,{]}
distribution, or cumulative density function, etc.

\end{description}

The probability density for the Zipf distribution is
\begin{gather}
\begin{split}p(x) = \frac{x^{-a}}{\zeta(a)},\end{split}\notag
\end{gather}
where \(\zeta\) is the Riemann Zeta function.

It is named for the American linguist George Kingsley Zipf, who noted
that the frequency of any word in a sample of a language is inversely
proportional to its rank in the frequency table.

Zipf, G. K., \emph{Selected Studies of the Principle of Relative Frequency
in Language}, Cambridge, MA: Harvard Univ. Press, 1932.

Draw samples from the distribution:

\begin{Verbatim}[commandchars=\\\{\}]
\PYG{g+gp}{\PYGZgt{}\PYGZgt{}\PYGZgt{} }\PYG{n}{a} \PYG{o}{=} \PYG{l+m+mf}{2.} \PYG{c}{\PYGZsh{} parameter}
\PYG{g+gp}{\PYGZgt{}\PYGZgt{}\PYGZgt{} }\PYG{n}{s} \PYG{o}{=} \PYG{n}{np}\PYG{o}{.}\PYG{n}{random}\PYG{o}{.}\PYG{n}{zipf}\PYG{p}{(}\PYG{n}{a}\PYG{p}{,} \PYG{l+m+mi}{1000}\PYG{p}{)}
\end{Verbatim}

Display the histogram of the samples, along with
the probability density function:

\begin{Verbatim}[commandchars=\\\{\}]
\PYG{g+gp}{\PYGZgt{}\PYGZgt{}\PYGZgt{} }\PYG{k+kn}{import} \PYG{n+nn}{matplotlib.pyplot} \PYG{k+kn}{as} \PYG{n+nn}{plt}
\PYG{g+gp}{\PYGZgt{}\PYGZgt{}\PYGZgt{} }\PYG{k+kn}{import} \PYG{n+nn}{scipy.special} \PYG{k+kn}{as} \PYG{n+nn}{sps}
\PYG{g+go}{Truncate s values at 50 so plot is interesting}
\PYG{g+gp}{\PYGZgt{}\PYGZgt{}\PYGZgt{} }\PYG{n}{count}\PYG{p}{,} \PYG{n}{bins}\PYG{p}{,} \PYG{n}{ignored} \PYG{o}{=} \PYG{n}{plt}\PYG{o}{.}\PYG{n}{hist}\PYG{p}{(}\PYG{n}{s}\PYG{p}{[}\PYG{n}{s}\PYG{o}{\PYGZlt{}}\PYG{l+m+mi}{50}\PYG{p}{]}\PYG{p}{,} \PYG{l+m+mi}{50}\PYG{p}{,} \PYG{n}{normed}\PYG{o}{=}\PYG{n+nb+bp}{True}\PYG{p}{)}
\PYG{g+gp}{\PYGZgt{}\PYGZgt{}\PYGZgt{} }\PYG{n}{x} \PYG{o}{=} \PYG{n}{np}\PYG{o}{.}\PYG{n}{arange}\PYG{p}{(}\PYG{l+m+mf}{1.}\PYG{p}{,} \PYG{l+m+mf}{50.}\PYG{p}{)}
\PYG{g+gp}{\PYGZgt{}\PYGZgt{}\PYGZgt{} }\PYG{n}{y} \PYG{o}{=} \PYG{n}{x}\PYG{o}{*}\PYG{o}{*}\PYG{p}{(}\PYG{o}{\PYGZhy{}}\PYG{n}{a}\PYG{p}{)}\PYG{o}{/}\PYG{n}{sps}\PYG{o}{.}\PYG{n}{zetac}\PYG{p}{(}\PYG{n}{a}\PYG{p}{)}
\PYG{g+gp}{\PYGZgt{}\PYGZgt{}\PYGZgt{} }\PYG{n}{plt}\PYG{o}{.}\PYG{n}{plot}\PYG{p}{(}\PYG{n}{x}\PYG{p}{,} \PYG{n}{y}\PYG{o}{/}\PYG{n+nb}{max}\PYG{p}{(}\PYG{n}{y}\PYG{p}{)}\PYG{p}{,} \PYG{n}{linewidth}\PYG{o}{=}\PYG{l+m+mi}{2}\PYG{p}{,} \PYG{n}{color}\PYG{o}{=}\PYG{l+s}{\PYGZsq{}}\PYG{l+s}{r}\PYG{l+s}{\PYGZsq{}}\PYG{p}{)}
\PYG{g+gp}{\PYGZgt{}\PYGZgt{}\PYGZgt{} }\PYG{n}{plt}\PYG{o}{.}\PYG{n}{show}\PYG{p}{(}\PYG{p}{)}
\end{Verbatim}

\end{fulllineitems}



\chapter{acsSpeciesActivities Module}
\label{acsSpeciesActivities:module-acsSpeciesActivities}\label{acsSpeciesActivities::doc}\label{acsSpeciesActivities:acsspeciesactivities-module}\index{acsSpeciesActivities (module)}
Function to evaluate the activity of each species during the simulation, 
catalyst substrate product or nothing. Moreover the script recognize all those molecules functioning as hub
\index{beta() (in module acsSpeciesActivities)}

\begin{fulllineitems}
\phantomsection\label{acsSpeciesActivities:acsSpeciesActivities.beta}\pysiglinewithargsret{\code{acsSpeciesActivities.}\bfcode{beta}}{\emph{a}, \emph{b}, \emph{size=None}}{}
The Beta distribution over \code{{[}0, 1{]}}.

The Beta distribution is a special case of the Dirichlet distribution,
and is related to the Gamma distribution.  It has the probability
distribution function
\begin{gather}
\begin{split}f(x; a,b) = \frac{1}{B(\alpha, \beta)} x^{\alpha - 1}
(1 - x)^{\beta - 1},\end{split}\notag
\end{gather}
where the normalisation, B, is the beta function,
\begin{gather}
\begin{split}B(\alpha, \beta) = \int_0^1 t^{\alpha - 1}
(1 - t)^{\beta - 1} dt.\end{split}\notag
\end{gather}
It is often seen in Bayesian inference and order statistics.
\begin{description}
\item[{a}] \leavevmode{[}float{]}
Alpha, non-negative.

\item[{b}] \leavevmode{[}float{]}
Beta, non-negative.

\item[{size}] \leavevmode{[}tuple of ints, optional{]}
The number of samples to draw.  The output is packed according to
the size given.

\end{description}
\begin{description}
\item[{out}] \leavevmode{[}ndarray{]}
Array of the given shape, containing values drawn from a
Beta distribution.

\end{description}

\end{fulllineitems}

\index{binomial() (in module acsSpeciesActivities)}

\begin{fulllineitems}
\phantomsection\label{acsSpeciesActivities:acsSpeciesActivities.binomial}\pysiglinewithargsret{\code{acsSpeciesActivities.}\bfcode{binomial}}{\emph{n}, \emph{p}, \emph{size=None}}{}
Draw samples from a binomial distribution.

Samples are drawn from a Binomial distribution with specified
parameters, n trials and p probability of success where
n an integer \textgreater{}= 0 and p is in the interval {[}0,1{]}. (n may be
input as a float, but it is truncated to an integer in use)
\begin{description}
\item[{n}] \leavevmode{[}float (but truncated to an integer){]}
parameter, \textgreater{}= 0.

\item[{p}] \leavevmode{[}float{]}
parameter, \textgreater{}= 0 and \textless{}=1.

\item[{size}] \leavevmode{[}\{tuple, int\}{]}
Output shape.  If the given shape is, e.g., \code{(m, n, k)}, then
\code{m * n * k} samples are drawn.

\end{description}
\begin{description}
\item[{samples}] \leavevmode{[}\{ndarray, scalar\}{]}
where the values are all integers in  {[}0, n{]}.

\end{description}
\begin{description}
\item[{scipy.stats.distributions.binom}] \leavevmode{[}probability density function,{]}
distribution or cumulative density function, etc.

\end{description}

The probability density for the Binomial distribution is
\begin{gather}
\begin{split}P(N) = \binom{n}{N}p^N(1-p)^{n-N},\end{split}\notag
\end{gather}
where \(n\) is the number of trials, \(p\) is the probability
of success, and \(N\) is the number of successes.

When estimating the standard error of a proportion in a population by
using a random sample, the normal distribution works well unless the
product p*n \textless{}=5, where p = population proportion estimate, and n =
number of samples, in which case the binomial distribution is used
instead. For example, a sample of 15 people shows 4 who are left
handed, and 11 who are right handed. Then p = 4/15 = 27\%. 0.27*15 = 4,
so the binomial distribution should be used in this case.

Draw samples from the distribution:

\begin{Verbatim}[commandchars=\\\{\}]
\PYG{g+gp}{\PYGZgt{}\PYGZgt{}\PYGZgt{} }\PYG{n}{n}\PYG{p}{,} \PYG{n}{p} \PYG{o}{=} \PYG{l+m+mi}{10}\PYG{p}{,} \PYG{o}{.}\PYG{l+m+mi}{5} \PYG{c}{\PYGZsh{} number of trials, probability of each trial}
\PYG{g+gp}{\PYGZgt{}\PYGZgt{}\PYGZgt{} }\PYG{n}{s} \PYG{o}{=} \PYG{n}{np}\PYG{o}{.}\PYG{n}{random}\PYG{o}{.}\PYG{n}{binomial}\PYG{p}{(}\PYG{n}{n}\PYG{p}{,} \PYG{n}{p}\PYG{p}{,} \PYG{l+m+mi}{1000}\PYG{p}{)}
\PYG{g+go}{\PYGZsh{} result of flipping a coin 10 times, tested 1000 times.}
\end{Verbatim}

A real world example. A company drills 9 wild-cat oil exploration
wells, each with an estimated probability of success of 0.1. All nine
wells fail. What is the probability of that happening?

Let's do 20,000 trials of the model, and count the number that
generate zero positive results.

\begin{Verbatim}[commandchars=\\\{\}]
\PYG{g+gp}{\PYGZgt{}\PYGZgt{}\PYGZgt{} }\PYG{n+nb}{sum}\PYG{p}{(}\PYG{n}{np}\PYG{o}{.}\PYG{n}{random}\PYG{o}{.}\PYG{n}{binomial}\PYG{p}{(}\PYG{l+m+mi}{9}\PYG{p}{,}\PYG{l+m+mf}{0.1}\PYG{p}{,}\PYG{l+m+mi}{20000}\PYG{p}{)}\PYG{o}{==}\PYG{l+m+mi}{0}\PYG{p}{)}\PYG{o}{/}\PYG{l+m+mf}{20000.}
\PYG{g+go}{answer = 0.38885, or 38\PYGZpc{}.}
\end{Verbatim}

\end{fulllineitems}

\index{chisquare() (in module acsSpeciesActivities)}

\begin{fulllineitems}
\phantomsection\label{acsSpeciesActivities:acsSpeciesActivities.chisquare}\pysiglinewithargsret{\code{acsSpeciesActivities.}\bfcode{chisquare}}{\emph{df}, \emph{size=None}}{}
Draw samples from a chi-square distribution.

When \emph{df} independent random variables, each with standard normal
distributions (mean 0, variance 1), are squared and summed, the
resulting distribution is chi-square (see Notes).  This distribution
is often used in hypothesis testing.
\begin{description}
\item[{df}] \leavevmode{[}int{]}
Number of degrees of freedom.

\item[{size}] \leavevmode{[}tuple of ints, int, optional{]}
Size of the returned array.  By default, a scalar is
returned.

\end{description}
\begin{description}
\item[{output}] \leavevmode{[}ndarray{]}
Samples drawn from the distribution, packed in a \emph{size}-shaped
array.

\end{description}
\begin{description}
\item[{ValueError}] \leavevmode
When \emph{df} \textless{}= 0 or when an inappropriate \emph{size} (e.g. \code{size=-1})
is given.

\end{description}

The variable obtained by summing the squares of \emph{df} independent,
standard normally distributed random variables:
\begin{gather}
\begin{split}Q = \sum_{i=0}^{\mathtt{df}} X^2_i\end{split}\notag
\end{gather}
is chi-square distributed, denoted
\begin{gather}
\begin{split}Q \sim \chi^2_k.\end{split}\notag
\end{gather}
The probability density function of the chi-squared distribution is
\begin{gather}
\begin{split}p(x) = \frac{(1/2)^{k/2}}{\Gamma(k/2)}
x^{k/2 - 1} e^{-x/2},\end{split}\notag
\end{gather}
where \(\Gamma\) is the gamma function,
\begin{gather}
\begin{split}\Gamma(x) = \int_0^{-\infty} t^{x - 1} e^{-t} dt.\end{split}\notag
\end{gather}
\href{http://www.itl.nist.gov/div898/handbook/eda/section3/eda3666.htm}{NIST/SEMATECH e-Handbook of Statistical Methods}

\begin{Verbatim}[commandchars=\\\{\}]
\PYG{g+gp}{\PYGZgt{}\PYGZgt{}\PYGZgt{} }\PYG{n}{np}\PYG{o}{.}\PYG{n}{random}\PYG{o}{.}\PYG{n}{chisquare}\PYG{p}{(}\PYG{l+m+mi}{2}\PYG{p}{,}\PYG{l+m+mi}{4}\PYG{p}{)}
\PYG{g+go}{array([ 1.89920014,  9.00867716,  3.13710533,  5.62318272])}
\end{Verbatim}

\end{fulllineitems}

\index{exponential() (in module acsSpeciesActivities)}

\begin{fulllineitems}
\phantomsection\label{acsSpeciesActivities:acsSpeciesActivities.exponential}\pysiglinewithargsret{\code{acsSpeciesActivities.}\bfcode{exponential}}{\emph{scale=1.0}, \emph{size=None}}{}
Exponential distribution.

Its probability density function is
\begin{gather}
\begin{split}f(x; \frac{1}{\beta}) = \frac{1}{\beta} \exp(-\frac{x}{\beta}),\end{split}\notag
\end{gather}
for \code{x \textgreater{} 0} and 0 elsewhere. \(\beta\) is the scale parameter,
which is the inverse of the rate parameter \(\lambda = 1/\beta\).
The rate parameter is an alternative, widely used parameterization
of the exponential distribution {\color{red}\bfseries{}{[}3{]}\_}.

The exponential distribution is a continuous analogue of the
geometric distribution.  It describes many common situations, such as
the size of raindrops measured over many rainstorms {\color{red}\bfseries{}{[}1{]}\_}, or the time
between page requests to Wikipedia {\color{red}\bfseries{}{[}2{]}\_}.
\begin{description}
\item[{scale}] \leavevmode{[}float{]}
The scale parameter, \(\beta = 1/\lambda\).

\item[{size}] \leavevmode{[}tuple of ints{]}
Number of samples to draw.  The output is shaped
according to \emph{size}.

\end{description}

\end{fulllineitems}

\index{f() (in module acsSpeciesActivities)}

\begin{fulllineitems}
\phantomsection\label{acsSpeciesActivities:acsSpeciesActivities.f}\pysiglinewithargsret{\code{acsSpeciesActivities.}\bfcode{f}}{\emph{dfnum}, \emph{dfden}, \emph{size=None}}{}
Draw samples from a F distribution.

Samples are drawn from an F distribution with specified parameters,
\emph{dfnum} (degrees of freedom in numerator) and \emph{dfden} (degrees of freedom
in denominator), where both parameters should be greater than zero.

The random variate of the F distribution (also known as the
Fisher distribution) is a continuous probability distribution
that arises in ANOVA tests, and is the ratio of two chi-square
variates.
\begin{description}
\item[{dfnum}] \leavevmode{[}float{]}
Degrees of freedom in numerator. Should be greater than zero.

\item[{dfden}] \leavevmode{[}float{]}
Degrees of freedom in denominator. Should be greater than zero.

\item[{size}] \leavevmode{[}\{tuple, int\}, optional{]}
Output shape.  If the given shape is, e.g., \code{(m, n, k)},
then \code{m * n * k} samples are drawn. By default only one sample
is returned.

\end{description}
\begin{description}
\item[{samples}] \leavevmode{[}\{ndarray, scalar\}{]}
Samples from the Fisher distribution.

\end{description}
\begin{description}
\item[{scipy.stats.distributions.f}] \leavevmode{[}probability density function,{]}
distribution or cumulative density function, etc.

\end{description}

The F statistic is used to compare in-group variances to between-group
variances. Calculating the distribution depends on the sampling, and
so it is a function of the respective degrees of freedom in the
problem.  The variable \emph{dfnum} is the number of samples minus one, the
between-groups degrees of freedom, while \emph{dfden} is the within-groups
degrees of freedom, the sum of the number of samples in each group
minus the number of groups.

An example from Glantz{[}1{]}, pp 47-40.
Two groups, children of diabetics (25 people) and children from people
without diabetes (25 controls). Fasting blood glucose was measured,
case group had a mean value of 86.1, controls had a mean value of
82.2. Standard deviations were 2.09 and 2.49 respectively. Are these
data consistent with the null hypothesis that the parents diabetic
status does not affect their children's blood glucose levels?
Calculating the F statistic from the data gives a value of 36.01.

Draw samples from the distribution:

\begin{Verbatim}[commandchars=\\\{\}]
\PYG{g+gp}{\PYGZgt{}\PYGZgt{}\PYGZgt{} }\PYG{n}{dfnum} \PYG{o}{=} \PYG{l+m+mf}{1.} \PYG{c}{\PYGZsh{} between group degrees of freedom}
\PYG{g+gp}{\PYGZgt{}\PYGZgt{}\PYGZgt{} }\PYG{n}{dfden} \PYG{o}{=} \PYG{l+m+mf}{48.} \PYG{c}{\PYGZsh{} within groups degrees of freedom}
\PYG{g+gp}{\PYGZgt{}\PYGZgt{}\PYGZgt{} }\PYG{n}{s} \PYG{o}{=} \PYG{n}{np}\PYG{o}{.}\PYG{n}{random}\PYG{o}{.}\PYG{n}{f}\PYG{p}{(}\PYG{n}{dfnum}\PYG{p}{,} \PYG{n}{dfden}\PYG{p}{,} \PYG{l+m+mi}{1000}\PYG{p}{)}
\end{Verbatim}

The lower bound for the top 1\% of the samples is :

\begin{Verbatim}[commandchars=\\\{\}]
\PYG{g+gp}{\PYGZgt{}\PYGZgt{}\PYGZgt{} }\PYG{n}{sort}\PYG{p}{(}\PYG{n}{s}\PYG{p}{)}\PYG{p}{[}\PYG{o}{\PYGZhy{}}\PYG{l+m+mi}{10}\PYG{p}{]}
\PYG{g+go}{7.61988120985}
\end{Verbatim}

So there is about a 1\% chance that the F statistic will exceed 7.62,
the measured value is 36, so the null hypothesis is rejected at the 1\%
level.

\end{fulllineitems}

\index{gamma() (in module acsSpeciesActivities)}

\begin{fulllineitems}
\phantomsection\label{acsSpeciesActivities:acsSpeciesActivities.gamma}\pysiglinewithargsret{\code{acsSpeciesActivities.}\bfcode{gamma}}{\emph{shape}, \emph{scale=1.0}, \emph{size=None}}{}
Draw samples from a Gamma distribution.

Samples are drawn from a Gamma distribution with specified parameters,
\emph{shape} (sometimes designated ``k'') and \emph{scale} (sometimes designated
``theta''), where both parameters are \textgreater{} 0.
\begin{description}
\item[{shape}] \leavevmode{[}scalar \textgreater{} 0{]}
The shape of the gamma distribution.

\item[{scale}] \leavevmode{[}scalar \textgreater{} 0, optional{]}
The scale of the gamma distribution.  Default is equal to 1.

\item[{size}] \leavevmode{[}shape\_tuple, optional{]}
Output shape.  If the given shape is, e.g., \code{(m, n, k)}, then
\code{m * n * k} samples are drawn.

\end{description}
\begin{description}
\item[{out}] \leavevmode{[}ndarray, float{]}
Returns one sample unless \emph{size} parameter is specified.

\end{description}
\begin{description}
\item[{scipy.stats.distributions.gamma}] \leavevmode{[}probability density function,{]}
distribution or cumulative density function, etc.

\end{description}

The probability density for the Gamma distribution is
\begin{gather}
\begin{split}p(x) = x^{k-1}\frac{e^{-x/\theta}}{\theta^k\Gamma(k)},\end{split}\notag
\end{gather}
where \(k\) is the shape and \(\theta\) the scale,
and \(\Gamma\) is the Gamma function.

The Gamma distribution is often used to model the times to failure of
electronic components, and arises naturally in processes for which the
waiting times between Poisson distributed events are relevant.

Draw samples from the distribution:

\begin{Verbatim}[commandchars=\\\{\}]
\PYG{g+gp}{\PYGZgt{}\PYGZgt{}\PYGZgt{} }\PYG{n}{shape}\PYG{p}{,} \PYG{n}{scale} \PYG{o}{=} \PYG{l+m+mf}{2.}\PYG{p}{,} \PYG{l+m+mf}{2.} \PYG{c}{\PYGZsh{} mean and dispersion}
\PYG{g+gp}{\PYGZgt{}\PYGZgt{}\PYGZgt{} }\PYG{n}{s} \PYG{o}{=} \PYG{n}{np}\PYG{o}{.}\PYG{n}{random}\PYG{o}{.}\PYG{n}{gamma}\PYG{p}{(}\PYG{n}{shape}\PYG{p}{,} \PYG{n}{scale}\PYG{p}{,} \PYG{l+m+mi}{1000}\PYG{p}{)}
\end{Verbatim}

Display the histogram of the samples, along with
the probability density function:

\begin{Verbatim}[commandchars=\\\{\}]
\PYG{g+gp}{\PYGZgt{}\PYGZgt{}\PYGZgt{} }\PYG{k+kn}{import} \PYG{n+nn}{matplotlib.pyplot} \PYG{k+kn}{as} \PYG{n+nn}{plt}
\PYG{g+gp}{\PYGZgt{}\PYGZgt{}\PYGZgt{} }\PYG{k+kn}{import} \PYG{n+nn}{scipy.special} \PYG{k+kn}{as} \PYG{n+nn}{sps}
\PYG{g+gp}{\PYGZgt{}\PYGZgt{}\PYGZgt{} }\PYG{n}{count}\PYG{p}{,} \PYG{n}{bins}\PYG{p}{,} \PYG{n}{ignored} \PYG{o}{=} \PYG{n}{plt}\PYG{o}{.}\PYG{n}{hist}\PYG{p}{(}\PYG{n}{s}\PYG{p}{,} \PYG{l+m+mi}{50}\PYG{p}{,} \PYG{n}{normed}\PYG{o}{=}\PYG{n+nb+bp}{True}\PYG{p}{)}
\PYG{g+gp}{\PYGZgt{}\PYGZgt{}\PYGZgt{} }\PYG{n}{y} \PYG{o}{=} \PYG{n}{bins}\PYG{o}{*}\PYG{o}{*}\PYG{p}{(}\PYG{n}{shape}\PYG{o}{\PYGZhy{}}\PYG{l+m+mi}{1}\PYG{p}{)}\PYG{o}{*}\PYG{p}{(}\PYG{n}{np}\PYG{o}{.}\PYG{n}{exp}\PYG{p}{(}\PYG{o}{\PYGZhy{}}\PYG{n}{bins}\PYG{o}{/}\PYG{n}{scale}\PYG{p}{)} \PYG{o}{/}
\PYG{g+gp}{... }                     \PYG{p}{(}\PYG{n}{sps}\PYG{o}{.}\PYG{n}{gamma}\PYG{p}{(}\PYG{n}{shape}\PYG{p}{)}\PYG{o}{*}\PYG{n}{scale}\PYG{o}{*}\PYG{o}{*}\PYG{n}{shape}\PYG{p}{)}\PYG{p}{)}
\PYG{g+gp}{\PYGZgt{}\PYGZgt{}\PYGZgt{} }\PYG{n}{plt}\PYG{o}{.}\PYG{n}{plot}\PYG{p}{(}\PYG{n}{bins}\PYG{p}{,} \PYG{n}{y}\PYG{p}{,} \PYG{n}{linewidth}\PYG{o}{=}\PYG{l+m+mi}{2}\PYG{p}{,} \PYG{n}{color}\PYG{o}{=}\PYG{l+s}{\PYGZsq{}}\PYG{l+s}{r}\PYG{l+s}{\PYGZsq{}}\PYG{p}{)}
\PYG{g+gp}{\PYGZgt{}\PYGZgt{}\PYGZgt{} }\PYG{n}{plt}\PYG{o}{.}\PYG{n}{show}\PYG{p}{(}\PYG{p}{)}
\end{Verbatim}

\end{fulllineitems}

\index{geometric() (in module acsSpeciesActivities)}

\begin{fulllineitems}
\phantomsection\label{acsSpeciesActivities:acsSpeciesActivities.geometric}\pysiglinewithargsret{\code{acsSpeciesActivities.}\bfcode{geometric}}{\emph{p}, \emph{size=None}}{}
Draw samples from the geometric distribution.

Bernoulli trials are experiments with one of two outcomes:
success or failure (an example of such an experiment is flipping
a coin).  The geometric distribution models the number of trials
that must be run in order to achieve success.  It is therefore
supported on the positive integers, \code{k = 1, 2, ...}.

The probability mass function of the geometric distribution is
\begin{gather}
\begin{split}f(k) = (1 - p)^{k - 1} p\end{split}\notag
\end{gather}
where \emph{p} is the probability of success of an individual trial.
\begin{description}
\item[{p}] \leavevmode{[}float{]}
The probability of success of an individual trial.

\item[{size}] \leavevmode{[}tuple of ints{]}
Number of values to draw from the distribution.  The output
is shaped according to \emph{size}.

\end{description}
\begin{description}
\item[{out}] \leavevmode{[}ndarray{]}
Samples from the geometric distribution, shaped according to
\emph{size}.

\end{description}

Draw ten thousand values from the geometric distribution,
with the probability of an individual success equal to 0.35:

\begin{Verbatim}[commandchars=\\\{\}]
\PYG{g+gp}{\PYGZgt{}\PYGZgt{}\PYGZgt{} }\PYG{n}{z} \PYG{o}{=} \PYG{n}{np}\PYG{o}{.}\PYG{n}{random}\PYG{o}{.}\PYG{n}{geometric}\PYG{p}{(}\PYG{n}{p}\PYG{o}{=}\PYG{l+m+mf}{0.35}\PYG{p}{,} \PYG{n}{size}\PYG{o}{=}\PYG{l+m+mi}{10000}\PYG{p}{)}
\end{Verbatim}

How many trials succeeded after a single run?

\begin{Verbatim}[commandchars=\\\{\}]
\PYG{g+gp}{\PYGZgt{}\PYGZgt{}\PYGZgt{} }\PYG{p}{(}\PYG{n}{z} \PYG{o}{==} \PYG{l+m+mi}{1}\PYG{p}{)}\PYG{o}{.}\PYG{n}{sum}\PYG{p}{(}\PYG{p}{)} \PYG{o}{/} \PYG{l+m+mf}{10000.}
\PYG{g+go}{0.34889999999999999 \PYGZsh{}random}
\end{Verbatim}

\end{fulllineitems}

\index{get\_state() (in module acsSpeciesActivities)}

\begin{fulllineitems}
\phantomsection\label{acsSpeciesActivities:acsSpeciesActivities.get_state}\pysiglinewithargsret{\code{acsSpeciesActivities.}\bfcode{get\_state}}{}{}
Return a tuple representing the internal state of the generator.

For more details, see \emph{set\_state}.
\begin{description}
\item[{out}] \leavevmode{[}tuple(str, ndarray of 624 uints, int, int, float){]}
The returned tuple has the following items:
\begin{enumerate}
\item {} 
the string `MT19937'.

\item {} 
a 1-D array of 624 unsigned integer keys.

\item {} 
an integer \code{pos}.

\item {} 
an integer \code{has\_gauss}.

\item {} 
a float \code{cached\_gaussian}.

\end{enumerate}

\end{description}

set\_state

\emph{set\_state} and \emph{get\_state} are not needed to work with any of the
random distributions in NumPy. If the internal state is manually altered,
the user should know exactly what he/she is doing.

\end{fulllineitems}

\index{gumbel() (in module acsSpeciesActivities)}

\begin{fulllineitems}
\phantomsection\label{acsSpeciesActivities:acsSpeciesActivities.gumbel}\pysiglinewithargsret{\code{acsSpeciesActivities.}\bfcode{gumbel}}{\emph{loc=0.0}, \emph{scale=1.0}, \emph{size=None}}{}
Gumbel distribution.

Draw samples from a Gumbel distribution with specified location and scale.
For more information on the Gumbel distribution, see Notes and References
below.
\begin{description}
\item[{loc}] \leavevmode{[}float{]}
The location of the mode of the distribution.

\item[{scale}] \leavevmode{[}float{]}
The scale parameter of the distribution.

\item[{size}] \leavevmode{[}tuple of ints{]}
Output shape.  If the given shape is, e.g., \code{(m, n, k)}, then
\code{m * n * k} samples are drawn.

\end{description}
\begin{description}
\item[{out}] \leavevmode{[}ndarray{]}
The samples

\end{description}

scipy.stats.gumbel\_l
scipy.stats.gumbel\_r
scipy.stats.genextreme
\begin{quote}

probability density function, distribution, or cumulative density
function, etc. for each of the above
\end{quote}

weibull

The Gumbel (or Smallest Extreme Value (SEV) or the Smallest Extreme Value
Type I) distribution is one of a class of Generalized Extreme Value (GEV)
distributions used in modeling extreme value problems.  The Gumbel is a
special case of the Extreme Value Type I distribution for maximums from
distributions with ``exponential-like'' tails.

The probability density for the Gumbel distribution is
\begin{gather}
\begin{split}p(x) = \frac{e^{-(x - \mu)/ \beta}}{\beta} e^{ -e^{-(x - \mu)/
\beta}},\end{split}\notag
\end{gather}
where \(\mu\) is the mode, a location parameter, and \(\beta\) is
the scale parameter.

The Gumbel (named for German mathematician Emil Julius Gumbel) was used
very early in the hydrology literature, for modeling the occurrence of
flood events. It is also used for modeling maximum wind speed and rainfall
rates.  It is a ``fat-tailed'' distribution - the probability of an event in
the tail of the distribution is larger than if one used a Gaussian, hence
the surprisingly frequent occurrence of 100-year floods. Floods were
initially modeled as a Gaussian process, which underestimated the frequency
of extreme events.

It is one of a class of extreme value distributions, the Generalized
Extreme Value (GEV) distributions, which also includes the Weibull and
Frechet.

The function has a mean of \(\mu + 0.57721\beta\) and a variance of
\(\frac{\pi^2}{6}\beta^2\).

Gumbel, E. J., \emph{Statistics of Extremes}, New York: Columbia University
Press, 1958.

Reiss, R.-D. and Thomas, M., \emph{Statistical Analysis of Extreme Values from
Insurance, Finance, Hydrology and Other Fields}, Basel: Birkhauser Verlag,
2001.

Draw samples from the distribution:

\begin{Verbatim}[commandchars=\\\{\}]
\PYG{g+gp}{\PYGZgt{}\PYGZgt{}\PYGZgt{} }\PYG{n}{mu}\PYG{p}{,} \PYG{n}{beta} \PYG{o}{=} \PYG{l+m+mi}{0}\PYG{p}{,} \PYG{l+m+mf}{0.1} \PYG{c}{\PYGZsh{} location and scale}
\PYG{g+gp}{\PYGZgt{}\PYGZgt{}\PYGZgt{} }\PYG{n}{s} \PYG{o}{=} \PYG{n}{np}\PYG{o}{.}\PYG{n}{random}\PYG{o}{.}\PYG{n}{gumbel}\PYG{p}{(}\PYG{n}{mu}\PYG{p}{,} \PYG{n}{beta}\PYG{p}{,} \PYG{l+m+mi}{1000}\PYG{p}{)}
\end{Verbatim}

Display the histogram of the samples, along with
the probability density function:

\begin{Verbatim}[commandchars=\\\{\}]
\PYG{g+gp}{\PYGZgt{}\PYGZgt{}\PYGZgt{} }\PYG{k+kn}{import} \PYG{n+nn}{matplotlib.pyplot} \PYG{k+kn}{as} \PYG{n+nn}{plt}
\PYG{g+gp}{\PYGZgt{}\PYGZgt{}\PYGZgt{} }\PYG{n}{count}\PYG{p}{,} \PYG{n}{bins}\PYG{p}{,} \PYG{n}{ignored} \PYG{o}{=} \PYG{n}{plt}\PYG{o}{.}\PYG{n}{hist}\PYG{p}{(}\PYG{n}{s}\PYG{p}{,} \PYG{l+m+mi}{30}\PYG{p}{,} \PYG{n}{normed}\PYG{o}{=}\PYG{n+nb+bp}{True}\PYG{p}{)}
\PYG{g+gp}{\PYGZgt{}\PYGZgt{}\PYGZgt{} }\PYG{n}{plt}\PYG{o}{.}\PYG{n}{plot}\PYG{p}{(}\PYG{n}{bins}\PYG{p}{,} \PYG{p}{(}\PYG{l+m+mi}{1}\PYG{o}{/}\PYG{n}{beta}\PYG{p}{)}\PYG{o}{*}\PYG{n}{np}\PYG{o}{.}\PYG{n}{exp}\PYG{p}{(}\PYG{o}{\PYGZhy{}}\PYG{p}{(}\PYG{n}{bins} \PYG{o}{\PYGZhy{}} \PYG{n}{mu}\PYG{p}{)}\PYG{o}{/}\PYG{n}{beta}\PYG{p}{)}
\PYG{g+gp}{... }         \PYG{o}{*} \PYG{n}{np}\PYG{o}{.}\PYG{n}{exp}\PYG{p}{(} \PYG{o}{\PYGZhy{}}\PYG{n}{np}\PYG{o}{.}\PYG{n}{exp}\PYG{p}{(} \PYG{o}{\PYGZhy{}}\PYG{p}{(}\PYG{n}{bins} \PYG{o}{\PYGZhy{}} \PYG{n}{mu}\PYG{p}{)} \PYG{o}{/}\PYG{n}{beta}\PYG{p}{)} \PYG{p}{)}\PYG{p}{,}
\PYG{g+gp}{... }         \PYG{n}{linewidth}\PYG{o}{=}\PYG{l+m+mi}{2}\PYG{p}{,} \PYG{n}{color}\PYG{o}{=}\PYG{l+s}{\PYGZsq{}}\PYG{l+s}{r}\PYG{l+s}{\PYGZsq{}}\PYG{p}{)}
\PYG{g+gp}{\PYGZgt{}\PYGZgt{}\PYGZgt{} }\PYG{n}{plt}\PYG{o}{.}\PYG{n}{show}\PYG{p}{(}\PYG{p}{)}
\end{Verbatim}

Show how an extreme value distribution can arise from a Gaussian process
and compare to a Gaussian:

\begin{Verbatim}[commandchars=\\\{\}]
\PYG{g+gp}{\PYGZgt{}\PYGZgt{}\PYGZgt{} }\PYG{n}{means} \PYG{o}{=} \PYG{p}{[}\PYG{p}{]}
\PYG{g+gp}{\PYGZgt{}\PYGZgt{}\PYGZgt{} }\PYG{n}{maxima} \PYG{o}{=} \PYG{p}{[}\PYG{p}{]}
\PYG{g+gp}{\PYGZgt{}\PYGZgt{}\PYGZgt{} }\PYG{k}{for} \PYG{n}{i} \PYG{o+ow}{in} \PYG{n+nb}{range}\PYG{p}{(}\PYG{l+m+mi}{0}\PYG{p}{,}\PYG{l+m+mi}{1000}\PYG{p}{)} \PYG{p}{:}
\PYG{g+gp}{... }   \PYG{n}{a} \PYG{o}{=} \PYG{n}{np}\PYG{o}{.}\PYG{n}{random}\PYG{o}{.}\PYG{n}{normal}\PYG{p}{(}\PYG{n}{mu}\PYG{p}{,} \PYG{n}{beta}\PYG{p}{,} \PYG{l+m+mi}{1000}\PYG{p}{)}
\PYG{g+gp}{... }   \PYG{n}{means}\PYG{o}{.}\PYG{n}{append}\PYG{p}{(}\PYG{n}{a}\PYG{o}{.}\PYG{n}{mean}\PYG{p}{(}\PYG{p}{)}\PYG{p}{)}
\PYG{g+gp}{... }   \PYG{n}{maxima}\PYG{o}{.}\PYG{n}{append}\PYG{p}{(}\PYG{n}{a}\PYG{o}{.}\PYG{n}{max}\PYG{p}{(}\PYG{p}{)}\PYG{p}{)}
\PYG{g+gp}{\PYGZgt{}\PYGZgt{}\PYGZgt{} }\PYG{n}{count}\PYG{p}{,} \PYG{n}{bins}\PYG{p}{,} \PYG{n}{ignored} \PYG{o}{=} \PYG{n}{plt}\PYG{o}{.}\PYG{n}{hist}\PYG{p}{(}\PYG{n}{maxima}\PYG{p}{,} \PYG{l+m+mi}{30}\PYG{p}{,} \PYG{n}{normed}\PYG{o}{=}\PYG{n+nb+bp}{True}\PYG{p}{)}
\PYG{g+gp}{\PYGZgt{}\PYGZgt{}\PYGZgt{} }\PYG{n}{beta} \PYG{o}{=} \PYG{n}{np}\PYG{o}{.}\PYG{n}{std}\PYG{p}{(}\PYG{n}{maxima}\PYG{p}{)}\PYG{o}{*}\PYG{n}{np}\PYG{o}{.}\PYG{n}{pi}\PYG{o}{/}\PYG{n}{np}\PYG{o}{.}\PYG{n}{sqrt}\PYG{p}{(}\PYG{l+m+mi}{6}\PYG{p}{)}
\PYG{g+gp}{\PYGZgt{}\PYGZgt{}\PYGZgt{} }\PYG{n}{mu} \PYG{o}{=} \PYG{n}{np}\PYG{o}{.}\PYG{n}{mean}\PYG{p}{(}\PYG{n}{maxima}\PYG{p}{)} \PYG{o}{\PYGZhy{}} \PYG{l+m+mf}{0.57721}\PYG{o}{*}\PYG{n}{beta}
\PYG{g+gp}{\PYGZgt{}\PYGZgt{}\PYGZgt{} }\PYG{n}{plt}\PYG{o}{.}\PYG{n}{plot}\PYG{p}{(}\PYG{n}{bins}\PYG{p}{,} \PYG{p}{(}\PYG{l+m+mi}{1}\PYG{o}{/}\PYG{n}{beta}\PYG{p}{)}\PYG{o}{*}\PYG{n}{np}\PYG{o}{.}\PYG{n}{exp}\PYG{p}{(}\PYG{o}{\PYGZhy{}}\PYG{p}{(}\PYG{n}{bins} \PYG{o}{\PYGZhy{}} \PYG{n}{mu}\PYG{p}{)}\PYG{o}{/}\PYG{n}{beta}\PYG{p}{)}
\PYG{g+gp}{... }         \PYG{o}{*} \PYG{n}{np}\PYG{o}{.}\PYG{n}{exp}\PYG{p}{(}\PYG{o}{\PYGZhy{}}\PYG{n}{np}\PYG{o}{.}\PYG{n}{exp}\PYG{p}{(}\PYG{o}{\PYGZhy{}}\PYG{p}{(}\PYG{n}{bins} \PYG{o}{\PYGZhy{}} \PYG{n}{mu}\PYG{p}{)}\PYG{o}{/}\PYG{n}{beta}\PYG{p}{)}\PYG{p}{)}\PYG{p}{,}
\PYG{g+gp}{... }         \PYG{n}{linewidth}\PYG{o}{=}\PYG{l+m+mi}{2}\PYG{p}{,} \PYG{n}{color}\PYG{o}{=}\PYG{l+s}{\PYGZsq{}}\PYG{l+s}{r}\PYG{l+s}{\PYGZsq{}}\PYG{p}{)}
\PYG{g+gp}{\PYGZgt{}\PYGZgt{}\PYGZgt{} }\PYG{n}{plt}\PYG{o}{.}\PYG{n}{plot}\PYG{p}{(}\PYG{n}{bins}\PYG{p}{,} \PYG{l+m+mi}{1}\PYG{o}{/}\PYG{p}{(}\PYG{n}{beta} \PYG{o}{*} \PYG{n}{np}\PYG{o}{.}\PYG{n}{sqrt}\PYG{p}{(}\PYG{l+m+mi}{2} \PYG{o}{*} \PYG{n}{np}\PYG{o}{.}\PYG{n}{pi}\PYG{p}{)}\PYG{p}{)}
\PYG{g+gp}{... }         \PYG{o}{*} \PYG{n}{np}\PYG{o}{.}\PYG{n}{exp}\PYG{p}{(}\PYG{o}{\PYGZhy{}}\PYG{p}{(}\PYG{n}{bins} \PYG{o}{\PYGZhy{}} \PYG{n}{mu}\PYG{p}{)}\PYG{o}{*}\PYG{o}{*}\PYG{l+m+mi}{2} \PYG{o}{/} \PYG{p}{(}\PYG{l+m+mi}{2} \PYG{o}{*} \PYG{n}{beta}\PYG{o}{*}\PYG{o}{*}\PYG{l+m+mi}{2}\PYG{p}{)}\PYG{p}{)}\PYG{p}{,}
\PYG{g+gp}{... }         \PYG{n}{linewidth}\PYG{o}{=}\PYG{l+m+mi}{2}\PYG{p}{,} \PYG{n}{color}\PYG{o}{=}\PYG{l+s}{\PYGZsq{}}\PYG{l+s}{g}\PYG{l+s}{\PYGZsq{}}\PYG{p}{)}
\PYG{g+gp}{\PYGZgt{}\PYGZgt{}\PYGZgt{} }\PYG{n}{plt}\PYG{o}{.}\PYG{n}{show}\PYG{p}{(}\PYG{p}{)}
\end{Verbatim}

\end{fulllineitems}

\index{hypergeometric() (in module acsSpeciesActivities)}

\begin{fulllineitems}
\phantomsection\label{acsSpeciesActivities:acsSpeciesActivities.hypergeometric}\pysiglinewithargsret{\code{acsSpeciesActivities.}\bfcode{hypergeometric}}{\emph{ngood}, \emph{nbad}, \emph{nsample}, \emph{size=None}}{}
Draw samples from a Hypergeometric distribution.

Samples are drawn from a Hypergeometric distribution with specified
parameters, ngood (ways to make a good selection), nbad (ways to make
a bad selection), and nsample = number of items sampled, which is less
than or equal to the sum ngood + nbad.
\begin{description}
\item[{ngood}] \leavevmode{[}int or array\_like{]}
Number of ways to make a good selection.  Must be nonnegative.

\item[{nbad}] \leavevmode{[}int or array\_like{]}
Number of ways to make a bad selection.  Must be nonnegative.

\item[{nsample}] \leavevmode{[}int or array\_like{]}
Number of items sampled.  Must be at least 1 and at most
\code{ngood + nbad}.

\item[{size}] \leavevmode{[}int or tuple of int{]}
Output shape.  If the given shape is, e.g., \code{(m, n, k)}, then
\code{m * n * k} samples are drawn.

\end{description}
\begin{description}
\item[{samples}] \leavevmode{[}ndarray or scalar{]}
The values are all integers in  {[}0, n{]}.

\end{description}
\begin{description}
\item[{scipy.stats.distributions.hypergeom}] \leavevmode{[}probability density function,{]}
distribution or cumulative density function, etc.

\end{description}

The probability density for the Hypergeometric distribution is
\begin{gather}
\begin{split}P(x) = \frac{\binom{m}{n}\binom{N-m}{n-x}}{\binom{N}{n}},\end{split}\notag
\end{gather}
where \(0 \le x \le m\) and \(n+m-N \le x \le n\)

for P(x) the probability of x successes, n = ngood, m = nbad, and
N = number of samples.

Consider an urn with black and white marbles in it, ngood of them
black and nbad are white. If you draw nsample balls without
replacement, then the Hypergeometric distribution describes the
distribution of black balls in the drawn sample.

Note that this distribution is very similar to the Binomial
distribution, except that in this case, samples are drawn without
replacement, whereas in the Binomial case samples are drawn with
replacement (or the sample space is infinite). As the sample space
becomes large, this distribution approaches the Binomial.

Draw samples from the distribution:

\begin{Verbatim}[commandchars=\\\{\}]
\PYG{g+gp}{\PYGZgt{}\PYGZgt{}\PYGZgt{} }\PYG{n}{ngood}\PYG{p}{,} \PYG{n}{nbad}\PYG{p}{,} \PYG{n}{nsamp} \PYG{o}{=} \PYG{l+m+mi}{100}\PYG{p}{,} \PYG{l+m+mi}{2}\PYG{p}{,} \PYG{l+m+mi}{10}
\PYG{g+go}{\PYGZsh{} number of good, number of bad, and number of samples}
\PYG{g+gp}{\PYGZgt{}\PYGZgt{}\PYGZgt{} }\PYG{n}{s} \PYG{o}{=} \PYG{n}{np}\PYG{o}{.}\PYG{n}{random}\PYG{o}{.}\PYG{n}{hypergeometric}\PYG{p}{(}\PYG{n}{ngood}\PYG{p}{,} \PYG{n}{nbad}\PYG{p}{,} \PYG{n}{nsamp}\PYG{p}{,} \PYG{l+m+mi}{1000}\PYG{p}{)}
\PYG{g+gp}{\PYGZgt{}\PYGZgt{}\PYGZgt{} }\PYG{n}{hist}\PYG{p}{(}\PYG{n}{s}\PYG{p}{)}
\PYG{g+go}{\PYGZsh{}   note that it is very unlikely to grab both bad items}
\end{Verbatim}

Suppose you have an urn with 15 white and 15 black marbles.
If you pull 15 marbles at random, how likely is it that
12 or more of them are one color?

\begin{Verbatim}[commandchars=\\\{\}]
\PYG{g+gp}{\PYGZgt{}\PYGZgt{}\PYGZgt{} }\PYG{n}{s} \PYG{o}{=} \PYG{n}{np}\PYG{o}{.}\PYG{n}{random}\PYG{o}{.}\PYG{n}{hypergeometric}\PYG{p}{(}\PYG{l+m+mi}{15}\PYG{p}{,} \PYG{l+m+mi}{15}\PYG{p}{,} \PYG{l+m+mi}{15}\PYG{p}{,} \PYG{l+m+mi}{100000}\PYG{p}{)}
\PYG{g+gp}{\PYGZgt{}\PYGZgt{}\PYGZgt{} }\PYG{n+nb}{sum}\PYG{p}{(}\PYG{n}{s}\PYG{o}{\PYGZgt{}}\PYG{o}{=}\PYG{l+m+mi}{12}\PYG{p}{)}\PYG{o}{/}\PYG{l+m+mf}{100000.} \PYG{o}{+} \PYG{n+nb}{sum}\PYG{p}{(}\PYG{n}{s}\PYG{o}{\PYGZlt{}}\PYG{o}{=}\PYG{l+m+mi}{3}\PYG{p}{)}\PYG{o}{/}\PYG{l+m+mf}{100000.}
\PYG{g+go}{\PYGZsh{}   answer = 0.003 ... pretty unlikely!}
\end{Verbatim}

\end{fulllineitems}

\index{laplace() (in module acsSpeciesActivities)}

\begin{fulllineitems}
\phantomsection\label{acsSpeciesActivities:acsSpeciesActivities.laplace}\pysiglinewithargsret{\code{acsSpeciesActivities.}\bfcode{laplace}}{\emph{loc=0.0}, \emph{scale=1.0}, \emph{size=None}}{}
Draw samples from the Laplace or double exponential distribution with
specified location (or mean) and scale (decay).

The Laplace distribution is similar to the Gaussian/normal distribution,
but is sharper at the peak and has fatter tails. It represents the
difference between two independent, identically distributed exponential
random variables.
\begin{description}
\item[{loc}] \leavevmode{[}float{]}
The position, \(\mu\), of the distribution peak.

\item[{scale}] \leavevmode{[}float{]}
\(\lambda\), the exponential decay.

\end{description}

It has the probability density function
\begin{gather}
\begin{split}f(x; \mu, \lambda) = \frac{1}{2\lambda}
\exp\left(-\frac{|x - \mu|}{\lambda}\right).\end{split}\notag
\end{gather}
The first law of Laplace, from 1774, states that the frequency of an error
can be expressed as an exponential function of the absolute magnitude of
the error, which leads to the Laplace distribution. For many problems in
Economics and Health sciences, this distribution seems to model the data
better than the standard Gaussian distribution

Draw samples from the distribution

\begin{Verbatim}[commandchars=\\\{\}]
\PYG{g+gp}{\PYGZgt{}\PYGZgt{}\PYGZgt{} }\PYG{n}{loc}\PYG{p}{,} \PYG{n}{scale} \PYG{o}{=} \PYG{l+m+mf}{0.}\PYG{p}{,} \PYG{l+m+mf}{1.}
\PYG{g+gp}{\PYGZgt{}\PYGZgt{}\PYGZgt{} }\PYG{n}{s} \PYG{o}{=} \PYG{n}{np}\PYG{o}{.}\PYG{n}{random}\PYG{o}{.}\PYG{n}{laplace}\PYG{p}{(}\PYG{n}{loc}\PYG{p}{,} \PYG{n}{scale}\PYG{p}{,} \PYG{l+m+mi}{1000}\PYG{p}{)}
\end{Verbatim}

Display the histogram of the samples, along with
the probability density function:

\begin{Verbatim}[commandchars=\\\{\}]
\PYG{g+gp}{\PYGZgt{}\PYGZgt{}\PYGZgt{} }\PYG{k+kn}{import} \PYG{n+nn}{matplotlib.pyplot} \PYG{k+kn}{as} \PYG{n+nn}{plt}
\PYG{g+gp}{\PYGZgt{}\PYGZgt{}\PYGZgt{} }\PYG{n}{count}\PYG{p}{,} \PYG{n}{bins}\PYG{p}{,} \PYG{n}{ignored} \PYG{o}{=} \PYG{n}{plt}\PYG{o}{.}\PYG{n}{hist}\PYG{p}{(}\PYG{n}{s}\PYG{p}{,} \PYG{l+m+mi}{30}\PYG{p}{,} \PYG{n}{normed}\PYG{o}{=}\PYG{n+nb+bp}{True}\PYG{p}{)}
\PYG{g+gp}{\PYGZgt{}\PYGZgt{}\PYGZgt{} }\PYG{n}{x} \PYG{o}{=} \PYG{n}{np}\PYG{o}{.}\PYG{n}{arange}\PYG{p}{(}\PYG{o}{\PYGZhy{}}\PYG{l+m+mf}{8.}\PYG{p}{,} \PYG{l+m+mf}{8.}\PYG{p}{,} \PYG{o}{.}\PYG{l+m+mo}{01}\PYG{p}{)}
\PYG{g+gp}{\PYGZgt{}\PYGZgt{}\PYGZgt{} }\PYG{n}{pdf} \PYG{o}{=} \PYG{n}{np}\PYG{o}{.}\PYG{n}{exp}\PYG{p}{(}\PYG{o}{\PYGZhy{}}\PYG{n+nb}{abs}\PYG{p}{(}\PYG{n}{x}\PYG{o}{\PYGZhy{}}\PYG{n}{loc}\PYG{o}{/}\PYG{n}{scale}\PYG{p}{)}\PYG{p}{)}\PYG{o}{/}\PYG{p}{(}\PYG{l+m+mf}{2.}\PYG{o}{*}\PYG{n}{scale}\PYG{p}{)}
\PYG{g+gp}{\PYGZgt{}\PYGZgt{}\PYGZgt{} }\PYG{n}{plt}\PYG{o}{.}\PYG{n}{plot}\PYG{p}{(}\PYG{n}{x}\PYG{p}{,} \PYG{n}{pdf}\PYG{p}{)}
\end{Verbatim}

Plot Gaussian for comparison:

\begin{Verbatim}[commandchars=\\\{\}]
\PYG{g+gp}{\PYGZgt{}\PYGZgt{}\PYGZgt{} }\PYG{n}{g} \PYG{o}{=} \PYG{p}{(}\PYG{l+m+mi}{1}\PYG{o}{/}\PYG{p}{(}\PYG{n}{scale} \PYG{o}{*} \PYG{n}{np}\PYG{o}{.}\PYG{n}{sqrt}\PYG{p}{(}\PYG{l+m+mi}{2} \PYG{o}{*} \PYG{n}{np}\PYG{o}{.}\PYG{n}{pi}\PYG{p}{)}\PYG{p}{)} \PYG{o}{*} 
\PYG{g+gp}{... }     \PYG{n}{np}\PYG{o}{.}\PYG{n}{exp}\PYG{p}{(} \PYG{o}{\PYGZhy{}} \PYG{p}{(}\PYG{n}{x} \PYG{o}{\PYGZhy{}} \PYG{n}{loc}\PYG{p}{)}\PYG{o}{*}\PYG{o}{*}\PYG{l+m+mi}{2} \PYG{o}{/} \PYG{p}{(}\PYG{l+m+mi}{2} \PYG{o}{*} \PYG{n}{scale}\PYG{o}{*}\PYG{o}{*}\PYG{l+m+mi}{2}\PYG{p}{)} \PYG{p}{)}\PYG{p}{)}
\PYG{g+gp}{\PYGZgt{}\PYGZgt{}\PYGZgt{} }\PYG{n}{plt}\PYG{o}{.}\PYG{n}{plot}\PYG{p}{(}\PYG{n}{x}\PYG{p}{,}\PYG{n}{g}\PYG{p}{)}
\end{Verbatim}

\end{fulllineitems}

\index{logistic() (in module acsSpeciesActivities)}

\begin{fulllineitems}
\phantomsection\label{acsSpeciesActivities:acsSpeciesActivities.logistic}\pysiglinewithargsret{\code{acsSpeciesActivities.}\bfcode{logistic}}{\emph{loc=0.0}, \emph{scale=1.0}, \emph{size=None}}{}
Draw samples from a Logistic distribution.

Samples are drawn from a Logistic distribution with specified
parameters, loc (location or mean, also median), and scale (\textgreater{}0).

loc : float

scale : float \textgreater{} 0.
\begin{description}
\item[{size}] \leavevmode{[}\{tuple, int\}{]}
Output shape.  If the given shape is, e.g., \code{(m, n, k)}, then
\code{m * n * k} samples are drawn.

\end{description}
\begin{description}
\item[{samples}] \leavevmode{[}\{ndarray, scalar\}{]}
where the values are all integers in  {[}0, n{]}.

\end{description}
\begin{description}
\item[{scipy.stats.distributions.logistic}] \leavevmode{[}probability density function,{]}
distribution or cumulative density function, etc.

\end{description}

The probability density for the Logistic distribution is
\begin{gather}
\begin{split}P(x) = P(x) = \frac{e^{-(x-\mu)/s}}{s(1+e^{-(x-\mu)/s})^2},\end{split}\notag
\end{gather}
where \(\mu\) = location and \(s\) = scale.

The Logistic distribution is used in Extreme Value problems where it
can act as a mixture of Gumbel distributions, in Epidemiology, and by
the World Chess Federation (FIDE) where it is used in the Elo ranking
system, assuming the performance of each player is a logistically
distributed random variable.

Draw samples from the distribution:

\begin{Verbatim}[commandchars=\\\{\}]
\PYG{g+gp}{\PYGZgt{}\PYGZgt{}\PYGZgt{} }\PYG{n}{loc}\PYG{p}{,} \PYG{n}{scale} \PYG{o}{=} \PYG{l+m+mi}{10}\PYG{p}{,} \PYG{l+m+mi}{1}
\PYG{g+gp}{\PYGZgt{}\PYGZgt{}\PYGZgt{} }\PYG{n}{s} \PYG{o}{=} \PYG{n}{np}\PYG{o}{.}\PYG{n}{random}\PYG{o}{.}\PYG{n}{logistic}\PYG{p}{(}\PYG{n}{loc}\PYG{p}{,} \PYG{n}{scale}\PYG{p}{,} \PYG{l+m+mi}{10000}\PYG{p}{)}
\PYG{g+gp}{\PYGZgt{}\PYGZgt{}\PYGZgt{} }\PYG{n}{count}\PYG{p}{,} \PYG{n}{bins}\PYG{p}{,} \PYG{n}{ignored} \PYG{o}{=} \PYG{n}{plt}\PYG{o}{.}\PYG{n}{hist}\PYG{p}{(}\PYG{n}{s}\PYG{p}{,} \PYG{n}{bins}\PYG{o}{=}\PYG{l+m+mi}{50}\PYG{p}{)}
\end{Verbatim}

\#   plot against distribution

\begin{Verbatim}[commandchars=\\\{\}]
\PYG{g+gp}{\PYGZgt{}\PYGZgt{}\PYGZgt{} }\PYG{k}{def} \PYG{n+nf}{logist}\PYG{p}{(}\PYG{n}{x}\PYG{p}{,} \PYG{n}{loc}\PYG{p}{,} \PYG{n}{scale}\PYG{p}{)}\PYG{p}{:}
\PYG{g+gp}{... }    \PYG{k}{return} \PYG{n}{exp}\PYG{p}{(}\PYG{p}{(}\PYG{n}{loc}\PYG{o}{\PYGZhy{}}\PYG{n}{x}\PYG{p}{)}\PYG{o}{/}\PYG{n}{scale}\PYG{p}{)}\PYG{o}{/}\PYG{p}{(}\PYG{n}{scale}\PYG{o}{*}\PYG{p}{(}\PYG{l+m+mi}{1}\PYG{o}{+}\PYG{n}{exp}\PYG{p}{(}\PYG{p}{(}\PYG{n}{loc}\PYG{o}{\PYGZhy{}}\PYG{n}{x}\PYG{p}{)}\PYG{o}{/}\PYG{n}{scale}\PYG{p}{)}\PYG{p}{)}\PYG{o}{*}\PYG{o}{*}\PYG{l+m+mi}{2}\PYG{p}{)}
\PYG{g+gp}{\PYGZgt{}\PYGZgt{}\PYGZgt{} }\PYG{n}{plt}\PYG{o}{.}\PYG{n}{plot}\PYG{p}{(}\PYG{n}{bins}\PYG{p}{,} \PYG{n}{logist}\PYG{p}{(}\PYG{n}{bins}\PYG{p}{,} \PYG{n}{loc}\PYG{p}{,} \PYG{n}{scale}\PYG{p}{)}\PYG{o}{*}\PYG{n}{count}\PYG{o}{.}\PYG{n}{max}\PYG{p}{(}\PYG{p}{)}\PYG{o}{/}\PYGZbs{}
\PYG{g+gp}{... }\PYG{n}{logist}\PYG{p}{(}\PYG{n}{bins}\PYG{p}{,} \PYG{n}{loc}\PYG{p}{,} \PYG{n}{scale}\PYG{p}{)}\PYG{o}{.}\PYG{n}{max}\PYG{p}{(}\PYG{p}{)}\PYG{p}{)}
\PYG{g+gp}{\PYGZgt{}\PYGZgt{}\PYGZgt{} }\PYG{n}{plt}\PYG{o}{.}\PYG{n}{show}\PYG{p}{(}\PYG{p}{)}
\end{Verbatim}

\end{fulllineitems}

\index{lognormal() (in module acsSpeciesActivities)}

\begin{fulllineitems}
\phantomsection\label{acsSpeciesActivities:acsSpeciesActivities.lognormal}\pysiglinewithargsret{\code{acsSpeciesActivities.}\bfcode{lognormal}}{\emph{mean=0.0}, \emph{sigma=1.0}, \emph{size=None}}{}
Return samples drawn from a log-normal distribution.

Draw samples from a log-normal distribution with specified mean,
standard deviation, and array shape.  Note that the mean and standard
deviation are not the values for the distribution itself, but of the
underlying normal distribution it is derived from.
\begin{description}
\item[{mean}] \leavevmode{[}float{]}
Mean value of the underlying normal distribution

\item[{sigma}] \leavevmode{[}float, \textgreater{} 0.{]}
Standard deviation of the underlying normal distribution

\item[{size}] \leavevmode{[}tuple of ints{]}
Output shape.  If the given shape is, e.g., \code{(m, n, k)}, then
\code{m * n * k} samples are drawn.

\end{description}
\begin{description}
\item[{samples}] \leavevmode{[}ndarray or float{]}
The desired samples. An array of the same shape as \emph{size} if given,
if \emph{size} is None a float is returned.

\end{description}
\begin{description}
\item[{scipy.stats.lognorm}] \leavevmode{[}probability density function, distribution,{]}
cumulative density function, etc.

\end{description}

A variable \emph{x} has a log-normal distribution if \emph{log(x)} is normally
distributed.  The probability density function for the log-normal
distribution is:
\begin{gather}
\begin{split}p(x) = \frac{1}{\sigma x \sqrt{2\pi}}
e^{(-\frac{(ln(x)-\mu)^2}{2\sigma^2})}\end{split}\notag
\end{gather}
where \(\mu\) is the mean and \(\sigma\) is the standard
deviation of the normally distributed logarithm of the variable.
A log-normal distribution results if a random variable is the \emph{product}
of a large number of independent, identically-distributed variables in
the same way that a normal distribution results if the variable is the
\emph{sum} of a large number of independent, identically-distributed
variables.

Limpert, E., Stahel, W. A., and Abbt, M., ``Log-normal Distributions
across the Sciences: Keys and Clues,'' \emph{BioScience}, Vol. 51, No. 5,
May, 2001.  \href{http://stat.ethz.ch/~stahel/lognormal/bioscience.pdf}{http://stat.ethz.ch/\textasciitilde{}stahel/lognormal/bioscience.pdf}

Reiss, R.D. and Thomas, M., \emph{Statistical Analysis of Extreme Values},
Basel: Birkhauser Verlag, 2001, pp. 31-32.

Draw samples from the distribution:

\begin{Verbatim}[commandchars=\\\{\}]
\PYG{g+gp}{\PYGZgt{}\PYGZgt{}\PYGZgt{} }\PYG{n}{mu}\PYG{p}{,} \PYG{n}{sigma} \PYG{o}{=} \PYG{l+m+mf}{3.}\PYG{p}{,} \PYG{l+m+mf}{1.} \PYG{c}{\PYGZsh{} mean and standard deviation}
\PYG{g+gp}{\PYGZgt{}\PYGZgt{}\PYGZgt{} }\PYG{n}{s} \PYG{o}{=} \PYG{n}{np}\PYG{o}{.}\PYG{n}{random}\PYG{o}{.}\PYG{n}{lognormal}\PYG{p}{(}\PYG{n}{mu}\PYG{p}{,} \PYG{n}{sigma}\PYG{p}{,} \PYG{l+m+mi}{1000}\PYG{p}{)}
\end{Verbatim}

Display the histogram of the samples, along with
the probability density function:

\begin{Verbatim}[commandchars=\\\{\}]
\PYG{g+gp}{\PYGZgt{}\PYGZgt{}\PYGZgt{} }\PYG{k+kn}{import} \PYG{n+nn}{matplotlib.pyplot} \PYG{k+kn}{as} \PYG{n+nn}{plt}
\PYG{g+gp}{\PYGZgt{}\PYGZgt{}\PYGZgt{} }\PYG{n}{count}\PYG{p}{,} \PYG{n}{bins}\PYG{p}{,} \PYG{n}{ignored} \PYG{o}{=} \PYG{n}{plt}\PYG{o}{.}\PYG{n}{hist}\PYG{p}{(}\PYG{n}{s}\PYG{p}{,} \PYG{l+m+mi}{100}\PYG{p}{,} \PYG{n}{normed}\PYG{o}{=}\PYG{n+nb+bp}{True}\PYG{p}{,} \PYG{n}{align}\PYG{o}{=}\PYG{l+s}{\PYGZsq{}}\PYG{l+s}{mid}\PYG{l+s}{\PYGZsq{}}\PYG{p}{)}
\end{Verbatim}

\begin{Verbatim}[commandchars=\\\{\}]
\PYG{g+gp}{\PYGZgt{}\PYGZgt{}\PYGZgt{} }\PYG{n}{x} \PYG{o}{=} \PYG{n}{np}\PYG{o}{.}\PYG{n}{linspace}\PYG{p}{(}\PYG{n+nb}{min}\PYG{p}{(}\PYG{n}{bins}\PYG{p}{)}\PYG{p}{,} \PYG{n+nb}{max}\PYG{p}{(}\PYG{n}{bins}\PYG{p}{)}\PYG{p}{,} \PYG{l+m+mi}{10000}\PYG{p}{)}
\PYG{g+gp}{\PYGZgt{}\PYGZgt{}\PYGZgt{} }\PYG{n}{pdf} \PYG{o}{=} \PYG{p}{(}\PYG{n}{np}\PYG{o}{.}\PYG{n}{exp}\PYG{p}{(}\PYG{o}{\PYGZhy{}}\PYG{p}{(}\PYG{n}{np}\PYG{o}{.}\PYG{n}{log}\PYG{p}{(}\PYG{n}{x}\PYG{p}{)} \PYG{o}{\PYGZhy{}} \PYG{n}{mu}\PYG{p}{)}\PYG{o}{*}\PYG{o}{*}\PYG{l+m+mi}{2} \PYG{o}{/} \PYG{p}{(}\PYG{l+m+mi}{2} \PYG{o}{*} \PYG{n}{sigma}\PYG{o}{*}\PYG{o}{*}\PYG{l+m+mi}{2}\PYG{p}{)}\PYG{p}{)}
\PYG{g+gp}{... }       \PYG{o}{/} \PYG{p}{(}\PYG{n}{x} \PYG{o}{*} \PYG{n}{sigma} \PYG{o}{*} \PYG{n}{np}\PYG{o}{.}\PYG{n}{sqrt}\PYG{p}{(}\PYG{l+m+mi}{2} \PYG{o}{*} \PYG{n}{np}\PYG{o}{.}\PYG{n}{pi}\PYG{p}{)}\PYG{p}{)}\PYG{p}{)}
\end{Verbatim}

\begin{Verbatim}[commandchars=\\\{\}]
\PYG{g+gp}{\PYGZgt{}\PYGZgt{}\PYGZgt{} }\PYG{n}{plt}\PYG{o}{.}\PYG{n}{plot}\PYG{p}{(}\PYG{n}{x}\PYG{p}{,} \PYG{n}{pdf}\PYG{p}{,} \PYG{n}{linewidth}\PYG{o}{=}\PYG{l+m+mi}{2}\PYG{p}{,} \PYG{n}{color}\PYG{o}{=}\PYG{l+s}{\PYGZsq{}}\PYG{l+s}{r}\PYG{l+s}{\PYGZsq{}}\PYG{p}{)}
\PYG{g+gp}{\PYGZgt{}\PYGZgt{}\PYGZgt{} }\PYG{n}{plt}\PYG{o}{.}\PYG{n}{axis}\PYG{p}{(}\PYG{l+s}{\PYGZsq{}}\PYG{l+s}{tight}\PYG{l+s}{\PYGZsq{}}\PYG{p}{)}
\PYG{g+gp}{\PYGZgt{}\PYGZgt{}\PYGZgt{} }\PYG{n}{plt}\PYG{o}{.}\PYG{n}{show}\PYG{p}{(}\PYG{p}{)}
\end{Verbatim}

Demonstrate that taking the products of random samples from a uniform
distribution can be fit well by a log-normal probability density function.

\begin{Verbatim}[commandchars=\\\{\}]
\PYG{g+gp}{\PYGZgt{}\PYGZgt{}\PYGZgt{} }\PYG{c}{\PYGZsh{} Generate a thousand samples: each is the product of 100 random}
\PYG{g+gp}{\PYGZgt{}\PYGZgt{}\PYGZgt{} }\PYG{c}{\PYGZsh{} values, drawn from a normal distribution.}
\PYG{g+gp}{\PYGZgt{}\PYGZgt{}\PYGZgt{} }\PYG{n}{b} \PYG{o}{=} \PYG{p}{[}\PYG{p}{]}
\PYG{g+gp}{\PYGZgt{}\PYGZgt{}\PYGZgt{} }\PYG{k}{for} \PYG{n}{i} \PYG{o+ow}{in} \PYG{n+nb}{range}\PYG{p}{(}\PYG{l+m+mi}{1000}\PYG{p}{)}\PYG{p}{:}
\PYG{g+gp}{... }   \PYG{n}{a} \PYG{o}{=} \PYG{l+m+mf}{10.} \PYG{o}{+} \PYG{n}{np}\PYG{o}{.}\PYG{n}{random}\PYG{o}{.}\PYG{n}{random}\PYG{p}{(}\PYG{l+m+mi}{100}\PYG{p}{)}
\PYG{g+gp}{... }   \PYG{n}{b}\PYG{o}{.}\PYG{n}{append}\PYG{p}{(}\PYG{n}{np}\PYG{o}{.}\PYG{n}{product}\PYG{p}{(}\PYG{n}{a}\PYG{p}{)}\PYG{p}{)}
\end{Verbatim}

\begin{Verbatim}[commandchars=\\\{\}]
\PYG{g+gp}{\PYGZgt{}\PYGZgt{}\PYGZgt{} }\PYG{n}{b} \PYG{o}{=} \PYG{n}{np}\PYG{o}{.}\PYG{n}{array}\PYG{p}{(}\PYG{n}{b}\PYG{p}{)} \PYG{o}{/} \PYG{n}{np}\PYG{o}{.}\PYG{n}{min}\PYG{p}{(}\PYG{n}{b}\PYG{p}{)} \PYG{c}{\PYGZsh{} scale values to be positive}
\PYG{g+gp}{\PYGZgt{}\PYGZgt{}\PYGZgt{} }\PYG{n}{count}\PYG{p}{,} \PYG{n}{bins}\PYG{p}{,} \PYG{n}{ignored} \PYG{o}{=} \PYG{n}{plt}\PYG{o}{.}\PYG{n}{hist}\PYG{p}{(}\PYG{n}{b}\PYG{p}{,} \PYG{l+m+mi}{100}\PYG{p}{,} \PYG{n}{normed}\PYG{o}{=}\PYG{n+nb+bp}{True}\PYG{p}{,} \PYG{n}{align}\PYG{o}{=}\PYG{l+s}{\PYGZsq{}}\PYG{l+s}{center}\PYG{l+s}{\PYGZsq{}}\PYG{p}{)}
\PYG{g+gp}{\PYGZgt{}\PYGZgt{}\PYGZgt{} }\PYG{n}{sigma} \PYG{o}{=} \PYG{n}{np}\PYG{o}{.}\PYG{n}{std}\PYG{p}{(}\PYG{n}{np}\PYG{o}{.}\PYG{n}{log}\PYG{p}{(}\PYG{n}{b}\PYG{p}{)}\PYG{p}{)}
\PYG{g+gp}{\PYGZgt{}\PYGZgt{}\PYGZgt{} }\PYG{n}{mu} \PYG{o}{=} \PYG{n}{np}\PYG{o}{.}\PYG{n}{mean}\PYG{p}{(}\PYG{n}{np}\PYG{o}{.}\PYG{n}{log}\PYG{p}{(}\PYG{n}{b}\PYG{p}{)}\PYG{p}{)}
\end{Verbatim}

\begin{Verbatim}[commandchars=\\\{\}]
\PYG{g+gp}{\PYGZgt{}\PYGZgt{}\PYGZgt{} }\PYG{n}{x} \PYG{o}{=} \PYG{n}{np}\PYG{o}{.}\PYG{n}{linspace}\PYG{p}{(}\PYG{n+nb}{min}\PYG{p}{(}\PYG{n}{bins}\PYG{p}{)}\PYG{p}{,} \PYG{n+nb}{max}\PYG{p}{(}\PYG{n}{bins}\PYG{p}{)}\PYG{p}{,} \PYG{l+m+mi}{10000}\PYG{p}{)}
\PYG{g+gp}{\PYGZgt{}\PYGZgt{}\PYGZgt{} }\PYG{n}{pdf} \PYG{o}{=} \PYG{p}{(}\PYG{n}{np}\PYG{o}{.}\PYG{n}{exp}\PYG{p}{(}\PYG{o}{\PYGZhy{}}\PYG{p}{(}\PYG{n}{np}\PYG{o}{.}\PYG{n}{log}\PYG{p}{(}\PYG{n}{x}\PYG{p}{)} \PYG{o}{\PYGZhy{}} \PYG{n}{mu}\PYG{p}{)}\PYG{o}{*}\PYG{o}{*}\PYG{l+m+mi}{2} \PYG{o}{/} \PYG{p}{(}\PYG{l+m+mi}{2} \PYG{o}{*} \PYG{n}{sigma}\PYG{o}{*}\PYG{o}{*}\PYG{l+m+mi}{2}\PYG{p}{)}\PYG{p}{)}
\PYG{g+gp}{... }       \PYG{o}{/} \PYG{p}{(}\PYG{n}{x} \PYG{o}{*} \PYG{n}{sigma} \PYG{o}{*} \PYG{n}{np}\PYG{o}{.}\PYG{n}{sqrt}\PYG{p}{(}\PYG{l+m+mi}{2} \PYG{o}{*} \PYG{n}{np}\PYG{o}{.}\PYG{n}{pi}\PYG{p}{)}\PYG{p}{)}\PYG{p}{)}
\end{Verbatim}

\begin{Verbatim}[commandchars=\\\{\}]
\PYG{g+gp}{\PYGZgt{}\PYGZgt{}\PYGZgt{} }\PYG{n}{plt}\PYG{o}{.}\PYG{n}{plot}\PYG{p}{(}\PYG{n}{x}\PYG{p}{,} \PYG{n}{pdf}\PYG{p}{,} \PYG{n}{color}\PYG{o}{=}\PYG{l+s}{\PYGZsq{}}\PYG{l+s}{r}\PYG{l+s}{\PYGZsq{}}\PYG{p}{,} \PYG{n}{linewidth}\PYG{o}{=}\PYG{l+m+mi}{2}\PYG{p}{)}
\PYG{g+gp}{\PYGZgt{}\PYGZgt{}\PYGZgt{} }\PYG{n}{plt}\PYG{o}{.}\PYG{n}{show}\PYG{p}{(}\PYG{p}{)}
\end{Verbatim}

\end{fulllineitems}

\index{logseries() (in module acsSpeciesActivities)}

\begin{fulllineitems}
\phantomsection\label{acsSpeciesActivities:acsSpeciesActivities.logseries}\pysiglinewithargsret{\code{acsSpeciesActivities.}\bfcode{logseries}}{\emph{p}, \emph{size=None}}{}
Draw samples from a Logarithmic Series distribution.

Samples are drawn from a Log Series distribution with specified
parameter, p (probability, 0 \textless{} p \textless{} 1).

loc : float

scale : float \textgreater{} 0.
\begin{description}
\item[{size}] \leavevmode{[}\{tuple, int\}{]}
Output shape.  If the given shape is, e.g., \code{(m, n, k)}, then
\code{m * n * k} samples are drawn.

\end{description}
\begin{description}
\item[{samples}] \leavevmode{[}\{ndarray, scalar\}{]}
where the values are all integers in  {[}0, n{]}.

\end{description}
\begin{description}
\item[{scipy.stats.distributions.logser}] \leavevmode{[}probability density function,{]}
distribution or cumulative density function, etc.

\end{description}

The probability density for the Log Series distribution is
\begin{gather}
\begin{split}P(k) = \frac{-p^k}{k \ln(1-p)},\end{split}\notag
\end{gather}
where p = probability.

The Log Series distribution is frequently used to represent species
richness and occurrence, first proposed by Fisher, Corbet, and
Williams in 1943 {[}2{]}.  It may also be used to model the numbers of
occupants seen in cars {[}3{]}.

Draw samples from the distribution:

\begin{Verbatim}[commandchars=\\\{\}]
\PYG{g+gp}{\PYGZgt{}\PYGZgt{}\PYGZgt{} }\PYG{n}{a} \PYG{o}{=} \PYG{o}{.}\PYG{l+m+mi}{6}
\PYG{g+gp}{\PYGZgt{}\PYGZgt{}\PYGZgt{} }\PYG{n}{s} \PYG{o}{=} \PYG{n}{np}\PYG{o}{.}\PYG{n}{random}\PYG{o}{.}\PYG{n}{logseries}\PYG{p}{(}\PYG{n}{a}\PYG{p}{,} \PYG{l+m+mi}{10000}\PYG{p}{)}
\PYG{g+gp}{\PYGZgt{}\PYGZgt{}\PYGZgt{} }\PYG{n}{count}\PYG{p}{,} \PYG{n}{bins}\PYG{p}{,} \PYG{n}{ignored} \PYG{o}{=} \PYG{n}{plt}\PYG{o}{.}\PYG{n}{hist}\PYG{p}{(}\PYG{n}{s}\PYG{p}{)}
\end{Verbatim}

\#   plot against distribution

\begin{Verbatim}[commandchars=\\\{\}]
\PYG{g+gp}{\PYGZgt{}\PYGZgt{}\PYGZgt{} }\PYG{k}{def} \PYG{n+nf}{logseries}\PYG{p}{(}\PYG{n}{k}\PYG{p}{,} \PYG{n}{p}\PYG{p}{)}\PYG{p}{:}
\PYG{g+gp}{... }    \PYG{k}{return} \PYG{o}{\PYGZhy{}}\PYG{n}{p}\PYG{o}{*}\PYG{o}{*}\PYG{n}{k}\PYG{o}{/}\PYG{p}{(}\PYG{n}{k}\PYG{o}{*}\PYG{n}{log}\PYG{p}{(}\PYG{l+m+mi}{1}\PYG{o}{\PYGZhy{}}\PYG{n}{p}\PYG{p}{)}\PYG{p}{)}
\PYG{g+gp}{\PYGZgt{}\PYGZgt{}\PYGZgt{} }\PYG{n}{plt}\PYG{o}{.}\PYG{n}{plot}\PYG{p}{(}\PYG{n}{bins}\PYG{p}{,} \PYG{n}{logseries}\PYG{p}{(}\PYG{n}{bins}\PYG{p}{,} \PYG{n}{a}\PYG{p}{)}\PYG{o}{*}\PYG{n}{count}\PYG{o}{.}\PYG{n}{max}\PYG{p}{(}\PYG{p}{)}\PYG{o}{/}
\PYG{g+go}{             logseries(bins, a).max(), \PYGZsq{}r\PYGZsq{})}
\PYG{g+gp}{\PYGZgt{}\PYGZgt{}\PYGZgt{} }\PYG{n}{plt}\PYG{o}{.}\PYG{n}{show}\PYG{p}{(}\PYG{p}{)}
\end{Verbatim}

\end{fulllineitems}

\index{multinomial() (in module acsSpeciesActivities)}

\begin{fulllineitems}
\phantomsection\label{acsSpeciesActivities:acsSpeciesActivities.multinomial}\pysiglinewithargsret{\code{acsSpeciesActivities.}\bfcode{multinomial}}{\emph{n}, \emph{pvals}, \emph{size=None}}{}
Draw samples from a multinomial distribution.

The multinomial distribution is a multivariate generalisation of the
binomial distribution.  Take an experiment with one of \code{p}
possible outcomes.  An example of such an experiment is throwing a dice,
where the outcome can be 1 through 6.  Each sample drawn from the
distribution represents \emph{n} such experiments.  Its values,
\code{X\_i = {[}X\_0, X\_1, ..., X\_p{]}}, represent the number of times the outcome
was \code{i}.
\begin{description}
\item[{n}] \leavevmode{[}int{]}
Number of experiments.

\item[{pvals}] \leavevmode{[}sequence of floats, length p{]}
Probabilities of each of the \code{p} different outcomes.  These
should sum to 1 (however, the last element is always assumed to
account for the remaining probability, as long as
\code{sum(pvals{[}:-1{]}) \textless{}= 1)}.

\item[{size}] \leavevmode{[}tuple of ints{]}
Given a \emph{size} of \code{(M, N, K)}, then \code{M*N*K} samples are drawn,
and the output shape becomes \code{(M, N, K, p)}, since each sample
has shape \code{(p,)}.

\end{description}

Throw a dice 20 times:

\begin{Verbatim}[commandchars=\\\{\}]
\PYG{g+gp}{\PYGZgt{}\PYGZgt{}\PYGZgt{} }\PYG{n}{np}\PYG{o}{.}\PYG{n}{random}\PYG{o}{.}\PYG{n}{multinomial}\PYG{p}{(}\PYG{l+m+mi}{20}\PYG{p}{,} \PYG{p}{[}\PYG{l+m+mi}{1}\PYG{o}{/}\PYG{l+m+mf}{6.}\PYG{p}{]}\PYG{o}{*}\PYG{l+m+mi}{6}\PYG{p}{,} \PYG{n}{size}\PYG{o}{=}\PYG{l+m+mi}{1}\PYG{p}{)}
\PYG{g+go}{array([[4, 1, 7, 5, 2, 1]])}
\end{Verbatim}

It landed 4 times on 1, once on 2, etc.

Now, throw the dice 20 times, and 20 times again:

\begin{Verbatim}[commandchars=\\\{\}]
\PYG{g+gp}{\PYGZgt{}\PYGZgt{}\PYGZgt{} }\PYG{n}{np}\PYG{o}{.}\PYG{n}{random}\PYG{o}{.}\PYG{n}{multinomial}\PYG{p}{(}\PYG{l+m+mi}{20}\PYG{p}{,} \PYG{p}{[}\PYG{l+m+mi}{1}\PYG{o}{/}\PYG{l+m+mf}{6.}\PYG{p}{]}\PYG{o}{*}\PYG{l+m+mi}{6}\PYG{p}{,} \PYG{n}{size}\PYG{o}{=}\PYG{l+m+mi}{2}\PYG{p}{)}
\PYG{g+go}{array([[3, 4, 3, 3, 4, 3],}
\PYG{g+go}{       [2, 4, 3, 4, 0, 7]])}
\end{Verbatim}

For the first run, we threw 3 times 1, 4 times 2, etc.  For the second,
we threw 2 times 1, 4 times 2, etc.

A loaded dice is more likely to land on number 6:

\begin{Verbatim}[commandchars=\\\{\}]
\PYG{g+gp}{\PYGZgt{}\PYGZgt{}\PYGZgt{} }\PYG{n}{np}\PYG{o}{.}\PYG{n}{random}\PYG{o}{.}\PYG{n}{multinomial}\PYG{p}{(}\PYG{l+m+mi}{100}\PYG{p}{,} \PYG{p}{[}\PYG{l+m+mi}{1}\PYG{o}{/}\PYG{l+m+mf}{7.}\PYG{p}{]}\PYG{o}{*}\PYG{l+m+mi}{5}\PYG{p}{)}
\PYG{g+go}{array([13, 16, 13, 16, 42])}
\end{Verbatim}

\end{fulllineitems}

\index{multivariate\_normal() (in module acsSpeciesActivities)}

\begin{fulllineitems}
\phantomsection\label{acsSpeciesActivities:acsSpeciesActivities.multivariate_normal}\pysiglinewithargsret{\code{acsSpeciesActivities.}\bfcode{multivariate\_normal}}{\emph{mean}, \emph{cov}\optional{, \emph{size}}}{}
Draw random samples from a multivariate normal distribution.

The multivariate normal, multinormal or Gaussian distribution is a
generalization of the one-dimensional normal distribution to higher
dimensions.  Such a distribution is specified by its mean and
covariance matrix.  These parameters are analogous to the mean
(average or ``center'') and variance (standard deviation, or ``width,''
squared) of the one-dimensional normal distribution.
\begin{description}
\item[{mean}] \leavevmode{[}1-D array\_like, of length N{]}
Mean of the N-dimensional distribution.

\item[{cov}] \leavevmode{[}2-D array\_like, of shape (N, N){]}
Covariance matrix of the distribution.  Must be symmetric and
positive semi-definite for ``physically meaningful'' results.

\item[{size}] \leavevmode{[}int or tuple of ints, optional{]}
Given a shape of, for example, \code{(m,n,k)}, \code{m*n*k} samples are
generated, and packed in an \emph{m}-by-\emph{n}-by-\emph{k} arrangement.  Because
each sample is \emph{N}-dimensional, the output shape is \code{(m,n,k,N)}.
If no shape is specified, a single (\emph{N}-D) sample is returned.

\end{description}
\begin{description}
\item[{out}] \leavevmode{[}ndarray{]}
The drawn samples, of shape \emph{size}, if that was provided.  If not,
the shape is \code{(N,)}.

In other words, each entry \code{out{[}i,j,...,:{]}} is an N-dimensional
value drawn from the distribution.

\end{description}

The mean is a coordinate in N-dimensional space, which represents the
location where samples are most likely to be generated.  This is
analogous to the peak of the bell curve for the one-dimensional or
univariate normal distribution.

Covariance indicates the level to which two variables vary together.
From the multivariate normal distribution, we draw N-dimensional
samples, \(X = [x_1, x_2, ... x_N]\).  The covariance matrix
element \(C_{ij}\) is the covariance of \(x_i\) and \(x_j\).
The element \(C_{ii}\) is the variance of \(x_i\) (i.e. its
``spread'').

Instead of specifying the full covariance matrix, popular
approximations include:
\begin{itemize}
\item {} 
Spherical covariance (\emph{cov} is a multiple of the identity matrix)

\item {} 
Diagonal covariance (\emph{cov} has non-negative elements, and only on
the diagonal)

\end{itemize}

This geometrical property can be seen in two dimensions by plotting
generated data-points:

\begin{Verbatim}[commandchars=\\\{\}]
\PYG{g+gp}{\PYGZgt{}\PYGZgt{}\PYGZgt{} }\PYG{n}{mean} \PYG{o}{=} \PYG{p}{[}\PYG{l+m+mi}{0}\PYG{p}{,}\PYG{l+m+mi}{0}\PYG{p}{]}
\PYG{g+gp}{\PYGZgt{}\PYGZgt{}\PYGZgt{} }\PYG{n}{cov} \PYG{o}{=} \PYG{p}{[}\PYG{p}{[}\PYG{l+m+mi}{1}\PYG{p}{,}\PYG{l+m+mi}{0}\PYG{p}{]}\PYG{p}{,}\PYG{p}{[}\PYG{l+m+mi}{0}\PYG{p}{,}\PYG{l+m+mi}{100}\PYG{p}{]}\PYG{p}{]} \PYG{c}{\PYGZsh{} diagonal covariance, points lie on x or y\PYGZhy{}axis}
\end{Verbatim}

\begin{Verbatim}[commandchars=\\\{\}]
\PYG{g+gp}{\PYGZgt{}\PYGZgt{}\PYGZgt{} }\PYG{k+kn}{import} \PYG{n+nn}{matplotlib.pyplot} \PYG{k+kn}{as} \PYG{n+nn}{plt}
\PYG{g+gp}{\PYGZgt{}\PYGZgt{}\PYGZgt{} }\PYG{n}{x}\PYG{p}{,}\PYG{n}{y} \PYG{o}{=} \PYG{n}{np}\PYG{o}{.}\PYG{n}{random}\PYG{o}{.}\PYG{n}{multivariate\PYGZus{}normal}\PYG{p}{(}\PYG{n}{mean}\PYG{p}{,}\PYG{n}{cov}\PYG{p}{,}\PYG{l+m+mi}{5000}\PYG{p}{)}\PYG{o}{.}\PYG{n}{T}
\PYG{g+gp}{\PYGZgt{}\PYGZgt{}\PYGZgt{} }\PYG{n}{plt}\PYG{o}{.}\PYG{n}{plot}\PYG{p}{(}\PYG{n}{x}\PYG{p}{,}\PYG{n}{y}\PYG{p}{,}\PYG{l+s}{\PYGZsq{}}\PYG{l+s}{x}\PYG{l+s}{\PYGZsq{}}\PYG{p}{)}\PYG{p}{;} \PYG{n}{plt}\PYG{o}{.}\PYG{n}{axis}\PYG{p}{(}\PYG{l+s}{\PYGZsq{}}\PYG{l+s}{equal}\PYG{l+s}{\PYGZsq{}}\PYG{p}{)}\PYG{p}{;} \PYG{n}{plt}\PYG{o}{.}\PYG{n}{show}\PYG{p}{(}\PYG{p}{)}
\end{Verbatim}

Note that the covariance matrix must be non-negative definite.

Papoulis, A., \emph{Probability, Random Variables, and Stochastic Processes},
3rd ed., New York: McGraw-Hill, 1991.

Duda, R. O., Hart, P. E., and Stork, D. G., \emph{Pattern Classification},
2nd ed., New York: Wiley, 2001.

\begin{Verbatim}[commandchars=\\\{\}]
\PYG{g+gp}{\PYGZgt{}\PYGZgt{}\PYGZgt{} }\PYG{n}{mean} \PYG{o}{=} \PYG{p}{(}\PYG{l+m+mi}{1}\PYG{p}{,}\PYG{l+m+mi}{2}\PYG{p}{)}
\PYG{g+gp}{\PYGZgt{}\PYGZgt{}\PYGZgt{} }\PYG{n}{cov} \PYG{o}{=} \PYG{p}{[}\PYG{p}{[}\PYG{l+m+mi}{1}\PYG{p}{,}\PYG{l+m+mi}{0}\PYG{p}{]}\PYG{p}{,}\PYG{p}{[}\PYG{l+m+mi}{1}\PYG{p}{,}\PYG{l+m+mi}{0}\PYG{p}{]}\PYG{p}{]}
\PYG{g+gp}{\PYGZgt{}\PYGZgt{}\PYGZgt{} }\PYG{n}{x} \PYG{o}{=} \PYG{n}{np}\PYG{o}{.}\PYG{n}{random}\PYG{o}{.}\PYG{n}{multivariate\PYGZus{}normal}\PYG{p}{(}\PYG{n}{mean}\PYG{p}{,}\PYG{n}{cov}\PYG{p}{,}\PYG{p}{(}\PYG{l+m+mi}{3}\PYG{p}{,}\PYG{l+m+mi}{3}\PYG{p}{)}\PYG{p}{)}
\PYG{g+gp}{\PYGZgt{}\PYGZgt{}\PYGZgt{} }\PYG{n}{x}\PYG{o}{.}\PYG{n}{shape}
\PYG{g+go}{(3, 3, 2)}
\end{Verbatim}

The following is probably true, given that 0.6 is roughly twice the
standard deviation:

\begin{Verbatim}[commandchars=\\\{\}]
\PYG{g+gp}{\PYGZgt{}\PYGZgt{}\PYGZgt{} }\PYG{k}{print} \PYG{n+nb}{list}\PYG{p}{(} \PYG{p}{(}\PYG{n}{x}\PYG{p}{[}\PYG{l+m+mi}{0}\PYG{p}{,}\PYG{l+m+mi}{0}\PYG{p}{,}\PYG{p}{:}\PYG{p}{]} \PYG{o}{\PYGZhy{}} \PYG{n}{mean}\PYG{p}{)} \PYG{o}{\PYGZlt{}} \PYG{l+m+mf}{0.6} \PYG{p}{)}
\PYG{g+go}{[True, True]}
\end{Verbatim}

\end{fulllineitems}

\index{negative\_binomial() (in module acsSpeciesActivities)}

\begin{fulllineitems}
\phantomsection\label{acsSpeciesActivities:acsSpeciesActivities.negative_binomial}\pysiglinewithargsret{\code{acsSpeciesActivities.}\bfcode{negative\_binomial}}{\emph{n}, \emph{p}, \emph{size=None}}{}
Draw samples from a negative\_binomial distribution.

Samples are drawn from a negative\_Binomial distribution with specified
parameters, \emph{n} trials and \emph{p} probability of success where \emph{n} is an
integer \textgreater{} 0 and \emph{p} is in the interval {[}0, 1{]}.
\begin{description}
\item[{n}] \leavevmode{[}int{]}
Parameter, \textgreater{} 0.

\item[{p}] \leavevmode{[}float{]}
Parameter, \textgreater{}= 0 and \textless{}=1.

\item[{size}] \leavevmode{[}int or tuple of ints{]}
Output shape. If the given shape is, e.g., \code{(m, n, k)}, then
\code{m * n * k} samples are drawn.

\end{description}
\begin{description}
\item[{samples}] \leavevmode{[}int or ndarray of ints{]}
Drawn samples.

\end{description}

The probability density for the Negative Binomial distribution is
\begin{gather}
\begin{split}P(N;n,p) = \binom{N+n-1}{n-1}p^{n}(1-p)^{N},\end{split}\notag
\end{gather}
where \(n-1\) is the number of successes, \(p\) is the probability
of success, and \(N+n-1\) is the number of trials.

The negative binomial distribution gives the probability of n-1 successes
and N failures in N+n-1 trials, and success on the (N+n)th trial.

If one throws a die repeatedly until the third time a ``1'' appears, then the
probability distribution of the number of non-``1''s that appear before the
third ``1'' is a negative binomial distribution.

Draw samples from the distribution:

A real world example. A company drills wild-cat oil exploration wells, each
with an estimated probability of success of 0.1.  What is the probability
of having one success for each successive well, that is what is the
probability of a single success after drilling 5 wells, after 6 wells,
etc.?

\begin{Verbatim}[commandchars=\\\{\}]
\PYG{g+gp}{\PYGZgt{}\PYGZgt{}\PYGZgt{} }\PYG{n}{s} \PYG{o}{=} \PYG{n}{np}\PYG{o}{.}\PYG{n}{random}\PYG{o}{.}\PYG{n}{negative\PYGZus{}binomial}\PYG{p}{(}\PYG{l+m+mi}{1}\PYG{p}{,} \PYG{l+m+mf}{0.1}\PYG{p}{,} \PYG{l+m+mi}{100000}\PYG{p}{)}
\PYG{g+gp}{\PYGZgt{}\PYGZgt{}\PYGZgt{} }\PYG{k}{for} \PYG{n}{i} \PYG{o+ow}{in} \PYG{n+nb}{range}\PYG{p}{(}\PYG{l+m+mi}{1}\PYG{p}{,} \PYG{l+m+mi}{11}\PYG{p}{)}\PYG{p}{:}
\PYG{g+gp}{... }   \PYG{n}{probability} \PYG{o}{=} \PYG{n+nb}{sum}\PYG{p}{(}\PYG{n}{s}\PYG{o}{\PYGZlt{}}\PYG{n}{i}\PYG{p}{)} \PYG{o}{/} \PYG{l+m+mf}{100000.}
\PYG{g+gp}{... }   \PYG{k}{print} \PYG{n}{i}\PYG{p}{,} \PYG{l+s}{\PYGZdq{}}\PYG{l+s}{wells drilled, probability of one success =}\PYG{l+s}{\PYGZdq{}}\PYG{p}{,} \PYG{n}{probability}
\end{Verbatim}

\end{fulllineitems}

\index{noncentral\_chisquare() (in module acsSpeciesActivities)}

\begin{fulllineitems}
\phantomsection\label{acsSpeciesActivities:acsSpeciesActivities.noncentral_chisquare}\pysiglinewithargsret{\code{acsSpeciesActivities.}\bfcode{noncentral\_chisquare}}{\emph{df}, \emph{nonc}, \emph{size=None}}{}
Draw samples from a noncentral chi-square distribution.

The noncentral \(\chi^2\) distribution is a generalisation of
the \(\chi^2\) distribution.
\begin{description}
\item[{df}] \leavevmode{[}int{]}
Degrees of freedom, should be \textgreater{}= 1.

\item[{nonc}] \leavevmode{[}float{]}
Non-centrality, should be \textgreater{} 0.

\item[{size}] \leavevmode{[}int or tuple of ints{]}
Shape of the output.

\end{description}

The probability density function for the noncentral Chi-square distribution
is
\begin{gather}
\begin{split}P(x;df,nonc) = \sum^{\infty}_{i=0}
\frac{e^{-nonc/2}(nonc/2)^{i}}{i!}P_{Y_{df+2i}}(x),\end{split}\notag
\end{gather}
where \(Y_{q}\) is the Chi-square with q degrees of freedom.

In Delhi (2007), it is noted that the noncentral chi-square is useful in
bombing and coverage problems, the probability of killing the point target
given by the noncentral chi-squared distribution.

Draw values from the distribution and plot the histogram

\begin{Verbatim}[commandchars=\\\{\}]
\PYG{g+gp}{\PYGZgt{}\PYGZgt{}\PYGZgt{} }\PYG{k+kn}{import} \PYG{n+nn}{matplotlib.pyplot} \PYG{k+kn}{as} \PYG{n+nn}{plt}
\PYG{g+gp}{\PYGZgt{}\PYGZgt{}\PYGZgt{} }\PYG{n}{values} \PYG{o}{=} \PYG{n}{plt}\PYG{o}{.}\PYG{n}{hist}\PYG{p}{(}\PYG{n}{np}\PYG{o}{.}\PYG{n}{random}\PYG{o}{.}\PYG{n}{noncentral\PYGZus{}chisquare}\PYG{p}{(}\PYG{l+m+mi}{3}\PYG{p}{,} \PYG{l+m+mi}{20}\PYG{p}{,} \PYG{l+m+mi}{100000}\PYG{p}{)}\PYG{p}{,}
\PYG{g+gp}{... }                  \PYG{n}{bins}\PYG{o}{=}\PYG{l+m+mi}{200}\PYG{p}{,} \PYG{n}{normed}\PYG{o}{=}\PYG{n+nb+bp}{True}\PYG{p}{)}
\PYG{g+gp}{\PYGZgt{}\PYGZgt{}\PYGZgt{} }\PYG{n}{plt}\PYG{o}{.}\PYG{n}{show}\PYG{p}{(}\PYG{p}{)}
\end{Verbatim}

Draw values from a noncentral chisquare with very small noncentrality,
and compare to a chisquare.

\begin{Verbatim}[commandchars=\\\{\}]
\PYG{g+gp}{\PYGZgt{}\PYGZgt{}\PYGZgt{} }\PYG{n}{plt}\PYG{o}{.}\PYG{n}{figure}\PYG{p}{(}\PYG{p}{)}
\PYG{g+gp}{\PYGZgt{}\PYGZgt{}\PYGZgt{} }\PYG{n}{values} \PYG{o}{=} \PYG{n}{plt}\PYG{o}{.}\PYG{n}{hist}\PYG{p}{(}\PYG{n}{np}\PYG{o}{.}\PYG{n}{random}\PYG{o}{.}\PYG{n}{noncentral\PYGZus{}chisquare}\PYG{p}{(}\PYG{l+m+mi}{3}\PYG{p}{,} \PYG{o}{.}\PYG{l+m+mo}{0000001}\PYG{p}{,} \PYG{l+m+mi}{100000}\PYG{p}{)}\PYG{p}{,}
\PYG{g+gp}{... }                  \PYG{n}{bins}\PYG{o}{=}\PYG{n}{np}\PYG{o}{.}\PYG{n}{arange}\PYG{p}{(}\PYG{l+m+mf}{0.}\PYG{p}{,} \PYG{l+m+mi}{25}\PYG{p}{,} \PYG{o}{.}\PYG{l+m+mi}{1}\PYG{p}{)}\PYG{p}{,} \PYG{n}{normed}\PYG{o}{=}\PYG{n+nb+bp}{True}\PYG{p}{)}
\PYG{g+gp}{\PYGZgt{}\PYGZgt{}\PYGZgt{} }\PYG{n}{values2} \PYG{o}{=} \PYG{n}{plt}\PYG{o}{.}\PYG{n}{hist}\PYG{p}{(}\PYG{n}{np}\PYG{o}{.}\PYG{n}{random}\PYG{o}{.}\PYG{n}{chisquare}\PYG{p}{(}\PYG{l+m+mi}{3}\PYG{p}{,} \PYG{l+m+mi}{100000}\PYG{p}{)}\PYG{p}{,}
\PYG{g+gp}{... }                   \PYG{n}{bins}\PYG{o}{=}\PYG{n}{np}\PYG{o}{.}\PYG{n}{arange}\PYG{p}{(}\PYG{l+m+mf}{0.}\PYG{p}{,} \PYG{l+m+mi}{25}\PYG{p}{,} \PYG{o}{.}\PYG{l+m+mi}{1}\PYG{p}{)}\PYG{p}{,} \PYG{n}{normed}\PYG{o}{=}\PYG{n+nb+bp}{True}\PYG{p}{)}
\PYG{g+gp}{\PYGZgt{}\PYGZgt{}\PYGZgt{} }\PYG{n}{plt}\PYG{o}{.}\PYG{n}{plot}\PYG{p}{(}\PYG{n}{values}\PYG{p}{[}\PYG{l+m+mi}{1}\PYG{p}{]}\PYG{p}{[}\PYG{l+m+mi}{0}\PYG{p}{:}\PYG{o}{\PYGZhy{}}\PYG{l+m+mi}{1}\PYG{p}{]}\PYG{p}{,} \PYG{n}{values}\PYG{p}{[}\PYG{l+m+mi}{0}\PYG{p}{]}\PYG{o}{\PYGZhy{}}\PYG{n}{values2}\PYG{p}{[}\PYG{l+m+mi}{0}\PYG{p}{]}\PYG{p}{,} \PYG{l+s}{\PYGZsq{}}\PYG{l+s}{ob}\PYG{l+s}{\PYGZsq{}}\PYG{p}{)}
\PYG{g+gp}{\PYGZgt{}\PYGZgt{}\PYGZgt{} }\PYG{n}{plt}\PYG{o}{.}\PYG{n}{show}\PYG{p}{(}\PYG{p}{)}
\end{Verbatim}

Demonstrate how large values of non-centrality lead to a more symmetric
distribution.

\begin{Verbatim}[commandchars=\\\{\}]
\PYG{g+gp}{\PYGZgt{}\PYGZgt{}\PYGZgt{} }\PYG{n}{plt}\PYG{o}{.}\PYG{n}{figure}\PYG{p}{(}\PYG{p}{)}
\PYG{g+gp}{\PYGZgt{}\PYGZgt{}\PYGZgt{} }\PYG{n}{values} \PYG{o}{=} \PYG{n}{plt}\PYG{o}{.}\PYG{n}{hist}\PYG{p}{(}\PYG{n}{np}\PYG{o}{.}\PYG{n}{random}\PYG{o}{.}\PYG{n}{noncentral\PYGZus{}chisquare}\PYG{p}{(}\PYG{l+m+mi}{3}\PYG{p}{,} \PYG{l+m+mi}{20}\PYG{p}{,} \PYG{l+m+mi}{100000}\PYG{p}{)}\PYG{p}{,}
\PYG{g+gp}{... }                  \PYG{n}{bins}\PYG{o}{=}\PYG{l+m+mi}{200}\PYG{p}{,} \PYG{n}{normed}\PYG{o}{=}\PYG{n+nb+bp}{True}\PYG{p}{)}
\PYG{g+gp}{\PYGZgt{}\PYGZgt{}\PYGZgt{} }\PYG{n}{plt}\PYG{o}{.}\PYG{n}{show}\PYG{p}{(}\PYG{p}{)}
\end{Verbatim}

\end{fulllineitems}

\index{noncentral\_f() (in module acsSpeciesActivities)}

\begin{fulllineitems}
\phantomsection\label{acsSpeciesActivities:acsSpeciesActivities.noncentral_f}\pysiglinewithargsret{\code{acsSpeciesActivities.}\bfcode{noncentral\_f}}{\emph{dfnum}, \emph{dfden}, \emph{nonc}, \emph{size=None}}{}
Draw samples from the noncentral F distribution.

Samples are drawn from an F distribution with specified parameters,
\emph{dfnum} (degrees of freedom in numerator) and \emph{dfden} (degrees of
freedom in denominator), where both parameters \textgreater{} 1.
\emph{nonc} is the non-centrality parameter.
\begin{description}
\item[{dfnum}] \leavevmode{[}int{]}
Parameter, should be \textgreater{} 1.

\item[{dfden}] \leavevmode{[}int{]}
Parameter, should be \textgreater{} 1.

\item[{nonc}] \leavevmode{[}float{]}
Parameter, should be \textgreater{}= 0.

\item[{size}] \leavevmode{[}int or tuple of ints{]}
Output shape. If the given shape is, e.g., \code{(m, n, k)}, then
\code{m * n * k} samples are drawn.

\end{description}
\begin{description}
\item[{samples}] \leavevmode{[}scalar or ndarray{]}
Drawn samples.

\end{description}

When calculating the power of an experiment (power = probability of
rejecting the null hypothesis when a specific alternative is true) the
non-central F statistic becomes important.  When the null hypothesis is
true, the F statistic follows a central F distribution. When the null
hypothesis is not true, then it follows a non-central F statistic.

Weisstein, Eric W. ``Noncentral F-Distribution.'' From MathWorld--A Wolfram
Web Resource.  \href{http://mathworld.wolfram.com/NoncentralF-Distribution.html}{http://mathworld.wolfram.com/NoncentralF-Distribution.html}

Wikipedia, ``Noncentral F distribution'',
\href{http://en.wikipedia.org/wiki/Noncentral\_F-distribution}{http://en.wikipedia.org/wiki/Noncentral\_F-distribution}

In a study, testing for a specific alternative to the null hypothesis
requires use of the Noncentral F distribution. We need to calculate the
area in the tail of the distribution that exceeds the value of the F
distribution for the null hypothesis.  We'll plot the two probability
distributions for comparison.

\begin{Verbatim}[commandchars=\\\{\}]
\PYG{g+gp}{\PYGZgt{}\PYGZgt{}\PYGZgt{} }\PYG{n}{dfnum} \PYG{o}{=} \PYG{l+m+mi}{3} \PYG{c}{\PYGZsh{} between group deg of freedom}
\PYG{g+gp}{\PYGZgt{}\PYGZgt{}\PYGZgt{} }\PYG{n}{dfden} \PYG{o}{=} \PYG{l+m+mi}{20} \PYG{c}{\PYGZsh{} within groups degrees of freedom}
\PYG{g+gp}{\PYGZgt{}\PYGZgt{}\PYGZgt{} }\PYG{n}{nonc} \PYG{o}{=} \PYG{l+m+mf}{3.0}
\PYG{g+gp}{\PYGZgt{}\PYGZgt{}\PYGZgt{} }\PYG{n}{nc\PYGZus{}vals} \PYG{o}{=} \PYG{n}{np}\PYG{o}{.}\PYG{n}{random}\PYG{o}{.}\PYG{n}{noncentral\PYGZus{}f}\PYG{p}{(}\PYG{n}{dfnum}\PYG{p}{,} \PYG{n}{dfden}\PYG{p}{,} \PYG{n}{nonc}\PYG{p}{,} \PYG{l+m+mi}{1000000}\PYG{p}{)}
\PYG{g+gp}{\PYGZgt{}\PYGZgt{}\PYGZgt{} }\PYG{n}{NF} \PYG{o}{=} \PYG{n}{np}\PYG{o}{.}\PYG{n}{histogram}\PYG{p}{(}\PYG{n}{nc\PYGZus{}vals}\PYG{p}{,} \PYG{n}{bins}\PYG{o}{=}\PYG{l+m+mi}{50}\PYG{p}{,} \PYG{n}{normed}\PYG{o}{=}\PYG{n+nb+bp}{True}\PYG{p}{)}
\PYG{g+gp}{\PYGZgt{}\PYGZgt{}\PYGZgt{} }\PYG{n}{c\PYGZus{}vals} \PYG{o}{=} \PYG{n}{np}\PYG{o}{.}\PYG{n}{random}\PYG{o}{.}\PYG{n}{f}\PYG{p}{(}\PYG{n}{dfnum}\PYG{p}{,} \PYG{n}{dfden}\PYG{p}{,} \PYG{l+m+mi}{1000000}\PYG{p}{)}
\PYG{g+gp}{\PYGZgt{}\PYGZgt{}\PYGZgt{} }\PYG{n}{F} \PYG{o}{=} \PYG{n}{np}\PYG{o}{.}\PYG{n}{histogram}\PYG{p}{(}\PYG{n}{c\PYGZus{}vals}\PYG{p}{,} \PYG{n}{bins}\PYG{o}{=}\PYG{l+m+mi}{50}\PYG{p}{,} \PYG{n}{normed}\PYG{o}{=}\PYG{n+nb+bp}{True}\PYG{p}{)}
\PYG{g+gp}{\PYGZgt{}\PYGZgt{}\PYGZgt{} }\PYG{n}{plt}\PYG{o}{.}\PYG{n}{plot}\PYG{p}{(}\PYG{n}{F}\PYG{p}{[}\PYG{l+m+mi}{1}\PYG{p}{]}\PYG{p}{[}\PYG{l+m+mi}{1}\PYG{p}{:}\PYG{p}{]}\PYG{p}{,} \PYG{n}{F}\PYG{p}{[}\PYG{l+m+mi}{0}\PYG{p}{]}\PYG{p}{)}
\PYG{g+gp}{\PYGZgt{}\PYGZgt{}\PYGZgt{} }\PYG{n}{plt}\PYG{o}{.}\PYG{n}{plot}\PYG{p}{(}\PYG{n}{NF}\PYG{p}{[}\PYG{l+m+mi}{1}\PYG{p}{]}\PYG{p}{[}\PYG{l+m+mi}{1}\PYG{p}{:}\PYG{p}{]}\PYG{p}{,} \PYG{n}{NF}\PYG{p}{[}\PYG{l+m+mi}{0}\PYG{p}{]}\PYG{p}{)}
\PYG{g+gp}{\PYGZgt{}\PYGZgt{}\PYGZgt{} }\PYG{n}{plt}\PYG{o}{.}\PYG{n}{show}\PYG{p}{(}\PYG{p}{)}
\end{Verbatim}

\end{fulllineitems}

\index{normal() (in module acsSpeciesActivities)}

\begin{fulllineitems}
\phantomsection\label{acsSpeciesActivities:acsSpeciesActivities.normal}\pysiglinewithargsret{\code{acsSpeciesActivities.}\bfcode{normal}}{\emph{loc=0.0}, \emph{scale=1.0}, \emph{size=None}}{}
Draw random samples from a normal (Gaussian) distribution.

The probability density function of the normal distribution, first
derived by De Moivre and 200 years later by both Gauss and Laplace
independently {\color{red}\bfseries{}{[}2{]}\_}, is often called the bell curve because of
its characteristic shape (see the example below).

The normal distributions occurs often in nature.  For example, it
describes the commonly occurring distribution of samples influenced
by a large number of tiny, random disturbances, each with its own
unique distribution {\color{red}\bfseries{}{[}2{]}\_}.
\begin{description}
\item[{loc}] \leavevmode{[}float{]}
Mean (``centre'') of the distribution.

\item[{scale}] \leavevmode{[}float{]}
Standard deviation (spread or ``width'') of the distribution.

\item[{size}] \leavevmode{[}tuple of ints{]}
Output shape.  If the given shape is, e.g., \code{(m, n, k)}, then
\code{m * n * k} samples are drawn.

\end{description}
\begin{description}
\item[{scipy.stats.distributions.norm}] \leavevmode{[}probability density function,{]}
distribution or cumulative density function, etc.

\end{description}

The probability density for the Gaussian distribution is
\begin{gather}
\begin{split}p(x) = \frac{1}{\sqrt{ 2 \pi \sigma^2 }}
e^{ - \frac{ (x - \mu)^2 } {2 \sigma^2} },\end{split}\notag
\end{gather}
where \(\mu\) is the mean and \(\sigma\) the standard deviation.
The square of the standard deviation, \(\sigma^2\), is called the
variance.

The function has its peak at the mean, and its ``spread'' increases with
the standard deviation (the function reaches 0.607 times its maximum at
\(x + \sigma\) and \(x - \sigma\) {\color{red}\bfseries{}{[}2{]}\_}).  This implies that
\emph{numpy.random.normal} is more likely to return samples lying close to the
mean, rather than those far away.

Draw samples from the distribution:

\begin{Verbatim}[commandchars=\\\{\}]
\PYG{g+gp}{\PYGZgt{}\PYGZgt{}\PYGZgt{} }\PYG{n}{mu}\PYG{p}{,} \PYG{n}{sigma} \PYG{o}{=} \PYG{l+m+mi}{0}\PYG{p}{,} \PYG{l+m+mf}{0.1} \PYG{c}{\PYGZsh{} mean and standard deviation}
\PYG{g+gp}{\PYGZgt{}\PYGZgt{}\PYGZgt{} }\PYG{n}{s} \PYG{o}{=} \PYG{n}{np}\PYG{o}{.}\PYG{n}{random}\PYG{o}{.}\PYG{n}{normal}\PYG{p}{(}\PYG{n}{mu}\PYG{p}{,} \PYG{n}{sigma}\PYG{p}{,} \PYG{l+m+mi}{1000}\PYG{p}{)}
\end{Verbatim}

Verify the mean and the variance:

\begin{Verbatim}[commandchars=\\\{\}]
\PYG{g+gp}{\PYGZgt{}\PYGZgt{}\PYGZgt{} }\PYG{n+nb}{abs}\PYG{p}{(}\PYG{n}{mu} \PYG{o}{\PYGZhy{}} \PYG{n}{np}\PYG{o}{.}\PYG{n}{mean}\PYG{p}{(}\PYG{n}{s}\PYG{p}{)}\PYG{p}{)} \PYG{o}{\PYGZlt{}} \PYG{l+m+mf}{0.01}
\PYG{g+go}{True}
\end{Verbatim}

\begin{Verbatim}[commandchars=\\\{\}]
\PYG{g+gp}{\PYGZgt{}\PYGZgt{}\PYGZgt{} }\PYG{n+nb}{abs}\PYG{p}{(}\PYG{n}{sigma} \PYG{o}{\PYGZhy{}} \PYG{n}{np}\PYG{o}{.}\PYG{n}{std}\PYG{p}{(}\PYG{n}{s}\PYG{p}{,} \PYG{n}{ddof}\PYG{o}{=}\PYG{l+m+mi}{1}\PYG{p}{)}\PYG{p}{)} \PYG{o}{\PYGZlt{}} \PYG{l+m+mf}{0.01}
\PYG{g+go}{True}
\end{Verbatim}

Display the histogram of the samples, along with
the probability density function:

\begin{Verbatim}[commandchars=\\\{\}]
\PYG{g+gp}{\PYGZgt{}\PYGZgt{}\PYGZgt{} }\PYG{k+kn}{import} \PYG{n+nn}{matplotlib.pyplot} \PYG{k+kn}{as} \PYG{n+nn}{plt}
\PYG{g+gp}{\PYGZgt{}\PYGZgt{}\PYGZgt{} }\PYG{n}{count}\PYG{p}{,} \PYG{n}{bins}\PYG{p}{,} \PYG{n}{ignored} \PYG{o}{=} \PYG{n}{plt}\PYG{o}{.}\PYG{n}{hist}\PYG{p}{(}\PYG{n}{s}\PYG{p}{,} \PYG{l+m+mi}{30}\PYG{p}{,} \PYG{n}{normed}\PYG{o}{=}\PYG{n+nb+bp}{True}\PYG{p}{)}
\PYG{g+gp}{\PYGZgt{}\PYGZgt{}\PYGZgt{} }\PYG{n}{plt}\PYG{o}{.}\PYG{n}{plot}\PYG{p}{(}\PYG{n}{bins}\PYG{p}{,} \PYG{l+m+mi}{1}\PYG{o}{/}\PYG{p}{(}\PYG{n}{sigma} \PYG{o}{*} \PYG{n}{np}\PYG{o}{.}\PYG{n}{sqrt}\PYG{p}{(}\PYG{l+m+mi}{2} \PYG{o}{*} \PYG{n}{np}\PYG{o}{.}\PYG{n}{pi}\PYG{p}{)}\PYG{p}{)} \PYG{o}{*}
\PYG{g+gp}{... }               \PYG{n}{np}\PYG{o}{.}\PYG{n}{exp}\PYG{p}{(} \PYG{o}{\PYGZhy{}} \PYG{p}{(}\PYG{n}{bins} \PYG{o}{\PYGZhy{}} \PYG{n}{mu}\PYG{p}{)}\PYG{o}{*}\PYG{o}{*}\PYG{l+m+mi}{2} \PYG{o}{/} \PYG{p}{(}\PYG{l+m+mi}{2} \PYG{o}{*} \PYG{n}{sigma}\PYG{o}{*}\PYG{o}{*}\PYG{l+m+mi}{2}\PYG{p}{)} \PYG{p}{)}\PYG{p}{,}
\PYG{g+gp}{... }         \PYG{n}{linewidth}\PYG{o}{=}\PYG{l+m+mi}{2}\PYG{p}{,} \PYG{n}{color}\PYG{o}{=}\PYG{l+s}{\PYGZsq{}}\PYG{l+s}{r}\PYG{l+s}{\PYGZsq{}}\PYG{p}{)}
\PYG{g+gp}{\PYGZgt{}\PYGZgt{}\PYGZgt{} }\PYG{n}{plt}\PYG{o}{.}\PYG{n}{show}\PYG{p}{(}\PYG{p}{)}
\end{Verbatim}

\end{fulllineitems}

\index{pareto() (in module acsSpeciesActivities)}

\begin{fulllineitems}
\phantomsection\label{acsSpeciesActivities:acsSpeciesActivities.pareto}\pysiglinewithargsret{\code{acsSpeciesActivities.}\bfcode{pareto}}{\emph{a}, \emph{size=None}}{}
Draw samples from a Pareto II or Lomax distribution with specified shape.

The Lomax or Pareto II distribution is a shifted Pareto distribution. The
classical Pareto distribution can be obtained from the Lomax distribution
by adding the location parameter m, see below. The smallest value of the
Lomax distribution is zero while for the classical Pareto distribution it
is m, where the standard Pareto distribution has location m=1.
Lomax can also be considered as a simplified version of the Generalized
Pareto distribution (available in SciPy), with the scale set to one and
the location set to zero.

The Pareto distribution must be greater than zero, and is unbounded above.
It is also known as the ``80-20 rule''.  In this distribution, 80 percent of
the weights are in the lowest 20 percent of the range, while the other 20
percent fill the remaining 80 percent of the range.
\begin{description}
\item[{shape}] \leavevmode{[}float, \textgreater{} 0.{]}
Shape of the distribution.

\item[{size}] \leavevmode{[}tuple of ints{]}
Output shape.  If the given shape is, e.g., \code{(m, n, k)}, then
\code{m * n * k} samples are drawn.

\end{description}
\begin{description}
\item[{scipy.stats.distributions.lomax.pdf}] \leavevmode{[}probability density function,{]}
distribution or cumulative density function, etc.

\item[{scipy.stats.distributions.genpareto.pdf}] \leavevmode{[}probability density function,{]}
distribution or cumulative density function, etc.

\end{description}

The probability density for the Pareto distribution is
\begin{gather}
\begin{split}p(x) = \frac{am^a}{x^{a+1}}\end{split}\notag
\end{gather}
where \(a\) is the shape and \(m\) the location

The Pareto distribution, named after the Italian economist Vilfredo Pareto,
is a power law probability distribution useful in many real world problems.
Outside the field of economics it is generally referred to as the Bradford
distribution. Pareto developed the distribution to describe the
distribution of wealth in an economy.  It has also found use in insurance,
web page access statistics, oil field sizes, and many other problems,
including the download frequency for projects in Sourceforge {[}1{]}.  It is
one of the so-called ``fat-tailed'' distributions.

Draw samples from the distribution:

\begin{Verbatim}[commandchars=\\\{\}]
\PYG{g+gp}{\PYGZgt{}\PYGZgt{}\PYGZgt{} }\PYG{n}{a}\PYG{p}{,} \PYG{n}{m} \PYG{o}{=} \PYG{l+m+mf}{3.}\PYG{p}{,} \PYG{l+m+mf}{1.} \PYG{c}{\PYGZsh{} shape and mode}
\PYG{g+gp}{\PYGZgt{}\PYGZgt{}\PYGZgt{} }\PYG{n}{s} \PYG{o}{=} \PYG{n}{np}\PYG{o}{.}\PYG{n}{random}\PYG{o}{.}\PYG{n}{pareto}\PYG{p}{(}\PYG{n}{a}\PYG{p}{,} \PYG{l+m+mi}{1000}\PYG{p}{)} \PYG{o}{+} \PYG{n}{m}
\end{Verbatim}

Display the histogram of the samples, along with
the probability density function:

\begin{Verbatim}[commandchars=\\\{\}]
\PYG{g+gp}{\PYGZgt{}\PYGZgt{}\PYGZgt{} }\PYG{k+kn}{import} \PYG{n+nn}{matplotlib.pyplot} \PYG{k+kn}{as} \PYG{n+nn}{plt}
\PYG{g+gp}{\PYGZgt{}\PYGZgt{}\PYGZgt{} }\PYG{n}{count}\PYG{p}{,} \PYG{n}{bins}\PYG{p}{,} \PYG{n}{ignored} \PYG{o}{=} \PYG{n}{plt}\PYG{o}{.}\PYG{n}{hist}\PYG{p}{(}\PYG{n}{s}\PYG{p}{,} \PYG{l+m+mi}{100}\PYG{p}{,} \PYG{n}{normed}\PYG{o}{=}\PYG{n+nb+bp}{True}\PYG{p}{,} \PYG{n}{align}\PYG{o}{=}\PYG{l+s}{\PYGZsq{}}\PYG{l+s}{center}\PYG{l+s}{\PYGZsq{}}\PYG{p}{)}
\PYG{g+gp}{\PYGZgt{}\PYGZgt{}\PYGZgt{} }\PYG{n}{fit} \PYG{o}{=} \PYG{n}{a}\PYG{o}{*}\PYG{n}{m}\PYG{o}{*}\PYG{o}{*}\PYG{n}{a}\PYG{o}{/}\PYG{n}{bins}\PYG{o}{*}\PYG{o}{*}\PYG{p}{(}\PYG{n}{a}\PYG{o}{+}\PYG{l+m+mi}{1}\PYG{p}{)}
\PYG{g+gp}{\PYGZgt{}\PYGZgt{}\PYGZgt{} }\PYG{n}{plt}\PYG{o}{.}\PYG{n}{plot}\PYG{p}{(}\PYG{n}{bins}\PYG{p}{,} \PYG{n+nb}{max}\PYG{p}{(}\PYG{n}{count}\PYG{p}{)}\PYG{o}{*}\PYG{n}{fit}\PYG{o}{/}\PYG{n+nb}{max}\PYG{p}{(}\PYG{n}{fit}\PYG{p}{)}\PYG{p}{,}\PYG{n}{linewidth}\PYG{o}{=}\PYG{l+m+mi}{2}\PYG{p}{,} \PYG{n}{color}\PYG{o}{=}\PYG{l+s}{\PYGZsq{}}\PYG{l+s}{r}\PYG{l+s}{\PYGZsq{}}\PYG{p}{)}
\PYG{g+gp}{\PYGZgt{}\PYGZgt{}\PYGZgt{} }\PYG{n}{plt}\PYG{o}{.}\PYG{n}{show}\PYG{p}{(}\PYG{p}{)}
\end{Verbatim}

\end{fulllineitems}

\index{permutation() (in module acsSpeciesActivities)}

\begin{fulllineitems}
\phantomsection\label{acsSpeciesActivities:acsSpeciesActivities.permutation}\pysiglinewithargsret{\code{acsSpeciesActivities.}\bfcode{permutation}}{\emph{x}}{}
Randomly permute a sequence, or return a permuted range.

If \emph{x} is a multi-dimensional array, it is only shuffled along its
first index.
\begin{description}
\item[{x}] \leavevmode{[}int or array\_like{]}
If \emph{x} is an integer, randomly permute \code{np.arange(x)}.
If \emph{x} is an array, make a copy and shuffle the elements
randomly.

\end{description}
\begin{description}
\item[{out}] \leavevmode{[}ndarray{]}
Permuted sequence or array range.

\end{description}

\begin{Verbatim}[commandchars=\\\{\}]
\PYG{g+gp}{\PYGZgt{}\PYGZgt{}\PYGZgt{} }\PYG{n}{np}\PYG{o}{.}\PYG{n}{random}\PYG{o}{.}\PYG{n}{permutation}\PYG{p}{(}\PYG{l+m+mi}{10}\PYG{p}{)}
\PYG{g+go}{array([1, 7, 4, 3, 0, 9, 2, 5, 8, 6])}
\end{Verbatim}

\begin{Verbatim}[commandchars=\\\{\}]
\PYG{g+gp}{\PYGZgt{}\PYGZgt{}\PYGZgt{} }\PYG{n}{np}\PYG{o}{.}\PYG{n}{random}\PYG{o}{.}\PYG{n}{permutation}\PYG{p}{(}\PYG{p}{[}\PYG{l+m+mi}{1}\PYG{p}{,} \PYG{l+m+mi}{4}\PYG{p}{,} \PYG{l+m+mi}{9}\PYG{p}{,} \PYG{l+m+mi}{12}\PYG{p}{,} \PYG{l+m+mi}{15}\PYG{p}{]}\PYG{p}{)}
\PYG{g+go}{array([15,  1,  9,  4, 12])}
\end{Verbatim}

\begin{Verbatim}[commandchars=\\\{\}]
\PYG{g+gp}{\PYGZgt{}\PYGZgt{}\PYGZgt{} }\PYG{n}{arr} \PYG{o}{=} \PYG{n}{np}\PYG{o}{.}\PYG{n}{arange}\PYG{p}{(}\PYG{l+m+mi}{9}\PYG{p}{)}\PYG{o}{.}\PYG{n}{reshape}\PYG{p}{(}\PYG{p}{(}\PYG{l+m+mi}{3}\PYG{p}{,} \PYG{l+m+mi}{3}\PYG{p}{)}\PYG{p}{)}
\PYG{g+gp}{\PYGZgt{}\PYGZgt{}\PYGZgt{} }\PYG{n}{np}\PYG{o}{.}\PYG{n}{random}\PYG{o}{.}\PYG{n}{permutation}\PYG{p}{(}\PYG{n}{arr}\PYG{p}{)}
\PYG{g+go}{array([[6, 7, 8],}
\PYG{g+go}{       [0, 1, 2],}
\PYG{g+go}{       [3, 4, 5]])}
\end{Verbatim}

\end{fulllineitems}

\index{poisson() (in module acsSpeciesActivities)}

\begin{fulllineitems}
\phantomsection\label{acsSpeciesActivities:acsSpeciesActivities.poisson}\pysiglinewithargsret{\code{acsSpeciesActivities.}\bfcode{poisson}}{\emph{lam=1.0}, \emph{size=None}}{}
Draw samples from a Poisson distribution.

The Poisson distribution is the limit of the Binomial
distribution for large N.
\begin{description}
\item[{lam}] \leavevmode{[}float{]}
Expectation of interval, should be \textgreater{}= 0.

\item[{size}] \leavevmode{[}int or tuple of ints, optional{]}
Output shape. If the given shape is, e.g., \code{(m, n, k)}, then
\code{m * n * k} samples are drawn.

\end{description}

The Poisson distribution
\begin{gather}
\begin{split}f(k; \lambda)=\frac{\lambda^k e^{-\lambda}}{k!}\end{split}\notag
\end{gather}
For events with an expected separation \(\lambda\) the Poisson
distribution \(f(k; \lambda)\) describes the probability of
\(k\) events occurring within the observed interval \(\lambda\).

Because the output is limited to the range of the C long type, a
ValueError is raised when \emph{lam} is within 10 sigma of the maximum
representable value.

Draw samples from the distribution:

\begin{Verbatim}[commandchars=\\\{\}]
\PYG{g+gp}{\PYGZgt{}\PYGZgt{}\PYGZgt{} }\PYG{k+kn}{import} \PYG{n+nn}{numpy} \PYG{k+kn}{as} \PYG{n+nn}{np}
\PYG{g+gp}{\PYGZgt{}\PYGZgt{}\PYGZgt{} }\PYG{n}{s} \PYG{o}{=} \PYG{n}{np}\PYG{o}{.}\PYG{n}{random}\PYG{o}{.}\PYG{n}{poisson}\PYG{p}{(}\PYG{l+m+mi}{5}\PYG{p}{,} \PYG{l+m+mi}{10000}\PYG{p}{)}
\end{Verbatim}

Display histogram of the sample:

\begin{Verbatim}[commandchars=\\\{\}]
\PYG{g+gp}{\PYGZgt{}\PYGZgt{}\PYGZgt{} }\PYG{k+kn}{import} \PYG{n+nn}{matplotlib.pyplot} \PYG{k+kn}{as} \PYG{n+nn}{plt}
\PYG{g+gp}{\PYGZgt{}\PYGZgt{}\PYGZgt{} }\PYG{n}{count}\PYG{p}{,} \PYG{n}{bins}\PYG{p}{,} \PYG{n}{ignored} \PYG{o}{=} \PYG{n}{plt}\PYG{o}{.}\PYG{n}{hist}\PYG{p}{(}\PYG{n}{s}\PYG{p}{,} \PYG{l+m+mi}{14}\PYG{p}{,} \PYG{n}{normed}\PYG{o}{=}\PYG{n+nb+bp}{True}\PYG{p}{)}
\PYG{g+gp}{\PYGZgt{}\PYGZgt{}\PYGZgt{} }\PYG{n}{plt}\PYG{o}{.}\PYG{n}{show}\PYG{p}{(}\PYG{p}{)}
\end{Verbatim}

\end{fulllineitems}

\index{power() (in module acsSpeciesActivities)}

\begin{fulllineitems}
\phantomsection\label{acsSpeciesActivities:acsSpeciesActivities.power}\pysiglinewithargsret{\code{acsSpeciesActivities.}\bfcode{power}}{\emph{a}, \emph{size=None}}{}
Draws samples in {[}0, 1{]} from a power distribution with positive
exponent a - 1.

Also known as the power function distribution.
\begin{description}
\item[{a}] \leavevmode{[}float{]}
parameter, \textgreater{} 0

\item[{size}] \leavevmode{[}tuple of ints{]}\begin{description}
\item[{Output shape.  If the given shape is, e.g., \code{(m, n, k)}, then}] \leavevmode
\code{m * n * k} samples are drawn.

\end{description}

\end{description}
\begin{description}
\item[{samples}] \leavevmode{[}\{ndarray, scalar\}{]}
The returned samples lie in {[}0, 1{]}.

\end{description}
\begin{description}
\item[{ValueError}] \leavevmode
If a\textless{}1.

\end{description}

The probability density function is
\begin{gather}
\begin{split}P(x; a) = ax^{a-1}, 0 \le x \le 1, a>0.\end{split}\notag
\end{gather}
The power function distribution is just the inverse of the Pareto
distribution. It may also be seen as a special case of the Beta
distribution.

It is used, for example, in modeling the over-reporting of insurance
claims.

Draw samples from the distribution:

\begin{Verbatim}[commandchars=\\\{\}]
\PYG{g+gp}{\PYGZgt{}\PYGZgt{}\PYGZgt{} }\PYG{n}{a} \PYG{o}{=} \PYG{l+m+mf}{5.} \PYG{c}{\PYGZsh{} shape}
\PYG{g+gp}{\PYGZgt{}\PYGZgt{}\PYGZgt{} }\PYG{n}{samples} \PYG{o}{=} \PYG{l+m+mi}{1000}
\PYG{g+gp}{\PYGZgt{}\PYGZgt{}\PYGZgt{} }\PYG{n}{s} \PYG{o}{=} \PYG{n}{np}\PYG{o}{.}\PYG{n}{random}\PYG{o}{.}\PYG{n}{power}\PYG{p}{(}\PYG{n}{a}\PYG{p}{,} \PYG{n}{samples}\PYG{p}{)}
\end{Verbatim}

Display the histogram of the samples, along with
the probability density function:

\begin{Verbatim}[commandchars=\\\{\}]
\PYG{g+gp}{\PYGZgt{}\PYGZgt{}\PYGZgt{} }\PYG{k+kn}{import} \PYG{n+nn}{matplotlib.pyplot} \PYG{k+kn}{as} \PYG{n+nn}{plt}
\PYG{g+gp}{\PYGZgt{}\PYGZgt{}\PYGZgt{} }\PYG{n}{count}\PYG{p}{,} \PYG{n}{bins}\PYG{p}{,} \PYG{n}{ignored} \PYG{o}{=} \PYG{n}{plt}\PYG{o}{.}\PYG{n}{hist}\PYG{p}{(}\PYG{n}{s}\PYG{p}{,} \PYG{n}{bins}\PYG{o}{=}\PYG{l+m+mi}{30}\PYG{p}{)}
\PYG{g+gp}{\PYGZgt{}\PYGZgt{}\PYGZgt{} }\PYG{n}{x} \PYG{o}{=} \PYG{n}{np}\PYG{o}{.}\PYG{n}{linspace}\PYG{p}{(}\PYG{l+m+mi}{0}\PYG{p}{,} \PYG{l+m+mi}{1}\PYG{p}{,} \PYG{l+m+mi}{100}\PYG{p}{)}
\PYG{g+gp}{\PYGZgt{}\PYGZgt{}\PYGZgt{} }\PYG{n}{y} \PYG{o}{=} \PYG{n}{a}\PYG{o}{*}\PYG{n}{x}\PYG{o}{*}\PYG{o}{*}\PYG{p}{(}\PYG{n}{a}\PYG{o}{\PYGZhy{}}\PYG{l+m+mf}{1.}\PYG{p}{)}
\PYG{g+gp}{\PYGZgt{}\PYGZgt{}\PYGZgt{} }\PYG{n}{normed\PYGZus{}y} \PYG{o}{=} \PYG{n}{samples}\PYG{o}{*}\PYG{n}{np}\PYG{o}{.}\PYG{n}{diff}\PYG{p}{(}\PYG{n}{bins}\PYG{p}{)}\PYG{p}{[}\PYG{l+m+mi}{0}\PYG{p}{]}\PYG{o}{*}\PYG{n}{y}
\PYG{g+gp}{\PYGZgt{}\PYGZgt{}\PYGZgt{} }\PYG{n}{plt}\PYG{o}{.}\PYG{n}{plot}\PYG{p}{(}\PYG{n}{x}\PYG{p}{,} \PYG{n}{normed\PYGZus{}y}\PYG{p}{)}
\PYG{g+gp}{\PYGZgt{}\PYGZgt{}\PYGZgt{} }\PYG{n}{plt}\PYG{o}{.}\PYG{n}{show}\PYG{p}{(}\PYG{p}{)}
\end{Verbatim}

Compare the power function distribution to the inverse of the Pareto.

\begin{Verbatim}[commandchars=\\\{\}]
\PYG{g+gp}{\PYGZgt{}\PYGZgt{}\PYGZgt{} }\PYG{k+kn}{from} \PYG{n+nn}{scipy} \PYG{k+kn}{import} \PYG{n}{stats}
\PYG{g+gp}{\PYGZgt{}\PYGZgt{}\PYGZgt{} }\PYG{n}{rvs} \PYG{o}{=} \PYG{n}{np}\PYG{o}{.}\PYG{n}{random}\PYG{o}{.}\PYG{n}{power}\PYG{p}{(}\PYG{l+m+mi}{5}\PYG{p}{,} \PYG{l+m+mi}{1000000}\PYG{p}{)}
\PYG{g+gp}{\PYGZgt{}\PYGZgt{}\PYGZgt{} }\PYG{n}{rvsp} \PYG{o}{=} \PYG{n}{np}\PYG{o}{.}\PYG{n}{random}\PYG{o}{.}\PYG{n}{pareto}\PYG{p}{(}\PYG{l+m+mi}{5}\PYG{p}{,} \PYG{l+m+mi}{1000000}\PYG{p}{)}
\PYG{g+gp}{\PYGZgt{}\PYGZgt{}\PYGZgt{} }\PYG{n}{xx} \PYG{o}{=} \PYG{n}{np}\PYG{o}{.}\PYG{n}{linspace}\PYG{p}{(}\PYG{l+m+mi}{0}\PYG{p}{,}\PYG{l+m+mi}{1}\PYG{p}{,}\PYG{l+m+mi}{100}\PYG{p}{)}
\PYG{g+gp}{\PYGZgt{}\PYGZgt{}\PYGZgt{} }\PYG{n}{powpdf} \PYG{o}{=} \PYG{n}{stats}\PYG{o}{.}\PYG{n}{powerlaw}\PYG{o}{.}\PYG{n}{pdf}\PYG{p}{(}\PYG{n}{xx}\PYG{p}{,}\PYG{l+m+mi}{5}\PYG{p}{)}
\end{Verbatim}

\begin{Verbatim}[commandchars=\\\{\}]
\PYG{g+gp}{\PYGZgt{}\PYGZgt{}\PYGZgt{} }\PYG{n}{plt}\PYG{o}{.}\PYG{n}{figure}\PYG{p}{(}\PYG{p}{)}
\PYG{g+gp}{\PYGZgt{}\PYGZgt{}\PYGZgt{} }\PYG{n}{plt}\PYG{o}{.}\PYG{n}{hist}\PYG{p}{(}\PYG{n}{rvs}\PYG{p}{,} \PYG{n}{bins}\PYG{o}{=}\PYG{l+m+mi}{50}\PYG{p}{,} \PYG{n}{normed}\PYG{o}{=}\PYG{n+nb+bp}{True}\PYG{p}{)}
\PYG{g+gp}{\PYGZgt{}\PYGZgt{}\PYGZgt{} }\PYG{n}{plt}\PYG{o}{.}\PYG{n}{plot}\PYG{p}{(}\PYG{n}{xx}\PYG{p}{,}\PYG{n}{powpdf}\PYG{p}{,}\PYG{l+s}{\PYGZsq{}}\PYG{l+s}{r\PYGZhy{}}\PYG{l+s}{\PYGZsq{}}\PYG{p}{)}
\PYG{g+gp}{\PYGZgt{}\PYGZgt{}\PYGZgt{} }\PYG{n}{plt}\PYG{o}{.}\PYG{n}{title}\PYG{p}{(}\PYG{l+s}{\PYGZsq{}}\PYG{l+s}{np.random.power(5)}\PYG{l+s}{\PYGZsq{}}\PYG{p}{)}
\end{Verbatim}

\begin{Verbatim}[commandchars=\\\{\}]
\PYG{g+gp}{\PYGZgt{}\PYGZgt{}\PYGZgt{} }\PYG{n}{plt}\PYG{o}{.}\PYG{n}{figure}\PYG{p}{(}\PYG{p}{)}
\PYG{g+gp}{\PYGZgt{}\PYGZgt{}\PYGZgt{} }\PYG{n}{plt}\PYG{o}{.}\PYG{n}{hist}\PYG{p}{(}\PYG{l+m+mf}{1.}\PYG{o}{/}\PYG{p}{(}\PYG{l+m+mf}{1.}\PYG{o}{+}\PYG{n}{rvsp}\PYG{p}{)}\PYG{p}{,} \PYG{n}{bins}\PYG{o}{=}\PYG{l+m+mi}{50}\PYG{p}{,} \PYG{n}{normed}\PYG{o}{=}\PYG{n+nb+bp}{True}\PYG{p}{)}
\PYG{g+gp}{\PYGZgt{}\PYGZgt{}\PYGZgt{} }\PYG{n}{plt}\PYG{o}{.}\PYG{n}{plot}\PYG{p}{(}\PYG{n}{xx}\PYG{p}{,}\PYG{n}{powpdf}\PYG{p}{,}\PYG{l+s}{\PYGZsq{}}\PYG{l+s}{r\PYGZhy{}}\PYG{l+s}{\PYGZsq{}}\PYG{p}{)}
\PYG{g+gp}{\PYGZgt{}\PYGZgt{}\PYGZgt{} }\PYG{n}{plt}\PYG{o}{.}\PYG{n}{title}\PYG{p}{(}\PYG{l+s}{\PYGZsq{}}\PYG{l+s}{inverse of 1 + np.random.pareto(5)}\PYG{l+s}{\PYGZsq{}}\PYG{p}{)}
\end{Verbatim}

\begin{Verbatim}[commandchars=\\\{\}]
\PYG{g+gp}{\PYGZgt{}\PYGZgt{}\PYGZgt{} }\PYG{n}{plt}\PYG{o}{.}\PYG{n}{figure}\PYG{p}{(}\PYG{p}{)}
\PYG{g+gp}{\PYGZgt{}\PYGZgt{}\PYGZgt{} }\PYG{n}{plt}\PYG{o}{.}\PYG{n}{hist}\PYG{p}{(}\PYG{l+m+mf}{1.}\PYG{o}{/}\PYG{p}{(}\PYG{l+m+mf}{1.}\PYG{o}{+}\PYG{n}{rvsp}\PYG{p}{)}\PYG{p}{,} \PYG{n}{bins}\PYG{o}{=}\PYG{l+m+mi}{50}\PYG{p}{,} \PYG{n}{normed}\PYG{o}{=}\PYG{n+nb+bp}{True}\PYG{p}{)}
\PYG{g+gp}{\PYGZgt{}\PYGZgt{}\PYGZgt{} }\PYG{n}{plt}\PYG{o}{.}\PYG{n}{plot}\PYG{p}{(}\PYG{n}{xx}\PYG{p}{,}\PYG{n}{powpdf}\PYG{p}{,}\PYG{l+s}{\PYGZsq{}}\PYG{l+s}{r\PYGZhy{}}\PYG{l+s}{\PYGZsq{}}\PYG{p}{)}
\PYG{g+gp}{\PYGZgt{}\PYGZgt{}\PYGZgt{} }\PYG{n}{plt}\PYG{o}{.}\PYG{n}{title}\PYG{p}{(}\PYG{l+s}{\PYGZsq{}}\PYG{l+s}{inverse of stats.pareto(5)}\PYG{l+s}{\PYGZsq{}}\PYG{p}{)}
\end{Verbatim}

\end{fulllineitems}

\index{rand() (in module acsSpeciesActivities)}

\begin{fulllineitems}
\phantomsection\label{acsSpeciesActivities:acsSpeciesActivities.rand}\pysiglinewithargsret{\code{acsSpeciesActivities.}\bfcode{rand}}{\emph{d0}, \emph{d1}, \emph{...}, \emph{dn}}{}
Random values in a given shape.

Create an array of the given shape and propagate it with
random samples from a uniform distribution
over \code{{[}0, 1)}.
\begin{description}
\item[{d0, d1, ..., dn}] \leavevmode{[}int, optional{]}
The dimensions of the returned array, should all be positive.
If no argument is given a single Python float is returned.

\end{description}
\begin{description}
\item[{out}] \leavevmode{[}ndarray, shape \code{(d0, d1, ..., dn)}{]}
Random values.

\end{description}

random

This is a convenience function. If you want an interface that
takes a shape-tuple as the first argument, refer to
np.random.random\_sample .

\begin{Verbatim}[commandchars=\\\{\}]
\PYG{g+gp}{\PYGZgt{}\PYGZgt{}\PYGZgt{} }\PYG{n}{np}\PYG{o}{.}\PYG{n}{random}\PYG{o}{.}\PYG{n}{rand}\PYG{p}{(}\PYG{l+m+mi}{3}\PYG{p}{,}\PYG{l+m+mi}{2}\PYG{p}{)}
\PYG{g+go}{array([[ 0.14022471,  0.96360618],  \PYGZsh{}random}
\PYG{g+go}{       [ 0.37601032,  0.25528411],  \PYGZsh{}random}
\PYG{g+go}{       [ 0.49313049,  0.94909878]]) \PYGZsh{}random}
\end{Verbatim}

\end{fulllineitems}

\index{randint() (in module acsSpeciesActivities)}

\begin{fulllineitems}
\phantomsection\label{acsSpeciesActivities:acsSpeciesActivities.randint}\pysiglinewithargsret{\code{acsSpeciesActivities.}\bfcode{randint}}{\emph{low}, \emph{high=None}, \emph{size=None}}{}
Return random integers from \emph{low} (inclusive) to \emph{high} (exclusive).

Return random integers from the ``discrete uniform'' distribution in the
``half-open'' interval {[}\emph{low}, \emph{high}). If \emph{high} is None (the default),
then results are from {[}0, \emph{low}).
\begin{description}
\item[{low}] \leavevmode{[}int{]}
Lowest (signed) integer to be drawn from the distribution (unless
\code{high=None}, in which case this parameter is the \emph{highest} such
integer).

\item[{high}] \leavevmode{[}int, optional{]}
If provided, one above the largest (signed) integer to be drawn
from the distribution (see above for behavior if \code{high=None}).

\item[{size}] \leavevmode{[}int or tuple of ints, optional{]}
Output shape. Default is None, in which case a single int is
returned.

\end{description}
\begin{description}
\item[{out}] \leavevmode{[}int or ndarray of ints{]}
\emph{size}-shaped array of random integers from the appropriate
distribution, or a single such random int if \emph{size} not provided.

\end{description}
\begin{description}
\item[{random.random\_integers}] \leavevmode{[}similar to \emph{randint}, only for the closed{]}
interval {[}\emph{low}, \emph{high}{]}, and 1 is the lowest value if \emph{high} is
omitted. In particular, this other one is the one to use to generate
uniformly distributed discrete non-integers.

\end{description}

\begin{Verbatim}[commandchars=\\\{\}]
\PYG{g+gp}{\PYGZgt{}\PYGZgt{}\PYGZgt{} }\PYG{n}{np}\PYG{o}{.}\PYG{n}{random}\PYG{o}{.}\PYG{n}{randint}\PYG{p}{(}\PYG{l+m+mi}{2}\PYG{p}{,} \PYG{n}{size}\PYG{o}{=}\PYG{l+m+mi}{10}\PYG{p}{)}
\PYG{g+go}{array([1, 0, 0, 0, 1, 1, 0, 0, 1, 0])}
\PYG{g+gp}{\PYGZgt{}\PYGZgt{}\PYGZgt{} }\PYG{n}{np}\PYG{o}{.}\PYG{n}{random}\PYG{o}{.}\PYG{n}{randint}\PYG{p}{(}\PYG{l+m+mi}{1}\PYG{p}{,} \PYG{n}{size}\PYG{o}{=}\PYG{l+m+mi}{10}\PYG{p}{)}
\PYG{g+go}{array([0, 0, 0, 0, 0, 0, 0, 0, 0, 0])}
\end{Verbatim}

Generate a 2 x 4 array of ints between 0 and 4, inclusive:

\begin{Verbatim}[commandchars=\\\{\}]
\PYG{g+gp}{\PYGZgt{}\PYGZgt{}\PYGZgt{} }\PYG{n}{np}\PYG{o}{.}\PYG{n}{random}\PYG{o}{.}\PYG{n}{randint}\PYG{p}{(}\PYG{l+m+mi}{5}\PYG{p}{,} \PYG{n}{size}\PYG{o}{=}\PYG{p}{(}\PYG{l+m+mi}{2}\PYG{p}{,} \PYG{l+m+mi}{4}\PYG{p}{)}\PYG{p}{)}
\PYG{g+go}{array([[4, 0, 2, 1],}
\PYG{g+go}{       [3, 2, 2, 0]])}
\end{Verbatim}

\end{fulllineitems}

\index{randn() (in module acsSpeciesActivities)}

\begin{fulllineitems}
\phantomsection\label{acsSpeciesActivities:acsSpeciesActivities.randn}\pysiglinewithargsret{\code{acsSpeciesActivities.}\bfcode{randn}}{\emph{d0}, \emph{d1}, \emph{...}, \emph{dn}}{}
Return a sample (or samples) from the ``standard normal'' distribution.

If positive, int\_like or int-convertible arguments are provided,
\emph{randn} generates an array of shape \code{(d0, d1, ..., dn)}, filled
with random floats sampled from a univariate ``normal'' (Gaussian)
distribution of mean 0 and variance 1 (if any of the \(d_i\) are
floats, they are first converted to integers by truncation). A single
float randomly sampled from the distribution is returned if no
argument is provided.

This is a convenience function.  If you want an interface that takes a
tuple as the first argument, use \emph{numpy.random.standard\_normal} instead.
\begin{description}
\item[{d0, d1, ..., dn}] \leavevmode{[}int, optional{]}
The dimensions of the returned array, should be all positive.
If no argument is given a single Python float is returned.

\end{description}
\begin{description}
\item[{Z}] \leavevmode{[}ndarray or float{]}
A \code{(d0, d1, ..., dn)}-shaped array of floating-point samples from
the standard normal distribution, or a single such float if
no parameters were supplied.

\end{description}

random.standard\_normal : Similar, but takes a tuple as its argument.

For random samples from \(N(\mu, \sigma^2)\), use:

\code{sigma * np.random.randn(...) + mu}

\begin{Verbatim}[commandchars=\\\{\}]
\PYG{g+gp}{\PYGZgt{}\PYGZgt{}\PYGZgt{} }\PYG{n}{np}\PYG{o}{.}\PYG{n}{random}\PYG{o}{.}\PYG{n}{randn}\PYG{p}{(}\PYG{p}{)}
\PYG{g+go}{2.1923875335537315 \PYGZsh{}random}
\end{Verbatim}

Two-by-four array of samples from N(3, 6.25):

\begin{Verbatim}[commandchars=\\\{\}]
\PYG{g+gp}{\PYGZgt{}\PYGZgt{}\PYGZgt{} }\PYG{l+m+mf}{2.5} \PYG{o}{*} \PYG{n}{np}\PYG{o}{.}\PYG{n}{random}\PYG{o}{.}\PYG{n}{randn}\PYG{p}{(}\PYG{l+m+mi}{2}\PYG{p}{,} \PYG{l+m+mi}{4}\PYG{p}{)} \PYG{o}{+} \PYG{l+m+mi}{3}
\PYG{g+go}{array([[\PYGZhy{}4.49401501,  4.00950034, \PYGZhy{}1.81814867,  7.29718677],  \PYGZsh{}random}
\PYG{g+go}{       [ 0.39924804,  4.68456316,  4.99394529,  4.84057254]]) \PYGZsh{}random}
\end{Verbatim}

\end{fulllineitems}

\index{random() (in module acsSpeciesActivities)}

\begin{fulllineitems}
\phantomsection\label{acsSpeciesActivities:acsSpeciesActivities.random}\pysiglinewithargsret{\code{acsSpeciesActivities.}\bfcode{random}}{}{}
random\_sample(size=None)

Return random floats in the half-open interval {[}0.0, 1.0).

Results are from the ``continuous uniform'' distribution over the
stated interval.  To sample \(Unif[a, b), b > a\) multiply
the output of \emph{random\_sample} by \emph{(b-a)} and add \emph{a}:

\begin{Verbatim}[commandchars=\\\{\}]
\PYG{p}{(}\PYG{n}{b} \PYG{o}{\PYGZhy{}} \PYG{n}{a}\PYG{p}{)} \PYG{o}{*} \PYG{n}{random\PYGZus{}sample}\PYG{p}{(}\PYG{p}{)} \PYG{o}{+} \PYG{n}{a}
\end{Verbatim}
\begin{description}
\item[{size}] \leavevmode{[}int or tuple of ints, optional{]}
Defines the shape of the returned array of random floats. If None
(the default), returns a single float.

\end{description}
\begin{description}
\item[{out}] \leavevmode{[}float or ndarray of floats{]}
Array of random floats of shape \emph{size} (unless \code{size=None}, in which
case a single float is returned).

\end{description}

\begin{Verbatim}[commandchars=\\\{\}]
\PYG{g+gp}{\PYGZgt{}\PYGZgt{}\PYGZgt{} }\PYG{n}{np}\PYG{o}{.}\PYG{n}{random}\PYG{o}{.}\PYG{n}{random\PYGZus{}sample}\PYG{p}{(}\PYG{p}{)}
\PYG{g+go}{0.47108547995356098}
\PYG{g+gp}{\PYGZgt{}\PYGZgt{}\PYGZgt{} }\PYG{n+nb}{type}\PYG{p}{(}\PYG{n}{np}\PYG{o}{.}\PYG{n}{random}\PYG{o}{.}\PYG{n}{random\PYGZus{}sample}\PYG{p}{(}\PYG{p}{)}\PYG{p}{)}
\PYG{g+go}{\PYGZlt{}type \PYGZsq{}float\PYGZsq{}\PYGZgt{}}
\PYG{g+gp}{\PYGZgt{}\PYGZgt{}\PYGZgt{} }\PYG{n}{np}\PYG{o}{.}\PYG{n}{random}\PYG{o}{.}\PYG{n}{random\PYGZus{}sample}\PYG{p}{(}\PYG{p}{(}\PYG{l+m+mi}{5}\PYG{p}{,}\PYG{p}{)}\PYG{p}{)}
\PYG{g+go}{array([ 0.30220482,  0.86820401,  0.1654503 ,  0.11659149,  0.54323428])}
\end{Verbatim}

Three-by-two array of random numbers from {[}-5, 0):

\begin{Verbatim}[commandchars=\\\{\}]
\PYG{g+gp}{\PYGZgt{}\PYGZgt{}\PYGZgt{} }\PYG{l+m+mi}{5} \PYG{o}{*} \PYG{n}{np}\PYG{o}{.}\PYG{n}{random}\PYG{o}{.}\PYG{n}{random\PYGZus{}sample}\PYG{p}{(}\PYG{p}{(}\PYG{l+m+mi}{3}\PYG{p}{,} \PYG{l+m+mi}{2}\PYG{p}{)}\PYG{p}{)} \PYG{o}{\PYGZhy{}} \PYG{l+m+mi}{5}
\PYG{g+go}{array([[\PYGZhy{}3.99149989, \PYGZhy{}0.52338984],}
\PYG{g+go}{       [\PYGZhy{}2.99091858, \PYGZhy{}0.79479508],}
\PYG{g+go}{       [\PYGZhy{}1.23204345, \PYGZhy{}1.75224494]])}
\end{Verbatim}

\end{fulllineitems}

\index{random\_integers() (in module acsSpeciesActivities)}

\begin{fulllineitems}
\phantomsection\label{acsSpeciesActivities:acsSpeciesActivities.random_integers}\pysiglinewithargsret{\code{acsSpeciesActivities.}\bfcode{random\_integers}}{\emph{low}, \emph{high=None}, \emph{size=None}}{}
Return random integers between \emph{low} and \emph{high}, inclusive.

Return random integers from the ``discrete uniform'' distribution in the
closed interval {[}\emph{low}, \emph{high}{]}.  If \emph{high} is None (the default),
then results are from {[}1, \emph{low}{]}.
\begin{description}
\item[{low}] \leavevmode{[}int{]}
Lowest (signed) integer to be drawn from the distribution (unless
\code{high=None}, in which case this parameter is the \emph{highest} such
integer).

\item[{high}] \leavevmode{[}int, optional{]}
If provided, the largest (signed) integer to be drawn from the
distribution (see above for behavior if \code{high=None}).

\item[{size}] \leavevmode{[}int or tuple of ints, optional{]}
Output shape. Default is None, in which case a single int is returned.

\end{description}
\begin{description}
\item[{out}] \leavevmode{[}int or ndarray of ints{]}
\emph{size}-shaped array of random integers from the appropriate
distribution, or a single such random int if \emph{size} not provided.

\end{description}
\begin{description}
\item[{random.randint}] \leavevmode{[}Similar to \emph{random\_integers}, only for the half-open{]}
interval {[}\emph{low}, \emph{high}), and 0 is the lowest value if \emph{high} is
omitted.

\end{description}

To sample from N evenly spaced floating-point numbers between a and b,
use:

\begin{Verbatim}[commandchars=\\\{\}]
\PYG{n}{a} \PYG{o}{+} \PYG{p}{(}\PYG{n}{b} \PYG{o}{\PYGZhy{}} \PYG{n}{a}\PYG{p}{)} \PYG{o}{*} \PYG{p}{(}\PYG{n}{np}\PYG{o}{.}\PYG{n}{random}\PYG{o}{.}\PYG{n}{random\PYGZus{}integers}\PYG{p}{(}\PYG{n}{N}\PYG{p}{)} \PYG{o}{\PYGZhy{}} \PYG{l+m+mi}{1}\PYG{p}{)} \PYG{o}{/} \PYG{p}{(}\PYG{n}{N} \PYG{o}{\PYGZhy{}} \PYG{l+m+mf}{1.}\PYG{p}{)}
\end{Verbatim}

\begin{Verbatim}[commandchars=\\\{\}]
\PYG{g+gp}{\PYGZgt{}\PYGZgt{}\PYGZgt{} }\PYG{n}{np}\PYG{o}{.}\PYG{n}{random}\PYG{o}{.}\PYG{n}{random\PYGZus{}integers}\PYG{p}{(}\PYG{l+m+mi}{5}\PYG{p}{)}
\PYG{g+go}{4}
\PYG{g+gp}{\PYGZgt{}\PYGZgt{}\PYGZgt{} }\PYG{n+nb}{type}\PYG{p}{(}\PYG{n}{np}\PYG{o}{.}\PYG{n}{random}\PYG{o}{.}\PYG{n}{random\PYGZus{}integers}\PYG{p}{(}\PYG{l+m+mi}{5}\PYG{p}{)}\PYG{p}{)}
\PYG{g+go}{\PYGZlt{}type \PYGZsq{}int\PYGZsq{}\PYGZgt{}}
\PYG{g+gp}{\PYGZgt{}\PYGZgt{}\PYGZgt{} }\PYG{n}{np}\PYG{o}{.}\PYG{n}{random}\PYG{o}{.}\PYG{n}{random\PYGZus{}integers}\PYG{p}{(}\PYG{l+m+mi}{5}\PYG{p}{,} \PYG{n}{size}\PYG{o}{=}\PYG{p}{(}\PYG{l+m+mf}{3.}\PYG{p}{,}\PYG{l+m+mf}{2.}\PYG{p}{)}\PYG{p}{)}
\PYG{g+go}{array([[5, 4],}
\PYG{g+go}{       [3, 3],}
\PYG{g+go}{       [4, 5]])}
\end{Verbatim}

Choose five random numbers from the set of five evenly-spaced
numbers between 0 and 2.5, inclusive (\emph{i.e.}, from the set
\({0, 5/8, 10/8, 15/8, 20/8}\)):

\begin{Verbatim}[commandchars=\\\{\}]
\PYG{g+gp}{\PYGZgt{}\PYGZgt{}\PYGZgt{} }\PYG{l+m+mf}{2.5} \PYG{o}{*} \PYG{p}{(}\PYG{n}{np}\PYG{o}{.}\PYG{n}{random}\PYG{o}{.}\PYG{n}{random\PYGZus{}integers}\PYG{p}{(}\PYG{l+m+mi}{5}\PYG{p}{,} \PYG{n}{size}\PYG{o}{=}\PYG{p}{(}\PYG{l+m+mi}{5}\PYG{p}{,}\PYG{p}{)}\PYG{p}{)} \PYG{o}{\PYGZhy{}} \PYG{l+m+mi}{1}\PYG{p}{)} \PYG{o}{/} \PYG{l+m+mf}{4.}
\PYG{g+go}{array([ 0.625,  1.25 ,  0.625,  0.625,  2.5  ])}
\end{Verbatim}

Roll two six sided dice 1000 times and sum the results:

\begin{Verbatim}[commandchars=\\\{\}]
\PYG{g+gp}{\PYGZgt{}\PYGZgt{}\PYGZgt{} }\PYG{n}{d1} \PYG{o}{=} \PYG{n}{np}\PYG{o}{.}\PYG{n}{random}\PYG{o}{.}\PYG{n}{random\PYGZus{}integers}\PYG{p}{(}\PYG{l+m+mi}{1}\PYG{p}{,} \PYG{l+m+mi}{6}\PYG{p}{,} \PYG{l+m+mi}{1000}\PYG{p}{)}
\PYG{g+gp}{\PYGZgt{}\PYGZgt{}\PYGZgt{} }\PYG{n}{d2} \PYG{o}{=} \PYG{n}{np}\PYG{o}{.}\PYG{n}{random}\PYG{o}{.}\PYG{n}{random\PYGZus{}integers}\PYG{p}{(}\PYG{l+m+mi}{1}\PYG{p}{,} \PYG{l+m+mi}{6}\PYG{p}{,} \PYG{l+m+mi}{1000}\PYG{p}{)}
\PYG{g+gp}{\PYGZgt{}\PYGZgt{}\PYGZgt{} }\PYG{n}{dsums} \PYG{o}{=} \PYG{n}{d1} \PYG{o}{+} \PYG{n}{d2}
\end{Verbatim}

Display results as a histogram:

\begin{Verbatim}[commandchars=\\\{\}]
\PYG{g+gp}{\PYGZgt{}\PYGZgt{}\PYGZgt{} }\PYG{k+kn}{import} \PYG{n+nn}{matplotlib.pyplot} \PYG{k+kn}{as} \PYG{n+nn}{plt}
\PYG{g+gp}{\PYGZgt{}\PYGZgt{}\PYGZgt{} }\PYG{n}{count}\PYG{p}{,} \PYG{n}{bins}\PYG{p}{,} \PYG{n}{ignored} \PYG{o}{=} \PYG{n}{plt}\PYG{o}{.}\PYG{n}{hist}\PYG{p}{(}\PYG{n}{dsums}\PYG{p}{,} \PYG{l+m+mi}{11}\PYG{p}{,} \PYG{n}{normed}\PYG{o}{=}\PYG{n+nb+bp}{True}\PYG{p}{)}
\PYG{g+gp}{\PYGZgt{}\PYGZgt{}\PYGZgt{} }\PYG{n}{plt}\PYG{o}{.}\PYG{n}{show}\PYG{p}{(}\PYG{p}{)}
\end{Verbatim}

\end{fulllineitems}

\index{random\_sample() (in module acsSpeciesActivities)}

\begin{fulllineitems}
\phantomsection\label{acsSpeciesActivities:acsSpeciesActivities.random_sample}\pysiglinewithargsret{\code{acsSpeciesActivities.}\bfcode{random\_sample}}{\emph{size=None}}{}
Return random floats in the half-open interval {[}0.0, 1.0).

Results are from the ``continuous uniform'' distribution over the
stated interval.  To sample \(Unif[a, b), b > a\) multiply
the output of \emph{random\_sample} by \emph{(b-a)} and add \emph{a}:

\begin{Verbatim}[commandchars=\\\{\}]
\PYG{p}{(}\PYG{n}{b} \PYG{o}{\PYGZhy{}} \PYG{n}{a}\PYG{p}{)} \PYG{o}{*} \PYG{n}{random\PYGZus{}sample}\PYG{p}{(}\PYG{p}{)} \PYG{o}{+} \PYG{n}{a}
\end{Verbatim}
\begin{description}
\item[{size}] \leavevmode{[}int or tuple of ints, optional{]}
Defines the shape of the returned array of random floats. If None
(the default), returns a single float.

\end{description}
\begin{description}
\item[{out}] \leavevmode{[}float or ndarray of floats{]}
Array of random floats of shape \emph{size} (unless \code{size=None}, in which
case a single float is returned).

\end{description}

\begin{Verbatim}[commandchars=\\\{\}]
\PYG{g+gp}{\PYGZgt{}\PYGZgt{}\PYGZgt{} }\PYG{n}{np}\PYG{o}{.}\PYG{n}{random}\PYG{o}{.}\PYG{n}{random\PYGZus{}sample}\PYG{p}{(}\PYG{p}{)}
\PYG{g+go}{0.47108547995356098}
\PYG{g+gp}{\PYGZgt{}\PYGZgt{}\PYGZgt{} }\PYG{n+nb}{type}\PYG{p}{(}\PYG{n}{np}\PYG{o}{.}\PYG{n}{random}\PYG{o}{.}\PYG{n}{random\PYGZus{}sample}\PYG{p}{(}\PYG{p}{)}\PYG{p}{)}
\PYG{g+go}{\PYGZlt{}type \PYGZsq{}float\PYGZsq{}\PYGZgt{}}
\PYG{g+gp}{\PYGZgt{}\PYGZgt{}\PYGZgt{} }\PYG{n}{np}\PYG{o}{.}\PYG{n}{random}\PYG{o}{.}\PYG{n}{random\PYGZus{}sample}\PYG{p}{(}\PYG{p}{(}\PYG{l+m+mi}{5}\PYG{p}{,}\PYG{p}{)}\PYG{p}{)}
\PYG{g+go}{array([ 0.30220482,  0.86820401,  0.1654503 ,  0.11659149,  0.54323428])}
\end{Verbatim}

Three-by-two array of random numbers from {[}-5, 0):

\begin{Verbatim}[commandchars=\\\{\}]
\PYG{g+gp}{\PYGZgt{}\PYGZgt{}\PYGZgt{} }\PYG{l+m+mi}{5} \PYG{o}{*} \PYG{n}{np}\PYG{o}{.}\PYG{n}{random}\PYG{o}{.}\PYG{n}{random\PYGZus{}sample}\PYG{p}{(}\PYG{p}{(}\PYG{l+m+mi}{3}\PYG{p}{,} \PYG{l+m+mi}{2}\PYG{p}{)}\PYG{p}{)} \PYG{o}{\PYGZhy{}} \PYG{l+m+mi}{5}
\PYG{g+go}{array([[\PYGZhy{}3.99149989, \PYGZhy{}0.52338984],}
\PYG{g+go}{       [\PYGZhy{}2.99091858, \PYGZhy{}0.79479508],}
\PYG{g+go}{       [\PYGZhy{}1.23204345, \PYGZhy{}1.75224494]])}
\end{Verbatim}

\end{fulllineitems}

\index{ranf() (in module acsSpeciesActivities)}

\begin{fulllineitems}
\phantomsection\label{acsSpeciesActivities:acsSpeciesActivities.ranf}\pysiglinewithargsret{\code{acsSpeciesActivities.}\bfcode{ranf}}{}{}
random\_sample(size=None)

Return random floats in the half-open interval {[}0.0, 1.0).

Results are from the ``continuous uniform'' distribution over the
stated interval.  To sample \(Unif[a, b), b > a\) multiply
the output of \emph{random\_sample} by \emph{(b-a)} and add \emph{a}:

\begin{Verbatim}[commandchars=\\\{\}]
\PYG{p}{(}\PYG{n}{b} \PYG{o}{\PYGZhy{}} \PYG{n}{a}\PYG{p}{)} \PYG{o}{*} \PYG{n}{random\PYGZus{}sample}\PYG{p}{(}\PYG{p}{)} \PYG{o}{+} \PYG{n}{a}
\end{Verbatim}
\begin{description}
\item[{size}] \leavevmode{[}int or tuple of ints, optional{]}
Defines the shape of the returned array of random floats. If None
(the default), returns a single float.

\end{description}
\begin{description}
\item[{out}] \leavevmode{[}float or ndarray of floats{]}
Array of random floats of shape \emph{size} (unless \code{size=None}, in which
case a single float is returned).

\end{description}

\begin{Verbatim}[commandchars=\\\{\}]
\PYG{g+gp}{\PYGZgt{}\PYGZgt{}\PYGZgt{} }\PYG{n}{np}\PYG{o}{.}\PYG{n}{random}\PYG{o}{.}\PYG{n}{random\PYGZus{}sample}\PYG{p}{(}\PYG{p}{)}
\PYG{g+go}{0.47108547995356098}
\PYG{g+gp}{\PYGZgt{}\PYGZgt{}\PYGZgt{} }\PYG{n+nb}{type}\PYG{p}{(}\PYG{n}{np}\PYG{o}{.}\PYG{n}{random}\PYG{o}{.}\PYG{n}{random\PYGZus{}sample}\PYG{p}{(}\PYG{p}{)}\PYG{p}{)}
\PYG{g+go}{\PYGZlt{}type \PYGZsq{}float\PYGZsq{}\PYGZgt{}}
\PYG{g+gp}{\PYGZgt{}\PYGZgt{}\PYGZgt{} }\PYG{n}{np}\PYG{o}{.}\PYG{n}{random}\PYG{o}{.}\PYG{n}{random\PYGZus{}sample}\PYG{p}{(}\PYG{p}{(}\PYG{l+m+mi}{5}\PYG{p}{,}\PYG{p}{)}\PYG{p}{)}
\PYG{g+go}{array([ 0.30220482,  0.86820401,  0.1654503 ,  0.11659149,  0.54323428])}
\end{Verbatim}

Three-by-two array of random numbers from {[}-5, 0):

\begin{Verbatim}[commandchars=\\\{\}]
\PYG{g+gp}{\PYGZgt{}\PYGZgt{}\PYGZgt{} }\PYG{l+m+mi}{5} \PYG{o}{*} \PYG{n}{np}\PYG{o}{.}\PYG{n}{random}\PYG{o}{.}\PYG{n}{random\PYGZus{}sample}\PYG{p}{(}\PYG{p}{(}\PYG{l+m+mi}{3}\PYG{p}{,} \PYG{l+m+mi}{2}\PYG{p}{)}\PYG{p}{)} \PYG{o}{\PYGZhy{}} \PYG{l+m+mi}{5}
\PYG{g+go}{array([[\PYGZhy{}3.99149989, \PYGZhy{}0.52338984],}
\PYG{g+go}{       [\PYGZhy{}2.99091858, \PYGZhy{}0.79479508],}
\PYG{g+go}{       [\PYGZhy{}1.23204345, \PYGZhy{}1.75224494]])}
\end{Verbatim}

\end{fulllineitems}

\index{rayleigh() (in module acsSpeciesActivities)}

\begin{fulllineitems}
\phantomsection\label{acsSpeciesActivities:acsSpeciesActivities.rayleigh}\pysiglinewithargsret{\code{acsSpeciesActivities.}\bfcode{rayleigh}}{\emph{scale=1.0}, \emph{size=None}}{}
Draw samples from a Rayleigh distribution.

The \(\chi\) and Weibull distributions are generalizations of the
Rayleigh.
\begin{description}
\item[{scale}] \leavevmode{[}scalar{]}
Scale, also equals the mode. Should be \textgreater{}= 0.

\item[{size}] \leavevmode{[}int or tuple of ints, optional{]}
Shape of the output. Default is None, in which case a single
value is returned.

\end{description}

The probability density function for the Rayleigh distribution is
\begin{gather}
\begin{split}P(x;scale) = \frac{x}{scale^2}e^{\frac{-x^2}{2 \cdotp scale^2}}\end{split}\notag
\end{gather}
The Rayleigh distribution arises if the wind speed and wind direction are
both gaussian variables, then the vector wind velocity forms a Rayleigh
distribution. The Rayleigh distribution is used to model the expected
output from wind turbines.

Draw values from the distribution and plot the histogram

\begin{Verbatim}[commandchars=\\\{\}]
\PYG{g+gp}{\PYGZgt{}\PYGZgt{}\PYGZgt{} }\PYG{n}{values} \PYG{o}{=} \PYG{n}{hist}\PYG{p}{(}\PYG{n}{np}\PYG{o}{.}\PYG{n}{random}\PYG{o}{.}\PYG{n}{rayleigh}\PYG{p}{(}\PYG{l+m+mi}{3}\PYG{p}{,} \PYG{l+m+mi}{100000}\PYG{p}{)}\PYG{p}{,} \PYG{n}{bins}\PYG{o}{=}\PYG{l+m+mi}{200}\PYG{p}{,} \PYG{n}{normed}\PYG{o}{=}\PYG{n+nb+bp}{True}\PYG{p}{)}
\end{Verbatim}

Wave heights tend to follow a Rayleigh distribution. If the mean wave
height is 1 meter, what fraction of waves are likely to be larger than 3
meters?

\begin{Verbatim}[commandchars=\\\{\}]
\PYG{g+gp}{\PYGZgt{}\PYGZgt{}\PYGZgt{} }\PYG{n}{meanvalue} \PYG{o}{=} \PYG{l+m+mi}{1}
\PYG{g+gp}{\PYGZgt{}\PYGZgt{}\PYGZgt{} }\PYG{n}{modevalue} \PYG{o}{=} \PYG{n}{np}\PYG{o}{.}\PYG{n}{sqrt}\PYG{p}{(}\PYG{l+m+mi}{2} \PYG{o}{/} \PYG{n}{np}\PYG{o}{.}\PYG{n}{pi}\PYG{p}{)} \PYG{o}{*} \PYG{n}{meanvalue}
\PYG{g+gp}{\PYGZgt{}\PYGZgt{}\PYGZgt{} }\PYG{n}{s} \PYG{o}{=} \PYG{n}{np}\PYG{o}{.}\PYG{n}{random}\PYG{o}{.}\PYG{n}{rayleigh}\PYG{p}{(}\PYG{n}{modevalue}\PYG{p}{,} \PYG{l+m+mi}{1000000}\PYG{p}{)}
\end{Verbatim}

The percentage of waves larger than 3 meters is:

\begin{Verbatim}[commandchars=\\\{\}]
\PYG{g+gp}{\PYGZgt{}\PYGZgt{}\PYGZgt{} }\PYG{l+m+mf}{100.}\PYG{o}{*}\PYG{n+nb}{sum}\PYG{p}{(}\PYG{n}{s}\PYG{o}{\PYGZgt{}}\PYG{l+m+mi}{3}\PYG{p}{)}\PYG{o}{/}\PYG{l+m+mf}{1000000.}
\PYG{g+go}{0.087300000000000003}
\end{Verbatim}

\end{fulllineitems}

\index{sample() (in module acsSpeciesActivities)}

\begin{fulllineitems}
\phantomsection\label{acsSpeciesActivities:acsSpeciesActivities.sample}\pysiglinewithargsret{\code{acsSpeciesActivities.}\bfcode{sample}}{}{}
random\_sample(size=None)

Return random floats in the half-open interval {[}0.0, 1.0).

Results are from the ``continuous uniform'' distribution over the
stated interval.  To sample \(Unif[a, b), b > a\) multiply
the output of \emph{random\_sample} by \emph{(b-a)} and add \emph{a}:

\begin{Verbatim}[commandchars=\\\{\}]
\PYG{p}{(}\PYG{n}{b} \PYG{o}{\PYGZhy{}} \PYG{n}{a}\PYG{p}{)} \PYG{o}{*} \PYG{n}{random\PYGZus{}sample}\PYG{p}{(}\PYG{p}{)} \PYG{o}{+} \PYG{n}{a}
\end{Verbatim}
\begin{description}
\item[{size}] \leavevmode{[}int or tuple of ints, optional{]}
Defines the shape of the returned array of random floats. If None
(the default), returns a single float.

\end{description}
\begin{description}
\item[{out}] \leavevmode{[}float or ndarray of floats{]}
Array of random floats of shape \emph{size} (unless \code{size=None}, in which
case a single float is returned).

\end{description}

\begin{Verbatim}[commandchars=\\\{\}]
\PYG{g+gp}{\PYGZgt{}\PYGZgt{}\PYGZgt{} }\PYG{n}{np}\PYG{o}{.}\PYG{n}{random}\PYG{o}{.}\PYG{n}{random\PYGZus{}sample}\PYG{p}{(}\PYG{p}{)}
\PYG{g+go}{0.47108547995356098}
\PYG{g+gp}{\PYGZgt{}\PYGZgt{}\PYGZgt{} }\PYG{n+nb}{type}\PYG{p}{(}\PYG{n}{np}\PYG{o}{.}\PYG{n}{random}\PYG{o}{.}\PYG{n}{random\PYGZus{}sample}\PYG{p}{(}\PYG{p}{)}\PYG{p}{)}
\PYG{g+go}{\PYGZlt{}type \PYGZsq{}float\PYGZsq{}\PYGZgt{}}
\PYG{g+gp}{\PYGZgt{}\PYGZgt{}\PYGZgt{} }\PYG{n}{np}\PYG{o}{.}\PYG{n}{random}\PYG{o}{.}\PYG{n}{random\PYGZus{}sample}\PYG{p}{(}\PYG{p}{(}\PYG{l+m+mi}{5}\PYG{p}{,}\PYG{p}{)}\PYG{p}{)}
\PYG{g+go}{array([ 0.30220482,  0.86820401,  0.1654503 ,  0.11659149,  0.54323428])}
\end{Verbatim}

Three-by-two array of random numbers from {[}-5, 0):

\begin{Verbatim}[commandchars=\\\{\}]
\PYG{g+gp}{\PYGZgt{}\PYGZgt{}\PYGZgt{} }\PYG{l+m+mi}{5} \PYG{o}{*} \PYG{n}{np}\PYG{o}{.}\PYG{n}{random}\PYG{o}{.}\PYG{n}{random\PYGZus{}sample}\PYG{p}{(}\PYG{p}{(}\PYG{l+m+mi}{3}\PYG{p}{,} \PYG{l+m+mi}{2}\PYG{p}{)}\PYG{p}{)} \PYG{o}{\PYGZhy{}} \PYG{l+m+mi}{5}
\PYG{g+go}{array([[\PYGZhy{}3.99149989, \PYGZhy{}0.52338984],}
\PYG{g+go}{       [\PYGZhy{}2.99091858, \PYGZhy{}0.79479508],}
\PYG{g+go}{       [\PYGZhy{}1.23204345, \PYGZhy{}1.75224494]])}
\end{Verbatim}

\end{fulllineitems}

\index{seed() (in module acsSpeciesActivities)}

\begin{fulllineitems}
\phantomsection\label{acsSpeciesActivities:acsSpeciesActivities.seed}\pysiglinewithargsret{\code{acsSpeciesActivities.}\bfcode{seed}}{\emph{seed=None}}{}
Seed the generator.

This method is called when \emph{RandomState} is initialized. It can be
called again to re-seed the generator. For details, see \emph{RandomState}.
\begin{description}
\item[{seed}] \leavevmode{[}int or array\_like, optional{]}
Seed for \emph{RandomState}.

\end{description}

RandomState

\end{fulllineitems}

\index{set\_state() (in module acsSpeciesActivities)}

\begin{fulllineitems}
\phantomsection\label{acsSpeciesActivities:acsSpeciesActivities.set_state}\pysiglinewithargsret{\code{acsSpeciesActivities.}\bfcode{set\_state}}{\emph{state}}{}
Set the internal state of the generator from a tuple.

For use if one has reason to manually (re-)set the internal state of the
``Mersenne Twister''{\color{red}\bfseries{}{[}1{]}\_} pseudo-random number generating algorithm.
\begin{description}
\item[{state}] \leavevmode{[}tuple(str, ndarray of 624 uints, int, int, float){]}
The \emph{state} tuple has the following items:
\begin{enumerate}
\item {} 
the string `MT19937', specifying the Mersenne Twister algorithm.

\item {} 
a 1-D array of 624 unsigned integers \code{keys}.

\item {} 
an integer \code{pos}.

\item {} 
an integer \code{has\_gauss}.

\item {} 
a float \code{cached\_gaussian}.

\end{enumerate}

\end{description}
\begin{description}
\item[{out}] \leavevmode{[}None{]}
Returns `None' on success.

\end{description}

get\_state

\emph{set\_state} and \emph{get\_state} are not needed to work with any of the
random distributions in NumPy. If the internal state is manually altered,
the user should know exactly what he/she is doing.

For backwards compatibility, the form (str, array of 624 uints, int) is
also accepted although it is missing some information about the cached
Gaussian value: \code{state = ('MT19937', keys, pos)}.

\end{fulllineitems}

\index{shuffle() (in module acsSpeciesActivities)}

\begin{fulllineitems}
\phantomsection\label{acsSpeciesActivities:acsSpeciesActivities.shuffle}\pysiglinewithargsret{\code{acsSpeciesActivities.}\bfcode{shuffle}}{\emph{x}}{}
Modify a sequence in-place by shuffling its contents.
\begin{description}
\item[{x}] \leavevmode{[}array\_like{]}
The array or list to be shuffled.

\end{description}

None

\begin{Verbatim}[commandchars=\\\{\}]
\PYG{g+gp}{\PYGZgt{}\PYGZgt{}\PYGZgt{} }\PYG{n}{arr} \PYG{o}{=} \PYG{n}{np}\PYG{o}{.}\PYG{n}{arange}\PYG{p}{(}\PYG{l+m+mi}{10}\PYG{p}{)}
\PYG{g+gp}{\PYGZgt{}\PYGZgt{}\PYGZgt{} }\PYG{n}{np}\PYG{o}{.}\PYG{n}{random}\PYG{o}{.}\PYG{n}{shuffle}\PYG{p}{(}\PYG{n}{arr}\PYG{p}{)}
\PYG{g+gp}{\PYGZgt{}\PYGZgt{}\PYGZgt{} }\PYG{n}{arr}
\PYG{g+go}{[1 7 5 2 9 4 3 6 0 8]}
\end{Verbatim}

This function only shuffles the array along the first index of a
multi-dimensional array:

\begin{Verbatim}[commandchars=\\\{\}]
\PYG{g+gp}{\PYGZgt{}\PYGZgt{}\PYGZgt{} }\PYG{n}{arr} \PYG{o}{=} \PYG{n}{np}\PYG{o}{.}\PYG{n}{arange}\PYG{p}{(}\PYG{l+m+mi}{9}\PYG{p}{)}\PYG{o}{.}\PYG{n}{reshape}\PYG{p}{(}\PYG{p}{(}\PYG{l+m+mi}{3}\PYG{p}{,} \PYG{l+m+mi}{3}\PYG{p}{)}\PYG{p}{)}
\PYG{g+gp}{\PYGZgt{}\PYGZgt{}\PYGZgt{} }\PYG{n}{np}\PYG{o}{.}\PYG{n}{random}\PYG{o}{.}\PYG{n}{shuffle}\PYG{p}{(}\PYG{n}{arr}\PYG{p}{)}
\PYG{g+gp}{\PYGZgt{}\PYGZgt{}\PYGZgt{} }\PYG{n}{arr}
\PYG{g+go}{array([[3, 4, 5],}
\PYG{g+go}{       [6, 7, 8],}
\PYG{g+go}{       [0, 1, 2]])}
\end{Verbatim}

\end{fulllineitems}

\index{standard\_cauchy() (in module acsSpeciesActivities)}

\begin{fulllineitems}
\phantomsection\label{acsSpeciesActivities:acsSpeciesActivities.standard_cauchy}\pysiglinewithargsret{\code{acsSpeciesActivities.}\bfcode{standard\_cauchy}}{\emph{size=None}}{}
Standard Cauchy distribution with mode = 0.

Also known as the Lorentz distribution.
\begin{description}
\item[{size}] \leavevmode{[}int or tuple of ints{]}
Shape of the output.

\end{description}
\begin{description}
\item[{samples}] \leavevmode{[}ndarray or scalar{]}
The drawn samples.

\end{description}

The probability density function for the full Cauchy distribution is
\begin{gather}
\begin{split}P(x; x_0, \gamma) = \frac{1}{\pi \gamma \bigl[ 1+
(\frac{x-x_0}{\gamma})^2 \bigr] }\end{split}\notag
\end{gather}
and the Standard Cauchy distribution just sets \(x_0=0\) and
\(\gamma=1\)

The Cauchy distribution arises in the solution to the driven harmonic
oscillator problem, and also describes spectral line broadening. It
also describes the distribution of values at which a line tilted at
a random angle will cut the x axis.

When studying hypothesis tests that assume normality, seeing how the
tests perform on data from a Cauchy distribution is a good indicator of
their sensitivity to a heavy-tailed distribution, since the Cauchy looks
very much like a Gaussian distribution, but with heavier tails.

Draw samples and plot the distribution:

\begin{Verbatim}[commandchars=\\\{\}]
\PYG{g+gp}{\PYGZgt{}\PYGZgt{}\PYGZgt{} }\PYG{n}{s} \PYG{o}{=} \PYG{n}{np}\PYG{o}{.}\PYG{n}{random}\PYG{o}{.}\PYG{n}{standard\PYGZus{}cauchy}\PYG{p}{(}\PYG{l+m+mi}{1000000}\PYG{p}{)}
\PYG{g+gp}{\PYGZgt{}\PYGZgt{}\PYGZgt{} }\PYG{n}{s} \PYG{o}{=} \PYG{n}{s}\PYG{p}{[}\PYG{p}{(}\PYG{n}{s}\PYG{o}{\PYGZgt{}}\PYG{o}{\PYGZhy{}}\PYG{l+m+mi}{25}\PYG{p}{)} \PYG{o}{\PYGZam{}} \PYG{p}{(}\PYG{n}{s}\PYG{o}{\PYGZlt{}}\PYG{l+m+mi}{25}\PYG{p}{)}\PYG{p}{]}  \PYG{c}{\PYGZsh{} truncate distribution so it plots well}
\PYG{g+gp}{\PYGZgt{}\PYGZgt{}\PYGZgt{} }\PYG{n}{plt}\PYG{o}{.}\PYG{n}{hist}\PYG{p}{(}\PYG{n}{s}\PYG{p}{,} \PYG{n}{bins}\PYG{o}{=}\PYG{l+m+mi}{100}\PYG{p}{)}
\PYG{g+gp}{\PYGZgt{}\PYGZgt{}\PYGZgt{} }\PYG{n}{plt}\PYG{o}{.}\PYG{n}{show}\PYG{p}{(}\PYG{p}{)}
\end{Verbatim}

\end{fulllineitems}

\index{standard\_exponential() (in module acsSpeciesActivities)}

\begin{fulllineitems}
\phantomsection\label{acsSpeciesActivities:acsSpeciesActivities.standard_exponential}\pysiglinewithargsret{\code{acsSpeciesActivities.}\bfcode{standard\_exponential}}{\emph{size=None}}{}
Draw samples from the standard exponential distribution.

\emph{standard\_exponential} is identical to the exponential distribution
with a scale parameter of 1.
\begin{description}
\item[{size}] \leavevmode{[}int or tuple of ints{]}
Shape of the output.

\end{description}
\begin{description}
\item[{out}] \leavevmode{[}float or ndarray{]}
Drawn samples.

\end{description}

Output a 3x8000 array:

\begin{Verbatim}[commandchars=\\\{\}]
\PYG{g+gp}{\PYGZgt{}\PYGZgt{}\PYGZgt{} }\PYG{n}{n} \PYG{o}{=} \PYG{n}{np}\PYG{o}{.}\PYG{n}{random}\PYG{o}{.}\PYG{n}{standard\PYGZus{}exponential}\PYG{p}{(}\PYG{p}{(}\PYG{l+m+mi}{3}\PYG{p}{,} \PYG{l+m+mi}{8000}\PYG{p}{)}\PYG{p}{)}
\end{Verbatim}

\end{fulllineitems}

\index{standard\_gamma() (in module acsSpeciesActivities)}

\begin{fulllineitems}
\phantomsection\label{acsSpeciesActivities:acsSpeciesActivities.standard_gamma}\pysiglinewithargsret{\code{acsSpeciesActivities.}\bfcode{standard\_gamma}}{\emph{shape}, \emph{size=None}}{}
Draw samples from a Standard Gamma distribution.

Samples are drawn from a Gamma distribution with specified parameters,
shape (sometimes designated ``k'') and scale=1.
\begin{description}
\item[{shape}] \leavevmode{[}float{]}
Parameter, should be \textgreater{} 0.

\item[{size}] \leavevmode{[}int or tuple of ints{]}
Output shape.  If the given shape is, e.g., \code{(m, n, k)}, then
\code{m * n * k} samples are drawn.

\end{description}
\begin{description}
\item[{samples}] \leavevmode{[}ndarray or scalar{]}
The drawn samples.

\end{description}
\begin{description}
\item[{scipy.stats.distributions.gamma}] \leavevmode{[}probability density function,{]}
distribution or cumulative density function, etc.

\end{description}

The probability density for the Gamma distribution is
\begin{gather}
\begin{split}p(x) = x^{k-1}\frac{e^{-x/\theta}}{\theta^k\Gamma(k)},\end{split}\notag
\end{gather}
where \(k\) is the shape and \(\theta\) the scale,
and \(\Gamma\) is the Gamma function.

The Gamma distribution is often used to model the times to failure of
electronic components, and arises naturally in processes for which the
waiting times between Poisson distributed events are relevant.

Draw samples from the distribution:

\begin{Verbatim}[commandchars=\\\{\}]
\PYG{g+gp}{\PYGZgt{}\PYGZgt{}\PYGZgt{} }\PYG{n}{shape}\PYG{p}{,} \PYG{n}{scale} \PYG{o}{=} \PYG{l+m+mf}{2.}\PYG{p}{,} \PYG{l+m+mf}{1.} \PYG{c}{\PYGZsh{} mean and width}
\PYG{g+gp}{\PYGZgt{}\PYGZgt{}\PYGZgt{} }\PYG{n}{s} \PYG{o}{=} \PYG{n}{np}\PYG{o}{.}\PYG{n}{random}\PYG{o}{.}\PYG{n}{standard\PYGZus{}gamma}\PYG{p}{(}\PYG{n}{shape}\PYG{p}{,} \PYG{l+m+mi}{1000000}\PYG{p}{)}
\end{Verbatim}

Display the histogram of the samples, along with
the probability density function:

\begin{Verbatim}[commandchars=\\\{\}]
\PYG{g+gp}{\PYGZgt{}\PYGZgt{}\PYGZgt{} }\PYG{k+kn}{import} \PYG{n+nn}{matplotlib.pyplot} \PYG{k+kn}{as} \PYG{n+nn}{plt}
\PYG{g+gp}{\PYGZgt{}\PYGZgt{}\PYGZgt{} }\PYG{k+kn}{import} \PYG{n+nn}{scipy.special} \PYG{k+kn}{as} \PYG{n+nn}{sps}
\PYG{g+gp}{\PYGZgt{}\PYGZgt{}\PYGZgt{} }\PYG{n}{count}\PYG{p}{,} \PYG{n}{bins}\PYG{p}{,} \PYG{n}{ignored} \PYG{o}{=} \PYG{n}{plt}\PYG{o}{.}\PYG{n}{hist}\PYG{p}{(}\PYG{n}{s}\PYG{p}{,} \PYG{l+m+mi}{50}\PYG{p}{,} \PYG{n}{normed}\PYG{o}{=}\PYG{n+nb+bp}{True}\PYG{p}{)}
\PYG{g+gp}{\PYGZgt{}\PYGZgt{}\PYGZgt{} }\PYG{n}{y} \PYG{o}{=} \PYG{n}{bins}\PYG{o}{*}\PYG{o}{*}\PYG{p}{(}\PYG{n}{shape}\PYG{o}{\PYGZhy{}}\PYG{l+m+mi}{1}\PYG{p}{)} \PYG{o}{*} \PYG{p}{(}\PYG{p}{(}\PYG{n}{np}\PYG{o}{.}\PYG{n}{exp}\PYG{p}{(}\PYG{o}{\PYGZhy{}}\PYG{n}{bins}\PYG{o}{/}\PYG{n}{scale}\PYG{p}{)}\PYG{p}{)}\PYG{o}{/} \PYGZbs{}
\PYG{g+gp}{... }                      \PYG{p}{(}\PYG{n}{sps}\PYG{o}{.}\PYG{n}{gamma}\PYG{p}{(}\PYG{n}{shape}\PYG{p}{)} \PYG{o}{*} \PYG{n}{scale}\PYG{o}{*}\PYG{o}{*}\PYG{n}{shape}\PYG{p}{)}\PYG{p}{)}
\PYG{g+gp}{\PYGZgt{}\PYGZgt{}\PYGZgt{} }\PYG{n}{plt}\PYG{o}{.}\PYG{n}{plot}\PYG{p}{(}\PYG{n}{bins}\PYG{p}{,} \PYG{n}{y}\PYG{p}{,} \PYG{n}{linewidth}\PYG{o}{=}\PYG{l+m+mi}{2}\PYG{p}{,} \PYG{n}{color}\PYG{o}{=}\PYG{l+s}{\PYGZsq{}}\PYG{l+s}{r}\PYG{l+s}{\PYGZsq{}}\PYG{p}{)}
\PYG{g+gp}{\PYGZgt{}\PYGZgt{}\PYGZgt{} }\PYG{n}{plt}\PYG{o}{.}\PYG{n}{show}\PYG{p}{(}\PYG{p}{)}
\end{Verbatim}

\end{fulllineitems}

\index{standard\_normal() (in module acsSpeciesActivities)}

\begin{fulllineitems}
\phantomsection\label{acsSpeciesActivities:acsSpeciesActivities.standard_normal}\pysiglinewithargsret{\code{acsSpeciesActivities.}\bfcode{standard\_normal}}{\emph{size=None}}{}
Returns samples from a Standard Normal distribution (mean=0, stdev=1).
\begin{description}
\item[{size}] \leavevmode{[}int or tuple of ints, optional{]}
Output shape. Default is None, in which case a single value is
returned.

\end{description}
\begin{description}
\item[{out}] \leavevmode{[}float or ndarray{]}
Drawn samples.

\end{description}

\begin{Verbatim}[commandchars=\\\{\}]
\PYG{g+gp}{\PYGZgt{}\PYGZgt{}\PYGZgt{} }\PYG{n}{s} \PYG{o}{=} \PYG{n}{np}\PYG{o}{.}\PYG{n}{random}\PYG{o}{.}\PYG{n}{standard\PYGZus{}normal}\PYG{p}{(}\PYG{l+m+mi}{8000}\PYG{p}{)}
\PYG{g+gp}{\PYGZgt{}\PYGZgt{}\PYGZgt{} }\PYG{n}{s}
\PYG{g+go}{array([ 0.6888893 ,  0.78096262, \PYGZhy{}0.89086505, ...,  0.49876311, \PYGZsh{}random}
\PYG{g+go}{       \PYGZhy{}0.38672696, \PYGZhy{}0.4685006 ])                               \PYGZsh{}random}
\PYG{g+gp}{\PYGZgt{}\PYGZgt{}\PYGZgt{} }\PYG{n}{s}\PYG{o}{.}\PYG{n}{shape}
\PYG{g+go}{(8000,)}
\PYG{g+gp}{\PYGZgt{}\PYGZgt{}\PYGZgt{} }\PYG{n}{s} \PYG{o}{=} \PYG{n}{np}\PYG{o}{.}\PYG{n}{random}\PYG{o}{.}\PYG{n}{standard\PYGZus{}normal}\PYG{p}{(}\PYG{n}{size}\PYG{o}{=}\PYG{p}{(}\PYG{l+m+mi}{3}\PYG{p}{,} \PYG{l+m+mi}{4}\PYG{p}{,} \PYG{l+m+mi}{2}\PYG{p}{)}\PYG{p}{)}
\PYG{g+gp}{\PYGZgt{}\PYGZgt{}\PYGZgt{} }\PYG{n}{s}\PYG{o}{.}\PYG{n}{shape}
\PYG{g+go}{(3, 4, 2)}
\end{Verbatim}

\end{fulllineitems}

\index{standard\_t() (in module acsSpeciesActivities)}

\begin{fulllineitems}
\phantomsection\label{acsSpeciesActivities:acsSpeciesActivities.standard_t}\pysiglinewithargsret{\code{acsSpeciesActivities.}\bfcode{standard\_t}}{\emph{df}, \emph{size=None}}{}
Standard Student's t distribution with df degrees of freedom.

A special case of the hyperbolic distribution.
As \emph{df} gets large, the result resembles that of the standard normal
distribution (\emph{standard\_normal}).
\begin{description}
\item[{df}] \leavevmode{[}int{]}
Degrees of freedom, should be \textgreater{} 0.

\item[{size}] \leavevmode{[}int or tuple of ints, optional{]}
Output shape. Default is None, in which case a single value is
returned.

\end{description}
\begin{description}
\item[{samples}] \leavevmode{[}ndarray or scalar{]}
Drawn samples.

\end{description}

The probability density function for the t distribution is
\begin{gather}
\begin{split}P(x, df) = \frac{\Gamma(\frac{df+1}{2})}{\sqrt{\pi df}
\Gamma(\frac{df}{2})}\Bigl( 1+\frac{x^2}{df} \Bigr)^{-(df+1)/2}\end{split}\notag
\end{gather}
The t test is based on an assumption that the data come from a Normal
distribution. The t test provides a way to test whether the sample mean
(that is the mean calculated from the data) is a good estimate of the true
mean.

The derivation of the t-distribution was forst published in 1908 by William
Gisset while working for the Guinness Brewery in Dublin. Due to proprietary
issues, he had to publish under a pseudonym, and so he used the name
Student.

From Dalgaard page 83 {\color{red}\bfseries{}{[}1{]}\_}, suppose the daily energy intake for 11
women in Kj is:

\begin{Verbatim}[commandchars=\\\{\}]
\PYG{g+gp}{\PYGZgt{}\PYGZgt{}\PYGZgt{} }\PYG{n}{intake} \PYG{o}{=} \PYG{n}{np}\PYG{o}{.}\PYG{n}{array}\PYG{p}{(}\PYG{p}{[}\PYG{l+m+mf}{5260.}\PYG{p}{,} \PYG{l+m+mi}{5470}\PYG{p}{,} \PYG{l+m+mi}{5640}\PYG{p}{,} \PYG{l+m+mi}{6180}\PYG{p}{,} \PYG{l+m+mi}{6390}\PYG{p}{,} \PYG{l+m+mi}{6515}\PYG{p}{,} \PYG{l+m+mi}{6805}\PYG{p}{,} \PYG{l+m+mi}{7515}\PYG{p}{,} \PYGZbs{}
\PYG{g+gp}{... }                   \PYG{l+m+mi}{7515}\PYG{p}{,} \PYG{l+m+mi}{8230}\PYG{p}{,} \PYG{l+m+mi}{8770}\PYG{p}{]}\PYG{p}{)}
\end{Verbatim}

Does their energy intake deviate systematically from the recommended
value of 7725 kJ?

We have 10 degrees of freedom, so is the sample mean within 95\% of the
recommended value?

\begin{Verbatim}[commandchars=\\\{\}]
\PYG{g+gp}{\PYGZgt{}\PYGZgt{}\PYGZgt{} }\PYG{n}{s} \PYG{o}{=} \PYG{n}{np}\PYG{o}{.}\PYG{n}{random}\PYG{o}{.}\PYG{n}{standard\PYGZus{}t}\PYG{p}{(}\PYG{l+m+mi}{10}\PYG{p}{,} \PYG{n}{size}\PYG{o}{=}\PYG{l+m+mi}{100000}\PYG{p}{)}
\PYG{g+gp}{\PYGZgt{}\PYGZgt{}\PYGZgt{} }\PYG{n}{np}\PYG{o}{.}\PYG{n}{mean}\PYG{p}{(}\PYG{n}{intake}\PYG{p}{)}
\PYG{g+go}{6753.636363636364}
\PYG{g+gp}{\PYGZgt{}\PYGZgt{}\PYGZgt{} }\PYG{n}{intake}\PYG{o}{.}\PYG{n}{std}\PYG{p}{(}\PYG{n}{ddof}\PYG{o}{=}\PYG{l+m+mi}{1}\PYG{p}{)}
\PYG{g+go}{1142.1232221373727}
\end{Verbatim}

Calculate the t statistic, setting the ddof parameter to the unbiased
value so the divisor in the standard deviation will be degrees of
freedom, N-1.

\begin{Verbatim}[commandchars=\\\{\}]
\PYG{g+gp}{\PYGZgt{}\PYGZgt{}\PYGZgt{} }\PYG{n}{t} \PYG{o}{=} \PYG{p}{(}\PYG{n}{np}\PYG{o}{.}\PYG{n}{mean}\PYG{p}{(}\PYG{n}{intake}\PYG{p}{)}\PYG{o}{\PYGZhy{}}\PYG{l+m+mi}{7725}\PYG{p}{)}\PYG{o}{/}\PYG{p}{(}\PYG{n}{intake}\PYG{o}{.}\PYG{n}{std}\PYG{p}{(}\PYG{n}{ddof}\PYG{o}{=}\PYG{l+m+mi}{1}\PYG{p}{)}\PYG{o}{/}\PYG{n}{np}\PYG{o}{.}\PYG{n}{sqrt}\PYG{p}{(}\PYG{n+nb}{len}\PYG{p}{(}\PYG{n}{intake}\PYG{p}{)}\PYG{p}{)}\PYG{p}{)}
\PYG{g+gp}{\PYGZgt{}\PYGZgt{}\PYGZgt{} }\PYG{k+kn}{import} \PYG{n+nn}{matplotlib.pyplot} \PYG{k+kn}{as} \PYG{n+nn}{plt}
\PYG{g+gp}{\PYGZgt{}\PYGZgt{}\PYGZgt{} }\PYG{n}{h} \PYG{o}{=} \PYG{n}{plt}\PYG{o}{.}\PYG{n}{hist}\PYG{p}{(}\PYG{n}{s}\PYG{p}{,} \PYG{n}{bins}\PYG{o}{=}\PYG{l+m+mi}{100}\PYG{p}{,} \PYG{n}{normed}\PYG{o}{=}\PYG{n+nb+bp}{True}\PYG{p}{)}
\end{Verbatim}

For a one-sided t-test, how far out in the distribution does the t
statistic appear?

\begin{Verbatim}[commandchars=\\\{\}]
\PYG{g+gp}{\PYGZgt{}\PYGZgt{}\PYGZgt{} }\PYG{o}{\PYGZgt{}\PYGZgt{}}\PYG{o}{\PYGZgt{}} \PYG{n}{np}\PYG{o}{.}\PYG{n}{sum}\PYG{p}{(}\PYG{n}{s}\PYG{o}{\PYGZlt{}}\PYG{n}{t}\PYG{p}{)} \PYG{o}{/} \PYG{n+nb}{float}\PYG{p}{(}\PYG{n+nb}{len}\PYG{p}{(}\PYG{n}{s}\PYG{p}{)}\PYG{p}{)}
\PYG{g+go}{0.0090699999999999999  \PYGZsh{}random}
\end{Verbatim}

So the p-value is about 0.009, which says the null hypothesis has a
probability of about 99\% of being true.

\end{fulllineitems}

\index{triangular() (in module acsSpeciesActivities)}

\begin{fulllineitems}
\phantomsection\label{acsSpeciesActivities:acsSpeciesActivities.triangular}\pysiglinewithargsret{\code{acsSpeciesActivities.}\bfcode{triangular}}{\emph{left}, \emph{mode}, \emph{right}, \emph{size=None}}{}
Draw samples from the triangular distribution.

The triangular distribution is a continuous probability distribution with
lower limit left, peak at mode, and upper limit right. Unlike the other
distributions, these parameters directly define the shape of the pdf.
\begin{description}
\item[{left}] \leavevmode{[}scalar{]}
Lower limit.

\item[{mode}] \leavevmode{[}scalar{]}
The value where the peak of the distribution occurs.
The value should fulfill the condition \code{left \textless{}= mode \textless{}= right}.

\item[{right}] \leavevmode{[}scalar{]}
Upper limit, should be larger than \emph{left}.

\item[{size}] \leavevmode{[}int or tuple of ints, optional{]}
Output shape. Default is None, in which case a single value is
returned.

\end{description}
\begin{description}
\item[{samples}] \leavevmode{[}ndarray or scalar{]}
The returned samples all lie in the interval {[}left, right{]}.

\end{description}

The probability density function for the Triangular distribution is
\begin{gather}
\begin{split}P(x;l, m, r) = \begin{cases}
\frac{2(x-l)}{(r-l)(m-l)}& \text{for $l \leq x \leq m$},\\
\frac{2(m-x)}{(r-l)(r-m)}& \text{for $m \leq x \leq r$},\\
0& \text{otherwise}.
\end{cases}\end{split}\notag
\end{gather}
The triangular distribution is often used in ill-defined problems where the
underlying distribution is not known, but some knowledge of the limits and
mode exists. Often it is used in simulations.

Draw values from the distribution and plot the histogram:

\begin{Verbatim}[commandchars=\\\{\}]
\PYG{g+gp}{\PYGZgt{}\PYGZgt{}\PYGZgt{} }\PYG{k+kn}{import} \PYG{n+nn}{matplotlib.pyplot} \PYG{k+kn}{as} \PYG{n+nn}{plt}
\PYG{g+gp}{\PYGZgt{}\PYGZgt{}\PYGZgt{} }\PYG{n}{h} \PYG{o}{=} \PYG{n}{plt}\PYG{o}{.}\PYG{n}{hist}\PYG{p}{(}\PYG{n}{np}\PYG{o}{.}\PYG{n}{random}\PYG{o}{.}\PYG{n}{triangular}\PYG{p}{(}\PYG{o}{\PYGZhy{}}\PYG{l+m+mi}{3}\PYG{p}{,} \PYG{l+m+mi}{0}\PYG{p}{,} \PYG{l+m+mi}{8}\PYG{p}{,} \PYG{l+m+mi}{100000}\PYG{p}{)}\PYG{p}{,} \PYG{n}{bins}\PYG{o}{=}\PYG{l+m+mi}{200}\PYG{p}{,}
\PYG{g+gp}{... }             \PYG{n}{normed}\PYG{o}{=}\PYG{n+nb+bp}{True}\PYG{p}{)}
\PYG{g+gp}{\PYGZgt{}\PYGZgt{}\PYGZgt{} }\PYG{n}{plt}\PYG{o}{.}\PYG{n}{show}\PYG{p}{(}\PYG{p}{)}
\end{Verbatim}

\end{fulllineitems}

\index{uniform() (in module acsSpeciesActivities)}

\begin{fulllineitems}
\phantomsection\label{acsSpeciesActivities:acsSpeciesActivities.uniform}\pysiglinewithargsret{\code{acsSpeciesActivities.}\bfcode{uniform}}{\emph{low=0.0}, \emph{high=1.0}, \emph{size=1}}{}
Draw samples from a uniform distribution.

Samples are uniformly distributed over the half-open interval
\code{{[}low, high)} (includes low, but excludes high).  In other words,
any value within the given interval is equally likely to be drawn
by \emph{uniform}.
\begin{description}
\item[{low}] \leavevmode{[}float, optional{]}
Lower boundary of the output interval.  All values generated will be
greater than or equal to low.  The default value is 0.

\item[{high}] \leavevmode{[}float{]}
Upper boundary of the output interval.  All values generated will be
less than high.  The default value is 1.0.

\item[{size}] \leavevmode{[}int or tuple of ints, optional{]}
Shape of output.  If the given size is, for example, (m,n,k),
m*n*k samples are generated.  If no shape is specified, a single sample
is returned.

\end{description}
\begin{description}
\item[{out}] \leavevmode{[}ndarray{]}
Drawn samples, with shape \emph{size}.

\end{description}

randint : Discrete uniform distribution, yielding integers.
random\_integers : Discrete uniform distribution over the closed
\begin{quote}

interval \code{{[}low, high{]}}.
\end{quote}

random\_sample : Floats uniformly distributed over \code{{[}0, 1)}.
random : Alias for \emph{random\_sample}.
rand : Convenience function that accepts dimensions as input, e.g.,
\begin{quote}

\code{rand(2,2)} would generate a 2-by-2 array of floats,
uniformly distributed over \code{{[}0, 1)}.
\end{quote}

The probability density function of the uniform distribution is
\begin{gather}
\begin{split}p(x) = \frac{1}{b - a}\end{split}\notag
\end{gather}
anywhere within the interval \code{{[}a, b)}, and zero elsewhere.

Draw samples from the distribution:

\begin{Verbatim}[commandchars=\\\{\}]
\PYG{g+gp}{\PYGZgt{}\PYGZgt{}\PYGZgt{} }\PYG{n}{s} \PYG{o}{=} \PYG{n}{np}\PYG{o}{.}\PYG{n}{random}\PYG{o}{.}\PYG{n}{uniform}\PYG{p}{(}\PYG{o}{\PYGZhy{}}\PYG{l+m+mi}{1}\PYG{p}{,}\PYG{l+m+mi}{0}\PYG{p}{,}\PYG{l+m+mi}{1000}\PYG{p}{)}
\end{Verbatim}

All values are within the given interval:

\begin{Verbatim}[commandchars=\\\{\}]
\PYG{g+gp}{\PYGZgt{}\PYGZgt{}\PYGZgt{} }\PYG{n}{np}\PYG{o}{.}\PYG{n}{all}\PYG{p}{(}\PYG{n}{s} \PYG{o}{\PYGZgt{}}\PYG{o}{=} \PYG{o}{\PYGZhy{}}\PYG{l+m+mi}{1}\PYG{p}{)}
\PYG{g+go}{True}
\PYG{g+gp}{\PYGZgt{}\PYGZgt{}\PYGZgt{} }\PYG{n}{np}\PYG{o}{.}\PYG{n}{all}\PYG{p}{(}\PYG{n}{s} \PYG{o}{\PYGZlt{}} \PYG{l+m+mi}{0}\PYG{p}{)}
\PYG{g+go}{True}
\end{Verbatim}

Display the histogram of the samples, along with the
probability density function:

\begin{Verbatim}[commandchars=\\\{\}]
\PYG{g+gp}{\PYGZgt{}\PYGZgt{}\PYGZgt{} }\PYG{k+kn}{import} \PYG{n+nn}{matplotlib.pyplot} \PYG{k+kn}{as} \PYG{n+nn}{plt}
\PYG{g+gp}{\PYGZgt{}\PYGZgt{}\PYGZgt{} }\PYG{n}{count}\PYG{p}{,} \PYG{n}{bins}\PYG{p}{,} \PYG{n}{ignored} \PYG{o}{=} \PYG{n}{plt}\PYG{o}{.}\PYG{n}{hist}\PYG{p}{(}\PYG{n}{s}\PYG{p}{,} \PYG{l+m+mi}{15}\PYG{p}{,} \PYG{n}{normed}\PYG{o}{=}\PYG{n+nb+bp}{True}\PYG{p}{)}
\PYG{g+gp}{\PYGZgt{}\PYGZgt{}\PYGZgt{} }\PYG{n}{plt}\PYG{o}{.}\PYG{n}{plot}\PYG{p}{(}\PYG{n}{bins}\PYG{p}{,} \PYG{n}{np}\PYG{o}{.}\PYG{n}{ones\PYGZus{}like}\PYG{p}{(}\PYG{n}{bins}\PYG{p}{)}\PYG{p}{,} \PYG{n}{linewidth}\PYG{o}{=}\PYG{l+m+mi}{2}\PYG{p}{,} \PYG{n}{color}\PYG{o}{=}\PYG{l+s}{\PYGZsq{}}\PYG{l+s}{r}\PYG{l+s}{\PYGZsq{}}\PYG{p}{)}
\PYG{g+gp}{\PYGZgt{}\PYGZgt{}\PYGZgt{} }\PYG{n}{plt}\PYG{o}{.}\PYG{n}{show}\PYG{p}{(}\PYG{p}{)}
\end{Verbatim}

\end{fulllineitems}

\index{vonmises() (in module acsSpeciesActivities)}

\begin{fulllineitems}
\phantomsection\label{acsSpeciesActivities:acsSpeciesActivities.vonmises}\pysiglinewithargsret{\code{acsSpeciesActivities.}\bfcode{vonmises}}{\emph{mu}, \emph{kappa}, \emph{size=None}}{}
Draw samples from a von Mises distribution.

Samples are drawn from a von Mises distribution with specified mode
(mu) and dispersion (kappa), on the interval {[}-pi, pi{]}.

The von Mises distribution (also known as the circular normal
distribution) is a continuous probability distribution on the unit
circle.  It may be thought of as the circular analogue of the normal
distribution.
\begin{description}
\item[{mu}] \leavevmode{[}float{]}
Mode (``center'') of the distribution.

\item[{kappa}] \leavevmode{[}float{]}
Dispersion of the distribution, has to be \textgreater{}=0.

\item[{size}] \leavevmode{[}int or tuple of int{]}
Output shape.  If the given shape is, e.g., \code{(m, n, k)}, then
\code{m * n * k} samples are drawn.

\end{description}
\begin{description}
\item[{samples}] \leavevmode{[}scalar or ndarray{]}
The returned samples, which are in the interval {[}-pi, pi{]}.

\end{description}
\begin{description}
\item[{scipy.stats.distributions.vonmises}] \leavevmode{[}probability density function,{]}
distribution, or cumulative density function, etc.

\end{description}

The probability density for the von Mises distribution is
\begin{gather}
\begin{split}p(x) = \frac{e^{\kappa cos(x-\mu)}}{2\pi I_0(\kappa)},\end{split}\notag
\end{gather}
where \(\mu\) is the mode and \(\kappa\) the dispersion,
and \(I_0(\kappa)\) is the modified Bessel function of order 0.

The von Mises is named for Richard Edler von Mises, who was born in
Austria-Hungary, in what is now the Ukraine.  He fled to the United
States in 1939 and became a professor at Harvard.  He worked in
probability theory, aerodynamics, fluid mechanics, and philosophy of
science.

Abramowitz, M. and Stegun, I. A. (ed.), \emph{Handbook of Mathematical
Functions}, New York: Dover, 1965.

von Mises, R., \emph{Mathematical Theory of Probability and Statistics},
New York: Academic Press, 1964.

Draw samples from the distribution:

\begin{Verbatim}[commandchars=\\\{\}]
\PYG{g+gp}{\PYGZgt{}\PYGZgt{}\PYGZgt{} }\PYG{n}{mu}\PYG{p}{,} \PYG{n}{kappa} \PYG{o}{=} \PYG{l+m+mf}{0.0}\PYG{p}{,} \PYG{l+m+mf}{4.0} \PYG{c}{\PYGZsh{} mean and dispersion}
\PYG{g+gp}{\PYGZgt{}\PYGZgt{}\PYGZgt{} }\PYG{n}{s} \PYG{o}{=} \PYG{n}{np}\PYG{o}{.}\PYG{n}{random}\PYG{o}{.}\PYG{n}{vonmises}\PYG{p}{(}\PYG{n}{mu}\PYG{p}{,} \PYG{n}{kappa}\PYG{p}{,} \PYG{l+m+mi}{1000}\PYG{p}{)}
\end{Verbatim}

Display the histogram of the samples, along with
the probability density function:

\begin{Verbatim}[commandchars=\\\{\}]
\PYG{g+gp}{\PYGZgt{}\PYGZgt{}\PYGZgt{} }\PYG{k+kn}{import} \PYG{n+nn}{matplotlib.pyplot} \PYG{k+kn}{as} \PYG{n+nn}{plt}
\PYG{g+gp}{\PYGZgt{}\PYGZgt{}\PYGZgt{} }\PYG{k+kn}{import} \PYG{n+nn}{scipy.special} \PYG{k+kn}{as} \PYG{n+nn}{sps}
\PYG{g+gp}{\PYGZgt{}\PYGZgt{}\PYGZgt{} }\PYG{n}{count}\PYG{p}{,} \PYG{n}{bins}\PYG{p}{,} \PYG{n}{ignored} \PYG{o}{=} \PYG{n}{plt}\PYG{o}{.}\PYG{n}{hist}\PYG{p}{(}\PYG{n}{s}\PYG{p}{,} \PYG{l+m+mi}{50}\PYG{p}{,} \PYG{n}{normed}\PYG{o}{=}\PYG{n+nb+bp}{True}\PYG{p}{)}
\PYG{g+gp}{\PYGZgt{}\PYGZgt{}\PYGZgt{} }\PYG{n}{x} \PYG{o}{=} \PYG{n}{np}\PYG{o}{.}\PYG{n}{arange}\PYG{p}{(}\PYG{o}{\PYGZhy{}}\PYG{n}{np}\PYG{o}{.}\PYG{n}{pi}\PYG{p}{,} \PYG{n}{np}\PYG{o}{.}\PYG{n}{pi}\PYG{p}{,} \PYG{l+m+mi}{2}\PYG{o}{*}\PYG{n}{np}\PYG{o}{.}\PYG{n}{pi}\PYG{o}{/}\PYG{l+m+mf}{50.}\PYG{p}{)}
\PYG{g+gp}{\PYGZgt{}\PYGZgt{}\PYGZgt{} }\PYG{n}{y} \PYG{o}{=} \PYG{o}{\PYGZhy{}}\PYG{n}{np}\PYG{o}{.}\PYG{n}{exp}\PYG{p}{(}\PYG{n}{kappa}\PYG{o}{*}\PYG{n}{np}\PYG{o}{.}\PYG{n}{cos}\PYG{p}{(}\PYG{n}{x}\PYG{o}{\PYGZhy{}}\PYG{n}{mu}\PYG{p}{)}\PYG{p}{)}\PYG{o}{/}\PYG{p}{(}\PYG{l+m+mi}{2}\PYG{o}{*}\PYG{n}{np}\PYG{o}{.}\PYG{n}{pi}\PYG{o}{*}\PYG{n}{sps}\PYG{o}{.}\PYG{n}{jn}\PYG{p}{(}\PYG{l+m+mi}{0}\PYG{p}{,}\PYG{n}{kappa}\PYG{p}{)}\PYG{p}{)}
\PYG{g+gp}{\PYGZgt{}\PYGZgt{}\PYGZgt{} }\PYG{n}{plt}\PYG{o}{.}\PYG{n}{plot}\PYG{p}{(}\PYG{n}{x}\PYG{p}{,} \PYG{n}{y}\PYG{o}{/}\PYG{n+nb}{max}\PYG{p}{(}\PYG{n}{y}\PYG{p}{)}\PYG{p}{,} \PYG{n}{linewidth}\PYG{o}{=}\PYG{l+m+mi}{2}\PYG{p}{,} \PYG{n}{color}\PYG{o}{=}\PYG{l+s}{\PYGZsq{}}\PYG{l+s}{r}\PYG{l+s}{\PYGZsq{}}\PYG{p}{)}
\PYG{g+gp}{\PYGZgt{}\PYGZgt{}\PYGZgt{} }\PYG{n}{plt}\PYG{o}{.}\PYG{n}{show}\PYG{p}{(}\PYG{p}{)}
\end{Verbatim}

\end{fulllineitems}

\index{wald() (in module acsSpeciesActivities)}

\begin{fulllineitems}
\phantomsection\label{acsSpeciesActivities:acsSpeciesActivities.wald}\pysiglinewithargsret{\code{acsSpeciesActivities.}\bfcode{wald}}{\emph{mean}, \emph{scale}, \emph{size=None}}{}
Draw samples from a Wald, or Inverse Gaussian, distribution.

As the scale approaches infinity, the distribution becomes more like a
Gaussian.

Some references claim that the Wald is an Inverse Gaussian with mean=1, but
this is by no means universal.

The Inverse Gaussian distribution was first studied in relationship to
Brownian motion. In 1956 M.C.K. Tweedie used the name Inverse Gaussian
because there is an inverse relationship between the time to cover a unit
distance and distance covered in unit time.
\begin{description}
\item[{mean}] \leavevmode{[}scalar{]}
Distribution mean, should be \textgreater{} 0.

\item[{scale}] \leavevmode{[}scalar{]}
Scale parameter, should be \textgreater{}= 0.

\item[{size}] \leavevmode{[}int or tuple of ints, optional{]}
Output shape. Default is None, in which case a single value is
returned.

\end{description}
\begin{description}
\item[{samples}] \leavevmode{[}ndarray or scalar{]}
Drawn sample, all greater than zero.

\end{description}

The probability density function for the Wald distribution is
\begin{gather}
\begin{split}P(x;mean,scale) = \sqrt{\frac{scale}{2\pi x^3}}e^
\frac{-scale(x-mean)^2}{2\cdotp mean^2x}\end{split}\notag
\end{gather}
As noted above the Inverse Gaussian distribution first arise from attempts
to model Brownian Motion. It is also a competitor to the Weibull for use in
reliability modeling and modeling stock returns and interest rate
processes.

Draw values from the distribution and plot the histogram:

\begin{Verbatim}[commandchars=\\\{\}]
\PYG{g+gp}{\PYGZgt{}\PYGZgt{}\PYGZgt{} }\PYG{k+kn}{import} \PYG{n+nn}{matplotlib.pyplot} \PYG{k+kn}{as} \PYG{n+nn}{plt}
\PYG{g+gp}{\PYGZgt{}\PYGZgt{}\PYGZgt{} }\PYG{n}{h} \PYG{o}{=} \PYG{n}{plt}\PYG{o}{.}\PYG{n}{hist}\PYG{p}{(}\PYG{n}{np}\PYG{o}{.}\PYG{n}{random}\PYG{o}{.}\PYG{n}{wald}\PYG{p}{(}\PYG{l+m+mi}{3}\PYG{p}{,} \PYG{l+m+mi}{2}\PYG{p}{,} \PYG{l+m+mi}{100000}\PYG{p}{)}\PYG{p}{,} \PYG{n}{bins}\PYG{o}{=}\PYG{l+m+mi}{200}\PYG{p}{,} \PYG{n}{normed}\PYG{o}{=}\PYG{n+nb+bp}{True}\PYG{p}{)}
\PYG{g+gp}{\PYGZgt{}\PYGZgt{}\PYGZgt{} }\PYG{n}{plt}\PYG{o}{.}\PYG{n}{show}\PYG{p}{(}\PYG{p}{)}
\end{Verbatim}

\end{fulllineitems}

\index{weibull() (in module acsSpeciesActivities)}

\begin{fulllineitems}
\phantomsection\label{acsSpeciesActivities:acsSpeciesActivities.weibull}\pysiglinewithargsret{\code{acsSpeciesActivities.}\bfcode{weibull}}{\emph{a}, \emph{size=None}}{}
Weibull distribution.

Draw samples from a 1-parameter Weibull distribution with the given
shape parameter \emph{a}.
\begin{gather}
\begin{split}X = (-ln(U))^{1/a}\end{split}\notag
\end{gather}
Here, U is drawn from the uniform distribution over (0,1{]}.

The more common 2-parameter Weibull, including a scale parameter
\(\lambda\) is just \(X = \lambda(-ln(U))^{1/a}\).
\begin{description}
\item[{a}] \leavevmode{[}float{]}
Shape of the distribution.

\item[{size}] \leavevmode{[}tuple of ints{]}
Output shape.  If the given shape is, e.g., \code{(m, n, k)}, then
\code{m * n * k} samples are drawn.

\end{description}

scipy.stats.distributions.weibull\_max
scipy.stats.distributions.weibull\_min
scipy.stats.distributions.genextreme
gumbel

The Weibull (or Type III asymptotic extreme value distribution for smallest
values, SEV Type III, or Rosin-Rammler distribution) is one of a class of
Generalized Extreme Value (GEV) distributions used in modeling extreme
value problems.  This class includes the Gumbel and Frechet distributions.

The probability density for the Weibull distribution is
\begin{gather}
\begin{split}p(x) = \frac{a}
{\lambda}(\frac{x}{\lambda})^{a-1}e^{-(x/\lambda)^a},\end{split}\notag
\end{gather}
where \(a\) is the shape and \(\lambda\) the scale.

The function has its peak (the mode) at
\(\lambda(\frac{a-1}{a})^{1/a}\).

When \code{a = 1}, the Weibull distribution reduces to the exponential
distribution.

Draw samples from the distribution:

\begin{Verbatim}[commandchars=\\\{\}]
\PYG{g+gp}{\PYGZgt{}\PYGZgt{}\PYGZgt{} }\PYG{n}{a} \PYG{o}{=} \PYG{l+m+mf}{5.} \PYG{c}{\PYGZsh{} shape}
\PYG{g+gp}{\PYGZgt{}\PYGZgt{}\PYGZgt{} }\PYG{n}{s} \PYG{o}{=} \PYG{n}{np}\PYG{o}{.}\PYG{n}{random}\PYG{o}{.}\PYG{n}{weibull}\PYG{p}{(}\PYG{n}{a}\PYG{p}{,} \PYG{l+m+mi}{1000}\PYG{p}{)}
\end{Verbatim}

Display the histogram of the samples, along with
the probability density function:

\begin{Verbatim}[commandchars=\\\{\}]
\PYG{g+gp}{\PYGZgt{}\PYGZgt{}\PYGZgt{} }\PYG{k+kn}{import} \PYG{n+nn}{matplotlib.pyplot} \PYG{k+kn}{as} \PYG{n+nn}{plt}
\PYG{g+gp}{\PYGZgt{}\PYGZgt{}\PYGZgt{} }\PYG{n}{x} \PYG{o}{=} \PYG{n}{np}\PYG{o}{.}\PYG{n}{arange}\PYG{p}{(}\PYG{l+m+mi}{1}\PYG{p}{,}\PYG{l+m+mf}{100.}\PYG{p}{)}\PYG{o}{/}\PYG{l+m+mf}{50.}
\PYG{g+gp}{\PYGZgt{}\PYGZgt{}\PYGZgt{} }\PYG{k}{def} \PYG{n+nf}{weib}\PYG{p}{(}\PYG{n}{x}\PYG{p}{,}\PYG{n}{n}\PYG{p}{,}\PYG{n}{a}\PYG{p}{)}\PYG{p}{:}
\PYG{g+gp}{... }    \PYG{k}{return} \PYG{p}{(}\PYG{n}{a} \PYG{o}{/} \PYG{n}{n}\PYG{p}{)} \PYG{o}{*} \PYG{p}{(}\PYG{n}{x} \PYG{o}{/} \PYG{n}{n}\PYG{p}{)}\PYG{o}{*}\PYG{o}{*}\PYG{p}{(}\PYG{n}{a} \PYG{o}{\PYGZhy{}} \PYG{l+m+mi}{1}\PYG{p}{)} \PYG{o}{*} \PYG{n}{np}\PYG{o}{.}\PYG{n}{exp}\PYG{p}{(}\PYG{o}{\PYGZhy{}}\PYG{p}{(}\PYG{n}{x} \PYG{o}{/} \PYG{n}{n}\PYG{p}{)}\PYG{o}{*}\PYG{o}{*}\PYG{n}{a}\PYG{p}{)}
\end{Verbatim}

\begin{Verbatim}[commandchars=\\\{\}]
\PYG{g+gp}{\PYGZgt{}\PYGZgt{}\PYGZgt{} }\PYG{n}{count}\PYG{p}{,} \PYG{n}{bins}\PYG{p}{,} \PYG{n}{ignored} \PYG{o}{=} \PYG{n}{plt}\PYG{o}{.}\PYG{n}{hist}\PYG{p}{(}\PYG{n}{np}\PYG{o}{.}\PYG{n}{random}\PYG{o}{.}\PYG{n}{weibull}\PYG{p}{(}\PYG{l+m+mf}{5.}\PYG{p}{,}\PYG{l+m+mi}{1000}\PYG{p}{)}\PYG{p}{)}
\PYG{g+gp}{\PYGZgt{}\PYGZgt{}\PYGZgt{} }\PYG{n}{x} \PYG{o}{=} \PYG{n}{np}\PYG{o}{.}\PYG{n}{arange}\PYG{p}{(}\PYG{l+m+mi}{1}\PYG{p}{,}\PYG{l+m+mf}{100.}\PYG{p}{)}\PYG{o}{/}\PYG{l+m+mf}{50.}
\PYG{g+gp}{\PYGZgt{}\PYGZgt{}\PYGZgt{} }\PYG{n}{scale} \PYG{o}{=} \PYG{n}{count}\PYG{o}{.}\PYG{n}{max}\PYG{p}{(}\PYG{p}{)}\PYG{o}{/}\PYG{n}{weib}\PYG{p}{(}\PYG{n}{x}\PYG{p}{,} \PYG{l+m+mf}{1.}\PYG{p}{,} \PYG{l+m+mf}{5.}\PYG{p}{)}\PYG{o}{.}\PYG{n}{max}\PYG{p}{(}\PYG{p}{)}
\PYG{g+gp}{\PYGZgt{}\PYGZgt{}\PYGZgt{} }\PYG{n}{plt}\PYG{o}{.}\PYG{n}{plot}\PYG{p}{(}\PYG{n}{x}\PYG{p}{,} \PYG{n}{weib}\PYG{p}{(}\PYG{n}{x}\PYG{p}{,} \PYG{l+m+mf}{1.}\PYG{p}{,} \PYG{l+m+mf}{5.}\PYG{p}{)}\PYG{o}{*}\PYG{n}{scale}\PYG{p}{)}
\PYG{g+gp}{\PYGZgt{}\PYGZgt{}\PYGZgt{} }\PYG{n}{plt}\PYG{o}{.}\PYG{n}{show}\PYG{p}{(}\PYG{p}{)}
\end{Verbatim}

\end{fulllineitems}

\index{zeroBeforeStrNum() (in module acsSpeciesActivities)}

\begin{fulllineitems}
\phantomsection\label{acsSpeciesActivities:acsSpeciesActivities.zeroBeforeStrNum}\pysiglinewithargsret{\code{acsSpeciesActivities.}\bfcode{zeroBeforeStrNum}}{\emph{tmpl}, \emph{tmpL}}{}
\end{fulllineitems}

\index{zipf() (in module acsSpeciesActivities)}

\begin{fulllineitems}
\phantomsection\label{acsSpeciesActivities:acsSpeciesActivities.zipf}\pysiglinewithargsret{\code{acsSpeciesActivities.}\bfcode{zipf}}{\emph{a}, \emph{size=None}}{}
Draw samples from a Zipf distribution.

Samples are drawn from a Zipf distribution with specified parameter
\emph{a} \textgreater{} 1.

The Zipf distribution (also known as the zeta distribution) is a
continuous probability distribution that satisfies Zipf's law: the
frequency of an item is inversely proportional to its rank in a
frequency table.
\begin{description}
\item[{a}] \leavevmode{[}float \textgreater{} 1{]}
Distribution parameter.

\item[{size}] \leavevmode{[}int or tuple of int, optional{]}
Output shape.  If the given shape is, e.g., \code{(m, n, k)}, then
\code{m * n * k} samples are drawn; a single integer is equivalent in
its result to providing a mono-tuple, i.e., a 1-D array of length
\emph{size} is returned.  The default is None, in which case a single
scalar is returned.

\end{description}
\begin{description}
\item[{samples}] \leavevmode{[}scalar or ndarray{]}
The returned samples are greater than or equal to one.

\end{description}
\begin{description}
\item[{scipy.stats.distributions.zipf}] \leavevmode{[}probability density function,{]}
distribution, or cumulative density function, etc.

\end{description}

The probability density for the Zipf distribution is
\begin{gather}
\begin{split}p(x) = \frac{x^{-a}}{\zeta(a)},\end{split}\notag
\end{gather}
where \(\zeta\) is the Riemann Zeta function.

It is named for the American linguist George Kingsley Zipf, who noted
that the frequency of any word in a sample of a language is inversely
proportional to its rank in the frequency table.

Zipf, G. K., \emph{Selected Studies of the Principle of Relative Frequency
in Language}, Cambridge, MA: Harvard Univ. Press, 1932.

Draw samples from the distribution:

\begin{Verbatim}[commandchars=\\\{\}]
\PYG{g+gp}{\PYGZgt{}\PYGZgt{}\PYGZgt{} }\PYG{n}{a} \PYG{o}{=} \PYG{l+m+mf}{2.} \PYG{c}{\PYGZsh{} parameter}
\PYG{g+gp}{\PYGZgt{}\PYGZgt{}\PYGZgt{} }\PYG{n}{s} \PYG{o}{=} \PYG{n}{np}\PYG{o}{.}\PYG{n}{random}\PYG{o}{.}\PYG{n}{zipf}\PYG{p}{(}\PYG{n}{a}\PYG{p}{,} \PYG{l+m+mi}{1000}\PYG{p}{)}
\end{Verbatim}

Display the histogram of the samples, along with
the probability density function:

\begin{Verbatim}[commandchars=\\\{\}]
\PYG{g+gp}{\PYGZgt{}\PYGZgt{}\PYGZgt{} }\PYG{k+kn}{import} \PYG{n+nn}{matplotlib.pyplot} \PYG{k+kn}{as} \PYG{n+nn}{plt}
\PYG{g+gp}{\PYGZgt{}\PYGZgt{}\PYGZgt{} }\PYG{k+kn}{import} \PYG{n+nn}{scipy.special} \PYG{k+kn}{as} \PYG{n+nn}{sps}
\PYG{g+go}{Truncate s values at 50 so plot is interesting}
\PYG{g+gp}{\PYGZgt{}\PYGZgt{}\PYGZgt{} }\PYG{n}{count}\PYG{p}{,} \PYG{n}{bins}\PYG{p}{,} \PYG{n}{ignored} \PYG{o}{=} \PYG{n}{plt}\PYG{o}{.}\PYG{n}{hist}\PYG{p}{(}\PYG{n}{s}\PYG{p}{[}\PYG{n}{s}\PYG{o}{\PYGZlt{}}\PYG{l+m+mi}{50}\PYG{p}{]}\PYG{p}{,} \PYG{l+m+mi}{50}\PYG{p}{,} \PYG{n}{normed}\PYG{o}{=}\PYG{n+nb+bp}{True}\PYG{p}{)}
\PYG{g+gp}{\PYGZgt{}\PYGZgt{}\PYGZgt{} }\PYG{n}{x} \PYG{o}{=} \PYG{n}{np}\PYG{o}{.}\PYG{n}{arange}\PYG{p}{(}\PYG{l+m+mf}{1.}\PYG{p}{,} \PYG{l+m+mf}{50.}\PYG{p}{)}
\PYG{g+gp}{\PYGZgt{}\PYGZgt{}\PYGZgt{} }\PYG{n}{y} \PYG{o}{=} \PYG{n}{x}\PYG{o}{*}\PYG{o}{*}\PYG{p}{(}\PYG{o}{\PYGZhy{}}\PYG{n}{a}\PYG{p}{)}\PYG{o}{/}\PYG{n}{sps}\PYG{o}{.}\PYG{n}{zetac}\PYG{p}{(}\PYG{n}{a}\PYG{p}{)}
\PYG{g+gp}{\PYGZgt{}\PYGZgt{}\PYGZgt{} }\PYG{n}{plt}\PYG{o}{.}\PYG{n}{plot}\PYG{p}{(}\PYG{n}{x}\PYG{p}{,} \PYG{n}{y}\PYG{o}{/}\PYG{n+nb}{max}\PYG{p}{(}\PYG{n}{y}\PYG{p}{)}\PYG{p}{,} \PYG{n}{linewidth}\PYG{o}{=}\PYG{l+m+mi}{2}\PYG{p}{,} \PYG{n}{color}\PYG{o}{=}\PYG{l+s}{\PYGZsq{}}\PYG{l+s}{r}\PYG{l+s}{\PYGZsq{}}\PYG{p}{)}
\PYG{g+gp}{\PYGZgt{}\PYGZgt{}\PYGZgt{} }\PYG{n}{plt}\PYG{o}{.}\PYG{n}{show}\PYG{p}{(}\PYG{p}{)}
\end{Verbatim}

\end{fulllineitems}



\chapter{acsStatesAnalysis Module}
\label{acsStatesAnalysis:module-acsStatesAnalysis}\label{acsStatesAnalysis:acsstatesanalysis-module}\label{acsStatesAnalysis::doc}\index{acsStatesAnalysis (module)}
Script to compute the distance between different state of the same simulation. 
comparison between t0 and t, t-1 and t adopting three different distance misure:
angle, euclidian distance and hamming distance. 
Moreover the script make an analysis of all the aggregative variables. 
\href{https://help.github.com/articles/fork-a-repo}{https://help.github.com/articles/fork-a-repo}
\index{beta() (in module acsStatesAnalysis)}

\begin{fulllineitems}
\phantomsection\label{acsStatesAnalysis:acsStatesAnalysis.beta}\pysiglinewithargsret{\code{acsStatesAnalysis.}\bfcode{beta}}{\emph{a}, \emph{b}, \emph{size=None}}{}
The Beta distribution over \code{{[}0, 1{]}}.

The Beta distribution is a special case of the Dirichlet distribution,
and is related to the Gamma distribution.  It has the probability
distribution function
\begin{gather}
\begin{split}f(x; a,b) = \frac{1}{B(\alpha, \beta)} x^{\alpha - 1}
(1 - x)^{\beta - 1},\end{split}\notag
\end{gather}
where the normalisation, B, is the beta function,
\begin{gather}
\begin{split}B(\alpha, \beta) = \int_0^1 t^{\alpha - 1}
(1 - t)^{\beta - 1} dt.\end{split}\notag
\end{gather}
It is often seen in Bayesian inference and order statistics.
\begin{description}
\item[{a}] \leavevmode{[}float{]}
Alpha, non-negative.

\item[{b}] \leavevmode{[}float{]}
Beta, non-negative.

\item[{size}] \leavevmode{[}tuple of ints, optional{]}
The number of samples to draw.  The output is packed according to
the size given.

\end{description}
\begin{description}
\item[{out}] \leavevmode{[}ndarray{]}
Array of the given shape, containing values drawn from a
Beta distribution.

\end{description}

\end{fulllineitems}

\index{binomial() (in module acsStatesAnalysis)}

\begin{fulllineitems}
\phantomsection\label{acsStatesAnalysis:acsStatesAnalysis.binomial}\pysiglinewithargsret{\code{acsStatesAnalysis.}\bfcode{binomial}}{\emph{n}, \emph{p}, \emph{size=None}}{}
Draw samples from a binomial distribution.

Samples are drawn from a Binomial distribution with specified
parameters, n trials and p probability of success where
n an integer \textgreater{}= 0 and p is in the interval {[}0,1{]}. (n may be
input as a float, but it is truncated to an integer in use)
\begin{description}
\item[{n}] \leavevmode{[}float (but truncated to an integer){]}
parameter, \textgreater{}= 0.

\item[{p}] \leavevmode{[}float{]}
parameter, \textgreater{}= 0 and \textless{}=1.

\item[{size}] \leavevmode{[}\{tuple, int\}{]}
Output shape.  If the given shape is, e.g., \code{(m, n, k)}, then
\code{m * n * k} samples are drawn.

\end{description}
\begin{description}
\item[{samples}] \leavevmode{[}\{ndarray, scalar\}{]}
where the values are all integers in  {[}0, n{]}.

\end{description}
\begin{description}
\item[{scipy.stats.distributions.binom}] \leavevmode{[}probability density function,{]}
distribution or cumulative density function, etc.

\end{description}

The probability density for the Binomial distribution is
\begin{gather}
\begin{split}P(N) = \binom{n}{N}p^N(1-p)^{n-N},\end{split}\notag
\end{gather}
where \(n\) is the number of trials, \(p\) is the probability
of success, and \(N\) is the number of successes.

When estimating the standard error of a proportion in a population by
using a random sample, the normal distribution works well unless the
product p*n \textless{}=5, where p = population proportion estimate, and n =
number of samples, in which case the binomial distribution is used
instead. For example, a sample of 15 people shows 4 who are left
handed, and 11 who are right handed. Then p = 4/15 = 27\%. 0.27*15 = 4,
so the binomial distribution should be used in this case.

Draw samples from the distribution:

\begin{Verbatim}[commandchars=\\\{\}]
\PYG{g+gp}{\PYGZgt{}\PYGZgt{}\PYGZgt{} }\PYG{n}{n}\PYG{p}{,} \PYG{n}{p} \PYG{o}{=} \PYG{l+m+mi}{10}\PYG{p}{,} \PYG{o}{.}\PYG{l+m+mi}{5} \PYG{c}{\PYGZsh{} number of trials, probability of each trial}
\PYG{g+gp}{\PYGZgt{}\PYGZgt{}\PYGZgt{} }\PYG{n}{s} \PYG{o}{=} \PYG{n}{np}\PYG{o}{.}\PYG{n}{random}\PYG{o}{.}\PYG{n}{binomial}\PYG{p}{(}\PYG{n}{n}\PYG{p}{,} \PYG{n}{p}\PYG{p}{,} \PYG{l+m+mi}{1000}\PYG{p}{)}
\PYG{g+go}{\PYGZsh{} result of flipping a coin 10 times, tested 1000 times.}
\end{Verbatim}

A real world example. A company drills 9 wild-cat oil exploration
wells, each with an estimated probability of success of 0.1. All nine
wells fail. What is the probability of that happening?

Let's do 20,000 trials of the model, and count the number that
generate zero positive results.

\begin{Verbatim}[commandchars=\\\{\}]
\PYG{g+gp}{\PYGZgt{}\PYGZgt{}\PYGZgt{} }\PYG{n+nb}{sum}\PYG{p}{(}\PYG{n}{np}\PYG{o}{.}\PYG{n}{random}\PYG{o}{.}\PYG{n}{binomial}\PYG{p}{(}\PYG{l+m+mi}{9}\PYG{p}{,}\PYG{l+m+mf}{0.1}\PYG{p}{,}\PYG{l+m+mi}{20000}\PYG{p}{)}\PYG{o}{==}\PYG{l+m+mi}{0}\PYG{p}{)}\PYG{o}{/}\PYG{l+m+mf}{20000.}
\PYG{g+go}{answer = 0.38885, or 38\PYGZpc{}.}
\end{Verbatim}

\end{fulllineitems}

\index{chisquare() (in module acsStatesAnalysis)}

\begin{fulllineitems}
\phantomsection\label{acsStatesAnalysis:acsStatesAnalysis.chisquare}\pysiglinewithargsret{\code{acsStatesAnalysis.}\bfcode{chisquare}}{\emph{df}, \emph{size=None}}{}
Draw samples from a chi-square distribution.

When \emph{df} independent random variables, each with standard normal
distributions (mean 0, variance 1), are squared and summed, the
resulting distribution is chi-square (see Notes).  This distribution
is often used in hypothesis testing.
\begin{description}
\item[{df}] \leavevmode{[}int{]}
Number of degrees of freedom.

\item[{size}] \leavevmode{[}tuple of ints, int, optional{]}
Size of the returned array.  By default, a scalar is
returned.

\end{description}
\begin{description}
\item[{output}] \leavevmode{[}ndarray{]}
Samples drawn from the distribution, packed in a \emph{size}-shaped
array.

\end{description}
\begin{description}
\item[{ValueError}] \leavevmode
When \emph{df} \textless{}= 0 or when an inappropriate \emph{size} (e.g. \code{size=-1})
is given.

\end{description}

The variable obtained by summing the squares of \emph{df} independent,
standard normally distributed random variables:
\begin{gather}
\begin{split}Q = \sum_{i=0}^{\mathtt{df}} X^2_i\end{split}\notag
\end{gather}
is chi-square distributed, denoted
\begin{gather}
\begin{split}Q \sim \chi^2_k.\end{split}\notag
\end{gather}
The probability density function of the chi-squared distribution is
\begin{gather}
\begin{split}p(x) = \frac{(1/2)^{k/2}}{\Gamma(k/2)}
x^{k/2 - 1} e^{-x/2},\end{split}\notag
\end{gather}
where \(\Gamma\) is the gamma function,
\begin{gather}
\begin{split}\Gamma(x) = \int_0^{-\infty} t^{x - 1} e^{-t} dt.\end{split}\notag
\end{gather}
\href{http://www.itl.nist.gov/div898/handbook/eda/section3/eda3666.htm}{NIST/SEMATECH e-Handbook of Statistical Methods}

\begin{Verbatim}[commandchars=\\\{\}]
\PYG{g+gp}{\PYGZgt{}\PYGZgt{}\PYGZgt{} }\PYG{n}{np}\PYG{o}{.}\PYG{n}{random}\PYG{o}{.}\PYG{n}{chisquare}\PYG{p}{(}\PYG{l+m+mi}{2}\PYG{p}{,}\PYG{l+m+mi}{4}\PYG{p}{)}
\PYG{g+go}{array([ 1.89920014,  9.00867716,  3.13710533,  5.62318272])}
\end{Verbatim}

\end{fulllineitems}

\index{distanceMisures() (in module acsStatesAnalysis)}

\begin{fulllineitems}
\phantomsection\label{acsStatesAnalysis:acsStatesAnalysis.distanceMisures}\pysiglinewithargsret{\code{acsStatesAnalysis.}\bfcode{distanceMisures}}{\emph{tmpSeqX}, \emph{tmpConcX}, \emph{tmpSeqY}, \emph{tmpConcY}, \emph{tmpIDs}}{}
Function to compute the angle between two multidimensional vectors

\end{fulllineitems}

\index{exponential() (in module acsStatesAnalysis)}

\begin{fulllineitems}
\phantomsection\label{acsStatesAnalysis:acsStatesAnalysis.exponential}\pysiglinewithargsret{\code{acsStatesAnalysis.}\bfcode{exponential}}{\emph{scale=1.0}, \emph{size=None}}{}
Exponential distribution.

Its probability density function is
\begin{gather}
\begin{split}f(x; \frac{1}{\beta}) = \frac{1}{\beta} \exp(-\frac{x}{\beta}),\end{split}\notag
\end{gather}
for \code{x \textgreater{} 0} and 0 elsewhere. \(\beta\) is the scale parameter,
which is the inverse of the rate parameter \(\lambda = 1/\beta\).
The rate parameter is an alternative, widely used parameterization
of the exponential distribution {\color{red}\bfseries{}{[}3{]}\_}.

The exponential distribution is a continuous analogue of the
geometric distribution.  It describes many common situations, such as
the size of raindrops measured over many rainstorms {\color{red}\bfseries{}{[}1{]}\_}, or the time
between page requests to Wikipedia {\color{red}\bfseries{}{[}2{]}\_}.
\begin{description}
\item[{scale}] \leavevmode{[}float{]}
The scale parameter, \(\beta = 1/\lambda\).

\item[{size}] \leavevmode{[}tuple of ints{]}
Number of samples to draw.  The output is shaped
according to \emph{size}.

\end{description}

\end{fulllineitems}

\index{f() (in module acsStatesAnalysis)}

\begin{fulllineitems}
\phantomsection\label{acsStatesAnalysis:acsStatesAnalysis.f}\pysiglinewithargsret{\code{acsStatesAnalysis.}\bfcode{f}}{\emph{dfnum}, \emph{dfden}, \emph{size=None}}{}
Draw samples from a F distribution.

Samples are drawn from an F distribution with specified parameters,
\emph{dfnum} (degrees of freedom in numerator) and \emph{dfden} (degrees of freedom
in denominator), where both parameters should be greater than zero.

The random variate of the F distribution (also known as the
Fisher distribution) is a continuous probability distribution
that arises in ANOVA tests, and is the ratio of two chi-square
variates.
\begin{description}
\item[{dfnum}] \leavevmode{[}float{]}
Degrees of freedom in numerator. Should be greater than zero.

\item[{dfden}] \leavevmode{[}float{]}
Degrees of freedom in denominator. Should be greater than zero.

\item[{size}] \leavevmode{[}\{tuple, int\}, optional{]}
Output shape.  If the given shape is, e.g., \code{(m, n, k)},
then \code{m * n * k} samples are drawn. By default only one sample
is returned.

\end{description}
\begin{description}
\item[{samples}] \leavevmode{[}\{ndarray, scalar\}{]}
Samples from the Fisher distribution.

\end{description}
\begin{description}
\item[{scipy.stats.distributions.f}] \leavevmode{[}probability density function,{]}
distribution or cumulative density function, etc.

\end{description}

The F statistic is used to compare in-group variances to between-group
variances. Calculating the distribution depends on the sampling, and
so it is a function of the respective degrees of freedom in the
problem.  The variable \emph{dfnum} is the number of samples minus one, the
between-groups degrees of freedom, while \emph{dfden} is the within-groups
degrees of freedom, the sum of the number of samples in each group
minus the number of groups.

An example from Glantz{[}1{]}, pp 47-40.
Two groups, children of diabetics (25 people) and children from people
without diabetes (25 controls). Fasting blood glucose was measured,
case group had a mean value of 86.1, controls had a mean value of
82.2. Standard deviations were 2.09 and 2.49 respectively. Are these
data consistent with the null hypothesis that the parents diabetic
status does not affect their children's blood glucose levels?
Calculating the F statistic from the data gives a value of 36.01.

Draw samples from the distribution:

\begin{Verbatim}[commandchars=\\\{\}]
\PYG{g+gp}{\PYGZgt{}\PYGZgt{}\PYGZgt{} }\PYG{n}{dfnum} \PYG{o}{=} \PYG{l+m+mf}{1.} \PYG{c}{\PYGZsh{} between group degrees of freedom}
\PYG{g+gp}{\PYGZgt{}\PYGZgt{}\PYGZgt{} }\PYG{n}{dfden} \PYG{o}{=} \PYG{l+m+mf}{48.} \PYG{c}{\PYGZsh{} within groups degrees of freedom}
\PYG{g+gp}{\PYGZgt{}\PYGZgt{}\PYGZgt{} }\PYG{n}{s} \PYG{o}{=} \PYG{n}{np}\PYG{o}{.}\PYG{n}{random}\PYG{o}{.}\PYG{n}{f}\PYG{p}{(}\PYG{n}{dfnum}\PYG{p}{,} \PYG{n}{dfden}\PYG{p}{,} \PYG{l+m+mi}{1000}\PYG{p}{)}
\end{Verbatim}

The lower bound for the top 1\% of the samples is :

\begin{Verbatim}[commandchars=\\\{\}]
\PYG{g+gp}{\PYGZgt{}\PYGZgt{}\PYGZgt{} }\PYG{n}{sort}\PYG{p}{(}\PYG{n}{s}\PYG{p}{)}\PYG{p}{[}\PYG{o}{\PYGZhy{}}\PYG{l+m+mi}{10}\PYG{p}{]}
\PYG{g+go}{7.61988120985}
\end{Verbatim}

So there is about a 1\% chance that the F statistic will exceed 7.62,
the measured value is 36, so the null hypothesis is rejected at the 1\%
level.

\end{fulllineitems}

\index{gamma() (in module acsStatesAnalysis)}

\begin{fulllineitems}
\phantomsection\label{acsStatesAnalysis:acsStatesAnalysis.gamma}\pysiglinewithargsret{\code{acsStatesAnalysis.}\bfcode{gamma}}{\emph{shape}, \emph{scale=1.0}, \emph{size=None}}{}
Draw samples from a Gamma distribution.

Samples are drawn from a Gamma distribution with specified parameters,
\emph{shape} (sometimes designated ``k'') and \emph{scale} (sometimes designated
``theta''), where both parameters are \textgreater{} 0.
\begin{description}
\item[{shape}] \leavevmode{[}scalar \textgreater{} 0{]}
The shape of the gamma distribution.

\item[{scale}] \leavevmode{[}scalar \textgreater{} 0, optional{]}
The scale of the gamma distribution.  Default is equal to 1.

\item[{size}] \leavevmode{[}shape\_tuple, optional{]}
Output shape.  If the given shape is, e.g., \code{(m, n, k)}, then
\code{m * n * k} samples are drawn.

\end{description}
\begin{description}
\item[{out}] \leavevmode{[}ndarray, float{]}
Returns one sample unless \emph{size} parameter is specified.

\end{description}
\begin{description}
\item[{scipy.stats.distributions.gamma}] \leavevmode{[}probability density function,{]}
distribution or cumulative density function, etc.

\end{description}

The probability density for the Gamma distribution is
\begin{gather}
\begin{split}p(x) = x^{k-1}\frac{e^{-x/\theta}}{\theta^k\Gamma(k)},\end{split}\notag
\end{gather}
where \(k\) is the shape and \(\theta\) the scale,
and \(\Gamma\) is the Gamma function.

The Gamma distribution is often used to model the times to failure of
electronic components, and arises naturally in processes for which the
waiting times between Poisson distributed events are relevant.

Draw samples from the distribution:

\begin{Verbatim}[commandchars=\\\{\}]
\PYG{g+gp}{\PYGZgt{}\PYGZgt{}\PYGZgt{} }\PYG{n}{shape}\PYG{p}{,} \PYG{n}{scale} \PYG{o}{=} \PYG{l+m+mf}{2.}\PYG{p}{,} \PYG{l+m+mf}{2.} \PYG{c}{\PYGZsh{} mean and dispersion}
\PYG{g+gp}{\PYGZgt{}\PYGZgt{}\PYGZgt{} }\PYG{n}{s} \PYG{o}{=} \PYG{n}{np}\PYG{o}{.}\PYG{n}{random}\PYG{o}{.}\PYG{n}{gamma}\PYG{p}{(}\PYG{n}{shape}\PYG{p}{,} \PYG{n}{scale}\PYG{p}{,} \PYG{l+m+mi}{1000}\PYG{p}{)}
\end{Verbatim}

Display the histogram of the samples, along with
the probability density function:

\begin{Verbatim}[commandchars=\\\{\}]
\PYG{g+gp}{\PYGZgt{}\PYGZgt{}\PYGZgt{} }\PYG{k+kn}{import} \PYG{n+nn}{matplotlib.pyplot} \PYG{k+kn}{as} \PYG{n+nn}{plt}
\PYG{g+gp}{\PYGZgt{}\PYGZgt{}\PYGZgt{} }\PYG{k+kn}{import} \PYG{n+nn}{scipy.special} \PYG{k+kn}{as} \PYG{n+nn}{sps}
\PYG{g+gp}{\PYGZgt{}\PYGZgt{}\PYGZgt{} }\PYG{n}{count}\PYG{p}{,} \PYG{n}{bins}\PYG{p}{,} \PYG{n}{ignored} \PYG{o}{=} \PYG{n}{plt}\PYG{o}{.}\PYG{n}{hist}\PYG{p}{(}\PYG{n}{s}\PYG{p}{,} \PYG{l+m+mi}{50}\PYG{p}{,} \PYG{n}{normed}\PYG{o}{=}\PYG{n+nb+bp}{True}\PYG{p}{)}
\PYG{g+gp}{\PYGZgt{}\PYGZgt{}\PYGZgt{} }\PYG{n}{y} \PYG{o}{=} \PYG{n}{bins}\PYG{o}{*}\PYG{o}{*}\PYG{p}{(}\PYG{n}{shape}\PYG{o}{\PYGZhy{}}\PYG{l+m+mi}{1}\PYG{p}{)}\PYG{o}{*}\PYG{p}{(}\PYG{n}{np}\PYG{o}{.}\PYG{n}{exp}\PYG{p}{(}\PYG{o}{\PYGZhy{}}\PYG{n}{bins}\PYG{o}{/}\PYG{n}{scale}\PYG{p}{)} \PYG{o}{/}
\PYG{g+gp}{... }                     \PYG{p}{(}\PYG{n}{sps}\PYG{o}{.}\PYG{n}{gamma}\PYG{p}{(}\PYG{n}{shape}\PYG{p}{)}\PYG{o}{*}\PYG{n}{scale}\PYG{o}{*}\PYG{o}{*}\PYG{n}{shape}\PYG{p}{)}\PYG{p}{)}
\PYG{g+gp}{\PYGZgt{}\PYGZgt{}\PYGZgt{} }\PYG{n}{plt}\PYG{o}{.}\PYG{n}{plot}\PYG{p}{(}\PYG{n}{bins}\PYG{p}{,} \PYG{n}{y}\PYG{p}{,} \PYG{n}{linewidth}\PYG{o}{=}\PYG{l+m+mi}{2}\PYG{p}{,} \PYG{n}{color}\PYG{o}{=}\PYG{l+s}{\PYGZsq{}}\PYG{l+s}{r}\PYG{l+s}{\PYGZsq{}}\PYG{p}{)}
\PYG{g+gp}{\PYGZgt{}\PYGZgt{}\PYGZgt{} }\PYG{n}{plt}\PYG{o}{.}\PYG{n}{show}\PYG{p}{(}\PYG{p}{)}
\end{Verbatim}

\end{fulllineitems}

\index{geometric() (in module acsStatesAnalysis)}

\begin{fulllineitems}
\phantomsection\label{acsStatesAnalysis:acsStatesAnalysis.geometric}\pysiglinewithargsret{\code{acsStatesAnalysis.}\bfcode{geometric}}{\emph{p}, \emph{size=None}}{}
Draw samples from the geometric distribution.

Bernoulli trials are experiments with one of two outcomes:
success or failure (an example of such an experiment is flipping
a coin).  The geometric distribution models the number of trials
that must be run in order to achieve success.  It is therefore
supported on the positive integers, \code{k = 1, 2, ...}.

The probability mass function of the geometric distribution is
\begin{gather}
\begin{split}f(k) = (1 - p)^{k - 1} p\end{split}\notag
\end{gather}
where \emph{p} is the probability of success of an individual trial.
\begin{description}
\item[{p}] \leavevmode{[}float{]}
The probability of success of an individual trial.

\item[{size}] \leavevmode{[}tuple of ints{]}
Number of values to draw from the distribution.  The output
is shaped according to \emph{size}.

\end{description}
\begin{description}
\item[{out}] \leavevmode{[}ndarray{]}
Samples from the geometric distribution, shaped according to
\emph{size}.

\end{description}

Draw ten thousand values from the geometric distribution,
with the probability of an individual success equal to 0.35:

\begin{Verbatim}[commandchars=\\\{\}]
\PYG{g+gp}{\PYGZgt{}\PYGZgt{}\PYGZgt{} }\PYG{n}{z} \PYG{o}{=} \PYG{n}{np}\PYG{o}{.}\PYG{n}{random}\PYG{o}{.}\PYG{n}{geometric}\PYG{p}{(}\PYG{n}{p}\PYG{o}{=}\PYG{l+m+mf}{0.35}\PYG{p}{,} \PYG{n}{size}\PYG{o}{=}\PYG{l+m+mi}{10000}\PYG{p}{)}
\end{Verbatim}

How many trials succeeded after a single run?

\begin{Verbatim}[commandchars=\\\{\}]
\PYG{g+gp}{\PYGZgt{}\PYGZgt{}\PYGZgt{} }\PYG{p}{(}\PYG{n}{z} \PYG{o}{==} \PYG{l+m+mi}{1}\PYG{p}{)}\PYG{o}{.}\PYG{n}{sum}\PYG{p}{(}\PYG{p}{)} \PYG{o}{/} \PYG{l+m+mf}{10000.}
\PYG{g+go}{0.34889999999999999 \PYGZsh{}random}
\end{Verbatim}

\end{fulllineitems}

\index{get\_state() (in module acsStatesAnalysis)}

\begin{fulllineitems}
\phantomsection\label{acsStatesAnalysis:acsStatesAnalysis.get_state}\pysiglinewithargsret{\code{acsStatesAnalysis.}\bfcode{get\_state}}{}{}
Return a tuple representing the internal state of the generator.

For more details, see \emph{set\_state}.
\begin{description}
\item[{out}] \leavevmode{[}tuple(str, ndarray of 624 uints, int, int, float){]}
The returned tuple has the following items:
\begin{enumerate}
\item {} 
the string `MT19937'.

\item {} 
a 1-D array of 624 unsigned integer keys.

\item {} 
an integer \code{pos}.

\item {} 
an integer \code{has\_gauss}.

\item {} 
a float \code{cached\_gaussian}.

\end{enumerate}

\end{description}

set\_state

\emph{set\_state} and \emph{get\_state} are not needed to work with any of the
random distributions in NumPy. If the internal state is manually altered,
the user should know exactly what he/she is doing.

\end{fulllineitems}

\index{gumbel() (in module acsStatesAnalysis)}

\begin{fulllineitems}
\phantomsection\label{acsStatesAnalysis:acsStatesAnalysis.gumbel}\pysiglinewithargsret{\code{acsStatesAnalysis.}\bfcode{gumbel}}{\emph{loc=0.0}, \emph{scale=1.0}, \emph{size=None}}{}
Gumbel distribution.

Draw samples from a Gumbel distribution with specified location and scale.
For more information on the Gumbel distribution, see Notes and References
below.
\begin{description}
\item[{loc}] \leavevmode{[}float{]}
The location of the mode of the distribution.

\item[{scale}] \leavevmode{[}float{]}
The scale parameter of the distribution.

\item[{size}] \leavevmode{[}tuple of ints{]}
Output shape.  If the given shape is, e.g., \code{(m, n, k)}, then
\code{m * n * k} samples are drawn.

\end{description}
\begin{description}
\item[{out}] \leavevmode{[}ndarray{]}
The samples

\end{description}

scipy.stats.gumbel\_l
scipy.stats.gumbel\_r
scipy.stats.genextreme
\begin{quote}

probability density function, distribution, or cumulative density
function, etc. for each of the above
\end{quote}

weibull

The Gumbel (or Smallest Extreme Value (SEV) or the Smallest Extreme Value
Type I) distribution is one of a class of Generalized Extreme Value (GEV)
distributions used in modeling extreme value problems.  The Gumbel is a
special case of the Extreme Value Type I distribution for maximums from
distributions with ``exponential-like'' tails.

The probability density for the Gumbel distribution is
\begin{gather}
\begin{split}p(x) = \frac{e^{-(x - \mu)/ \beta}}{\beta} e^{ -e^{-(x - \mu)/
\beta}},\end{split}\notag
\end{gather}
where \(\mu\) is the mode, a location parameter, and \(\beta\) is
the scale parameter.

The Gumbel (named for German mathematician Emil Julius Gumbel) was used
very early in the hydrology literature, for modeling the occurrence of
flood events. It is also used for modeling maximum wind speed and rainfall
rates.  It is a ``fat-tailed'' distribution - the probability of an event in
the tail of the distribution is larger than if one used a Gaussian, hence
the surprisingly frequent occurrence of 100-year floods. Floods were
initially modeled as a Gaussian process, which underestimated the frequency
of extreme events.

It is one of a class of extreme value distributions, the Generalized
Extreme Value (GEV) distributions, which also includes the Weibull and
Frechet.

The function has a mean of \(\mu + 0.57721\beta\) and a variance of
\(\frac{\pi^2}{6}\beta^2\).

Gumbel, E. J., \emph{Statistics of Extremes}, New York: Columbia University
Press, 1958.

Reiss, R.-D. and Thomas, M., \emph{Statistical Analysis of Extreme Values from
Insurance, Finance, Hydrology and Other Fields}, Basel: Birkhauser Verlag,
2001.

Draw samples from the distribution:

\begin{Verbatim}[commandchars=\\\{\}]
\PYG{g+gp}{\PYGZgt{}\PYGZgt{}\PYGZgt{} }\PYG{n}{mu}\PYG{p}{,} \PYG{n}{beta} \PYG{o}{=} \PYG{l+m+mi}{0}\PYG{p}{,} \PYG{l+m+mf}{0.1} \PYG{c}{\PYGZsh{} location and scale}
\PYG{g+gp}{\PYGZgt{}\PYGZgt{}\PYGZgt{} }\PYG{n}{s} \PYG{o}{=} \PYG{n}{np}\PYG{o}{.}\PYG{n}{random}\PYG{o}{.}\PYG{n}{gumbel}\PYG{p}{(}\PYG{n}{mu}\PYG{p}{,} \PYG{n}{beta}\PYG{p}{,} \PYG{l+m+mi}{1000}\PYG{p}{)}
\end{Verbatim}

Display the histogram of the samples, along with
the probability density function:

\begin{Verbatim}[commandchars=\\\{\}]
\PYG{g+gp}{\PYGZgt{}\PYGZgt{}\PYGZgt{} }\PYG{k+kn}{import} \PYG{n+nn}{matplotlib.pyplot} \PYG{k+kn}{as} \PYG{n+nn}{plt}
\PYG{g+gp}{\PYGZgt{}\PYGZgt{}\PYGZgt{} }\PYG{n}{count}\PYG{p}{,} \PYG{n}{bins}\PYG{p}{,} \PYG{n}{ignored} \PYG{o}{=} \PYG{n}{plt}\PYG{o}{.}\PYG{n}{hist}\PYG{p}{(}\PYG{n}{s}\PYG{p}{,} \PYG{l+m+mi}{30}\PYG{p}{,} \PYG{n}{normed}\PYG{o}{=}\PYG{n+nb+bp}{True}\PYG{p}{)}
\PYG{g+gp}{\PYGZgt{}\PYGZgt{}\PYGZgt{} }\PYG{n}{plt}\PYG{o}{.}\PYG{n}{plot}\PYG{p}{(}\PYG{n}{bins}\PYG{p}{,} \PYG{p}{(}\PYG{l+m+mi}{1}\PYG{o}{/}\PYG{n}{beta}\PYG{p}{)}\PYG{o}{*}\PYG{n}{np}\PYG{o}{.}\PYG{n}{exp}\PYG{p}{(}\PYG{o}{\PYGZhy{}}\PYG{p}{(}\PYG{n}{bins} \PYG{o}{\PYGZhy{}} \PYG{n}{mu}\PYG{p}{)}\PYG{o}{/}\PYG{n}{beta}\PYG{p}{)}
\PYG{g+gp}{... }         \PYG{o}{*} \PYG{n}{np}\PYG{o}{.}\PYG{n}{exp}\PYG{p}{(} \PYG{o}{\PYGZhy{}}\PYG{n}{np}\PYG{o}{.}\PYG{n}{exp}\PYG{p}{(} \PYG{o}{\PYGZhy{}}\PYG{p}{(}\PYG{n}{bins} \PYG{o}{\PYGZhy{}} \PYG{n}{mu}\PYG{p}{)} \PYG{o}{/}\PYG{n}{beta}\PYG{p}{)} \PYG{p}{)}\PYG{p}{,}
\PYG{g+gp}{... }         \PYG{n}{linewidth}\PYG{o}{=}\PYG{l+m+mi}{2}\PYG{p}{,} \PYG{n}{color}\PYG{o}{=}\PYG{l+s}{\PYGZsq{}}\PYG{l+s}{r}\PYG{l+s}{\PYGZsq{}}\PYG{p}{)}
\PYG{g+gp}{\PYGZgt{}\PYGZgt{}\PYGZgt{} }\PYG{n}{plt}\PYG{o}{.}\PYG{n}{show}\PYG{p}{(}\PYG{p}{)}
\end{Verbatim}

Show how an extreme value distribution can arise from a Gaussian process
and compare to a Gaussian:

\begin{Verbatim}[commandchars=\\\{\}]
\PYG{g+gp}{\PYGZgt{}\PYGZgt{}\PYGZgt{} }\PYG{n}{means} \PYG{o}{=} \PYG{p}{[}\PYG{p}{]}
\PYG{g+gp}{\PYGZgt{}\PYGZgt{}\PYGZgt{} }\PYG{n}{maxima} \PYG{o}{=} \PYG{p}{[}\PYG{p}{]}
\PYG{g+gp}{\PYGZgt{}\PYGZgt{}\PYGZgt{} }\PYG{k}{for} \PYG{n}{i} \PYG{o+ow}{in} \PYG{n+nb}{range}\PYG{p}{(}\PYG{l+m+mi}{0}\PYG{p}{,}\PYG{l+m+mi}{1000}\PYG{p}{)} \PYG{p}{:}
\PYG{g+gp}{... }   \PYG{n}{a} \PYG{o}{=} \PYG{n}{np}\PYG{o}{.}\PYG{n}{random}\PYG{o}{.}\PYG{n}{normal}\PYG{p}{(}\PYG{n}{mu}\PYG{p}{,} \PYG{n}{beta}\PYG{p}{,} \PYG{l+m+mi}{1000}\PYG{p}{)}
\PYG{g+gp}{... }   \PYG{n}{means}\PYG{o}{.}\PYG{n}{append}\PYG{p}{(}\PYG{n}{a}\PYG{o}{.}\PYG{n}{mean}\PYG{p}{(}\PYG{p}{)}\PYG{p}{)}
\PYG{g+gp}{... }   \PYG{n}{maxima}\PYG{o}{.}\PYG{n}{append}\PYG{p}{(}\PYG{n}{a}\PYG{o}{.}\PYG{n}{max}\PYG{p}{(}\PYG{p}{)}\PYG{p}{)}
\PYG{g+gp}{\PYGZgt{}\PYGZgt{}\PYGZgt{} }\PYG{n}{count}\PYG{p}{,} \PYG{n}{bins}\PYG{p}{,} \PYG{n}{ignored} \PYG{o}{=} \PYG{n}{plt}\PYG{o}{.}\PYG{n}{hist}\PYG{p}{(}\PYG{n}{maxima}\PYG{p}{,} \PYG{l+m+mi}{30}\PYG{p}{,} \PYG{n}{normed}\PYG{o}{=}\PYG{n+nb+bp}{True}\PYG{p}{)}
\PYG{g+gp}{\PYGZgt{}\PYGZgt{}\PYGZgt{} }\PYG{n}{beta} \PYG{o}{=} \PYG{n}{np}\PYG{o}{.}\PYG{n}{std}\PYG{p}{(}\PYG{n}{maxima}\PYG{p}{)}\PYG{o}{*}\PYG{n}{np}\PYG{o}{.}\PYG{n}{pi}\PYG{o}{/}\PYG{n}{np}\PYG{o}{.}\PYG{n}{sqrt}\PYG{p}{(}\PYG{l+m+mi}{6}\PYG{p}{)}
\PYG{g+gp}{\PYGZgt{}\PYGZgt{}\PYGZgt{} }\PYG{n}{mu} \PYG{o}{=} \PYG{n}{np}\PYG{o}{.}\PYG{n}{mean}\PYG{p}{(}\PYG{n}{maxima}\PYG{p}{)} \PYG{o}{\PYGZhy{}} \PYG{l+m+mf}{0.57721}\PYG{o}{*}\PYG{n}{beta}
\PYG{g+gp}{\PYGZgt{}\PYGZgt{}\PYGZgt{} }\PYG{n}{plt}\PYG{o}{.}\PYG{n}{plot}\PYG{p}{(}\PYG{n}{bins}\PYG{p}{,} \PYG{p}{(}\PYG{l+m+mi}{1}\PYG{o}{/}\PYG{n}{beta}\PYG{p}{)}\PYG{o}{*}\PYG{n}{np}\PYG{o}{.}\PYG{n}{exp}\PYG{p}{(}\PYG{o}{\PYGZhy{}}\PYG{p}{(}\PYG{n}{bins} \PYG{o}{\PYGZhy{}} \PYG{n}{mu}\PYG{p}{)}\PYG{o}{/}\PYG{n}{beta}\PYG{p}{)}
\PYG{g+gp}{... }         \PYG{o}{*} \PYG{n}{np}\PYG{o}{.}\PYG{n}{exp}\PYG{p}{(}\PYG{o}{\PYGZhy{}}\PYG{n}{np}\PYG{o}{.}\PYG{n}{exp}\PYG{p}{(}\PYG{o}{\PYGZhy{}}\PYG{p}{(}\PYG{n}{bins} \PYG{o}{\PYGZhy{}} \PYG{n}{mu}\PYG{p}{)}\PYG{o}{/}\PYG{n}{beta}\PYG{p}{)}\PYG{p}{)}\PYG{p}{,}
\PYG{g+gp}{... }         \PYG{n}{linewidth}\PYG{o}{=}\PYG{l+m+mi}{2}\PYG{p}{,} \PYG{n}{color}\PYG{o}{=}\PYG{l+s}{\PYGZsq{}}\PYG{l+s}{r}\PYG{l+s}{\PYGZsq{}}\PYG{p}{)}
\PYG{g+gp}{\PYGZgt{}\PYGZgt{}\PYGZgt{} }\PYG{n}{plt}\PYG{o}{.}\PYG{n}{plot}\PYG{p}{(}\PYG{n}{bins}\PYG{p}{,} \PYG{l+m+mi}{1}\PYG{o}{/}\PYG{p}{(}\PYG{n}{beta} \PYG{o}{*} \PYG{n}{np}\PYG{o}{.}\PYG{n}{sqrt}\PYG{p}{(}\PYG{l+m+mi}{2} \PYG{o}{*} \PYG{n}{np}\PYG{o}{.}\PYG{n}{pi}\PYG{p}{)}\PYG{p}{)}
\PYG{g+gp}{... }         \PYG{o}{*} \PYG{n}{np}\PYG{o}{.}\PYG{n}{exp}\PYG{p}{(}\PYG{o}{\PYGZhy{}}\PYG{p}{(}\PYG{n}{bins} \PYG{o}{\PYGZhy{}} \PYG{n}{mu}\PYG{p}{)}\PYG{o}{*}\PYG{o}{*}\PYG{l+m+mi}{2} \PYG{o}{/} \PYG{p}{(}\PYG{l+m+mi}{2} \PYG{o}{*} \PYG{n}{beta}\PYG{o}{*}\PYG{o}{*}\PYG{l+m+mi}{2}\PYG{p}{)}\PYG{p}{)}\PYG{p}{,}
\PYG{g+gp}{... }         \PYG{n}{linewidth}\PYG{o}{=}\PYG{l+m+mi}{2}\PYG{p}{,} \PYG{n}{color}\PYG{o}{=}\PYG{l+s}{\PYGZsq{}}\PYG{l+s}{g}\PYG{l+s}{\PYGZsq{}}\PYG{p}{)}
\PYG{g+gp}{\PYGZgt{}\PYGZgt{}\PYGZgt{} }\PYG{n}{plt}\PYG{o}{.}\PYG{n}{show}\PYG{p}{(}\PYG{p}{)}
\end{Verbatim}

\end{fulllineitems}

\index{hypergeometric() (in module acsStatesAnalysis)}

\begin{fulllineitems}
\phantomsection\label{acsStatesAnalysis:acsStatesAnalysis.hypergeometric}\pysiglinewithargsret{\code{acsStatesAnalysis.}\bfcode{hypergeometric}}{\emph{ngood}, \emph{nbad}, \emph{nsample}, \emph{size=None}}{}
Draw samples from a Hypergeometric distribution.

Samples are drawn from a Hypergeometric distribution with specified
parameters, ngood (ways to make a good selection), nbad (ways to make
a bad selection), and nsample = number of items sampled, which is less
than or equal to the sum ngood + nbad.
\begin{description}
\item[{ngood}] \leavevmode{[}int or array\_like{]}
Number of ways to make a good selection.  Must be nonnegative.

\item[{nbad}] \leavevmode{[}int or array\_like{]}
Number of ways to make a bad selection.  Must be nonnegative.

\item[{nsample}] \leavevmode{[}int or array\_like{]}
Number of items sampled.  Must be at least 1 and at most
\code{ngood + nbad}.

\item[{size}] \leavevmode{[}int or tuple of int{]}
Output shape.  If the given shape is, e.g., \code{(m, n, k)}, then
\code{m * n * k} samples are drawn.

\end{description}
\begin{description}
\item[{samples}] \leavevmode{[}ndarray or scalar{]}
The values are all integers in  {[}0, n{]}.

\end{description}
\begin{description}
\item[{scipy.stats.distributions.hypergeom}] \leavevmode{[}probability density function,{]}
distribution or cumulative density function, etc.

\end{description}

The probability density for the Hypergeometric distribution is
\begin{gather}
\begin{split}P(x) = \frac{\binom{m}{n}\binom{N-m}{n-x}}{\binom{N}{n}},\end{split}\notag
\end{gather}
where \(0 \le x \le m\) and \(n+m-N \le x \le n\)

for P(x) the probability of x successes, n = ngood, m = nbad, and
N = number of samples.

Consider an urn with black and white marbles in it, ngood of them
black and nbad are white. If you draw nsample balls without
replacement, then the Hypergeometric distribution describes the
distribution of black balls in the drawn sample.

Note that this distribution is very similar to the Binomial
distribution, except that in this case, samples are drawn without
replacement, whereas in the Binomial case samples are drawn with
replacement (or the sample space is infinite). As the sample space
becomes large, this distribution approaches the Binomial.

Draw samples from the distribution:

\begin{Verbatim}[commandchars=\\\{\}]
\PYG{g+gp}{\PYGZgt{}\PYGZgt{}\PYGZgt{} }\PYG{n}{ngood}\PYG{p}{,} \PYG{n}{nbad}\PYG{p}{,} \PYG{n}{nsamp} \PYG{o}{=} \PYG{l+m+mi}{100}\PYG{p}{,} \PYG{l+m+mi}{2}\PYG{p}{,} \PYG{l+m+mi}{10}
\PYG{g+go}{\PYGZsh{} number of good, number of bad, and number of samples}
\PYG{g+gp}{\PYGZgt{}\PYGZgt{}\PYGZgt{} }\PYG{n}{s} \PYG{o}{=} \PYG{n}{np}\PYG{o}{.}\PYG{n}{random}\PYG{o}{.}\PYG{n}{hypergeometric}\PYG{p}{(}\PYG{n}{ngood}\PYG{p}{,} \PYG{n}{nbad}\PYG{p}{,} \PYG{n}{nsamp}\PYG{p}{,} \PYG{l+m+mi}{1000}\PYG{p}{)}
\PYG{g+gp}{\PYGZgt{}\PYGZgt{}\PYGZgt{} }\PYG{n}{hist}\PYG{p}{(}\PYG{n}{s}\PYG{p}{)}
\PYG{g+go}{\PYGZsh{}   note that it is very unlikely to grab both bad items}
\end{Verbatim}

Suppose you have an urn with 15 white and 15 black marbles.
If you pull 15 marbles at random, how likely is it that
12 or more of them are one color?

\begin{Verbatim}[commandchars=\\\{\}]
\PYG{g+gp}{\PYGZgt{}\PYGZgt{}\PYGZgt{} }\PYG{n}{s} \PYG{o}{=} \PYG{n}{np}\PYG{o}{.}\PYG{n}{random}\PYG{o}{.}\PYG{n}{hypergeometric}\PYG{p}{(}\PYG{l+m+mi}{15}\PYG{p}{,} \PYG{l+m+mi}{15}\PYG{p}{,} \PYG{l+m+mi}{15}\PYG{p}{,} \PYG{l+m+mi}{100000}\PYG{p}{)}
\PYG{g+gp}{\PYGZgt{}\PYGZgt{}\PYGZgt{} }\PYG{n+nb}{sum}\PYG{p}{(}\PYG{n}{s}\PYG{o}{\PYGZgt{}}\PYG{o}{=}\PYG{l+m+mi}{12}\PYG{p}{)}\PYG{o}{/}\PYG{l+m+mf}{100000.} \PYG{o}{+} \PYG{n+nb}{sum}\PYG{p}{(}\PYG{n}{s}\PYG{o}{\PYGZlt{}}\PYG{o}{=}\PYG{l+m+mi}{3}\PYG{p}{)}\PYG{o}{/}\PYG{l+m+mf}{100000.}
\PYG{g+go}{\PYGZsh{}   answer = 0.003 ... pretty unlikely!}
\end{Verbatim}

\end{fulllineitems}

\index{laplace() (in module acsStatesAnalysis)}

\begin{fulllineitems}
\phantomsection\label{acsStatesAnalysis:acsStatesAnalysis.laplace}\pysiglinewithargsret{\code{acsStatesAnalysis.}\bfcode{laplace}}{\emph{loc=0.0}, \emph{scale=1.0}, \emph{size=None}}{}
Draw samples from the Laplace or double exponential distribution with
specified location (or mean) and scale (decay).

The Laplace distribution is similar to the Gaussian/normal distribution,
but is sharper at the peak and has fatter tails. It represents the
difference between two independent, identically distributed exponential
random variables.
\begin{description}
\item[{loc}] \leavevmode{[}float{]}
The position, \(\mu\), of the distribution peak.

\item[{scale}] \leavevmode{[}float{]}
\(\lambda\), the exponential decay.

\end{description}

It has the probability density function
\begin{gather}
\begin{split}f(x; \mu, \lambda) = \frac{1}{2\lambda}
\exp\left(-\frac{|x - \mu|}{\lambda}\right).\end{split}\notag
\end{gather}
The first law of Laplace, from 1774, states that the frequency of an error
can be expressed as an exponential function of the absolute magnitude of
the error, which leads to the Laplace distribution. For many problems in
Economics and Health sciences, this distribution seems to model the data
better than the standard Gaussian distribution

Draw samples from the distribution

\begin{Verbatim}[commandchars=\\\{\}]
\PYG{g+gp}{\PYGZgt{}\PYGZgt{}\PYGZgt{} }\PYG{n}{loc}\PYG{p}{,} \PYG{n}{scale} \PYG{o}{=} \PYG{l+m+mf}{0.}\PYG{p}{,} \PYG{l+m+mf}{1.}
\PYG{g+gp}{\PYGZgt{}\PYGZgt{}\PYGZgt{} }\PYG{n}{s} \PYG{o}{=} \PYG{n}{np}\PYG{o}{.}\PYG{n}{random}\PYG{o}{.}\PYG{n}{laplace}\PYG{p}{(}\PYG{n}{loc}\PYG{p}{,} \PYG{n}{scale}\PYG{p}{,} \PYG{l+m+mi}{1000}\PYG{p}{)}
\end{Verbatim}

Display the histogram of the samples, along with
the probability density function:

\begin{Verbatim}[commandchars=\\\{\}]
\PYG{g+gp}{\PYGZgt{}\PYGZgt{}\PYGZgt{} }\PYG{k+kn}{import} \PYG{n+nn}{matplotlib.pyplot} \PYG{k+kn}{as} \PYG{n+nn}{plt}
\PYG{g+gp}{\PYGZgt{}\PYGZgt{}\PYGZgt{} }\PYG{n}{count}\PYG{p}{,} \PYG{n}{bins}\PYG{p}{,} \PYG{n}{ignored} \PYG{o}{=} \PYG{n}{plt}\PYG{o}{.}\PYG{n}{hist}\PYG{p}{(}\PYG{n}{s}\PYG{p}{,} \PYG{l+m+mi}{30}\PYG{p}{,} \PYG{n}{normed}\PYG{o}{=}\PYG{n+nb+bp}{True}\PYG{p}{)}
\PYG{g+gp}{\PYGZgt{}\PYGZgt{}\PYGZgt{} }\PYG{n}{x} \PYG{o}{=} \PYG{n}{np}\PYG{o}{.}\PYG{n}{arange}\PYG{p}{(}\PYG{o}{\PYGZhy{}}\PYG{l+m+mf}{8.}\PYG{p}{,} \PYG{l+m+mf}{8.}\PYG{p}{,} \PYG{o}{.}\PYG{l+m+mo}{01}\PYG{p}{)}
\PYG{g+gp}{\PYGZgt{}\PYGZgt{}\PYGZgt{} }\PYG{n}{pdf} \PYG{o}{=} \PYG{n}{np}\PYG{o}{.}\PYG{n}{exp}\PYG{p}{(}\PYG{o}{\PYGZhy{}}\PYG{n+nb}{abs}\PYG{p}{(}\PYG{n}{x}\PYG{o}{\PYGZhy{}}\PYG{n}{loc}\PYG{o}{/}\PYG{n}{scale}\PYG{p}{)}\PYG{p}{)}\PYG{o}{/}\PYG{p}{(}\PYG{l+m+mf}{2.}\PYG{o}{*}\PYG{n}{scale}\PYG{p}{)}
\PYG{g+gp}{\PYGZgt{}\PYGZgt{}\PYGZgt{} }\PYG{n}{plt}\PYG{o}{.}\PYG{n}{plot}\PYG{p}{(}\PYG{n}{x}\PYG{p}{,} \PYG{n}{pdf}\PYG{p}{)}
\end{Verbatim}

Plot Gaussian for comparison:

\begin{Verbatim}[commandchars=\\\{\}]
\PYG{g+gp}{\PYGZgt{}\PYGZgt{}\PYGZgt{} }\PYG{n}{g} \PYG{o}{=} \PYG{p}{(}\PYG{l+m+mi}{1}\PYG{o}{/}\PYG{p}{(}\PYG{n}{scale} \PYG{o}{*} \PYG{n}{np}\PYG{o}{.}\PYG{n}{sqrt}\PYG{p}{(}\PYG{l+m+mi}{2} \PYG{o}{*} \PYG{n}{np}\PYG{o}{.}\PYG{n}{pi}\PYG{p}{)}\PYG{p}{)} \PYG{o}{*} 
\PYG{g+gp}{... }     \PYG{n}{np}\PYG{o}{.}\PYG{n}{exp}\PYG{p}{(} \PYG{o}{\PYGZhy{}} \PYG{p}{(}\PYG{n}{x} \PYG{o}{\PYGZhy{}} \PYG{n}{loc}\PYG{p}{)}\PYG{o}{*}\PYG{o}{*}\PYG{l+m+mi}{2} \PYG{o}{/} \PYG{p}{(}\PYG{l+m+mi}{2} \PYG{o}{*} \PYG{n}{scale}\PYG{o}{*}\PYG{o}{*}\PYG{l+m+mi}{2}\PYG{p}{)} \PYG{p}{)}\PYG{p}{)}
\PYG{g+gp}{\PYGZgt{}\PYGZgt{}\PYGZgt{} }\PYG{n}{plt}\PYG{o}{.}\PYG{n}{plot}\PYG{p}{(}\PYG{n}{x}\PYG{p}{,}\PYG{n}{g}\PYG{p}{)}
\end{Verbatim}

\end{fulllineitems}

\index{logistic() (in module acsStatesAnalysis)}

\begin{fulllineitems}
\phantomsection\label{acsStatesAnalysis:acsStatesAnalysis.logistic}\pysiglinewithargsret{\code{acsStatesAnalysis.}\bfcode{logistic}}{\emph{loc=0.0}, \emph{scale=1.0}, \emph{size=None}}{}
Draw samples from a Logistic distribution.

Samples are drawn from a Logistic distribution with specified
parameters, loc (location or mean, also median), and scale (\textgreater{}0).

loc : float

scale : float \textgreater{} 0.
\begin{description}
\item[{size}] \leavevmode{[}\{tuple, int\}{]}
Output shape.  If the given shape is, e.g., \code{(m, n, k)}, then
\code{m * n * k} samples are drawn.

\end{description}
\begin{description}
\item[{samples}] \leavevmode{[}\{ndarray, scalar\}{]}
where the values are all integers in  {[}0, n{]}.

\end{description}
\begin{description}
\item[{scipy.stats.distributions.logistic}] \leavevmode{[}probability density function,{]}
distribution or cumulative density function, etc.

\end{description}

The probability density for the Logistic distribution is
\begin{gather}
\begin{split}P(x) = P(x) = \frac{e^{-(x-\mu)/s}}{s(1+e^{-(x-\mu)/s})^2},\end{split}\notag
\end{gather}
where \(\mu\) = location and \(s\) = scale.

The Logistic distribution is used in Extreme Value problems where it
can act as a mixture of Gumbel distributions, in Epidemiology, and by
the World Chess Federation (FIDE) where it is used in the Elo ranking
system, assuming the performance of each player is a logistically
distributed random variable.

Draw samples from the distribution:

\begin{Verbatim}[commandchars=\\\{\}]
\PYG{g+gp}{\PYGZgt{}\PYGZgt{}\PYGZgt{} }\PYG{n}{loc}\PYG{p}{,} \PYG{n}{scale} \PYG{o}{=} \PYG{l+m+mi}{10}\PYG{p}{,} \PYG{l+m+mi}{1}
\PYG{g+gp}{\PYGZgt{}\PYGZgt{}\PYGZgt{} }\PYG{n}{s} \PYG{o}{=} \PYG{n}{np}\PYG{o}{.}\PYG{n}{random}\PYG{o}{.}\PYG{n}{logistic}\PYG{p}{(}\PYG{n}{loc}\PYG{p}{,} \PYG{n}{scale}\PYG{p}{,} \PYG{l+m+mi}{10000}\PYG{p}{)}
\PYG{g+gp}{\PYGZgt{}\PYGZgt{}\PYGZgt{} }\PYG{n}{count}\PYG{p}{,} \PYG{n}{bins}\PYG{p}{,} \PYG{n}{ignored} \PYG{o}{=} \PYG{n}{plt}\PYG{o}{.}\PYG{n}{hist}\PYG{p}{(}\PYG{n}{s}\PYG{p}{,} \PYG{n}{bins}\PYG{o}{=}\PYG{l+m+mi}{50}\PYG{p}{)}
\end{Verbatim}

\#   plot against distribution

\begin{Verbatim}[commandchars=\\\{\}]
\PYG{g+gp}{\PYGZgt{}\PYGZgt{}\PYGZgt{} }\PYG{k}{def} \PYG{n+nf}{logist}\PYG{p}{(}\PYG{n}{x}\PYG{p}{,} \PYG{n}{loc}\PYG{p}{,} \PYG{n}{scale}\PYG{p}{)}\PYG{p}{:}
\PYG{g+gp}{... }    \PYG{k}{return} \PYG{n}{exp}\PYG{p}{(}\PYG{p}{(}\PYG{n}{loc}\PYG{o}{\PYGZhy{}}\PYG{n}{x}\PYG{p}{)}\PYG{o}{/}\PYG{n}{scale}\PYG{p}{)}\PYG{o}{/}\PYG{p}{(}\PYG{n}{scale}\PYG{o}{*}\PYG{p}{(}\PYG{l+m+mi}{1}\PYG{o}{+}\PYG{n}{exp}\PYG{p}{(}\PYG{p}{(}\PYG{n}{loc}\PYG{o}{\PYGZhy{}}\PYG{n}{x}\PYG{p}{)}\PYG{o}{/}\PYG{n}{scale}\PYG{p}{)}\PYG{p}{)}\PYG{o}{*}\PYG{o}{*}\PYG{l+m+mi}{2}\PYG{p}{)}
\PYG{g+gp}{\PYGZgt{}\PYGZgt{}\PYGZgt{} }\PYG{n}{plt}\PYG{o}{.}\PYG{n}{plot}\PYG{p}{(}\PYG{n}{bins}\PYG{p}{,} \PYG{n}{logist}\PYG{p}{(}\PYG{n}{bins}\PYG{p}{,} \PYG{n}{loc}\PYG{p}{,} \PYG{n}{scale}\PYG{p}{)}\PYG{o}{*}\PYG{n}{count}\PYG{o}{.}\PYG{n}{max}\PYG{p}{(}\PYG{p}{)}\PYG{o}{/}\PYGZbs{}
\PYG{g+gp}{... }\PYG{n}{logist}\PYG{p}{(}\PYG{n}{bins}\PYG{p}{,} \PYG{n}{loc}\PYG{p}{,} \PYG{n}{scale}\PYG{p}{)}\PYG{o}{.}\PYG{n}{max}\PYG{p}{(}\PYG{p}{)}\PYG{p}{)}
\PYG{g+gp}{\PYGZgt{}\PYGZgt{}\PYGZgt{} }\PYG{n}{plt}\PYG{o}{.}\PYG{n}{show}\PYG{p}{(}\PYG{p}{)}
\end{Verbatim}

\end{fulllineitems}

\index{lognormal() (in module acsStatesAnalysis)}

\begin{fulllineitems}
\phantomsection\label{acsStatesAnalysis:acsStatesAnalysis.lognormal}\pysiglinewithargsret{\code{acsStatesAnalysis.}\bfcode{lognormal}}{\emph{mean=0.0}, \emph{sigma=1.0}, \emph{size=None}}{}
Return samples drawn from a log-normal distribution.

Draw samples from a log-normal distribution with specified mean,
standard deviation, and array shape.  Note that the mean and standard
deviation are not the values for the distribution itself, but of the
underlying normal distribution it is derived from.
\begin{description}
\item[{mean}] \leavevmode{[}float{]}
Mean value of the underlying normal distribution

\item[{sigma}] \leavevmode{[}float, \textgreater{} 0.{]}
Standard deviation of the underlying normal distribution

\item[{size}] \leavevmode{[}tuple of ints{]}
Output shape.  If the given shape is, e.g., \code{(m, n, k)}, then
\code{m * n * k} samples are drawn.

\end{description}
\begin{description}
\item[{samples}] \leavevmode{[}ndarray or float{]}
The desired samples. An array of the same shape as \emph{size} if given,
if \emph{size} is None a float is returned.

\end{description}
\begin{description}
\item[{scipy.stats.lognorm}] \leavevmode{[}probability density function, distribution,{]}
cumulative density function, etc.

\end{description}

A variable \emph{x} has a log-normal distribution if \emph{log(x)} is normally
distributed.  The probability density function for the log-normal
distribution is:
\begin{gather}
\begin{split}p(x) = \frac{1}{\sigma x \sqrt{2\pi}}
e^{(-\frac{(ln(x)-\mu)^2}{2\sigma^2})}\end{split}\notag
\end{gather}
where \(\mu\) is the mean and \(\sigma\) is the standard
deviation of the normally distributed logarithm of the variable.
A log-normal distribution results if a random variable is the \emph{product}
of a large number of independent, identically-distributed variables in
the same way that a normal distribution results if the variable is the
\emph{sum} of a large number of independent, identically-distributed
variables.

Limpert, E., Stahel, W. A., and Abbt, M., ``Log-normal Distributions
across the Sciences: Keys and Clues,'' \emph{BioScience}, Vol. 51, No. 5,
May, 2001.  \href{http://stat.ethz.ch/~stahel/lognormal/bioscience.pdf}{http://stat.ethz.ch/\textasciitilde{}stahel/lognormal/bioscience.pdf}

Reiss, R.D. and Thomas, M., \emph{Statistical Analysis of Extreme Values},
Basel: Birkhauser Verlag, 2001, pp. 31-32.

Draw samples from the distribution:

\begin{Verbatim}[commandchars=\\\{\}]
\PYG{g+gp}{\PYGZgt{}\PYGZgt{}\PYGZgt{} }\PYG{n}{mu}\PYG{p}{,} \PYG{n}{sigma} \PYG{o}{=} \PYG{l+m+mf}{3.}\PYG{p}{,} \PYG{l+m+mf}{1.} \PYG{c}{\PYGZsh{} mean and standard deviation}
\PYG{g+gp}{\PYGZgt{}\PYGZgt{}\PYGZgt{} }\PYG{n}{s} \PYG{o}{=} \PYG{n}{np}\PYG{o}{.}\PYG{n}{random}\PYG{o}{.}\PYG{n}{lognormal}\PYG{p}{(}\PYG{n}{mu}\PYG{p}{,} \PYG{n}{sigma}\PYG{p}{,} \PYG{l+m+mi}{1000}\PYG{p}{)}
\end{Verbatim}

Display the histogram of the samples, along with
the probability density function:

\begin{Verbatim}[commandchars=\\\{\}]
\PYG{g+gp}{\PYGZgt{}\PYGZgt{}\PYGZgt{} }\PYG{k+kn}{import} \PYG{n+nn}{matplotlib.pyplot} \PYG{k+kn}{as} \PYG{n+nn}{plt}
\PYG{g+gp}{\PYGZgt{}\PYGZgt{}\PYGZgt{} }\PYG{n}{count}\PYG{p}{,} \PYG{n}{bins}\PYG{p}{,} \PYG{n}{ignored} \PYG{o}{=} \PYG{n}{plt}\PYG{o}{.}\PYG{n}{hist}\PYG{p}{(}\PYG{n}{s}\PYG{p}{,} \PYG{l+m+mi}{100}\PYG{p}{,} \PYG{n}{normed}\PYG{o}{=}\PYG{n+nb+bp}{True}\PYG{p}{,} \PYG{n}{align}\PYG{o}{=}\PYG{l+s}{\PYGZsq{}}\PYG{l+s}{mid}\PYG{l+s}{\PYGZsq{}}\PYG{p}{)}
\end{Verbatim}

\begin{Verbatim}[commandchars=\\\{\}]
\PYG{g+gp}{\PYGZgt{}\PYGZgt{}\PYGZgt{} }\PYG{n}{x} \PYG{o}{=} \PYG{n}{np}\PYG{o}{.}\PYG{n}{linspace}\PYG{p}{(}\PYG{n+nb}{min}\PYG{p}{(}\PYG{n}{bins}\PYG{p}{)}\PYG{p}{,} \PYG{n+nb}{max}\PYG{p}{(}\PYG{n}{bins}\PYG{p}{)}\PYG{p}{,} \PYG{l+m+mi}{10000}\PYG{p}{)}
\PYG{g+gp}{\PYGZgt{}\PYGZgt{}\PYGZgt{} }\PYG{n}{pdf} \PYG{o}{=} \PYG{p}{(}\PYG{n}{np}\PYG{o}{.}\PYG{n}{exp}\PYG{p}{(}\PYG{o}{\PYGZhy{}}\PYG{p}{(}\PYG{n}{np}\PYG{o}{.}\PYG{n}{log}\PYG{p}{(}\PYG{n}{x}\PYG{p}{)} \PYG{o}{\PYGZhy{}} \PYG{n}{mu}\PYG{p}{)}\PYG{o}{*}\PYG{o}{*}\PYG{l+m+mi}{2} \PYG{o}{/} \PYG{p}{(}\PYG{l+m+mi}{2} \PYG{o}{*} \PYG{n}{sigma}\PYG{o}{*}\PYG{o}{*}\PYG{l+m+mi}{2}\PYG{p}{)}\PYG{p}{)}
\PYG{g+gp}{... }       \PYG{o}{/} \PYG{p}{(}\PYG{n}{x} \PYG{o}{*} \PYG{n}{sigma} \PYG{o}{*} \PYG{n}{np}\PYG{o}{.}\PYG{n}{sqrt}\PYG{p}{(}\PYG{l+m+mi}{2} \PYG{o}{*} \PYG{n}{np}\PYG{o}{.}\PYG{n}{pi}\PYG{p}{)}\PYG{p}{)}\PYG{p}{)}
\end{Verbatim}

\begin{Verbatim}[commandchars=\\\{\}]
\PYG{g+gp}{\PYGZgt{}\PYGZgt{}\PYGZgt{} }\PYG{n}{plt}\PYG{o}{.}\PYG{n}{plot}\PYG{p}{(}\PYG{n}{x}\PYG{p}{,} \PYG{n}{pdf}\PYG{p}{,} \PYG{n}{linewidth}\PYG{o}{=}\PYG{l+m+mi}{2}\PYG{p}{,} \PYG{n}{color}\PYG{o}{=}\PYG{l+s}{\PYGZsq{}}\PYG{l+s}{r}\PYG{l+s}{\PYGZsq{}}\PYG{p}{)}
\PYG{g+gp}{\PYGZgt{}\PYGZgt{}\PYGZgt{} }\PYG{n}{plt}\PYG{o}{.}\PYG{n}{axis}\PYG{p}{(}\PYG{l+s}{\PYGZsq{}}\PYG{l+s}{tight}\PYG{l+s}{\PYGZsq{}}\PYG{p}{)}
\PYG{g+gp}{\PYGZgt{}\PYGZgt{}\PYGZgt{} }\PYG{n}{plt}\PYG{o}{.}\PYG{n}{show}\PYG{p}{(}\PYG{p}{)}
\end{Verbatim}

Demonstrate that taking the products of random samples from a uniform
distribution can be fit well by a log-normal probability density function.

\begin{Verbatim}[commandchars=\\\{\}]
\PYG{g+gp}{\PYGZgt{}\PYGZgt{}\PYGZgt{} }\PYG{c}{\PYGZsh{} Generate a thousand samples: each is the product of 100 random}
\PYG{g+gp}{\PYGZgt{}\PYGZgt{}\PYGZgt{} }\PYG{c}{\PYGZsh{} values, drawn from a normal distribution.}
\PYG{g+gp}{\PYGZgt{}\PYGZgt{}\PYGZgt{} }\PYG{n}{b} \PYG{o}{=} \PYG{p}{[}\PYG{p}{]}
\PYG{g+gp}{\PYGZgt{}\PYGZgt{}\PYGZgt{} }\PYG{k}{for} \PYG{n}{i} \PYG{o+ow}{in} \PYG{n+nb}{range}\PYG{p}{(}\PYG{l+m+mi}{1000}\PYG{p}{)}\PYG{p}{:}
\PYG{g+gp}{... }   \PYG{n}{a} \PYG{o}{=} \PYG{l+m+mf}{10.} \PYG{o}{+} \PYG{n}{np}\PYG{o}{.}\PYG{n}{random}\PYG{o}{.}\PYG{n}{random}\PYG{p}{(}\PYG{l+m+mi}{100}\PYG{p}{)}
\PYG{g+gp}{... }   \PYG{n}{b}\PYG{o}{.}\PYG{n}{append}\PYG{p}{(}\PYG{n}{np}\PYG{o}{.}\PYG{n}{product}\PYG{p}{(}\PYG{n}{a}\PYG{p}{)}\PYG{p}{)}
\end{Verbatim}

\begin{Verbatim}[commandchars=\\\{\}]
\PYG{g+gp}{\PYGZgt{}\PYGZgt{}\PYGZgt{} }\PYG{n}{b} \PYG{o}{=} \PYG{n}{np}\PYG{o}{.}\PYG{n}{array}\PYG{p}{(}\PYG{n}{b}\PYG{p}{)} \PYG{o}{/} \PYG{n}{np}\PYG{o}{.}\PYG{n}{min}\PYG{p}{(}\PYG{n}{b}\PYG{p}{)} \PYG{c}{\PYGZsh{} scale values to be positive}
\PYG{g+gp}{\PYGZgt{}\PYGZgt{}\PYGZgt{} }\PYG{n}{count}\PYG{p}{,} \PYG{n}{bins}\PYG{p}{,} \PYG{n}{ignored} \PYG{o}{=} \PYG{n}{plt}\PYG{o}{.}\PYG{n}{hist}\PYG{p}{(}\PYG{n}{b}\PYG{p}{,} \PYG{l+m+mi}{100}\PYG{p}{,} \PYG{n}{normed}\PYG{o}{=}\PYG{n+nb+bp}{True}\PYG{p}{,} \PYG{n}{align}\PYG{o}{=}\PYG{l+s}{\PYGZsq{}}\PYG{l+s}{center}\PYG{l+s}{\PYGZsq{}}\PYG{p}{)}
\PYG{g+gp}{\PYGZgt{}\PYGZgt{}\PYGZgt{} }\PYG{n}{sigma} \PYG{o}{=} \PYG{n}{np}\PYG{o}{.}\PYG{n}{std}\PYG{p}{(}\PYG{n}{np}\PYG{o}{.}\PYG{n}{log}\PYG{p}{(}\PYG{n}{b}\PYG{p}{)}\PYG{p}{)}
\PYG{g+gp}{\PYGZgt{}\PYGZgt{}\PYGZgt{} }\PYG{n}{mu} \PYG{o}{=} \PYG{n}{np}\PYG{o}{.}\PYG{n}{mean}\PYG{p}{(}\PYG{n}{np}\PYG{o}{.}\PYG{n}{log}\PYG{p}{(}\PYG{n}{b}\PYG{p}{)}\PYG{p}{)}
\end{Verbatim}

\begin{Verbatim}[commandchars=\\\{\}]
\PYG{g+gp}{\PYGZgt{}\PYGZgt{}\PYGZgt{} }\PYG{n}{x} \PYG{o}{=} \PYG{n}{np}\PYG{o}{.}\PYG{n}{linspace}\PYG{p}{(}\PYG{n+nb}{min}\PYG{p}{(}\PYG{n}{bins}\PYG{p}{)}\PYG{p}{,} \PYG{n+nb}{max}\PYG{p}{(}\PYG{n}{bins}\PYG{p}{)}\PYG{p}{,} \PYG{l+m+mi}{10000}\PYG{p}{)}
\PYG{g+gp}{\PYGZgt{}\PYGZgt{}\PYGZgt{} }\PYG{n}{pdf} \PYG{o}{=} \PYG{p}{(}\PYG{n}{np}\PYG{o}{.}\PYG{n}{exp}\PYG{p}{(}\PYG{o}{\PYGZhy{}}\PYG{p}{(}\PYG{n}{np}\PYG{o}{.}\PYG{n}{log}\PYG{p}{(}\PYG{n}{x}\PYG{p}{)} \PYG{o}{\PYGZhy{}} \PYG{n}{mu}\PYG{p}{)}\PYG{o}{*}\PYG{o}{*}\PYG{l+m+mi}{2} \PYG{o}{/} \PYG{p}{(}\PYG{l+m+mi}{2} \PYG{o}{*} \PYG{n}{sigma}\PYG{o}{*}\PYG{o}{*}\PYG{l+m+mi}{2}\PYG{p}{)}\PYG{p}{)}
\PYG{g+gp}{... }       \PYG{o}{/} \PYG{p}{(}\PYG{n}{x} \PYG{o}{*} \PYG{n}{sigma} \PYG{o}{*} \PYG{n}{np}\PYG{o}{.}\PYG{n}{sqrt}\PYG{p}{(}\PYG{l+m+mi}{2} \PYG{o}{*} \PYG{n}{np}\PYG{o}{.}\PYG{n}{pi}\PYG{p}{)}\PYG{p}{)}\PYG{p}{)}
\end{Verbatim}

\begin{Verbatim}[commandchars=\\\{\}]
\PYG{g+gp}{\PYGZgt{}\PYGZgt{}\PYGZgt{} }\PYG{n}{plt}\PYG{o}{.}\PYG{n}{plot}\PYG{p}{(}\PYG{n}{x}\PYG{p}{,} \PYG{n}{pdf}\PYG{p}{,} \PYG{n}{color}\PYG{o}{=}\PYG{l+s}{\PYGZsq{}}\PYG{l+s}{r}\PYG{l+s}{\PYGZsq{}}\PYG{p}{,} \PYG{n}{linewidth}\PYG{o}{=}\PYG{l+m+mi}{2}\PYG{p}{)}
\PYG{g+gp}{\PYGZgt{}\PYGZgt{}\PYGZgt{} }\PYG{n}{plt}\PYG{o}{.}\PYG{n}{show}\PYG{p}{(}\PYG{p}{)}
\end{Verbatim}

\end{fulllineitems}

\index{logseries() (in module acsStatesAnalysis)}

\begin{fulllineitems}
\phantomsection\label{acsStatesAnalysis:acsStatesAnalysis.logseries}\pysiglinewithargsret{\code{acsStatesAnalysis.}\bfcode{logseries}}{\emph{p}, \emph{size=None}}{}
Draw samples from a Logarithmic Series distribution.

Samples are drawn from a Log Series distribution with specified
parameter, p (probability, 0 \textless{} p \textless{} 1).

loc : float

scale : float \textgreater{} 0.
\begin{description}
\item[{size}] \leavevmode{[}\{tuple, int\}{]}
Output shape.  If the given shape is, e.g., \code{(m, n, k)}, then
\code{m * n * k} samples are drawn.

\end{description}
\begin{description}
\item[{samples}] \leavevmode{[}\{ndarray, scalar\}{]}
where the values are all integers in  {[}0, n{]}.

\end{description}
\begin{description}
\item[{scipy.stats.distributions.logser}] \leavevmode{[}probability density function,{]}
distribution or cumulative density function, etc.

\end{description}

The probability density for the Log Series distribution is
\begin{gather}
\begin{split}P(k) = \frac{-p^k}{k \ln(1-p)},\end{split}\notag
\end{gather}
where p = probability.

The Log Series distribution is frequently used to represent species
richness and occurrence, first proposed by Fisher, Corbet, and
Williams in 1943 {[}2{]}.  It may also be used to model the numbers of
occupants seen in cars {[}3{]}.

Draw samples from the distribution:

\begin{Verbatim}[commandchars=\\\{\}]
\PYG{g+gp}{\PYGZgt{}\PYGZgt{}\PYGZgt{} }\PYG{n}{a} \PYG{o}{=} \PYG{o}{.}\PYG{l+m+mi}{6}
\PYG{g+gp}{\PYGZgt{}\PYGZgt{}\PYGZgt{} }\PYG{n}{s} \PYG{o}{=} \PYG{n}{np}\PYG{o}{.}\PYG{n}{random}\PYG{o}{.}\PYG{n}{logseries}\PYG{p}{(}\PYG{n}{a}\PYG{p}{,} \PYG{l+m+mi}{10000}\PYG{p}{)}
\PYG{g+gp}{\PYGZgt{}\PYGZgt{}\PYGZgt{} }\PYG{n}{count}\PYG{p}{,} \PYG{n}{bins}\PYG{p}{,} \PYG{n}{ignored} \PYG{o}{=} \PYG{n}{plt}\PYG{o}{.}\PYG{n}{hist}\PYG{p}{(}\PYG{n}{s}\PYG{p}{)}
\end{Verbatim}

\#   plot against distribution

\begin{Verbatim}[commandchars=\\\{\}]
\PYG{g+gp}{\PYGZgt{}\PYGZgt{}\PYGZgt{} }\PYG{k}{def} \PYG{n+nf}{logseries}\PYG{p}{(}\PYG{n}{k}\PYG{p}{,} \PYG{n}{p}\PYG{p}{)}\PYG{p}{:}
\PYG{g+gp}{... }    \PYG{k}{return} \PYG{o}{\PYGZhy{}}\PYG{n}{p}\PYG{o}{*}\PYG{o}{*}\PYG{n}{k}\PYG{o}{/}\PYG{p}{(}\PYG{n}{k}\PYG{o}{*}\PYG{n}{log}\PYG{p}{(}\PYG{l+m+mi}{1}\PYG{o}{\PYGZhy{}}\PYG{n}{p}\PYG{p}{)}\PYG{p}{)}
\PYG{g+gp}{\PYGZgt{}\PYGZgt{}\PYGZgt{} }\PYG{n}{plt}\PYG{o}{.}\PYG{n}{plot}\PYG{p}{(}\PYG{n}{bins}\PYG{p}{,} \PYG{n}{logseries}\PYG{p}{(}\PYG{n}{bins}\PYG{p}{,} \PYG{n}{a}\PYG{p}{)}\PYG{o}{*}\PYG{n}{count}\PYG{o}{.}\PYG{n}{max}\PYG{p}{(}\PYG{p}{)}\PYG{o}{/}
\PYG{g+go}{             logseries(bins, a).max(), \PYGZsq{}r\PYGZsq{})}
\PYG{g+gp}{\PYGZgt{}\PYGZgt{}\PYGZgt{} }\PYG{n}{plt}\PYG{o}{.}\PYG{n}{show}\PYG{p}{(}\PYG{p}{)}
\end{Verbatim}

\end{fulllineitems}

\index{multinomial() (in module acsStatesAnalysis)}

\begin{fulllineitems}
\phantomsection\label{acsStatesAnalysis:acsStatesAnalysis.multinomial}\pysiglinewithargsret{\code{acsStatesAnalysis.}\bfcode{multinomial}}{\emph{n}, \emph{pvals}, \emph{size=None}}{}
Draw samples from a multinomial distribution.

The multinomial distribution is a multivariate generalisation of the
binomial distribution.  Take an experiment with one of \code{p}
possible outcomes.  An example of such an experiment is throwing a dice,
where the outcome can be 1 through 6.  Each sample drawn from the
distribution represents \emph{n} such experiments.  Its values,
\code{X\_i = {[}X\_0, X\_1, ..., X\_p{]}}, represent the number of times the outcome
was \code{i}.
\begin{description}
\item[{n}] \leavevmode{[}int{]}
Number of experiments.

\item[{pvals}] \leavevmode{[}sequence of floats, length p{]}
Probabilities of each of the \code{p} different outcomes.  These
should sum to 1 (however, the last element is always assumed to
account for the remaining probability, as long as
\code{sum(pvals{[}:-1{]}) \textless{}= 1)}.

\item[{size}] \leavevmode{[}tuple of ints{]}
Given a \emph{size} of \code{(M, N, K)}, then \code{M*N*K} samples are drawn,
and the output shape becomes \code{(M, N, K, p)}, since each sample
has shape \code{(p,)}.

\end{description}

Throw a dice 20 times:

\begin{Verbatim}[commandchars=\\\{\}]
\PYG{g+gp}{\PYGZgt{}\PYGZgt{}\PYGZgt{} }\PYG{n}{np}\PYG{o}{.}\PYG{n}{random}\PYG{o}{.}\PYG{n}{multinomial}\PYG{p}{(}\PYG{l+m+mi}{20}\PYG{p}{,} \PYG{p}{[}\PYG{l+m+mi}{1}\PYG{o}{/}\PYG{l+m+mf}{6.}\PYG{p}{]}\PYG{o}{*}\PYG{l+m+mi}{6}\PYG{p}{,} \PYG{n}{size}\PYG{o}{=}\PYG{l+m+mi}{1}\PYG{p}{)}
\PYG{g+go}{array([[4, 1, 7, 5, 2, 1]])}
\end{Verbatim}

It landed 4 times on 1, once on 2, etc.

Now, throw the dice 20 times, and 20 times again:

\begin{Verbatim}[commandchars=\\\{\}]
\PYG{g+gp}{\PYGZgt{}\PYGZgt{}\PYGZgt{} }\PYG{n}{np}\PYG{o}{.}\PYG{n}{random}\PYG{o}{.}\PYG{n}{multinomial}\PYG{p}{(}\PYG{l+m+mi}{20}\PYG{p}{,} \PYG{p}{[}\PYG{l+m+mi}{1}\PYG{o}{/}\PYG{l+m+mf}{6.}\PYG{p}{]}\PYG{o}{*}\PYG{l+m+mi}{6}\PYG{p}{,} \PYG{n}{size}\PYG{o}{=}\PYG{l+m+mi}{2}\PYG{p}{)}
\PYG{g+go}{array([[3, 4, 3, 3, 4, 3],}
\PYG{g+go}{       [2, 4, 3, 4, 0, 7]])}
\end{Verbatim}

For the first run, we threw 3 times 1, 4 times 2, etc.  For the second,
we threw 2 times 1, 4 times 2, etc.

A loaded dice is more likely to land on number 6:

\begin{Verbatim}[commandchars=\\\{\}]
\PYG{g+gp}{\PYGZgt{}\PYGZgt{}\PYGZgt{} }\PYG{n}{np}\PYG{o}{.}\PYG{n}{random}\PYG{o}{.}\PYG{n}{multinomial}\PYG{p}{(}\PYG{l+m+mi}{100}\PYG{p}{,} \PYG{p}{[}\PYG{l+m+mi}{1}\PYG{o}{/}\PYG{l+m+mf}{7.}\PYG{p}{]}\PYG{o}{*}\PYG{l+m+mi}{5}\PYG{p}{)}
\PYG{g+go}{array([13, 16, 13, 16, 42])}
\end{Verbatim}

\end{fulllineitems}

\index{multivariate\_normal() (in module acsStatesAnalysis)}

\begin{fulllineitems}
\phantomsection\label{acsStatesAnalysis:acsStatesAnalysis.multivariate_normal}\pysiglinewithargsret{\code{acsStatesAnalysis.}\bfcode{multivariate\_normal}}{\emph{mean}, \emph{cov}\optional{, \emph{size}}}{}
Draw random samples from a multivariate normal distribution.

The multivariate normal, multinormal or Gaussian distribution is a
generalization of the one-dimensional normal distribution to higher
dimensions.  Such a distribution is specified by its mean and
covariance matrix.  These parameters are analogous to the mean
(average or ``center'') and variance (standard deviation, or ``width,''
squared) of the one-dimensional normal distribution.
\begin{description}
\item[{mean}] \leavevmode{[}1-D array\_like, of length N{]}
Mean of the N-dimensional distribution.

\item[{cov}] \leavevmode{[}2-D array\_like, of shape (N, N){]}
Covariance matrix of the distribution.  Must be symmetric and
positive semi-definite for ``physically meaningful'' results.

\item[{size}] \leavevmode{[}int or tuple of ints, optional{]}
Given a shape of, for example, \code{(m,n,k)}, \code{m*n*k} samples are
generated, and packed in an \emph{m}-by-\emph{n}-by-\emph{k} arrangement.  Because
each sample is \emph{N}-dimensional, the output shape is \code{(m,n,k,N)}.
If no shape is specified, a single (\emph{N}-D) sample is returned.

\end{description}
\begin{description}
\item[{out}] \leavevmode{[}ndarray{]}
The drawn samples, of shape \emph{size}, if that was provided.  If not,
the shape is \code{(N,)}.

In other words, each entry \code{out{[}i,j,...,:{]}} is an N-dimensional
value drawn from the distribution.

\end{description}

The mean is a coordinate in N-dimensional space, which represents the
location where samples are most likely to be generated.  This is
analogous to the peak of the bell curve for the one-dimensional or
univariate normal distribution.

Covariance indicates the level to which two variables vary together.
From the multivariate normal distribution, we draw N-dimensional
samples, \(X = [x_1, x_2, ... x_N]\).  The covariance matrix
element \(C_{ij}\) is the covariance of \(x_i\) and \(x_j\).
The element \(C_{ii}\) is the variance of \(x_i\) (i.e. its
``spread'').

Instead of specifying the full covariance matrix, popular
approximations include:
\begin{itemize}
\item {} 
Spherical covariance (\emph{cov} is a multiple of the identity matrix)

\item {} 
Diagonal covariance (\emph{cov} has non-negative elements, and only on
the diagonal)

\end{itemize}

This geometrical property can be seen in two dimensions by plotting
generated data-points:

\begin{Verbatim}[commandchars=\\\{\}]
\PYG{g+gp}{\PYGZgt{}\PYGZgt{}\PYGZgt{} }\PYG{n}{mean} \PYG{o}{=} \PYG{p}{[}\PYG{l+m+mi}{0}\PYG{p}{,}\PYG{l+m+mi}{0}\PYG{p}{]}
\PYG{g+gp}{\PYGZgt{}\PYGZgt{}\PYGZgt{} }\PYG{n}{cov} \PYG{o}{=} \PYG{p}{[}\PYG{p}{[}\PYG{l+m+mi}{1}\PYG{p}{,}\PYG{l+m+mi}{0}\PYG{p}{]}\PYG{p}{,}\PYG{p}{[}\PYG{l+m+mi}{0}\PYG{p}{,}\PYG{l+m+mi}{100}\PYG{p}{]}\PYG{p}{]} \PYG{c}{\PYGZsh{} diagonal covariance, points lie on x or y\PYGZhy{}axis}
\end{Verbatim}

\begin{Verbatim}[commandchars=\\\{\}]
\PYG{g+gp}{\PYGZgt{}\PYGZgt{}\PYGZgt{} }\PYG{k+kn}{import} \PYG{n+nn}{matplotlib.pyplot} \PYG{k+kn}{as} \PYG{n+nn}{plt}
\PYG{g+gp}{\PYGZgt{}\PYGZgt{}\PYGZgt{} }\PYG{n}{x}\PYG{p}{,}\PYG{n}{y} \PYG{o}{=} \PYG{n}{np}\PYG{o}{.}\PYG{n}{random}\PYG{o}{.}\PYG{n}{multivariate\PYGZus{}normal}\PYG{p}{(}\PYG{n}{mean}\PYG{p}{,}\PYG{n}{cov}\PYG{p}{,}\PYG{l+m+mi}{5000}\PYG{p}{)}\PYG{o}{.}\PYG{n}{T}
\PYG{g+gp}{\PYGZgt{}\PYGZgt{}\PYGZgt{} }\PYG{n}{plt}\PYG{o}{.}\PYG{n}{plot}\PYG{p}{(}\PYG{n}{x}\PYG{p}{,}\PYG{n}{y}\PYG{p}{,}\PYG{l+s}{\PYGZsq{}}\PYG{l+s}{x}\PYG{l+s}{\PYGZsq{}}\PYG{p}{)}\PYG{p}{;} \PYG{n}{plt}\PYG{o}{.}\PYG{n}{axis}\PYG{p}{(}\PYG{l+s}{\PYGZsq{}}\PYG{l+s}{equal}\PYG{l+s}{\PYGZsq{}}\PYG{p}{)}\PYG{p}{;} \PYG{n}{plt}\PYG{o}{.}\PYG{n}{show}\PYG{p}{(}\PYG{p}{)}
\end{Verbatim}

Note that the covariance matrix must be non-negative definite.

Papoulis, A., \emph{Probability, Random Variables, and Stochastic Processes},
3rd ed., New York: McGraw-Hill, 1991.

Duda, R. O., Hart, P. E., and Stork, D. G., \emph{Pattern Classification},
2nd ed., New York: Wiley, 2001.

\begin{Verbatim}[commandchars=\\\{\}]
\PYG{g+gp}{\PYGZgt{}\PYGZgt{}\PYGZgt{} }\PYG{n}{mean} \PYG{o}{=} \PYG{p}{(}\PYG{l+m+mi}{1}\PYG{p}{,}\PYG{l+m+mi}{2}\PYG{p}{)}
\PYG{g+gp}{\PYGZgt{}\PYGZgt{}\PYGZgt{} }\PYG{n}{cov} \PYG{o}{=} \PYG{p}{[}\PYG{p}{[}\PYG{l+m+mi}{1}\PYG{p}{,}\PYG{l+m+mi}{0}\PYG{p}{]}\PYG{p}{,}\PYG{p}{[}\PYG{l+m+mi}{1}\PYG{p}{,}\PYG{l+m+mi}{0}\PYG{p}{]}\PYG{p}{]}
\PYG{g+gp}{\PYGZgt{}\PYGZgt{}\PYGZgt{} }\PYG{n}{x} \PYG{o}{=} \PYG{n}{np}\PYG{o}{.}\PYG{n}{random}\PYG{o}{.}\PYG{n}{multivariate\PYGZus{}normal}\PYG{p}{(}\PYG{n}{mean}\PYG{p}{,}\PYG{n}{cov}\PYG{p}{,}\PYG{p}{(}\PYG{l+m+mi}{3}\PYG{p}{,}\PYG{l+m+mi}{3}\PYG{p}{)}\PYG{p}{)}
\PYG{g+gp}{\PYGZgt{}\PYGZgt{}\PYGZgt{} }\PYG{n}{x}\PYG{o}{.}\PYG{n}{shape}
\PYG{g+go}{(3, 3, 2)}
\end{Verbatim}

The following is probably true, given that 0.6 is roughly twice the
standard deviation:

\begin{Verbatim}[commandchars=\\\{\}]
\PYG{g+gp}{\PYGZgt{}\PYGZgt{}\PYGZgt{} }\PYG{k}{print} \PYG{n+nb}{list}\PYG{p}{(} \PYG{p}{(}\PYG{n}{x}\PYG{p}{[}\PYG{l+m+mi}{0}\PYG{p}{,}\PYG{l+m+mi}{0}\PYG{p}{,}\PYG{p}{:}\PYG{p}{]} \PYG{o}{\PYGZhy{}} \PYG{n}{mean}\PYG{p}{)} \PYG{o}{\PYGZlt{}} \PYG{l+m+mf}{0.6} \PYG{p}{)}
\PYG{g+go}{[True, True]}
\end{Verbatim}

\end{fulllineitems}

\index{negative\_binomial() (in module acsStatesAnalysis)}

\begin{fulllineitems}
\phantomsection\label{acsStatesAnalysis:acsStatesAnalysis.negative_binomial}\pysiglinewithargsret{\code{acsStatesAnalysis.}\bfcode{negative\_binomial}}{\emph{n}, \emph{p}, \emph{size=None}}{}
Draw samples from a negative\_binomial distribution.

Samples are drawn from a negative\_Binomial distribution with specified
parameters, \emph{n} trials and \emph{p} probability of success where \emph{n} is an
integer \textgreater{} 0 and \emph{p} is in the interval {[}0, 1{]}.
\begin{description}
\item[{n}] \leavevmode{[}int{]}
Parameter, \textgreater{} 0.

\item[{p}] \leavevmode{[}float{]}
Parameter, \textgreater{}= 0 and \textless{}=1.

\item[{size}] \leavevmode{[}int or tuple of ints{]}
Output shape. If the given shape is, e.g., \code{(m, n, k)}, then
\code{m * n * k} samples are drawn.

\end{description}
\begin{description}
\item[{samples}] \leavevmode{[}int or ndarray of ints{]}
Drawn samples.

\end{description}

The probability density for the Negative Binomial distribution is
\begin{gather}
\begin{split}P(N;n,p) = \binom{N+n-1}{n-1}p^{n}(1-p)^{N},\end{split}\notag
\end{gather}
where \(n-1\) is the number of successes, \(p\) is the probability
of success, and \(N+n-1\) is the number of trials.

The negative binomial distribution gives the probability of n-1 successes
and N failures in N+n-1 trials, and success on the (N+n)th trial.

If one throws a die repeatedly until the third time a ``1'' appears, then the
probability distribution of the number of non-``1''s that appear before the
third ``1'' is a negative binomial distribution.

Draw samples from the distribution:

A real world example. A company drills wild-cat oil exploration wells, each
with an estimated probability of success of 0.1.  What is the probability
of having one success for each successive well, that is what is the
probability of a single success after drilling 5 wells, after 6 wells,
etc.?

\begin{Verbatim}[commandchars=\\\{\}]
\PYG{g+gp}{\PYGZgt{}\PYGZgt{}\PYGZgt{} }\PYG{n}{s} \PYG{o}{=} \PYG{n}{np}\PYG{o}{.}\PYG{n}{random}\PYG{o}{.}\PYG{n}{negative\PYGZus{}binomial}\PYG{p}{(}\PYG{l+m+mi}{1}\PYG{p}{,} \PYG{l+m+mf}{0.1}\PYG{p}{,} \PYG{l+m+mi}{100000}\PYG{p}{)}
\PYG{g+gp}{\PYGZgt{}\PYGZgt{}\PYGZgt{} }\PYG{k}{for} \PYG{n}{i} \PYG{o+ow}{in} \PYG{n+nb}{range}\PYG{p}{(}\PYG{l+m+mi}{1}\PYG{p}{,} \PYG{l+m+mi}{11}\PYG{p}{)}\PYG{p}{:}
\PYG{g+gp}{... }   \PYG{n}{probability} \PYG{o}{=} \PYG{n+nb}{sum}\PYG{p}{(}\PYG{n}{s}\PYG{o}{\PYGZlt{}}\PYG{n}{i}\PYG{p}{)} \PYG{o}{/} \PYG{l+m+mf}{100000.}
\PYG{g+gp}{... }   \PYG{k}{print} \PYG{n}{i}\PYG{p}{,} \PYG{l+s}{\PYGZdq{}}\PYG{l+s}{wells drilled, probability of one success =}\PYG{l+s}{\PYGZdq{}}\PYG{p}{,} \PYG{n}{probability}
\end{Verbatim}

\end{fulllineitems}

\index{noncentral\_chisquare() (in module acsStatesAnalysis)}

\begin{fulllineitems}
\phantomsection\label{acsStatesAnalysis:acsStatesAnalysis.noncentral_chisquare}\pysiglinewithargsret{\code{acsStatesAnalysis.}\bfcode{noncentral\_chisquare}}{\emph{df}, \emph{nonc}, \emph{size=None}}{}
Draw samples from a noncentral chi-square distribution.

The noncentral \(\chi^2\) distribution is a generalisation of
the \(\chi^2\) distribution.
\begin{description}
\item[{df}] \leavevmode{[}int{]}
Degrees of freedom, should be \textgreater{}= 1.

\item[{nonc}] \leavevmode{[}float{]}
Non-centrality, should be \textgreater{} 0.

\item[{size}] \leavevmode{[}int or tuple of ints{]}
Shape of the output.

\end{description}

The probability density function for the noncentral Chi-square distribution
is
\begin{gather}
\begin{split}P(x;df,nonc) = \sum^{\infty}_{i=0}
\frac{e^{-nonc/2}(nonc/2)^{i}}{i!}P_{Y_{df+2i}}(x),\end{split}\notag
\end{gather}
where \(Y_{q}\) is the Chi-square with q degrees of freedom.

In Delhi (2007), it is noted that the noncentral chi-square is useful in
bombing and coverage problems, the probability of killing the point target
given by the noncentral chi-squared distribution.

Draw values from the distribution and plot the histogram

\begin{Verbatim}[commandchars=\\\{\}]
\PYG{g+gp}{\PYGZgt{}\PYGZgt{}\PYGZgt{} }\PYG{k+kn}{import} \PYG{n+nn}{matplotlib.pyplot} \PYG{k+kn}{as} \PYG{n+nn}{plt}
\PYG{g+gp}{\PYGZgt{}\PYGZgt{}\PYGZgt{} }\PYG{n}{values} \PYG{o}{=} \PYG{n}{plt}\PYG{o}{.}\PYG{n}{hist}\PYG{p}{(}\PYG{n}{np}\PYG{o}{.}\PYG{n}{random}\PYG{o}{.}\PYG{n}{noncentral\PYGZus{}chisquare}\PYG{p}{(}\PYG{l+m+mi}{3}\PYG{p}{,} \PYG{l+m+mi}{20}\PYG{p}{,} \PYG{l+m+mi}{100000}\PYG{p}{)}\PYG{p}{,}
\PYG{g+gp}{... }                  \PYG{n}{bins}\PYG{o}{=}\PYG{l+m+mi}{200}\PYG{p}{,} \PYG{n}{normed}\PYG{o}{=}\PYG{n+nb+bp}{True}\PYG{p}{)}
\PYG{g+gp}{\PYGZgt{}\PYGZgt{}\PYGZgt{} }\PYG{n}{plt}\PYG{o}{.}\PYG{n}{show}\PYG{p}{(}\PYG{p}{)}
\end{Verbatim}

Draw values from a noncentral chisquare with very small noncentrality,
and compare to a chisquare.

\begin{Verbatim}[commandchars=\\\{\}]
\PYG{g+gp}{\PYGZgt{}\PYGZgt{}\PYGZgt{} }\PYG{n}{plt}\PYG{o}{.}\PYG{n}{figure}\PYG{p}{(}\PYG{p}{)}
\PYG{g+gp}{\PYGZgt{}\PYGZgt{}\PYGZgt{} }\PYG{n}{values} \PYG{o}{=} \PYG{n}{plt}\PYG{o}{.}\PYG{n}{hist}\PYG{p}{(}\PYG{n}{np}\PYG{o}{.}\PYG{n}{random}\PYG{o}{.}\PYG{n}{noncentral\PYGZus{}chisquare}\PYG{p}{(}\PYG{l+m+mi}{3}\PYG{p}{,} \PYG{o}{.}\PYG{l+m+mo}{0000001}\PYG{p}{,} \PYG{l+m+mi}{100000}\PYG{p}{)}\PYG{p}{,}
\PYG{g+gp}{... }                  \PYG{n}{bins}\PYG{o}{=}\PYG{n}{np}\PYG{o}{.}\PYG{n}{arange}\PYG{p}{(}\PYG{l+m+mf}{0.}\PYG{p}{,} \PYG{l+m+mi}{25}\PYG{p}{,} \PYG{o}{.}\PYG{l+m+mi}{1}\PYG{p}{)}\PYG{p}{,} \PYG{n}{normed}\PYG{o}{=}\PYG{n+nb+bp}{True}\PYG{p}{)}
\PYG{g+gp}{\PYGZgt{}\PYGZgt{}\PYGZgt{} }\PYG{n}{values2} \PYG{o}{=} \PYG{n}{plt}\PYG{o}{.}\PYG{n}{hist}\PYG{p}{(}\PYG{n}{np}\PYG{o}{.}\PYG{n}{random}\PYG{o}{.}\PYG{n}{chisquare}\PYG{p}{(}\PYG{l+m+mi}{3}\PYG{p}{,} \PYG{l+m+mi}{100000}\PYG{p}{)}\PYG{p}{,}
\PYG{g+gp}{... }                   \PYG{n}{bins}\PYG{o}{=}\PYG{n}{np}\PYG{o}{.}\PYG{n}{arange}\PYG{p}{(}\PYG{l+m+mf}{0.}\PYG{p}{,} \PYG{l+m+mi}{25}\PYG{p}{,} \PYG{o}{.}\PYG{l+m+mi}{1}\PYG{p}{)}\PYG{p}{,} \PYG{n}{normed}\PYG{o}{=}\PYG{n+nb+bp}{True}\PYG{p}{)}
\PYG{g+gp}{\PYGZgt{}\PYGZgt{}\PYGZgt{} }\PYG{n}{plt}\PYG{o}{.}\PYG{n}{plot}\PYG{p}{(}\PYG{n}{values}\PYG{p}{[}\PYG{l+m+mi}{1}\PYG{p}{]}\PYG{p}{[}\PYG{l+m+mi}{0}\PYG{p}{:}\PYG{o}{\PYGZhy{}}\PYG{l+m+mi}{1}\PYG{p}{]}\PYG{p}{,} \PYG{n}{values}\PYG{p}{[}\PYG{l+m+mi}{0}\PYG{p}{]}\PYG{o}{\PYGZhy{}}\PYG{n}{values2}\PYG{p}{[}\PYG{l+m+mi}{0}\PYG{p}{]}\PYG{p}{,} \PYG{l+s}{\PYGZsq{}}\PYG{l+s}{ob}\PYG{l+s}{\PYGZsq{}}\PYG{p}{)}
\PYG{g+gp}{\PYGZgt{}\PYGZgt{}\PYGZgt{} }\PYG{n}{plt}\PYG{o}{.}\PYG{n}{show}\PYG{p}{(}\PYG{p}{)}
\end{Verbatim}

Demonstrate how large values of non-centrality lead to a more symmetric
distribution.

\begin{Verbatim}[commandchars=\\\{\}]
\PYG{g+gp}{\PYGZgt{}\PYGZgt{}\PYGZgt{} }\PYG{n}{plt}\PYG{o}{.}\PYG{n}{figure}\PYG{p}{(}\PYG{p}{)}
\PYG{g+gp}{\PYGZgt{}\PYGZgt{}\PYGZgt{} }\PYG{n}{values} \PYG{o}{=} \PYG{n}{plt}\PYG{o}{.}\PYG{n}{hist}\PYG{p}{(}\PYG{n}{np}\PYG{o}{.}\PYG{n}{random}\PYG{o}{.}\PYG{n}{noncentral\PYGZus{}chisquare}\PYG{p}{(}\PYG{l+m+mi}{3}\PYG{p}{,} \PYG{l+m+mi}{20}\PYG{p}{,} \PYG{l+m+mi}{100000}\PYG{p}{)}\PYG{p}{,}
\PYG{g+gp}{... }                  \PYG{n}{bins}\PYG{o}{=}\PYG{l+m+mi}{200}\PYG{p}{,} \PYG{n}{normed}\PYG{o}{=}\PYG{n+nb+bp}{True}\PYG{p}{)}
\PYG{g+gp}{\PYGZgt{}\PYGZgt{}\PYGZgt{} }\PYG{n}{plt}\PYG{o}{.}\PYG{n}{show}\PYG{p}{(}\PYG{p}{)}
\end{Verbatim}

\end{fulllineitems}

\index{noncentral\_f() (in module acsStatesAnalysis)}

\begin{fulllineitems}
\phantomsection\label{acsStatesAnalysis:acsStatesAnalysis.noncentral_f}\pysiglinewithargsret{\code{acsStatesAnalysis.}\bfcode{noncentral\_f}}{\emph{dfnum}, \emph{dfden}, \emph{nonc}, \emph{size=None}}{}
Draw samples from the noncentral F distribution.

Samples are drawn from an F distribution with specified parameters,
\emph{dfnum} (degrees of freedom in numerator) and \emph{dfden} (degrees of
freedom in denominator), where both parameters \textgreater{} 1.
\emph{nonc} is the non-centrality parameter.
\begin{description}
\item[{dfnum}] \leavevmode{[}int{]}
Parameter, should be \textgreater{} 1.

\item[{dfden}] \leavevmode{[}int{]}
Parameter, should be \textgreater{} 1.

\item[{nonc}] \leavevmode{[}float{]}
Parameter, should be \textgreater{}= 0.

\item[{size}] \leavevmode{[}int or tuple of ints{]}
Output shape. If the given shape is, e.g., \code{(m, n, k)}, then
\code{m * n * k} samples are drawn.

\end{description}
\begin{description}
\item[{samples}] \leavevmode{[}scalar or ndarray{]}
Drawn samples.

\end{description}

When calculating the power of an experiment (power = probability of
rejecting the null hypothesis when a specific alternative is true) the
non-central F statistic becomes important.  When the null hypothesis is
true, the F statistic follows a central F distribution. When the null
hypothesis is not true, then it follows a non-central F statistic.

Weisstein, Eric W. ``Noncentral F-Distribution.'' From MathWorld--A Wolfram
Web Resource.  \href{http://mathworld.wolfram.com/NoncentralF-Distribution.html}{http://mathworld.wolfram.com/NoncentralF-Distribution.html}

Wikipedia, ``Noncentral F distribution'',
\href{http://en.wikipedia.org/wiki/Noncentral\_F-distribution}{http://en.wikipedia.org/wiki/Noncentral\_F-distribution}

In a study, testing for a specific alternative to the null hypothesis
requires use of the Noncentral F distribution. We need to calculate the
area in the tail of the distribution that exceeds the value of the F
distribution for the null hypothesis.  We'll plot the two probability
distributions for comparison.

\begin{Verbatim}[commandchars=\\\{\}]
\PYG{g+gp}{\PYGZgt{}\PYGZgt{}\PYGZgt{} }\PYG{n}{dfnum} \PYG{o}{=} \PYG{l+m+mi}{3} \PYG{c}{\PYGZsh{} between group deg of freedom}
\PYG{g+gp}{\PYGZgt{}\PYGZgt{}\PYGZgt{} }\PYG{n}{dfden} \PYG{o}{=} \PYG{l+m+mi}{20} \PYG{c}{\PYGZsh{} within groups degrees of freedom}
\PYG{g+gp}{\PYGZgt{}\PYGZgt{}\PYGZgt{} }\PYG{n}{nonc} \PYG{o}{=} \PYG{l+m+mf}{3.0}
\PYG{g+gp}{\PYGZgt{}\PYGZgt{}\PYGZgt{} }\PYG{n}{nc\PYGZus{}vals} \PYG{o}{=} \PYG{n}{np}\PYG{o}{.}\PYG{n}{random}\PYG{o}{.}\PYG{n}{noncentral\PYGZus{}f}\PYG{p}{(}\PYG{n}{dfnum}\PYG{p}{,} \PYG{n}{dfden}\PYG{p}{,} \PYG{n}{nonc}\PYG{p}{,} \PYG{l+m+mi}{1000000}\PYG{p}{)}
\PYG{g+gp}{\PYGZgt{}\PYGZgt{}\PYGZgt{} }\PYG{n}{NF} \PYG{o}{=} \PYG{n}{np}\PYG{o}{.}\PYG{n}{histogram}\PYG{p}{(}\PYG{n}{nc\PYGZus{}vals}\PYG{p}{,} \PYG{n}{bins}\PYG{o}{=}\PYG{l+m+mi}{50}\PYG{p}{,} \PYG{n}{normed}\PYG{o}{=}\PYG{n+nb+bp}{True}\PYG{p}{)}
\PYG{g+gp}{\PYGZgt{}\PYGZgt{}\PYGZgt{} }\PYG{n}{c\PYGZus{}vals} \PYG{o}{=} \PYG{n}{np}\PYG{o}{.}\PYG{n}{random}\PYG{o}{.}\PYG{n}{f}\PYG{p}{(}\PYG{n}{dfnum}\PYG{p}{,} \PYG{n}{dfden}\PYG{p}{,} \PYG{l+m+mi}{1000000}\PYG{p}{)}
\PYG{g+gp}{\PYGZgt{}\PYGZgt{}\PYGZgt{} }\PYG{n}{F} \PYG{o}{=} \PYG{n}{np}\PYG{o}{.}\PYG{n}{histogram}\PYG{p}{(}\PYG{n}{c\PYGZus{}vals}\PYG{p}{,} \PYG{n}{bins}\PYG{o}{=}\PYG{l+m+mi}{50}\PYG{p}{,} \PYG{n}{normed}\PYG{o}{=}\PYG{n+nb+bp}{True}\PYG{p}{)}
\PYG{g+gp}{\PYGZgt{}\PYGZgt{}\PYGZgt{} }\PYG{n}{plt}\PYG{o}{.}\PYG{n}{plot}\PYG{p}{(}\PYG{n}{F}\PYG{p}{[}\PYG{l+m+mi}{1}\PYG{p}{]}\PYG{p}{[}\PYG{l+m+mi}{1}\PYG{p}{:}\PYG{p}{]}\PYG{p}{,} \PYG{n}{F}\PYG{p}{[}\PYG{l+m+mi}{0}\PYG{p}{]}\PYG{p}{)}
\PYG{g+gp}{\PYGZgt{}\PYGZgt{}\PYGZgt{} }\PYG{n}{plt}\PYG{o}{.}\PYG{n}{plot}\PYG{p}{(}\PYG{n}{NF}\PYG{p}{[}\PYG{l+m+mi}{1}\PYG{p}{]}\PYG{p}{[}\PYG{l+m+mi}{1}\PYG{p}{:}\PYG{p}{]}\PYG{p}{,} \PYG{n}{NF}\PYG{p}{[}\PYG{l+m+mi}{0}\PYG{p}{]}\PYG{p}{)}
\PYG{g+gp}{\PYGZgt{}\PYGZgt{}\PYGZgt{} }\PYG{n}{plt}\PYG{o}{.}\PYG{n}{show}\PYG{p}{(}\PYG{p}{)}
\end{Verbatim}

\end{fulllineitems}

\index{normal() (in module acsStatesAnalysis)}

\begin{fulllineitems}
\phantomsection\label{acsStatesAnalysis:acsStatesAnalysis.normal}\pysiglinewithargsret{\code{acsStatesAnalysis.}\bfcode{normal}}{\emph{loc=0.0}, \emph{scale=1.0}, \emph{size=None}}{}
Draw random samples from a normal (Gaussian) distribution.

The probability density function of the normal distribution, first
derived by De Moivre and 200 years later by both Gauss and Laplace
independently {\color{red}\bfseries{}{[}2{]}\_}, is often called the bell curve because of
its characteristic shape (see the example below).

The normal distributions occurs often in nature.  For example, it
describes the commonly occurring distribution of samples influenced
by a large number of tiny, random disturbances, each with its own
unique distribution {\color{red}\bfseries{}{[}2{]}\_}.
\begin{description}
\item[{loc}] \leavevmode{[}float{]}
Mean (``centre'') of the distribution.

\item[{scale}] \leavevmode{[}float{]}
Standard deviation (spread or ``width'') of the distribution.

\item[{size}] \leavevmode{[}tuple of ints{]}
Output shape.  If the given shape is, e.g., \code{(m, n, k)}, then
\code{m * n * k} samples are drawn.

\end{description}
\begin{description}
\item[{scipy.stats.distributions.norm}] \leavevmode{[}probability density function,{]}
distribution or cumulative density function, etc.

\end{description}

The probability density for the Gaussian distribution is
\begin{gather}
\begin{split}p(x) = \frac{1}{\sqrt{ 2 \pi \sigma^2 }}
e^{ - \frac{ (x - \mu)^2 } {2 \sigma^2} },\end{split}\notag
\end{gather}
where \(\mu\) is the mean and \(\sigma\) the standard deviation.
The square of the standard deviation, \(\sigma^2\), is called the
variance.

The function has its peak at the mean, and its ``spread'' increases with
the standard deviation (the function reaches 0.607 times its maximum at
\(x + \sigma\) and \(x - \sigma\) {\color{red}\bfseries{}{[}2{]}\_}).  This implies that
\emph{numpy.random.normal} is more likely to return samples lying close to the
mean, rather than those far away.

Draw samples from the distribution:

\begin{Verbatim}[commandchars=\\\{\}]
\PYG{g+gp}{\PYGZgt{}\PYGZgt{}\PYGZgt{} }\PYG{n}{mu}\PYG{p}{,} \PYG{n}{sigma} \PYG{o}{=} \PYG{l+m+mi}{0}\PYG{p}{,} \PYG{l+m+mf}{0.1} \PYG{c}{\PYGZsh{} mean and standard deviation}
\PYG{g+gp}{\PYGZgt{}\PYGZgt{}\PYGZgt{} }\PYG{n}{s} \PYG{o}{=} \PYG{n}{np}\PYG{o}{.}\PYG{n}{random}\PYG{o}{.}\PYG{n}{normal}\PYG{p}{(}\PYG{n}{mu}\PYG{p}{,} \PYG{n}{sigma}\PYG{p}{,} \PYG{l+m+mi}{1000}\PYG{p}{)}
\end{Verbatim}

Verify the mean and the variance:

\begin{Verbatim}[commandchars=\\\{\}]
\PYG{g+gp}{\PYGZgt{}\PYGZgt{}\PYGZgt{} }\PYG{n+nb}{abs}\PYG{p}{(}\PYG{n}{mu} \PYG{o}{\PYGZhy{}} \PYG{n}{np}\PYG{o}{.}\PYG{n}{mean}\PYG{p}{(}\PYG{n}{s}\PYG{p}{)}\PYG{p}{)} \PYG{o}{\PYGZlt{}} \PYG{l+m+mf}{0.01}
\PYG{g+go}{True}
\end{Verbatim}

\begin{Verbatim}[commandchars=\\\{\}]
\PYG{g+gp}{\PYGZgt{}\PYGZgt{}\PYGZgt{} }\PYG{n+nb}{abs}\PYG{p}{(}\PYG{n}{sigma} \PYG{o}{\PYGZhy{}} \PYG{n}{np}\PYG{o}{.}\PYG{n}{std}\PYG{p}{(}\PYG{n}{s}\PYG{p}{,} \PYG{n}{ddof}\PYG{o}{=}\PYG{l+m+mi}{1}\PYG{p}{)}\PYG{p}{)} \PYG{o}{\PYGZlt{}} \PYG{l+m+mf}{0.01}
\PYG{g+go}{True}
\end{Verbatim}

Display the histogram of the samples, along with
the probability density function:

\begin{Verbatim}[commandchars=\\\{\}]
\PYG{g+gp}{\PYGZgt{}\PYGZgt{}\PYGZgt{} }\PYG{k+kn}{import} \PYG{n+nn}{matplotlib.pyplot} \PYG{k+kn}{as} \PYG{n+nn}{plt}
\PYG{g+gp}{\PYGZgt{}\PYGZgt{}\PYGZgt{} }\PYG{n}{count}\PYG{p}{,} \PYG{n}{bins}\PYG{p}{,} \PYG{n}{ignored} \PYG{o}{=} \PYG{n}{plt}\PYG{o}{.}\PYG{n}{hist}\PYG{p}{(}\PYG{n}{s}\PYG{p}{,} \PYG{l+m+mi}{30}\PYG{p}{,} \PYG{n}{normed}\PYG{o}{=}\PYG{n+nb+bp}{True}\PYG{p}{)}
\PYG{g+gp}{\PYGZgt{}\PYGZgt{}\PYGZgt{} }\PYG{n}{plt}\PYG{o}{.}\PYG{n}{plot}\PYG{p}{(}\PYG{n}{bins}\PYG{p}{,} \PYG{l+m+mi}{1}\PYG{o}{/}\PYG{p}{(}\PYG{n}{sigma} \PYG{o}{*} \PYG{n}{np}\PYG{o}{.}\PYG{n}{sqrt}\PYG{p}{(}\PYG{l+m+mi}{2} \PYG{o}{*} \PYG{n}{np}\PYG{o}{.}\PYG{n}{pi}\PYG{p}{)}\PYG{p}{)} \PYG{o}{*}
\PYG{g+gp}{... }               \PYG{n}{np}\PYG{o}{.}\PYG{n}{exp}\PYG{p}{(} \PYG{o}{\PYGZhy{}} \PYG{p}{(}\PYG{n}{bins} \PYG{o}{\PYGZhy{}} \PYG{n}{mu}\PYG{p}{)}\PYG{o}{*}\PYG{o}{*}\PYG{l+m+mi}{2} \PYG{o}{/} \PYG{p}{(}\PYG{l+m+mi}{2} \PYG{o}{*} \PYG{n}{sigma}\PYG{o}{*}\PYG{o}{*}\PYG{l+m+mi}{2}\PYG{p}{)} \PYG{p}{)}\PYG{p}{,}
\PYG{g+gp}{... }         \PYG{n}{linewidth}\PYG{o}{=}\PYG{l+m+mi}{2}\PYG{p}{,} \PYG{n}{color}\PYG{o}{=}\PYG{l+s}{\PYGZsq{}}\PYG{l+s}{r}\PYG{l+s}{\PYGZsq{}}\PYG{p}{)}
\PYG{g+gp}{\PYGZgt{}\PYGZgt{}\PYGZgt{} }\PYG{n}{plt}\PYG{o}{.}\PYG{n}{show}\PYG{p}{(}\PYG{p}{)}
\end{Verbatim}

\end{fulllineitems}

\index{pareto() (in module acsStatesAnalysis)}

\begin{fulllineitems}
\phantomsection\label{acsStatesAnalysis:acsStatesAnalysis.pareto}\pysiglinewithargsret{\code{acsStatesAnalysis.}\bfcode{pareto}}{\emph{a}, \emph{size=None}}{}
Draw samples from a Pareto II or Lomax distribution with specified shape.

The Lomax or Pareto II distribution is a shifted Pareto distribution. The
classical Pareto distribution can be obtained from the Lomax distribution
by adding the location parameter m, see below. The smallest value of the
Lomax distribution is zero while for the classical Pareto distribution it
is m, where the standard Pareto distribution has location m=1.
Lomax can also be considered as a simplified version of the Generalized
Pareto distribution (available in SciPy), with the scale set to one and
the location set to zero.

The Pareto distribution must be greater than zero, and is unbounded above.
It is also known as the ``80-20 rule''.  In this distribution, 80 percent of
the weights are in the lowest 20 percent of the range, while the other 20
percent fill the remaining 80 percent of the range.
\begin{description}
\item[{shape}] \leavevmode{[}float, \textgreater{} 0.{]}
Shape of the distribution.

\item[{size}] \leavevmode{[}tuple of ints{]}
Output shape.  If the given shape is, e.g., \code{(m, n, k)}, then
\code{m * n * k} samples are drawn.

\end{description}
\begin{description}
\item[{scipy.stats.distributions.lomax.pdf}] \leavevmode{[}probability density function,{]}
distribution or cumulative density function, etc.

\item[{scipy.stats.distributions.genpareto.pdf}] \leavevmode{[}probability density function,{]}
distribution or cumulative density function, etc.

\end{description}

The probability density for the Pareto distribution is
\begin{gather}
\begin{split}p(x) = \frac{am^a}{x^{a+1}}\end{split}\notag
\end{gather}
where \(a\) is the shape and \(m\) the location

The Pareto distribution, named after the Italian economist Vilfredo Pareto,
is a power law probability distribution useful in many real world problems.
Outside the field of economics it is generally referred to as the Bradford
distribution. Pareto developed the distribution to describe the
distribution of wealth in an economy.  It has also found use in insurance,
web page access statistics, oil field sizes, and many other problems,
including the download frequency for projects in Sourceforge {[}1{]}.  It is
one of the so-called ``fat-tailed'' distributions.

Draw samples from the distribution:

\begin{Verbatim}[commandchars=\\\{\}]
\PYG{g+gp}{\PYGZgt{}\PYGZgt{}\PYGZgt{} }\PYG{n}{a}\PYG{p}{,} \PYG{n}{m} \PYG{o}{=} \PYG{l+m+mf}{3.}\PYG{p}{,} \PYG{l+m+mf}{1.} \PYG{c}{\PYGZsh{} shape and mode}
\PYG{g+gp}{\PYGZgt{}\PYGZgt{}\PYGZgt{} }\PYG{n}{s} \PYG{o}{=} \PYG{n}{np}\PYG{o}{.}\PYG{n}{random}\PYG{o}{.}\PYG{n}{pareto}\PYG{p}{(}\PYG{n}{a}\PYG{p}{,} \PYG{l+m+mi}{1000}\PYG{p}{)} \PYG{o}{+} \PYG{n}{m}
\end{Verbatim}

Display the histogram of the samples, along with
the probability density function:

\begin{Verbatim}[commandchars=\\\{\}]
\PYG{g+gp}{\PYGZgt{}\PYGZgt{}\PYGZgt{} }\PYG{k+kn}{import} \PYG{n+nn}{matplotlib.pyplot} \PYG{k+kn}{as} \PYG{n+nn}{plt}
\PYG{g+gp}{\PYGZgt{}\PYGZgt{}\PYGZgt{} }\PYG{n}{count}\PYG{p}{,} \PYG{n}{bins}\PYG{p}{,} \PYG{n}{ignored} \PYG{o}{=} \PYG{n}{plt}\PYG{o}{.}\PYG{n}{hist}\PYG{p}{(}\PYG{n}{s}\PYG{p}{,} \PYG{l+m+mi}{100}\PYG{p}{,} \PYG{n}{normed}\PYG{o}{=}\PYG{n+nb+bp}{True}\PYG{p}{,} \PYG{n}{align}\PYG{o}{=}\PYG{l+s}{\PYGZsq{}}\PYG{l+s}{center}\PYG{l+s}{\PYGZsq{}}\PYG{p}{)}
\PYG{g+gp}{\PYGZgt{}\PYGZgt{}\PYGZgt{} }\PYG{n}{fit} \PYG{o}{=} \PYG{n}{a}\PYG{o}{*}\PYG{n}{m}\PYG{o}{*}\PYG{o}{*}\PYG{n}{a}\PYG{o}{/}\PYG{n}{bins}\PYG{o}{*}\PYG{o}{*}\PYG{p}{(}\PYG{n}{a}\PYG{o}{+}\PYG{l+m+mi}{1}\PYG{p}{)}
\PYG{g+gp}{\PYGZgt{}\PYGZgt{}\PYGZgt{} }\PYG{n}{plt}\PYG{o}{.}\PYG{n}{plot}\PYG{p}{(}\PYG{n}{bins}\PYG{p}{,} \PYG{n+nb}{max}\PYG{p}{(}\PYG{n}{count}\PYG{p}{)}\PYG{o}{*}\PYG{n}{fit}\PYG{o}{/}\PYG{n+nb}{max}\PYG{p}{(}\PYG{n}{fit}\PYG{p}{)}\PYG{p}{,}\PYG{n}{linewidth}\PYG{o}{=}\PYG{l+m+mi}{2}\PYG{p}{,} \PYG{n}{color}\PYG{o}{=}\PYG{l+s}{\PYGZsq{}}\PYG{l+s}{r}\PYG{l+s}{\PYGZsq{}}\PYG{p}{)}
\PYG{g+gp}{\PYGZgt{}\PYGZgt{}\PYGZgt{} }\PYG{n}{plt}\PYG{o}{.}\PYG{n}{show}\PYG{p}{(}\PYG{p}{)}
\end{Verbatim}

\end{fulllineitems}

\index{permutation() (in module acsStatesAnalysis)}

\begin{fulllineitems}
\phantomsection\label{acsStatesAnalysis:acsStatesAnalysis.permutation}\pysiglinewithargsret{\code{acsStatesAnalysis.}\bfcode{permutation}}{\emph{x}}{}
Randomly permute a sequence, or return a permuted range.

If \emph{x} is a multi-dimensional array, it is only shuffled along its
first index.
\begin{description}
\item[{x}] \leavevmode{[}int or array\_like{]}
If \emph{x} is an integer, randomly permute \code{np.arange(x)}.
If \emph{x} is an array, make a copy and shuffle the elements
randomly.

\end{description}
\begin{description}
\item[{out}] \leavevmode{[}ndarray{]}
Permuted sequence or array range.

\end{description}

\begin{Verbatim}[commandchars=\\\{\}]
\PYG{g+gp}{\PYGZgt{}\PYGZgt{}\PYGZgt{} }\PYG{n}{np}\PYG{o}{.}\PYG{n}{random}\PYG{o}{.}\PYG{n}{permutation}\PYG{p}{(}\PYG{l+m+mi}{10}\PYG{p}{)}
\PYG{g+go}{array([1, 7, 4, 3, 0, 9, 2, 5, 8, 6])}
\end{Verbatim}

\begin{Verbatim}[commandchars=\\\{\}]
\PYG{g+gp}{\PYGZgt{}\PYGZgt{}\PYGZgt{} }\PYG{n}{np}\PYG{o}{.}\PYG{n}{random}\PYG{o}{.}\PYG{n}{permutation}\PYG{p}{(}\PYG{p}{[}\PYG{l+m+mi}{1}\PYG{p}{,} \PYG{l+m+mi}{4}\PYG{p}{,} \PYG{l+m+mi}{9}\PYG{p}{,} \PYG{l+m+mi}{12}\PYG{p}{,} \PYG{l+m+mi}{15}\PYG{p}{]}\PYG{p}{)}
\PYG{g+go}{array([15,  1,  9,  4, 12])}
\end{Verbatim}

\begin{Verbatim}[commandchars=\\\{\}]
\PYG{g+gp}{\PYGZgt{}\PYGZgt{}\PYGZgt{} }\PYG{n}{arr} \PYG{o}{=} \PYG{n}{np}\PYG{o}{.}\PYG{n}{arange}\PYG{p}{(}\PYG{l+m+mi}{9}\PYG{p}{)}\PYG{o}{.}\PYG{n}{reshape}\PYG{p}{(}\PYG{p}{(}\PYG{l+m+mi}{3}\PYG{p}{,} \PYG{l+m+mi}{3}\PYG{p}{)}\PYG{p}{)}
\PYG{g+gp}{\PYGZgt{}\PYGZgt{}\PYGZgt{} }\PYG{n}{np}\PYG{o}{.}\PYG{n}{random}\PYG{o}{.}\PYG{n}{permutation}\PYG{p}{(}\PYG{n}{arr}\PYG{p}{)}
\PYG{g+go}{array([[6, 7, 8],}
\PYG{g+go}{       [0, 1, 2],}
\PYG{g+go}{       [3, 4, 5]])}
\end{Verbatim}

\end{fulllineitems}

\index{poisson() (in module acsStatesAnalysis)}

\begin{fulllineitems}
\phantomsection\label{acsStatesAnalysis:acsStatesAnalysis.poisson}\pysiglinewithargsret{\code{acsStatesAnalysis.}\bfcode{poisson}}{\emph{lam=1.0}, \emph{size=None}}{}
Draw samples from a Poisson distribution.

The Poisson distribution is the limit of the Binomial
distribution for large N.
\begin{description}
\item[{lam}] \leavevmode{[}float{]}
Expectation of interval, should be \textgreater{}= 0.

\item[{size}] \leavevmode{[}int or tuple of ints, optional{]}
Output shape. If the given shape is, e.g., \code{(m, n, k)}, then
\code{m * n * k} samples are drawn.

\end{description}

The Poisson distribution
\begin{gather}
\begin{split}f(k; \lambda)=\frac{\lambda^k e^{-\lambda}}{k!}\end{split}\notag
\end{gather}
For events with an expected separation \(\lambda\) the Poisson
distribution \(f(k; \lambda)\) describes the probability of
\(k\) events occurring within the observed interval \(\lambda\).

Because the output is limited to the range of the C long type, a
ValueError is raised when \emph{lam} is within 10 sigma of the maximum
representable value.

Draw samples from the distribution:

\begin{Verbatim}[commandchars=\\\{\}]
\PYG{g+gp}{\PYGZgt{}\PYGZgt{}\PYGZgt{} }\PYG{k+kn}{import} \PYG{n+nn}{numpy} \PYG{k+kn}{as} \PYG{n+nn}{np}
\PYG{g+gp}{\PYGZgt{}\PYGZgt{}\PYGZgt{} }\PYG{n}{s} \PYG{o}{=} \PYG{n}{np}\PYG{o}{.}\PYG{n}{random}\PYG{o}{.}\PYG{n}{poisson}\PYG{p}{(}\PYG{l+m+mi}{5}\PYG{p}{,} \PYG{l+m+mi}{10000}\PYG{p}{)}
\end{Verbatim}

Display histogram of the sample:

\begin{Verbatim}[commandchars=\\\{\}]
\PYG{g+gp}{\PYGZgt{}\PYGZgt{}\PYGZgt{} }\PYG{k+kn}{import} \PYG{n+nn}{matplotlib.pyplot} \PYG{k+kn}{as} \PYG{n+nn}{plt}
\PYG{g+gp}{\PYGZgt{}\PYGZgt{}\PYGZgt{} }\PYG{n}{count}\PYG{p}{,} \PYG{n}{bins}\PYG{p}{,} \PYG{n}{ignored} \PYG{o}{=} \PYG{n}{plt}\PYG{o}{.}\PYG{n}{hist}\PYG{p}{(}\PYG{n}{s}\PYG{p}{,} \PYG{l+m+mi}{14}\PYG{p}{,} \PYG{n}{normed}\PYG{o}{=}\PYG{n+nb+bp}{True}\PYG{p}{)}
\PYG{g+gp}{\PYGZgt{}\PYGZgt{}\PYGZgt{} }\PYG{n}{plt}\PYG{o}{.}\PYG{n}{show}\PYG{p}{(}\PYG{p}{)}
\end{Verbatim}

\end{fulllineitems}

\index{power() (in module acsStatesAnalysis)}

\begin{fulllineitems}
\phantomsection\label{acsStatesAnalysis:acsStatesAnalysis.power}\pysiglinewithargsret{\code{acsStatesAnalysis.}\bfcode{power}}{\emph{a}, \emph{size=None}}{}
Draws samples in {[}0, 1{]} from a power distribution with positive
exponent a - 1.

Also known as the power function distribution.
\begin{description}
\item[{a}] \leavevmode{[}float{]}
parameter, \textgreater{} 0

\item[{size}] \leavevmode{[}tuple of ints{]}\begin{description}
\item[{Output shape.  If the given shape is, e.g., \code{(m, n, k)}, then}] \leavevmode
\code{m * n * k} samples are drawn.

\end{description}

\end{description}
\begin{description}
\item[{samples}] \leavevmode{[}\{ndarray, scalar\}{]}
The returned samples lie in {[}0, 1{]}.

\end{description}
\begin{description}
\item[{ValueError}] \leavevmode
If a\textless{}1.

\end{description}

The probability density function is
\begin{gather}
\begin{split}P(x; a) = ax^{a-1}, 0 \le x \le 1, a>0.\end{split}\notag
\end{gather}
The power function distribution is just the inverse of the Pareto
distribution. It may also be seen as a special case of the Beta
distribution.

It is used, for example, in modeling the over-reporting of insurance
claims.

Draw samples from the distribution:

\begin{Verbatim}[commandchars=\\\{\}]
\PYG{g+gp}{\PYGZgt{}\PYGZgt{}\PYGZgt{} }\PYG{n}{a} \PYG{o}{=} \PYG{l+m+mf}{5.} \PYG{c}{\PYGZsh{} shape}
\PYG{g+gp}{\PYGZgt{}\PYGZgt{}\PYGZgt{} }\PYG{n}{samples} \PYG{o}{=} \PYG{l+m+mi}{1000}
\PYG{g+gp}{\PYGZgt{}\PYGZgt{}\PYGZgt{} }\PYG{n}{s} \PYG{o}{=} \PYG{n}{np}\PYG{o}{.}\PYG{n}{random}\PYG{o}{.}\PYG{n}{power}\PYG{p}{(}\PYG{n}{a}\PYG{p}{,} \PYG{n}{samples}\PYG{p}{)}
\end{Verbatim}

Display the histogram of the samples, along with
the probability density function:

\begin{Verbatim}[commandchars=\\\{\}]
\PYG{g+gp}{\PYGZgt{}\PYGZgt{}\PYGZgt{} }\PYG{k+kn}{import} \PYG{n+nn}{matplotlib.pyplot} \PYG{k+kn}{as} \PYG{n+nn}{plt}
\PYG{g+gp}{\PYGZgt{}\PYGZgt{}\PYGZgt{} }\PYG{n}{count}\PYG{p}{,} \PYG{n}{bins}\PYG{p}{,} \PYG{n}{ignored} \PYG{o}{=} \PYG{n}{plt}\PYG{o}{.}\PYG{n}{hist}\PYG{p}{(}\PYG{n}{s}\PYG{p}{,} \PYG{n}{bins}\PYG{o}{=}\PYG{l+m+mi}{30}\PYG{p}{)}
\PYG{g+gp}{\PYGZgt{}\PYGZgt{}\PYGZgt{} }\PYG{n}{x} \PYG{o}{=} \PYG{n}{np}\PYG{o}{.}\PYG{n}{linspace}\PYG{p}{(}\PYG{l+m+mi}{0}\PYG{p}{,} \PYG{l+m+mi}{1}\PYG{p}{,} \PYG{l+m+mi}{100}\PYG{p}{)}
\PYG{g+gp}{\PYGZgt{}\PYGZgt{}\PYGZgt{} }\PYG{n}{y} \PYG{o}{=} \PYG{n}{a}\PYG{o}{*}\PYG{n}{x}\PYG{o}{*}\PYG{o}{*}\PYG{p}{(}\PYG{n}{a}\PYG{o}{\PYGZhy{}}\PYG{l+m+mf}{1.}\PYG{p}{)}
\PYG{g+gp}{\PYGZgt{}\PYGZgt{}\PYGZgt{} }\PYG{n}{normed\PYGZus{}y} \PYG{o}{=} \PYG{n}{samples}\PYG{o}{*}\PYG{n}{np}\PYG{o}{.}\PYG{n}{diff}\PYG{p}{(}\PYG{n}{bins}\PYG{p}{)}\PYG{p}{[}\PYG{l+m+mi}{0}\PYG{p}{]}\PYG{o}{*}\PYG{n}{y}
\PYG{g+gp}{\PYGZgt{}\PYGZgt{}\PYGZgt{} }\PYG{n}{plt}\PYG{o}{.}\PYG{n}{plot}\PYG{p}{(}\PYG{n}{x}\PYG{p}{,} \PYG{n}{normed\PYGZus{}y}\PYG{p}{)}
\PYG{g+gp}{\PYGZgt{}\PYGZgt{}\PYGZgt{} }\PYG{n}{plt}\PYG{o}{.}\PYG{n}{show}\PYG{p}{(}\PYG{p}{)}
\end{Verbatim}

Compare the power function distribution to the inverse of the Pareto.

\begin{Verbatim}[commandchars=\\\{\}]
\PYG{g+gp}{\PYGZgt{}\PYGZgt{}\PYGZgt{} }\PYG{k+kn}{from} \PYG{n+nn}{scipy} \PYG{k+kn}{import} \PYG{n}{stats}
\PYG{g+gp}{\PYGZgt{}\PYGZgt{}\PYGZgt{} }\PYG{n}{rvs} \PYG{o}{=} \PYG{n}{np}\PYG{o}{.}\PYG{n}{random}\PYG{o}{.}\PYG{n}{power}\PYG{p}{(}\PYG{l+m+mi}{5}\PYG{p}{,} \PYG{l+m+mi}{1000000}\PYG{p}{)}
\PYG{g+gp}{\PYGZgt{}\PYGZgt{}\PYGZgt{} }\PYG{n}{rvsp} \PYG{o}{=} \PYG{n}{np}\PYG{o}{.}\PYG{n}{random}\PYG{o}{.}\PYG{n}{pareto}\PYG{p}{(}\PYG{l+m+mi}{5}\PYG{p}{,} \PYG{l+m+mi}{1000000}\PYG{p}{)}
\PYG{g+gp}{\PYGZgt{}\PYGZgt{}\PYGZgt{} }\PYG{n}{xx} \PYG{o}{=} \PYG{n}{np}\PYG{o}{.}\PYG{n}{linspace}\PYG{p}{(}\PYG{l+m+mi}{0}\PYG{p}{,}\PYG{l+m+mi}{1}\PYG{p}{,}\PYG{l+m+mi}{100}\PYG{p}{)}
\PYG{g+gp}{\PYGZgt{}\PYGZgt{}\PYGZgt{} }\PYG{n}{powpdf} \PYG{o}{=} \PYG{n}{stats}\PYG{o}{.}\PYG{n}{powerlaw}\PYG{o}{.}\PYG{n}{pdf}\PYG{p}{(}\PYG{n}{xx}\PYG{p}{,}\PYG{l+m+mi}{5}\PYG{p}{)}
\end{Verbatim}

\begin{Verbatim}[commandchars=\\\{\}]
\PYG{g+gp}{\PYGZgt{}\PYGZgt{}\PYGZgt{} }\PYG{n}{plt}\PYG{o}{.}\PYG{n}{figure}\PYG{p}{(}\PYG{p}{)}
\PYG{g+gp}{\PYGZgt{}\PYGZgt{}\PYGZgt{} }\PYG{n}{plt}\PYG{o}{.}\PYG{n}{hist}\PYG{p}{(}\PYG{n}{rvs}\PYG{p}{,} \PYG{n}{bins}\PYG{o}{=}\PYG{l+m+mi}{50}\PYG{p}{,} \PYG{n}{normed}\PYG{o}{=}\PYG{n+nb+bp}{True}\PYG{p}{)}
\PYG{g+gp}{\PYGZgt{}\PYGZgt{}\PYGZgt{} }\PYG{n}{plt}\PYG{o}{.}\PYG{n}{plot}\PYG{p}{(}\PYG{n}{xx}\PYG{p}{,}\PYG{n}{powpdf}\PYG{p}{,}\PYG{l+s}{\PYGZsq{}}\PYG{l+s}{r\PYGZhy{}}\PYG{l+s}{\PYGZsq{}}\PYG{p}{)}
\PYG{g+gp}{\PYGZgt{}\PYGZgt{}\PYGZgt{} }\PYG{n}{plt}\PYG{o}{.}\PYG{n}{title}\PYG{p}{(}\PYG{l+s}{\PYGZsq{}}\PYG{l+s}{np.random.power(5)}\PYG{l+s}{\PYGZsq{}}\PYG{p}{)}
\end{Verbatim}

\begin{Verbatim}[commandchars=\\\{\}]
\PYG{g+gp}{\PYGZgt{}\PYGZgt{}\PYGZgt{} }\PYG{n}{plt}\PYG{o}{.}\PYG{n}{figure}\PYG{p}{(}\PYG{p}{)}
\PYG{g+gp}{\PYGZgt{}\PYGZgt{}\PYGZgt{} }\PYG{n}{plt}\PYG{o}{.}\PYG{n}{hist}\PYG{p}{(}\PYG{l+m+mf}{1.}\PYG{o}{/}\PYG{p}{(}\PYG{l+m+mf}{1.}\PYG{o}{+}\PYG{n}{rvsp}\PYG{p}{)}\PYG{p}{,} \PYG{n}{bins}\PYG{o}{=}\PYG{l+m+mi}{50}\PYG{p}{,} \PYG{n}{normed}\PYG{o}{=}\PYG{n+nb+bp}{True}\PYG{p}{)}
\PYG{g+gp}{\PYGZgt{}\PYGZgt{}\PYGZgt{} }\PYG{n}{plt}\PYG{o}{.}\PYG{n}{plot}\PYG{p}{(}\PYG{n}{xx}\PYG{p}{,}\PYG{n}{powpdf}\PYG{p}{,}\PYG{l+s}{\PYGZsq{}}\PYG{l+s}{r\PYGZhy{}}\PYG{l+s}{\PYGZsq{}}\PYG{p}{)}
\PYG{g+gp}{\PYGZgt{}\PYGZgt{}\PYGZgt{} }\PYG{n}{plt}\PYG{o}{.}\PYG{n}{title}\PYG{p}{(}\PYG{l+s}{\PYGZsq{}}\PYG{l+s}{inverse of 1 + np.random.pareto(5)}\PYG{l+s}{\PYGZsq{}}\PYG{p}{)}
\end{Verbatim}

\begin{Verbatim}[commandchars=\\\{\}]
\PYG{g+gp}{\PYGZgt{}\PYGZgt{}\PYGZgt{} }\PYG{n}{plt}\PYG{o}{.}\PYG{n}{figure}\PYG{p}{(}\PYG{p}{)}
\PYG{g+gp}{\PYGZgt{}\PYGZgt{}\PYGZgt{} }\PYG{n}{plt}\PYG{o}{.}\PYG{n}{hist}\PYG{p}{(}\PYG{l+m+mf}{1.}\PYG{o}{/}\PYG{p}{(}\PYG{l+m+mf}{1.}\PYG{o}{+}\PYG{n}{rvsp}\PYG{p}{)}\PYG{p}{,} \PYG{n}{bins}\PYG{o}{=}\PYG{l+m+mi}{50}\PYG{p}{,} \PYG{n}{normed}\PYG{o}{=}\PYG{n+nb+bp}{True}\PYG{p}{)}
\PYG{g+gp}{\PYGZgt{}\PYGZgt{}\PYGZgt{} }\PYG{n}{plt}\PYG{o}{.}\PYG{n}{plot}\PYG{p}{(}\PYG{n}{xx}\PYG{p}{,}\PYG{n}{powpdf}\PYG{p}{,}\PYG{l+s}{\PYGZsq{}}\PYG{l+s}{r\PYGZhy{}}\PYG{l+s}{\PYGZsq{}}\PYG{p}{)}
\PYG{g+gp}{\PYGZgt{}\PYGZgt{}\PYGZgt{} }\PYG{n}{plt}\PYG{o}{.}\PYG{n}{title}\PYG{p}{(}\PYG{l+s}{\PYGZsq{}}\PYG{l+s}{inverse of stats.pareto(5)}\PYG{l+s}{\PYGZsq{}}\PYG{p}{)}
\end{Verbatim}

\end{fulllineitems}

\index{rand() (in module acsStatesAnalysis)}

\begin{fulllineitems}
\phantomsection\label{acsStatesAnalysis:acsStatesAnalysis.rand}\pysiglinewithargsret{\code{acsStatesAnalysis.}\bfcode{rand}}{\emph{d0}, \emph{d1}, \emph{...}, \emph{dn}}{}
Random values in a given shape.

Create an array of the given shape and propagate it with
random samples from a uniform distribution
over \code{{[}0, 1)}.
\begin{description}
\item[{d0, d1, ..., dn}] \leavevmode{[}int, optional{]}
The dimensions of the returned array, should all be positive.
If no argument is given a single Python float is returned.

\end{description}
\begin{description}
\item[{out}] \leavevmode{[}ndarray, shape \code{(d0, d1, ..., dn)}{]}
Random values.

\end{description}

random

This is a convenience function. If you want an interface that
takes a shape-tuple as the first argument, refer to
np.random.random\_sample .

\begin{Verbatim}[commandchars=\\\{\}]
\PYG{g+gp}{\PYGZgt{}\PYGZgt{}\PYGZgt{} }\PYG{n}{np}\PYG{o}{.}\PYG{n}{random}\PYG{o}{.}\PYG{n}{rand}\PYG{p}{(}\PYG{l+m+mi}{3}\PYG{p}{,}\PYG{l+m+mi}{2}\PYG{p}{)}
\PYG{g+go}{array([[ 0.14022471,  0.96360618],  \PYGZsh{}random}
\PYG{g+go}{       [ 0.37601032,  0.25528411],  \PYGZsh{}random}
\PYG{g+go}{       [ 0.49313049,  0.94909878]]) \PYGZsh{}random}
\end{Verbatim}

\end{fulllineitems}

\index{randint() (in module acsStatesAnalysis)}

\begin{fulllineitems}
\phantomsection\label{acsStatesAnalysis:acsStatesAnalysis.randint}\pysiglinewithargsret{\code{acsStatesAnalysis.}\bfcode{randint}}{\emph{low}, \emph{high=None}, \emph{size=None}}{}
Return random integers from \emph{low} (inclusive) to \emph{high} (exclusive).

Return random integers from the ``discrete uniform'' distribution in the
``half-open'' interval {[}\emph{low}, \emph{high}). If \emph{high} is None (the default),
then results are from {[}0, \emph{low}).
\begin{description}
\item[{low}] \leavevmode{[}int{]}
Lowest (signed) integer to be drawn from the distribution (unless
\code{high=None}, in which case this parameter is the \emph{highest} such
integer).

\item[{high}] \leavevmode{[}int, optional{]}
If provided, one above the largest (signed) integer to be drawn
from the distribution (see above for behavior if \code{high=None}).

\item[{size}] \leavevmode{[}int or tuple of ints, optional{]}
Output shape. Default is None, in which case a single int is
returned.

\end{description}
\begin{description}
\item[{out}] \leavevmode{[}int or ndarray of ints{]}
\emph{size}-shaped array of random integers from the appropriate
distribution, or a single such random int if \emph{size} not provided.

\end{description}
\begin{description}
\item[{random.random\_integers}] \leavevmode{[}similar to \emph{randint}, only for the closed{]}
interval {[}\emph{low}, \emph{high}{]}, and 1 is the lowest value if \emph{high} is
omitted. In particular, this other one is the one to use to generate
uniformly distributed discrete non-integers.

\end{description}

\begin{Verbatim}[commandchars=\\\{\}]
\PYG{g+gp}{\PYGZgt{}\PYGZgt{}\PYGZgt{} }\PYG{n}{np}\PYG{o}{.}\PYG{n}{random}\PYG{o}{.}\PYG{n}{randint}\PYG{p}{(}\PYG{l+m+mi}{2}\PYG{p}{,} \PYG{n}{size}\PYG{o}{=}\PYG{l+m+mi}{10}\PYG{p}{)}
\PYG{g+go}{array([1, 0, 0, 0, 1, 1, 0, 0, 1, 0])}
\PYG{g+gp}{\PYGZgt{}\PYGZgt{}\PYGZgt{} }\PYG{n}{np}\PYG{o}{.}\PYG{n}{random}\PYG{o}{.}\PYG{n}{randint}\PYG{p}{(}\PYG{l+m+mi}{1}\PYG{p}{,} \PYG{n}{size}\PYG{o}{=}\PYG{l+m+mi}{10}\PYG{p}{)}
\PYG{g+go}{array([0, 0, 0, 0, 0, 0, 0, 0, 0, 0])}
\end{Verbatim}

Generate a 2 x 4 array of ints between 0 and 4, inclusive:

\begin{Verbatim}[commandchars=\\\{\}]
\PYG{g+gp}{\PYGZgt{}\PYGZgt{}\PYGZgt{} }\PYG{n}{np}\PYG{o}{.}\PYG{n}{random}\PYG{o}{.}\PYG{n}{randint}\PYG{p}{(}\PYG{l+m+mi}{5}\PYG{p}{,} \PYG{n}{size}\PYG{o}{=}\PYG{p}{(}\PYG{l+m+mi}{2}\PYG{p}{,} \PYG{l+m+mi}{4}\PYG{p}{)}\PYG{p}{)}
\PYG{g+go}{array([[4, 0, 2, 1],}
\PYG{g+go}{       [3, 2, 2, 0]])}
\end{Verbatim}

\end{fulllineitems}

\index{randn() (in module acsStatesAnalysis)}

\begin{fulllineitems}
\phantomsection\label{acsStatesAnalysis:acsStatesAnalysis.randn}\pysiglinewithargsret{\code{acsStatesAnalysis.}\bfcode{randn}}{\emph{d0}, \emph{d1}, \emph{...}, \emph{dn}}{}
Return a sample (or samples) from the ``standard normal'' distribution.

If positive, int\_like or int-convertible arguments are provided,
\emph{randn} generates an array of shape \code{(d0, d1, ..., dn)}, filled
with random floats sampled from a univariate ``normal'' (Gaussian)
distribution of mean 0 and variance 1 (if any of the \(d_i\) are
floats, they are first converted to integers by truncation). A single
float randomly sampled from the distribution is returned if no
argument is provided.

This is a convenience function.  If you want an interface that takes a
tuple as the first argument, use \emph{numpy.random.standard\_normal} instead.
\begin{description}
\item[{d0, d1, ..., dn}] \leavevmode{[}int, optional{]}
The dimensions of the returned array, should be all positive.
If no argument is given a single Python float is returned.

\end{description}
\begin{description}
\item[{Z}] \leavevmode{[}ndarray or float{]}
A \code{(d0, d1, ..., dn)}-shaped array of floating-point samples from
the standard normal distribution, or a single such float if
no parameters were supplied.

\end{description}

random.standard\_normal : Similar, but takes a tuple as its argument.

For random samples from \(N(\mu, \sigma^2)\), use:

\code{sigma * np.random.randn(...) + mu}

\begin{Verbatim}[commandchars=\\\{\}]
\PYG{g+gp}{\PYGZgt{}\PYGZgt{}\PYGZgt{} }\PYG{n}{np}\PYG{o}{.}\PYG{n}{random}\PYG{o}{.}\PYG{n}{randn}\PYG{p}{(}\PYG{p}{)}
\PYG{g+go}{2.1923875335537315 \PYGZsh{}random}
\end{Verbatim}

Two-by-four array of samples from N(3, 6.25):

\begin{Verbatim}[commandchars=\\\{\}]
\PYG{g+gp}{\PYGZgt{}\PYGZgt{}\PYGZgt{} }\PYG{l+m+mf}{2.5} \PYG{o}{*} \PYG{n}{np}\PYG{o}{.}\PYG{n}{random}\PYG{o}{.}\PYG{n}{randn}\PYG{p}{(}\PYG{l+m+mi}{2}\PYG{p}{,} \PYG{l+m+mi}{4}\PYG{p}{)} \PYG{o}{+} \PYG{l+m+mi}{3}
\PYG{g+go}{array([[\PYGZhy{}4.49401501,  4.00950034, \PYGZhy{}1.81814867,  7.29718677],  \PYGZsh{}random}
\PYG{g+go}{       [ 0.39924804,  4.68456316,  4.99394529,  4.84057254]]) \PYGZsh{}random}
\end{Verbatim}

\end{fulllineitems}

\index{random() (in module acsStatesAnalysis)}

\begin{fulllineitems}
\phantomsection\label{acsStatesAnalysis:acsStatesAnalysis.random}\pysiglinewithargsret{\code{acsStatesAnalysis.}\bfcode{random}}{}{}
random\_sample(size=None)

Return random floats in the half-open interval {[}0.0, 1.0).

Results are from the ``continuous uniform'' distribution over the
stated interval.  To sample \(Unif[a, b), b > a\) multiply
the output of \emph{random\_sample} by \emph{(b-a)} and add \emph{a}:

\begin{Verbatim}[commandchars=\\\{\}]
\PYG{p}{(}\PYG{n}{b} \PYG{o}{\PYGZhy{}} \PYG{n}{a}\PYG{p}{)} \PYG{o}{*} \PYG{n}{random\PYGZus{}sample}\PYG{p}{(}\PYG{p}{)} \PYG{o}{+} \PYG{n}{a}
\end{Verbatim}
\begin{description}
\item[{size}] \leavevmode{[}int or tuple of ints, optional{]}
Defines the shape of the returned array of random floats. If None
(the default), returns a single float.

\end{description}
\begin{description}
\item[{out}] \leavevmode{[}float or ndarray of floats{]}
Array of random floats of shape \emph{size} (unless \code{size=None}, in which
case a single float is returned).

\end{description}

\begin{Verbatim}[commandchars=\\\{\}]
\PYG{g+gp}{\PYGZgt{}\PYGZgt{}\PYGZgt{} }\PYG{n}{np}\PYG{o}{.}\PYG{n}{random}\PYG{o}{.}\PYG{n}{random\PYGZus{}sample}\PYG{p}{(}\PYG{p}{)}
\PYG{g+go}{0.47108547995356098}
\PYG{g+gp}{\PYGZgt{}\PYGZgt{}\PYGZgt{} }\PYG{n+nb}{type}\PYG{p}{(}\PYG{n}{np}\PYG{o}{.}\PYG{n}{random}\PYG{o}{.}\PYG{n}{random\PYGZus{}sample}\PYG{p}{(}\PYG{p}{)}\PYG{p}{)}
\PYG{g+go}{\PYGZlt{}type \PYGZsq{}float\PYGZsq{}\PYGZgt{}}
\PYG{g+gp}{\PYGZgt{}\PYGZgt{}\PYGZgt{} }\PYG{n}{np}\PYG{o}{.}\PYG{n}{random}\PYG{o}{.}\PYG{n}{random\PYGZus{}sample}\PYG{p}{(}\PYG{p}{(}\PYG{l+m+mi}{5}\PYG{p}{,}\PYG{p}{)}\PYG{p}{)}
\PYG{g+go}{array([ 0.30220482,  0.86820401,  0.1654503 ,  0.11659149,  0.54323428])}
\end{Verbatim}

Three-by-two array of random numbers from {[}-5, 0):

\begin{Verbatim}[commandchars=\\\{\}]
\PYG{g+gp}{\PYGZgt{}\PYGZgt{}\PYGZgt{} }\PYG{l+m+mi}{5} \PYG{o}{*} \PYG{n}{np}\PYG{o}{.}\PYG{n}{random}\PYG{o}{.}\PYG{n}{random\PYGZus{}sample}\PYG{p}{(}\PYG{p}{(}\PYG{l+m+mi}{3}\PYG{p}{,} \PYG{l+m+mi}{2}\PYG{p}{)}\PYG{p}{)} \PYG{o}{\PYGZhy{}} \PYG{l+m+mi}{5}
\PYG{g+go}{array([[\PYGZhy{}3.99149989, \PYGZhy{}0.52338984],}
\PYG{g+go}{       [\PYGZhy{}2.99091858, \PYGZhy{}0.79479508],}
\PYG{g+go}{       [\PYGZhy{}1.23204345, \PYGZhy{}1.75224494]])}
\end{Verbatim}

\end{fulllineitems}

\index{random\_integers() (in module acsStatesAnalysis)}

\begin{fulllineitems}
\phantomsection\label{acsStatesAnalysis:acsStatesAnalysis.random_integers}\pysiglinewithargsret{\code{acsStatesAnalysis.}\bfcode{random\_integers}}{\emph{low}, \emph{high=None}, \emph{size=None}}{}
Return random integers between \emph{low} and \emph{high}, inclusive.

Return random integers from the ``discrete uniform'' distribution in the
closed interval {[}\emph{low}, \emph{high}{]}.  If \emph{high} is None (the default),
then results are from {[}1, \emph{low}{]}.
\begin{description}
\item[{low}] \leavevmode{[}int{]}
Lowest (signed) integer to be drawn from the distribution (unless
\code{high=None}, in which case this parameter is the \emph{highest} such
integer).

\item[{high}] \leavevmode{[}int, optional{]}
If provided, the largest (signed) integer to be drawn from the
distribution (see above for behavior if \code{high=None}).

\item[{size}] \leavevmode{[}int or tuple of ints, optional{]}
Output shape. Default is None, in which case a single int is returned.

\end{description}
\begin{description}
\item[{out}] \leavevmode{[}int or ndarray of ints{]}
\emph{size}-shaped array of random integers from the appropriate
distribution, or a single such random int if \emph{size} not provided.

\end{description}
\begin{description}
\item[{random.randint}] \leavevmode{[}Similar to \emph{random\_integers}, only for the half-open{]}
interval {[}\emph{low}, \emph{high}), and 0 is the lowest value if \emph{high} is
omitted.

\end{description}

To sample from N evenly spaced floating-point numbers between a and b,
use:

\begin{Verbatim}[commandchars=\\\{\}]
\PYG{n}{a} \PYG{o}{+} \PYG{p}{(}\PYG{n}{b} \PYG{o}{\PYGZhy{}} \PYG{n}{a}\PYG{p}{)} \PYG{o}{*} \PYG{p}{(}\PYG{n}{np}\PYG{o}{.}\PYG{n}{random}\PYG{o}{.}\PYG{n}{random\PYGZus{}integers}\PYG{p}{(}\PYG{n}{N}\PYG{p}{)} \PYG{o}{\PYGZhy{}} \PYG{l+m+mi}{1}\PYG{p}{)} \PYG{o}{/} \PYG{p}{(}\PYG{n}{N} \PYG{o}{\PYGZhy{}} \PYG{l+m+mf}{1.}\PYG{p}{)}
\end{Verbatim}

\begin{Verbatim}[commandchars=\\\{\}]
\PYG{g+gp}{\PYGZgt{}\PYGZgt{}\PYGZgt{} }\PYG{n}{np}\PYG{o}{.}\PYG{n}{random}\PYG{o}{.}\PYG{n}{random\PYGZus{}integers}\PYG{p}{(}\PYG{l+m+mi}{5}\PYG{p}{)}
\PYG{g+go}{4}
\PYG{g+gp}{\PYGZgt{}\PYGZgt{}\PYGZgt{} }\PYG{n+nb}{type}\PYG{p}{(}\PYG{n}{np}\PYG{o}{.}\PYG{n}{random}\PYG{o}{.}\PYG{n}{random\PYGZus{}integers}\PYG{p}{(}\PYG{l+m+mi}{5}\PYG{p}{)}\PYG{p}{)}
\PYG{g+go}{\PYGZlt{}type \PYGZsq{}int\PYGZsq{}\PYGZgt{}}
\PYG{g+gp}{\PYGZgt{}\PYGZgt{}\PYGZgt{} }\PYG{n}{np}\PYG{o}{.}\PYG{n}{random}\PYG{o}{.}\PYG{n}{random\PYGZus{}integers}\PYG{p}{(}\PYG{l+m+mi}{5}\PYG{p}{,} \PYG{n}{size}\PYG{o}{=}\PYG{p}{(}\PYG{l+m+mf}{3.}\PYG{p}{,}\PYG{l+m+mf}{2.}\PYG{p}{)}\PYG{p}{)}
\PYG{g+go}{array([[5, 4],}
\PYG{g+go}{       [3, 3],}
\PYG{g+go}{       [4, 5]])}
\end{Verbatim}

Choose five random numbers from the set of five evenly-spaced
numbers between 0 and 2.5, inclusive (\emph{i.e.}, from the set
\({0, 5/8, 10/8, 15/8, 20/8}\)):

\begin{Verbatim}[commandchars=\\\{\}]
\PYG{g+gp}{\PYGZgt{}\PYGZgt{}\PYGZgt{} }\PYG{l+m+mf}{2.5} \PYG{o}{*} \PYG{p}{(}\PYG{n}{np}\PYG{o}{.}\PYG{n}{random}\PYG{o}{.}\PYG{n}{random\PYGZus{}integers}\PYG{p}{(}\PYG{l+m+mi}{5}\PYG{p}{,} \PYG{n}{size}\PYG{o}{=}\PYG{p}{(}\PYG{l+m+mi}{5}\PYG{p}{,}\PYG{p}{)}\PYG{p}{)} \PYG{o}{\PYGZhy{}} \PYG{l+m+mi}{1}\PYG{p}{)} \PYG{o}{/} \PYG{l+m+mf}{4.}
\PYG{g+go}{array([ 0.625,  1.25 ,  0.625,  0.625,  2.5  ])}
\end{Verbatim}

Roll two six sided dice 1000 times and sum the results:

\begin{Verbatim}[commandchars=\\\{\}]
\PYG{g+gp}{\PYGZgt{}\PYGZgt{}\PYGZgt{} }\PYG{n}{d1} \PYG{o}{=} \PYG{n}{np}\PYG{o}{.}\PYG{n}{random}\PYG{o}{.}\PYG{n}{random\PYGZus{}integers}\PYG{p}{(}\PYG{l+m+mi}{1}\PYG{p}{,} \PYG{l+m+mi}{6}\PYG{p}{,} \PYG{l+m+mi}{1000}\PYG{p}{)}
\PYG{g+gp}{\PYGZgt{}\PYGZgt{}\PYGZgt{} }\PYG{n}{d2} \PYG{o}{=} \PYG{n}{np}\PYG{o}{.}\PYG{n}{random}\PYG{o}{.}\PYG{n}{random\PYGZus{}integers}\PYG{p}{(}\PYG{l+m+mi}{1}\PYG{p}{,} \PYG{l+m+mi}{6}\PYG{p}{,} \PYG{l+m+mi}{1000}\PYG{p}{)}
\PYG{g+gp}{\PYGZgt{}\PYGZgt{}\PYGZgt{} }\PYG{n}{dsums} \PYG{o}{=} \PYG{n}{d1} \PYG{o}{+} \PYG{n}{d2}
\end{Verbatim}

Display results as a histogram:

\begin{Verbatim}[commandchars=\\\{\}]
\PYG{g+gp}{\PYGZgt{}\PYGZgt{}\PYGZgt{} }\PYG{k+kn}{import} \PYG{n+nn}{matplotlib.pyplot} \PYG{k+kn}{as} \PYG{n+nn}{plt}
\PYG{g+gp}{\PYGZgt{}\PYGZgt{}\PYGZgt{} }\PYG{n}{count}\PYG{p}{,} \PYG{n}{bins}\PYG{p}{,} \PYG{n}{ignored} \PYG{o}{=} \PYG{n}{plt}\PYG{o}{.}\PYG{n}{hist}\PYG{p}{(}\PYG{n}{dsums}\PYG{p}{,} \PYG{l+m+mi}{11}\PYG{p}{,} \PYG{n}{normed}\PYG{o}{=}\PYG{n+nb+bp}{True}\PYG{p}{)}
\PYG{g+gp}{\PYGZgt{}\PYGZgt{}\PYGZgt{} }\PYG{n}{plt}\PYG{o}{.}\PYG{n}{show}\PYG{p}{(}\PYG{p}{)}
\end{Verbatim}

\end{fulllineitems}

\index{random\_sample() (in module acsStatesAnalysis)}

\begin{fulllineitems}
\phantomsection\label{acsStatesAnalysis:acsStatesAnalysis.random_sample}\pysiglinewithargsret{\code{acsStatesAnalysis.}\bfcode{random\_sample}}{\emph{size=None}}{}
Return random floats in the half-open interval {[}0.0, 1.0).

Results are from the ``continuous uniform'' distribution over the
stated interval.  To sample \(Unif[a, b), b > a\) multiply
the output of \emph{random\_sample} by \emph{(b-a)} and add \emph{a}:

\begin{Verbatim}[commandchars=\\\{\}]
\PYG{p}{(}\PYG{n}{b} \PYG{o}{\PYGZhy{}} \PYG{n}{a}\PYG{p}{)} \PYG{o}{*} \PYG{n}{random\PYGZus{}sample}\PYG{p}{(}\PYG{p}{)} \PYG{o}{+} \PYG{n}{a}
\end{Verbatim}
\begin{description}
\item[{size}] \leavevmode{[}int or tuple of ints, optional{]}
Defines the shape of the returned array of random floats. If None
(the default), returns a single float.

\end{description}
\begin{description}
\item[{out}] \leavevmode{[}float or ndarray of floats{]}
Array of random floats of shape \emph{size} (unless \code{size=None}, in which
case a single float is returned).

\end{description}

\begin{Verbatim}[commandchars=\\\{\}]
\PYG{g+gp}{\PYGZgt{}\PYGZgt{}\PYGZgt{} }\PYG{n}{np}\PYG{o}{.}\PYG{n}{random}\PYG{o}{.}\PYG{n}{random\PYGZus{}sample}\PYG{p}{(}\PYG{p}{)}
\PYG{g+go}{0.47108547995356098}
\PYG{g+gp}{\PYGZgt{}\PYGZgt{}\PYGZgt{} }\PYG{n+nb}{type}\PYG{p}{(}\PYG{n}{np}\PYG{o}{.}\PYG{n}{random}\PYG{o}{.}\PYG{n}{random\PYGZus{}sample}\PYG{p}{(}\PYG{p}{)}\PYG{p}{)}
\PYG{g+go}{\PYGZlt{}type \PYGZsq{}float\PYGZsq{}\PYGZgt{}}
\PYG{g+gp}{\PYGZgt{}\PYGZgt{}\PYGZgt{} }\PYG{n}{np}\PYG{o}{.}\PYG{n}{random}\PYG{o}{.}\PYG{n}{random\PYGZus{}sample}\PYG{p}{(}\PYG{p}{(}\PYG{l+m+mi}{5}\PYG{p}{,}\PYG{p}{)}\PYG{p}{)}
\PYG{g+go}{array([ 0.30220482,  0.86820401,  0.1654503 ,  0.11659149,  0.54323428])}
\end{Verbatim}

Three-by-two array of random numbers from {[}-5, 0):

\begin{Verbatim}[commandchars=\\\{\}]
\PYG{g+gp}{\PYGZgt{}\PYGZgt{}\PYGZgt{} }\PYG{l+m+mi}{5} \PYG{o}{*} \PYG{n}{np}\PYG{o}{.}\PYG{n}{random}\PYG{o}{.}\PYG{n}{random\PYGZus{}sample}\PYG{p}{(}\PYG{p}{(}\PYG{l+m+mi}{3}\PYG{p}{,} \PYG{l+m+mi}{2}\PYG{p}{)}\PYG{p}{)} \PYG{o}{\PYGZhy{}} \PYG{l+m+mi}{5}
\PYG{g+go}{array([[\PYGZhy{}3.99149989, \PYGZhy{}0.52338984],}
\PYG{g+go}{       [\PYGZhy{}2.99091858, \PYGZhy{}0.79479508],}
\PYG{g+go}{       [\PYGZhy{}1.23204345, \PYGZhy{}1.75224494]])}
\end{Verbatim}

\end{fulllineitems}

\index{ranf() (in module acsStatesAnalysis)}

\begin{fulllineitems}
\phantomsection\label{acsStatesAnalysis:acsStatesAnalysis.ranf}\pysiglinewithargsret{\code{acsStatesAnalysis.}\bfcode{ranf}}{}{}
random\_sample(size=None)

Return random floats in the half-open interval {[}0.0, 1.0).

Results are from the ``continuous uniform'' distribution over the
stated interval.  To sample \(Unif[a, b), b > a\) multiply
the output of \emph{random\_sample} by \emph{(b-a)} and add \emph{a}:

\begin{Verbatim}[commandchars=\\\{\}]
\PYG{p}{(}\PYG{n}{b} \PYG{o}{\PYGZhy{}} \PYG{n}{a}\PYG{p}{)} \PYG{o}{*} \PYG{n}{random\PYGZus{}sample}\PYG{p}{(}\PYG{p}{)} \PYG{o}{+} \PYG{n}{a}
\end{Verbatim}
\begin{description}
\item[{size}] \leavevmode{[}int or tuple of ints, optional{]}
Defines the shape of the returned array of random floats. If None
(the default), returns a single float.

\end{description}
\begin{description}
\item[{out}] \leavevmode{[}float or ndarray of floats{]}
Array of random floats of shape \emph{size} (unless \code{size=None}, in which
case a single float is returned).

\end{description}

\begin{Verbatim}[commandchars=\\\{\}]
\PYG{g+gp}{\PYGZgt{}\PYGZgt{}\PYGZgt{} }\PYG{n}{np}\PYG{o}{.}\PYG{n}{random}\PYG{o}{.}\PYG{n}{random\PYGZus{}sample}\PYG{p}{(}\PYG{p}{)}
\PYG{g+go}{0.47108547995356098}
\PYG{g+gp}{\PYGZgt{}\PYGZgt{}\PYGZgt{} }\PYG{n+nb}{type}\PYG{p}{(}\PYG{n}{np}\PYG{o}{.}\PYG{n}{random}\PYG{o}{.}\PYG{n}{random\PYGZus{}sample}\PYG{p}{(}\PYG{p}{)}\PYG{p}{)}
\PYG{g+go}{\PYGZlt{}type \PYGZsq{}float\PYGZsq{}\PYGZgt{}}
\PYG{g+gp}{\PYGZgt{}\PYGZgt{}\PYGZgt{} }\PYG{n}{np}\PYG{o}{.}\PYG{n}{random}\PYG{o}{.}\PYG{n}{random\PYGZus{}sample}\PYG{p}{(}\PYG{p}{(}\PYG{l+m+mi}{5}\PYG{p}{,}\PYG{p}{)}\PYG{p}{)}
\PYG{g+go}{array([ 0.30220482,  0.86820401,  0.1654503 ,  0.11659149,  0.54323428])}
\end{Verbatim}

Three-by-two array of random numbers from {[}-5, 0):

\begin{Verbatim}[commandchars=\\\{\}]
\PYG{g+gp}{\PYGZgt{}\PYGZgt{}\PYGZgt{} }\PYG{l+m+mi}{5} \PYG{o}{*} \PYG{n}{np}\PYG{o}{.}\PYG{n}{random}\PYG{o}{.}\PYG{n}{random\PYGZus{}sample}\PYG{p}{(}\PYG{p}{(}\PYG{l+m+mi}{3}\PYG{p}{,} \PYG{l+m+mi}{2}\PYG{p}{)}\PYG{p}{)} \PYG{o}{\PYGZhy{}} \PYG{l+m+mi}{5}
\PYG{g+go}{array([[\PYGZhy{}3.99149989, \PYGZhy{}0.52338984],}
\PYG{g+go}{       [\PYGZhy{}2.99091858, \PYGZhy{}0.79479508],}
\PYG{g+go}{       [\PYGZhy{}1.23204345, \PYGZhy{}1.75224494]])}
\end{Verbatim}

\end{fulllineitems}

\index{rayleigh() (in module acsStatesAnalysis)}

\begin{fulllineitems}
\phantomsection\label{acsStatesAnalysis:acsStatesAnalysis.rayleigh}\pysiglinewithargsret{\code{acsStatesAnalysis.}\bfcode{rayleigh}}{\emph{scale=1.0}, \emph{size=None}}{}
Draw samples from a Rayleigh distribution.

The \(\chi\) and Weibull distributions are generalizations of the
Rayleigh.
\begin{description}
\item[{scale}] \leavevmode{[}scalar{]}
Scale, also equals the mode. Should be \textgreater{}= 0.

\item[{size}] \leavevmode{[}int or tuple of ints, optional{]}
Shape of the output. Default is None, in which case a single
value is returned.

\end{description}

The probability density function for the Rayleigh distribution is
\begin{gather}
\begin{split}P(x;scale) = \frac{x}{scale^2}e^{\frac{-x^2}{2 \cdotp scale^2}}\end{split}\notag
\end{gather}
The Rayleigh distribution arises if the wind speed and wind direction are
both gaussian variables, then the vector wind velocity forms a Rayleigh
distribution. The Rayleigh distribution is used to model the expected
output from wind turbines.

Draw values from the distribution and plot the histogram

\begin{Verbatim}[commandchars=\\\{\}]
\PYG{g+gp}{\PYGZgt{}\PYGZgt{}\PYGZgt{} }\PYG{n}{values} \PYG{o}{=} \PYG{n}{hist}\PYG{p}{(}\PYG{n}{np}\PYG{o}{.}\PYG{n}{random}\PYG{o}{.}\PYG{n}{rayleigh}\PYG{p}{(}\PYG{l+m+mi}{3}\PYG{p}{,} \PYG{l+m+mi}{100000}\PYG{p}{)}\PYG{p}{,} \PYG{n}{bins}\PYG{o}{=}\PYG{l+m+mi}{200}\PYG{p}{,} \PYG{n}{normed}\PYG{o}{=}\PYG{n+nb+bp}{True}\PYG{p}{)}
\end{Verbatim}

Wave heights tend to follow a Rayleigh distribution. If the mean wave
height is 1 meter, what fraction of waves are likely to be larger than 3
meters?

\begin{Verbatim}[commandchars=\\\{\}]
\PYG{g+gp}{\PYGZgt{}\PYGZgt{}\PYGZgt{} }\PYG{n}{meanvalue} \PYG{o}{=} \PYG{l+m+mi}{1}
\PYG{g+gp}{\PYGZgt{}\PYGZgt{}\PYGZgt{} }\PYG{n}{modevalue} \PYG{o}{=} \PYG{n}{np}\PYG{o}{.}\PYG{n}{sqrt}\PYG{p}{(}\PYG{l+m+mi}{2} \PYG{o}{/} \PYG{n}{np}\PYG{o}{.}\PYG{n}{pi}\PYG{p}{)} \PYG{o}{*} \PYG{n}{meanvalue}
\PYG{g+gp}{\PYGZgt{}\PYGZgt{}\PYGZgt{} }\PYG{n}{s} \PYG{o}{=} \PYG{n}{np}\PYG{o}{.}\PYG{n}{random}\PYG{o}{.}\PYG{n}{rayleigh}\PYG{p}{(}\PYG{n}{modevalue}\PYG{p}{,} \PYG{l+m+mi}{1000000}\PYG{p}{)}
\end{Verbatim}

The percentage of waves larger than 3 meters is:

\begin{Verbatim}[commandchars=\\\{\}]
\PYG{g+gp}{\PYGZgt{}\PYGZgt{}\PYGZgt{} }\PYG{l+m+mf}{100.}\PYG{o}{*}\PYG{n+nb}{sum}\PYG{p}{(}\PYG{n}{s}\PYG{o}{\PYGZgt{}}\PYG{l+m+mi}{3}\PYG{p}{)}\PYG{o}{/}\PYG{l+m+mf}{1000000.}
\PYG{g+go}{0.087300000000000003}
\end{Verbatim}

\end{fulllineitems}

\index{returnZeroSpeciesList() (in module acsStatesAnalysis)}

\begin{fulllineitems}
\phantomsection\label{acsStatesAnalysis:acsStatesAnalysis.returnZeroSpeciesList}\pysiglinewithargsret{\code{acsStatesAnalysis.}\bfcode{returnZeroSpeciesList}}{\emph{tmpLastSpeciesFile}}{}
Function to create a zero vector for each species (NO COMPLEXES)

\end{fulllineitems}

\index{sample() (in module acsStatesAnalysis)}

\begin{fulllineitems}
\phantomsection\label{acsStatesAnalysis:acsStatesAnalysis.sample}\pysiglinewithargsret{\code{acsStatesAnalysis.}\bfcode{sample}}{}{}
random\_sample(size=None)

Return random floats in the half-open interval {[}0.0, 1.0).

Results are from the ``continuous uniform'' distribution over the
stated interval.  To sample \(Unif[a, b), b > a\) multiply
the output of \emph{random\_sample} by \emph{(b-a)} and add \emph{a}:

\begin{Verbatim}[commandchars=\\\{\}]
\PYG{p}{(}\PYG{n}{b} \PYG{o}{\PYGZhy{}} \PYG{n}{a}\PYG{p}{)} \PYG{o}{*} \PYG{n}{random\PYGZus{}sample}\PYG{p}{(}\PYG{p}{)} \PYG{o}{+} \PYG{n}{a}
\end{Verbatim}
\begin{description}
\item[{size}] \leavevmode{[}int or tuple of ints, optional{]}
Defines the shape of the returned array of random floats. If None
(the default), returns a single float.

\end{description}
\begin{description}
\item[{out}] \leavevmode{[}float or ndarray of floats{]}
Array of random floats of shape \emph{size} (unless \code{size=None}, in which
case a single float is returned).

\end{description}

\begin{Verbatim}[commandchars=\\\{\}]
\PYG{g+gp}{\PYGZgt{}\PYGZgt{}\PYGZgt{} }\PYG{n}{np}\PYG{o}{.}\PYG{n}{random}\PYG{o}{.}\PYG{n}{random\PYGZus{}sample}\PYG{p}{(}\PYG{p}{)}
\PYG{g+go}{0.47108547995356098}
\PYG{g+gp}{\PYGZgt{}\PYGZgt{}\PYGZgt{} }\PYG{n+nb}{type}\PYG{p}{(}\PYG{n}{np}\PYG{o}{.}\PYG{n}{random}\PYG{o}{.}\PYG{n}{random\PYGZus{}sample}\PYG{p}{(}\PYG{p}{)}\PYG{p}{)}
\PYG{g+go}{\PYGZlt{}type \PYGZsq{}float\PYGZsq{}\PYGZgt{}}
\PYG{g+gp}{\PYGZgt{}\PYGZgt{}\PYGZgt{} }\PYG{n}{np}\PYG{o}{.}\PYG{n}{random}\PYG{o}{.}\PYG{n}{random\PYGZus{}sample}\PYG{p}{(}\PYG{p}{(}\PYG{l+m+mi}{5}\PYG{p}{,}\PYG{p}{)}\PYG{p}{)}
\PYG{g+go}{array([ 0.30220482,  0.86820401,  0.1654503 ,  0.11659149,  0.54323428])}
\end{Verbatim}

Three-by-two array of random numbers from {[}-5, 0):

\begin{Verbatim}[commandchars=\\\{\}]
\PYG{g+gp}{\PYGZgt{}\PYGZgt{}\PYGZgt{} }\PYG{l+m+mi}{5} \PYG{o}{*} \PYG{n}{np}\PYG{o}{.}\PYG{n}{random}\PYG{o}{.}\PYG{n}{random\PYGZus{}sample}\PYG{p}{(}\PYG{p}{(}\PYG{l+m+mi}{3}\PYG{p}{,} \PYG{l+m+mi}{2}\PYG{p}{)}\PYG{p}{)} \PYG{o}{\PYGZhy{}} \PYG{l+m+mi}{5}
\PYG{g+go}{array([[\PYGZhy{}3.99149989, \PYGZhy{}0.52338984],}
\PYG{g+go}{       [\PYGZhy{}2.99091858, \PYGZhy{}0.79479508],}
\PYG{g+go}{       [\PYGZhy{}1.23204345, \PYGZhy{}1.75224494]])}
\end{Verbatim}

\end{fulllineitems}

\index{seed() (in module acsStatesAnalysis)}

\begin{fulllineitems}
\phantomsection\label{acsStatesAnalysis:acsStatesAnalysis.seed}\pysiglinewithargsret{\code{acsStatesAnalysis.}\bfcode{seed}}{\emph{seed=None}}{}
Seed the generator.

This method is called when \emph{RandomState} is initialized. It can be
called again to re-seed the generator. For details, see \emph{RandomState}.
\begin{description}
\item[{seed}] \leavevmode{[}int or array\_like, optional{]}
Seed for \emph{RandomState}.

\end{description}

RandomState

\end{fulllineitems}

\index{set\_state() (in module acsStatesAnalysis)}

\begin{fulllineitems}
\phantomsection\label{acsStatesAnalysis:acsStatesAnalysis.set_state}\pysiglinewithargsret{\code{acsStatesAnalysis.}\bfcode{set\_state}}{\emph{state}}{}
Set the internal state of the generator from a tuple.

For use if one has reason to manually (re-)set the internal state of the
``Mersenne Twister''{\color{red}\bfseries{}{[}1{]}\_} pseudo-random number generating algorithm.
\begin{description}
\item[{state}] \leavevmode{[}tuple(str, ndarray of 624 uints, int, int, float){]}
The \emph{state} tuple has the following items:
\begin{enumerate}
\item {} 
the string `MT19937', specifying the Mersenne Twister algorithm.

\item {} 
a 1-D array of 624 unsigned integers \code{keys}.

\item {} 
an integer \code{pos}.

\item {} 
an integer \code{has\_gauss}.

\item {} 
a float \code{cached\_gaussian}.

\end{enumerate}

\end{description}
\begin{description}
\item[{out}] \leavevmode{[}None{]}
Returns `None' on success.

\end{description}

get\_state

\emph{set\_state} and \emph{get\_state} are not needed to work with any of the
random distributions in NumPy. If the internal state is manually altered,
the user should know exactly what he/she is doing.

For backwards compatibility, the form (str, array of 624 uints, int) is
also accepted although it is missing some information about the cached
Gaussian value: \code{state = ('MT19937', keys, pos)}.

\end{fulllineitems}

\index{shuffle() (in module acsStatesAnalysis)}

\begin{fulllineitems}
\phantomsection\label{acsStatesAnalysis:acsStatesAnalysis.shuffle}\pysiglinewithargsret{\code{acsStatesAnalysis.}\bfcode{shuffle}}{\emph{x}}{}
Modify a sequence in-place by shuffling its contents.
\begin{description}
\item[{x}] \leavevmode{[}array\_like{]}
The array or list to be shuffled.

\end{description}

None

\begin{Verbatim}[commandchars=\\\{\}]
\PYG{g+gp}{\PYGZgt{}\PYGZgt{}\PYGZgt{} }\PYG{n}{arr} \PYG{o}{=} \PYG{n}{np}\PYG{o}{.}\PYG{n}{arange}\PYG{p}{(}\PYG{l+m+mi}{10}\PYG{p}{)}
\PYG{g+gp}{\PYGZgt{}\PYGZgt{}\PYGZgt{} }\PYG{n}{np}\PYG{o}{.}\PYG{n}{random}\PYG{o}{.}\PYG{n}{shuffle}\PYG{p}{(}\PYG{n}{arr}\PYG{p}{)}
\PYG{g+gp}{\PYGZgt{}\PYGZgt{}\PYGZgt{} }\PYG{n}{arr}
\PYG{g+go}{[1 7 5 2 9 4 3 6 0 8]}
\end{Verbatim}

This function only shuffles the array along the first index of a
multi-dimensional array:

\begin{Verbatim}[commandchars=\\\{\}]
\PYG{g+gp}{\PYGZgt{}\PYGZgt{}\PYGZgt{} }\PYG{n}{arr} \PYG{o}{=} \PYG{n}{np}\PYG{o}{.}\PYG{n}{arange}\PYG{p}{(}\PYG{l+m+mi}{9}\PYG{p}{)}\PYG{o}{.}\PYG{n}{reshape}\PYG{p}{(}\PYG{p}{(}\PYG{l+m+mi}{3}\PYG{p}{,} \PYG{l+m+mi}{3}\PYG{p}{)}\PYG{p}{)}
\PYG{g+gp}{\PYGZgt{}\PYGZgt{}\PYGZgt{} }\PYG{n}{np}\PYG{o}{.}\PYG{n}{random}\PYG{o}{.}\PYG{n}{shuffle}\PYG{p}{(}\PYG{n}{arr}\PYG{p}{)}
\PYG{g+gp}{\PYGZgt{}\PYGZgt{}\PYGZgt{} }\PYG{n}{arr}
\PYG{g+go}{array([[3, 4, 5],}
\PYG{g+go}{       [6, 7, 8],}
\PYG{g+go}{       [0, 1, 2]])}
\end{Verbatim}

\end{fulllineitems}

\index{standard\_cauchy() (in module acsStatesAnalysis)}

\begin{fulllineitems}
\phantomsection\label{acsStatesAnalysis:acsStatesAnalysis.standard_cauchy}\pysiglinewithargsret{\code{acsStatesAnalysis.}\bfcode{standard\_cauchy}}{\emph{size=None}}{}
Standard Cauchy distribution with mode = 0.

Also known as the Lorentz distribution.
\begin{description}
\item[{size}] \leavevmode{[}int or tuple of ints{]}
Shape of the output.

\end{description}
\begin{description}
\item[{samples}] \leavevmode{[}ndarray or scalar{]}
The drawn samples.

\end{description}

The probability density function for the full Cauchy distribution is
\begin{gather}
\begin{split}P(x; x_0, \gamma) = \frac{1}{\pi \gamma \bigl[ 1+
(\frac{x-x_0}{\gamma})^2 \bigr] }\end{split}\notag
\end{gather}
and the Standard Cauchy distribution just sets \(x_0=0\) and
\(\gamma=1\)

The Cauchy distribution arises in the solution to the driven harmonic
oscillator problem, and also describes spectral line broadening. It
also describes the distribution of values at which a line tilted at
a random angle will cut the x axis.

When studying hypothesis tests that assume normality, seeing how the
tests perform on data from a Cauchy distribution is a good indicator of
their sensitivity to a heavy-tailed distribution, since the Cauchy looks
very much like a Gaussian distribution, but with heavier tails.

Draw samples and plot the distribution:

\begin{Verbatim}[commandchars=\\\{\}]
\PYG{g+gp}{\PYGZgt{}\PYGZgt{}\PYGZgt{} }\PYG{n}{s} \PYG{o}{=} \PYG{n}{np}\PYG{o}{.}\PYG{n}{random}\PYG{o}{.}\PYG{n}{standard\PYGZus{}cauchy}\PYG{p}{(}\PYG{l+m+mi}{1000000}\PYG{p}{)}
\PYG{g+gp}{\PYGZgt{}\PYGZgt{}\PYGZgt{} }\PYG{n}{s} \PYG{o}{=} \PYG{n}{s}\PYG{p}{[}\PYG{p}{(}\PYG{n}{s}\PYG{o}{\PYGZgt{}}\PYG{o}{\PYGZhy{}}\PYG{l+m+mi}{25}\PYG{p}{)} \PYG{o}{\PYGZam{}} \PYG{p}{(}\PYG{n}{s}\PYG{o}{\PYGZlt{}}\PYG{l+m+mi}{25}\PYG{p}{)}\PYG{p}{]}  \PYG{c}{\PYGZsh{} truncate distribution so it plots well}
\PYG{g+gp}{\PYGZgt{}\PYGZgt{}\PYGZgt{} }\PYG{n}{plt}\PYG{o}{.}\PYG{n}{hist}\PYG{p}{(}\PYG{n}{s}\PYG{p}{,} \PYG{n}{bins}\PYG{o}{=}\PYG{l+m+mi}{100}\PYG{p}{)}
\PYG{g+gp}{\PYGZgt{}\PYGZgt{}\PYGZgt{} }\PYG{n}{plt}\PYG{o}{.}\PYG{n}{show}\PYG{p}{(}\PYG{p}{)}
\end{Verbatim}

\end{fulllineitems}

\index{standard\_exponential() (in module acsStatesAnalysis)}

\begin{fulllineitems}
\phantomsection\label{acsStatesAnalysis:acsStatesAnalysis.standard_exponential}\pysiglinewithargsret{\code{acsStatesAnalysis.}\bfcode{standard\_exponential}}{\emph{size=None}}{}
Draw samples from the standard exponential distribution.

\emph{standard\_exponential} is identical to the exponential distribution
with a scale parameter of 1.
\begin{description}
\item[{size}] \leavevmode{[}int or tuple of ints{]}
Shape of the output.

\end{description}
\begin{description}
\item[{out}] \leavevmode{[}float or ndarray{]}
Drawn samples.

\end{description}

Output a 3x8000 array:

\begin{Verbatim}[commandchars=\\\{\}]
\PYG{g+gp}{\PYGZgt{}\PYGZgt{}\PYGZgt{} }\PYG{n}{n} \PYG{o}{=} \PYG{n}{np}\PYG{o}{.}\PYG{n}{random}\PYG{o}{.}\PYG{n}{standard\PYGZus{}exponential}\PYG{p}{(}\PYG{p}{(}\PYG{l+m+mi}{3}\PYG{p}{,} \PYG{l+m+mi}{8000}\PYG{p}{)}\PYG{p}{)}
\end{Verbatim}

\end{fulllineitems}

\index{standard\_gamma() (in module acsStatesAnalysis)}

\begin{fulllineitems}
\phantomsection\label{acsStatesAnalysis:acsStatesAnalysis.standard_gamma}\pysiglinewithargsret{\code{acsStatesAnalysis.}\bfcode{standard\_gamma}}{\emph{shape}, \emph{size=None}}{}
Draw samples from a Standard Gamma distribution.

Samples are drawn from a Gamma distribution with specified parameters,
shape (sometimes designated ``k'') and scale=1.
\begin{description}
\item[{shape}] \leavevmode{[}float{]}
Parameter, should be \textgreater{} 0.

\item[{size}] \leavevmode{[}int or tuple of ints{]}
Output shape.  If the given shape is, e.g., \code{(m, n, k)}, then
\code{m * n * k} samples are drawn.

\end{description}
\begin{description}
\item[{samples}] \leavevmode{[}ndarray or scalar{]}
The drawn samples.

\end{description}
\begin{description}
\item[{scipy.stats.distributions.gamma}] \leavevmode{[}probability density function,{]}
distribution or cumulative density function, etc.

\end{description}

The probability density for the Gamma distribution is
\begin{gather}
\begin{split}p(x) = x^{k-1}\frac{e^{-x/\theta}}{\theta^k\Gamma(k)},\end{split}\notag
\end{gather}
where \(k\) is the shape and \(\theta\) the scale,
and \(\Gamma\) is the Gamma function.

The Gamma distribution is often used to model the times to failure of
electronic components, and arises naturally in processes for which the
waiting times between Poisson distributed events are relevant.

Draw samples from the distribution:

\begin{Verbatim}[commandchars=\\\{\}]
\PYG{g+gp}{\PYGZgt{}\PYGZgt{}\PYGZgt{} }\PYG{n}{shape}\PYG{p}{,} \PYG{n}{scale} \PYG{o}{=} \PYG{l+m+mf}{2.}\PYG{p}{,} \PYG{l+m+mf}{1.} \PYG{c}{\PYGZsh{} mean and width}
\PYG{g+gp}{\PYGZgt{}\PYGZgt{}\PYGZgt{} }\PYG{n}{s} \PYG{o}{=} \PYG{n}{np}\PYG{o}{.}\PYG{n}{random}\PYG{o}{.}\PYG{n}{standard\PYGZus{}gamma}\PYG{p}{(}\PYG{n}{shape}\PYG{p}{,} \PYG{l+m+mi}{1000000}\PYG{p}{)}
\end{Verbatim}

Display the histogram of the samples, along with
the probability density function:

\begin{Verbatim}[commandchars=\\\{\}]
\PYG{g+gp}{\PYGZgt{}\PYGZgt{}\PYGZgt{} }\PYG{k+kn}{import} \PYG{n+nn}{matplotlib.pyplot} \PYG{k+kn}{as} \PYG{n+nn}{plt}
\PYG{g+gp}{\PYGZgt{}\PYGZgt{}\PYGZgt{} }\PYG{k+kn}{import} \PYG{n+nn}{scipy.special} \PYG{k+kn}{as} \PYG{n+nn}{sps}
\PYG{g+gp}{\PYGZgt{}\PYGZgt{}\PYGZgt{} }\PYG{n}{count}\PYG{p}{,} \PYG{n}{bins}\PYG{p}{,} \PYG{n}{ignored} \PYG{o}{=} \PYG{n}{plt}\PYG{o}{.}\PYG{n}{hist}\PYG{p}{(}\PYG{n}{s}\PYG{p}{,} \PYG{l+m+mi}{50}\PYG{p}{,} \PYG{n}{normed}\PYG{o}{=}\PYG{n+nb+bp}{True}\PYG{p}{)}
\PYG{g+gp}{\PYGZgt{}\PYGZgt{}\PYGZgt{} }\PYG{n}{y} \PYG{o}{=} \PYG{n}{bins}\PYG{o}{*}\PYG{o}{*}\PYG{p}{(}\PYG{n}{shape}\PYG{o}{\PYGZhy{}}\PYG{l+m+mi}{1}\PYG{p}{)} \PYG{o}{*} \PYG{p}{(}\PYG{p}{(}\PYG{n}{np}\PYG{o}{.}\PYG{n}{exp}\PYG{p}{(}\PYG{o}{\PYGZhy{}}\PYG{n}{bins}\PYG{o}{/}\PYG{n}{scale}\PYG{p}{)}\PYG{p}{)}\PYG{o}{/} \PYGZbs{}
\PYG{g+gp}{... }                      \PYG{p}{(}\PYG{n}{sps}\PYG{o}{.}\PYG{n}{gamma}\PYG{p}{(}\PYG{n}{shape}\PYG{p}{)} \PYG{o}{*} \PYG{n}{scale}\PYG{o}{*}\PYG{o}{*}\PYG{n}{shape}\PYG{p}{)}\PYG{p}{)}
\PYG{g+gp}{\PYGZgt{}\PYGZgt{}\PYGZgt{} }\PYG{n}{plt}\PYG{o}{.}\PYG{n}{plot}\PYG{p}{(}\PYG{n}{bins}\PYG{p}{,} \PYG{n}{y}\PYG{p}{,} \PYG{n}{linewidth}\PYG{o}{=}\PYG{l+m+mi}{2}\PYG{p}{,} \PYG{n}{color}\PYG{o}{=}\PYG{l+s}{\PYGZsq{}}\PYG{l+s}{r}\PYG{l+s}{\PYGZsq{}}\PYG{p}{)}
\PYG{g+gp}{\PYGZgt{}\PYGZgt{}\PYGZgt{} }\PYG{n}{plt}\PYG{o}{.}\PYG{n}{show}\PYG{p}{(}\PYG{p}{)}
\end{Verbatim}

\end{fulllineitems}

\index{standard\_normal() (in module acsStatesAnalysis)}

\begin{fulllineitems}
\phantomsection\label{acsStatesAnalysis:acsStatesAnalysis.standard_normal}\pysiglinewithargsret{\code{acsStatesAnalysis.}\bfcode{standard\_normal}}{\emph{size=None}}{}
Returns samples from a Standard Normal distribution (mean=0, stdev=1).
\begin{description}
\item[{size}] \leavevmode{[}int or tuple of ints, optional{]}
Output shape. Default is None, in which case a single value is
returned.

\end{description}
\begin{description}
\item[{out}] \leavevmode{[}float or ndarray{]}
Drawn samples.

\end{description}

\begin{Verbatim}[commandchars=\\\{\}]
\PYG{g+gp}{\PYGZgt{}\PYGZgt{}\PYGZgt{} }\PYG{n}{s} \PYG{o}{=} \PYG{n}{np}\PYG{o}{.}\PYG{n}{random}\PYG{o}{.}\PYG{n}{standard\PYGZus{}normal}\PYG{p}{(}\PYG{l+m+mi}{8000}\PYG{p}{)}
\PYG{g+gp}{\PYGZgt{}\PYGZgt{}\PYGZgt{} }\PYG{n}{s}
\PYG{g+go}{array([ 0.6888893 ,  0.78096262, \PYGZhy{}0.89086505, ...,  0.49876311, \PYGZsh{}random}
\PYG{g+go}{       \PYGZhy{}0.38672696, \PYGZhy{}0.4685006 ])                               \PYGZsh{}random}
\PYG{g+gp}{\PYGZgt{}\PYGZgt{}\PYGZgt{} }\PYG{n}{s}\PYG{o}{.}\PYG{n}{shape}
\PYG{g+go}{(8000,)}
\PYG{g+gp}{\PYGZgt{}\PYGZgt{}\PYGZgt{} }\PYG{n}{s} \PYG{o}{=} \PYG{n}{np}\PYG{o}{.}\PYG{n}{random}\PYG{o}{.}\PYG{n}{standard\PYGZus{}normal}\PYG{p}{(}\PYG{n}{size}\PYG{o}{=}\PYG{p}{(}\PYG{l+m+mi}{3}\PYG{p}{,} \PYG{l+m+mi}{4}\PYG{p}{,} \PYG{l+m+mi}{2}\PYG{p}{)}\PYG{p}{)}
\PYG{g+gp}{\PYGZgt{}\PYGZgt{}\PYGZgt{} }\PYG{n}{s}\PYG{o}{.}\PYG{n}{shape}
\PYG{g+go}{(3, 4, 2)}
\end{Verbatim}

\end{fulllineitems}

\index{standard\_t() (in module acsStatesAnalysis)}

\begin{fulllineitems}
\phantomsection\label{acsStatesAnalysis:acsStatesAnalysis.standard_t}\pysiglinewithargsret{\code{acsStatesAnalysis.}\bfcode{standard\_t}}{\emph{df}, \emph{size=None}}{}
Standard Student's t distribution with df degrees of freedom.

A special case of the hyperbolic distribution.
As \emph{df} gets large, the result resembles that of the standard normal
distribution (\emph{standard\_normal}).
\begin{description}
\item[{df}] \leavevmode{[}int{]}
Degrees of freedom, should be \textgreater{} 0.

\item[{size}] \leavevmode{[}int or tuple of ints, optional{]}
Output shape. Default is None, in which case a single value is
returned.

\end{description}
\begin{description}
\item[{samples}] \leavevmode{[}ndarray or scalar{]}
Drawn samples.

\end{description}

The probability density function for the t distribution is
\begin{gather}
\begin{split}P(x, df) = \frac{\Gamma(\frac{df+1}{2})}{\sqrt{\pi df}
\Gamma(\frac{df}{2})}\Bigl( 1+\frac{x^2}{df} \Bigr)^{-(df+1)/2}\end{split}\notag
\end{gather}
The t test is based on an assumption that the data come from a Normal
distribution. The t test provides a way to test whether the sample mean
(that is the mean calculated from the data) is a good estimate of the true
mean.

The derivation of the t-distribution was forst published in 1908 by William
Gisset while working for the Guinness Brewery in Dublin. Due to proprietary
issues, he had to publish under a pseudonym, and so he used the name
Student.

From Dalgaard page 83 {\color{red}\bfseries{}{[}1{]}\_}, suppose the daily energy intake for 11
women in Kj is:

\begin{Verbatim}[commandchars=\\\{\}]
\PYG{g+gp}{\PYGZgt{}\PYGZgt{}\PYGZgt{} }\PYG{n}{intake} \PYG{o}{=} \PYG{n}{np}\PYG{o}{.}\PYG{n}{array}\PYG{p}{(}\PYG{p}{[}\PYG{l+m+mf}{5260.}\PYG{p}{,} \PYG{l+m+mi}{5470}\PYG{p}{,} \PYG{l+m+mi}{5640}\PYG{p}{,} \PYG{l+m+mi}{6180}\PYG{p}{,} \PYG{l+m+mi}{6390}\PYG{p}{,} \PYG{l+m+mi}{6515}\PYG{p}{,} \PYG{l+m+mi}{6805}\PYG{p}{,} \PYG{l+m+mi}{7515}\PYG{p}{,} \PYGZbs{}
\PYG{g+gp}{... }                   \PYG{l+m+mi}{7515}\PYG{p}{,} \PYG{l+m+mi}{8230}\PYG{p}{,} \PYG{l+m+mi}{8770}\PYG{p}{]}\PYG{p}{)}
\end{Verbatim}

Does their energy intake deviate systematically from the recommended
value of 7725 kJ?

We have 10 degrees of freedom, so is the sample mean within 95\% of the
recommended value?

\begin{Verbatim}[commandchars=\\\{\}]
\PYG{g+gp}{\PYGZgt{}\PYGZgt{}\PYGZgt{} }\PYG{n}{s} \PYG{o}{=} \PYG{n}{np}\PYG{o}{.}\PYG{n}{random}\PYG{o}{.}\PYG{n}{standard\PYGZus{}t}\PYG{p}{(}\PYG{l+m+mi}{10}\PYG{p}{,} \PYG{n}{size}\PYG{o}{=}\PYG{l+m+mi}{100000}\PYG{p}{)}
\PYG{g+gp}{\PYGZgt{}\PYGZgt{}\PYGZgt{} }\PYG{n}{np}\PYG{o}{.}\PYG{n}{mean}\PYG{p}{(}\PYG{n}{intake}\PYG{p}{)}
\PYG{g+go}{6753.636363636364}
\PYG{g+gp}{\PYGZgt{}\PYGZgt{}\PYGZgt{} }\PYG{n}{intake}\PYG{o}{.}\PYG{n}{std}\PYG{p}{(}\PYG{n}{ddof}\PYG{o}{=}\PYG{l+m+mi}{1}\PYG{p}{)}
\PYG{g+go}{1142.1232221373727}
\end{Verbatim}

Calculate the t statistic, setting the ddof parameter to the unbiased
value so the divisor in the standard deviation will be degrees of
freedom, N-1.

\begin{Verbatim}[commandchars=\\\{\}]
\PYG{g+gp}{\PYGZgt{}\PYGZgt{}\PYGZgt{} }\PYG{n}{t} \PYG{o}{=} \PYG{p}{(}\PYG{n}{np}\PYG{o}{.}\PYG{n}{mean}\PYG{p}{(}\PYG{n}{intake}\PYG{p}{)}\PYG{o}{\PYGZhy{}}\PYG{l+m+mi}{7725}\PYG{p}{)}\PYG{o}{/}\PYG{p}{(}\PYG{n}{intake}\PYG{o}{.}\PYG{n}{std}\PYG{p}{(}\PYG{n}{ddof}\PYG{o}{=}\PYG{l+m+mi}{1}\PYG{p}{)}\PYG{o}{/}\PYG{n}{np}\PYG{o}{.}\PYG{n}{sqrt}\PYG{p}{(}\PYG{n+nb}{len}\PYG{p}{(}\PYG{n}{intake}\PYG{p}{)}\PYG{p}{)}\PYG{p}{)}
\PYG{g+gp}{\PYGZgt{}\PYGZgt{}\PYGZgt{} }\PYG{k+kn}{import} \PYG{n+nn}{matplotlib.pyplot} \PYG{k+kn}{as} \PYG{n+nn}{plt}
\PYG{g+gp}{\PYGZgt{}\PYGZgt{}\PYGZgt{} }\PYG{n}{h} \PYG{o}{=} \PYG{n}{plt}\PYG{o}{.}\PYG{n}{hist}\PYG{p}{(}\PYG{n}{s}\PYG{p}{,} \PYG{n}{bins}\PYG{o}{=}\PYG{l+m+mi}{100}\PYG{p}{,} \PYG{n}{normed}\PYG{o}{=}\PYG{n+nb+bp}{True}\PYG{p}{)}
\end{Verbatim}

For a one-sided t-test, how far out in the distribution does the t
statistic appear?

\begin{Verbatim}[commandchars=\\\{\}]
\PYG{g+gp}{\PYGZgt{}\PYGZgt{}\PYGZgt{} }\PYG{o}{\PYGZgt{}\PYGZgt{}}\PYG{o}{\PYGZgt{}} \PYG{n}{np}\PYG{o}{.}\PYG{n}{sum}\PYG{p}{(}\PYG{n}{s}\PYG{o}{\PYGZlt{}}\PYG{n}{t}\PYG{p}{)} \PYG{o}{/} \PYG{n+nb}{float}\PYG{p}{(}\PYG{n+nb}{len}\PYG{p}{(}\PYG{n}{s}\PYG{p}{)}\PYG{p}{)}
\PYG{g+go}{0.0090699999999999999  \PYGZsh{}random}
\end{Verbatim}

So the p-value is about 0.009, which says the null hypothesis has a
probability of about 99\% of being true.

\end{fulllineitems}

\index{triangular() (in module acsStatesAnalysis)}

\begin{fulllineitems}
\phantomsection\label{acsStatesAnalysis:acsStatesAnalysis.triangular}\pysiglinewithargsret{\code{acsStatesAnalysis.}\bfcode{triangular}}{\emph{left}, \emph{mode}, \emph{right}, \emph{size=None}}{}
Draw samples from the triangular distribution.

The triangular distribution is a continuous probability distribution with
lower limit left, peak at mode, and upper limit right. Unlike the other
distributions, these parameters directly define the shape of the pdf.
\begin{description}
\item[{left}] \leavevmode{[}scalar{]}
Lower limit.

\item[{mode}] \leavevmode{[}scalar{]}
The value where the peak of the distribution occurs.
The value should fulfill the condition \code{left \textless{}= mode \textless{}= right}.

\item[{right}] \leavevmode{[}scalar{]}
Upper limit, should be larger than \emph{left}.

\item[{size}] \leavevmode{[}int or tuple of ints, optional{]}
Output shape. Default is None, in which case a single value is
returned.

\end{description}
\begin{description}
\item[{samples}] \leavevmode{[}ndarray or scalar{]}
The returned samples all lie in the interval {[}left, right{]}.

\end{description}

The probability density function for the Triangular distribution is
\begin{gather}
\begin{split}P(x;l, m, r) = \begin{cases}
\frac{2(x-l)}{(r-l)(m-l)}& \text{for $l \leq x \leq m$},\\
\frac{2(m-x)}{(r-l)(r-m)}& \text{for $m \leq x \leq r$},\\
0& \text{otherwise}.
\end{cases}\end{split}\notag
\end{gather}
The triangular distribution is often used in ill-defined problems where the
underlying distribution is not known, but some knowledge of the limits and
mode exists. Often it is used in simulations.

Draw values from the distribution and plot the histogram:

\begin{Verbatim}[commandchars=\\\{\}]
\PYG{g+gp}{\PYGZgt{}\PYGZgt{}\PYGZgt{} }\PYG{k+kn}{import} \PYG{n+nn}{matplotlib.pyplot} \PYG{k+kn}{as} \PYG{n+nn}{plt}
\PYG{g+gp}{\PYGZgt{}\PYGZgt{}\PYGZgt{} }\PYG{n}{h} \PYG{o}{=} \PYG{n}{plt}\PYG{o}{.}\PYG{n}{hist}\PYG{p}{(}\PYG{n}{np}\PYG{o}{.}\PYG{n}{random}\PYG{o}{.}\PYG{n}{triangular}\PYG{p}{(}\PYG{o}{\PYGZhy{}}\PYG{l+m+mi}{3}\PYG{p}{,} \PYG{l+m+mi}{0}\PYG{p}{,} \PYG{l+m+mi}{8}\PYG{p}{,} \PYG{l+m+mi}{100000}\PYG{p}{)}\PYG{p}{,} \PYG{n}{bins}\PYG{o}{=}\PYG{l+m+mi}{200}\PYG{p}{,}
\PYG{g+gp}{... }             \PYG{n}{normed}\PYG{o}{=}\PYG{n+nb+bp}{True}\PYG{p}{)}
\PYG{g+gp}{\PYGZgt{}\PYGZgt{}\PYGZgt{} }\PYG{n}{plt}\PYG{o}{.}\PYG{n}{show}\PYG{p}{(}\PYG{p}{)}
\end{Verbatim}

\end{fulllineitems}

\index{uniform() (in module acsStatesAnalysis)}

\begin{fulllineitems}
\phantomsection\label{acsStatesAnalysis:acsStatesAnalysis.uniform}\pysiglinewithargsret{\code{acsStatesAnalysis.}\bfcode{uniform}}{\emph{low=0.0}, \emph{high=1.0}, \emph{size=1}}{}
Draw samples from a uniform distribution.

Samples are uniformly distributed over the half-open interval
\code{{[}low, high)} (includes low, but excludes high).  In other words,
any value within the given interval is equally likely to be drawn
by \emph{uniform}.
\begin{description}
\item[{low}] \leavevmode{[}float, optional{]}
Lower boundary of the output interval.  All values generated will be
greater than or equal to low.  The default value is 0.

\item[{high}] \leavevmode{[}float{]}
Upper boundary of the output interval.  All values generated will be
less than high.  The default value is 1.0.

\item[{size}] \leavevmode{[}int or tuple of ints, optional{]}
Shape of output.  If the given size is, for example, (m,n,k),
m*n*k samples are generated.  If no shape is specified, a single sample
is returned.

\end{description}
\begin{description}
\item[{out}] \leavevmode{[}ndarray{]}
Drawn samples, with shape \emph{size}.

\end{description}

randint : Discrete uniform distribution, yielding integers.
random\_integers : Discrete uniform distribution over the closed
\begin{quote}

interval \code{{[}low, high{]}}.
\end{quote}

random\_sample : Floats uniformly distributed over \code{{[}0, 1)}.
random : Alias for \emph{random\_sample}.
rand : Convenience function that accepts dimensions as input, e.g.,
\begin{quote}

\code{rand(2,2)} would generate a 2-by-2 array of floats,
uniformly distributed over \code{{[}0, 1)}.
\end{quote}

The probability density function of the uniform distribution is
\begin{gather}
\begin{split}p(x) = \frac{1}{b - a}\end{split}\notag
\end{gather}
anywhere within the interval \code{{[}a, b)}, and zero elsewhere.

Draw samples from the distribution:

\begin{Verbatim}[commandchars=\\\{\}]
\PYG{g+gp}{\PYGZgt{}\PYGZgt{}\PYGZgt{} }\PYG{n}{s} \PYG{o}{=} \PYG{n}{np}\PYG{o}{.}\PYG{n}{random}\PYG{o}{.}\PYG{n}{uniform}\PYG{p}{(}\PYG{o}{\PYGZhy{}}\PYG{l+m+mi}{1}\PYG{p}{,}\PYG{l+m+mi}{0}\PYG{p}{,}\PYG{l+m+mi}{1000}\PYG{p}{)}
\end{Verbatim}

All values are within the given interval:

\begin{Verbatim}[commandchars=\\\{\}]
\PYG{g+gp}{\PYGZgt{}\PYGZgt{}\PYGZgt{} }\PYG{n}{np}\PYG{o}{.}\PYG{n}{all}\PYG{p}{(}\PYG{n}{s} \PYG{o}{\PYGZgt{}}\PYG{o}{=} \PYG{o}{\PYGZhy{}}\PYG{l+m+mi}{1}\PYG{p}{)}
\PYG{g+go}{True}
\PYG{g+gp}{\PYGZgt{}\PYGZgt{}\PYGZgt{} }\PYG{n}{np}\PYG{o}{.}\PYG{n}{all}\PYG{p}{(}\PYG{n}{s} \PYG{o}{\PYGZlt{}} \PYG{l+m+mi}{0}\PYG{p}{)}
\PYG{g+go}{True}
\end{Verbatim}

Display the histogram of the samples, along with the
probability density function:

\begin{Verbatim}[commandchars=\\\{\}]
\PYG{g+gp}{\PYGZgt{}\PYGZgt{}\PYGZgt{} }\PYG{k+kn}{import} \PYG{n+nn}{matplotlib.pyplot} \PYG{k+kn}{as} \PYG{n+nn}{plt}
\PYG{g+gp}{\PYGZgt{}\PYGZgt{}\PYGZgt{} }\PYG{n}{count}\PYG{p}{,} \PYG{n}{bins}\PYG{p}{,} \PYG{n}{ignored} \PYG{o}{=} \PYG{n}{plt}\PYG{o}{.}\PYG{n}{hist}\PYG{p}{(}\PYG{n}{s}\PYG{p}{,} \PYG{l+m+mi}{15}\PYG{p}{,} \PYG{n}{normed}\PYG{o}{=}\PYG{n+nb+bp}{True}\PYG{p}{)}
\PYG{g+gp}{\PYGZgt{}\PYGZgt{}\PYGZgt{} }\PYG{n}{plt}\PYG{o}{.}\PYG{n}{plot}\PYG{p}{(}\PYG{n}{bins}\PYG{p}{,} \PYG{n}{np}\PYG{o}{.}\PYG{n}{ones\PYGZus{}like}\PYG{p}{(}\PYG{n}{bins}\PYG{p}{)}\PYG{p}{,} \PYG{n}{linewidth}\PYG{o}{=}\PYG{l+m+mi}{2}\PYG{p}{,} \PYG{n}{color}\PYG{o}{=}\PYG{l+s}{\PYGZsq{}}\PYG{l+s}{r}\PYG{l+s}{\PYGZsq{}}\PYG{p}{)}
\PYG{g+gp}{\PYGZgt{}\PYGZgt{}\PYGZgt{} }\PYG{n}{plt}\PYG{o}{.}\PYG{n}{show}\PYG{p}{(}\PYG{p}{)}
\end{Verbatim}

\end{fulllineitems}

\index{vonmises() (in module acsStatesAnalysis)}

\begin{fulllineitems}
\phantomsection\label{acsStatesAnalysis:acsStatesAnalysis.vonmises}\pysiglinewithargsret{\code{acsStatesAnalysis.}\bfcode{vonmises}}{\emph{mu}, \emph{kappa}, \emph{size=None}}{}
Draw samples from a von Mises distribution.

Samples are drawn from a von Mises distribution with specified mode
(mu) and dispersion (kappa), on the interval {[}-pi, pi{]}.

The von Mises distribution (also known as the circular normal
distribution) is a continuous probability distribution on the unit
circle.  It may be thought of as the circular analogue of the normal
distribution.
\begin{description}
\item[{mu}] \leavevmode{[}float{]}
Mode (``center'') of the distribution.

\item[{kappa}] \leavevmode{[}float{]}
Dispersion of the distribution, has to be \textgreater{}=0.

\item[{size}] \leavevmode{[}int or tuple of int{]}
Output shape.  If the given shape is, e.g., \code{(m, n, k)}, then
\code{m * n * k} samples are drawn.

\end{description}
\begin{description}
\item[{samples}] \leavevmode{[}scalar or ndarray{]}
The returned samples, which are in the interval {[}-pi, pi{]}.

\end{description}
\begin{description}
\item[{scipy.stats.distributions.vonmises}] \leavevmode{[}probability density function,{]}
distribution, or cumulative density function, etc.

\end{description}

The probability density for the von Mises distribution is
\begin{gather}
\begin{split}p(x) = \frac{e^{\kappa cos(x-\mu)}}{2\pi I_0(\kappa)},\end{split}\notag
\end{gather}
where \(\mu\) is the mode and \(\kappa\) the dispersion,
and \(I_0(\kappa)\) is the modified Bessel function of order 0.

The von Mises is named for Richard Edler von Mises, who was born in
Austria-Hungary, in what is now the Ukraine.  He fled to the United
States in 1939 and became a professor at Harvard.  He worked in
probability theory, aerodynamics, fluid mechanics, and philosophy of
science.

Abramowitz, M. and Stegun, I. A. (ed.), \emph{Handbook of Mathematical
Functions}, New York: Dover, 1965.

von Mises, R., \emph{Mathematical Theory of Probability and Statistics},
New York: Academic Press, 1964.

Draw samples from the distribution:

\begin{Verbatim}[commandchars=\\\{\}]
\PYG{g+gp}{\PYGZgt{}\PYGZgt{}\PYGZgt{} }\PYG{n}{mu}\PYG{p}{,} \PYG{n}{kappa} \PYG{o}{=} \PYG{l+m+mf}{0.0}\PYG{p}{,} \PYG{l+m+mf}{4.0} \PYG{c}{\PYGZsh{} mean and dispersion}
\PYG{g+gp}{\PYGZgt{}\PYGZgt{}\PYGZgt{} }\PYG{n}{s} \PYG{o}{=} \PYG{n}{np}\PYG{o}{.}\PYG{n}{random}\PYG{o}{.}\PYG{n}{vonmises}\PYG{p}{(}\PYG{n}{mu}\PYG{p}{,} \PYG{n}{kappa}\PYG{p}{,} \PYG{l+m+mi}{1000}\PYG{p}{)}
\end{Verbatim}

Display the histogram of the samples, along with
the probability density function:

\begin{Verbatim}[commandchars=\\\{\}]
\PYG{g+gp}{\PYGZgt{}\PYGZgt{}\PYGZgt{} }\PYG{k+kn}{import} \PYG{n+nn}{matplotlib.pyplot} \PYG{k+kn}{as} \PYG{n+nn}{plt}
\PYG{g+gp}{\PYGZgt{}\PYGZgt{}\PYGZgt{} }\PYG{k+kn}{import} \PYG{n+nn}{scipy.special} \PYG{k+kn}{as} \PYG{n+nn}{sps}
\PYG{g+gp}{\PYGZgt{}\PYGZgt{}\PYGZgt{} }\PYG{n}{count}\PYG{p}{,} \PYG{n}{bins}\PYG{p}{,} \PYG{n}{ignored} \PYG{o}{=} \PYG{n}{plt}\PYG{o}{.}\PYG{n}{hist}\PYG{p}{(}\PYG{n}{s}\PYG{p}{,} \PYG{l+m+mi}{50}\PYG{p}{,} \PYG{n}{normed}\PYG{o}{=}\PYG{n+nb+bp}{True}\PYG{p}{)}
\PYG{g+gp}{\PYGZgt{}\PYGZgt{}\PYGZgt{} }\PYG{n}{x} \PYG{o}{=} \PYG{n}{np}\PYG{o}{.}\PYG{n}{arange}\PYG{p}{(}\PYG{o}{\PYGZhy{}}\PYG{n}{np}\PYG{o}{.}\PYG{n}{pi}\PYG{p}{,} \PYG{n}{np}\PYG{o}{.}\PYG{n}{pi}\PYG{p}{,} \PYG{l+m+mi}{2}\PYG{o}{*}\PYG{n}{np}\PYG{o}{.}\PYG{n}{pi}\PYG{o}{/}\PYG{l+m+mf}{50.}\PYG{p}{)}
\PYG{g+gp}{\PYGZgt{}\PYGZgt{}\PYGZgt{} }\PYG{n}{y} \PYG{o}{=} \PYG{o}{\PYGZhy{}}\PYG{n}{np}\PYG{o}{.}\PYG{n}{exp}\PYG{p}{(}\PYG{n}{kappa}\PYG{o}{*}\PYG{n}{np}\PYG{o}{.}\PYG{n}{cos}\PYG{p}{(}\PYG{n}{x}\PYG{o}{\PYGZhy{}}\PYG{n}{mu}\PYG{p}{)}\PYG{p}{)}\PYG{o}{/}\PYG{p}{(}\PYG{l+m+mi}{2}\PYG{o}{*}\PYG{n}{np}\PYG{o}{.}\PYG{n}{pi}\PYG{o}{*}\PYG{n}{sps}\PYG{o}{.}\PYG{n}{jn}\PYG{p}{(}\PYG{l+m+mi}{0}\PYG{p}{,}\PYG{n}{kappa}\PYG{p}{)}\PYG{p}{)}
\PYG{g+gp}{\PYGZgt{}\PYGZgt{}\PYGZgt{} }\PYG{n}{plt}\PYG{o}{.}\PYG{n}{plot}\PYG{p}{(}\PYG{n}{x}\PYG{p}{,} \PYG{n}{y}\PYG{o}{/}\PYG{n+nb}{max}\PYG{p}{(}\PYG{n}{y}\PYG{p}{)}\PYG{p}{,} \PYG{n}{linewidth}\PYG{o}{=}\PYG{l+m+mi}{2}\PYG{p}{,} \PYG{n}{color}\PYG{o}{=}\PYG{l+s}{\PYGZsq{}}\PYG{l+s}{r}\PYG{l+s}{\PYGZsq{}}\PYG{p}{)}
\PYG{g+gp}{\PYGZgt{}\PYGZgt{}\PYGZgt{} }\PYG{n}{plt}\PYG{o}{.}\PYG{n}{show}\PYG{p}{(}\PYG{p}{)}
\end{Verbatim}

\end{fulllineitems}

\index{wald() (in module acsStatesAnalysis)}

\begin{fulllineitems}
\phantomsection\label{acsStatesAnalysis:acsStatesAnalysis.wald}\pysiglinewithargsret{\code{acsStatesAnalysis.}\bfcode{wald}}{\emph{mean}, \emph{scale}, \emph{size=None}}{}
Draw samples from a Wald, or Inverse Gaussian, distribution.

As the scale approaches infinity, the distribution becomes more like a
Gaussian.

Some references claim that the Wald is an Inverse Gaussian with mean=1, but
this is by no means universal.

The Inverse Gaussian distribution was first studied in relationship to
Brownian motion. In 1956 M.C.K. Tweedie used the name Inverse Gaussian
because there is an inverse relationship between the time to cover a unit
distance and distance covered in unit time.
\begin{description}
\item[{mean}] \leavevmode{[}scalar{]}
Distribution mean, should be \textgreater{} 0.

\item[{scale}] \leavevmode{[}scalar{]}
Scale parameter, should be \textgreater{}= 0.

\item[{size}] \leavevmode{[}int or tuple of ints, optional{]}
Output shape. Default is None, in which case a single value is
returned.

\end{description}
\begin{description}
\item[{samples}] \leavevmode{[}ndarray or scalar{]}
Drawn sample, all greater than zero.

\end{description}

The probability density function for the Wald distribution is
\begin{gather}
\begin{split}P(x;mean,scale) = \sqrt{\frac{scale}{2\pi x^3}}e^
\frac{-scale(x-mean)^2}{2\cdotp mean^2x}\end{split}\notag
\end{gather}
As noted above the Inverse Gaussian distribution first arise from attempts
to model Brownian Motion. It is also a competitor to the Weibull for use in
reliability modeling and modeling stock returns and interest rate
processes.

Draw values from the distribution and plot the histogram:

\begin{Verbatim}[commandchars=\\\{\}]
\PYG{g+gp}{\PYGZgt{}\PYGZgt{}\PYGZgt{} }\PYG{k+kn}{import} \PYG{n+nn}{matplotlib.pyplot} \PYG{k+kn}{as} \PYG{n+nn}{plt}
\PYG{g+gp}{\PYGZgt{}\PYGZgt{}\PYGZgt{} }\PYG{n}{h} \PYG{o}{=} \PYG{n}{plt}\PYG{o}{.}\PYG{n}{hist}\PYG{p}{(}\PYG{n}{np}\PYG{o}{.}\PYG{n}{random}\PYG{o}{.}\PYG{n}{wald}\PYG{p}{(}\PYG{l+m+mi}{3}\PYG{p}{,} \PYG{l+m+mi}{2}\PYG{p}{,} \PYG{l+m+mi}{100000}\PYG{p}{)}\PYG{p}{,} \PYG{n}{bins}\PYG{o}{=}\PYG{l+m+mi}{200}\PYG{p}{,} \PYG{n}{normed}\PYG{o}{=}\PYG{n+nb+bp}{True}\PYG{p}{)}
\PYG{g+gp}{\PYGZgt{}\PYGZgt{}\PYGZgt{} }\PYG{n}{plt}\PYG{o}{.}\PYG{n}{show}\PYG{p}{(}\PYG{p}{)}
\end{Verbatim}

\end{fulllineitems}

\index{weibull() (in module acsStatesAnalysis)}

\begin{fulllineitems}
\phantomsection\label{acsStatesAnalysis:acsStatesAnalysis.weibull}\pysiglinewithargsret{\code{acsStatesAnalysis.}\bfcode{weibull}}{\emph{a}, \emph{size=None}}{}
Weibull distribution.

Draw samples from a 1-parameter Weibull distribution with the given
shape parameter \emph{a}.
\begin{gather}
\begin{split}X = (-ln(U))^{1/a}\end{split}\notag
\end{gather}
Here, U is drawn from the uniform distribution over (0,1{]}.

The more common 2-parameter Weibull, including a scale parameter
\(\lambda\) is just \(X = \lambda(-ln(U))^{1/a}\).
\begin{description}
\item[{a}] \leavevmode{[}float{]}
Shape of the distribution.

\item[{size}] \leavevmode{[}tuple of ints{]}
Output shape.  If the given shape is, e.g., \code{(m, n, k)}, then
\code{m * n * k} samples are drawn.

\end{description}

scipy.stats.distributions.weibull\_max
scipy.stats.distributions.weibull\_min
scipy.stats.distributions.genextreme
gumbel

The Weibull (or Type III asymptotic extreme value distribution for smallest
values, SEV Type III, or Rosin-Rammler distribution) is one of a class of
Generalized Extreme Value (GEV) distributions used in modeling extreme
value problems.  This class includes the Gumbel and Frechet distributions.

The probability density for the Weibull distribution is
\begin{gather}
\begin{split}p(x) = \frac{a}
{\lambda}(\frac{x}{\lambda})^{a-1}e^{-(x/\lambda)^a},\end{split}\notag
\end{gather}
where \(a\) is the shape and \(\lambda\) the scale.

The function has its peak (the mode) at
\(\lambda(\frac{a-1}{a})^{1/a}\).

When \code{a = 1}, the Weibull distribution reduces to the exponential
distribution.

Draw samples from the distribution:

\begin{Verbatim}[commandchars=\\\{\}]
\PYG{g+gp}{\PYGZgt{}\PYGZgt{}\PYGZgt{} }\PYG{n}{a} \PYG{o}{=} \PYG{l+m+mf}{5.} \PYG{c}{\PYGZsh{} shape}
\PYG{g+gp}{\PYGZgt{}\PYGZgt{}\PYGZgt{} }\PYG{n}{s} \PYG{o}{=} \PYG{n}{np}\PYG{o}{.}\PYG{n}{random}\PYG{o}{.}\PYG{n}{weibull}\PYG{p}{(}\PYG{n}{a}\PYG{p}{,} \PYG{l+m+mi}{1000}\PYG{p}{)}
\end{Verbatim}

Display the histogram of the samples, along with
the probability density function:

\begin{Verbatim}[commandchars=\\\{\}]
\PYG{g+gp}{\PYGZgt{}\PYGZgt{}\PYGZgt{} }\PYG{k+kn}{import} \PYG{n+nn}{matplotlib.pyplot} \PYG{k+kn}{as} \PYG{n+nn}{plt}
\PYG{g+gp}{\PYGZgt{}\PYGZgt{}\PYGZgt{} }\PYG{n}{x} \PYG{o}{=} \PYG{n}{np}\PYG{o}{.}\PYG{n}{arange}\PYG{p}{(}\PYG{l+m+mi}{1}\PYG{p}{,}\PYG{l+m+mf}{100.}\PYG{p}{)}\PYG{o}{/}\PYG{l+m+mf}{50.}
\PYG{g+gp}{\PYGZgt{}\PYGZgt{}\PYGZgt{} }\PYG{k}{def} \PYG{n+nf}{weib}\PYG{p}{(}\PYG{n}{x}\PYG{p}{,}\PYG{n}{n}\PYG{p}{,}\PYG{n}{a}\PYG{p}{)}\PYG{p}{:}
\PYG{g+gp}{... }    \PYG{k}{return} \PYG{p}{(}\PYG{n}{a} \PYG{o}{/} \PYG{n}{n}\PYG{p}{)} \PYG{o}{*} \PYG{p}{(}\PYG{n}{x} \PYG{o}{/} \PYG{n}{n}\PYG{p}{)}\PYG{o}{*}\PYG{o}{*}\PYG{p}{(}\PYG{n}{a} \PYG{o}{\PYGZhy{}} \PYG{l+m+mi}{1}\PYG{p}{)} \PYG{o}{*} \PYG{n}{np}\PYG{o}{.}\PYG{n}{exp}\PYG{p}{(}\PYG{o}{\PYGZhy{}}\PYG{p}{(}\PYG{n}{x} \PYG{o}{/} \PYG{n}{n}\PYG{p}{)}\PYG{o}{*}\PYG{o}{*}\PYG{n}{a}\PYG{p}{)}
\end{Verbatim}

\begin{Verbatim}[commandchars=\\\{\}]
\PYG{g+gp}{\PYGZgt{}\PYGZgt{}\PYGZgt{} }\PYG{n}{count}\PYG{p}{,} \PYG{n}{bins}\PYG{p}{,} \PYG{n}{ignored} \PYG{o}{=} \PYG{n}{plt}\PYG{o}{.}\PYG{n}{hist}\PYG{p}{(}\PYG{n}{np}\PYG{o}{.}\PYG{n}{random}\PYG{o}{.}\PYG{n}{weibull}\PYG{p}{(}\PYG{l+m+mf}{5.}\PYG{p}{,}\PYG{l+m+mi}{1000}\PYG{p}{)}\PYG{p}{)}
\PYG{g+gp}{\PYGZgt{}\PYGZgt{}\PYGZgt{} }\PYG{n}{x} \PYG{o}{=} \PYG{n}{np}\PYG{o}{.}\PYG{n}{arange}\PYG{p}{(}\PYG{l+m+mi}{1}\PYG{p}{,}\PYG{l+m+mf}{100.}\PYG{p}{)}\PYG{o}{/}\PYG{l+m+mf}{50.}
\PYG{g+gp}{\PYGZgt{}\PYGZgt{}\PYGZgt{} }\PYG{n}{scale} \PYG{o}{=} \PYG{n}{count}\PYG{o}{.}\PYG{n}{max}\PYG{p}{(}\PYG{p}{)}\PYG{o}{/}\PYG{n}{weib}\PYG{p}{(}\PYG{n}{x}\PYG{p}{,} \PYG{l+m+mf}{1.}\PYG{p}{,} \PYG{l+m+mf}{5.}\PYG{p}{)}\PYG{o}{.}\PYG{n}{max}\PYG{p}{(}\PYG{p}{)}
\PYG{g+gp}{\PYGZgt{}\PYGZgt{}\PYGZgt{} }\PYG{n}{plt}\PYG{o}{.}\PYG{n}{plot}\PYG{p}{(}\PYG{n}{x}\PYG{p}{,} \PYG{n}{weib}\PYG{p}{(}\PYG{n}{x}\PYG{p}{,} \PYG{l+m+mf}{1.}\PYG{p}{,} \PYG{l+m+mf}{5.}\PYG{p}{)}\PYG{o}{*}\PYG{n}{scale}\PYG{p}{)}
\PYG{g+gp}{\PYGZgt{}\PYGZgt{}\PYGZgt{} }\PYG{n}{plt}\PYG{o}{.}\PYG{n}{show}\PYG{p}{(}\PYG{p}{)}
\end{Verbatim}

\end{fulllineitems}

\index{zeroBeforeStrNum() (in module acsStatesAnalysis)}

\begin{fulllineitems}
\phantomsection\label{acsStatesAnalysis:acsStatesAnalysis.zeroBeforeStrNum}\pysiglinewithargsret{\code{acsStatesAnalysis.}\bfcode{zeroBeforeStrNum}}{\emph{tmpl}, \emph{tmpL}}{}
Function to create string zero string vector before graph filename.
According to the total number of reactions N zeros will be add before the instant reaction number 
(e.g. reaction 130 of 10000 the string became `00130')

\end{fulllineitems}

\index{zipf() (in module acsStatesAnalysis)}

\begin{fulllineitems}
\phantomsection\label{acsStatesAnalysis:acsStatesAnalysis.zipf}\pysiglinewithargsret{\code{acsStatesAnalysis.}\bfcode{zipf}}{\emph{a}, \emph{size=None}}{}
Draw samples from a Zipf distribution.

Samples are drawn from a Zipf distribution with specified parameter
\emph{a} \textgreater{} 1.

The Zipf distribution (also known as the zeta distribution) is a
continuous probability distribution that satisfies Zipf's law: the
frequency of an item is inversely proportional to its rank in a
frequency table.
\begin{description}
\item[{a}] \leavevmode{[}float \textgreater{} 1{]}
Distribution parameter.

\item[{size}] \leavevmode{[}int or tuple of int, optional{]}
Output shape.  If the given shape is, e.g., \code{(m, n, k)}, then
\code{m * n * k} samples are drawn; a single integer is equivalent in
its result to providing a mono-tuple, i.e., a 1-D array of length
\emph{size} is returned.  The default is None, in which case a single
scalar is returned.

\end{description}
\begin{description}
\item[{samples}] \leavevmode{[}scalar or ndarray{]}
The returned samples are greater than or equal to one.

\end{description}
\begin{description}
\item[{scipy.stats.distributions.zipf}] \leavevmode{[}probability density function,{]}
distribution, or cumulative density function, etc.

\end{description}

The probability density for the Zipf distribution is
\begin{gather}
\begin{split}p(x) = \frac{x^{-a}}{\zeta(a)},\end{split}\notag
\end{gather}
where \(\zeta\) is the Riemann Zeta function.

It is named for the American linguist George Kingsley Zipf, who noted
that the frequency of any word in a sample of a language is inversely
proportional to its rank in the frequency table.

Zipf, G. K., \emph{Selected Studies of the Principle of Relative Frequency
in Language}, Cambridge, MA: Harvard Univ. Press, 1932.

Draw samples from the distribution:

\begin{Verbatim}[commandchars=\\\{\}]
\PYG{g+gp}{\PYGZgt{}\PYGZgt{}\PYGZgt{} }\PYG{n}{a} \PYG{o}{=} \PYG{l+m+mf}{2.} \PYG{c}{\PYGZsh{} parameter}
\PYG{g+gp}{\PYGZgt{}\PYGZgt{}\PYGZgt{} }\PYG{n}{s} \PYG{o}{=} \PYG{n}{np}\PYG{o}{.}\PYG{n}{random}\PYG{o}{.}\PYG{n}{zipf}\PYG{p}{(}\PYG{n}{a}\PYG{p}{,} \PYG{l+m+mi}{1000}\PYG{p}{)}
\end{Verbatim}

Display the histogram of the samples, along with
the probability density function:

\begin{Verbatim}[commandchars=\\\{\}]
\PYG{g+gp}{\PYGZgt{}\PYGZgt{}\PYGZgt{} }\PYG{k+kn}{import} \PYG{n+nn}{matplotlib.pyplot} \PYG{k+kn}{as} \PYG{n+nn}{plt}
\PYG{g+gp}{\PYGZgt{}\PYGZgt{}\PYGZgt{} }\PYG{k+kn}{import} \PYG{n+nn}{scipy.special} \PYG{k+kn}{as} \PYG{n+nn}{sps}
\PYG{g+go}{Truncate s values at 50 so plot is interesting}
\PYG{g+gp}{\PYGZgt{}\PYGZgt{}\PYGZgt{} }\PYG{n}{count}\PYG{p}{,} \PYG{n}{bins}\PYG{p}{,} \PYG{n}{ignored} \PYG{o}{=} \PYG{n}{plt}\PYG{o}{.}\PYG{n}{hist}\PYG{p}{(}\PYG{n}{s}\PYG{p}{[}\PYG{n}{s}\PYG{o}{\PYGZlt{}}\PYG{l+m+mi}{50}\PYG{p}{]}\PYG{p}{,} \PYG{l+m+mi}{50}\PYG{p}{,} \PYG{n}{normed}\PYG{o}{=}\PYG{n+nb+bp}{True}\PYG{p}{)}
\PYG{g+gp}{\PYGZgt{}\PYGZgt{}\PYGZgt{} }\PYG{n}{x} \PYG{o}{=} \PYG{n}{np}\PYG{o}{.}\PYG{n}{arange}\PYG{p}{(}\PYG{l+m+mf}{1.}\PYG{p}{,} \PYG{l+m+mf}{50.}\PYG{p}{)}
\PYG{g+gp}{\PYGZgt{}\PYGZgt{}\PYGZgt{} }\PYG{n}{y} \PYG{o}{=} \PYG{n}{x}\PYG{o}{*}\PYG{o}{*}\PYG{p}{(}\PYG{o}{\PYGZhy{}}\PYG{n}{a}\PYG{p}{)}\PYG{o}{/}\PYG{n}{sps}\PYG{o}{.}\PYG{n}{zetac}\PYG{p}{(}\PYG{n}{a}\PYG{p}{)}
\PYG{g+gp}{\PYGZgt{}\PYGZgt{}\PYGZgt{} }\PYG{n}{plt}\PYG{o}{.}\PYG{n}{plot}\PYG{p}{(}\PYG{n}{x}\PYG{p}{,} \PYG{n}{y}\PYG{o}{/}\PYG{n+nb}{max}\PYG{p}{(}\PYG{n}{y}\PYG{p}{)}\PYG{p}{,} \PYG{n}{linewidth}\PYG{o}{=}\PYG{l+m+mi}{2}\PYG{p}{,} \PYG{n}{color}\PYG{o}{=}\PYG{l+s}{\PYGZsq{}}\PYG{l+s}{r}\PYG{l+s}{\PYGZsq{}}\PYG{p}{)}
\PYG{g+gp}{\PYGZgt{}\PYGZgt{}\PYGZgt{} }\PYG{n}{plt}\PYG{o}{.}\PYG{n}{show}\PYG{p}{(}\PYG{p}{)}
\end{Verbatim}

\end{fulllineitems}



\chapter{initializator Module}
\label{initializator:initializator-module}\label{initializator::doc}\label{initializator:module-initializator}\index{initializator (module)}
Script to initialize random catalytic nets 
python \textless{}path\textgreater{}/GIT/ACS\_analysis/initializator.py -t2 -a0 -o \textasciitilde{}/Documents/lavoro/protocell/init/
\begin{quote}

-k3 -d2 -K10 -f2 -n2 -s6 -m6 -p7 -I \textasciitilde{}/Documents/lavoro/protocell/init/acsm2s.conf 
-N600 -B600 -x1 -O0 -H10 -v2.5 -c0.5 -F TEST -i 1 -S2 -u -P2 -S2 -A0.1
\end{quote}

2015 - experiments on synchronization, init command
python \textless{}path\textgreater{}/initializator.py t2 -H20 -K20 -u -v1.0 -P2 -S2 -F \textless{}folder\_name\textgreater{} -A0.01
and
python \textless{}path\textgreater{}/initializator.py t2 -H20 -K20 -u -v2.5 -P2 -S2 -F \textless{}folder\_name\textgreater{} -A0.01


\chapter{lib Package}
\label{lib:lib-package}\label{lib::doc}

\section{Subpackages}
\label{lib:subpackages}

\subsection{IO Package}
\label{lib.IO:io-package}\label{lib.IO::doc}

\subsubsection{\texttt{IO} Package}
\label{lib.IO:id1}\phantomsection\label{lib.IO:module-lib.IO}\index{lib.IO (module)}

\subsubsection{\texttt{readfiles} Module}
\label{lib.IO:readfiles-module}\label{lib.IO:module-lib.IO.readfiles}\index{lib.IO.readfiles (module)}\index{beta() (in module lib.IO.readfiles)}

\begin{fulllineitems}
\phantomsection\label{lib.IO:lib.IO.readfiles.beta}\pysiglinewithargsret{\code{lib.IO.readfiles.}\bfcode{beta}}{\emph{a}, \emph{b}, \emph{size=None}}{}
The Beta distribution over \code{{[}0, 1{]}}.

The Beta distribution is a special case of the Dirichlet distribution,
and is related to the Gamma distribution.  It has the probability
distribution function
\begin{gather}
\begin{split}f(x; a,b) = \frac{1}{B(\alpha, \beta)} x^{\alpha - 1}
(1 - x)^{\beta - 1},\end{split}\notag
\end{gather}
where the normalisation, B, is the beta function,
\begin{gather}
\begin{split}B(\alpha, \beta) = \int_0^1 t^{\alpha - 1}
(1 - t)^{\beta - 1} dt.\end{split}\notag
\end{gather}
It is often seen in Bayesian inference and order statistics.
\begin{description}
\item[{a}] \leavevmode{[}float{]}
Alpha, non-negative.

\item[{b}] \leavevmode{[}float{]}
Beta, non-negative.

\item[{size}] \leavevmode{[}tuple of ints, optional{]}
The number of samples to draw.  The output is packed according to
the size given.

\end{description}
\begin{description}
\item[{out}] \leavevmode{[}ndarray{]}
Array of the given shape, containing values drawn from a
Beta distribution.

\end{description}

\end{fulllineitems}

\index{binomial() (in module lib.IO.readfiles)}

\begin{fulllineitems}
\phantomsection\label{lib.IO:lib.IO.readfiles.binomial}\pysiglinewithargsret{\code{lib.IO.readfiles.}\bfcode{binomial}}{\emph{n}, \emph{p}, \emph{size=None}}{}
Draw samples from a binomial distribution.

Samples are drawn from a Binomial distribution with specified
parameters, n trials and p probability of success where
n an integer \textgreater{}= 0 and p is in the interval {[}0,1{]}. (n may be
input as a float, but it is truncated to an integer in use)
\begin{description}
\item[{n}] \leavevmode{[}float (but truncated to an integer){]}
parameter, \textgreater{}= 0.

\item[{p}] \leavevmode{[}float{]}
parameter, \textgreater{}= 0 and \textless{}=1.

\item[{size}] \leavevmode{[}\{tuple, int\}{]}
Output shape.  If the given shape is, e.g., \code{(m, n, k)}, then
\code{m * n * k} samples are drawn.

\end{description}
\begin{description}
\item[{samples}] \leavevmode{[}\{ndarray, scalar\}{]}
where the values are all integers in  {[}0, n{]}.

\end{description}
\begin{description}
\item[{scipy.stats.distributions.binom}] \leavevmode{[}probability density function,{]}
distribution or cumulative density function, etc.

\end{description}

The probability density for the Binomial distribution is
\begin{gather}
\begin{split}P(N) = \binom{n}{N}p^N(1-p)^{n-N},\end{split}\notag
\end{gather}
where \(n\) is the number of trials, \(p\) is the probability
of success, and \(N\) is the number of successes.

When estimating the standard error of a proportion in a population by
using a random sample, the normal distribution works well unless the
product p*n \textless{}=5, where p = population proportion estimate, and n =
number of samples, in which case the binomial distribution is used
instead. For example, a sample of 15 people shows 4 who are left
handed, and 11 who are right handed. Then p = 4/15 = 27\%. 0.27*15 = 4,
so the binomial distribution should be used in this case.

Draw samples from the distribution:

\begin{Verbatim}[commandchars=\\\{\}]
\PYG{g+gp}{\PYGZgt{}\PYGZgt{}\PYGZgt{} }\PYG{n}{n}\PYG{p}{,} \PYG{n}{p} \PYG{o}{=} \PYG{l+m+mi}{10}\PYG{p}{,} \PYG{o}{.}\PYG{l+m+mi}{5} \PYG{c}{\PYGZsh{} number of trials, probability of each trial}
\PYG{g+gp}{\PYGZgt{}\PYGZgt{}\PYGZgt{} }\PYG{n}{s} \PYG{o}{=} \PYG{n}{np}\PYG{o}{.}\PYG{n}{random}\PYG{o}{.}\PYG{n}{binomial}\PYG{p}{(}\PYG{n}{n}\PYG{p}{,} \PYG{n}{p}\PYG{p}{,} \PYG{l+m+mi}{1000}\PYG{p}{)}
\PYG{g+go}{\PYGZsh{} result of flipping a coin 10 times, tested 1000 times.}
\end{Verbatim}

A real world example. A company drills 9 wild-cat oil exploration
wells, each with an estimated probability of success of 0.1. All nine
wells fail. What is the probability of that happening?

Let's do 20,000 trials of the model, and count the number that
generate zero positive results.

\begin{Verbatim}[commandchars=\\\{\}]
\PYG{g+gp}{\PYGZgt{}\PYGZgt{}\PYGZgt{} }\PYG{n+nb}{sum}\PYG{p}{(}\PYG{n}{np}\PYG{o}{.}\PYG{n}{random}\PYG{o}{.}\PYG{n}{binomial}\PYG{p}{(}\PYG{l+m+mi}{9}\PYG{p}{,}\PYG{l+m+mf}{0.1}\PYG{p}{,}\PYG{l+m+mi}{20000}\PYG{p}{)}\PYG{o}{==}\PYG{l+m+mi}{0}\PYG{p}{)}\PYG{o}{/}\PYG{l+m+mf}{20000.}
\PYG{g+go}{answer = 0.38885, or 38\PYGZpc{}.}
\end{Verbatim}

\end{fulllineitems}

\index{chisquare() (in module lib.IO.readfiles)}

\begin{fulllineitems}
\phantomsection\label{lib.IO:lib.IO.readfiles.chisquare}\pysiglinewithargsret{\code{lib.IO.readfiles.}\bfcode{chisquare}}{\emph{df}, \emph{size=None}}{}
Draw samples from a chi-square distribution.

When \emph{df} independent random variables, each with standard normal
distributions (mean 0, variance 1), are squared and summed, the
resulting distribution is chi-square (see Notes).  This distribution
is often used in hypothesis testing.
\begin{description}
\item[{df}] \leavevmode{[}int{]}
Number of degrees of freedom.

\item[{size}] \leavevmode{[}tuple of ints, int, optional{]}
Size of the returned array.  By default, a scalar is
returned.

\end{description}
\begin{description}
\item[{output}] \leavevmode{[}ndarray{]}
Samples drawn from the distribution, packed in a \emph{size}-shaped
array.

\end{description}
\begin{description}
\item[{ValueError}] \leavevmode
When \emph{df} \textless{}= 0 or when an inappropriate \emph{size} (e.g. \code{size=-1})
is given.

\end{description}

The variable obtained by summing the squares of \emph{df} independent,
standard normally distributed random variables:
\begin{gather}
\begin{split}Q = \sum_{i=0}^{\mathtt{df}} X^2_i\end{split}\notag
\end{gather}
is chi-square distributed, denoted
\begin{gather}
\begin{split}Q \sim \chi^2_k.\end{split}\notag
\end{gather}
The probability density function of the chi-squared distribution is
\begin{gather}
\begin{split}p(x) = \frac{(1/2)^{k/2}}{\Gamma(k/2)}
x^{k/2 - 1} e^{-x/2},\end{split}\notag
\end{gather}
where \(\Gamma\) is the gamma function,
\begin{gather}
\begin{split}\Gamma(x) = \int_0^{-\infty} t^{x - 1} e^{-t} dt.\end{split}\notag
\end{gather}
\href{http://www.itl.nist.gov/div898/handbook/eda/section3/eda3666.htm}{NIST/SEMATECH e-Handbook of Statistical Methods}

\begin{Verbatim}[commandchars=\\\{\}]
\PYG{g+gp}{\PYGZgt{}\PYGZgt{}\PYGZgt{} }\PYG{n}{np}\PYG{o}{.}\PYG{n}{random}\PYG{o}{.}\PYG{n}{chisquare}\PYG{p}{(}\PYG{l+m+mi}{2}\PYG{p}{,}\PYG{l+m+mi}{4}\PYG{p}{)}
\PYG{g+go}{array([ 1.89920014,  9.00867716,  3.13710533,  5.62318272])}
\end{Verbatim}

\end{fulllineitems}

\index{exponential() (in module lib.IO.readfiles)}

\begin{fulllineitems}
\phantomsection\label{lib.IO:lib.IO.readfiles.exponential}\pysiglinewithargsret{\code{lib.IO.readfiles.}\bfcode{exponential}}{\emph{scale=1.0}, \emph{size=None}}{}
Exponential distribution.

Its probability density function is
\begin{gather}
\begin{split}f(x; \frac{1}{\beta}) = \frac{1}{\beta} \exp(-\frac{x}{\beta}),\end{split}\notag
\end{gather}
for \code{x \textgreater{} 0} and 0 elsewhere. \(\beta\) is the scale parameter,
which is the inverse of the rate parameter \(\lambda = 1/\beta\).
The rate parameter is an alternative, widely used parameterization
of the exponential distribution {\color{red}\bfseries{}{[}3{]}\_}.

The exponential distribution is a continuous analogue of the
geometric distribution.  It describes many common situations, such as
the size of raindrops measured over many rainstorms {\color{red}\bfseries{}{[}1{]}\_}, or the time
between page requests to Wikipedia {\color{red}\bfseries{}{[}2{]}\_}.
\begin{description}
\item[{scale}] \leavevmode{[}float{]}
The scale parameter, \(\beta = 1/\lambda\).

\item[{size}] \leavevmode{[}tuple of ints{]}
Number of samples to draw.  The output is shaped
according to \emph{size}.

\end{description}

\end{fulllineitems}

\index{f() (in module lib.IO.readfiles)}

\begin{fulllineitems}
\phantomsection\label{lib.IO:lib.IO.readfiles.f}\pysiglinewithargsret{\code{lib.IO.readfiles.}\bfcode{f}}{\emph{dfnum}, \emph{dfden}, \emph{size=None}}{}
Draw samples from a F distribution.

Samples are drawn from an F distribution with specified parameters,
\emph{dfnum} (degrees of freedom in numerator) and \emph{dfden} (degrees of freedom
in denominator), where both parameters should be greater than zero.

The random variate of the F distribution (also known as the
Fisher distribution) is a continuous probability distribution
that arises in ANOVA tests, and is the ratio of two chi-square
variates.
\begin{description}
\item[{dfnum}] \leavevmode{[}float{]}
Degrees of freedom in numerator. Should be greater than zero.

\item[{dfden}] \leavevmode{[}float{]}
Degrees of freedom in denominator. Should be greater than zero.

\item[{size}] \leavevmode{[}\{tuple, int\}, optional{]}
Output shape.  If the given shape is, e.g., \code{(m, n, k)},
then \code{m * n * k} samples are drawn. By default only one sample
is returned.

\end{description}
\begin{description}
\item[{samples}] \leavevmode{[}\{ndarray, scalar\}{]}
Samples from the Fisher distribution.

\end{description}
\begin{description}
\item[{scipy.stats.distributions.f}] \leavevmode{[}probability density function,{]}
distribution or cumulative density function, etc.

\end{description}

The F statistic is used to compare in-group variances to between-group
variances. Calculating the distribution depends on the sampling, and
so it is a function of the respective degrees of freedom in the
problem.  The variable \emph{dfnum} is the number of samples minus one, the
between-groups degrees of freedom, while \emph{dfden} is the within-groups
degrees of freedom, the sum of the number of samples in each group
minus the number of groups.

An example from Glantz{[}1{]}, pp 47-40.
Two groups, children of diabetics (25 people) and children from people
without diabetes (25 controls). Fasting blood glucose was measured,
case group had a mean value of 86.1, controls had a mean value of
82.2. Standard deviations were 2.09 and 2.49 respectively. Are these
data consistent with the null hypothesis that the parents diabetic
status does not affect their children's blood glucose levels?
Calculating the F statistic from the data gives a value of 36.01.

Draw samples from the distribution:

\begin{Verbatim}[commandchars=\\\{\}]
\PYG{g+gp}{\PYGZgt{}\PYGZgt{}\PYGZgt{} }\PYG{n}{dfnum} \PYG{o}{=} \PYG{l+m+mf}{1.} \PYG{c}{\PYGZsh{} between group degrees of freedom}
\PYG{g+gp}{\PYGZgt{}\PYGZgt{}\PYGZgt{} }\PYG{n}{dfden} \PYG{o}{=} \PYG{l+m+mf}{48.} \PYG{c}{\PYGZsh{} within groups degrees of freedom}
\PYG{g+gp}{\PYGZgt{}\PYGZgt{}\PYGZgt{} }\PYG{n}{s} \PYG{o}{=} \PYG{n}{np}\PYG{o}{.}\PYG{n}{random}\PYG{o}{.}\PYG{n}{f}\PYG{p}{(}\PYG{n}{dfnum}\PYG{p}{,} \PYG{n}{dfden}\PYG{p}{,} \PYG{l+m+mi}{1000}\PYG{p}{)}
\end{Verbatim}

The lower bound for the top 1\% of the samples is :

\begin{Verbatim}[commandchars=\\\{\}]
\PYG{g+gp}{\PYGZgt{}\PYGZgt{}\PYGZgt{} }\PYG{n}{sort}\PYG{p}{(}\PYG{n}{s}\PYG{p}{)}\PYG{p}{[}\PYG{o}{\PYGZhy{}}\PYG{l+m+mi}{10}\PYG{p}{]}
\PYG{g+go}{7.61988120985}
\end{Verbatim}

So there is about a 1\% chance that the F statistic will exceed 7.62,
the measured value is 36, so the null hypothesis is rejected at the 1\%
level.

\end{fulllineitems}

\index{gamma() (in module lib.IO.readfiles)}

\begin{fulllineitems}
\phantomsection\label{lib.IO:lib.IO.readfiles.gamma}\pysiglinewithargsret{\code{lib.IO.readfiles.}\bfcode{gamma}}{\emph{shape}, \emph{scale=1.0}, \emph{size=None}}{}
Draw samples from a Gamma distribution.

Samples are drawn from a Gamma distribution with specified parameters,
\emph{shape} (sometimes designated ``k'') and \emph{scale} (sometimes designated
``theta''), where both parameters are \textgreater{} 0.
\begin{description}
\item[{shape}] \leavevmode{[}scalar \textgreater{} 0{]}
The shape of the gamma distribution.

\item[{scale}] \leavevmode{[}scalar \textgreater{} 0, optional{]}
The scale of the gamma distribution.  Default is equal to 1.

\item[{size}] \leavevmode{[}shape\_tuple, optional{]}
Output shape.  If the given shape is, e.g., \code{(m, n, k)}, then
\code{m * n * k} samples are drawn.

\end{description}
\begin{description}
\item[{out}] \leavevmode{[}ndarray, float{]}
Returns one sample unless \emph{size} parameter is specified.

\end{description}
\begin{description}
\item[{scipy.stats.distributions.gamma}] \leavevmode{[}probability density function,{]}
distribution or cumulative density function, etc.

\end{description}

The probability density for the Gamma distribution is
\begin{gather}
\begin{split}p(x) = x^{k-1}\frac{e^{-x/\theta}}{\theta^k\Gamma(k)},\end{split}\notag
\end{gather}
where \(k\) is the shape and \(\theta\) the scale,
and \(\Gamma\) is the Gamma function.

The Gamma distribution is often used to model the times to failure of
electronic components, and arises naturally in processes for which the
waiting times between Poisson distributed events are relevant.

Draw samples from the distribution:

\begin{Verbatim}[commandchars=\\\{\}]
\PYG{g+gp}{\PYGZgt{}\PYGZgt{}\PYGZgt{} }\PYG{n}{shape}\PYG{p}{,} \PYG{n}{scale} \PYG{o}{=} \PYG{l+m+mf}{2.}\PYG{p}{,} \PYG{l+m+mf}{2.} \PYG{c}{\PYGZsh{} mean and dispersion}
\PYG{g+gp}{\PYGZgt{}\PYGZgt{}\PYGZgt{} }\PYG{n}{s} \PYG{o}{=} \PYG{n}{np}\PYG{o}{.}\PYG{n}{random}\PYG{o}{.}\PYG{n}{gamma}\PYG{p}{(}\PYG{n}{shape}\PYG{p}{,} \PYG{n}{scale}\PYG{p}{,} \PYG{l+m+mi}{1000}\PYG{p}{)}
\end{Verbatim}

Display the histogram of the samples, along with
the probability density function:

\begin{Verbatim}[commandchars=\\\{\}]
\PYG{g+gp}{\PYGZgt{}\PYGZgt{}\PYGZgt{} }\PYG{k+kn}{import} \PYG{n+nn}{matplotlib.pyplot} \PYG{k+kn}{as} \PYG{n+nn}{plt}
\PYG{g+gp}{\PYGZgt{}\PYGZgt{}\PYGZgt{} }\PYG{k+kn}{import} \PYG{n+nn}{scipy.special} \PYG{k+kn}{as} \PYG{n+nn}{sps}
\PYG{g+gp}{\PYGZgt{}\PYGZgt{}\PYGZgt{} }\PYG{n}{count}\PYG{p}{,} \PYG{n}{bins}\PYG{p}{,} \PYG{n}{ignored} \PYG{o}{=} \PYG{n}{plt}\PYG{o}{.}\PYG{n}{hist}\PYG{p}{(}\PYG{n}{s}\PYG{p}{,} \PYG{l+m+mi}{50}\PYG{p}{,} \PYG{n}{normed}\PYG{o}{=}\PYG{n+nb+bp}{True}\PYG{p}{)}
\PYG{g+gp}{\PYGZgt{}\PYGZgt{}\PYGZgt{} }\PYG{n}{y} \PYG{o}{=} \PYG{n}{bins}\PYG{o}{*}\PYG{o}{*}\PYG{p}{(}\PYG{n}{shape}\PYG{o}{\PYGZhy{}}\PYG{l+m+mi}{1}\PYG{p}{)}\PYG{o}{*}\PYG{p}{(}\PYG{n}{np}\PYG{o}{.}\PYG{n}{exp}\PYG{p}{(}\PYG{o}{\PYGZhy{}}\PYG{n}{bins}\PYG{o}{/}\PYG{n}{scale}\PYG{p}{)} \PYG{o}{/}
\PYG{g+gp}{... }                     \PYG{p}{(}\PYG{n}{sps}\PYG{o}{.}\PYG{n}{gamma}\PYG{p}{(}\PYG{n}{shape}\PYG{p}{)}\PYG{o}{*}\PYG{n}{scale}\PYG{o}{*}\PYG{o}{*}\PYG{n}{shape}\PYG{p}{)}\PYG{p}{)}
\PYG{g+gp}{\PYGZgt{}\PYGZgt{}\PYGZgt{} }\PYG{n}{plt}\PYG{o}{.}\PYG{n}{plot}\PYG{p}{(}\PYG{n}{bins}\PYG{p}{,} \PYG{n}{y}\PYG{p}{,} \PYG{n}{linewidth}\PYG{o}{=}\PYG{l+m+mi}{2}\PYG{p}{,} \PYG{n}{color}\PYG{o}{=}\PYG{l+s}{\PYGZsq{}}\PYG{l+s}{r}\PYG{l+s}{\PYGZsq{}}\PYG{p}{)}
\PYG{g+gp}{\PYGZgt{}\PYGZgt{}\PYGZgt{} }\PYG{n}{plt}\PYG{o}{.}\PYG{n}{show}\PYG{p}{(}\PYG{p}{)}
\end{Verbatim}

\end{fulllineitems}

\index{geometric() (in module lib.IO.readfiles)}

\begin{fulllineitems}
\phantomsection\label{lib.IO:lib.IO.readfiles.geometric}\pysiglinewithargsret{\code{lib.IO.readfiles.}\bfcode{geometric}}{\emph{p}, \emph{size=None}}{}
Draw samples from the geometric distribution.

Bernoulli trials are experiments with one of two outcomes:
success or failure (an example of such an experiment is flipping
a coin).  The geometric distribution models the number of trials
that must be run in order to achieve success.  It is therefore
supported on the positive integers, \code{k = 1, 2, ...}.

The probability mass function of the geometric distribution is
\begin{gather}
\begin{split}f(k) = (1 - p)^{k - 1} p\end{split}\notag
\end{gather}
where \emph{p} is the probability of success of an individual trial.
\begin{description}
\item[{p}] \leavevmode{[}float{]}
The probability of success of an individual trial.

\item[{size}] \leavevmode{[}tuple of ints{]}
Number of values to draw from the distribution.  The output
is shaped according to \emph{size}.

\end{description}
\begin{description}
\item[{out}] \leavevmode{[}ndarray{]}
Samples from the geometric distribution, shaped according to
\emph{size}.

\end{description}

Draw ten thousand values from the geometric distribution,
with the probability of an individual success equal to 0.35:

\begin{Verbatim}[commandchars=\\\{\}]
\PYG{g+gp}{\PYGZgt{}\PYGZgt{}\PYGZgt{} }\PYG{n}{z} \PYG{o}{=} \PYG{n}{np}\PYG{o}{.}\PYG{n}{random}\PYG{o}{.}\PYG{n}{geometric}\PYG{p}{(}\PYG{n}{p}\PYG{o}{=}\PYG{l+m+mf}{0.35}\PYG{p}{,} \PYG{n}{size}\PYG{o}{=}\PYG{l+m+mi}{10000}\PYG{p}{)}
\end{Verbatim}

How many trials succeeded after a single run?

\begin{Verbatim}[commandchars=\\\{\}]
\PYG{g+gp}{\PYGZgt{}\PYGZgt{}\PYGZgt{} }\PYG{p}{(}\PYG{n}{z} \PYG{o}{==} \PYG{l+m+mi}{1}\PYG{p}{)}\PYG{o}{.}\PYG{n}{sum}\PYG{p}{(}\PYG{p}{)} \PYG{o}{/} \PYG{l+m+mf}{10000.}
\PYG{g+go}{0.34889999999999999 \PYGZsh{}random}
\end{Verbatim}

\end{fulllineitems}

\index{get\_state() (in module lib.IO.readfiles)}

\begin{fulllineitems}
\phantomsection\label{lib.IO:lib.IO.readfiles.get_state}\pysiglinewithargsret{\code{lib.IO.readfiles.}\bfcode{get\_state}}{}{}
Return a tuple representing the internal state of the generator.

For more details, see \emph{set\_state}.
\begin{description}
\item[{out}] \leavevmode{[}tuple(str, ndarray of 624 uints, int, int, float){]}
The returned tuple has the following items:
\begin{enumerate}
\item {} 
the string `MT19937'.

\item {} 
a 1-D array of 624 unsigned integer keys.

\item {} 
an integer \code{pos}.

\item {} 
an integer \code{has\_gauss}.

\item {} 
a float \code{cached\_gaussian}.

\end{enumerate}

\end{description}

set\_state

\emph{set\_state} and \emph{get\_state} are not needed to work with any of the
random distributions in NumPy. If the internal state is manually altered,
the user should know exactly what he/she is doing.

\end{fulllineitems}

\index{gumbel() (in module lib.IO.readfiles)}

\begin{fulllineitems}
\phantomsection\label{lib.IO:lib.IO.readfiles.gumbel}\pysiglinewithargsret{\code{lib.IO.readfiles.}\bfcode{gumbel}}{\emph{loc=0.0}, \emph{scale=1.0}, \emph{size=None}}{}
Gumbel distribution.

Draw samples from a Gumbel distribution with specified location and scale.
For more information on the Gumbel distribution, see Notes and References
below.
\begin{description}
\item[{loc}] \leavevmode{[}float{]}
The location of the mode of the distribution.

\item[{scale}] \leavevmode{[}float{]}
The scale parameter of the distribution.

\item[{size}] \leavevmode{[}tuple of ints{]}
Output shape.  If the given shape is, e.g., \code{(m, n, k)}, then
\code{m * n * k} samples are drawn.

\end{description}
\begin{description}
\item[{out}] \leavevmode{[}ndarray{]}
The samples

\end{description}

scipy.stats.gumbel\_l
scipy.stats.gumbel\_r
scipy.stats.genextreme
\begin{quote}

probability density function, distribution, or cumulative density
function, etc. for each of the above
\end{quote}

weibull

The Gumbel (or Smallest Extreme Value (SEV) or the Smallest Extreme Value
Type I) distribution is one of a class of Generalized Extreme Value (GEV)
distributions used in modeling extreme value problems.  The Gumbel is a
special case of the Extreme Value Type I distribution for maximums from
distributions with ``exponential-like'' tails.

The probability density for the Gumbel distribution is
\begin{gather}
\begin{split}p(x) = \frac{e^{-(x - \mu)/ \beta}}{\beta} e^{ -e^{-(x - \mu)/
\beta}},\end{split}\notag
\end{gather}
where \(\mu\) is the mode, a location parameter, and \(\beta\) is
the scale parameter.

The Gumbel (named for German mathematician Emil Julius Gumbel) was used
very early in the hydrology literature, for modeling the occurrence of
flood events. It is also used for modeling maximum wind speed and rainfall
rates.  It is a ``fat-tailed'' distribution - the probability of an event in
the tail of the distribution is larger than if one used a Gaussian, hence
the surprisingly frequent occurrence of 100-year floods. Floods were
initially modeled as a Gaussian process, which underestimated the frequency
of extreme events.

It is one of a class of extreme value distributions, the Generalized
Extreme Value (GEV) distributions, which also includes the Weibull and
Frechet.

The function has a mean of \(\mu + 0.57721\beta\) and a variance of
\(\frac{\pi^2}{6}\beta^2\).

Gumbel, E. J., \emph{Statistics of Extremes}, New York: Columbia University
Press, 1958.

Reiss, R.-D. and Thomas, M., \emph{Statistical Analysis of Extreme Values from
Insurance, Finance, Hydrology and Other Fields}, Basel: Birkhauser Verlag,
2001.

Draw samples from the distribution:

\begin{Verbatim}[commandchars=\\\{\}]
\PYG{g+gp}{\PYGZgt{}\PYGZgt{}\PYGZgt{} }\PYG{n}{mu}\PYG{p}{,} \PYG{n}{beta} \PYG{o}{=} \PYG{l+m+mi}{0}\PYG{p}{,} \PYG{l+m+mf}{0.1} \PYG{c}{\PYGZsh{} location and scale}
\PYG{g+gp}{\PYGZgt{}\PYGZgt{}\PYGZgt{} }\PYG{n}{s} \PYG{o}{=} \PYG{n}{np}\PYG{o}{.}\PYG{n}{random}\PYG{o}{.}\PYG{n}{gumbel}\PYG{p}{(}\PYG{n}{mu}\PYG{p}{,} \PYG{n}{beta}\PYG{p}{,} \PYG{l+m+mi}{1000}\PYG{p}{)}
\end{Verbatim}

Display the histogram of the samples, along with
the probability density function:

\begin{Verbatim}[commandchars=\\\{\}]
\PYG{g+gp}{\PYGZgt{}\PYGZgt{}\PYGZgt{} }\PYG{k+kn}{import} \PYG{n+nn}{matplotlib.pyplot} \PYG{k+kn}{as} \PYG{n+nn}{plt}
\PYG{g+gp}{\PYGZgt{}\PYGZgt{}\PYGZgt{} }\PYG{n}{count}\PYG{p}{,} \PYG{n}{bins}\PYG{p}{,} \PYG{n}{ignored} \PYG{o}{=} \PYG{n}{plt}\PYG{o}{.}\PYG{n}{hist}\PYG{p}{(}\PYG{n}{s}\PYG{p}{,} \PYG{l+m+mi}{30}\PYG{p}{,} \PYG{n}{normed}\PYG{o}{=}\PYG{n+nb+bp}{True}\PYG{p}{)}
\PYG{g+gp}{\PYGZgt{}\PYGZgt{}\PYGZgt{} }\PYG{n}{plt}\PYG{o}{.}\PYG{n}{plot}\PYG{p}{(}\PYG{n}{bins}\PYG{p}{,} \PYG{p}{(}\PYG{l+m+mi}{1}\PYG{o}{/}\PYG{n}{beta}\PYG{p}{)}\PYG{o}{*}\PYG{n}{np}\PYG{o}{.}\PYG{n}{exp}\PYG{p}{(}\PYG{o}{\PYGZhy{}}\PYG{p}{(}\PYG{n}{bins} \PYG{o}{\PYGZhy{}} \PYG{n}{mu}\PYG{p}{)}\PYG{o}{/}\PYG{n}{beta}\PYG{p}{)}
\PYG{g+gp}{... }         \PYG{o}{*} \PYG{n}{np}\PYG{o}{.}\PYG{n}{exp}\PYG{p}{(} \PYG{o}{\PYGZhy{}}\PYG{n}{np}\PYG{o}{.}\PYG{n}{exp}\PYG{p}{(} \PYG{o}{\PYGZhy{}}\PYG{p}{(}\PYG{n}{bins} \PYG{o}{\PYGZhy{}} \PYG{n}{mu}\PYG{p}{)} \PYG{o}{/}\PYG{n}{beta}\PYG{p}{)} \PYG{p}{)}\PYG{p}{,}
\PYG{g+gp}{... }         \PYG{n}{linewidth}\PYG{o}{=}\PYG{l+m+mi}{2}\PYG{p}{,} \PYG{n}{color}\PYG{o}{=}\PYG{l+s}{\PYGZsq{}}\PYG{l+s}{r}\PYG{l+s}{\PYGZsq{}}\PYG{p}{)}
\PYG{g+gp}{\PYGZgt{}\PYGZgt{}\PYGZgt{} }\PYG{n}{plt}\PYG{o}{.}\PYG{n}{show}\PYG{p}{(}\PYG{p}{)}
\end{Verbatim}

Show how an extreme value distribution can arise from a Gaussian process
and compare to a Gaussian:

\begin{Verbatim}[commandchars=\\\{\}]
\PYG{g+gp}{\PYGZgt{}\PYGZgt{}\PYGZgt{} }\PYG{n}{means} \PYG{o}{=} \PYG{p}{[}\PYG{p}{]}
\PYG{g+gp}{\PYGZgt{}\PYGZgt{}\PYGZgt{} }\PYG{n}{maxima} \PYG{o}{=} \PYG{p}{[}\PYG{p}{]}
\PYG{g+gp}{\PYGZgt{}\PYGZgt{}\PYGZgt{} }\PYG{k}{for} \PYG{n}{i} \PYG{o+ow}{in} \PYG{n+nb}{range}\PYG{p}{(}\PYG{l+m+mi}{0}\PYG{p}{,}\PYG{l+m+mi}{1000}\PYG{p}{)} \PYG{p}{:}
\PYG{g+gp}{... }   \PYG{n}{a} \PYG{o}{=} \PYG{n}{np}\PYG{o}{.}\PYG{n}{random}\PYG{o}{.}\PYG{n}{normal}\PYG{p}{(}\PYG{n}{mu}\PYG{p}{,} \PYG{n}{beta}\PYG{p}{,} \PYG{l+m+mi}{1000}\PYG{p}{)}
\PYG{g+gp}{... }   \PYG{n}{means}\PYG{o}{.}\PYG{n}{append}\PYG{p}{(}\PYG{n}{a}\PYG{o}{.}\PYG{n}{mean}\PYG{p}{(}\PYG{p}{)}\PYG{p}{)}
\PYG{g+gp}{... }   \PYG{n}{maxima}\PYG{o}{.}\PYG{n}{append}\PYG{p}{(}\PYG{n}{a}\PYG{o}{.}\PYG{n}{max}\PYG{p}{(}\PYG{p}{)}\PYG{p}{)}
\PYG{g+gp}{\PYGZgt{}\PYGZgt{}\PYGZgt{} }\PYG{n}{count}\PYG{p}{,} \PYG{n}{bins}\PYG{p}{,} \PYG{n}{ignored} \PYG{o}{=} \PYG{n}{plt}\PYG{o}{.}\PYG{n}{hist}\PYG{p}{(}\PYG{n}{maxima}\PYG{p}{,} \PYG{l+m+mi}{30}\PYG{p}{,} \PYG{n}{normed}\PYG{o}{=}\PYG{n+nb+bp}{True}\PYG{p}{)}
\PYG{g+gp}{\PYGZgt{}\PYGZgt{}\PYGZgt{} }\PYG{n}{beta} \PYG{o}{=} \PYG{n}{np}\PYG{o}{.}\PYG{n}{std}\PYG{p}{(}\PYG{n}{maxima}\PYG{p}{)}\PYG{o}{*}\PYG{n}{np}\PYG{o}{.}\PYG{n}{pi}\PYG{o}{/}\PYG{n}{np}\PYG{o}{.}\PYG{n}{sqrt}\PYG{p}{(}\PYG{l+m+mi}{6}\PYG{p}{)}
\PYG{g+gp}{\PYGZgt{}\PYGZgt{}\PYGZgt{} }\PYG{n}{mu} \PYG{o}{=} \PYG{n}{np}\PYG{o}{.}\PYG{n}{mean}\PYG{p}{(}\PYG{n}{maxima}\PYG{p}{)} \PYG{o}{\PYGZhy{}} \PYG{l+m+mf}{0.57721}\PYG{o}{*}\PYG{n}{beta}
\PYG{g+gp}{\PYGZgt{}\PYGZgt{}\PYGZgt{} }\PYG{n}{plt}\PYG{o}{.}\PYG{n}{plot}\PYG{p}{(}\PYG{n}{bins}\PYG{p}{,} \PYG{p}{(}\PYG{l+m+mi}{1}\PYG{o}{/}\PYG{n}{beta}\PYG{p}{)}\PYG{o}{*}\PYG{n}{np}\PYG{o}{.}\PYG{n}{exp}\PYG{p}{(}\PYG{o}{\PYGZhy{}}\PYG{p}{(}\PYG{n}{bins} \PYG{o}{\PYGZhy{}} \PYG{n}{mu}\PYG{p}{)}\PYG{o}{/}\PYG{n}{beta}\PYG{p}{)}
\PYG{g+gp}{... }         \PYG{o}{*} \PYG{n}{np}\PYG{o}{.}\PYG{n}{exp}\PYG{p}{(}\PYG{o}{\PYGZhy{}}\PYG{n}{np}\PYG{o}{.}\PYG{n}{exp}\PYG{p}{(}\PYG{o}{\PYGZhy{}}\PYG{p}{(}\PYG{n}{bins} \PYG{o}{\PYGZhy{}} \PYG{n}{mu}\PYG{p}{)}\PYG{o}{/}\PYG{n}{beta}\PYG{p}{)}\PYG{p}{)}\PYG{p}{,}
\PYG{g+gp}{... }         \PYG{n}{linewidth}\PYG{o}{=}\PYG{l+m+mi}{2}\PYG{p}{,} \PYG{n}{color}\PYG{o}{=}\PYG{l+s}{\PYGZsq{}}\PYG{l+s}{r}\PYG{l+s}{\PYGZsq{}}\PYG{p}{)}
\PYG{g+gp}{\PYGZgt{}\PYGZgt{}\PYGZgt{} }\PYG{n}{plt}\PYG{o}{.}\PYG{n}{plot}\PYG{p}{(}\PYG{n}{bins}\PYG{p}{,} \PYG{l+m+mi}{1}\PYG{o}{/}\PYG{p}{(}\PYG{n}{beta} \PYG{o}{*} \PYG{n}{np}\PYG{o}{.}\PYG{n}{sqrt}\PYG{p}{(}\PYG{l+m+mi}{2} \PYG{o}{*} \PYG{n}{np}\PYG{o}{.}\PYG{n}{pi}\PYG{p}{)}\PYG{p}{)}
\PYG{g+gp}{... }         \PYG{o}{*} \PYG{n}{np}\PYG{o}{.}\PYG{n}{exp}\PYG{p}{(}\PYG{o}{\PYGZhy{}}\PYG{p}{(}\PYG{n}{bins} \PYG{o}{\PYGZhy{}} \PYG{n}{mu}\PYG{p}{)}\PYG{o}{*}\PYG{o}{*}\PYG{l+m+mi}{2} \PYG{o}{/} \PYG{p}{(}\PYG{l+m+mi}{2} \PYG{o}{*} \PYG{n}{beta}\PYG{o}{*}\PYG{o}{*}\PYG{l+m+mi}{2}\PYG{p}{)}\PYG{p}{)}\PYG{p}{,}
\PYG{g+gp}{... }         \PYG{n}{linewidth}\PYG{o}{=}\PYG{l+m+mi}{2}\PYG{p}{,} \PYG{n}{color}\PYG{o}{=}\PYG{l+s}{\PYGZsq{}}\PYG{l+s}{g}\PYG{l+s}{\PYGZsq{}}\PYG{p}{)}
\PYG{g+gp}{\PYGZgt{}\PYGZgt{}\PYGZgt{} }\PYG{n}{plt}\PYG{o}{.}\PYG{n}{show}\PYG{p}{(}\PYG{p}{)}
\end{Verbatim}

\end{fulllineitems}

\index{hypergeometric() (in module lib.IO.readfiles)}

\begin{fulllineitems}
\phantomsection\label{lib.IO:lib.IO.readfiles.hypergeometric}\pysiglinewithargsret{\code{lib.IO.readfiles.}\bfcode{hypergeometric}}{\emph{ngood}, \emph{nbad}, \emph{nsample}, \emph{size=None}}{}
Draw samples from a Hypergeometric distribution.

Samples are drawn from a Hypergeometric distribution with specified
parameters, ngood (ways to make a good selection), nbad (ways to make
a bad selection), and nsample = number of items sampled, which is less
than or equal to the sum ngood + nbad.
\begin{description}
\item[{ngood}] \leavevmode{[}int or array\_like{]}
Number of ways to make a good selection.  Must be nonnegative.

\item[{nbad}] \leavevmode{[}int or array\_like{]}
Number of ways to make a bad selection.  Must be nonnegative.

\item[{nsample}] \leavevmode{[}int or array\_like{]}
Number of items sampled.  Must be at least 1 and at most
\code{ngood + nbad}.

\item[{size}] \leavevmode{[}int or tuple of int{]}
Output shape.  If the given shape is, e.g., \code{(m, n, k)}, then
\code{m * n * k} samples are drawn.

\end{description}
\begin{description}
\item[{samples}] \leavevmode{[}ndarray or scalar{]}
The values are all integers in  {[}0, n{]}.

\end{description}
\begin{description}
\item[{scipy.stats.distributions.hypergeom}] \leavevmode{[}probability density function,{]}
distribution or cumulative density function, etc.

\end{description}

The probability density for the Hypergeometric distribution is
\begin{gather}
\begin{split}P(x) = \frac{\binom{m}{n}\binom{N-m}{n-x}}{\binom{N}{n}},\end{split}\notag
\end{gather}
where \(0 \le x \le m\) and \(n+m-N \le x \le n\)

for P(x) the probability of x successes, n = ngood, m = nbad, and
N = number of samples.

Consider an urn with black and white marbles in it, ngood of them
black and nbad are white. If you draw nsample balls without
replacement, then the Hypergeometric distribution describes the
distribution of black balls in the drawn sample.

Note that this distribution is very similar to the Binomial
distribution, except that in this case, samples are drawn without
replacement, whereas in the Binomial case samples are drawn with
replacement (or the sample space is infinite). As the sample space
becomes large, this distribution approaches the Binomial.

Draw samples from the distribution:

\begin{Verbatim}[commandchars=\\\{\}]
\PYG{g+gp}{\PYGZgt{}\PYGZgt{}\PYGZgt{} }\PYG{n}{ngood}\PYG{p}{,} \PYG{n}{nbad}\PYG{p}{,} \PYG{n}{nsamp} \PYG{o}{=} \PYG{l+m+mi}{100}\PYG{p}{,} \PYG{l+m+mi}{2}\PYG{p}{,} \PYG{l+m+mi}{10}
\PYG{g+go}{\PYGZsh{} number of good, number of bad, and number of samples}
\PYG{g+gp}{\PYGZgt{}\PYGZgt{}\PYGZgt{} }\PYG{n}{s} \PYG{o}{=} \PYG{n}{np}\PYG{o}{.}\PYG{n}{random}\PYG{o}{.}\PYG{n}{hypergeometric}\PYG{p}{(}\PYG{n}{ngood}\PYG{p}{,} \PYG{n}{nbad}\PYG{p}{,} \PYG{n}{nsamp}\PYG{p}{,} \PYG{l+m+mi}{1000}\PYG{p}{)}
\PYG{g+gp}{\PYGZgt{}\PYGZgt{}\PYGZgt{} }\PYG{n}{hist}\PYG{p}{(}\PYG{n}{s}\PYG{p}{)}
\PYG{g+go}{\PYGZsh{}   note that it is very unlikely to grab both bad items}
\end{Verbatim}

Suppose you have an urn with 15 white and 15 black marbles.
If you pull 15 marbles at random, how likely is it that
12 or more of them are one color?

\begin{Verbatim}[commandchars=\\\{\}]
\PYG{g+gp}{\PYGZgt{}\PYGZgt{}\PYGZgt{} }\PYG{n}{s} \PYG{o}{=} \PYG{n}{np}\PYG{o}{.}\PYG{n}{random}\PYG{o}{.}\PYG{n}{hypergeometric}\PYG{p}{(}\PYG{l+m+mi}{15}\PYG{p}{,} \PYG{l+m+mi}{15}\PYG{p}{,} \PYG{l+m+mi}{15}\PYG{p}{,} \PYG{l+m+mi}{100000}\PYG{p}{)}
\PYG{g+gp}{\PYGZgt{}\PYGZgt{}\PYGZgt{} }\PYG{n+nb}{sum}\PYG{p}{(}\PYG{n}{s}\PYG{o}{\PYGZgt{}}\PYG{o}{=}\PYG{l+m+mi}{12}\PYG{p}{)}\PYG{o}{/}\PYG{l+m+mf}{100000.} \PYG{o}{+} \PYG{n+nb}{sum}\PYG{p}{(}\PYG{n}{s}\PYG{o}{\PYGZlt{}}\PYG{o}{=}\PYG{l+m+mi}{3}\PYG{p}{)}\PYG{o}{/}\PYG{l+m+mf}{100000.}
\PYG{g+go}{\PYGZsh{}   answer = 0.003 ... pretty unlikely!}
\end{Verbatim}

\end{fulllineitems}

\index{laplace() (in module lib.IO.readfiles)}

\begin{fulllineitems}
\phantomsection\label{lib.IO:lib.IO.readfiles.laplace}\pysiglinewithargsret{\code{lib.IO.readfiles.}\bfcode{laplace}}{\emph{loc=0.0}, \emph{scale=1.0}, \emph{size=None}}{}
Draw samples from the Laplace or double exponential distribution with
specified location (or mean) and scale (decay).

The Laplace distribution is similar to the Gaussian/normal distribution,
but is sharper at the peak and has fatter tails. It represents the
difference between two independent, identically distributed exponential
random variables.
\begin{description}
\item[{loc}] \leavevmode{[}float{]}
The position, \(\mu\), of the distribution peak.

\item[{scale}] \leavevmode{[}float{]}
\(\lambda\), the exponential decay.

\end{description}

It has the probability density function
\begin{gather}
\begin{split}f(x; \mu, \lambda) = \frac{1}{2\lambda}
\exp\left(-\frac{|x - \mu|}{\lambda}\right).\end{split}\notag
\end{gather}
The first law of Laplace, from 1774, states that the frequency of an error
can be expressed as an exponential function of the absolute magnitude of
the error, which leads to the Laplace distribution. For many problems in
Economics and Health sciences, this distribution seems to model the data
better than the standard Gaussian distribution

Draw samples from the distribution

\begin{Verbatim}[commandchars=\\\{\}]
\PYG{g+gp}{\PYGZgt{}\PYGZgt{}\PYGZgt{} }\PYG{n}{loc}\PYG{p}{,} \PYG{n}{scale} \PYG{o}{=} \PYG{l+m+mf}{0.}\PYG{p}{,} \PYG{l+m+mf}{1.}
\PYG{g+gp}{\PYGZgt{}\PYGZgt{}\PYGZgt{} }\PYG{n}{s} \PYG{o}{=} \PYG{n}{np}\PYG{o}{.}\PYG{n}{random}\PYG{o}{.}\PYG{n}{laplace}\PYG{p}{(}\PYG{n}{loc}\PYG{p}{,} \PYG{n}{scale}\PYG{p}{,} \PYG{l+m+mi}{1000}\PYG{p}{)}
\end{Verbatim}

Display the histogram of the samples, along with
the probability density function:

\begin{Verbatim}[commandchars=\\\{\}]
\PYG{g+gp}{\PYGZgt{}\PYGZgt{}\PYGZgt{} }\PYG{k+kn}{import} \PYG{n+nn}{matplotlib.pyplot} \PYG{k+kn}{as} \PYG{n+nn}{plt}
\PYG{g+gp}{\PYGZgt{}\PYGZgt{}\PYGZgt{} }\PYG{n}{count}\PYG{p}{,} \PYG{n}{bins}\PYG{p}{,} \PYG{n}{ignored} \PYG{o}{=} \PYG{n}{plt}\PYG{o}{.}\PYG{n}{hist}\PYG{p}{(}\PYG{n}{s}\PYG{p}{,} \PYG{l+m+mi}{30}\PYG{p}{,} \PYG{n}{normed}\PYG{o}{=}\PYG{n+nb+bp}{True}\PYG{p}{)}
\PYG{g+gp}{\PYGZgt{}\PYGZgt{}\PYGZgt{} }\PYG{n}{x} \PYG{o}{=} \PYG{n}{np}\PYG{o}{.}\PYG{n}{arange}\PYG{p}{(}\PYG{o}{\PYGZhy{}}\PYG{l+m+mf}{8.}\PYG{p}{,} \PYG{l+m+mf}{8.}\PYG{p}{,} \PYG{o}{.}\PYG{l+m+mo}{01}\PYG{p}{)}
\PYG{g+gp}{\PYGZgt{}\PYGZgt{}\PYGZgt{} }\PYG{n}{pdf} \PYG{o}{=} \PYG{n}{np}\PYG{o}{.}\PYG{n}{exp}\PYG{p}{(}\PYG{o}{\PYGZhy{}}\PYG{n+nb}{abs}\PYG{p}{(}\PYG{n}{x}\PYG{o}{\PYGZhy{}}\PYG{n}{loc}\PYG{o}{/}\PYG{n}{scale}\PYG{p}{)}\PYG{p}{)}\PYG{o}{/}\PYG{p}{(}\PYG{l+m+mf}{2.}\PYG{o}{*}\PYG{n}{scale}\PYG{p}{)}
\PYG{g+gp}{\PYGZgt{}\PYGZgt{}\PYGZgt{} }\PYG{n}{plt}\PYG{o}{.}\PYG{n}{plot}\PYG{p}{(}\PYG{n}{x}\PYG{p}{,} \PYG{n}{pdf}\PYG{p}{)}
\end{Verbatim}

Plot Gaussian for comparison:

\begin{Verbatim}[commandchars=\\\{\}]
\PYG{g+gp}{\PYGZgt{}\PYGZgt{}\PYGZgt{} }\PYG{n}{g} \PYG{o}{=} \PYG{p}{(}\PYG{l+m+mi}{1}\PYG{o}{/}\PYG{p}{(}\PYG{n}{scale} \PYG{o}{*} \PYG{n}{np}\PYG{o}{.}\PYG{n}{sqrt}\PYG{p}{(}\PYG{l+m+mi}{2} \PYG{o}{*} \PYG{n}{np}\PYG{o}{.}\PYG{n}{pi}\PYG{p}{)}\PYG{p}{)} \PYG{o}{*} 
\PYG{g+gp}{... }     \PYG{n}{np}\PYG{o}{.}\PYG{n}{exp}\PYG{p}{(} \PYG{o}{\PYGZhy{}} \PYG{p}{(}\PYG{n}{x} \PYG{o}{\PYGZhy{}} \PYG{n}{loc}\PYG{p}{)}\PYG{o}{*}\PYG{o}{*}\PYG{l+m+mi}{2} \PYG{o}{/} \PYG{p}{(}\PYG{l+m+mi}{2} \PYG{o}{*} \PYG{n}{scale}\PYG{o}{*}\PYG{o}{*}\PYG{l+m+mi}{2}\PYG{p}{)} \PYG{p}{)}\PYG{p}{)}
\PYG{g+gp}{\PYGZgt{}\PYGZgt{}\PYGZgt{} }\PYG{n}{plt}\PYG{o}{.}\PYG{n}{plot}\PYG{p}{(}\PYG{n}{x}\PYG{p}{,}\PYG{n}{g}\PYG{p}{)}
\end{Verbatim}

\end{fulllineitems}

\index{loadAllData() (in module lib.IO.readfiles)}

\begin{fulllineitems}
\phantomsection\label{lib.IO:lib.IO.readfiles.loadAllData}\pysiglinewithargsret{\code{lib.IO.readfiles.}\bfcode{loadAllData}}{\emph{tmpPath}, \emph{tmpFname}}{}
\end{fulllineitems}

\index{loadRandomSeed() (in module lib.IO.readfiles)}

\begin{fulllineitems}
\phantomsection\label{lib.IO:lib.IO.readfiles.loadRandomSeed}\pysiglinewithargsret{\code{lib.IO.readfiles.}\bfcode{loadRandomSeed}}{\emph{tmpRndPath}}{}
Function to load a previously saved random seed

\end{fulllineitems}

\index{logistic() (in module lib.IO.readfiles)}

\begin{fulllineitems}
\phantomsection\label{lib.IO:lib.IO.readfiles.logistic}\pysiglinewithargsret{\code{lib.IO.readfiles.}\bfcode{logistic}}{\emph{loc=0.0}, \emph{scale=1.0}, \emph{size=None}}{}
Draw samples from a Logistic distribution.

Samples are drawn from a Logistic distribution with specified
parameters, loc (location or mean, also median), and scale (\textgreater{}0).

loc : float

scale : float \textgreater{} 0.
\begin{description}
\item[{size}] \leavevmode{[}\{tuple, int\}{]}
Output shape.  If the given shape is, e.g., \code{(m, n, k)}, then
\code{m * n * k} samples are drawn.

\end{description}
\begin{description}
\item[{samples}] \leavevmode{[}\{ndarray, scalar\}{]}
where the values are all integers in  {[}0, n{]}.

\end{description}
\begin{description}
\item[{scipy.stats.distributions.logistic}] \leavevmode{[}probability density function,{]}
distribution or cumulative density function, etc.

\end{description}

The probability density for the Logistic distribution is
\begin{gather}
\begin{split}P(x) = P(x) = \frac{e^{-(x-\mu)/s}}{s(1+e^{-(x-\mu)/s})^2},\end{split}\notag
\end{gather}
where \(\mu\) = location and \(s\) = scale.

The Logistic distribution is used in Extreme Value problems where it
can act as a mixture of Gumbel distributions, in Epidemiology, and by
the World Chess Federation (FIDE) where it is used in the Elo ranking
system, assuming the performance of each player is a logistically
distributed random variable.

Draw samples from the distribution:

\begin{Verbatim}[commandchars=\\\{\}]
\PYG{g+gp}{\PYGZgt{}\PYGZgt{}\PYGZgt{} }\PYG{n}{loc}\PYG{p}{,} \PYG{n}{scale} \PYG{o}{=} \PYG{l+m+mi}{10}\PYG{p}{,} \PYG{l+m+mi}{1}
\PYG{g+gp}{\PYGZgt{}\PYGZgt{}\PYGZgt{} }\PYG{n}{s} \PYG{o}{=} \PYG{n}{np}\PYG{o}{.}\PYG{n}{random}\PYG{o}{.}\PYG{n}{logistic}\PYG{p}{(}\PYG{n}{loc}\PYG{p}{,} \PYG{n}{scale}\PYG{p}{,} \PYG{l+m+mi}{10000}\PYG{p}{)}
\PYG{g+gp}{\PYGZgt{}\PYGZgt{}\PYGZgt{} }\PYG{n}{count}\PYG{p}{,} \PYG{n}{bins}\PYG{p}{,} \PYG{n}{ignored} \PYG{o}{=} \PYG{n}{plt}\PYG{o}{.}\PYG{n}{hist}\PYG{p}{(}\PYG{n}{s}\PYG{p}{,} \PYG{n}{bins}\PYG{o}{=}\PYG{l+m+mi}{50}\PYG{p}{)}
\end{Verbatim}

\#   plot against distribution

\begin{Verbatim}[commandchars=\\\{\}]
\PYG{g+gp}{\PYGZgt{}\PYGZgt{}\PYGZgt{} }\PYG{k}{def} \PYG{n+nf}{logist}\PYG{p}{(}\PYG{n}{x}\PYG{p}{,} \PYG{n}{loc}\PYG{p}{,} \PYG{n}{scale}\PYG{p}{)}\PYG{p}{:}
\PYG{g+gp}{... }    \PYG{k}{return} \PYG{n}{exp}\PYG{p}{(}\PYG{p}{(}\PYG{n}{loc}\PYG{o}{\PYGZhy{}}\PYG{n}{x}\PYG{p}{)}\PYG{o}{/}\PYG{n}{scale}\PYG{p}{)}\PYG{o}{/}\PYG{p}{(}\PYG{n}{scale}\PYG{o}{*}\PYG{p}{(}\PYG{l+m+mi}{1}\PYG{o}{+}\PYG{n}{exp}\PYG{p}{(}\PYG{p}{(}\PYG{n}{loc}\PYG{o}{\PYGZhy{}}\PYG{n}{x}\PYG{p}{)}\PYG{o}{/}\PYG{n}{scale}\PYG{p}{)}\PYG{p}{)}\PYG{o}{*}\PYG{o}{*}\PYG{l+m+mi}{2}\PYG{p}{)}
\PYG{g+gp}{\PYGZgt{}\PYGZgt{}\PYGZgt{} }\PYG{n}{plt}\PYG{o}{.}\PYG{n}{plot}\PYG{p}{(}\PYG{n}{bins}\PYG{p}{,} \PYG{n}{logist}\PYG{p}{(}\PYG{n}{bins}\PYG{p}{,} \PYG{n}{loc}\PYG{p}{,} \PYG{n}{scale}\PYG{p}{)}\PYG{o}{*}\PYG{n}{count}\PYG{o}{.}\PYG{n}{max}\PYG{p}{(}\PYG{p}{)}\PYG{o}{/}\PYGZbs{}
\PYG{g+gp}{... }\PYG{n}{logist}\PYG{p}{(}\PYG{n}{bins}\PYG{p}{,} \PYG{n}{loc}\PYG{p}{,} \PYG{n}{scale}\PYG{p}{)}\PYG{o}{.}\PYG{n}{max}\PYG{p}{(}\PYG{p}{)}\PYG{p}{)}
\PYG{g+gp}{\PYGZgt{}\PYGZgt{}\PYGZgt{} }\PYG{n}{plt}\PYG{o}{.}\PYG{n}{show}\PYG{p}{(}\PYG{p}{)}
\end{Verbatim}

\end{fulllineitems}

\index{lognormal() (in module lib.IO.readfiles)}

\begin{fulllineitems}
\phantomsection\label{lib.IO:lib.IO.readfiles.lognormal}\pysiglinewithargsret{\code{lib.IO.readfiles.}\bfcode{lognormal}}{\emph{mean=0.0}, \emph{sigma=1.0}, \emph{size=None}}{}
Return samples drawn from a log-normal distribution.

Draw samples from a log-normal distribution with specified mean,
standard deviation, and array shape.  Note that the mean and standard
deviation are not the values for the distribution itself, but of the
underlying normal distribution it is derived from.
\begin{description}
\item[{mean}] \leavevmode{[}float{]}
Mean value of the underlying normal distribution

\item[{sigma}] \leavevmode{[}float, \textgreater{} 0.{]}
Standard deviation of the underlying normal distribution

\item[{size}] \leavevmode{[}tuple of ints{]}
Output shape.  If the given shape is, e.g., \code{(m, n, k)}, then
\code{m * n * k} samples are drawn.

\end{description}
\begin{description}
\item[{samples}] \leavevmode{[}ndarray or float{]}
The desired samples. An array of the same shape as \emph{size} if given,
if \emph{size} is None a float is returned.

\end{description}
\begin{description}
\item[{scipy.stats.lognorm}] \leavevmode{[}probability density function, distribution,{]}
cumulative density function, etc.

\end{description}

A variable \emph{x} has a log-normal distribution if \emph{log(x)} is normally
distributed.  The probability density function for the log-normal
distribution is:
\begin{gather}
\begin{split}p(x) = \frac{1}{\sigma x \sqrt{2\pi}}
e^{(-\frac{(ln(x)-\mu)^2}{2\sigma^2})}\end{split}\notag
\end{gather}
where \(\mu\) is the mean and \(\sigma\) is the standard
deviation of the normally distributed logarithm of the variable.
A log-normal distribution results if a random variable is the \emph{product}
of a large number of independent, identically-distributed variables in
the same way that a normal distribution results if the variable is the
\emph{sum} of a large number of independent, identically-distributed
variables.

Limpert, E., Stahel, W. A., and Abbt, M., ``Log-normal Distributions
across the Sciences: Keys and Clues,'' \emph{BioScience}, Vol. 51, No. 5,
May, 2001.  \href{http://stat.ethz.ch/~stahel/lognormal/bioscience.pdf}{http://stat.ethz.ch/\textasciitilde{}stahel/lognormal/bioscience.pdf}

Reiss, R.D. and Thomas, M., \emph{Statistical Analysis of Extreme Values},
Basel: Birkhauser Verlag, 2001, pp. 31-32.

Draw samples from the distribution:

\begin{Verbatim}[commandchars=\\\{\}]
\PYG{g+gp}{\PYGZgt{}\PYGZgt{}\PYGZgt{} }\PYG{n}{mu}\PYG{p}{,} \PYG{n}{sigma} \PYG{o}{=} \PYG{l+m+mf}{3.}\PYG{p}{,} \PYG{l+m+mf}{1.} \PYG{c}{\PYGZsh{} mean and standard deviation}
\PYG{g+gp}{\PYGZgt{}\PYGZgt{}\PYGZgt{} }\PYG{n}{s} \PYG{o}{=} \PYG{n}{np}\PYG{o}{.}\PYG{n}{random}\PYG{o}{.}\PYG{n}{lognormal}\PYG{p}{(}\PYG{n}{mu}\PYG{p}{,} \PYG{n}{sigma}\PYG{p}{,} \PYG{l+m+mi}{1000}\PYG{p}{)}
\end{Verbatim}

Display the histogram of the samples, along with
the probability density function:

\begin{Verbatim}[commandchars=\\\{\}]
\PYG{g+gp}{\PYGZgt{}\PYGZgt{}\PYGZgt{} }\PYG{k+kn}{import} \PYG{n+nn}{matplotlib.pyplot} \PYG{k+kn}{as} \PYG{n+nn}{plt}
\PYG{g+gp}{\PYGZgt{}\PYGZgt{}\PYGZgt{} }\PYG{n}{count}\PYG{p}{,} \PYG{n}{bins}\PYG{p}{,} \PYG{n}{ignored} \PYG{o}{=} \PYG{n}{plt}\PYG{o}{.}\PYG{n}{hist}\PYG{p}{(}\PYG{n}{s}\PYG{p}{,} \PYG{l+m+mi}{100}\PYG{p}{,} \PYG{n}{normed}\PYG{o}{=}\PYG{n+nb+bp}{True}\PYG{p}{,} \PYG{n}{align}\PYG{o}{=}\PYG{l+s}{\PYGZsq{}}\PYG{l+s}{mid}\PYG{l+s}{\PYGZsq{}}\PYG{p}{)}
\end{Verbatim}

\begin{Verbatim}[commandchars=\\\{\}]
\PYG{g+gp}{\PYGZgt{}\PYGZgt{}\PYGZgt{} }\PYG{n}{x} \PYG{o}{=} \PYG{n}{np}\PYG{o}{.}\PYG{n}{linspace}\PYG{p}{(}\PYG{n+nb}{min}\PYG{p}{(}\PYG{n}{bins}\PYG{p}{)}\PYG{p}{,} \PYG{n+nb}{max}\PYG{p}{(}\PYG{n}{bins}\PYG{p}{)}\PYG{p}{,} \PYG{l+m+mi}{10000}\PYG{p}{)}
\PYG{g+gp}{\PYGZgt{}\PYGZgt{}\PYGZgt{} }\PYG{n}{pdf} \PYG{o}{=} \PYG{p}{(}\PYG{n}{np}\PYG{o}{.}\PYG{n}{exp}\PYG{p}{(}\PYG{o}{\PYGZhy{}}\PYG{p}{(}\PYG{n}{np}\PYG{o}{.}\PYG{n}{log}\PYG{p}{(}\PYG{n}{x}\PYG{p}{)} \PYG{o}{\PYGZhy{}} \PYG{n}{mu}\PYG{p}{)}\PYG{o}{*}\PYG{o}{*}\PYG{l+m+mi}{2} \PYG{o}{/} \PYG{p}{(}\PYG{l+m+mi}{2} \PYG{o}{*} \PYG{n}{sigma}\PYG{o}{*}\PYG{o}{*}\PYG{l+m+mi}{2}\PYG{p}{)}\PYG{p}{)}
\PYG{g+gp}{... }       \PYG{o}{/} \PYG{p}{(}\PYG{n}{x} \PYG{o}{*} \PYG{n}{sigma} \PYG{o}{*} \PYG{n}{np}\PYG{o}{.}\PYG{n}{sqrt}\PYG{p}{(}\PYG{l+m+mi}{2} \PYG{o}{*} \PYG{n}{np}\PYG{o}{.}\PYG{n}{pi}\PYG{p}{)}\PYG{p}{)}\PYG{p}{)}
\end{Verbatim}

\begin{Verbatim}[commandchars=\\\{\}]
\PYG{g+gp}{\PYGZgt{}\PYGZgt{}\PYGZgt{} }\PYG{n}{plt}\PYG{o}{.}\PYG{n}{plot}\PYG{p}{(}\PYG{n}{x}\PYG{p}{,} \PYG{n}{pdf}\PYG{p}{,} \PYG{n}{linewidth}\PYG{o}{=}\PYG{l+m+mi}{2}\PYG{p}{,} \PYG{n}{color}\PYG{o}{=}\PYG{l+s}{\PYGZsq{}}\PYG{l+s}{r}\PYG{l+s}{\PYGZsq{}}\PYG{p}{)}
\PYG{g+gp}{\PYGZgt{}\PYGZgt{}\PYGZgt{} }\PYG{n}{plt}\PYG{o}{.}\PYG{n}{axis}\PYG{p}{(}\PYG{l+s}{\PYGZsq{}}\PYG{l+s}{tight}\PYG{l+s}{\PYGZsq{}}\PYG{p}{)}
\PYG{g+gp}{\PYGZgt{}\PYGZgt{}\PYGZgt{} }\PYG{n}{plt}\PYG{o}{.}\PYG{n}{show}\PYG{p}{(}\PYG{p}{)}
\end{Verbatim}

Demonstrate that taking the products of random samples from a uniform
distribution can be fit well by a log-normal probability density function.

\begin{Verbatim}[commandchars=\\\{\}]
\PYG{g+gp}{\PYGZgt{}\PYGZgt{}\PYGZgt{} }\PYG{c}{\PYGZsh{} Generate a thousand samples: each is the product of 100 random}
\PYG{g+gp}{\PYGZgt{}\PYGZgt{}\PYGZgt{} }\PYG{c}{\PYGZsh{} values, drawn from a normal distribution.}
\PYG{g+gp}{\PYGZgt{}\PYGZgt{}\PYGZgt{} }\PYG{n}{b} \PYG{o}{=} \PYG{p}{[}\PYG{p}{]}
\PYG{g+gp}{\PYGZgt{}\PYGZgt{}\PYGZgt{} }\PYG{k}{for} \PYG{n}{i} \PYG{o+ow}{in} \PYG{n+nb}{range}\PYG{p}{(}\PYG{l+m+mi}{1000}\PYG{p}{)}\PYG{p}{:}
\PYG{g+gp}{... }   \PYG{n}{a} \PYG{o}{=} \PYG{l+m+mf}{10.} \PYG{o}{+} \PYG{n}{np}\PYG{o}{.}\PYG{n}{random}\PYG{o}{.}\PYG{n}{random}\PYG{p}{(}\PYG{l+m+mi}{100}\PYG{p}{)}
\PYG{g+gp}{... }   \PYG{n}{b}\PYG{o}{.}\PYG{n}{append}\PYG{p}{(}\PYG{n}{np}\PYG{o}{.}\PYG{n}{product}\PYG{p}{(}\PYG{n}{a}\PYG{p}{)}\PYG{p}{)}
\end{Verbatim}

\begin{Verbatim}[commandchars=\\\{\}]
\PYG{g+gp}{\PYGZgt{}\PYGZgt{}\PYGZgt{} }\PYG{n}{b} \PYG{o}{=} \PYG{n}{np}\PYG{o}{.}\PYG{n}{array}\PYG{p}{(}\PYG{n}{b}\PYG{p}{)} \PYG{o}{/} \PYG{n}{np}\PYG{o}{.}\PYG{n}{min}\PYG{p}{(}\PYG{n}{b}\PYG{p}{)} \PYG{c}{\PYGZsh{} scale values to be positive}
\PYG{g+gp}{\PYGZgt{}\PYGZgt{}\PYGZgt{} }\PYG{n}{count}\PYG{p}{,} \PYG{n}{bins}\PYG{p}{,} \PYG{n}{ignored} \PYG{o}{=} \PYG{n}{plt}\PYG{o}{.}\PYG{n}{hist}\PYG{p}{(}\PYG{n}{b}\PYG{p}{,} \PYG{l+m+mi}{100}\PYG{p}{,} \PYG{n}{normed}\PYG{o}{=}\PYG{n+nb+bp}{True}\PYG{p}{,} \PYG{n}{align}\PYG{o}{=}\PYG{l+s}{\PYGZsq{}}\PYG{l+s}{center}\PYG{l+s}{\PYGZsq{}}\PYG{p}{)}
\PYG{g+gp}{\PYGZgt{}\PYGZgt{}\PYGZgt{} }\PYG{n}{sigma} \PYG{o}{=} \PYG{n}{np}\PYG{o}{.}\PYG{n}{std}\PYG{p}{(}\PYG{n}{np}\PYG{o}{.}\PYG{n}{log}\PYG{p}{(}\PYG{n}{b}\PYG{p}{)}\PYG{p}{)}
\PYG{g+gp}{\PYGZgt{}\PYGZgt{}\PYGZgt{} }\PYG{n}{mu} \PYG{o}{=} \PYG{n}{np}\PYG{o}{.}\PYG{n}{mean}\PYG{p}{(}\PYG{n}{np}\PYG{o}{.}\PYG{n}{log}\PYG{p}{(}\PYG{n}{b}\PYG{p}{)}\PYG{p}{)}
\end{Verbatim}

\begin{Verbatim}[commandchars=\\\{\}]
\PYG{g+gp}{\PYGZgt{}\PYGZgt{}\PYGZgt{} }\PYG{n}{x} \PYG{o}{=} \PYG{n}{np}\PYG{o}{.}\PYG{n}{linspace}\PYG{p}{(}\PYG{n+nb}{min}\PYG{p}{(}\PYG{n}{bins}\PYG{p}{)}\PYG{p}{,} \PYG{n+nb}{max}\PYG{p}{(}\PYG{n}{bins}\PYG{p}{)}\PYG{p}{,} \PYG{l+m+mi}{10000}\PYG{p}{)}
\PYG{g+gp}{\PYGZgt{}\PYGZgt{}\PYGZgt{} }\PYG{n}{pdf} \PYG{o}{=} \PYG{p}{(}\PYG{n}{np}\PYG{o}{.}\PYG{n}{exp}\PYG{p}{(}\PYG{o}{\PYGZhy{}}\PYG{p}{(}\PYG{n}{np}\PYG{o}{.}\PYG{n}{log}\PYG{p}{(}\PYG{n}{x}\PYG{p}{)} \PYG{o}{\PYGZhy{}} \PYG{n}{mu}\PYG{p}{)}\PYG{o}{*}\PYG{o}{*}\PYG{l+m+mi}{2} \PYG{o}{/} \PYG{p}{(}\PYG{l+m+mi}{2} \PYG{o}{*} \PYG{n}{sigma}\PYG{o}{*}\PYG{o}{*}\PYG{l+m+mi}{2}\PYG{p}{)}\PYG{p}{)}
\PYG{g+gp}{... }       \PYG{o}{/} \PYG{p}{(}\PYG{n}{x} \PYG{o}{*} \PYG{n}{sigma} \PYG{o}{*} \PYG{n}{np}\PYG{o}{.}\PYG{n}{sqrt}\PYG{p}{(}\PYG{l+m+mi}{2} \PYG{o}{*} \PYG{n}{np}\PYG{o}{.}\PYG{n}{pi}\PYG{p}{)}\PYG{p}{)}\PYG{p}{)}
\end{Verbatim}

\begin{Verbatim}[commandchars=\\\{\}]
\PYG{g+gp}{\PYGZgt{}\PYGZgt{}\PYGZgt{} }\PYG{n}{plt}\PYG{o}{.}\PYG{n}{plot}\PYG{p}{(}\PYG{n}{x}\PYG{p}{,} \PYG{n}{pdf}\PYG{p}{,} \PYG{n}{color}\PYG{o}{=}\PYG{l+s}{\PYGZsq{}}\PYG{l+s}{r}\PYG{l+s}{\PYGZsq{}}\PYG{p}{,} \PYG{n}{linewidth}\PYG{o}{=}\PYG{l+m+mi}{2}\PYG{p}{)}
\PYG{g+gp}{\PYGZgt{}\PYGZgt{}\PYGZgt{} }\PYG{n}{plt}\PYG{o}{.}\PYG{n}{show}\PYG{p}{(}\PYG{p}{)}
\end{Verbatim}

\end{fulllineitems}

\index{logseries() (in module lib.IO.readfiles)}

\begin{fulllineitems}
\phantomsection\label{lib.IO:lib.IO.readfiles.logseries}\pysiglinewithargsret{\code{lib.IO.readfiles.}\bfcode{logseries}}{\emph{p}, \emph{size=None}}{}
Draw samples from a Logarithmic Series distribution.

Samples are drawn from a Log Series distribution with specified
parameter, p (probability, 0 \textless{} p \textless{} 1).

loc : float

scale : float \textgreater{} 0.
\begin{description}
\item[{size}] \leavevmode{[}\{tuple, int\}{]}
Output shape.  If the given shape is, e.g., \code{(m, n, k)}, then
\code{m * n * k} samples are drawn.

\end{description}
\begin{description}
\item[{samples}] \leavevmode{[}\{ndarray, scalar\}{]}
where the values are all integers in  {[}0, n{]}.

\end{description}
\begin{description}
\item[{scipy.stats.distributions.logser}] \leavevmode{[}probability density function,{]}
distribution or cumulative density function, etc.

\end{description}

The probability density for the Log Series distribution is
\begin{gather}
\begin{split}P(k) = \frac{-p^k}{k \ln(1-p)},\end{split}\notag
\end{gather}
where p = probability.

The Log Series distribution is frequently used to represent species
richness and occurrence, first proposed by Fisher, Corbet, and
Williams in 1943 {[}2{]}.  It may also be used to model the numbers of
occupants seen in cars {[}3{]}.

Draw samples from the distribution:

\begin{Verbatim}[commandchars=\\\{\}]
\PYG{g+gp}{\PYGZgt{}\PYGZgt{}\PYGZgt{} }\PYG{n}{a} \PYG{o}{=} \PYG{o}{.}\PYG{l+m+mi}{6}
\PYG{g+gp}{\PYGZgt{}\PYGZgt{}\PYGZgt{} }\PYG{n}{s} \PYG{o}{=} \PYG{n}{np}\PYG{o}{.}\PYG{n}{random}\PYG{o}{.}\PYG{n}{logseries}\PYG{p}{(}\PYG{n}{a}\PYG{p}{,} \PYG{l+m+mi}{10000}\PYG{p}{)}
\PYG{g+gp}{\PYGZgt{}\PYGZgt{}\PYGZgt{} }\PYG{n}{count}\PYG{p}{,} \PYG{n}{bins}\PYG{p}{,} \PYG{n}{ignored} \PYG{o}{=} \PYG{n}{plt}\PYG{o}{.}\PYG{n}{hist}\PYG{p}{(}\PYG{n}{s}\PYG{p}{)}
\end{Verbatim}

\#   plot against distribution

\begin{Verbatim}[commandchars=\\\{\}]
\PYG{g+gp}{\PYGZgt{}\PYGZgt{}\PYGZgt{} }\PYG{k}{def} \PYG{n+nf}{logseries}\PYG{p}{(}\PYG{n}{k}\PYG{p}{,} \PYG{n}{p}\PYG{p}{)}\PYG{p}{:}
\PYG{g+gp}{... }    \PYG{k}{return} \PYG{o}{\PYGZhy{}}\PYG{n}{p}\PYG{o}{*}\PYG{o}{*}\PYG{n}{k}\PYG{o}{/}\PYG{p}{(}\PYG{n}{k}\PYG{o}{*}\PYG{n}{log}\PYG{p}{(}\PYG{l+m+mi}{1}\PYG{o}{\PYGZhy{}}\PYG{n}{p}\PYG{p}{)}\PYG{p}{)}
\PYG{g+gp}{\PYGZgt{}\PYGZgt{}\PYGZgt{} }\PYG{n}{plt}\PYG{o}{.}\PYG{n}{plot}\PYG{p}{(}\PYG{n}{bins}\PYG{p}{,} \PYG{n}{logseries}\PYG{p}{(}\PYG{n}{bins}\PYG{p}{,} \PYG{n}{a}\PYG{p}{)}\PYG{o}{*}\PYG{n}{count}\PYG{o}{.}\PYG{n}{max}\PYG{p}{(}\PYG{p}{)}\PYG{o}{/}
\PYG{g+go}{             logseries(bins, a).max(), \PYGZsq{}r\PYGZsq{})}
\PYG{g+gp}{\PYGZgt{}\PYGZgt{}\PYGZgt{} }\PYG{n}{plt}\PYG{o}{.}\PYG{n}{show}\PYG{p}{(}\PYG{p}{)}
\end{Verbatim}

\end{fulllineitems}

\index{multinomial() (in module lib.IO.readfiles)}

\begin{fulllineitems}
\phantomsection\label{lib.IO:lib.IO.readfiles.multinomial}\pysiglinewithargsret{\code{lib.IO.readfiles.}\bfcode{multinomial}}{\emph{n}, \emph{pvals}, \emph{size=None}}{}
Draw samples from a multinomial distribution.

The multinomial distribution is a multivariate generalisation of the
binomial distribution.  Take an experiment with one of \code{p}
possible outcomes.  An example of such an experiment is throwing a dice,
where the outcome can be 1 through 6.  Each sample drawn from the
distribution represents \emph{n} such experiments.  Its values,
\code{X\_i = {[}X\_0, X\_1, ..., X\_p{]}}, represent the number of times the outcome
was \code{i}.
\begin{description}
\item[{n}] \leavevmode{[}int{]}
Number of experiments.

\item[{pvals}] \leavevmode{[}sequence of floats, length p{]}
Probabilities of each of the \code{p} different outcomes.  These
should sum to 1 (however, the last element is always assumed to
account for the remaining probability, as long as
\code{sum(pvals{[}:-1{]}) \textless{}= 1)}.

\item[{size}] \leavevmode{[}tuple of ints{]}
Given a \emph{size} of \code{(M, N, K)}, then \code{M*N*K} samples are drawn,
and the output shape becomes \code{(M, N, K, p)}, since each sample
has shape \code{(p,)}.

\end{description}

Throw a dice 20 times:

\begin{Verbatim}[commandchars=\\\{\}]
\PYG{g+gp}{\PYGZgt{}\PYGZgt{}\PYGZgt{} }\PYG{n}{np}\PYG{o}{.}\PYG{n}{random}\PYG{o}{.}\PYG{n}{multinomial}\PYG{p}{(}\PYG{l+m+mi}{20}\PYG{p}{,} \PYG{p}{[}\PYG{l+m+mi}{1}\PYG{o}{/}\PYG{l+m+mf}{6.}\PYG{p}{]}\PYG{o}{*}\PYG{l+m+mi}{6}\PYG{p}{,} \PYG{n}{size}\PYG{o}{=}\PYG{l+m+mi}{1}\PYG{p}{)}
\PYG{g+go}{array([[4, 1, 7, 5, 2, 1]])}
\end{Verbatim}

It landed 4 times on 1, once on 2, etc.

Now, throw the dice 20 times, and 20 times again:

\begin{Verbatim}[commandchars=\\\{\}]
\PYG{g+gp}{\PYGZgt{}\PYGZgt{}\PYGZgt{} }\PYG{n}{np}\PYG{o}{.}\PYG{n}{random}\PYG{o}{.}\PYG{n}{multinomial}\PYG{p}{(}\PYG{l+m+mi}{20}\PYG{p}{,} \PYG{p}{[}\PYG{l+m+mi}{1}\PYG{o}{/}\PYG{l+m+mf}{6.}\PYG{p}{]}\PYG{o}{*}\PYG{l+m+mi}{6}\PYG{p}{,} \PYG{n}{size}\PYG{o}{=}\PYG{l+m+mi}{2}\PYG{p}{)}
\PYG{g+go}{array([[3, 4, 3, 3, 4, 3],}
\PYG{g+go}{       [2, 4, 3, 4, 0, 7]])}
\end{Verbatim}

For the first run, we threw 3 times 1, 4 times 2, etc.  For the second,
we threw 2 times 1, 4 times 2, etc.

A loaded dice is more likely to land on number 6:

\begin{Verbatim}[commandchars=\\\{\}]
\PYG{g+gp}{\PYGZgt{}\PYGZgt{}\PYGZgt{} }\PYG{n}{np}\PYG{o}{.}\PYG{n}{random}\PYG{o}{.}\PYG{n}{multinomial}\PYG{p}{(}\PYG{l+m+mi}{100}\PYG{p}{,} \PYG{p}{[}\PYG{l+m+mi}{1}\PYG{o}{/}\PYG{l+m+mf}{7.}\PYG{p}{]}\PYG{o}{*}\PYG{l+m+mi}{5}\PYG{p}{)}
\PYG{g+go}{array([13, 16, 13, 16, 42])}
\end{Verbatim}

\end{fulllineitems}

\index{multivariate\_normal() (in module lib.IO.readfiles)}

\begin{fulllineitems}
\phantomsection\label{lib.IO:lib.IO.readfiles.multivariate_normal}\pysiglinewithargsret{\code{lib.IO.readfiles.}\bfcode{multivariate\_normal}}{\emph{mean}, \emph{cov}\optional{, \emph{size}}}{}
Draw random samples from a multivariate normal distribution.

The multivariate normal, multinormal or Gaussian distribution is a
generalization of the one-dimensional normal distribution to higher
dimensions.  Such a distribution is specified by its mean and
covariance matrix.  These parameters are analogous to the mean
(average or ``center'') and variance (standard deviation, or ``width,''
squared) of the one-dimensional normal distribution.
\begin{description}
\item[{mean}] \leavevmode{[}1-D array\_like, of length N{]}
Mean of the N-dimensional distribution.

\item[{cov}] \leavevmode{[}2-D array\_like, of shape (N, N){]}
Covariance matrix of the distribution.  Must be symmetric and
positive semi-definite for ``physically meaningful'' results.

\item[{size}] \leavevmode{[}int or tuple of ints, optional{]}
Given a shape of, for example, \code{(m,n,k)}, \code{m*n*k} samples are
generated, and packed in an \emph{m}-by-\emph{n}-by-\emph{k} arrangement.  Because
each sample is \emph{N}-dimensional, the output shape is \code{(m,n,k,N)}.
If no shape is specified, a single (\emph{N}-D) sample is returned.

\end{description}
\begin{description}
\item[{out}] \leavevmode{[}ndarray{]}
The drawn samples, of shape \emph{size}, if that was provided.  If not,
the shape is \code{(N,)}.

In other words, each entry \code{out{[}i,j,...,:{]}} is an N-dimensional
value drawn from the distribution.

\end{description}

The mean is a coordinate in N-dimensional space, which represents the
location where samples are most likely to be generated.  This is
analogous to the peak of the bell curve for the one-dimensional or
univariate normal distribution.

Covariance indicates the level to which two variables vary together.
From the multivariate normal distribution, we draw N-dimensional
samples, \(X = [x_1, x_2, ... x_N]\).  The covariance matrix
element \(C_{ij}\) is the covariance of \(x_i\) and \(x_j\).
The element \(C_{ii}\) is the variance of \(x_i\) (i.e. its
``spread'').

Instead of specifying the full covariance matrix, popular
approximations include:
\begin{itemize}
\item {} 
Spherical covariance (\emph{cov} is a multiple of the identity matrix)

\item {} 
Diagonal covariance (\emph{cov} has non-negative elements, and only on
the diagonal)

\end{itemize}

This geometrical property can be seen in two dimensions by plotting
generated data-points:

\begin{Verbatim}[commandchars=\\\{\}]
\PYG{g+gp}{\PYGZgt{}\PYGZgt{}\PYGZgt{} }\PYG{n}{mean} \PYG{o}{=} \PYG{p}{[}\PYG{l+m+mi}{0}\PYG{p}{,}\PYG{l+m+mi}{0}\PYG{p}{]}
\PYG{g+gp}{\PYGZgt{}\PYGZgt{}\PYGZgt{} }\PYG{n}{cov} \PYG{o}{=} \PYG{p}{[}\PYG{p}{[}\PYG{l+m+mi}{1}\PYG{p}{,}\PYG{l+m+mi}{0}\PYG{p}{]}\PYG{p}{,}\PYG{p}{[}\PYG{l+m+mi}{0}\PYG{p}{,}\PYG{l+m+mi}{100}\PYG{p}{]}\PYG{p}{]} \PYG{c}{\PYGZsh{} diagonal covariance, points lie on x or y\PYGZhy{}axis}
\end{Verbatim}

\begin{Verbatim}[commandchars=\\\{\}]
\PYG{g+gp}{\PYGZgt{}\PYGZgt{}\PYGZgt{} }\PYG{k+kn}{import} \PYG{n+nn}{matplotlib.pyplot} \PYG{k+kn}{as} \PYG{n+nn}{plt}
\PYG{g+gp}{\PYGZgt{}\PYGZgt{}\PYGZgt{} }\PYG{n}{x}\PYG{p}{,}\PYG{n}{y} \PYG{o}{=} \PYG{n}{np}\PYG{o}{.}\PYG{n}{random}\PYG{o}{.}\PYG{n}{multivariate\PYGZus{}normal}\PYG{p}{(}\PYG{n}{mean}\PYG{p}{,}\PYG{n}{cov}\PYG{p}{,}\PYG{l+m+mi}{5000}\PYG{p}{)}\PYG{o}{.}\PYG{n}{T}
\PYG{g+gp}{\PYGZgt{}\PYGZgt{}\PYGZgt{} }\PYG{n}{plt}\PYG{o}{.}\PYG{n}{plot}\PYG{p}{(}\PYG{n}{x}\PYG{p}{,}\PYG{n}{y}\PYG{p}{,}\PYG{l+s}{\PYGZsq{}}\PYG{l+s}{x}\PYG{l+s}{\PYGZsq{}}\PYG{p}{)}\PYG{p}{;} \PYG{n}{plt}\PYG{o}{.}\PYG{n}{axis}\PYG{p}{(}\PYG{l+s}{\PYGZsq{}}\PYG{l+s}{equal}\PYG{l+s}{\PYGZsq{}}\PYG{p}{)}\PYG{p}{;} \PYG{n}{plt}\PYG{o}{.}\PYG{n}{show}\PYG{p}{(}\PYG{p}{)}
\end{Verbatim}

Note that the covariance matrix must be non-negative definite.

Papoulis, A., \emph{Probability, Random Variables, and Stochastic Processes},
3rd ed., New York: McGraw-Hill, 1991.

Duda, R. O., Hart, P. E., and Stork, D. G., \emph{Pattern Classification},
2nd ed., New York: Wiley, 2001.

\begin{Verbatim}[commandchars=\\\{\}]
\PYG{g+gp}{\PYGZgt{}\PYGZgt{}\PYGZgt{} }\PYG{n}{mean} \PYG{o}{=} \PYG{p}{(}\PYG{l+m+mi}{1}\PYG{p}{,}\PYG{l+m+mi}{2}\PYG{p}{)}
\PYG{g+gp}{\PYGZgt{}\PYGZgt{}\PYGZgt{} }\PYG{n}{cov} \PYG{o}{=} \PYG{p}{[}\PYG{p}{[}\PYG{l+m+mi}{1}\PYG{p}{,}\PYG{l+m+mi}{0}\PYG{p}{]}\PYG{p}{,}\PYG{p}{[}\PYG{l+m+mi}{1}\PYG{p}{,}\PYG{l+m+mi}{0}\PYG{p}{]}\PYG{p}{]}
\PYG{g+gp}{\PYGZgt{}\PYGZgt{}\PYGZgt{} }\PYG{n}{x} \PYG{o}{=} \PYG{n}{np}\PYG{o}{.}\PYG{n}{random}\PYG{o}{.}\PYG{n}{multivariate\PYGZus{}normal}\PYG{p}{(}\PYG{n}{mean}\PYG{p}{,}\PYG{n}{cov}\PYG{p}{,}\PYG{p}{(}\PYG{l+m+mi}{3}\PYG{p}{,}\PYG{l+m+mi}{3}\PYG{p}{)}\PYG{p}{)}
\PYG{g+gp}{\PYGZgt{}\PYGZgt{}\PYGZgt{} }\PYG{n}{x}\PYG{o}{.}\PYG{n}{shape}
\PYG{g+go}{(3, 3, 2)}
\end{Verbatim}

The following is probably true, given that 0.6 is roughly twice the
standard deviation:

\begin{Verbatim}[commandchars=\\\{\}]
\PYG{g+gp}{\PYGZgt{}\PYGZgt{}\PYGZgt{} }\PYG{k}{print} \PYG{n+nb}{list}\PYG{p}{(} \PYG{p}{(}\PYG{n}{x}\PYG{p}{[}\PYG{l+m+mi}{0}\PYG{p}{,}\PYG{l+m+mi}{0}\PYG{p}{,}\PYG{p}{:}\PYG{p}{]} \PYG{o}{\PYGZhy{}} \PYG{n}{mean}\PYG{p}{)} \PYG{o}{\PYGZlt{}} \PYG{l+m+mf}{0.6} \PYG{p}{)}
\PYG{g+go}{[True, True]}
\end{Verbatim}

\end{fulllineitems}

\index{negative\_binomial() (in module lib.IO.readfiles)}

\begin{fulllineitems}
\phantomsection\label{lib.IO:lib.IO.readfiles.negative_binomial}\pysiglinewithargsret{\code{lib.IO.readfiles.}\bfcode{negative\_binomial}}{\emph{n}, \emph{p}, \emph{size=None}}{}
Draw samples from a negative\_binomial distribution.

Samples are drawn from a negative\_Binomial distribution with specified
parameters, \emph{n} trials and \emph{p} probability of success where \emph{n} is an
integer \textgreater{} 0 and \emph{p} is in the interval {[}0, 1{]}.
\begin{description}
\item[{n}] \leavevmode{[}int{]}
Parameter, \textgreater{} 0.

\item[{p}] \leavevmode{[}float{]}
Parameter, \textgreater{}= 0 and \textless{}=1.

\item[{size}] \leavevmode{[}int or tuple of ints{]}
Output shape. If the given shape is, e.g., \code{(m, n, k)}, then
\code{m * n * k} samples are drawn.

\end{description}
\begin{description}
\item[{samples}] \leavevmode{[}int or ndarray of ints{]}
Drawn samples.

\end{description}

The probability density for the Negative Binomial distribution is
\begin{gather}
\begin{split}P(N;n,p) = \binom{N+n-1}{n-1}p^{n}(1-p)^{N},\end{split}\notag
\end{gather}
where \(n-1\) is the number of successes, \(p\) is the probability
of success, and \(N+n-1\) is the number of trials.

The negative binomial distribution gives the probability of n-1 successes
and N failures in N+n-1 trials, and success on the (N+n)th trial.

If one throws a die repeatedly until the third time a ``1'' appears, then the
probability distribution of the number of non-``1''s that appear before the
third ``1'' is a negative binomial distribution.

Draw samples from the distribution:

A real world example. A company drills wild-cat oil exploration wells, each
with an estimated probability of success of 0.1.  What is the probability
of having one success for each successive well, that is what is the
probability of a single success after drilling 5 wells, after 6 wells,
etc.?

\begin{Verbatim}[commandchars=\\\{\}]
\PYG{g+gp}{\PYGZgt{}\PYGZgt{}\PYGZgt{} }\PYG{n}{s} \PYG{o}{=} \PYG{n}{np}\PYG{o}{.}\PYG{n}{random}\PYG{o}{.}\PYG{n}{negative\PYGZus{}binomial}\PYG{p}{(}\PYG{l+m+mi}{1}\PYG{p}{,} \PYG{l+m+mf}{0.1}\PYG{p}{,} \PYG{l+m+mi}{100000}\PYG{p}{)}
\PYG{g+gp}{\PYGZgt{}\PYGZgt{}\PYGZgt{} }\PYG{k}{for} \PYG{n}{i} \PYG{o+ow}{in} \PYG{n+nb}{range}\PYG{p}{(}\PYG{l+m+mi}{1}\PYG{p}{,} \PYG{l+m+mi}{11}\PYG{p}{)}\PYG{p}{:}
\PYG{g+gp}{... }   \PYG{n}{probability} \PYG{o}{=} \PYG{n+nb}{sum}\PYG{p}{(}\PYG{n}{s}\PYG{o}{\PYGZlt{}}\PYG{n}{i}\PYG{p}{)} \PYG{o}{/} \PYG{l+m+mf}{100000.}
\PYG{g+gp}{... }   \PYG{k}{print} \PYG{n}{i}\PYG{p}{,} \PYG{l+s}{\PYGZdq{}}\PYG{l+s}{wells drilled, probability of one success =}\PYG{l+s}{\PYGZdq{}}\PYG{p}{,} \PYG{n}{probability}
\end{Verbatim}

\end{fulllineitems}

\index{noncentral\_chisquare() (in module lib.IO.readfiles)}

\begin{fulllineitems}
\phantomsection\label{lib.IO:lib.IO.readfiles.noncentral_chisquare}\pysiglinewithargsret{\code{lib.IO.readfiles.}\bfcode{noncentral\_chisquare}}{\emph{df}, \emph{nonc}, \emph{size=None}}{}
Draw samples from a noncentral chi-square distribution.

The noncentral \(\chi^2\) distribution is a generalisation of
the \(\chi^2\) distribution.
\begin{description}
\item[{df}] \leavevmode{[}int{]}
Degrees of freedom, should be \textgreater{}= 1.

\item[{nonc}] \leavevmode{[}float{]}
Non-centrality, should be \textgreater{} 0.

\item[{size}] \leavevmode{[}int or tuple of ints{]}
Shape of the output.

\end{description}

The probability density function for the noncentral Chi-square distribution
is
\begin{gather}
\begin{split}P(x;df,nonc) = \sum^{\infty}_{i=0}
\frac{e^{-nonc/2}(nonc/2)^{i}}{i!}P_{Y_{df+2i}}(x),\end{split}\notag
\end{gather}
where \(Y_{q}\) is the Chi-square with q degrees of freedom.

In Delhi (2007), it is noted that the noncentral chi-square is useful in
bombing and coverage problems, the probability of killing the point target
given by the noncentral chi-squared distribution.

Draw values from the distribution and plot the histogram

\begin{Verbatim}[commandchars=\\\{\}]
\PYG{g+gp}{\PYGZgt{}\PYGZgt{}\PYGZgt{} }\PYG{k+kn}{import} \PYG{n+nn}{matplotlib.pyplot} \PYG{k+kn}{as} \PYG{n+nn}{plt}
\PYG{g+gp}{\PYGZgt{}\PYGZgt{}\PYGZgt{} }\PYG{n}{values} \PYG{o}{=} \PYG{n}{plt}\PYG{o}{.}\PYG{n}{hist}\PYG{p}{(}\PYG{n}{np}\PYG{o}{.}\PYG{n}{random}\PYG{o}{.}\PYG{n}{noncentral\PYGZus{}chisquare}\PYG{p}{(}\PYG{l+m+mi}{3}\PYG{p}{,} \PYG{l+m+mi}{20}\PYG{p}{,} \PYG{l+m+mi}{100000}\PYG{p}{)}\PYG{p}{,}
\PYG{g+gp}{... }                  \PYG{n}{bins}\PYG{o}{=}\PYG{l+m+mi}{200}\PYG{p}{,} \PYG{n}{normed}\PYG{o}{=}\PYG{n+nb+bp}{True}\PYG{p}{)}
\PYG{g+gp}{\PYGZgt{}\PYGZgt{}\PYGZgt{} }\PYG{n}{plt}\PYG{o}{.}\PYG{n}{show}\PYG{p}{(}\PYG{p}{)}
\end{Verbatim}

Draw values from a noncentral chisquare with very small noncentrality,
and compare to a chisquare.

\begin{Verbatim}[commandchars=\\\{\}]
\PYG{g+gp}{\PYGZgt{}\PYGZgt{}\PYGZgt{} }\PYG{n}{plt}\PYG{o}{.}\PYG{n}{figure}\PYG{p}{(}\PYG{p}{)}
\PYG{g+gp}{\PYGZgt{}\PYGZgt{}\PYGZgt{} }\PYG{n}{values} \PYG{o}{=} \PYG{n}{plt}\PYG{o}{.}\PYG{n}{hist}\PYG{p}{(}\PYG{n}{np}\PYG{o}{.}\PYG{n}{random}\PYG{o}{.}\PYG{n}{noncentral\PYGZus{}chisquare}\PYG{p}{(}\PYG{l+m+mi}{3}\PYG{p}{,} \PYG{o}{.}\PYG{l+m+mo}{0000001}\PYG{p}{,} \PYG{l+m+mi}{100000}\PYG{p}{)}\PYG{p}{,}
\PYG{g+gp}{... }                  \PYG{n}{bins}\PYG{o}{=}\PYG{n}{np}\PYG{o}{.}\PYG{n}{arange}\PYG{p}{(}\PYG{l+m+mf}{0.}\PYG{p}{,} \PYG{l+m+mi}{25}\PYG{p}{,} \PYG{o}{.}\PYG{l+m+mi}{1}\PYG{p}{)}\PYG{p}{,} \PYG{n}{normed}\PYG{o}{=}\PYG{n+nb+bp}{True}\PYG{p}{)}
\PYG{g+gp}{\PYGZgt{}\PYGZgt{}\PYGZgt{} }\PYG{n}{values2} \PYG{o}{=} \PYG{n}{plt}\PYG{o}{.}\PYG{n}{hist}\PYG{p}{(}\PYG{n}{np}\PYG{o}{.}\PYG{n}{random}\PYG{o}{.}\PYG{n}{chisquare}\PYG{p}{(}\PYG{l+m+mi}{3}\PYG{p}{,} \PYG{l+m+mi}{100000}\PYG{p}{)}\PYG{p}{,}
\PYG{g+gp}{... }                   \PYG{n}{bins}\PYG{o}{=}\PYG{n}{np}\PYG{o}{.}\PYG{n}{arange}\PYG{p}{(}\PYG{l+m+mf}{0.}\PYG{p}{,} \PYG{l+m+mi}{25}\PYG{p}{,} \PYG{o}{.}\PYG{l+m+mi}{1}\PYG{p}{)}\PYG{p}{,} \PYG{n}{normed}\PYG{o}{=}\PYG{n+nb+bp}{True}\PYG{p}{)}
\PYG{g+gp}{\PYGZgt{}\PYGZgt{}\PYGZgt{} }\PYG{n}{plt}\PYG{o}{.}\PYG{n}{plot}\PYG{p}{(}\PYG{n}{values}\PYG{p}{[}\PYG{l+m+mi}{1}\PYG{p}{]}\PYG{p}{[}\PYG{l+m+mi}{0}\PYG{p}{:}\PYG{o}{\PYGZhy{}}\PYG{l+m+mi}{1}\PYG{p}{]}\PYG{p}{,} \PYG{n}{values}\PYG{p}{[}\PYG{l+m+mi}{0}\PYG{p}{]}\PYG{o}{\PYGZhy{}}\PYG{n}{values2}\PYG{p}{[}\PYG{l+m+mi}{0}\PYG{p}{]}\PYG{p}{,} \PYG{l+s}{\PYGZsq{}}\PYG{l+s}{ob}\PYG{l+s}{\PYGZsq{}}\PYG{p}{)}
\PYG{g+gp}{\PYGZgt{}\PYGZgt{}\PYGZgt{} }\PYG{n}{plt}\PYG{o}{.}\PYG{n}{show}\PYG{p}{(}\PYG{p}{)}
\end{Verbatim}

Demonstrate how large values of non-centrality lead to a more symmetric
distribution.

\begin{Verbatim}[commandchars=\\\{\}]
\PYG{g+gp}{\PYGZgt{}\PYGZgt{}\PYGZgt{} }\PYG{n}{plt}\PYG{o}{.}\PYG{n}{figure}\PYG{p}{(}\PYG{p}{)}
\PYG{g+gp}{\PYGZgt{}\PYGZgt{}\PYGZgt{} }\PYG{n}{values} \PYG{o}{=} \PYG{n}{plt}\PYG{o}{.}\PYG{n}{hist}\PYG{p}{(}\PYG{n}{np}\PYG{o}{.}\PYG{n}{random}\PYG{o}{.}\PYG{n}{noncentral\PYGZus{}chisquare}\PYG{p}{(}\PYG{l+m+mi}{3}\PYG{p}{,} \PYG{l+m+mi}{20}\PYG{p}{,} \PYG{l+m+mi}{100000}\PYG{p}{)}\PYG{p}{,}
\PYG{g+gp}{... }                  \PYG{n}{bins}\PYG{o}{=}\PYG{l+m+mi}{200}\PYG{p}{,} \PYG{n}{normed}\PYG{o}{=}\PYG{n+nb+bp}{True}\PYG{p}{)}
\PYG{g+gp}{\PYGZgt{}\PYGZgt{}\PYGZgt{} }\PYG{n}{plt}\PYG{o}{.}\PYG{n}{show}\PYG{p}{(}\PYG{p}{)}
\end{Verbatim}

\end{fulllineitems}

\index{noncentral\_f() (in module lib.IO.readfiles)}

\begin{fulllineitems}
\phantomsection\label{lib.IO:lib.IO.readfiles.noncentral_f}\pysiglinewithargsret{\code{lib.IO.readfiles.}\bfcode{noncentral\_f}}{\emph{dfnum}, \emph{dfden}, \emph{nonc}, \emph{size=None}}{}
Draw samples from the noncentral F distribution.

Samples are drawn from an F distribution with specified parameters,
\emph{dfnum} (degrees of freedom in numerator) and \emph{dfden} (degrees of
freedom in denominator), where both parameters \textgreater{} 1.
\emph{nonc} is the non-centrality parameter.
\begin{description}
\item[{dfnum}] \leavevmode{[}int{]}
Parameter, should be \textgreater{} 1.

\item[{dfden}] \leavevmode{[}int{]}
Parameter, should be \textgreater{} 1.

\item[{nonc}] \leavevmode{[}float{]}
Parameter, should be \textgreater{}= 0.

\item[{size}] \leavevmode{[}int or tuple of ints{]}
Output shape. If the given shape is, e.g., \code{(m, n, k)}, then
\code{m * n * k} samples are drawn.

\end{description}
\begin{description}
\item[{samples}] \leavevmode{[}scalar or ndarray{]}
Drawn samples.

\end{description}

When calculating the power of an experiment (power = probability of
rejecting the null hypothesis when a specific alternative is true) the
non-central F statistic becomes important.  When the null hypothesis is
true, the F statistic follows a central F distribution. When the null
hypothesis is not true, then it follows a non-central F statistic.

Weisstein, Eric W. ``Noncentral F-Distribution.'' From MathWorld--A Wolfram
Web Resource.  \href{http://mathworld.wolfram.com/NoncentralF-Distribution.html}{http://mathworld.wolfram.com/NoncentralF-Distribution.html}

Wikipedia, ``Noncentral F distribution'',
\href{http://en.wikipedia.org/wiki/Noncentral\_F-distribution}{http://en.wikipedia.org/wiki/Noncentral\_F-distribution}

In a study, testing for a specific alternative to the null hypothesis
requires use of the Noncentral F distribution. We need to calculate the
area in the tail of the distribution that exceeds the value of the F
distribution for the null hypothesis.  We'll plot the two probability
distributions for comparison.

\begin{Verbatim}[commandchars=\\\{\}]
\PYG{g+gp}{\PYGZgt{}\PYGZgt{}\PYGZgt{} }\PYG{n}{dfnum} \PYG{o}{=} \PYG{l+m+mi}{3} \PYG{c}{\PYGZsh{} between group deg of freedom}
\PYG{g+gp}{\PYGZgt{}\PYGZgt{}\PYGZgt{} }\PYG{n}{dfden} \PYG{o}{=} \PYG{l+m+mi}{20} \PYG{c}{\PYGZsh{} within groups degrees of freedom}
\PYG{g+gp}{\PYGZgt{}\PYGZgt{}\PYGZgt{} }\PYG{n}{nonc} \PYG{o}{=} \PYG{l+m+mf}{3.0}
\PYG{g+gp}{\PYGZgt{}\PYGZgt{}\PYGZgt{} }\PYG{n}{nc\PYGZus{}vals} \PYG{o}{=} \PYG{n}{np}\PYG{o}{.}\PYG{n}{random}\PYG{o}{.}\PYG{n}{noncentral\PYGZus{}f}\PYG{p}{(}\PYG{n}{dfnum}\PYG{p}{,} \PYG{n}{dfden}\PYG{p}{,} \PYG{n}{nonc}\PYG{p}{,} \PYG{l+m+mi}{1000000}\PYG{p}{)}
\PYG{g+gp}{\PYGZgt{}\PYGZgt{}\PYGZgt{} }\PYG{n}{NF} \PYG{o}{=} \PYG{n}{np}\PYG{o}{.}\PYG{n}{histogram}\PYG{p}{(}\PYG{n}{nc\PYGZus{}vals}\PYG{p}{,} \PYG{n}{bins}\PYG{o}{=}\PYG{l+m+mi}{50}\PYG{p}{,} \PYG{n}{normed}\PYG{o}{=}\PYG{n+nb+bp}{True}\PYG{p}{)}
\PYG{g+gp}{\PYGZgt{}\PYGZgt{}\PYGZgt{} }\PYG{n}{c\PYGZus{}vals} \PYG{o}{=} \PYG{n}{np}\PYG{o}{.}\PYG{n}{random}\PYG{o}{.}\PYG{n}{f}\PYG{p}{(}\PYG{n}{dfnum}\PYG{p}{,} \PYG{n}{dfden}\PYG{p}{,} \PYG{l+m+mi}{1000000}\PYG{p}{)}
\PYG{g+gp}{\PYGZgt{}\PYGZgt{}\PYGZgt{} }\PYG{n}{F} \PYG{o}{=} \PYG{n}{np}\PYG{o}{.}\PYG{n}{histogram}\PYG{p}{(}\PYG{n}{c\PYGZus{}vals}\PYG{p}{,} \PYG{n}{bins}\PYG{o}{=}\PYG{l+m+mi}{50}\PYG{p}{,} \PYG{n}{normed}\PYG{o}{=}\PYG{n+nb+bp}{True}\PYG{p}{)}
\PYG{g+gp}{\PYGZgt{}\PYGZgt{}\PYGZgt{} }\PYG{n}{plt}\PYG{o}{.}\PYG{n}{plot}\PYG{p}{(}\PYG{n}{F}\PYG{p}{[}\PYG{l+m+mi}{1}\PYG{p}{]}\PYG{p}{[}\PYG{l+m+mi}{1}\PYG{p}{:}\PYG{p}{]}\PYG{p}{,} \PYG{n}{F}\PYG{p}{[}\PYG{l+m+mi}{0}\PYG{p}{]}\PYG{p}{)}
\PYG{g+gp}{\PYGZgt{}\PYGZgt{}\PYGZgt{} }\PYG{n}{plt}\PYG{o}{.}\PYG{n}{plot}\PYG{p}{(}\PYG{n}{NF}\PYG{p}{[}\PYG{l+m+mi}{1}\PYG{p}{]}\PYG{p}{[}\PYG{l+m+mi}{1}\PYG{p}{:}\PYG{p}{]}\PYG{p}{,} \PYG{n}{NF}\PYG{p}{[}\PYG{l+m+mi}{0}\PYG{p}{]}\PYG{p}{)}
\PYG{g+gp}{\PYGZgt{}\PYGZgt{}\PYGZgt{} }\PYG{n}{plt}\PYG{o}{.}\PYG{n}{show}\PYG{p}{(}\PYG{p}{)}
\end{Verbatim}

\end{fulllineitems}

\index{normal() (in module lib.IO.readfiles)}

\begin{fulllineitems}
\phantomsection\label{lib.IO:lib.IO.readfiles.normal}\pysiglinewithargsret{\code{lib.IO.readfiles.}\bfcode{normal}}{\emph{loc=0.0}, \emph{scale=1.0}, \emph{size=None}}{}
Draw random samples from a normal (Gaussian) distribution.

The probability density function of the normal distribution, first
derived by De Moivre and 200 years later by both Gauss and Laplace
independently {\color{red}\bfseries{}{[}2{]}\_}, is often called the bell curve because of
its characteristic shape (see the example below).

The normal distributions occurs often in nature.  For example, it
describes the commonly occurring distribution of samples influenced
by a large number of tiny, random disturbances, each with its own
unique distribution {\color{red}\bfseries{}{[}2{]}\_}.
\begin{description}
\item[{loc}] \leavevmode{[}float{]}
Mean (``centre'') of the distribution.

\item[{scale}] \leavevmode{[}float{]}
Standard deviation (spread or ``width'') of the distribution.

\item[{size}] \leavevmode{[}tuple of ints{]}
Output shape.  If the given shape is, e.g., \code{(m, n, k)}, then
\code{m * n * k} samples are drawn.

\end{description}
\begin{description}
\item[{scipy.stats.distributions.norm}] \leavevmode{[}probability density function,{]}
distribution or cumulative density function, etc.

\end{description}

The probability density for the Gaussian distribution is
\begin{gather}
\begin{split}p(x) = \frac{1}{\sqrt{ 2 \pi \sigma^2 }}
e^{ - \frac{ (x - \mu)^2 } {2 \sigma^2} },\end{split}\notag
\end{gather}
where \(\mu\) is the mean and \(\sigma\) the standard deviation.
The square of the standard deviation, \(\sigma^2\), is called the
variance.

The function has its peak at the mean, and its ``spread'' increases with
the standard deviation (the function reaches 0.607 times its maximum at
\(x + \sigma\) and \(x - \sigma\) {\color{red}\bfseries{}{[}2{]}\_}).  This implies that
\emph{numpy.random.normal} is more likely to return samples lying close to the
mean, rather than those far away.

Draw samples from the distribution:

\begin{Verbatim}[commandchars=\\\{\}]
\PYG{g+gp}{\PYGZgt{}\PYGZgt{}\PYGZgt{} }\PYG{n}{mu}\PYG{p}{,} \PYG{n}{sigma} \PYG{o}{=} \PYG{l+m+mi}{0}\PYG{p}{,} \PYG{l+m+mf}{0.1} \PYG{c}{\PYGZsh{} mean and standard deviation}
\PYG{g+gp}{\PYGZgt{}\PYGZgt{}\PYGZgt{} }\PYG{n}{s} \PYG{o}{=} \PYG{n}{np}\PYG{o}{.}\PYG{n}{random}\PYG{o}{.}\PYG{n}{normal}\PYG{p}{(}\PYG{n}{mu}\PYG{p}{,} \PYG{n}{sigma}\PYG{p}{,} \PYG{l+m+mi}{1000}\PYG{p}{)}
\end{Verbatim}

Verify the mean and the variance:

\begin{Verbatim}[commandchars=\\\{\}]
\PYG{g+gp}{\PYGZgt{}\PYGZgt{}\PYGZgt{} }\PYG{n+nb}{abs}\PYG{p}{(}\PYG{n}{mu} \PYG{o}{\PYGZhy{}} \PYG{n}{np}\PYG{o}{.}\PYG{n}{mean}\PYG{p}{(}\PYG{n}{s}\PYG{p}{)}\PYG{p}{)} \PYG{o}{\PYGZlt{}} \PYG{l+m+mf}{0.01}
\PYG{g+go}{True}
\end{Verbatim}

\begin{Verbatim}[commandchars=\\\{\}]
\PYG{g+gp}{\PYGZgt{}\PYGZgt{}\PYGZgt{} }\PYG{n+nb}{abs}\PYG{p}{(}\PYG{n}{sigma} \PYG{o}{\PYGZhy{}} \PYG{n}{np}\PYG{o}{.}\PYG{n}{std}\PYG{p}{(}\PYG{n}{s}\PYG{p}{,} \PYG{n}{ddof}\PYG{o}{=}\PYG{l+m+mi}{1}\PYG{p}{)}\PYG{p}{)} \PYG{o}{\PYGZlt{}} \PYG{l+m+mf}{0.01}
\PYG{g+go}{True}
\end{Verbatim}

Display the histogram of the samples, along with
the probability density function:

\begin{Verbatim}[commandchars=\\\{\}]
\PYG{g+gp}{\PYGZgt{}\PYGZgt{}\PYGZgt{} }\PYG{k+kn}{import} \PYG{n+nn}{matplotlib.pyplot} \PYG{k+kn}{as} \PYG{n+nn}{plt}
\PYG{g+gp}{\PYGZgt{}\PYGZgt{}\PYGZgt{} }\PYG{n}{count}\PYG{p}{,} \PYG{n}{bins}\PYG{p}{,} \PYG{n}{ignored} \PYG{o}{=} \PYG{n}{plt}\PYG{o}{.}\PYG{n}{hist}\PYG{p}{(}\PYG{n}{s}\PYG{p}{,} \PYG{l+m+mi}{30}\PYG{p}{,} \PYG{n}{normed}\PYG{o}{=}\PYG{n+nb+bp}{True}\PYG{p}{)}
\PYG{g+gp}{\PYGZgt{}\PYGZgt{}\PYGZgt{} }\PYG{n}{plt}\PYG{o}{.}\PYG{n}{plot}\PYG{p}{(}\PYG{n}{bins}\PYG{p}{,} \PYG{l+m+mi}{1}\PYG{o}{/}\PYG{p}{(}\PYG{n}{sigma} \PYG{o}{*} \PYG{n}{np}\PYG{o}{.}\PYG{n}{sqrt}\PYG{p}{(}\PYG{l+m+mi}{2} \PYG{o}{*} \PYG{n}{np}\PYG{o}{.}\PYG{n}{pi}\PYG{p}{)}\PYG{p}{)} \PYG{o}{*}
\PYG{g+gp}{... }               \PYG{n}{np}\PYG{o}{.}\PYG{n}{exp}\PYG{p}{(} \PYG{o}{\PYGZhy{}} \PYG{p}{(}\PYG{n}{bins} \PYG{o}{\PYGZhy{}} \PYG{n}{mu}\PYG{p}{)}\PYG{o}{*}\PYG{o}{*}\PYG{l+m+mi}{2} \PYG{o}{/} \PYG{p}{(}\PYG{l+m+mi}{2} \PYG{o}{*} \PYG{n}{sigma}\PYG{o}{*}\PYG{o}{*}\PYG{l+m+mi}{2}\PYG{p}{)} \PYG{p}{)}\PYG{p}{,}
\PYG{g+gp}{... }         \PYG{n}{linewidth}\PYG{o}{=}\PYG{l+m+mi}{2}\PYG{p}{,} \PYG{n}{color}\PYG{o}{=}\PYG{l+s}{\PYGZsq{}}\PYG{l+s}{r}\PYG{l+s}{\PYGZsq{}}\PYG{p}{)}
\PYG{g+gp}{\PYGZgt{}\PYGZgt{}\PYGZgt{} }\PYG{n}{plt}\PYG{o}{.}\PYG{n}{show}\PYG{p}{(}\PYG{p}{)}
\end{Verbatim}

\end{fulllineitems}

\index{pareto() (in module lib.IO.readfiles)}

\begin{fulllineitems}
\phantomsection\label{lib.IO:lib.IO.readfiles.pareto}\pysiglinewithargsret{\code{lib.IO.readfiles.}\bfcode{pareto}}{\emph{a}, \emph{size=None}}{}
Draw samples from a Pareto II or Lomax distribution with specified shape.

The Lomax or Pareto II distribution is a shifted Pareto distribution. The
classical Pareto distribution can be obtained from the Lomax distribution
by adding the location parameter m, see below. The smallest value of the
Lomax distribution is zero while for the classical Pareto distribution it
is m, where the standard Pareto distribution has location m=1.
Lomax can also be considered as a simplified version of the Generalized
Pareto distribution (available in SciPy), with the scale set to one and
the location set to zero.

The Pareto distribution must be greater than zero, and is unbounded above.
It is also known as the ``80-20 rule''.  In this distribution, 80 percent of
the weights are in the lowest 20 percent of the range, while the other 20
percent fill the remaining 80 percent of the range.
\begin{description}
\item[{shape}] \leavevmode{[}float, \textgreater{} 0.{]}
Shape of the distribution.

\item[{size}] \leavevmode{[}tuple of ints{]}
Output shape.  If the given shape is, e.g., \code{(m, n, k)}, then
\code{m * n * k} samples are drawn.

\end{description}
\begin{description}
\item[{scipy.stats.distributions.lomax.pdf}] \leavevmode{[}probability density function,{]}
distribution or cumulative density function, etc.

\item[{scipy.stats.distributions.genpareto.pdf}] \leavevmode{[}probability density function,{]}
distribution or cumulative density function, etc.

\end{description}

The probability density for the Pareto distribution is
\begin{gather}
\begin{split}p(x) = \frac{am^a}{x^{a+1}}\end{split}\notag
\end{gather}
where \(a\) is the shape and \(m\) the location

The Pareto distribution, named after the Italian economist Vilfredo Pareto,
is a power law probability distribution useful in many real world problems.
Outside the field of economics it is generally referred to as the Bradford
distribution. Pareto developed the distribution to describe the
distribution of wealth in an economy.  It has also found use in insurance,
web page access statistics, oil field sizes, and many other problems,
including the download frequency for projects in Sourceforge {[}1{]}.  It is
one of the so-called ``fat-tailed'' distributions.

Draw samples from the distribution:

\begin{Verbatim}[commandchars=\\\{\}]
\PYG{g+gp}{\PYGZgt{}\PYGZgt{}\PYGZgt{} }\PYG{n}{a}\PYG{p}{,} \PYG{n}{m} \PYG{o}{=} \PYG{l+m+mf}{3.}\PYG{p}{,} \PYG{l+m+mf}{1.} \PYG{c}{\PYGZsh{} shape and mode}
\PYG{g+gp}{\PYGZgt{}\PYGZgt{}\PYGZgt{} }\PYG{n}{s} \PYG{o}{=} \PYG{n}{np}\PYG{o}{.}\PYG{n}{random}\PYG{o}{.}\PYG{n}{pareto}\PYG{p}{(}\PYG{n}{a}\PYG{p}{,} \PYG{l+m+mi}{1000}\PYG{p}{)} \PYG{o}{+} \PYG{n}{m}
\end{Verbatim}

Display the histogram of the samples, along with
the probability density function:

\begin{Verbatim}[commandchars=\\\{\}]
\PYG{g+gp}{\PYGZgt{}\PYGZgt{}\PYGZgt{} }\PYG{k+kn}{import} \PYG{n+nn}{matplotlib.pyplot} \PYG{k+kn}{as} \PYG{n+nn}{plt}
\PYG{g+gp}{\PYGZgt{}\PYGZgt{}\PYGZgt{} }\PYG{n}{count}\PYG{p}{,} \PYG{n}{bins}\PYG{p}{,} \PYG{n}{ignored} \PYG{o}{=} \PYG{n}{plt}\PYG{o}{.}\PYG{n}{hist}\PYG{p}{(}\PYG{n}{s}\PYG{p}{,} \PYG{l+m+mi}{100}\PYG{p}{,} \PYG{n}{normed}\PYG{o}{=}\PYG{n+nb+bp}{True}\PYG{p}{,} \PYG{n}{align}\PYG{o}{=}\PYG{l+s}{\PYGZsq{}}\PYG{l+s}{center}\PYG{l+s}{\PYGZsq{}}\PYG{p}{)}
\PYG{g+gp}{\PYGZgt{}\PYGZgt{}\PYGZgt{} }\PYG{n}{fit} \PYG{o}{=} \PYG{n}{a}\PYG{o}{*}\PYG{n}{m}\PYG{o}{*}\PYG{o}{*}\PYG{n}{a}\PYG{o}{/}\PYG{n}{bins}\PYG{o}{*}\PYG{o}{*}\PYG{p}{(}\PYG{n}{a}\PYG{o}{+}\PYG{l+m+mi}{1}\PYG{p}{)}
\PYG{g+gp}{\PYGZgt{}\PYGZgt{}\PYGZgt{} }\PYG{n}{plt}\PYG{o}{.}\PYG{n}{plot}\PYG{p}{(}\PYG{n}{bins}\PYG{p}{,} \PYG{n+nb}{max}\PYG{p}{(}\PYG{n}{count}\PYG{p}{)}\PYG{o}{*}\PYG{n}{fit}\PYG{o}{/}\PYG{n+nb}{max}\PYG{p}{(}\PYG{n}{fit}\PYG{p}{)}\PYG{p}{,}\PYG{n}{linewidth}\PYG{o}{=}\PYG{l+m+mi}{2}\PYG{p}{,} \PYG{n}{color}\PYG{o}{=}\PYG{l+s}{\PYGZsq{}}\PYG{l+s}{r}\PYG{l+s}{\PYGZsq{}}\PYG{p}{)}
\PYG{g+gp}{\PYGZgt{}\PYGZgt{}\PYGZgt{} }\PYG{n}{plt}\PYG{o}{.}\PYG{n}{show}\PYG{p}{(}\PYG{p}{)}
\end{Verbatim}

\end{fulllineitems}

\index{permutation() (in module lib.IO.readfiles)}

\begin{fulllineitems}
\phantomsection\label{lib.IO:lib.IO.readfiles.permutation}\pysiglinewithargsret{\code{lib.IO.readfiles.}\bfcode{permutation}}{\emph{x}}{}
Randomly permute a sequence, or return a permuted range.

If \emph{x} is a multi-dimensional array, it is only shuffled along its
first index.
\begin{description}
\item[{x}] \leavevmode{[}int or array\_like{]}
If \emph{x} is an integer, randomly permute \code{np.arange(x)}.
If \emph{x} is an array, make a copy and shuffle the elements
randomly.

\end{description}
\begin{description}
\item[{out}] \leavevmode{[}ndarray{]}
Permuted sequence or array range.

\end{description}

\begin{Verbatim}[commandchars=\\\{\}]
\PYG{g+gp}{\PYGZgt{}\PYGZgt{}\PYGZgt{} }\PYG{n}{np}\PYG{o}{.}\PYG{n}{random}\PYG{o}{.}\PYG{n}{permutation}\PYG{p}{(}\PYG{l+m+mi}{10}\PYG{p}{)}
\PYG{g+go}{array([1, 7, 4, 3, 0, 9, 2, 5, 8, 6])}
\end{Verbatim}

\begin{Verbatim}[commandchars=\\\{\}]
\PYG{g+gp}{\PYGZgt{}\PYGZgt{}\PYGZgt{} }\PYG{n}{np}\PYG{o}{.}\PYG{n}{random}\PYG{o}{.}\PYG{n}{permutation}\PYG{p}{(}\PYG{p}{[}\PYG{l+m+mi}{1}\PYG{p}{,} \PYG{l+m+mi}{4}\PYG{p}{,} \PYG{l+m+mi}{9}\PYG{p}{,} \PYG{l+m+mi}{12}\PYG{p}{,} \PYG{l+m+mi}{15}\PYG{p}{]}\PYG{p}{)}
\PYG{g+go}{array([15,  1,  9,  4, 12])}
\end{Verbatim}

\begin{Verbatim}[commandchars=\\\{\}]
\PYG{g+gp}{\PYGZgt{}\PYGZgt{}\PYGZgt{} }\PYG{n}{arr} \PYG{o}{=} \PYG{n}{np}\PYG{o}{.}\PYG{n}{arange}\PYG{p}{(}\PYG{l+m+mi}{9}\PYG{p}{)}\PYG{o}{.}\PYG{n}{reshape}\PYG{p}{(}\PYG{p}{(}\PYG{l+m+mi}{3}\PYG{p}{,} \PYG{l+m+mi}{3}\PYG{p}{)}\PYG{p}{)}
\PYG{g+gp}{\PYGZgt{}\PYGZgt{}\PYGZgt{} }\PYG{n}{np}\PYG{o}{.}\PYG{n}{random}\PYG{o}{.}\PYG{n}{permutation}\PYG{p}{(}\PYG{n}{arr}\PYG{p}{)}
\PYG{g+go}{array([[6, 7, 8],}
\PYG{g+go}{       [0, 1, 2],}
\PYG{g+go}{       [3, 4, 5]])}
\end{Verbatim}

\end{fulllineitems}

\index{poisson() (in module lib.IO.readfiles)}

\begin{fulllineitems}
\phantomsection\label{lib.IO:lib.IO.readfiles.poisson}\pysiglinewithargsret{\code{lib.IO.readfiles.}\bfcode{poisson}}{\emph{lam=1.0}, \emph{size=None}}{}
Draw samples from a Poisson distribution.

The Poisson distribution is the limit of the Binomial
distribution for large N.
\begin{description}
\item[{lam}] \leavevmode{[}float{]}
Expectation of interval, should be \textgreater{}= 0.

\item[{size}] \leavevmode{[}int or tuple of ints, optional{]}
Output shape. If the given shape is, e.g., \code{(m, n, k)}, then
\code{m * n * k} samples are drawn.

\end{description}

The Poisson distribution
\begin{gather}
\begin{split}f(k; \lambda)=\frac{\lambda^k e^{-\lambda}}{k!}\end{split}\notag
\end{gather}
For events with an expected separation \(\lambda\) the Poisson
distribution \(f(k; \lambda)\) describes the probability of
\(k\) events occurring within the observed interval \(\lambda\).

Because the output is limited to the range of the C long type, a
ValueError is raised when \emph{lam} is within 10 sigma of the maximum
representable value.

Draw samples from the distribution:

\begin{Verbatim}[commandchars=\\\{\}]
\PYG{g+gp}{\PYGZgt{}\PYGZgt{}\PYGZgt{} }\PYG{k+kn}{import} \PYG{n+nn}{numpy} \PYG{k+kn}{as} \PYG{n+nn}{np}
\PYG{g+gp}{\PYGZgt{}\PYGZgt{}\PYGZgt{} }\PYG{n}{s} \PYG{o}{=} \PYG{n}{np}\PYG{o}{.}\PYG{n}{random}\PYG{o}{.}\PYG{n}{poisson}\PYG{p}{(}\PYG{l+m+mi}{5}\PYG{p}{,} \PYG{l+m+mi}{10000}\PYG{p}{)}
\end{Verbatim}

Display histogram of the sample:

\begin{Verbatim}[commandchars=\\\{\}]
\PYG{g+gp}{\PYGZgt{}\PYGZgt{}\PYGZgt{} }\PYG{k+kn}{import} \PYG{n+nn}{matplotlib.pyplot} \PYG{k+kn}{as} \PYG{n+nn}{plt}
\PYG{g+gp}{\PYGZgt{}\PYGZgt{}\PYGZgt{} }\PYG{n}{count}\PYG{p}{,} \PYG{n}{bins}\PYG{p}{,} \PYG{n}{ignored} \PYG{o}{=} \PYG{n}{plt}\PYG{o}{.}\PYG{n}{hist}\PYG{p}{(}\PYG{n}{s}\PYG{p}{,} \PYG{l+m+mi}{14}\PYG{p}{,} \PYG{n}{normed}\PYG{o}{=}\PYG{n+nb+bp}{True}\PYG{p}{)}
\PYG{g+gp}{\PYGZgt{}\PYGZgt{}\PYGZgt{} }\PYG{n}{plt}\PYG{o}{.}\PYG{n}{show}\PYG{p}{(}\PYG{p}{)}
\end{Verbatim}

\end{fulllineitems}

\index{power() (in module lib.IO.readfiles)}

\begin{fulllineitems}
\phantomsection\label{lib.IO:lib.IO.readfiles.power}\pysiglinewithargsret{\code{lib.IO.readfiles.}\bfcode{power}}{\emph{a}, \emph{size=None}}{}
Draws samples in {[}0, 1{]} from a power distribution with positive
exponent a - 1.

Also known as the power function distribution.
\begin{description}
\item[{a}] \leavevmode{[}float{]}
parameter, \textgreater{} 0

\item[{size}] \leavevmode{[}tuple of ints{]}\begin{description}
\item[{Output shape.  If the given shape is, e.g., \code{(m, n, k)}, then}] \leavevmode
\code{m * n * k} samples are drawn.

\end{description}

\end{description}
\begin{description}
\item[{samples}] \leavevmode{[}\{ndarray, scalar\}{]}
The returned samples lie in {[}0, 1{]}.

\end{description}
\begin{description}
\item[{ValueError}] \leavevmode
If a\textless{}1.

\end{description}

The probability density function is
\begin{gather}
\begin{split}P(x; a) = ax^{a-1}, 0 \le x \le 1, a>0.\end{split}\notag
\end{gather}
The power function distribution is just the inverse of the Pareto
distribution. It may also be seen as a special case of the Beta
distribution.

It is used, for example, in modeling the over-reporting of insurance
claims.

Draw samples from the distribution:

\begin{Verbatim}[commandchars=\\\{\}]
\PYG{g+gp}{\PYGZgt{}\PYGZgt{}\PYGZgt{} }\PYG{n}{a} \PYG{o}{=} \PYG{l+m+mf}{5.} \PYG{c}{\PYGZsh{} shape}
\PYG{g+gp}{\PYGZgt{}\PYGZgt{}\PYGZgt{} }\PYG{n}{samples} \PYG{o}{=} \PYG{l+m+mi}{1000}
\PYG{g+gp}{\PYGZgt{}\PYGZgt{}\PYGZgt{} }\PYG{n}{s} \PYG{o}{=} \PYG{n}{np}\PYG{o}{.}\PYG{n}{random}\PYG{o}{.}\PYG{n}{power}\PYG{p}{(}\PYG{n}{a}\PYG{p}{,} \PYG{n}{samples}\PYG{p}{)}
\end{Verbatim}

Display the histogram of the samples, along with
the probability density function:

\begin{Verbatim}[commandchars=\\\{\}]
\PYG{g+gp}{\PYGZgt{}\PYGZgt{}\PYGZgt{} }\PYG{k+kn}{import} \PYG{n+nn}{matplotlib.pyplot} \PYG{k+kn}{as} \PYG{n+nn}{plt}
\PYG{g+gp}{\PYGZgt{}\PYGZgt{}\PYGZgt{} }\PYG{n}{count}\PYG{p}{,} \PYG{n}{bins}\PYG{p}{,} \PYG{n}{ignored} \PYG{o}{=} \PYG{n}{plt}\PYG{o}{.}\PYG{n}{hist}\PYG{p}{(}\PYG{n}{s}\PYG{p}{,} \PYG{n}{bins}\PYG{o}{=}\PYG{l+m+mi}{30}\PYG{p}{)}
\PYG{g+gp}{\PYGZgt{}\PYGZgt{}\PYGZgt{} }\PYG{n}{x} \PYG{o}{=} \PYG{n}{np}\PYG{o}{.}\PYG{n}{linspace}\PYG{p}{(}\PYG{l+m+mi}{0}\PYG{p}{,} \PYG{l+m+mi}{1}\PYG{p}{,} \PYG{l+m+mi}{100}\PYG{p}{)}
\PYG{g+gp}{\PYGZgt{}\PYGZgt{}\PYGZgt{} }\PYG{n}{y} \PYG{o}{=} \PYG{n}{a}\PYG{o}{*}\PYG{n}{x}\PYG{o}{*}\PYG{o}{*}\PYG{p}{(}\PYG{n}{a}\PYG{o}{\PYGZhy{}}\PYG{l+m+mf}{1.}\PYG{p}{)}
\PYG{g+gp}{\PYGZgt{}\PYGZgt{}\PYGZgt{} }\PYG{n}{normed\PYGZus{}y} \PYG{o}{=} \PYG{n}{samples}\PYG{o}{*}\PYG{n}{np}\PYG{o}{.}\PYG{n}{diff}\PYG{p}{(}\PYG{n}{bins}\PYG{p}{)}\PYG{p}{[}\PYG{l+m+mi}{0}\PYG{p}{]}\PYG{o}{*}\PYG{n}{y}
\PYG{g+gp}{\PYGZgt{}\PYGZgt{}\PYGZgt{} }\PYG{n}{plt}\PYG{o}{.}\PYG{n}{plot}\PYG{p}{(}\PYG{n}{x}\PYG{p}{,} \PYG{n}{normed\PYGZus{}y}\PYG{p}{)}
\PYG{g+gp}{\PYGZgt{}\PYGZgt{}\PYGZgt{} }\PYG{n}{plt}\PYG{o}{.}\PYG{n}{show}\PYG{p}{(}\PYG{p}{)}
\end{Verbatim}

Compare the power function distribution to the inverse of the Pareto.

\begin{Verbatim}[commandchars=\\\{\}]
\PYG{g+gp}{\PYGZgt{}\PYGZgt{}\PYGZgt{} }\PYG{k+kn}{from} \PYG{n+nn}{scipy} \PYG{k+kn}{import} \PYG{n}{stats}
\PYG{g+gp}{\PYGZgt{}\PYGZgt{}\PYGZgt{} }\PYG{n}{rvs} \PYG{o}{=} \PYG{n}{np}\PYG{o}{.}\PYG{n}{random}\PYG{o}{.}\PYG{n}{power}\PYG{p}{(}\PYG{l+m+mi}{5}\PYG{p}{,} \PYG{l+m+mi}{1000000}\PYG{p}{)}
\PYG{g+gp}{\PYGZgt{}\PYGZgt{}\PYGZgt{} }\PYG{n}{rvsp} \PYG{o}{=} \PYG{n}{np}\PYG{o}{.}\PYG{n}{random}\PYG{o}{.}\PYG{n}{pareto}\PYG{p}{(}\PYG{l+m+mi}{5}\PYG{p}{,} \PYG{l+m+mi}{1000000}\PYG{p}{)}
\PYG{g+gp}{\PYGZgt{}\PYGZgt{}\PYGZgt{} }\PYG{n}{xx} \PYG{o}{=} \PYG{n}{np}\PYG{o}{.}\PYG{n}{linspace}\PYG{p}{(}\PYG{l+m+mi}{0}\PYG{p}{,}\PYG{l+m+mi}{1}\PYG{p}{,}\PYG{l+m+mi}{100}\PYG{p}{)}
\PYG{g+gp}{\PYGZgt{}\PYGZgt{}\PYGZgt{} }\PYG{n}{powpdf} \PYG{o}{=} \PYG{n}{stats}\PYG{o}{.}\PYG{n}{powerlaw}\PYG{o}{.}\PYG{n}{pdf}\PYG{p}{(}\PYG{n}{xx}\PYG{p}{,}\PYG{l+m+mi}{5}\PYG{p}{)}
\end{Verbatim}

\begin{Verbatim}[commandchars=\\\{\}]
\PYG{g+gp}{\PYGZgt{}\PYGZgt{}\PYGZgt{} }\PYG{n}{plt}\PYG{o}{.}\PYG{n}{figure}\PYG{p}{(}\PYG{p}{)}
\PYG{g+gp}{\PYGZgt{}\PYGZgt{}\PYGZgt{} }\PYG{n}{plt}\PYG{o}{.}\PYG{n}{hist}\PYG{p}{(}\PYG{n}{rvs}\PYG{p}{,} \PYG{n}{bins}\PYG{o}{=}\PYG{l+m+mi}{50}\PYG{p}{,} \PYG{n}{normed}\PYG{o}{=}\PYG{n+nb+bp}{True}\PYG{p}{)}
\PYG{g+gp}{\PYGZgt{}\PYGZgt{}\PYGZgt{} }\PYG{n}{plt}\PYG{o}{.}\PYG{n}{plot}\PYG{p}{(}\PYG{n}{xx}\PYG{p}{,}\PYG{n}{powpdf}\PYG{p}{,}\PYG{l+s}{\PYGZsq{}}\PYG{l+s}{r\PYGZhy{}}\PYG{l+s}{\PYGZsq{}}\PYG{p}{)}
\PYG{g+gp}{\PYGZgt{}\PYGZgt{}\PYGZgt{} }\PYG{n}{plt}\PYG{o}{.}\PYG{n}{title}\PYG{p}{(}\PYG{l+s}{\PYGZsq{}}\PYG{l+s}{np.random.power(5)}\PYG{l+s}{\PYGZsq{}}\PYG{p}{)}
\end{Verbatim}

\begin{Verbatim}[commandchars=\\\{\}]
\PYG{g+gp}{\PYGZgt{}\PYGZgt{}\PYGZgt{} }\PYG{n}{plt}\PYG{o}{.}\PYG{n}{figure}\PYG{p}{(}\PYG{p}{)}
\PYG{g+gp}{\PYGZgt{}\PYGZgt{}\PYGZgt{} }\PYG{n}{plt}\PYG{o}{.}\PYG{n}{hist}\PYG{p}{(}\PYG{l+m+mf}{1.}\PYG{o}{/}\PYG{p}{(}\PYG{l+m+mf}{1.}\PYG{o}{+}\PYG{n}{rvsp}\PYG{p}{)}\PYG{p}{,} \PYG{n}{bins}\PYG{o}{=}\PYG{l+m+mi}{50}\PYG{p}{,} \PYG{n}{normed}\PYG{o}{=}\PYG{n+nb+bp}{True}\PYG{p}{)}
\PYG{g+gp}{\PYGZgt{}\PYGZgt{}\PYGZgt{} }\PYG{n}{plt}\PYG{o}{.}\PYG{n}{plot}\PYG{p}{(}\PYG{n}{xx}\PYG{p}{,}\PYG{n}{powpdf}\PYG{p}{,}\PYG{l+s}{\PYGZsq{}}\PYG{l+s}{r\PYGZhy{}}\PYG{l+s}{\PYGZsq{}}\PYG{p}{)}
\PYG{g+gp}{\PYGZgt{}\PYGZgt{}\PYGZgt{} }\PYG{n}{plt}\PYG{o}{.}\PYG{n}{title}\PYG{p}{(}\PYG{l+s}{\PYGZsq{}}\PYG{l+s}{inverse of 1 + np.random.pareto(5)}\PYG{l+s}{\PYGZsq{}}\PYG{p}{)}
\end{Verbatim}

\begin{Verbatim}[commandchars=\\\{\}]
\PYG{g+gp}{\PYGZgt{}\PYGZgt{}\PYGZgt{} }\PYG{n}{plt}\PYG{o}{.}\PYG{n}{figure}\PYG{p}{(}\PYG{p}{)}
\PYG{g+gp}{\PYGZgt{}\PYGZgt{}\PYGZgt{} }\PYG{n}{plt}\PYG{o}{.}\PYG{n}{hist}\PYG{p}{(}\PYG{l+m+mf}{1.}\PYG{o}{/}\PYG{p}{(}\PYG{l+m+mf}{1.}\PYG{o}{+}\PYG{n}{rvsp}\PYG{p}{)}\PYG{p}{,} \PYG{n}{bins}\PYG{o}{=}\PYG{l+m+mi}{50}\PYG{p}{,} \PYG{n}{normed}\PYG{o}{=}\PYG{n+nb+bp}{True}\PYG{p}{)}
\PYG{g+gp}{\PYGZgt{}\PYGZgt{}\PYGZgt{} }\PYG{n}{plt}\PYG{o}{.}\PYG{n}{plot}\PYG{p}{(}\PYG{n}{xx}\PYG{p}{,}\PYG{n}{powpdf}\PYG{p}{,}\PYG{l+s}{\PYGZsq{}}\PYG{l+s}{r\PYGZhy{}}\PYG{l+s}{\PYGZsq{}}\PYG{p}{)}
\PYG{g+gp}{\PYGZgt{}\PYGZgt{}\PYGZgt{} }\PYG{n}{plt}\PYG{o}{.}\PYG{n}{title}\PYG{p}{(}\PYG{l+s}{\PYGZsq{}}\PYG{l+s}{inverse of stats.pareto(5)}\PYG{l+s}{\PYGZsq{}}\PYG{p}{)}
\end{Verbatim}

\end{fulllineitems}

\index{rand() (in module lib.IO.readfiles)}

\begin{fulllineitems}
\phantomsection\label{lib.IO:lib.IO.readfiles.rand}\pysiglinewithargsret{\code{lib.IO.readfiles.}\bfcode{rand}}{\emph{d0}, \emph{d1}, \emph{...}, \emph{dn}}{}
Random values in a given shape.

Create an array of the given shape and propagate it with
random samples from a uniform distribution
over \code{{[}0, 1)}.
\begin{description}
\item[{d0, d1, ..., dn}] \leavevmode{[}int, optional{]}
The dimensions of the returned array, should all be positive.
If no argument is given a single Python float is returned.

\end{description}
\begin{description}
\item[{out}] \leavevmode{[}ndarray, shape \code{(d0, d1, ..., dn)}{]}
Random values.

\end{description}

random

This is a convenience function. If you want an interface that
takes a shape-tuple as the first argument, refer to
np.random.random\_sample .

\begin{Verbatim}[commandchars=\\\{\}]
\PYG{g+gp}{\PYGZgt{}\PYGZgt{}\PYGZgt{} }\PYG{n}{np}\PYG{o}{.}\PYG{n}{random}\PYG{o}{.}\PYG{n}{rand}\PYG{p}{(}\PYG{l+m+mi}{3}\PYG{p}{,}\PYG{l+m+mi}{2}\PYG{p}{)}
\PYG{g+go}{array([[ 0.14022471,  0.96360618],  \PYGZsh{}random}
\PYG{g+go}{       [ 0.37601032,  0.25528411],  \PYGZsh{}random}
\PYG{g+go}{       [ 0.49313049,  0.94909878]]) \PYGZsh{}random}
\end{Verbatim}

\end{fulllineitems}

\index{randint() (in module lib.IO.readfiles)}

\begin{fulllineitems}
\phantomsection\label{lib.IO:lib.IO.readfiles.randint}\pysiglinewithargsret{\code{lib.IO.readfiles.}\bfcode{randint}}{\emph{low}, \emph{high=None}, \emph{size=None}}{}
Return random integers from \emph{low} (inclusive) to \emph{high} (exclusive).

Return random integers from the ``discrete uniform'' distribution in the
``half-open'' interval {[}\emph{low}, \emph{high}). If \emph{high} is None (the default),
then results are from {[}0, \emph{low}).
\begin{description}
\item[{low}] \leavevmode{[}int{]}
Lowest (signed) integer to be drawn from the distribution (unless
\code{high=None}, in which case this parameter is the \emph{highest} such
integer).

\item[{high}] \leavevmode{[}int, optional{]}
If provided, one above the largest (signed) integer to be drawn
from the distribution (see above for behavior if \code{high=None}).

\item[{size}] \leavevmode{[}int or tuple of ints, optional{]}
Output shape. Default is None, in which case a single int is
returned.

\end{description}
\begin{description}
\item[{out}] \leavevmode{[}int or ndarray of ints{]}
\emph{size}-shaped array of random integers from the appropriate
distribution, or a single such random int if \emph{size} not provided.

\end{description}
\begin{description}
\item[{random.random\_integers}] \leavevmode{[}similar to \emph{randint}, only for the closed{]}
interval {[}\emph{low}, \emph{high}{]}, and 1 is the lowest value if \emph{high} is
omitted. In particular, this other one is the one to use to generate
uniformly distributed discrete non-integers.

\end{description}

\begin{Verbatim}[commandchars=\\\{\}]
\PYG{g+gp}{\PYGZgt{}\PYGZgt{}\PYGZgt{} }\PYG{n}{np}\PYG{o}{.}\PYG{n}{random}\PYG{o}{.}\PYG{n}{randint}\PYG{p}{(}\PYG{l+m+mi}{2}\PYG{p}{,} \PYG{n}{size}\PYG{o}{=}\PYG{l+m+mi}{10}\PYG{p}{)}
\PYG{g+go}{array([1, 0, 0, 0, 1, 1, 0, 0, 1, 0])}
\PYG{g+gp}{\PYGZgt{}\PYGZgt{}\PYGZgt{} }\PYG{n}{np}\PYG{o}{.}\PYG{n}{random}\PYG{o}{.}\PYG{n}{randint}\PYG{p}{(}\PYG{l+m+mi}{1}\PYG{p}{,} \PYG{n}{size}\PYG{o}{=}\PYG{l+m+mi}{10}\PYG{p}{)}
\PYG{g+go}{array([0, 0, 0, 0, 0, 0, 0, 0, 0, 0])}
\end{Verbatim}

Generate a 2 x 4 array of ints between 0 and 4, inclusive:

\begin{Verbatim}[commandchars=\\\{\}]
\PYG{g+gp}{\PYGZgt{}\PYGZgt{}\PYGZgt{} }\PYG{n}{np}\PYG{o}{.}\PYG{n}{random}\PYG{o}{.}\PYG{n}{randint}\PYG{p}{(}\PYG{l+m+mi}{5}\PYG{p}{,} \PYG{n}{size}\PYG{o}{=}\PYG{p}{(}\PYG{l+m+mi}{2}\PYG{p}{,} \PYG{l+m+mi}{4}\PYG{p}{)}\PYG{p}{)}
\PYG{g+go}{array([[4, 0, 2, 1],}
\PYG{g+go}{       [3, 2, 2, 0]])}
\end{Verbatim}

\end{fulllineitems}

\index{randn() (in module lib.IO.readfiles)}

\begin{fulllineitems}
\phantomsection\label{lib.IO:lib.IO.readfiles.randn}\pysiglinewithargsret{\code{lib.IO.readfiles.}\bfcode{randn}}{\emph{d0}, \emph{d1}, \emph{...}, \emph{dn}}{}
Return a sample (or samples) from the ``standard normal'' distribution.

If positive, int\_like or int-convertible arguments are provided,
\emph{randn} generates an array of shape \code{(d0, d1, ..., dn)}, filled
with random floats sampled from a univariate ``normal'' (Gaussian)
distribution of mean 0 and variance 1 (if any of the \(d_i\) are
floats, they are first converted to integers by truncation). A single
float randomly sampled from the distribution is returned if no
argument is provided.

This is a convenience function.  If you want an interface that takes a
tuple as the first argument, use \emph{numpy.random.standard\_normal} instead.
\begin{description}
\item[{d0, d1, ..., dn}] \leavevmode{[}int, optional{]}
The dimensions of the returned array, should be all positive.
If no argument is given a single Python float is returned.

\end{description}
\begin{description}
\item[{Z}] \leavevmode{[}ndarray or float{]}
A \code{(d0, d1, ..., dn)}-shaped array of floating-point samples from
the standard normal distribution, or a single such float if
no parameters were supplied.

\end{description}

random.standard\_normal : Similar, but takes a tuple as its argument.

For random samples from \(N(\mu, \sigma^2)\), use:

\code{sigma * np.random.randn(...) + mu}

\begin{Verbatim}[commandchars=\\\{\}]
\PYG{g+gp}{\PYGZgt{}\PYGZgt{}\PYGZgt{} }\PYG{n}{np}\PYG{o}{.}\PYG{n}{random}\PYG{o}{.}\PYG{n}{randn}\PYG{p}{(}\PYG{p}{)}
\PYG{g+go}{2.1923875335537315 \PYGZsh{}random}
\end{Verbatim}

Two-by-four array of samples from N(3, 6.25):

\begin{Verbatim}[commandchars=\\\{\}]
\PYG{g+gp}{\PYGZgt{}\PYGZgt{}\PYGZgt{} }\PYG{l+m+mf}{2.5} \PYG{o}{*} \PYG{n}{np}\PYG{o}{.}\PYG{n}{random}\PYG{o}{.}\PYG{n}{randn}\PYG{p}{(}\PYG{l+m+mi}{2}\PYG{p}{,} \PYG{l+m+mi}{4}\PYG{p}{)} \PYG{o}{+} \PYG{l+m+mi}{3}
\PYG{g+go}{array([[\PYGZhy{}4.49401501,  4.00950034, \PYGZhy{}1.81814867,  7.29718677],  \PYGZsh{}random}
\PYG{g+go}{       [ 0.39924804,  4.68456316,  4.99394529,  4.84057254]]) \PYGZsh{}random}
\end{Verbatim}

\end{fulllineitems}

\index{random() (in module lib.IO.readfiles)}

\begin{fulllineitems}
\phantomsection\label{lib.IO:lib.IO.readfiles.random}\pysiglinewithargsret{\code{lib.IO.readfiles.}\bfcode{random}}{}{}
random\_sample(size=None)

Return random floats in the half-open interval {[}0.0, 1.0).

Results are from the ``continuous uniform'' distribution over the
stated interval.  To sample \(Unif[a, b), b > a\) multiply
the output of \emph{random\_sample} by \emph{(b-a)} and add \emph{a}:

\begin{Verbatim}[commandchars=\\\{\}]
\PYG{p}{(}\PYG{n}{b} \PYG{o}{\PYGZhy{}} \PYG{n}{a}\PYG{p}{)} \PYG{o}{*} \PYG{n}{random\PYGZus{}sample}\PYG{p}{(}\PYG{p}{)} \PYG{o}{+} \PYG{n}{a}
\end{Verbatim}
\begin{description}
\item[{size}] \leavevmode{[}int or tuple of ints, optional{]}
Defines the shape of the returned array of random floats. If None
(the default), returns a single float.

\end{description}
\begin{description}
\item[{out}] \leavevmode{[}float or ndarray of floats{]}
Array of random floats of shape \emph{size} (unless \code{size=None}, in which
case a single float is returned).

\end{description}

\begin{Verbatim}[commandchars=\\\{\}]
\PYG{g+gp}{\PYGZgt{}\PYGZgt{}\PYGZgt{} }\PYG{n}{np}\PYG{o}{.}\PYG{n}{random}\PYG{o}{.}\PYG{n}{random\PYGZus{}sample}\PYG{p}{(}\PYG{p}{)}
\PYG{g+go}{0.47108547995356098}
\PYG{g+gp}{\PYGZgt{}\PYGZgt{}\PYGZgt{} }\PYG{n+nb}{type}\PYG{p}{(}\PYG{n}{np}\PYG{o}{.}\PYG{n}{random}\PYG{o}{.}\PYG{n}{random\PYGZus{}sample}\PYG{p}{(}\PYG{p}{)}\PYG{p}{)}
\PYG{g+go}{\PYGZlt{}type \PYGZsq{}float\PYGZsq{}\PYGZgt{}}
\PYG{g+gp}{\PYGZgt{}\PYGZgt{}\PYGZgt{} }\PYG{n}{np}\PYG{o}{.}\PYG{n}{random}\PYG{o}{.}\PYG{n}{random\PYGZus{}sample}\PYG{p}{(}\PYG{p}{(}\PYG{l+m+mi}{5}\PYG{p}{,}\PYG{p}{)}\PYG{p}{)}
\PYG{g+go}{array([ 0.30220482,  0.86820401,  0.1654503 ,  0.11659149,  0.54323428])}
\end{Verbatim}

Three-by-two array of random numbers from {[}-5, 0):

\begin{Verbatim}[commandchars=\\\{\}]
\PYG{g+gp}{\PYGZgt{}\PYGZgt{}\PYGZgt{} }\PYG{l+m+mi}{5} \PYG{o}{*} \PYG{n}{np}\PYG{o}{.}\PYG{n}{random}\PYG{o}{.}\PYG{n}{random\PYGZus{}sample}\PYG{p}{(}\PYG{p}{(}\PYG{l+m+mi}{3}\PYG{p}{,} \PYG{l+m+mi}{2}\PYG{p}{)}\PYG{p}{)} \PYG{o}{\PYGZhy{}} \PYG{l+m+mi}{5}
\PYG{g+go}{array([[\PYGZhy{}3.99149989, \PYGZhy{}0.52338984],}
\PYG{g+go}{       [\PYGZhy{}2.99091858, \PYGZhy{}0.79479508],}
\PYG{g+go}{       [\PYGZhy{}1.23204345, \PYGZhy{}1.75224494]])}
\end{Verbatim}

\end{fulllineitems}

\index{random\_integers() (in module lib.IO.readfiles)}

\begin{fulllineitems}
\phantomsection\label{lib.IO:lib.IO.readfiles.random_integers}\pysiglinewithargsret{\code{lib.IO.readfiles.}\bfcode{random\_integers}}{\emph{low}, \emph{high=None}, \emph{size=None}}{}
Return random integers between \emph{low} and \emph{high}, inclusive.

Return random integers from the ``discrete uniform'' distribution in the
closed interval {[}\emph{low}, \emph{high}{]}.  If \emph{high} is None (the default),
then results are from {[}1, \emph{low}{]}.
\begin{description}
\item[{low}] \leavevmode{[}int{]}
Lowest (signed) integer to be drawn from the distribution (unless
\code{high=None}, in which case this parameter is the \emph{highest} such
integer).

\item[{high}] \leavevmode{[}int, optional{]}
If provided, the largest (signed) integer to be drawn from the
distribution (see above for behavior if \code{high=None}).

\item[{size}] \leavevmode{[}int or tuple of ints, optional{]}
Output shape. Default is None, in which case a single int is returned.

\end{description}
\begin{description}
\item[{out}] \leavevmode{[}int or ndarray of ints{]}
\emph{size}-shaped array of random integers from the appropriate
distribution, or a single such random int if \emph{size} not provided.

\end{description}
\begin{description}
\item[{random.randint}] \leavevmode{[}Similar to \emph{random\_integers}, only for the half-open{]}
interval {[}\emph{low}, \emph{high}), and 0 is the lowest value if \emph{high} is
omitted.

\end{description}

To sample from N evenly spaced floating-point numbers between a and b,
use:

\begin{Verbatim}[commandchars=\\\{\}]
\PYG{n}{a} \PYG{o}{+} \PYG{p}{(}\PYG{n}{b} \PYG{o}{\PYGZhy{}} \PYG{n}{a}\PYG{p}{)} \PYG{o}{*} \PYG{p}{(}\PYG{n}{np}\PYG{o}{.}\PYG{n}{random}\PYG{o}{.}\PYG{n}{random\PYGZus{}integers}\PYG{p}{(}\PYG{n}{N}\PYG{p}{)} \PYG{o}{\PYGZhy{}} \PYG{l+m+mi}{1}\PYG{p}{)} \PYG{o}{/} \PYG{p}{(}\PYG{n}{N} \PYG{o}{\PYGZhy{}} \PYG{l+m+mf}{1.}\PYG{p}{)}
\end{Verbatim}

\begin{Verbatim}[commandchars=\\\{\}]
\PYG{g+gp}{\PYGZgt{}\PYGZgt{}\PYGZgt{} }\PYG{n}{np}\PYG{o}{.}\PYG{n}{random}\PYG{o}{.}\PYG{n}{random\PYGZus{}integers}\PYG{p}{(}\PYG{l+m+mi}{5}\PYG{p}{)}
\PYG{g+go}{4}
\PYG{g+gp}{\PYGZgt{}\PYGZgt{}\PYGZgt{} }\PYG{n+nb}{type}\PYG{p}{(}\PYG{n}{np}\PYG{o}{.}\PYG{n}{random}\PYG{o}{.}\PYG{n}{random\PYGZus{}integers}\PYG{p}{(}\PYG{l+m+mi}{5}\PYG{p}{)}\PYG{p}{)}
\PYG{g+go}{\PYGZlt{}type \PYGZsq{}int\PYGZsq{}\PYGZgt{}}
\PYG{g+gp}{\PYGZgt{}\PYGZgt{}\PYGZgt{} }\PYG{n}{np}\PYG{o}{.}\PYG{n}{random}\PYG{o}{.}\PYG{n}{random\PYGZus{}integers}\PYG{p}{(}\PYG{l+m+mi}{5}\PYG{p}{,} \PYG{n}{size}\PYG{o}{=}\PYG{p}{(}\PYG{l+m+mf}{3.}\PYG{p}{,}\PYG{l+m+mf}{2.}\PYG{p}{)}\PYG{p}{)}
\PYG{g+go}{array([[5, 4],}
\PYG{g+go}{       [3, 3],}
\PYG{g+go}{       [4, 5]])}
\end{Verbatim}

Choose five random numbers from the set of five evenly-spaced
numbers between 0 and 2.5, inclusive (\emph{i.e.}, from the set
\({0, 5/8, 10/8, 15/8, 20/8}\)):

\begin{Verbatim}[commandchars=\\\{\}]
\PYG{g+gp}{\PYGZgt{}\PYGZgt{}\PYGZgt{} }\PYG{l+m+mf}{2.5} \PYG{o}{*} \PYG{p}{(}\PYG{n}{np}\PYG{o}{.}\PYG{n}{random}\PYG{o}{.}\PYG{n}{random\PYGZus{}integers}\PYG{p}{(}\PYG{l+m+mi}{5}\PYG{p}{,} \PYG{n}{size}\PYG{o}{=}\PYG{p}{(}\PYG{l+m+mi}{5}\PYG{p}{,}\PYG{p}{)}\PYG{p}{)} \PYG{o}{\PYGZhy{}} \PYG{l+m+mi}{1}\PYG{p}{)} \PYG{o}{/} \PYG{l+m+mf}{4.}
\PYG{g+go}{array([ 0.625,  1.25 ,  0.625,  0.625,  2.5  ])}
\end{Verbatim}

Roll two six sided dice 1000 times and sum the results:

\begin{Verbatim}[commandchars=\\\{\}]
\PYG{g+gp}{\PYGZgt{}\PYGZgt{}\PYGZgt{} }\PYG{n}{d1} \PYG{o}{=} \PYG{n}{np}\PYG{o}{.}\PYG{n}{random}\PYG{o}{.}\PYG{n}{random\PYGZus{}integers}\PYG{p}{(}\PYG{l+m+mi}{1}\PYG{p}{,} \PYG{l+m+mi}{6}\PYG{p}{,} \PYG{l+m+mi}{1000}\PYG{p}{)}
\PYG{g+gp}{\PYGZgt{}\PYGZgt{}\PYGZgt{} }\PYG{n}{d2} \PYG{o}{=} \PYG{n}{np}\PYG{o}{.}\PYG{n}{random}\PYG{o}{.}\PYG{n}{random\PYGZus{}integers}\PYG{p}{(}\PYG{l+m+mi}{1}\PYG{p}{,} \PYG{l+m+mi}{6}\PYG{p}{,} \PYG{l+m+mi}{1000}\PYG{p}{)}
\PYG{g+gp}{\PYGZgt{}\PYGZgt{}\PYGZgt{} }\PYG{n}{dsums} \PYG{o}{=} \PYG{n}{d1} \PYG{o}{+} \PYG{n}{d2}
\end{Verbatim}

Display results as a histogram:

\begin{Verbatim}[commandchars=\\\{\}]
\PYG{g+gp}{\PYGZgt{}\PYGZgt{}\PYGZgt{} }\PYG{k+kn}{import} \PYG{n+nn}{matplotlib.pyplot} \PYG{k+kn}{as} \PYG{n+nn}{plt}
\PYG{g+gp}{\PYGZgt{}\PYGZgt{}\PYGZgt{} }\PYG{n}{count}\PYG{p}{,} \PYG{n}{bins}\PYG{p}{,} \PYG{n}{ignored} \PYG{o}{=} \PYG{n}{plt}\PYG{o}{.}\PYG{n}{hist}\PYG{p}{(}\PYG{n}{dsums}\PYG{p}{,} \PYG{l+m+mi}{11}\PYG{p}{,} \PYG{n}{normed}\PYG{o}{=}\PYG{n+nb+bp}{True}\PYG{p}{)}
\PYG{g+gp}{\PYGZgt{}\PYGZgt{}\PYGZgt{} }\PYG{n}{plt}\PYG{o}{.}\PYG{n}{show}\PYG{p}{(}\PYG{p}{)}
\end{Verbatim}

\end{fulllineitems}

\index{random\_sample() (in module lib.IO.readfiles)}

\begin{fulllineitems}
\phantomsection\label{lib.IO:lib.IO.readfiles.random_sample}\pysiglinewithargsret{\code{lib.IO.readfiles.}\bfcode{random\_sample}}{\emph{size=None}}{}
Return random floats in the half-open interval {[}0.0, 1.0).

Results are from the ``continuous uniform'' distribution over the
stated interval.  To sample \(Unif[a, b), b > a\) multiply
the output of \emph{random\_sample} by \emph{(b-a)} and add \emph{a}:

\begin{Verbatim}[commandchars=\\\{\}]
\PYG{p}{(}\PYG{n}{b} \PYG{o}{\PYGZhy{}} \PYG{n}{a}\PYG{p}{)} \PYG{o}{*} \PYG{n}{random\PYGZus{}sample}\PYG{p}{(}\PYG{p}{)} \PYG{o}{+} \PYG{n}{a}
\end{Verbatim}
\begin{description}
\item[{size}] \leavevmode{[}int or tuple of ints, optional{]}
Defines the shape of the returned array of random floats. If None
(the default), returns a single float.

\end{description}
\begin{description}
\item[{out}] \leavevmode{[}float or ndarray of floats{]}
Array of random floats of shape \emph{size} (unless \code{size=None}, in which
case a single float is returned).

\end{description}

\begin{Verbatim}[commandchars=\\\{\}]
\PYG{g+gp}{\PYGZgt{}\PYGZgt{}\PYGZgt{} }\PYG{n}{np}\PYG{o}{.}\PYG{n}{random}\PYG{o}{.}\PYG{n}{random\PYGZus{}sample}\PYG{p}{(}\PYG{p}{)}
\PYG{g+go}{0.47108547995356098}
\PYG{g+gp}{\PYGZgt{}\PYGZgt{}\PYGZgt{} }\PYG{n+nb}{type}\PYG{p}{(}\PYG{n}{np}\PYG{o}{.}\PYG{n}{random}\PYG{o}{.}\PYG{n}{random\PYGZus{}sample}\PYG{p}{(}\PYG{p}{)}\PYG{p}{)}
\PYG{g+go}{\PYGZlt{}type \PYGZsq{}float\PYGZsq{}\PYGZgt{}}
\PYG{g+gp}{\PYGZgt{}\PYGZgt{}\PYGZgt{} }\PYG{n}{np}\PYG{o}{.}\PYG{n}{random}\PYG{o}{.}\PYG{n}{random\PYGZus{}sample}\PYG{p}{(}\PYG{p}{(}\PYG{l+m+mi}{5}\PYG{p}{,}\PYG{p}{)}\PYG{p}{)}
\PYG{g+go}{array([ 0.30220482,  0.86820401,  0.1654503 ,  0.11659149,  0.54323428])}
\end{Verbatim}

Three-by-two array of random numbers from {[}-5, 0):

\begin{Verbatim}[commandchars=\\\{\}]
\PYG{g+gp}{\PYGZgt{}\PYGZgt{}\PYGZgt{} }\PYG{l+m+mi}{5} \PYG{o}{*} \PYG{n}{np}\PYG{o}{.}\PYG{n}{random}\PYG{o}{.}\PYG{n}{random\PYGZus{}sample}\PYG{p}{(}\PYG{p}{(}\PYG{l+m+mi}{3}\PYG{p}{,} \PYG{l+m+mi}{2}\PYG{p}{)}\PYG{p}{)} \PYG{o}{\PYGZhy{}} \PYG{l+m+mi}{5}
\PYG{g+go}{array([[\PYGZhy{}3.99149989, \PYGZhy{}0.52338984],}
\PYG{g+go}{       [\PYGZhy{}2.99091858, \PYGZhy{}0.79479508],}
\PYG{g+go}{       [\PYGZhy{}1.23204345, \PYGZhy{}1.75224494]])}
\end{Verbatim}

\end{fulllineitems}

\index{ranf() (in module lib.IO.readfiles)}

\begin{fulllineitems}
\phantomsection\label{lib.IO:lib.IO.readfiles.ranf}\pysiglinewithargsret{\code{lib.IO.readfiles.}\bfcode{ranf}}{}{}
random\_sample(size=None)

Return random floats in the half-open interval {[}0.0, 1.0).

Results are from the ``continuous uniform'' distribution over the
stated interval.  To sample \(Unif[a, b), b > a\) multiply
the output of \emph{random\_sample} by \emph{(b-a)} and add \emph{a}:

\begin{Verbatim}[commandchars=\\\{\}]
\PYG{p}{(}\PYG{n}{b} \PYG{o}{\PYGZhy{}} \PYG{n}{a}\PYG{p}{)} \PYG{o}{*} \PYG{n}{random\PYGZus{}sample}\PYG{p}{(}\PYG{p}{)} \PYG{o}{+} \PYG{n}{a}
\end{Verbatim}
\begin{description}
\item[{size}] \leavevmode{[}int or tuple of ints, optional{]}
Defines the shape of the returned array of random floats. If None
(the default), returns a single float.

\end{description}
\begin{description}
\item[{out}] \leavevmode{[}float or ndarray of floats{]}
Array of random floats of shape \emph{size} (unless \code{size=None}, in which
case a single float is returned).

\end{description}

\begin{Verbatim}[commandchars=\\\{\}]
\PYG{g+gp}{\PYGZgt{}\PYGZgt{}\PYGZgt{} }\PYG{n}{np}\PYG{o}{.}\PYG{n}{random}\PYG{o}{.}\PYG{n}{random\PYGZus{}sample}\PYG{p}{(}\PYG{p}{)}
\PYG{g+go}{0.47108547995356098}
\PYG{g+gp}{\PYGZgt{}\PYGZgt{}\PYGZgt{} }\PYG{n+nb}{type}\PYG{p}{(}\PYG{n}{np}\PYG{o}{.}\PYG{n}{random}\PYG{o}{.}\PYG{n}{random\PYGZus{}sample}\PYG{p}{(}\PYG{p}{)}\PYG{p}{)}
\PYG{g+go}{\PYGZlt{}type \PYGZsq{}float\PYGZsq{}\PYGZgt{}}
\PYG{g+gp}{\PYGZgt{}\PYGZgt{}\PYGZgt{} }\PYG{n}{np}\PYG{o}{.}\PYG{n}{random}\PYG{o}{.}\PYG{n}{random\PYGZus{}sample}\PYG{p}{(}\PYG{p}{(}\PYG{l+m+mi}{5}\PYG{p}{,}\PYG{p}{)}\PYG{p}{)}
\PYG{g+go}{array([ 0.30220482,  0.86820401,  0.1654503 ,  0.11659149,  0.54323428])}
\end{Verbatim}

Three-by-two array of random numbers from {[}-5, 0):

\begin{Verbatim}[commandchars=\\\{\}]
\PYG{g+gp}{\PYGZgt{}\PYGZgt{}\PYGZgt{} }\PYG{l+m+mi}{5} \PYG{o}{*} \PYG{n}{np}\PYG{o}{.}\PYG{n}{random}\PYG{o}{.}\PYG{n}{random\PYGZus{}sample}\PYG{p}{(}\PYG{p}{(}\PYG{l+m+mi}{3}\PYG{p}{,} \PYG{l+m+mi}{2}\PYG{p}{)}\PYG{p}{)} \PYG{o}{\PYGZhy{}} \PYG{l+m+mi}{5}
\PYG{g+go}{array([[\PYGZhy{}3.99149989, \PYGZhy{}0.52338984],}
\PYG{g+go}{       [\PYGZhy{}2.99091858, \PYGZhy{}0.79479508],}
\PYG{g+go}{       [\PYGZhy{}1.23204345, \PYGZhy{}1.75224494]])}
\end{Verbatim}

\end{fulllineitems}

\index{rayleigh() (in module lib.IO.readfiles)}

\begin{fulllineitems}
\phantomsection\label{lib.IO:lib.IO.readfiles.rayleigh}\pysiglinewithargsret{\code{lib.IO.readfiles.}\bfcode{rayleigh}}{\emph{scale=1.0}, \emph{size=None}}{}
Draw samples from a Rayleigh distribution.

The \(\chi\) and Weibull distributions are generalizations of the
Rayleigh.
\begin{description}
\item[{scale}] \leavevmode{[}scalar{]}
Scale, also equals the mode. Should be \textgreater{}= 0.

\item[{size}] \leavevmode{[}int or tuple of ints, optional{]}
Shape of the output. Default is None, in which case a single
value is returned.

\end{description}

The probability density function for the Rayleigh distribution is
\begin{gather}
\begin{split}P(x;scale) = \frac{x}{scale^2}e^{\frac{-x^2}{2 \cdotp scale^2}}\end{split}\notag
\end{gather}
The Rayleigh distribution arises if the wind speed and wind direction are
both gaussian variables, then the vector wind velocity forms a Rayleigh
distribution. The Rayleigh distribution is used to model the expected
output from wind turbines.

Draw values from the distribution and plot the histogram

\begin{Verbatim}[commandchars=\\\{\}]
\PYG{g+gp}{\PYGZgt{}\PYGZgt{}\PYGZgt{} }\PYG{n}{values} \PYG{o}{=} \PYG{n}{hist}\PYG{p}{(}\PYG{n}{np}\PYG{o}{.}\PYG{n}{random}\PYG{o}{.}\PYG{n}{rayleigh}\PYG{p}{(}\PYG{l+m+mi}{3}\PYG{p}{,} \PYG{l+m+mi}{100000}\PYG{p}{)}\PYG{p}{,} \PYG{n}{bins}\PYG{o}{=}\PYG{l+m+mi}{200}\PYG{p}{,} \PYG{n}{normed}\PYG{o}{=}\PYG{n+nb+bp}{True}\PYG{p}{)}
\end{Verbatim}

Wave heights tend to follow a Rayleigh distribution. If the mean wave
height is 1 meter, what fraction of waves are likely to be larger than 3
meters?

\begin{Verbatim}[commandchars=\\\{\}]
\PYG{g+gp}{\PYGZgt{}\PYGZgt{}\PYGZgt{} }\PYG{n}{meanvalue} \PYG{o}{=} \PYG{l+m+mi}{1}
\PYG{g+gp}{\PYGZgt{}\PYGZgt{}\PYGZgt{} }\PYG{n}{modevalue} \PYG{o}{=} \PYG{n}{np}\PYG{o}{.}\PYG{n}{sqrt}\PYG{p}{(}\PYG{l+m+mi}{2} \PYG{o}{/} \PYG{n}{np}\PYG{o}{.}\PYG{n}{pi}\PYG{p}{)} \PYG{o}{*} \PYG{n}{meanvalue}
\PYG{g+gp}{\PYGZgt{}\PYGZgt{}\PYGZgt{} }\PYG{n}{s} \PYG{o}{=} \PYG{n}{np}\PYG{o}{.}\PYG{n}{random}\PYG{o}{.}\PYG{n}{rayleigh}\PYG{p}{(}\PYG{n}{modevalue}\PYG{p}{,} \PYG{l+m+mi}{1000000}\PYG{p}{)}
\end{Verbatim}

The percentage of waves larger than 3 meters is:

\begin{Verbatim}[commandchars=\\\{\}]
\PYG{g+gp}{\PYGZgt{}\PYGZgt{}\PYGZgt{} }\PYG{l+m+mf}{100.}\PYG{o}{*}\PYG{n+nb}{sum}\PYG{p}{(}\PYG{n}{s}\PYG{o}{\PYGZgt{}}\PYG{l+m+mi}{3}\PYG{p}{)}\PYG{o}{/}\PYG{l+m+mf}{1000000.}
\PYG{g+go}{0.087300000000000003}
\end{Verbatim}

\end{fulllineitems}

\index{readBufferedID() (in module lib.IO.readfiles)}

\begin{fulllineitems}
\phantomsection\label{lib.IO:lib.IO.readfiles.readBufferedID}\pysiglinewithargsret{\code{lib.IO.readfiles.}\bfcode{readBufferedID}}{\emph{tmpPath}}{}
\end{fulllineitems}

\index{readCSTRflux() (in module lib.IO.readfiles)}

\begin{fulllineitems}
\phantomsection\label{lib.IO:lib.IO.readfiles.readCSTRflux}\pysiglinewithargsret{\code{lib.IO.readfiles.}\bfcode{readCSTRflux}}{\emph{tmpPath}}{}
\end{fulllineitems}

\index{readConfFile() (in module lib.IO.readfiles)}

\begin{fulllineitems}
\phantomsection\label{lib.IO:lib.IO.readfiles.readConfFile}\pysiglinewithargsret{\code{lib.IO.readfiles.}\bfcode{readConfFile}}{\emph{tmpPath}}{}
\end{fulllineitems}

\index{readInitConfFile() (in module lib.IO.readfiles)}

\begin{fulllineitems}
\phantomsection\label{lib.IO:lib.IO.readfiles.readInitConfFile}\pysiglinewithargsret{\code{lib.IO.readfiles.}\bfcode{readInitConfFile}}{\emph{tmpPath}}{}
\end{fulllineitems}

\index{read\_sims\_conf\_file() (in module lib.IO.readfiles)}

\begin{fulllineitems}
\phantomsection\label{lib.IO:lib.IO.readfiles.read_sims_conf_file}\pysiglinewithargsret{\code{lib.IO.readfiles.}\bfcode{read\_sims\_conf\_file}}{\emph{paramFile='acsm2s.conf'}}{}
\end{fulllineitems}

\index{sample() (in module lib.IO.readfiles)}

\begin{fulllineitems}
\phantomsection\label{lib.IO:lib.IO.readfiles.sample}\pysiglinewithargsret{\code{lib.IO.readfiles.}\bfcode{sample}}{}{}
random\_sample(size=None)

Return random floats in the half-open interval {[}0.0, 1.0).

Results are from the ``continuous uniform'' distribution over the
stated interval.  To sample \(Unif[a, b), b > a\) multiply
the output of \emph{random\_sample} by \emph{(b-a)} and add \emph{a}:

\begin{Verbatim}[commandchars=\\\{\}]
\PYG{p}{(}\PYG{n}{b} \PYG{o}{\PYGZhy{}} \PYG{n}{a}\PYG{p}{)} \PYG{o}{*} \PYG{n}{random\PYGZus{}sample}\PYG{p}{(}\PYG{p}{)} \PYG{o}{+} \PYG{n}{a}
\end{Verbatim}
\begin{description}
\item[{size}] \leavevmode{[}int or tuple of ints, optional{]}
Defines the shape of the returned array of random floats. If None
(the default), returns a single float.

\end{description}
\begin{description}
\item[{out}] \leavevmode{[}float or ndarray of floats{]}
Array of random floats of shape \emph{size} (unless \code{size=None}, in which
case a single float is returned).

\end{description}

\begin{Verbatim}[commandchars=\\\{\}]
\PYG{g+gp}{\PYGZgt{}\PYGZgt{}\PYGZgt{} }\PYG{n}{np}\PYG{o}{.}\PYG{n}{random}\PYG{o}{.}\PYG{n}{random\PYGZus{}sample}\PYG{p}{(}\PYG{p}{)}
\PYG{g+go}{0.47108547995356098}
\PYG{g+gp}{\PYGZgt{}\PYGZgt{}\PYGZgt{} }\PYG{n+nb}{type}\PYG{p}{(}\PYG{n}{np}\PYG{o}{.}\PYG{n}{random}\PYG{o}{.}\PYG{n}{random\PYGZus{}sample}\PYG{p}{(}\PYG{p}{)}\PYG{p}{)}
\PYG{g+go}{\PYGZlt{}type \PYGZsq{}float\PYGZsq{}\PYGZgt{}}
\PYG{g+gp}{\PYGZgt{}\PYGZgt{}\PYGZgt{} }\PYG{n}{np}\PYG{o}{.}\PYG{n}{random}\PYG{o}{.}\PYG{n}{random\PYGZus{}sample}\PYG{p}{(}\PYG{p}{(}\PYG{l+m+mi}{5}\PYG{p}{,}\PYG{p}{)}\PYG{p}{)}
\PYG{g+go}{array([ 0.30220482,  0.86820401,  0.1654503 ,  0.11659149,  0.54323428])}
\end{Verbatim}

Three-by-two array of random numbers from {[}-5, 0):

\begin{Verbatim}[commandchars=\\\{\}]
\PYG{g+gp}{\PYGZgt{}\PYGZgt{}\PYGZgt{} }\PYG{l+m+mi}{5} \PYG{o}{*} \PYG{n}{np}\PYG{o}{.}\PYG{n}{random}\PYG{o}{.}\PYG{n}{random\PYGZus{}sample}\PYG{p}{(}\PYG{p}{(}\PYG{l+m+mi}{3}\PYG{p}{,} \PYG{l+m+mi}{2}\PYG{p}{)}\PYG{p}{)} \PYG{o}{\PYGZhy{}} \PYG{l+m+mi}{5}
\PYG{g+go}{array([[\PYGZhy{}3.99149989, \PYGZhy{}0.52338984],}
\PYG{g+go}{       [\PYGZhy{}2.99091858, \PYGZhy{}0.79479508],}
\PYG{g+go}{       [\PYGZhy{}1.23204345, \PYGZhy{}1.75224494]])}
\end{Verbatim}

\end{fulllineitems}

\index{seed() (in module lib.IO.readfiles)}

\begin{fulllineitems}
\phantomsection\label{lib.IO:lib.IO.readfiles.seed}\pysiglinewithargsret{\code{lib.IO.readfiles.}\bfcode{seed}}{\emph{seed=None}}{}
Seed the generator.

This method is called when \emph{RandomState} is initialized. It can be
called again to re-seed the generator. For details, see \emph{RandomState}.
\begin{description}
\item[{seed}] \leavevmode{[}int or array\_like, optional{]}
Seed for \emph{RandomState}.

\end{description}

RandomState

\end{fulllineitems}

\index{set\_state() (in module lib.IO.readfiles)}

\begin{fulllineitems}
\phantomsection\label{lib.IO:lib.IO.readfiles.set_state}\pysiglinewithargsret{\code{lib.IO.readfiles.}\bfcode{set\_state}}{\emph{state}}{}
Set the internal state of the generator from a tuple.

For use if one has reason to manually (re-)set the internal state of the
``Mersenne Twister''{\color{red}\bfseries{}{[}1{]}\_} pseudo-random number generating algorithm.
\begin{description}
\item[{state}] \leavevmode{[}tuple(str, ndarray of 624 uints, int, int, float){]}
The \emph{state} tuple has the following items:
\begin{enumerate}
\item {} 
the string `MT19937', specifying the Mersenne Twister algorithm.

\item {} 
a 1-D array of 624 unsigned integers \code{keys}.

\item {} 
an integer \code{pos}.

\item {} 
an integer \code{has\_gauss}.

\item {} 
a float \code{cached\_gaussian}.

\end{enumerate}

\end{description}
\begin{description}
\item[{out}] \leavevmode{[}None{]}
Returns `None' on success.

\end{description}

get\_state

\emph{set\_state} and \emph{get\_state} are not needed to work with any of the
random distributions in NumPy. If the internal state is manually altered,
the user should know exactly what he/she is doing.

For backwards compatibility, the form (str, array of 624 uints, int) is
also accepted although it is missing some information about the cached
Gaussian value: \code{state = ('MT19937', keys, pos)}.

\end{fulllineitems}

\index{shuffle() (in module lib.IO.readfiles)}

\begin{fulllineitems}
\phantomsection\label{lib.IO:lib.IO.readfiles.shuffle}\pysiglinewithargsret{\code{lib.IO.readfiles.}\bfcode{shuffle}}{\emph{x}}{}
Modify a sequence in-place by shuffling its contents.
\begin{description}
\item[{x}] \leavevmode{[}array\_like{]}
The array or list to be shuffled.

\end{description}

None

\begin{Verbatim}[commandchars=\\\{\}]
\PYG{g+gp}{\PYGZgt{}\PYGZgt{}\PYGZgt{} }\PYG{n}{arr} \PYG{o}{=} \PYG{n}{np}\PYG{o}{.}\PYG{n}{arange}\PYG{p}{(}\PYG{l+m+mi}{10}\PYG{p}{)}
\PYG{g+gp}{\PYGZgt{}\PYGZgt{}\PYGZgt{} }\PYG{n}{np}\PYG{o}{.}\PYG{n}{random}\PYG{o}{.}\PYG{n}{shuffle}\PYG{p}{(}\PYG{n}{arr}\PYG{p}{)}
\PYG{g+gp}{\PYGZgt{}\PYGZgt{}\PYGZgt{} }\PYG{n}{arr}
\PYG{g+go}{[1 7 5 2 9 4 3 6 0 8]}
\end{Verbatim}

This function only shuffles the array along the first index of a
multi-dimensional array:

\begin{Verbatim}[commandchars=\\\{\}]
\PYG{g+gp}{\PYGZgt{}\PYGZgt{}\PYGZgt{} }\PYG{n}{arr} \PYG{o}{=} \PYG{n}{np}\PYG{o}{.}\PYG{n}{arange}\PYG{p}{(}\PYG{l+m+mi}{9}\PYG{p}{)}\PYG{o}{.}\PYG{n}{reshape}\PYG{p}{(}\PYG{p}{(}\PYG{l+m+mi}{3}\PYG{p}{,} \PYG{l+m+mi}{3}\PYG{p}{)}\PYG{p}{)}
\PYG{g+gp}{\PYGZgt{}\PYGZgt{}\PYGZgt{} }\PYG{n}{np}\PYG{o}{.}\PYG{n}{random}\PYG{o}{.}\PYG{n}{shuffle}\PYG{p}{(}\PYG{n}{arr}\PYG{p}{)}
\PYG{g+gp}{\PYGZgt{}\PYGZgt{}\PYGZgt{} }\PYG{n}{arr}
\PYG{g+go}{array([[3, 4, 5],}
\PYG{g+go}{       [6, 7, 8],}
\PYG{g+go}{       [0, 1, 2]])}
\end{Verbatim}

\end{fulllineitems}

\index{splitRctParsLine() (in module lib.IO.readfiles)}

\begin{fulllineitems}
\phantomsection\label{lib.IO:lib.IO.readfiles.splitRctParsLine}\pysiglinewithargsret{\code{lib.IO.readfiles.}\bfcode{splitRctParsLine}}{\emph{tmpLine}}{}
\end{fulllineitems}

\index{standard\_cauchy() (in module lib.IO.readfiles)}

\begin{fulllineitems}
\phantomsection\label{lib.IO:lib.IO.readfiles.standard_cauchy}\pysiglinewithargsret{\code{lib.IO.readfiles.}\bfcode{standard\_cauchy}}{\emph{size=None}}{}
Standard Cauchy distribution with mode = 0.

Also known as the Lorentz distribution.
\begin{description}
\item[{size}] \leavevmode{[}int or tuple of ints{]}
Shape of the output.

\end{description}
\begin{description}
\item[{samples}] \leavevmode{[}ndarray or scalar{]}
The drawn samples.

\end{description}

The probability density function for the full Cauchy distribution is
\begin{gather}
\begin{split}P(x; x_0, \gamma) = \frac{1}{\pi \gamma \bigl[ 1+
(\frac{x-x_0}{\gamma})^2 \bigr] }\end{split}\notag
\end{gather}
and the Standard Cauchy distribution just sets \(x_0=0\) and
\(\gamma=1\)

The Cauchy distribution arises in the solution to the driven harmonic
oscillator problem, and also describes spectral line broadening. It
also describes the distribution of values at which a line tilted at
a random angle will cut the x axis.

When studying hypothesis tests that assume normality, seeing how the
tests perform on data from a Cauchy distribution is a good indicator of
their sensitivity to a heavy-tailed distribution, since the Cauchy looks
very much like a Gaussian distribution, but with heavier tails.

Draw samples and plot the distribution:

\begin{Verbatim}[commandchars=\\\{\}]
\PYG{g+gp}{\PYGZgt{}\PYGZgt{}\PYGZgt{} }\PYG{n}{s} \PYG{o}{=} \PYG{n}{np}\PYG{o}{.}\PYG{n}{random}\PYG{o}{.}\PYG{n}{standard\PYGZus{}cauchy}\PYG{p}{(}\PYG{l+m+mi}{1000000}\PYG{p}{)}
\PYG{g+gp}{\PYGZgt{}\PYGZgt{}\PYGZgt{} }\PYG{n}{s} \PYG{o}{=} \PYG{n}{s}\PYG{p}{[}\PYG{p}{(}\PYG{n}{s}\PYG{o}{\PYGZgt{}}\PYG{o}{\PYGZhy{}}\PYG{l+m+mi}{25}\PYG{p}{)} \PYG{o}{\PYGZam{}} \PYG{p}{(}\PYG{n}{s}\PYG{o}{\PYGZlt{}}\PYG{l+m+mi}{25}\PYG{p}{)}\PYG{p}{]}  \PYG{c}{\PYGZsh{} truncate distribution so it plots well}
\PYG{g+gp}{\PYGZgt{}\PYGZgt{}\PYGZgt{} }\PYG{n}{plt}\PYG{o}{.}\PYG{n}{hist}\PYG{p}{(}\PYG{n}{s}\PYG{p}{,} \PYG{n}{bins}\PYG{o}{=}\PYG{l+m+mi}{100}\PYG{p}{)}
\PYG{g+gp}{\PYGZgt{}\PYGZgt{}\PYGZgt{} }\PYG{n}{plt}\PYG{o}{.}\PYG{n}{show}\PYG{p}{(}\PYG{p}{)}
\end{Verbatim}

\end{fulllineitems}

\index{standard\_exponential() (in module lib.IO.readfiles)}

\begin{fulllineitems}
\phantomsection\label{lib.IO:lib.IO.readfiles.standard_exponential}\pysiglinewithargsret{\code{lib.IO.readfiles.}\bfcode{standard\_exponential}}{\emph{size=None}}{}
Draw samples from the standard exponential distribution.

\emph{standard\_exponential} is identical to the exponential distribution
with a scale parameter of 1.
\begin{description}
\item[{size}] \leavevmode{[}int or tuple of ints{]}
Shape of the output.

\end{description}
\begin{description}
\item[{out}] \leavevmode{[}float or ndarray{]}
Drawn samples.

\end{description}

Output a 3x8000 array:

\begin{Verbatim}[commandchars=\\\{\}]
\PYG{g+gp}{\PYGZgt{}\PYGZgt{}\PYGZgt{} }\PYG{n}{n} \PYG{o}{=} \PYG{n}{np}\PYG{o}{.}\PYG{n}{random}\PYG{o}{.}\PYG{n}{standard\PYGZus{}exponential}\PYG{p}{(}\PYG{p}{(}\PYG{l+m+mi}{3}\PYG{p}{,} \PYG{l+m+mi}{8000}\PYG{p}{)}\PYG{p}{)}
\end{Verbatim}

\end{fulllineitems}

\index{standard\_gamma() (in module lib.IO.readfiles)}

\begin{fulllineitems}
\phantomsection\label{lib.IO:lib.IO.readfiles.standard_gamma}\pysiglinewithargsret{\code{lib.IO.readfiles.}\bfcode{standard\_gamma}}{\emph{shape}, \emph{size=None}}{}
Draw samples from a Standard Gamma distribution.

Samples are drawn from a Gamma distribution with specified parameters,
shape (sometimes designated ``k'') and scale=1.
\begin{description}
\item[{shape}] \leavevmode{[}float{]}
Parameter, should be \textgreater{} 0.

\item[{size}] \leavevmode{[}int or tuple of ints{]}
Output shape.  If the given shape is, e.g., \code{(m, n, k)}, then
\code{m * n * k} samples are drawn.

\end{description}
\begin{description}
\item[{samples}] \leavevmode{[}ndarray or scalar{]}
The drawn samples.

\end{description}
\begin{description}
\item[{scipy.stats.distributions.gamma}] \leavevmode{[}probability density function,{]}
distribution or cumulative density function, etc.

\end{description}

The probability density for the Gamma distribution is
\begin{gather}
\begin{split}p(x) = x^{k-1}\frac{e^{-x/\theta}}{\theta^k\Gamma(k)},\end{split}\notag
\end{gather}
where \(k\) is the shape and \(\theta\) the scale,
and \(\Gamma\) is the Gamma function.

The Gamma distribution is often used to model the times to failure of
electronic components, and arises naturally in processes for which the
waiting times between Poisson distributed events are relevant.

Draw samples from the distribution:

\begin{Verbatim}[commandchars=\\\{\}]
\PYG{g+gp}{\PYGZgt{}\PYGZgt{}\PYGZgt{} }\PYG{n}{shape}\PYG{p}{,} \PYG{n}{scale} \PYG{o}{=} \PYG{l+m+mf}{2.}\PYG{p}{,} \PYG{l+m+mf}{1.} \PYG{c}{\PYGZsh{} mean and width}
\PYG{g+gp}{\PYGZgt{}\PYGZgt{}\PYGZgt{} }\PYG{n}{s} \PYG{o}{=} \PYG{n}{np}\PYG{o}{.}\PYG{n}{random}\PYG{o}{.}\PYG{n}{standard\PYGZus{}gamma}\PYG{p}{(}\PYG{n}{shape}\PYG{p}{,} \PYG{l+m+mi}{1000000}\PYG{p}{)}
\end{Verbatim}

Display the histogram of the samples, along with
the probability density function:

\begin{Verbatim}[commandchars=\\\{\}]
\PYG{g+gp}{\PYGZgt{}\PYGZgt{}\PYGZgt{} }\PYG{k+kn}{import} \PYG{n+nn}{matplotlib.pyplot} \PYG{k+kn}{as} \PYG{n+nn}{plt}
\PYG{g+gp}{\PYGZgt{}\PYGZgt{}\PYGZgt{} }\PYG{k+kn}{import} \PYG{n+nn}{scipy.special} \PYG{k+kn}{as} \PYG{n+nn}{sps}
\PYG{g+gp}{\PYGZgt{}\PYGZgt{}\PYGZgt{} }\PYG{n}{count}\PYG{p}{,} \PYG{n}{bins}\PYG{p}{,} \PYG{n}{ignored} \PYG{o}{=} \PYG{n}{plt}\PYG{o}{.}\PYG{n}{hist}\PYG{p}{(}\PYG{n}{s}\PYG{p}{,} \PYG{l+m+mi}{50}\PYG{p}{,} \PYG{n}{normed}\PYG{o}{=}\PYG{n+nb+bp}{True}\PYG{p}{)}
\PYG{g+gp}{\PYGZgt{}\PYGZgt{}\PYGZgt{} }\PYG{n}{y} \PYG{o}{=} \PYG{n}{bins}\PYG{o}{*}\PYG{o}{*}\PYG{p}{(}\PYG{n}{shape}\PYG{o}{\PYGZhy{}}\PYG{l+m+mi}{1}\PYG{p}{)} \PYG{o}{*} \PYG{p}{(}\PYG{p}{(}\PYG{n}{np}\PYG{o}{.}\PYG{n}{exp}\PYG{p}{(}\PYG{o}{\PYGZhy{}}\PYG{n}{bins}\PYG{o}{/}\PYG{n}{scale}\PYG{p}{)}\PYG{p}{)}\PYG{o}{/} \PYGZbs{}
\PYG{g+gp}{... }                      \PYG{p}{(}\PYG{n}{sps}\PYG{o}{.}\PYG{n}{gamma}\PYG{p}{(}\PYG{n}{shape}\PYG{p}{)} \PYG{o}{*} \PYG{n}{scale}\PYG{o}{*}\PYG{o}{*}\PYG{n}{shape}\PYG{p}{)}\PYG{p}{)}
\PYG{g+gp}{\PYGZgt{}\PYGZgt{}\PYGZgt{} }\PYG{n}{plt}\PYG{o}{.}\PYG{n}{plot}\PYG{p}{(}\PYG{n}{bins}\PYG{p}{,} \PYG{n}{y}\PYG{p}{,} \PYG{n}{linewidth}\PYG{o}{=}\PYG{l+m+mi}{2}\PYG{p}{,} \PYG{n}{color}\PYG{o}{=}\PYG{l+s}{\PYGZsq{}}\PYG{l+s}{r}\PYG{l+s}{\PYGZsq{}}\PYG{p}{)}
\PYG{g+gp}{\PYGZgt{}\PYGZgt{}\PYGZgt{} }\PYG{n}{plt}\PYG{o}{.}\PYG{n}{show}\PYG{p}{(}\PYG{p}{)}
\end{Verbatim}

\end{fulllineitems}

\index{standard\_normal() (in module lib.IO.readfiles)}

\begin{fulllineitems}
\phantomsection\label{lib.IO:lib.IO.readfiles.standard_normal}\pysiglinewithargsret{\code{lib.IO.readfiles.}\bfcode{standard\_normal}}{\emph{size=None}}{}
Returns samples from a Standard Normal distribution (mean=0, stdev=1).
\begin{description}
\item[{size}] \leavevmode{[}int or tuple of ints, optional{]}
Output shape. Default is None, in which case a single value is
returned.

\end{description}
\begin{description}
\item[{out}] \leavevmode{[}float or ndarray{]}
Drawn samples.

\end{description}

\begin{Verbatim}[commandchars=\\\{\}]
\PYG{g+gp}{\PYGZgt{}\PYGZgt{}\PYGZgt{} }\PYG{n}{s} \PYG{o}{=} \PYG{n}{np}\PYG{o}{.}\PYG{n}{random}\PYG{o}{.}\PYG{n}{standard\PYGZus{}normal}\PYG{p}{(}\PYG{l+m+mi}{8000}\PYG{p}{)}
\PYG{g+gp}{\PYGZgt{}\PYGZgt{}\PYGZgt{} }\PYG{n}{s}
\PYG{g+go}{array([ 0.6888893 ,  0.78096262, \PYGZhy{}0.89086505, ...,  0.49876311, \PYGZsh{}random}
\PYG{g+go}{       \PYGZhy{}0.38672696, \PYGZhy{}0.4685006 ])                               \PYGZsh{}random}
\PYG{g+gp}{\PYGZgt{}\PYGZgt{}\PYGZgt{} }\PYG{n}{s}\PYG{o}{.}\PYG{n}{shape}
\PYG{g+go}{(8000,)}
\PYG{g+gp}{\PYGZgt{}\PYGZgt{}\PYGZgt{} }\PYG{n}{s} \PYG{o}{=} \PYG{n}{np}\PYG{o}{.}\PYG{n}{random}\PYG{o}{.}\PYG{n}{standard\PYGZus{}normal}\PYG{p}{(}\PYG{n}{size}\PYG{o}{=}\PYG{p}{(}\PYG{l+m+mi}{3}\PYG{p}{,} \PYG{l+m+mi}{4}\PYG{p}{,} \PYG{l+m+mi}{2}\PYG{p}{)}\PYG{p}{)}
\PYG{g+gp}{\PYGZgt{}\PYGZgt{}\PYGZgt{} }\PYG{n}{s}\PYG{o}{.}\PYG{n}{shape}
\PYG{g+go}{(3, 4, 2)}
\end{Verbatim}

\end{fulllineitems}

\index{standard\_t() (in module lib.IO.readfiles)}

\begin{fulllineitems}
\phantomsection\label{lib.IO:lib.IO.readfiles.standard_t}\pysiglinewithargsret{\code{lib.IO.readfiles.}\bfcode{standard\_t}}{\emph{df}, \emph{size=None}}{}
Standard Student's t distribution with df degrees of freedom.

A special case of the hyperbolic distribution.
As \emph{df} gets large, the result resembles that of the standard normal
distribution (\emph{standard\_normal}).
\begin{description}
\item[{df}] \leavevmode{[}int{]}
Degrees of freedom, should be \textgreater{} 0.

\item[{size}] \leavevmode{[}int or tuple of ints, optional{]}
Output shape. Default is None, in which case a single value is
returned.

\end{description}
\begin{description}
\item[{samples}] \leavevmode{[}ndarray or scalar{]}
Drawn samples.

\end{description}

The probability density function for the t distribution is
\begin{gather}
\begin{split}P(x, df) = \frac{\Gamma(\frac{df+1}{2})}{\sqrt{\pi df}
\Gamma(\frac{df}{2})}\Bigl( 1+\frac{x^2}{df} \Bigr)^{-(df+1)/2}\end{split}\notag
\end{gather}
The t test is based on an assumption that the data come from a Normal
distribution. The t test provides a way to test whether the sample mean
(that is the mean calculated from the data) is a good estimate of the true
mean.

The derivation of the t-distribution was forst published in 1908 by William
Gisset while working for the Guinness Brewery in Dublin. Due to proprietary
issues, he had to publish under a pseudonym, and so he used the name
Student.

From Dalgaard page 83 {\color{red}\bfseries{}{[}1{]}\_}, suppose the daily energy intake for 11
women in Kj is:

\begin{Verbatim}[commandchars=\\\{\}]
\PYG{g+gp}{\PYGZgt{}\PYGZgt{}\PYGZgt{} }\PYG{n}{intake} \PYG{o}{=} \PYG{n}{np}\PYG{o}{.}\PYG{n}{array}\PYG{p}{(}\PYG{p}{[}\PYG{l+m+mf}{5260.}\PYG{p}{,} \PYG{l+m+mi}{5470}\PYG{p}{,} \PYG{l+m+mi}{5640}\PYG{p}{,} \PYG{l+m+mi}{6180}\PYG{p}{,} \PYG{l+m+mi}{6390}\PYG{p}{,} \PYG{l+m+mi}{6515}\PYG{p}{,} \PYG{l+m+mi}{6805}\PYG{p}{,} \PYG{l+m+mi}{7515}\PYG{p}{,} \PYGZbs{}
\PYG{g+gp}{... }                   \PYG{l+m+mi}{7515}\PYG{p}{,} \PYG{l+m+mi}{8230}\PYG{p}{,} \PYG{l+m+mi}{8770}\PYG{p}{]}\PYG{p}{)}
\end{Verbatim}

Does their energy intake deviate systematically from the recommended
value of 7725 kJ?

We have 10 degrees of freedom, so is the sample mean within 95\% of the
recommended value?

\begin{Verbatim}[commandchars=\\\{\}]
\PYG{g+gp}{\PYGZgt{}\PYGZgt{}\PYGZgt{} }\PYG{n}{s} \PYG{o}{=} \PYG{n}{np}\PYG{o}{.}\PYG{n}{random}\PYG{o}{.}\PYG{n}{standard\PYGZus{}t}\PYG{p}{(}\PYG{l+m+mi}{10}\PYG{p}{,} \PYG{n}{size}\PYG{o}{=}\PYG{l+m+mi}{100000}\PYG{p}{)}
\PYG{g+gp}{\PYGZgt{}\PYGZgt{}\PYGZgt{} }\PYG{n}{np}\PYG{o}{.}\PYG{n}{mean}\PYG{p}{(}\PYG{n}{intake}\PYG{p}{)}
\PYG{g+go}{6753.636363636364}
\PYG{g+gp}{\PYGZgt{}\PYGZgt{}\PYGZgt{} }\PYG{n}{intake}\PYG{o}{.}\PYG{n}{std}\PYG{p}{(}\PYG{n}{ddof}\PYG{o}{=}\PYG{l+m+mi}{1}\PYG{p}{)}
\PYG{g+go}{1142.1232221373727}
\end{Verbatim}

Calculate the t statistic, setting the ddof parameter to the unbiased
value so the divisor in the standard deviation will be degrees of
freedom, N-1.

\begin{Verbatim}[commandchars=\\\{\}]
\PYG{g+gp}{\PYGZgt{}\PYGZgt{}\PYGZgt{} }\PYG{n}{t} \PYG{o}{=} \PYG{p}{(}\PYG{n}{np}\PYG{o}{.}\PYG{n}{mean}\PYG{p}{(}\PYG{n}{intake}\PYG{p}{)}\PYG{o}{\PYGZhy{}}\PYG{l+m+mi}{7725}\PYG{p}{)}\PYG{o}{/}\PYG{p}{(}\PYG{n}{intake}\PYG{o}{.}\PYG{n}{std}\PYG{p}{(}\PYG{n}{ddof}\PYG{o}{=}\PYG{l+m+mi}{1}\PYG{p}{)}\PYG{o}{/}\PYG{n}{np}\PYG{o}{.}\PYG{n}{sqrt}\PYG{p}{(}\PYG{n+nb}{len}\PYG{p}{(}\PYG{n}{intake}\PYG{p}{)}\PYG{p}{)}\PYG{p}{)}
\PYG{g+gp}{\PYGZgt{}\PYGZgt{}\PYGZgt{} }\PYG{k+kn}{import} \PYG{n+nn}{matplotlib.pyplot} \PYG{k+kn}{as} \PYG{n+nn}{plt}
\PYG{g+gp}{\PYGZgt{}\PYGZgt{}\PYGZgt{} }\PYG{n}{h} \PYG{o}{=} \PYG{n}{plt}\PYG{o}{.}\PYG{n}{hist}\PYG{p}{(}\PYG{n}{s}\PYG{p}{,} \PYG{n}{bins}\PYG{o}{=}\PYG{l+m+mi}{100}\PYG{p}{,} \PYG{n}{normed}\PYG{o}{=}\PYG{n+nb+bp}{True}\PYG{p}{)}
\end{Verbatim}

For a one-sided t-test, how far out in the distribution does the t
statistic appear?

\begin{Verbatim}[commandchars=\\\{\}]
\PYG{g+gp}{\PYGZgt{}\PYGZgt{}\PYGZgt{} }\PYG{o}{\PYGZgt{}\PYGZgt{}}\PYG{o}{\PYGZgt{}} \PYG{n}{np}\PYG{o}{.}\PYG{n}{sum}\PYG{p}{(}\PYG{n}{s}\PYG{o}{\PYGZlt{}}\PYG{n}{t}\PYG{p}{)} \PYG{o}{/} \PYG{n+nb}{float}\PYG{p}{(}\PYG{n+nb}{len}\PYG{p}{(}\PYG{n}{s}\PYG{p}{)}\PYG{p}{)}
\PYG{g+go}{0.0090699999999999999  \PYGZsh{}random}
\end{Verbatim}

So the p-value is about 0.009, which says the null hypothesis has a
probability of about 99\% of being true.

\end{fulllineitems}

\index{triangular() (in module lib.IO.readfiles)}

\begin{fulllineitems}
\phantomsection\label{lib.IO:lib.IO.readfiles.triangular}\pysiglinewithargsret{\code{lib.IO.readfiles.}\bfcode{triangular}}{\emph{left}, \emph{mode}, \emph{right}, \emph{size=None}}{}
Draw samples from the triangular distribution.

The triangular distribution is a continuous probability distribution with
lower limit left, peak at mode, and upper limit right. Unlike the other
distributions, these parameters directly define the shape of the pdf.
\begin{description}
\item[{left}] \leavevmode{[}scalar{]}
Lower limit.

\item[{mode}] \leavevmode{[}scalar{]}
The value where the peak of the distribution occurs.
The value should fulfill the condition \code{left \textless{}= mode \textless{}= right}.

\item[{right}] \leavevmode{[}scalar{]}
Upper limit, should be larger than \emph{left}.

\item[{size}] \leavevmode{[}int or tuple of ints, optional{]}
Output shape. Default is None, in which case a single value is
returned.

\end{description}
\begin{description}
\item[{samples}] \leavevmode{[}ndarray or scalar{]}
The returned samples all lie in the interval {[}left, right{]}.

\end{description}

The probability density function for the Triangular distribution is
\begin{gather}
\begin{split}P(x;l, m, r) = \begin{cases}
\frac{2(x-l)}{(r-l)(m-l)}& \text{for $l \leq x \leq m$},\\
\frac{2(m-x)}{(r-l)(r-m)}& \text{for $m \leq x \leq r$},\\
0& \text{otherwise}.
\end{cases}\end{split}\notag
\end{gather}
The triangular distribution is often used in ill-defined problems where the
underlying distribution is not known, but some knowledge of the limits and
mode exists. Often it is used in simulations.

Draw values from the distribution and plot the histogram:

\begin{Verbatim}[commandchars=\\\{\}]
\PYG{g+gp}{\PYGZgt{}\PYGZgt{}\PYGZgt{} }\PYG{k+kn}{import} \PYG{n+nn}{matplotlib.pyplot} \PYG{k+kn}{as} \PYG{n+nn}{plt}
\PYG{g+gp}{\PYGZgt{}\PYGZgt{}\PYGZgt{} }\PYG{n}{h} \PYG{o}{=} \PYG{n}{plt}\PYG{o}{.}\PYG{n}{hist}\PYG{p}{(}\PYG{n}{np}\PYG{o}{.}\PYG{n}{random}\PYG{o}{.}\PYG{n}{triangular}\PYG{p}{(}\PYG{o}{\PYGZhy{}}\PYG{l+m+mi}{3}\PYG{p}{,} \PYG{l+m+mi}{0}\PYG{p}{,} \PYG{l+m+mi}{8}\PYG{p}{,} \PYG{l+m+mi}{100000}\PYG{p}{)}\PYG{p}{,} \PYG{n}{bins}\PYG{o}{=}\PYG{l+m+mi}{200}\PYG{p}{,}
\PYG{g+gp}{... }             \PYG{n}{normed}\PYG{o}{=}\PYG{n+nb+bp}{True}\PYG{p}{)}
\PYG{g+gp}{\PYGZgt{}\PYGZgt{}\PYGZgt{} }\PYG{n}{plt}\PYG{o}{.}\PYG{n}{show}\PYG{p}{(}\PYG{p}{)}
\end{Verbatim}

\end{fulllineitems}

\index{uniform() (in module lib.IO.readfiles)}

\begin{fulllineitems}
\phantomsection\label{lib.IO:lib.IO.readfiles.uniform}\pysiglinewithargsret{\code{lib.IO.readfiles.}\bfcode{uniform}}{\emph{low=0.0}, \emph{high=1.0}, \emph{size=1}}{}
Draw samples from a uniform distribution.

Samples are uniformly distributed over the half-open interval
\code{{[}low, high)} (includes low, but excludes high).  In other words,
any value within the given interval is equally likely to be drawn
by \emph{uniform}.
\begin{description}
\item[{low}] \leavevmode{[}float, optional{]}
Lower boundary of the output interval.  All values generated will be
greater than or equal to low.  The default value is 0.

\item[{high}] \leavevmode{[}float{]}
Upper boundary of the output interval.  All values generated will be
less than high.  The default value is 1.0.

\item[{size}] \leavevmode{[}int or tuple of ints, optional{]}
Shape of output.  If the given size is, for example, (m,n,k),
m*n*k samples are generated.  If no shape is specified, a single sample
is returned.

\end{description}
\begin{description}
\item[{out}] \leavevmode{[}ndarray{]}
Drawn samples, with shape \emph{size}.

\end{description}

randint : Discrete uniform distribution, yielding integers.
random\_integers : Discrete uniform distribution over the closed
\begin{quote}

interval \code{{[}low, high{]}}.
\end{quote}

random\_sample : Floats uniformly distributed over \code{{[}0, 1)}.
random : Alias for \emph{random\_sample}.
rand : Convenience function that accepts dimensions as input, e.g.,
\begin{quote}

\code{rand(2,2)} would generate a 2-by-2 array of floats,
uniformly distributed over \code{{[}0, 1)}.
\end{quote}

The probability density function of the uniform distribution is
\begin{gather}
\begin{split}p(x) = \frac{1}{b - a}\end{split}\notag
\end{gather}
anywhere within the interval \code{{[}a, b)}, and zero elsewhere.

Draw samples from the distribution:

\begin{Verbatim}[commandchars=\\\{\}]
\PYG{g+gp}{\PYGZgt{}\PYGZgt{}\PYGZgt{} }\PYG{n}{s} \PYG{o}{=} \PYG{n}{np}\PYG{o}{.}\PYG{n}{random}\PYG{o}{.}\PYG{n}{uniform}\PYG{p}{(}\PYG{o}{\PYGZhy{}}\PYG{l+m+mi}{1}\PYG{p}{,}\PYG{l+m+mi}{0}\PYG{p}{,}\PYG{l+m+mi}{1000}\PYG{p}{)}
\end{Verbatim}

All values are within the given interval:

\begin{Verbatim}[commandchars=\\\{\}]
\PYG{g+gp}{\PYGZgt{}\PYGZgt{}\PYGZgt{} }\PYG{n}{np}\PYG{o}{.}\PYG{n}{all}\PYG{p}{(}\PYG{n}{s} \PYG{o}{\PYGZgt{}}\PYG{o}{=} \PYG{o}{\PYGZhy{}}\PYG{l+m+mi}{1}\PYG{p}{)}
\PYG{g+go}{True}
\PYG{g+gp}{\PYGZgt{}\PYGZgt{}\PYGZgt{} }\PYG{n}{np}\PYG{o}{.}\PYG{n}{all}\PYG{p}{(}\PYG{n}{s} \PYG{o}{\PYGZlt{}} \PYG{l+m+mi}{0}\PYG{p}{)}
\PYG{g+go}{True}
\end{Verbatim}

Display the histogram of the samples, along with the
probability density function:

\begin{Verbatim}[commandchars=\\\{\}]
\PYG{g+gp}{\PYGZgt{}\PYGZgt{}\PYGZgt{} }\PYG{k+kn}{import} \PYG{n+nn}{matplotlib.pyplot} \PYG{k+kn}{as} \PYG{n+nn}{plt}
\PYG{g+gp}{\PYGZgt{}\PYGZgt{}\PYGZgt{} }\PYG{n}{count}\PYG{p}{,} \PYG{n}{bins}\PYG{p}{,} \PYG{n}{ignored} \PYG{o}{=} \PYG{n}{plt}\PYG{o}{.}\PYG{n}{hist}\PYG{p}{(}\PYG{n}{s}\PYG{p}{,} \PYG{l+m+mi}{15}\PYG{p}{,} \PYG{n}{normed}\PYG{o}{=}\PYG{n+nb+bp}{True}\PYG{p}{)}
\PYG{g+gp}{\PYGZgt{}\PYGZgt{}\PYGZgt{} }\PYG{n}{plt}\PYG{o}{.}\PYG{n}{plot}\PYG{p}{(}\PYG{n}{bins}\PYG{p}{,} \PYG{n}{np}\PYG{o}{.}\PYG{n}{ones\PYGZus{}like}\PYG{p}{(}\PYG{n}{bins}\PYG{p}{)}\PYG{p}{,} \PYG{n}{linewidth}\PYG{o}{=}\PYG{l+m+mi}{2}\PYG{p}{,} \PYG{n}{color}\PYG{o}{=}\PYG{l+s}{\PYGZsq{}}\PYG{l+s}{r}\PYG{l+s}{\PYGZsq{}}\PYG{p}{)}
\PYG{g+gp}{\PYGZgt{}\PYGZgt{}\PYGZgt{} }\PYG{n}{plt}\PYG{o}{.}\PYG{n}{show}\PYG{p}{(}\PYG{p}{)}
\end{Verbatim}

\end{fulllineitems}

\index{vonmises() (in module lib.IO.readfiles)}

\begin{fulllineitems}
\phantomsection\label{lib.IO:lib.IO.readfiles.vonmises}\pysiglinewithargsret{\code{lib.IO.readfiles.}\bfcode{vonmises}}{\emph{mu}, \emph{kappa}, \emph{size=None}}{}
Draw samples from a von Mises distribution.

Samples are drawn from a von Mises distribution with specified mode
(mu) and dispersion (kappa), on the interval {[}-pi, pi{]}.

The von Mises distribution (also known as the circular normal
distribution) is a continuous probability distribution on the unit
circle.  It may be thought of as the circular analogue of the normal
distribution.
\begin{description}
\item[{mu}] \leavevmode{[}float{]}
Mode (``center'') of the distribution.

\item[{kappa}] \leavevmode{[}float{]}
Dispersion of the distribution, has to be \textgreater{}=0.

\item[{size}] \leavevmode{[}int or tuple of int{]}
Output shape.  If the given shape is, e.g., \code{(m, n, k)}, then
\code{m * n * k} samples are drawn.

\end{description}
\begin{description}
\item[{samples}] \leavevmode{[}scalar or ndarray{]}
The returned samples, which are in the interval {[}-pi, pi{]}.

\end{description}
\begin{description}
\item[{scipy.stats.distributions.vonmises}] \leavevmode{[}probability density function,{]}
distribution, or cumulative density function, etc.

\end{description}

The probability density for the von Mises distribution is
\begin{gather}
\begin{split}p(x) = \frac{e^{\kappa cos(x-\mu)}}{2\pi I_0(\kappa)},\end{split}\notag
\end{gather}
where \(\mu\) is the mode and \(\kappa\) the dispersion,
and \(I_0(\kappa)\) is the modified Bessel function of order 0.

The von Mises is named for Richard Edler von Mises, who was born in
Austria-Hungary, in what is now the Ukraine.  He fled to the United
States in 1939 and became a professor at Harvard.  He worked in
probability theory, aerodynamics, fluid mechanics, and philosophy of
science.

Abramowitz, M. and Stegun, I. A. (ed.), \emph{Handbook of Mathematical
Functions}, New York: Dover, 1965.

von Mises, R., \emph{Mathematical Theory of Probability and Statistics},
New York: Academic Press, 1964.

Draw samples from the distribution:

\begin{Verbatim}[commandchars=\\\{\}]
\PYG{g+gp}{\PYGZgt{}\PYGZgt{}\PYGZgt{} }\PYG{n}{mu}\PYG{p}{,} \PYG{n}{kappa} \PYG{o}{=} \PYG{l+m+mf}{0.0}\PYG{p}{,} \PYG{l+m+mf}{4.0} \PYG{c}{\PYGZsh{} mean and dispersion}
\PYG{g+gp}{\PYGZgt{}\PYGZgt{}\PYGZgt{} }\PYG{n}{s} \PYG{o}{=} \PYG{n}{np}\PYG{o}{.}\PYG{n}{random}\PYG{o}{.}\PYG{n}{vonmises}\PYG{p}{(}\PYG{n}{mu}\PYG{p}{,} \PYG{n}{kappa}\PYG{p}{,} \PYG{l+m+mi}{1000}\PYG{p}{)}
\end{Verbatim}

Display the histogram of the samples, along with
the probability density function:

\begin{Verbatim}[commandchars=\\\{\}]
\PYG{g+gp}{\PYGZgt{}\PYGZgt{}\PYGZgt{} }\PYG{k+kn}{import} \PYG{n+nn}{matplotlib.pyplot} \PYG{k+kn}{as} \PYG{n+nn}{plt}
\PYG{g+gp}{\PYGZgt{}\PYGZgt{}\PYGZgt{} }\PYG{k+kn}{import} \PYG{n+nn}{scipy.special} \PYG{k+kn}{as} \PYG{n+nn}{sps}
\PYG{g+gp}{\PYGZgt{}\PYGZgt{}\PYGZgt{} }\PYG{n}{count}\PYG{p}{,} \PYG{n}{bins}\PYG{p}{,} \PYG{n}{ignored} \PYG{o}{=} \PYG{n}{plt}\PYG{o}{.}\PYG{n}{hist}\PYG{p}{(}\PYG{n}{s}\PYG{p}{,} \PYG{l+m+mi}{50}\PYG{p}{,} \PYG{n}{normed}\PYG{o}{=}\PYG{n+nb+bp}{True}\PYG{p}{)}
\PYG{g+gp}{\PYGZgt{}\PYGZgt{}\PYGZgt{} }\PYG{n}{x} \PYG{o}{=} \PYG{n}{np}\PYG{o}{.}\PYG{n}{arange}\PYG{p}{(}\PYG{o}{\PYGZhy{}}\PYG{n}{np}\PYG{o}{.}\PYG{n}{pi}\PYG{p}{,} \PYG{n}{np}\PYG{o}{.}\PYG{n}{pi}\PYG{p}{,} \PYG{l+m+mi}{2}\PYG{o}{*}\PYG{n}{np}\PYG{o}{.}\PYG{n}{pi}\PYG{o}{/}\PYG{l+m+mf}{50.}\PYG{p}{)}
\PYG{g+gp}{\PYGZgt{}\PYGZgt{}\PYGZgt{} }\PYG{n}{y} \PYG{o}{=} \PYG{o}{\PYGZhy{}}\PYG{n}{np}\PYG{o}{.}\PYG{n}{exp}\PYG{p}{(}\PYG{n}{kappa}\PYG{o}{*}\PYG{n}{np}\PYG{o}{.}\PYG{n}{cos}\PYG{p}{(}\PYG{n}{x}\PYG{o}{\PYGZhy{}}\PYG{n}{mu}\PYG{p}{)}\PYG{p}{)}\PYG{o}{/}\PYG{p}{(}\PYG{l+m+mi}{2}\PYG{o}{*}\PYG{n}{np}\PYG{o}{.}\PYG{n}{pi}\PYG{o}{*}\PYG{n}{sps}\PYG{o}{.}\PYG{n}{jn}\PYG{p}{(}\PYG{l+m+mi}{0}\PYG{p}{,}\PYG{n}{kappa}\PYG{p}{)}\PYG{p}{)}
\PYG{g+gp}{\PYGZgt{}\PYGZgt{}\PYGZgt{} }\PYG{n}{plt}\PYG{o}{.}\PYG{n}{plot}\PYG{p}{(}\PYG{n}{x}\PYG{p}{,} \PYG{n}{y}\PYG{o}{/}\PYG{n+nb}{max}\PYG{p}{(}\PYG{n}{y}\PYG{p}{)}\PYG{p}{,} \PYG{n}{linewidth}\PYG{o}{=}\PYG{l+m+mi}{2}\PYG{p}{,} \PYG{n}{color}\PYG{o}{=}\PYG{l+s}{\PYGZsq{}}\PYG{l+s}{r}\PYG{l+s}{\PYGZsq{}}\PYG{p}{)}
\PYG{g+gp}{\PYGZgt{}\PYGZgt{}\PYGZgt{} }\PYG{n}{plt}\PYG{o}{.}\PYG{n}{show}\PYG{p}{(}\PYG{p}{)}
\end{Verbatim}

\end{fulllineitems}

\index{wald() (in module lib.IO.readfiles)}

\begin{fulllineitems}
\phantomsection\label{lib.IO:lib.IO.readfiles.wald}\pysiglinewithargsret{\code{lib.IO.readfiles.}\bfcode{wald}}{\emph{mean}, \emph{scale}, \emph{size=None}}{}
Draw samples from a Wald, or Inverse Gaussian, distribution.

As the scale approaches infinity, the distribution becomes more like a
Gaussian.

Some references claim that the Wald is an Inverse Gaussian with mean=1, but
this is by no means universal.

The Inverse Gaussian distribution was first studied in relationship to
Brownian motion. In 1956 M.C.K. Tweedie used the name Inverse Gaussian
because there is an inverse relationship between the time to cover a unit
distance and distance covered in unit time.
\begin{description}
\item[{mean}] \leavevmode{[}scalar{]}
Distribution mean, should be \textgreater{} 0.

\item[{scale}] \leavevmode{[}scalar{]}
Scale parameter, should be \textgreater{}= 0.

\item[{size}] \leavevmode{[}int or tuple of ints, optional{]}
Output shape. Default is None, in which case a single value is
returned.

\end{description}
\begin{description}
\item[{samples}] \leavevmode{[}ndarray or scalar{]}
Drawn sample, all greater than zero.

\end{description}

The probability density function for the Wald distribution is
\begin{gather}
\begin{split}P(x;mean,scale) = \sqrt{\frac{scale}{2\pi x^3}}e^
\frac{-scale(x-mean)^2}{2\cdotp mean^2x}\end{split}\notag
\end{gather}
As noted above the Inverse Gaussian distribution first arise from attempts
to model Brownian Motion. It is also a competitor to the Weibull for use in
reliability modeling and modeling stock returns and interest rate
processes.

Draw values from the distribution and plot the histogram:

\begin{Verbatim}[commandchars=\\\{\}]
\PYG{g+gp}{\PYGZgt{}\PYGZgt{}\PYGZgt{} }\PYG{k+kn}{import} \PYG{n+nn}{matplotlib.pyplot} \PYG{k+kn}{as} \PYG{n+nn}{plt}
\PYG{g+gp}{\PYGZgt{}\PYGZgt{}\PYGZgt{} }\PYG{n}{h} \PYG{o}{=} \PYG{n}{plt}\PYG{o}{.}\PYG{n}{hist}\PYG{p}{(}\PYG{n}{np}\PYG{o}{.}\PYG{n}{random}\PYG{o}{.}\PYG{n}{wald}\PYG{p}{(}\PYG{l+m+mi}{3}\PYG{p}{,} \PYG{l+m+mi}{2}\PYG{p}{,} \PYG{l+m+mi}{100000}\PYG{p}{)}\PYG{p}{,} \PYG{n}{bins}\PYG{o}{=}\PYG{l+m+mi}{200}\PYG{p}{,} \PYG{n}{normed}\PYG{o}{=}\PYG{n+nb+bp}{True}\PYG{p}{)}
\PYG{g+gp}{\PYGZgt{}\PYGZgt{}\PYGZgt{} }\PYG{n}{plt}\PYG{o}{.}\PYG{n}{show}\PYG{p}{(}\PYG{p}{)}
\end{Verbatim}

\end{fulllineitems}

\index{weibull() (in module lib.IO.readfiles)}

\begin{fulllineitems}
\phantomsection\label{lib.IO:lib.IO.readfiles.weibull}\pysiglinewithargsret{\code{lib.IO.readfiles.}\bfcode{weibull}}{\emph{a}, \emph{size=None}}{}
Weibull distribution.

Draw samples from a 1-parameter Weibull distribution with the given
shape parameter \emph{a}.
\begin{gather}
\begin{split}X = (-ln(U))^{1/a}\end{split}\notag
\end{gather}
Here, U is drawn from the uniform distribution over (0,1{]}.

The more common 2-parameter Weibull, including a scale parameter
\(\lambda\) is just \(X = \lambda(-ln(U))^{1/a}\).
\begin{description}
\item[{a}] \leavevmode{[}float{]}
Shape of the distribution.

\item[{size}] \leavevmode{[}tuple of ints{]}
Output shape.  If the given shape is, e.g., \code{(m, n, k)}, then
\code{m * n * k} samples are drawn.

\end{description}

scipy.stats.distributions.weibull\_max
scipy.stats.distributions.weibull\_min
scipy.stats.distributions.genextreme
gumbel

The Weibull (or Type III asymptotic extreme value distribution for smallest
values, SEV Type III, or Rosin-Rammler distribution) is one of a class of
Generalized Extreme Value (GEV) distributions used in modeling extreme
value problems.  This class includes the Gumbel and Frechet distributions.

The probability density for the Weibull distribution is
\begin{gather}
\begin{split}p(x) = \frac{a}
{\lambda}(\frac{x}{\lambda})^{a-1}e^{-(x/\lambda)^a},\end{split}\notag
\end{gather}
where \(a\) is the shape and \(\lambda\) the scale.

The function has its peak (the mode) at
\(\lambda(\frac{a-1}{a})^{1/a}\).

When \code{a = 1}, the Weibull distribution reduces to the exponential
distribution.

Draw samples from the distribution:

\begin{Verbatim}[commandchars=\\\{\}]
\PYG{g+gp}{\PYGZgt{}\PYGZgt{}\PYGZgt{} }\PYG{n}{a} \PYG{o}{=} \PYG{l+m+mf}{5.} \PYG{c}{\PYGZsh{} shape}
\PYG{g+gp}{\PYGZgt{}\PYGZgt{}\PYGZgt{} }\PYG{n}{s} \PYG{o}{=} \PYG{n}{np}\PYG{o}{.}\PYG{n}{random}\PYG{o}{.}\PYG{n}{weibull}\PYG{p}{(}\PYG{n}{a}\PYG{p}{,} \PYG{l+m+mi}{1000}\PYG{p}{)}
\end{Verbatim}

Display the histogram of the samples, along with
the probability density function:

\begin{Verbatim}[commandchars=\\\{\}]
\PYG{g+gp}{\PYGZgt{}\PYGZgt{}\PYGZgt{} }\PYG{k+kn}{import} \PYG{n+nn}{matplotlib.pyplot} \PYG{k+kn}{as} \PYG{n+nn}{plt}
\PYG{g+gp}{\PYGZgt{}\PYGZgt{}\PYGZgt{} }\PYG{n}{x} \PYG{o}{=} \PYG{n}{np}\PYG{o}{.}\PYG{n}{arange}\PYG{p}{(}\PYG{l+m+mi}{1}\PYG{p}{,}\PYG{l+m+mf}{100.}\PYG{p}{)}\PYG{o}{/}\PYG{l+m+mf}{50.}
\PYG{g+gp}{\PYGZgt{}\PYGZgt{}\PYGZgt{} }\PYG{k}{def} \PYG{n+nf}{weib}\PYG{p}{(}\PYG{n}{x}\PYG{p}{,}\PYG{n}{n}\PYG{p}{,}\PYG{n}{a}\PYG{p}{)}\PYG{p}{:}
\PYG{g+gp}{... }    \PYG{k}{return} \PYG{p}{(}\PYG{n}{a} \PYG{o}{/} \PYG{n}{n}\PYG{p}{)} \PYG{o}{*} \PYG{p}{(}\PYG{n}{x} \PYG{o}{/} \PYG{n}{n}\PYG{p}{)}\PYG{o}{*}\PYG{o}{*}\PYG{p}{(}\PYG{n}{a} \PYG{o}{\PYGZhy{}} \PYG{l+m+mi}{1}\PYG{p}{)} \PYG{o}{*} \PYG{n}{np}\PYG{o}{.}\PYG{n}{exp}\PYG{p}{(}\PYG{o}{\PYGZhy{}}\PYG{p}{(}\PYG{n}{x} \PYG{o}{/} \PYG{n}{n}\PYG{p}{)}\PYG{o}{*}\PYG{o}{*}\PYG{n}{a}\PYG{p}{)}
\end{Verbatim}

\begin{Verbatim}[commandchars=\\\{\}]
\PYG{g+gp}{\PYGZgt{}\PYGZgt{}\PYGZgt{} }\PYG{n}{count}\PYG{p}{,} \PYG{n}{bins}\PYG{p}{,} \PYG{n}{ignored} \PYG{o}{=} \PYG{n}{plt}\PYG{o}{.}\PYG{n}{hist}\PYG{p}{(}\PYG{n}{np}\PYG{o}{.}\PYG{n}{random}\PYG{o}{.}\PYG{n}{weibull}\PYG{p}{(}\PYG{l+m+mf}{5.}\PYG{p}{,}\PYG{l+m+mi}{1000}\PYG{p}{)}\PYG{p}{)}
\PYG{g+gp}{\PYGZgt{}\PYGZgt{}\PYGZgt{} }\PYG{n}{x} \PYG{o}{=} \PYG{n}{np}\PYG{o}{.}\PYG{n}{arange}\PYG{p}{(}\PYG{l+m+mi}{1}\PYG{p}{,}\PYG{l+m+mf}{100.}\PYG{p}{)}\PYG{o}{/}\PYG{l+m+mf}{50.}
\PYG{g+gp}{\PYGZgt{}\PYGZgt{}\PYGZgt{} }\PYG{n}{scale} \PYG{o}{=} \PYG{n}{count}\PYG{o}{.}\PYG{n}{max}\PYG{p}{(}\PYG{p}{)}\PYG{o}{/}\PYG{n}{weib}\PYG{p}{(}\PYG{n}{x}\PYG{p}{,} \PYG{l+m+mf}{1.}\PYG{p}{,} \PYG{l+m+mf}{5.}\PYG{p}{)}\PYG{o}{.}\PYG{n}{max}\PYG{p}{(}\PYG{p}{)}
\PYG{g+gp}{\PYGZgt{}\PYGZgt{}\PYGZgt{} }\PYG{n}{plt}\PYG{o}{.}\PYG{n}{plot}\PYG{p}{(}\PYG{n}{x}\PYG{p}{,} \PYG{n}{weib}\PYG{p}{(}\PYG{n}{x}\PYG{p}{,} \PYG{l+m+mf}{1.}\PYG{p}{,} \PYG{l+m+mf}{5.}\PYG{p}{)}\PYG{o}{*}\PYG{n}{scale}\PYG{p}{)}
\PYG{g+gp}{\PYGZgt{}\PYGZgt{}\PYGZgt{} }\PYG{n}{plt}\PYG{o}{.}\PYG{n}{show}\PYG{p}{(}\PYG{p}{)}
\end{Verbatim}

\end{fulllineitems}

\index{zeroBeforeStrNum() (in module lib.IO.readfiles)}

\begin{fulllineitems}
\phantomsection\label{lib.IO:lib.IO.readfiles.zeroBeforeStrNum}\pysiglinewithargsret{\code{lib.IO.readfiles.}\bfcode{zeroBeforeStrNum}}{\emph{tmpl}, \emph{tmpL}}{}
\end{fulllineitems}

\index{zipf() (in module lib.IO.readfiles)}

\begin{fulllineitems}
\phantomsection\label{lib.IO:lib.IO.readfiles.zipf}\pysiglinewithargsret{\code{lib.IO.readfiles.}\bfcode{zipf}}{\emph{a}, \emph{size=None}}{}
Draw samples from a Zipf distribution.

Samples are drawn from a Zipf distribution with specified parameter
\emph{a} \textgreater{} 1.

The Zipf distribution (also known as the zeta distribution) is a
continuous probability distribution that satisfies Zipf's law: the
frequency of an item is inversely proportional to its rank in a
frequency table.
\begin{description}
\item[{a}] \leavevmode{[}float \textgreater{} 1{]}
Distribution parameter.

\item[{size}] \leavevmode{[}int or tuple of int, optional{]}
Output shape.  If the given shape is, e.g., \code{(m, n, k)}, then
\code{m * n * k} samples are drawn; a single integer is equivalent in
its result to providing a mono-tuple, i.e., a 1-D array of length
\emph{size} is returned.  The default is None, in which case a single
scalar is returned.

\end{description}
\begin{description}
\item[{samples}] \leavevmode{[}scalar or ndarray{]}
The returned samples are greater than or equal to one.

\end{description}
\begin{description}
\item[{scipy.stats.distributions.zipf}] \leavevmode{[}probability density function,{]}
distribution, or cumulative density function, etc.

\end{description}

The probability density for the Zipf distribution is
\begin{gather}
\begin{split}p(x) = \frac{x^{-a}}{\zeta(a)},\end{split}\notag
\end{gather}
where \(\zeta\) is the Riemann Zeta function.

It is named for the American linguist George Kingsley Zipf, who noted
that the frequency of any word in a sample of a language is inversely
proportional to its rank in the frequency table.

Zipf, G. K., \emph{Selected Studies of the Principle of Relative Frequency
in Language}, Cambridge, MA: Harvard Univ. Press, 1932.

Draw samples from the distribution:

\begin{Verbatim}[commandchars=\\\{\}]
\PYG{g+gp}{\PYGZgt{}\PYGZgt{}\PYGZgt{} }\PYG{n}{a} \PYG{o}{=} \PYG{l+m+mf}{2.} \PYG{c}{\PYGZsh{} parameter}
\PYG{g+gp}{\PYGZgt{}\PYGZgt{}\PYGZgt{} }\PYG{n}{s} \PYG{o}{=} \PYG{n}{np}\PYG{o}{.}\PYG{n}{random}\PYG{o}{.}\PYG{n}{zipf}\PYG{p}{(}\PYG{n}{a}\PYG{p}{,} \PYG{l+m+mi}{1000}\PYG{p}{)}
\end{Verbatim}

Display the histogram of the samples, along with
the probability density function:

\begin{Verbatim}[commandchars=\\\{\}]
\PYG{g+gp}{\PYGZgt{}\PYGZgt{}\PYGZgt{} }\PYG{k+kn}{import} \PYG{n+nn}{matplotlib.pyplot} \PYG{k+kn}{as} \PYG{n+nn}{plt}
\PYG{g+gp}{\PYGZgt{}\PYGZgt{}\PYGZgt{} }\PYG{k+kn}{import} \PYG{n+nn}{scipy.special} \PYG{k+kn}{as} \PYG{n+nn}{sps}
\PYG{g+go}{Truncate s values at 50 so plot is interesting}
\PYG{g+gp}{\PYGZgt{}\PYGZgt{}\PYGZgt{} }\PYG{n}{count}\PYG{p}{,} \PYG{n}{bins}\PYG{p}{,} \PYG{n}{ignored} \PYG{o}{=} \PYG{n}{plt}\PYG{o}{.}\PYG{n}{hist}\PYG{p}{(}\PYG{n}{s}\PYG{p}{[}\PYG{n}{s}\PYG{o}{\PYGZlt{}}\PYG{l+m+mi}{50}\PYG{p}{]}\PYG{p}{,} \PYG{l+m+mi}{50}\PYG{p}{,} \PYG{n}{normed}\PYG{o}{=}\PYG{n+nb+bp}{True}\PYG{p}{)}
\PYG{g+gp}{\PYGZgt{}\PYGZgt{}\PYGZgt{} }\PYG{n}{x} \PYG{o}{=} \PYG{n}{np}\PYG{o}{.}\PYG{n}{arange}\PYG{p}{(}\PYG{l+m+mf}{1.}\PYG{p}{,} \PYG{l+m+mf}{50.}\PYG{p}{)}
\PYG{g+gp}{\PYGZgt{}\PYGZgt{}\PYGZgt{} }\PYG{n}{y} \PYG{o}{=} \PYG{n}{x}\PYG{o}{*}\PYG{o}{*}\PYG{p}{(}\PYG{o}{\PYGZhy{}}\PYG{n}{a}\PYG{p}{)}\PYG{o}{/}\PYG{n}{sps}\PYG{o}{.}\PYG{n}{zetac}\PYG{p}{(}\PYG{n}{a}\PYG{p}{)}
\PYG{g+gp}{\PYGZgt{}\PYGZgt{}\PYGZgt{} }\PYG{n}{plt}\PYG{o}{.}\PYG{n}{plot}\PYG{p}{(}\PYG{n}{x}\PYG{p}{,} \PYG{n}{y}\PYG{o}{/}\PYG{n+nb}{max}\PYG{p}{(}\PYG{n}{y}\PYG{p}{)}\PYG{p}{,} \PYG{n}{linewidth}\PYG{o}{=}\PYG{l+m+mi}{2}\PYG{p}{,} \PYG{n}{color}\PYG{o}{=}\PYG{l+s}{\PYGZsq{}}\PYG{l+s}{r}\PYG{l+s}{\PYGZsq{}}\PYG{p}{)}
\PYG{g+gp}{\PYGZgt{}\PYGZgt{}\PYGZgt{} }\PYG{n}{plt}\PYG{o}{.}\PYG{n}{show}\PYG{p}{(}\PYG{p}{)}
\end{Verbatim}

\end{fulllineitems}



\subsubsection{\texttt{writefiles} Module}
\label{lib.IO:writefiles-module}\label{lib.IO:module-lib.IO.writefiles}\index{lib.IO.writefiles (module)}\index{beta() (in module lib.IO.writefiles)}

\begin{fulllineitems}
\phantomsection\label{lib.IO:lib.IO.writefiles.beta}\pysiglinewithargsret{\code{lib.IO.writefiles.}\bfcode{beta}}{\emph{a}, \emph{b}, \emph{size=None}}{}
The Beta distribution over \code{{[}0, 1{]}}.

The Beta distribution is a special case of the Dirichlet distribution,
and is related to the Gamma distribution.  It has the probability
distribution function
\begin{gather}
\begin{split}f(x; a,b) = \frac{1}{B(\alpha, \beta)} x^{\alpha - 1}
(1 - x)^{\beta - 1},\end{split}\notag
\end{gather}
where the normalisation, B, is the beta function,
\begin{gather}
\begin{split}B(\alpha, \beta) = \int_0^1 t^{\alpha - 1}
(1 - t)^{\beta - 1} dt.\end{split}\notag
\end{gather}
It is often seen in Bayesian inference and order statistics.
\begin{description}
\item[{a}] \leavevmode{[}float{]}
Alpha, non-negative.

\item[{b}] \leavevmode{[}float{]}
Beta, non-negative.

\item[{size}] \leavevmode{[}tuple of ints, optional{]}
The number of samples to draw.  The output is packed according to
the size given.

\end{description}
\begin{description}
\item[{out}] \leavevmode{[}ndarray{]}
Array of the given shape, containing values drawn from a
Beta distribution.

\end{description}

\end{fulllineitems}

\index{binomial() (in module lib.IO.writefiles)}

\begin{fulllineitems}
\phantomsection\label{lib.IO:lib.IO.writefiles.binomial}\pysiglinewithargsret{\code{lib.IO.writefiles.}\bfcode{binomial}}{\emph{n}, \emph{p}, \emph{size=None}}{}
Draw samples from a binomial distribution.

Samples are drawn from a Binomial distribution with specified
parameters, n trials and p probability of success where
n an integer \textgreater{}= 0 and p is in the interval {[}0,1{]}. (n may be
input as a float, but it is truncated to an integer in use)
\begin{description}
\item[{n}] \leavevmode{[}float (but truncated to an integer){]}
parameter, \textgreater{}= 0.

\item[{p}] \leavevmode{[}float{]}
parameter, \textgreater{}= 0 and \textless{}=1.

\item[{size}] \leavevmode{[}\{tuple, int\}{]}
Output shape.  If the given shape is, e.g., \code{(m, n, k)}, then
\code{m * n * k} samples are drawn.

\end{description}
\begin{description}
\item[{samples}] \leavevmode{[}\{ndarray, scalar\}{]}
where the values are all integers in  {[}0, n{]}.

\end{description}
\begin{description}
\item[{scipy.stats.distributions.binom}] \leavevmode{[}probability density function,{]}
distribution or cumulative density function, etc.

\end{description}

The probability density for the Binomial distribution is
\begin{gather}
\begin{split}P(N) = \binom{n}{N}p^N(1-p)^{n-N},\end{split}\notag
\end{gather}
where \(n\) is the number of trials, \(p\) is the probability
of success, and \(N\) is the number of successes.

When estimating the standard error of a proportion in a population by
using a random sample, the normal distribution works well unless the
product p*n \textless{}=5, where p = population proportion estimate, and n =
number of samples, in which case the binomial distribution is used
instead. For example, a sample of 15 people shows 4 who are left
handed, and 11 who are right handed. Then p = 4/15 = 27\%. 0.27*15 = 4,
so the binomial distribution should be used in this case.

Draw samples from the distribution:

\begin{Verbatim}[commandchars=\\\{\}]
\PYG{g+gp}{\PYGZgt{}\PYGZgt{}\PYGZgt{} }\PYG{n}{n}\PYG{p}{,} \PYG{n}{p} \PYG{o}{=} \PYG{l+m+mi}{10}\PYG{p}{,} \PYG{o}{.}\PYG{l+m+mi}{5} \PYG{c}{\PYGZsh{} number of trials, probability of each trial}
\PYG{g+gp}{\PYGZgt{}\PYGZgt{}\PYGZgt{} }\PYG{n}{s} \PYG{o}{=} \PYG{n}{np}\PYG{o}{.}\PYG{n}{random}\PYG{o}{.}\PYG{n}{binomial}\PYG{p}{(}\PYG{n}{n}\PYG{p}{,} \PYG{n}{p}\PYG{p}{,} \PYG{l+m+mi}{1000}\PYG{p}{)}
\PYG{g+go}{\PYGZsh{} result of flipping a coin 10 times, tested 1000 times.}
\end{Verbatim}

A real world example. A company drills 9 wild-cat oil exploration
wells, each with an estimated probability of success of 0.1. All nine
wells fail. What is the probability of that happening?

Let's do 20,000 trials of the model, and count the number that
generate zero positive results.

\begin{Verbatim}[commandchars=\\\{\}]
\PYG{g+gp}{\PYGZgt{}\PYGZgt{}\PYGZgt{} }\PYG{n+nb}{sum}\PYG{p}{(}\PYG{n}{np}\PYG{o}{.}\PYG{n}{random}\PYG{o}{.}\PYG{n}{binomial}\PYG{p}{(}\PYG{l+m+mi}{9}\PYG{p}{,}\PYG{l+m+mf}{0.1}\PYG{p}{,}\PYG{l+m+mi}{20000}\PYG{p}{)}\PYG{o}{==}\PYG{l+m+mi}{0}\PYG{p}{)}\PYG{o}{/}\PYG{l+m+mf}{20000.}
\PYG{g+go}{answer = 0.38885, or 38\PYGZpc{}.}
\end{Verbatim}

\end{fulllineitems}

\index{chisquare() (in module lib.IO.writefiles)}

\begin{fulllineitems}
\phantomsection\label{lib.IO:lib.IO.writefiles.chisquare}\pysiglinewithargsret{\code{lib.IO.writefiles.}\bfcode{chisquare}}{\emph{df}, \emph{size=None}}{}
Draw samples from a chi-square distribution.

When \emph{df} independent random variables, each with standard normal
distributions (mean 0, variance 1), are squared and summed, the
resulting distribution is chi-square (see Notes).  This distribution
is often used in hypothesis testing.
\begin{description}
\item[{df}] \leavevmode{[}int{]}
Number of degrees of freedom.

\item[{size}] \leavevmode{[}tuple of ints, int, optional{]}
Size of the returned array.  By default, a scalar is
returned.

\end{description}
\begin{description}
\item[{output}] \leavevmode{[}ndarray{]}
Samples drawn from the distribution, packed in a \emph{size}-shaped
array.

\end{description}
\begin{description}
\item[{ValueError}] \leavevmode
When \emph{df} \textless{}= 0 or when an inappropriate \emph{size} (e.g. \code{size=-1})
is given.

\end{description}

The variable obtained by summing the squares of \emph{df} independent,
standard normally distributed random variables:
\begin{gather}
\begin{split}Q = \sum_{i=0}^{\mathtt{df}} X^2_i\end{split}\notag
\end{gather}
is chi-square distributed, denoted
\begin{gather}
\begin{split}Q \sim \chi^2_k.\end{split}\notag
\end{gather}
The probability density function of the chi-squared distribution is
\begin{gather}
\begin{split}p(x) = \frac{(1/2)^{k/2}}{\Gamma(k/2)}
x^{k/2 - 1} e^{-x/2},\end{split}\notag
\end{gather}
where \(\Gamma\) is the gamma function,
\begin{gather}
\begin{split}\Gamma(x) = \int_0^{-\infty} t^{x - 1} e^{-t} dt.\end{split}\notag
\end{gather}
\href{http://www.itl.nist.gov/div898/handbook/eda/section3/eda3666.htm}{NIST/SEMATECH e-Handbook of Statistical Methods}

\begin{Verbatim}[commandchars=\\\{\}]
\PYG{g+gp}{\PYGZgt{}\PYGZgt{}\PYGZgt{} }\PYG{n}{np}\PYG{o}{.}\PYG{n}{random}\PYG{o}{.}\PYG{n}{chisquare}\PYG{p}{(}\PYG{l+m+mi}{2}\PYG{p}{,}\PYG{l+m+mi}{4}\PYG{p}{)}
\PYG{g+go}{array([ 1.89920014,  9.00867716,  3.13710533,  5.62318272])}
\end{Verbatim}

\end{fulllineitems}

\index{exponential() (in module lib.IO.writefiles)}

\begin{fulllineitems}
\phantomsection\label{lib.IO:lib.IO.writefiles.exponential}\pysiglinewithargsret{\code{lib.IO.writefiles.}\bfcode{exponential}}{\emph{scale=1.0}, \emph{size=None}}{}
Exponential distribution.

Its probability density function is
\begin{gather}
\begin{split}f(x; \frac{1}{\beta}) = \frac{1}{\beta} \exp(-\frac{x}{\beta}),\end{split}\notag
\end{gather}
for \code{x \textgreater{} 0} and 0 elsewhere. \(\beta\) is the scale parameter,
which is the inverse of the rate parameter \(\lambda = 1/\beta\).
The rate parameter is an alternative, widely used parameterization
of the exponential distribution {\color{red}\bfseries{}{[}3{]}\_}.

The exponential distribution is a continuous analogue of the
geometric distribution.  It describes many common situations, such as
the size of raindrops measured over many rainstorms {\color{red}\bfseries{}{[}1{]}\_}, or the time
between page requests to Wikipedia {\color{red}\bfseries{}{[}2{]}\_}.
\begin{description}
\item[{scale}] \leavevmode{[}float{]}
The scale parameter, \(\beta = 1/\lambda\).

\item[{size}] \leavevmode{[}tuple of ints{]}
Number of samples to draw.  The output is shaped
according to \emph{size}.

\end{description}

\end{fulllineitems}

\index{f() (in module lib.IO.writefiles)}

\begin{fulllineitems}
\phantomsection\label{lib.IO:lib.IO.writefiles.f}\pysiglinewithargsret{\code{lib.IO.writefiles.}\bfcode{f}}{\emph{dfnum}, \emph{dfden}, \emph{size=None}}{}
Draw samples from a F distribution.

Samples are drawn from an F distribution with specified parameters,
\emph{dfnum} (degrees of freedom in numerator) and \emph{dfden} (degrees of freedom
in denominator), where both parameters should be greater than zero.

The random variate of the F distribution (also known as the
Fisher distribution) is a continuous probability distribution
that arises in ANOVA tests, and is the ratio of two chi-square
variates.
\begin{description}
\item[{dfnum}] \leavevmode{[}float{]}
Degrees of freedom in numerator. Should be greater than zero.

\item[{dfden}] \leavevmode{[}float{]}
Degrees of freedom in denominator. Should be greater than zero.

\item[{size}] \leavevmode{[}\{tuple, int\}, optional{]}
Output shape.  If the given shape is, e.g., \code{(m, n, k)},
then \code{m * n * k} samples are drawn. By default only one sample
is returned.

\end{description}
\begin{description}
\item[{samples}] \leavevmode{[}\{ndarray, scalar\}{]}
Samples from the Fisher distribution.

\end{description}
\begin{description}
\item[{scipy.stats.distributions.f}] \leavevmode{[}probability density function,{]}
distribution or cumulative density function, etc.

\end{description}

The F statistic is used to compare in-group variances to between-group
variances. Calculating the distribution depends on the sampling, and
so it is a function of the respective degrees of freedom in the
problem.  The variable \emph{dfnum} is the number of samples minus one, the
between-groups degrees of freedom, while \emph{dfden} is the within-groups
degrees of freedom, the sum of the number of samples in each group
minus the number of groups.

An example from Glantz{[}1{]}, pp 47-40.
Two groups, children of diabetics (25 people) and children from people
without diabetes (25 controls). Fasting blood glucose was measured,
case group had a mean value of 86.1, controls had a mean value of
82.2. Standard deviations were 2.09 and 2.49 respectively. Are these
data consistent with the null hypothesis that the parents diabetic
status does not affect their children's blood glucose levels?
Calculating the F statistic from the data gives a value of 36.01.

Draw samples from the distribution:

\begin{Verbatim}[commandchars=\\\{\}]
\PYG{g+gp}{\PYGZgt{}\PYGZgt{}\PYGZgt{} }\PYG{n}{dfnum} \PYG{o}{=} \PYG{l+m+mf}{1.} \PYG{c}{\PYGZsh{} between group degrees of freedom}
\PYG{g+gp}{\PYGZgt{}\PYGZgt{}\PYGZgt{} }\PYG{n}{dfden} \PYG{o}{=} \PYG{l+m+mf}{48.} \PYG{c}{\PYGZsh{} within groups degrees of freedom}
\PYG{g+gp}{\PYGZgt{}\PYGZgt{}\PYGZgt{} }\PYG{n}{s} \PYG{o}{=} \PYG{n}{np}\PYG{o}{.}\PYG{n}{random}\PYG{o}{.}\PYG{n}{f}\PYG{p}{(}\PYG{n}{dfnum}\PYG{p}{,} \PYG{n}{dfden}\PYG{p}{,} \PYG{l+m+mi}{1000}\PYG{p}{)}
\end{Verbatim}

The lower bound for the top 1\% of the samples is :

\begin{Verbatim}[commandchars=\\\{\}]
\PYG{g+gp}{\PYGZgt{}\PYGZgt{}\PYGZgt{} }\PYG{n}{sort}\PYG{p}{(}\PYG{n}{s}\PYG{p}{)}\PYG{p}{[}\PYG{o}{\PYGZhy{}}\PYG{l+m+mi}{10}\PYG{p}{]}
\PYG{g+go}{7.61988120985}
\end{Verbatim}

So there is about a 1\% chance that the F statistic will exceed 7.62,
the measured value is 36, so the null hypothesis is rejected at the 1\%
level.

\end{fulllineitems}

\index{gamma() (in module lib.IO.writefiles)}

\begin{fulllineitems}
\phantomsection\label{lib.IO:lib.IO.writefiles.gamma}\pysiglinewithargsret{\code{lib.IO.writefiles.}\bfcode{gamma}}{\emph{shape}, \emph{scale=1.0}, \emph{size=None}}{}
Draw samples from a Gamma distribution.

Samples are drawn from a Gamma distribution with specified parameters,
\emph{shape} (sometimes designated ``k'') and \emph{scale} (sometimes designated
``theta''), where both parameters are \textgreater{} 0.
\begin{description}
\item[{shape}] \leavevmode{[}scalar \textgreater{} 0{]}
The shape of the gamma distribution.

\item[{scale}] \leavevmode{[}scalar \textgreater{} 0, optional{]}
The scale of the gamma distribution.  Default is equal to 1.

\item[{size}] \leavevmode{[}shape\_tuple, optional{]}
Output shape.  If the given shape is, e.g., \code{(m, n, k)}, then
\code{m * n * k} samples are drawn.

\end{description}
\begin{description}
\item[{out}] \leavevmode{[}ndarray, float{]}
Returns one sample unless \emph{size} parameter is specified.

\end{description}
\begin{description}
\item[{scipy.stats.distributions.gamma}] \leavevmode{[}probability density function,{]}
distribution or cumulative density function, etc.

\end{description}

The probability density for the Gamma distribution is
\begin{gather}
\begin{split}p(x) = x^{k-1}\frac{e^{-x/\theta}}{\theta^k\Gamma(k)},\end{split}\notag
\end{gather}
where \(k\) is the shape and \(\theta\) the scale,
and \(\Gamma\) is the Gamma function.

The Gamma distribution is often used to model the times to failure of
electronic components, and arises naturally in processes for which the
waiting times between Poisson distributed events are relevant.

Draw samples from the distribution:

\begin{Verbatim}[commandchars=\\\{\}]
\PYG{g+gp}{\PYGZgt{}\PYGZgt{}\PYGZgt{} }\PYG{n}{shape}\PYG{p}{,} \PYG{n}{scale} \PYG{o}{=} \PYG{l+m+mf}{2.}\PYG{p}{,} \PYG{l+m+mf}{2.} \PYG{c}{\PYGZsh{} mean and dispersion}
\PYG{g+gp}{\PYGZgt{}\PYGZgt{}\PYGZgt{} }\PYG{n}{s} \PYG{o}{=} \PYG{n}{np}\PYG{o}{.}\PYG{n}{random}\PYG{o}{.}\PYG{n}{gamma}\PYG{p}{(}\PYG{n}{shape}\PYG{p}{,} \PYG{n}{scale}\PYG{p}{,} \PYG{l+m+mi}{1000}\PYG{p}{)}
\end{Verbatim}

Display the histogram of the samples, along with
the probability density function:

\begin{Verbatim}[commandchars=\\\{\}]
\PYG{g+gp}{\PYGZgt{}\PYGZgt{}\PYGZgt{} }\PYG{k+kn}{import} \PYG{n+nn}{matplotlib.pyplot} \PYG{k+kn}{as} \PYG{n+nn}{plt}
\PYG{g+gp}{\PYGZgt{}\PYGZgt{}\PYGZgt{} }\PYG{k+kn}{import} \PYG{n+nn}{scipy.special} \PYG{k+kn}{as} \PYG{n+nn}{sps}
\PYG{g+gp}{\PYGZgt{}\PYGZgt{}\PYGZgt{} }\PYG{n}{count}\PYG{p}{,} \PYG{n}{bins}\PYG{p}{,} \PYG{n}{ignored} \PYG{o}{=} \PYG{n}{plt}\PYG{o}{.}\PYG{n}{hist}\PYG{p}{(}\PYG{n}{s}\PYG{p}{,} \PYG{l+m+mi}{50}\PYG{p}{,} \PYG{n}{normed}\PYG{o}{=}\PYG{n+nb+bp}{True}\PYG{p}{)}
\PYG{g+gp}{\PYGZgt{}\PYGZgt{}\PYGZgt{} }\PYG{n}{y} \PYG{o}{=} \PYG{n}{bins}\PYG{o}{*}\PYG{o}{*}\PYG{p}{(}\PYG{n}{shape}\PYG{o}{\PYGZhy{}}\PYG{l+m+mi}{1}\PYG{p}{)}\PYG{o}{*}\PYG{p}{(}\PYG{n}{np}\PYG{o}{.}\PYG{n}{exp}\PYG{p}{(}\PYG{o}{\PYGZhy{}}\PYG{n}{bins}\PYG{o}{/}\PYG{n}{scale}\PYG{p}{)} \PYG{o}{/}
\PYG{g+gp}{... }                     \PYG{p}{(}\PYG{n}{sps}\PYG{o}{.}\PYG{n}{gamma}\PYG{p}{(}\PYG{n}{shape}\PYG{p}{)}\PYG{o}{*}\PYG{n}{scale}\PYG{o}{*}\PYG{o}{*}\PYG{n}{shape}\PYG{p}{)}\PYG{p}{)}
\PYG{g+gp}{\PYGZgt{}\PYGZgt{}\PYGZgt{} }\PYG{n}{plt}\PYG{o}{.}\PYG{n}{plot}\PYG{p}{(}\PYG{n}{bins}\PYG{p}{,} \PYG{n}{y}\PYG{p}{,} \PYG{n}{linewidth}\PYG{o}{=}\PYG{l+m+mi}{2}\PYG{p}{,} \PYG{n}{color}\PYG{o}{=}\PYG{l+s}{\PYGZsq{}}\PYG{l+s}{r}\PYG{l+s}{\PYGZsq{}}\PYG{p}{)}
\PYG{g+gp}{\PYGZgt{}\PYGZgt{}\PYGZgt{} }\PYG{n}{plt}\PYG{o}{.}\PYG{n}{show}\PYG{p}{(}\PYG{p}{)}
\end{Verbatim}

\end{fulllineitems}

\index{geometric() (in module lib.IO.writefiles)}

\begin{fulllineitems}
\phantomsection\label{lib.IO:lib.IO.writefiles.geometric}\pysiglinewithargsret{\code{lib.IO.writefiles.}\bfcode{geometric}}{\emph{p}, \emph{size=None}}{}
Draw samples from the geometric distribution.

Bernoulli trials are experiments with one of two outcomes:
success or failure (an example of such an experiment is flipping
a coin).  The geometric distribution models the number of trials
that must be run in order to achieve success.  It is therefore
supported on the positive integers, \code{k = 1, 2, ...}.

The probability mass function of the geometric distribution is
\begin{gather}
\begin{split}f(k) = (1 - p)^{k - 1} p\end{split}\notag
\end{gather}
where \emph{p} is the probability of success of an individual trial.
\begin{description}
\item[{p}] \leavevmode{[}float{]}
The probability of success of an individual trial.

\item[{size}] \leavevmode{[}tuple of ints{]}
Number of values to draw from the distribution.  The output
is shaped according to \emph{size}.

\end{description}
\begin{description}
\item[{out}] \leavevmode{[}ndarray{]}
Samples from the geometric distribution, shaped according to
\emph{size}.

\end{description}

Draw ten thousand values from the geometric distribution,
with the probability of an individual success equal to 0.35:

\begin{Verbatim}[commandchars=\\\{\}]
\PYG{g+gp}{\PYGZgt{}\PYGZgt{}\PYGZgt{} }\PYG{n}{z} \PYG{o}{=} \PYG{n}{np}\PYG{o}{.}\PYG{n}{random}\PYG{o}{.}\PYG{n}{geometric}\PYG{p}{(}\PYG{n}{p}\PYG{o}{=}\PYG{l+m+mf}{0.35}\PYG{p}{,} \PYG{n}{size}\PYG{o}{=}\PYG{l+m+mi}{10000}\PYG{p}{)}
\end{Verbatim}

How many trials succeeded after a single run?

\begin{Verbatim}[commandchars=\\\{\}]
\PYG{g+gp}{\PYGZgt{}\PYGZgt{}\PYGZgt{} }\PYG{p}{(}\PYG{n}{z} \PYG{o}{==} \PYG{l+m+mi}{1}\PYG{p}{)}\PYG{o}{.}\PYG{n}{sum}\PYG{p}{(}\PYG{p}{)} \PYG{o}{/} \PYG{l+m+mf}{10000.}
\PYG{g+go}{0.34889999999999999 \PYGZsh{}random}
\end{Verbatim}

\end{fulllineitems}

\index{get\_state() (in module lib.IO.writefiles)}

\begin{fulllineitems}
\phantomsection\label{lib.IO:lib.IO.writefiles.get_state}\pysiglinewithargsret{\code{lib.IO.writefiles.}\bfcode{get\_state}}{}{}
Return a tuple representing the internal state of the generator.

For more details, see \emph{set\_state}.
\begin{description}
\item[{out}] \leavevmode{[}tuple(str, ndarray of 624 uints, int, int, float){]}
The returned tuple has the following items:
\begin{enumerate}
\item {} 
the string `MT19937'.

\item {} 
a 1-D array of 624 unsigned integer keys.

\item {} 
an integer \code{pos}.

\item {} 
an integer \code{has\_gauss}.

\item {} 
a float \code{cached\_gaussian}.

\end{enumerate}

\end{description}

set\_state

\emph{set\_state} and \emph{get\_state} are not needed to work with any of the
random distributions in NumPy. If the internal state is manually altered,
the user should know exactly what he/she is doing.

\end{fulllineitems}

\index{gumbel() (in module lib.IO.writefiles)}

\begin{fulllineitems}
\phantomsection\label{lib.IO:lib.IO.writefiles.gumbel}\pysiglinewithargsret{\code{lib.IO.writefiles.}\bfcode{gumbel}}{\emph{loc=0.0}, \emph{scale=1.0}, \emph{size=None}}{}
Gumbel distribution.

Draw samples from a Gumbel distribution with specified location and scale.
For more information on the Gumbel distribution, see Notes and References
below.
\begin{description}
\item[{loc}] \leavevmode{[}float{]}
The location of the mode of the distribution.

\item[{scale}] \leavevmode{[}float{]}
The scale parameter of the distribution.

\item[{size}] \leavevmode{[}tuple of ints{]}
Output shape.  If the given shape is, e.g., \code{(m, n, k)}, then
\code{m * n * k} samples are drawn.

\end{description}
\begin{description}
\item[{out}] \leavevmode{[}ndarray{]}
The samples

\end{description}

scipy.stats.gumbel\_l
scipy.stats.gumbel\_r
scipy.stats.genextreme
\begin{quote}

probability density function, distribution, or cumulative density
function, etc. for each of the above
\end{quote}

weibull

The Gumbel (or Smallest Extreme Value (SEV) or the Smallest Extreme Value
Type I) distribution is one of a class of Generalized Extreme Value (GEV)
distributions used in modeling extreme value problems.  The Gumbel is a
special case of the Extreme Value Type I distribution for maximums from
distributions with ``exponential-like'' tails.

The probability density for the Gumbel distribution is
\begin{gather}
\begin{split}p(x) = \frac{e^{-(x - \mu)/ \beta}}{\beta} e^{ -e^{-(x - \mu)/
\beta}},\end{split}\notag
\end{gather}
where \(\mu\) is the mode, a location parameter, and \(\beta\) is
the scale parameter.

The Gumbel (named for German mathematician Emil Julius Gumbel) was used
very early in the hydrology literature, for modeling the occurrence of
flood events. It is also used for modeling maximum wind speed and rainfall
rates.  It is a ``fat-tailed'' distribution - the probability of an event in
the tail of the distribution is larger than if one used a Gaussian, hence
the surprisingly frequent occurrence of 100-year floods. Floods were
initially modeled as a Gaussian process, which underestimated the frequency
of extreme events.

It is one of a class of extreme value distributions, the Generalized
Extreme Value (GEV) distributions, which also includes the Weibull and
Frechet.

The function has a mean of \(\mu + 0.57721\beta\) and a variance of
\(\frac{\pi^2}{6}\beta^2\).

Gumbel, E. J., \emph{Statistics of Extremes}, New York: Columbia University
Press, 1958.

Reiss, R.-D. and Thomas, M., \emph{Statistical Analysis of Extreme Values from
Insurance, Finance, Hydrology and Other Fields}, Basel: Birkhauser Verlag,
2001.

Draw samples from the distribution:

\begin{Verbatim}[commandchars=\\\{\}]
\PYG{g+gp}{\PYGZgt{}\PYGZgt{}\PYGZgt{} }\PYG{n}{mu}\PYG{p}{,} \PYG{n}{beta} \PYG{o}{=} \PYG{l+m+mi}{0}\PYG{p}{,} \PYG{l+m+mf}{0.1} \PYG{c}{\PYGZsh{} location and scale}
\PYG{g+gp}{\PYGZgt{}\PYGZgt{}\PYGZgt{} }\PYG{n}{s} \PYG{o}{=} \PYG{n}{np}\PYG{o}{.}\PYG{n}{random}\PYG{o}{.}\PYG{n}{gumbel}\PYG{p}{(}\PYG{n}{mu}\PYG{p}{,} \PYG{n}{beta}\PYG{p}{,} \PYG{l+m+mi}{1000}\PYG{p}{)}
\end{Verbatim}

Display the histogram of the samples, along with
the probability density function:

\begin{Verbatim}[commandchars=\\\{\}]
\PYG{g+gp}{\PYGZgt{}\PYGZgt{}\PYGZgt{} }\PYG{k+kn}{import} \PYG{n+nn}{matplotlib.pyplot} \PYG{k+kn}{as} \PYG{n+nn}{plt}
\PYG{g+gp}{\PYGZgt{}\PYGZgt{}\PYGZgt{} }\PYG{n}{count}\PYG{p}{,} \PYG{n}{bins}\PYG{p}{,} \PYG{n}{ignored} \PYG{o}{=} \PYG{n}{plt}\PYG{o}{.}\PYG{n}{hist}\PYG{p}{(}\PYG{n}{s}\PYG{p}{,} \PYG{l+m+mi}{30}\PYG{p}{,} \PYG{n}{normed}\PYG{o}{=}\PYG{n+nb+bp}{True}\PYG{p}{)}
\PYG{g+gp}{\PYGZgt{}\PYGZgt{}\PYGZgt{} }\PYG{n}{plt}\PYG{o}{.}\PYG{n}{plot}\PYG{p}{(}\PYG{n}{bins}\PYG{p}{,} \PYG{p}{(}\PYG{l+m+mi}{1}\PYG{o}{/}\PYG{n}{beta}\PYG{p}{)}\PYG{o}{*}\PYG{n}{np}\PYG{o}{.}\PYG{n}{exp}\PYG{p}{(}\PYG{o}{\PYGZhy{}}\PYG{p}{(}\PYG{n}{bins} \PYG{o}{\PYGZhy{}} \PYG{n}{mu}\PYG{p}{)}\PYG{o}{/}\PYG{n}{beta}\PYG{p}{)}
\PYG{g+gp}{... }         \PYG{o}{*} \PYG{n}{np}\PYG{o}{.}\PYG{n}{exp}\PYG{p}{(} \PYG{o}{\PYGZhy{}}\PYG{n}{np}\PYG{o}{.}\PYG{n}{exp}\PYG{p}{(} \PYG{o}{\PYGZhy{}}\PYG{p}{(}\PYG{n}{bins} \PYG{o}{\PYGZhy{}} \PYG{n}{mu}\PYG{p}{)} \PYG{o}{/}\PYG{n}{beta}\PYG{p}{)} \PYG{p}{)}\PYG{p}{,}
\PYG{g+gp}{... }         \PYG{n}{linewidth}\PYG{o}{=}\PYG{l+m+mi}{2}\PYG{p}{,} \PYG{n}{color}\PYG{o}{=}\PYG{l+s}{\PYGZsq{}}\PYG{l+s}{r}\PYG{l+s}{\PYGZsq{}}\PYG{p}{)}
\PYG{g+gp}{\PYGZgt{}\PYGZgt{}\PYGZgt{} }\PYG{n}{plt}\PYG{o}{.}\PYG{n}{show}\PYG{p}{(}\PYG{p}{)}
\end{Verbatim}

Show how an extreme value distribution can arise from a Gaussian process
and compare to a Gaussian:

\begin{Verbatim}[commandchars=\\\{\}]
\PYG{g+gp}{\PYGZgt{}\PYGZgt{}\PYGZgt{} }\PYG{n}{means} \PYG{o}{=} \PYG{p}{[}\PYG{p}{]}
\PYG{g+gp}{\PYGZgt{}\PYGZgt{}\PYGZgt{} }\PYG{n}{maxima} \PYG{o}{=} \PYG{p}{[}\PYG{p}{]}
\PYG{g+gp}{\PYGZgt{}\PYGZgt{}\PYGZgt{} }\PYG{k}{for} \PYG{n}{i} \PYG{o+ow}{in} \PYG{n+nb}{range}\PYG{p}{(}\PYG{l+m+mi}{0}\PYG{p}{,}\PYG{l+m+mi}{1000}\PYG{p}{)} \PYG{p}{:}
\PYG{g+gp}{... }   \PYG{n}{a} \PYG{o}{=} \PYG{n}{np}\PYG{o}{.}\PYG{n}{random}\PYG{o}{.}\PYG{n}{normal}\PYG{p}{(}\PYG{n}{mu}\PYG{p}{,} \PYG{n}{beta}\PYG{p}{,} \PYG{l+m+mi}{1000}\PYG{p}{)}
\PYG{g+gp}{... }   \PYG{n}{means}\PYG{o}{.}\PYG{n}{append}\PYG{p}{(}\PYG{n}{a}\PYG{o}{.}\PYG{n}{mean}\PYG{p}{(}\PYG{p}{)}\PYG{p}{)}
\PYG{g+gp}{... }   \PYG{n}{maxima}\PYG{o}{.}\PYG{n}{append}\PYG{p}{(}\PYG{n}{a}\PYG{o}{.}\PYG{n}{max}\PYG{p}{(}\PYG{p}{)}\PYG{p}{)}
\PYG{g+gp}{\PYGZgt{}\PYGZgt{}\PYGZgt{} }\PYG{n}{count}\PYG{p}{,} \PYG{n}{bins}\PYG{p}{,} \PYG{n}{ignored} \PYG{o}{=} \PYG{n}{plt}\PYG{o}{.}\PYG{n}{hist}\PYG{p}{(}\PYG{n}{maxima}\PYG{p}{,} \PYG{l+m+mi}{30}\PYG{p}{,} \PYG{n}{normed}\PYG{o}{=}\PYG{n+nb+bp}{True}\PYG{p}{)}
\PYG{g+gp}{\PYGZgt{}\PYGZgt{}\PYGZgt{} }\PYG{n}{beta} \PYG{o}{=} \PYG{n}{np}\PYG{o}{.}\PYG{n}{std}\PYG{p}{(}\PYG{n}{maxima}\PYG{p}{)}\PYG{o}{*}\PYG{n}{np}\PYG{o}{.}\PYG{n}{pi}\PYG{o}{/}\PYG{n}{np}\PYG{o}{.}\PYG{n}{sqrt}\PYG{p}{(}\PYG{l+m+mi}{6}\PYG{p}{)}
\PYG{g+gp}{\PYGZgt{}\PYGZgt{}\PYGZgt{} }\PYG{n}{mu} \PYG{o}{=} \PYG{n}{np}\PYG{o}{.}\PYG{n}{mean}\PYG{p}{(}\PYG{n}{maxima}\PYG{p}{)} \PYG{o}{\PYGZhy{}} \PYG{l+m+mf}{0.57721}\PYG{o}{*}\PYG{n}{beta}
\PYG{g+gp}{\PYGZgt{}\PYGZgt{}\PYGZgt{} }\PYG{n}{plt}\PYG{o}{.}\PYG{n}{plot}\PYG{p}{(}\PYG{n}{bins}\PYG{p}{,} \PYG{p}{(}\PYG{l+m+mi}{1}\PYG{o}{/}\PYG{n}{beta}\PYG{p}{)}\PYG{o}{*}\PYG{n}{np}\PYG{o}{.}\PYG{n}{exp}\PYG{p}{(}\PYG{o}{\PYGZhy{}}\PYG{p}{(}\PYG{n}{bins} \PYG{o}{\PYGZhy{}} \PYG{n}{mu}\PYG{p}{)}\PYG{o}{/}\PYG{n}{beta}\PYG{p}{)}
\PYG{g+gp}{... }         \PYG{o}{*} \PYG{n}{np}\PYG{o}{.}\PYG{n}{exp}\PYG{p}{(}\PYG{o}{\PYGZhy{}}\PYG{n}{np}\PYG{o}{.}\PYG{n}{exp}\PYG{p}{(}\PYG{o}{\PYGZhy{}}\PYG{p}{(}\PYG{n}{bins} \PYG{o}{\PYGZhy{}} \PYG{n}{mu}\PYG{p}{)}\PYG{o}{/}\PYG{n}{beta}\PYG{p}{)}\PYG{p}{)}\PYG{p}{,}
\PYG{g+gp}{... }         \PYG{n}{linewidth}\PYG{o}{=}\PYG{l+m+mi}{2}\PYG{p}{,} \PYG{n}{color}\PYG{o}{=}\PYG{l+s}{\PYGZsq{}}\PYG{l+s}{r}\PYG{l+s}{\PYGZsq{}}\PYG{p}{)}
\PYG{g+gp}{\PYGZgt{}\PYGZgt{}\PYGZgt{} }\PYG{n}{plt}\PYG{o}{.}\PYG{n}{plot}\PYG{p}{(}\PYG{n}{bins}\PYG{p}{,} \PYG{l+m+mi}{1}\PYG{o}{/}\PYG{p}{(}\PYG{n}{beta} \PYG{o}{*} \PYG{n}{np}\PYG{o}{.}\PYG{n}{sqrt}\PYG{p}{(}\PYG{l+m+mi}{2} \PYG{o}{*} \PYG{n}{np}\PYG{o}{.}\PYG{n}{pi}\PYG{p}{)}\PYG{p}{)}
\PYG{g+gp}{... }         \PYG{o}{*} \PYG{n}{np}\PYG{o}{.}\PYG{n}{exp}\PYG{p}{(}\PYG{o}{\PYGZhy{}}\PYG{p}{(}\PYG{n}{bins} \PYG{o}{\PYGZhy{}} \PYG{n}{mu}\PYG{p}{)}\PYG{o}{*}\PYG{o}{*}\PYG{l+m+mi}{2} \PYG{o}{/} \PYG{p}{(}\PYG{l+m+mi}{2} \PYG{o}{*} \PYG{n}{beta}\PYG{o}{*}\PYG{o}{*}\PYG{l+m+mi}{2}\PYG{p}{)}\PYG{p}{)}\PYG{p}{,}
\PYG{g+gp}{... }         \PYG{n}{linewidth}\PYG{o}{=}\PYG{l+m+mi}{2}\PYG{p}{,} \PYG{n}{color}\PYG{o}{=}\PYG{l+s}{\PYGZsq{}}\PYG{l+s}{g}\PYG{l+s}{\PYGZsq{}}\PYG{p}{)}
\PYG{g+gp}{\PYGZgt{}\PYGZgt{}\PYGZgt{} }\PYG{n}{plt}\PYG{o}{.}\PYG{n}{show}\PYG{p}{(}\PYG{p}{)}
\end{Verbatim}

\end{fulllineitems}

\index{hypergeometric() (in module lib.IO.writefiles)}

\begin{fulllineitems}
\phantomsection\label{lib.IO:lib.IO.writefiles.hypergeometric}\pysiglinewithargsret{\code{lib.IO.writefiles.}\bfcode{hypergeometric}}{\emph{ngood}, \emph{nbad}, \emph{nsample}, \emph{size=None}}{}
Draw samples from a Hypergeometric distribution.

Samples are drawn from a Hypergeometric distribution with specified
parameters, ngood (ways to make a good selection), nbad (ways to make
a bad selection), and nsample = number of items sampled, which is less
than or equal to the sum ngood + nbad.
\begin{description}
\item[{ngood}] \leavevmode{[}int or array\_like{]}
Number of ways to make a good selection.  Must be nonnegative.

\item[{nbad}] \leavevmode{[}int or array\_like{]}
Number of ways to make a bad selection.  Must be nonnegative.

\item[{nsample}] \leavevmode{[}int or array\_like{]}
Number of items sampled.  Must be at least 1 and at most
\code{ngood + nbad}.

\item[{size}] \leavevmode{[}int or tuple of int{]}
Output shape.  If the given shape is, e.g., \code{(m, n, k)}, then
\code{m * n * k} samples are drawn.

\end{description}
\begin{description}
\item[{samples}] \leavevmode{[}ndarray or scalar{]}
The values are all integers in  {[}0, n{]}.

\end{description}
\begin{description}
\item[{scipy.stats.distributions.hypergeom}] \leavevmode{[}probability density function,{]}
distribution or cumulative density function, etc.

\end{description}

The probability density for the Hypergeometric distribution is
\begin{gather}
\begin{split}P(x) = \frac{\binom{m}{n}\binom{N-m}{n-x}}{\binom{N}{n}},\end{split}\notag
\end{gather}
where \(0 \le x \le m\) and \(n+m-N \le x \le n\)

for P(x) the probability of x successes, n = ngood, m = nbad, and
N = number of samples.

Consider an urn with black and white marbles in it, ngood of them
black and nbad are white. If you draw nsample balls without
replacement, then the Hypergeometric distribution describes the
distribution of black balls in the drawn sample.

Note that this distribution is very similar to the Binomial
distribution, except that in this case, samples are drawn without
replacement, whereas in the Binomial case samples are drawn with
replacement (or the sample space is infinite). As the sample space
becomes large, this distribution approaches the Binomial.

Draw samples from the distribution:

\begin{Verbatim}[commandchars=\\\{\}]
\PYG{g+gp}{\PYGZgt{}\PYGZgt{}\PYGZgt{} }\PYG{n}{ngood}\PYG{p}{,} \PYG{n}{nbad}\PYG{p}{,} \PYG{n}{nsamp} \PYG{o}{=} \PYG{l+m+mi}{100}\PYG{p}{,} \PYG{l+m+mi}{2}\PYG{p}{,} \PYG{l+m+mi}{10}
\PYG{g+go}{\PYGZsh{} number of good, number of bad, and number of samples}
\PYG{g+gp}{\PYGZgt{}\PYGZgt{}\PYGZgt{} }\PYG{n}{s} \PYG{o}{=} \PYG{n}{np}\PYG{o}{.}\PYG{n}{random}\PYG{o}{.}\PYG{n}{hypergeometric}\PYG{p}{(}\PYG{n}{ngood}\PYG{p}{,} \PYG{n}{nbad}\PYG{p}{,} \PYG{n}{nsamp}\PYG{p}{,} \PYG{l+m+mi}{1000}\PYG{p}{)}
\PYG{g+gp}{\PYGZgt{}\PYGZgt{}\PYGZgt{} }\PYG{n}{hist}\PYG{p}{(}\PYG{n}{s}\PYG{p}{)}
\PYG{g+go}{\PYGZsh{}   note that it is very unlikely to grab both bad items}
\end{Verbatim}

Suppose you have an urn with 15 white and 15 black marbles.
If you pull 15 marbles at random, how likely is it that
12 or more of them are one color?

\begin{Verbatim}[commandchars=\\\{\}]
\PYG{g+gp}{\PYGZgt{}\PYGZgt{}\PYGZgt{} }\PYG{n}{s} \PYG{o}{=} \PYG{n}{np}\PYG{o}{.}\PYG{n}{random}\PYG{o}{.}\PYG{n}{hypergeometric}\PYG{p}{(}\PYG{l+m+mi}{15}\PYG{p}{,} \PYG{l+m+mi}{15}\PYG{p}{,} \PYG{l+m+mi}{15}\PYG{p}{,} \PYG{l+m+mi}{100000}\PYG{p}{)}
\PYG{g+gp}{\PYGZgt{}\PYGZgt{}\PYGZgt{} }\PYG{n+nb}{sum}\PYG{p}{(}\PYG{n}{s}\PYG{o}{\PYGZgt{}}\PYG{o}{=}\PYG{l+m+mi}{12}\PYG{p}{)}\PYG{o}{/}\PYG{l+m+mf}{100000.} \PYG{o}{+} \PYG{n+nb}{sum}\PYG{p}{(}\PYG{n}{s}\PYG{o}{\PYGZlt{}}\PYG{o}{=}\PYG{l+m+mi}{3}\PYG{p}{)}\PYG{o}{/}\PYG{l+m+mf}{100000.}
\PYG{g+go}{\PYGZsh{}   answer = 0.003 ... pretty unlikely!}
\end{Verbatim}

\end{fulllineitems}

\index{laplace() (in module lib.IO.writefiles)}

\begin{fulllineitems}
\phantomsection\label{lib.IO:lib.IO.writefiles.laplace}\pysiglinewithargsret{\code{lib.IO.writefiles.}\bfcode{laplace}}{\emph{loc=0.0}, \emph{scale=1.0}, \emph{size=None}}{}
Draw samples from the Laplace or double exponential distribution with
specified location (or mean) and scale (decay).

The Laplace distribution is similar to the Gaussian/normal distribution,
but is sharper at the peak and has fatter tails. It represents the
difference between two independent, identically distributed exponential
random variables.
\begin{description}
\item[{loc}] \leavevmode{[}float{]}
The position, \(\mu\), of the distribution peak.

\item[{scale}] \leavevmode{[}float{]}
\(\lambda\), the exponential decay.

\end{description}

It has the probability density function
\begin{gather}
\begin{split}f(x; \mu, \lambda) = \frac{1}{2\lambda}
\exp\left(-\frac{|x - \mu|}{\lambda}\right).\end{split}\notag
\end{gather}
The first law of Laplace, from 1774, states that the frequency of an error
can be expressed as an exponential function of the absolute magnitude of
the error, which leads to the Laplace distribution. For many problems in
Economics and Health sciences, this distribution seems to model the data
better than the standard Gaussian distribution

Draw samples from the distribution

\begin{Verbatim}[commandchars=\\\{\}]
\PYG{g+gp}{\PYGZgt{}\PYGZgt{}\PYGZgt{} }\PYG{n}{loc}\PYG{p}{,} \PYG{n}{scale} \PYG{o}{=} \PYG{l+m+mf}{0.}\PYG{p}{,} \PYG{l+m+mf}{1.}
\PYG{g+gp}{\PYGZgt{}\PYGZgt{}\PYGZgt{} }\PYG{n}{s} \PYG{o}{=} \PYG{n}{np}\PYG{o}{.}\PYG{n}{random}\PYG{o}{.}\PYG{n}{laplace}\PYG{p}{(}\PYG{n}{loc}\PYG{p}{,} \PYG{n}{scale}\PYG{p}{,} \PYG{l+m+mi}{1000}\PYG{p}{)}
\end{Verbatim}

Display the histogram of the samples, along with
the probability density function:

\begin{Verbatim}[commandchars=\\\{\}]
\PYG{g+gp}{\PYGZgt{}\PYGZgt{}\PYGZgt{} }\PYG{k+kn}{import} \PYG{n+nn}{matplotlib.pyplot} \PYG{k+kn}{as} \PYG{n+nn}{plt}
\PYG{g+gp}{\PYGZgt{}\PYGZgt{}\PYGZgt{} }\PYG{n}{count}\PYG{p}{,} \PYG{n}{bins}\PYG{p}{,} \PYG{n}{ignored} \PYG{o}{=} \PYG{n}{plt}\PYG{o}{.}\PYG{n}{hist}\PYG{p}{(}\PYG{n}{s}\PYG{p}{,} \PYG{l+m+mi}{30}\PYG{p}{,} \PYG{n}{normed}\PYG{o}{=}\PYG{n+nb+bp}{True}\PYG{p}{)}
\PYG{g+gp}{\PYGZgt{}\PYGZgt{}\PYGZgt{} }\PYG{n}{x} \PYG{o}{=} \PYG{n}{np}\PYG{o}{.}\PYG{n}{arange}\PYG{p}{(}\PYG{o}{\PYGZhy{}}\PYG{l+m+mf}{8.}\PYG{p}{,} \PYG{l+m+mf}{8.}\PYG{p}{,} \PYG{o}{.}\PYG{l+m+mo}{01}\PYG{p}{)}
\PYG{g+gp}{\PYGZgt{}\PYGZgt{}\PYGZgt{} }\PYG{n}{pdf} \PYG{o}{=} \PYG{n}{np}\PYG{o}{.}\PYG{n}{exp}\PYG{p}{(}\PYG{o}{\PYGZhy{}}\PYG{n+nb}{abs}\PYG{p}{(}\PYG{n}{x}\PYG{o}{\PYGZhy{}}\PYG{n}{loc}\PYG{o}{/}\PYG{n}{scale}\PYG{p}{)}\PYG{p}{)}\PYG{o}{/}\PYG{p}{(}\PYG{l+m+mf}{2.}\PYG{o}{*}\PYG{n}{scale}\PYG{p}{)}
\PYG{g+gp}{\PYGZgt{}\PYGZgt{}\PYGZgt{} }\PYG{n}{plt}\PYG{o}{.}\PYG{n}{plot}\PYG{p}{(}\PYG{n}{x}\PYG{p}{,} \PYG{n}{pdf}\PYG{p}{)}
\end{Verbatim}

Plot Gaussian for comparison:

\begin{Verbatim}[commandchars=\\\{\}]
\PYG{g+gp}{\PYGZgt{}\PYGZgt{}\PYGZgt{} }\PYG{n}{g} \PYG{o}{=} \PYG{p}{(}\PYG{l+m+mi}{1}\PYG{o}{/}\PYG{p}{(}\PYG{n}{scale} \PYG{o}{*} \PYG{n}{np}\PYG{o}{.}\PYG{n}{sqrt}\PYG{p}{(}\PYG{l+m+mi}{2} \PYG{o}{*} \PYG{n}{np}\PYG{o}{.}\PYG{n}{pi}\PYG{p}{)}\PYG{p}{)} \PYG{o}{*} 
\PYG{g+gp}{... }     \PYG{n}{np}\PYG{o}{.}\PYG{n}{exp}\PYG{p}{(} \PYG{o}{\PYGZhy{}} \PYG{p}{(}\PYG{n}{x} \PYG{o}{\PYGZhy{}} \PYG{n}{loc}\PYG{p}{)}\PYG{o}{*}\PYG{o}{*}\PYG{l+m+mi}{2} \PYG{o}{/} \PYG{p}{(}\PYG{l+m+mi}{2} \PYG{o}{*} \PYG{n}{scale}\PYG{o}{*}\PYG{o}{*}\PYG{l+m+mi}{2}\PYG{p}{)} \PYG{p}{)}\PYG{p}{)}
\PYG{g+gp}{\PYGZgt{}\PYGZgt{}\PYGZgt{} }\PYG{n}{plt}\PYG{o}{.}\PYG{n}{plot}\PYG{p}{(}\PYG{n}{x}\PYG{p}{,}\PYG{n}{g}\PYG{p}{)}
\end{Verbatim}

\end{fulllineitems}

\index{logistic() (in module lib.IO.writefiles)}

\begin{fulllineitems}
\phantomsection\label{lib.IO:lib.IO.writefiles.logistic}\pysiglinewithargsret{\code{lib.IO.writefiles.}\bfcode{logistic}}{\emph{loc=0.0}, \emph{scale=1.0}, \emph{size=None}}{}
Draw samples from a Logistic distribution.

Samples are drawn from a Logistic distribution with specified
parameters, loc (location or mean, also median), and scale (\textgreater{}0).

loc : float

scale : float \textgreater{} 0.
\begin{description}
\item[{size}] \leavevmode{[}\{tuple, int\}{]}
Output shape.  If the given shape is, e.g., \code{(m, n, k)}, then
\code{m * n * k} samples are drawn.

\end{description}
\begin{description}
\item[{samples}] \leavevmode{[}\{ndarray, scalar\}{]}
where the values are all integers in  {[}0, n{]}.

\end{description}
\begin{description}
\item[{scipy.stats.distributions.logistic}] \leavevmode{[}probability density function,{]}
distribution or cumulative density function, etc.

\end{description}

The probability density for the Logistic distribution is
\begin{gather}
\begin{split}P(x) = P(x) = \frac{e^{-(x-\mu)/s}}{s(1+e^{-(x-\mu)/s})^2},\end{split}\notag
\end{gather}
where \(\mu\) = location and \(s\) = scale.

The Logistic distribution is used in Extreme Value problems where it
can act as a mixture of Gumbel distributions, in Epidemiology, and by
the World Chess Federation (FIDE) where it is used in the Elo ranking
system, assuming the performance of each player is a logistically
distributed random variable.

Draw samples from the distribution:

\begin{Verbatim}[commandchars=\\\{\}]
\PYG{g+gp}{\PYGZgt{}\PYGZgt{}\PYGZgt{} }\PYG{n}{loc}\PYG{p}{,} \PYG{n}{scale} \PYG{o}{=} \PYG{l+m+mi}{10}\PYG{p}{,} \PYG{l+m+mi}{1}
\PYG{g+gp}{\PYGZgt{}\PYGZgt{}\PYGZgt{} }\PYG{n}{s} \PYG{o}{=} \PYG{n}{np}\PYG{o}{.}\PYG{n}{random}\PYG{o}{.}\PYG{n}{logistic}\PYG{p}{(}\PYG{n}{loc}\PYG{p}{,} \PYG{n}{scale}\PYG{p}{,} \PYG{l+m+mi}{10000}\PYG{p}{)}
\PYG{g+gp}{\PYGZgt{}\PYGZgt{}\PYGZgt{} }\PYG{n}{count}\PYG{p}{,} \PYG{n}{bins}\PYG{p}{,} \PYG{n}{ignored} \PYG{o}{=} \PYG{n}{plt}\PYG{o}{.}\PYG{n}{hist}\PYG{p}{(}\PYG{n}{s}\PYG{p}{,} \PYG{n}{bins}\PYG{o}{=}\PYG{l+m+mi}{50}\PYG{p}{)}
\end{Verbatim}

\#   plot against distribution

\begin{Verbatim}[commandchars=\\\{\}]
\PYG{g+gp}{\PYGZgt{}\PYGZgt{}\PYGZgt{} }\PYG{k}{def} \PYG{n+nf}{logist}\PYG{p}{(}\PYG{n}{x}\PYG{p}{,} \PYG{n}{loc}\PYG{p}{,} \PYG{n}{scale}\PYG{p}{)}\PYG{p}{:}
\PYG{g+gp}{... }    \PYG{k}{return} \PYG{n}{exp}\PYG{p}{(}\PYG{p}{(}\PYG{n}{loc}\PYG{o}{\PYGZhy{}}\PYG{n}{x}\PYG{p}{)}\PYG{o}{/}\PYG{n}{scale}\PYG{p}{)}\PYG{o}{/}\PYG{p}{(}\PYG{n}{scale}\PYG{o}{*}\PYG{p}{(}\PYG{l+m+mi}{1}\PYG{o}{+}\PYG{n}{exp}\PYG{p}{(}\PYG{p}{(}\PYG{n}{loc}\PYG{o}{\PYGZhy{}}\PYG{n}{x}\PYG{p}{)}\PYG{o}{/}\PYG{n}{scale}\PYG{p}{)}\PYG{p}{)}\PYG{o}{*}\PYG{o}{*}\PYG{l+m+mi}{2}\PYG{p}{)}
\PYG{g+gp}{\PYGZgt{}\PYGZgt{}\PYGZgt{} }\PYG{n}{plt}\PYG{o}{.}\PYG{n}{plot}\PYG{p}{(}\PYG{n}{bins}\PYG{p}{,} \PYG{n}{logist}\PYG{p}{(}\PYG{n}{bins}\PYG{p}{,} \PYG{n}{loc}\PYG{p}{,} \PYG{n}{scale}\PYG{p}{)}\PYG{o}{*}\PYG{n}{count}\PYG{o}{.}\PYG{n}{max}\PYG{p}{(}\PYG{p}{)}\PYG{o}{/}\PYGZbs{}
\PYG{g+gp}{... }\PYG{n}{logist}\PYG{p}{(}\PYG{n}{bins}\PYG{p}{,} \PYG{n}{loc}\PYG{p}{,} \PYG{n}{scale}\PYG{p}{)}\PYG{o}{.}\PYG{n}{max}\PYG{p}{(}\PYG{p}{)}\PYG{p}{)}
\PYG{g+gp}{\PYGZgt{}\PYGZgt{}\PYGZgt{} }\PYG{n}{plt}\PYG{o}{.}\PYG{n}{show}\PYG{p}{(}\PYG{p}{)}
\end{Verbatim}

\end{fulllineitems}

\index{lognormal() (in module lib.IO.writefiles)}

\begin{fulllineitems}
\phantomsection\label{lib.IO:lib.IO.writefiles.lognormal}\pysiglinewithargsret{\code{lib.IO.writefiles.}\bfcode{lognormal}}{\emph{mean=0.0}, \emph{sigma=1.0}, \emph{size=None}}{}
Return samples drawn from a log-normal distribution.

Draw samples from a log-normal distribution with specified mean,
standard deviation, and array shape.  Note that the mean and standard
deviation are not the values for the distribution itself, but of the
underlying normal distribution it is derived from.
\begin{description}
\item[{mean}] \leavevmode{[}float{]}
Mean value of the underlying normal distribution

\item[{sigma}] \leavevmode{[}float, \textgreater{} 0.{]}
Standard deviation of the underlying normal distribution

\item[{size}] \leavevmode{[}tuple of ints{]}
Output shape.  If the given shape is, e.g., \code{(m, n, k)}, then
\code{m * n * k} samples are drawn.

\end{description}
\begin{description}
\item[{samples}] \leavevmode{[}ndarray or float{]}
The desired samples. An array of the same shape as \emph{size} if given,
if \emph{size} is None a float is returned.

\end{description}
\begin{description}
\item[{scipy.stats.lognorm}] \leavevmode{[}probability density function, distribution,{]}
cumulative density function, etc.

\end{description}

A variable \emph{x} has a log-normal distribution if \emph{log(x)} is normally
distributed.  The probability density function for the log-normal
distribution is:
\begin{gather}
\begin{split}p(x) = \frac{1}{\sigma x \sqrt{2\pi}}
e^{(-\frac{(ln(x)-\mu)^2}{2\sigma^2})}\end{split}\notag
\end{gather}
where \(\mu\) is the mean and \(\sigma\) is the standard
deviation of the normally distributed logarithm of the variable.
A log-normal distribution results if a random variable is the \emph{product}
of a large number of independent, identically-distributed variables in
the same way that a normal distribution results if the variable is the
\emph{sum} of a large number of independent, identically-distributed
variables.

Limpert, E., Stahel, W. A., and Abbt, M., ``Log-normal Distributions
across the Sciences: Keys and Clues,'' \emph{BioScience}, Vol. 51, No. 5,
May, 2001.  \href{http://stat.ethz.ch/~stahel/lognormal/bioscience.pdf}{http://stat.ethz.ch/\textasciitilde{}stahel/lognormal/bioscience.pdf}

Reiss, R.D. and Thomas, M., \emph{Statistical Analysis of Extreme Values},
Basel: Birkhauser Verlag, 2001, pp. 31-32.

Draw samples from the distribution:

\begin{Verbatim}[commandchars=\\\{\}]
\PYG{g+gp}{\PYGZgt{}\PYGZgt{}\PYGZgt{} }\PYG{n}{mu}\PYG{p}{,} \PYG{n}{sigma} \PYG{o}{=} \PYG{l+m+mf}{3.}\PYG{p}{,} \PYG{l+m+mf}{1.} \PYG{c}{\PYGZsh{} mean and standard deviation}
\PYG{g+gp}{\PYGZgt{}\PYGZgt{}\PYGZgt{} }\PYG{n}{s} \PYG{o}{=} \PYG{n}{np}\PYG{o}{.}\PYG{n}{random}\PYG{o}{.}\PYG{n}{lognormal}\PYG{p}{(}\PYG{n}{mu}\PYG{p}{,} \PYG{n}{sigma}\PYG{p}{,} \PYG{l+m+mi}{1000}\PYG{p}{)}
\end{Verbatim}

Display the histogram of the samples, along with
the probability density function:

\begin{Verbatim}[commandchars=\\\{\}]
\PYG{g+gp}{\PYGZgt{}\PYGZgt{}\PYGZgt{} }\PYG{k+kn}{import} \PYG{n+nn}{matplotlib.pyplot} \PYG{k+kn}{as} \PYG{n+nn}{plt}
\PYG{g+gp}{\PYGZgt{}\PYGZgt{}\PYGZgt{} }\PYG{n}{count}\PYG{p}{,} \PYG{n}{bins}\PYG{p}{,} \PYG{n}{ignored} \PYG{o}{=} \PYG{n}{plt}\PYG{o}{.}\PYG{n}{hist}\PYG{p}{(}\PYG{n}{s}\PYG{p}{,} \PYG{l+m+mi}{100}\PYG{p}{,} \PYG{n}{normed}\PYG{o}{=}\PYG{n+nb+bp}{True}\PYG{p}{,} \PYG{n}{align}\PYG{o}{=}\PYG{l+s}{\PYGZsq{}}\PYG{l+s}{mid}\PYG{l+s}{\PYGZsq{}}\PYG{p}{)}
\end{Verbatim}

\begin{Verbatim}[commandchars=\\\{\}]
\PYG{g+gp}{\PYGZgt{}\PYGZgt{}\PYGZgt{} }\PYG{n}{x} \PYG{o}{=} \PYG{n}{np}\PYG{o}{.}\PYG{n}{linspace}\PYG{p}{(}\PYG{n+nb}{min}\PYG{p}{(}\PYG{n}{bins}\PYG{p}{)}\PYG{p}{,} \PYG{n+nb}{max}\PYG{p}{(}\PYG{n}{bins}\PYG{p}{)}\PYG{p}{,} \PYG{l+m+mi}{10000}\PYG{p}{)}
\PYG{g+gp}{\PYGZgt{}\PYGZgt{}\PYGZgt{} }\PYG{n}{pdf} \PYG{o}{=} \PYG{p}{(}\PYG{n}{np}\PYG{o}{.}\PYG{n}{exp}\PYG{p}{(}\PYG{o}{\PYGZhy{}}\PYG{p}{(}\PYG{n}{np}\PYG{o}{.}\PYG{n}{log}\PYG{p}{(}\PYG{n}{x}\PYG{p}{)} \PYG{o}{\PYGZhy{}} \PYG{n}{mu}\PYG{p}{)}\PYG{o}{*}\PYG{o}{*}\PYG{l+m+mi}{2} \PYG{o}{/} \PYG{p}{(}\PYG{l+m+mi}{2} \PYG{o}{*} \PYG{n}{sigma}\PYG{o}{*}\PYG{o}{*}\PYG{l+m+mi}{2}\PYG{p}{)}\PYG{p}{)}
\PYG{g+gp}{... }       \PYG{o}{/} \PYG{p}{(}\PYG{n}{x} \PYG{o}{*} \PYG{n}{sigma} \PYG{o}{*} \PYG{n}{np}\PYG{o}{.}\PYG{n}{sqrt}\PYG{p}{(}\PYG{l+m+mi}{2} \PYG{o}{*} \PYG{n}{np}\PYG{o}{.}\PYG{n}{pi}\PYG{p}{)}\PYG{p}{)}\PYG{p}{)}
\end{Verbatim}

\begin{Verbatim}[commandchars=\\\{\}]
\PYG{g+gp}{\PYGZgt{}\PYGZgt{}\PYGZgt{} }\PYG{n}{plt}\PYG{o}{.}\PYG{n}{plot}\PYG{p}{(}\PYG{n}{x}\PYG{p}{,} \PYG{n}{pdf}\PYG{p}{,} \PYG{n}{linewidth}\PYG{o}{=}\PYG{l+m+mi}{2}\PYG{p}{,} \PYG{n}{color}\PYG{o}{=}\PYG{l+s}{\PYGZsq{}}\PYG{l+s}{r}\PYG{l+s}{\PYGZsq{}}\PYG{p}{)}
\PYG{g+gp}{\PYGZgt{}\PYGZgt{}\PYGZgt{} }\PYG{n}{plt}\PYG{o}{.}\PYG{n}{axis}\PYG{p}{(}\PYG{l+s}{\PYGZsq{}}\PYG{l+s}{tight}\PYG{l+s}{\PYGZsq{}}\PYG{p}{)}
\PYG{g+gp}{\PYGZgt{}\PYGZgt{}\PYGZgt{} }\PYG{n}{plt}\PYG{o}{.}\PYG{n}{show}\PYG{p}{(}\PYG{p}{)}
\end{Verbatim}

Demonstrate that taking the products of random samples from a uniform
distribution can be fit well by a log-normal probability density function.

\begin{Verbatim}[commandchars=\\\{\}]
\PYG{g+gp}{\PYGZgt{}\PYGZgt{}\PYGZgt{} }\PYG{c}{\PYGZsh{} Generate a thousand samples: each is the product of 100 random}
\PYG{g+gp}{\PYGZgt{}\PYGZgt{}\PYGZgt{} }\PYG{c}{\PYGZsh{} values, drawn from a normal distribution.}
\PYG{g+gp}{\PYGZgt{}\PYGZgt{}\PYGZgt{} }\PYG{n}{b} \PYG{o}{=} \PYG{p}{[}\PYG{p}{]}
\PYG{g+gp}{\PYGZgt{}\PYGZgt{}\PYGZgt{} }\PYG{k}{for} \PYG{n}{i} \PYG{o+ow}{in} \PYG{n+nb}{range}\PYG{p}{(}\PYG{l+m+mi}{1000}\PYG{p}{)}\PYG{p}{:}
\PYG{g+gp}{... }   \PYG{n}{a} \PYG{o}{=} \PYG{l+m+mf}{10.} \PYG{o}{+} \PYG{n}{np}\PYG{o}{.}\PYG{n}{random}\PYG{o}{.}\PYG{n}{random}\PYG{p}{(}\PYG{l+m+mi}{100}\PYG{p}{)}
\PYG{g+gp}{... }   \PYG{n}{b}\PYG{o}{.}\PYG{n}{append}\PYG{p}{(}\PYG{n}{np}\PYG{o}{.}\PYG{n}{product}\PYG{p}{(}\PYG{n}{a}\PYG{p}{)}\PYG{p}{)}
\end{Verbatim}

\begin{Verbatim}[commandchars=\\\{\}]
\PYG{g+gp}{\PYGZgt{}\PYGZgt{}\PYGZgt{} }\PYG{n}{b} \PYG{o}{=} \PYG{n}{np}\PYG{o}{.}\PYG{n}{array}\PYG{p}{(}\PYG{n}{b}\PYG{p}{)} \PYG{o}{/} \PYG{n}{np}\PYG{o}{.}\PYG{n}{min}\PYG{p}{(}\PYG{n}{b}\PYG{p}{)} \PYG{c}{\PYGZsh{} scale values to be positive}
\PYG{g+gp}{\PYGZgt{}\PYGZgt{}\PYGZgt{} }\PYG{n}{count}\PYG{p}{,} \PYG{n}{bins}\PYG{p}{,} \PYG{n}{ignored} \PYG{o}{=} \PYG{n}{plt}\PYG{o}{.}\PYG{n}{hist}\PYG{p}{(}\PYG{n}{b}\PYG{p}{,} \PYG{l+m+mi}{100}\PYG{p}{,} \PYG{n}{normed}\PYG{o}{=}\PYG{n+nb+bp}{True}\PYG{p}{,} \PYG{n}{align}\PYG{o}{=}\PYG{l+s}{\PYGZsq{}}\PYG{l+s}{center}\PYG{l+s}{\PYGZsq{}}\PYG{p}{)}
\PYG{g+gp}{\PYGZgt{}\PYGZgt{}\PYGZgt{} }\PYG{n}{sigma} \PYG{o}{=} \PYG{n}{np}\PYG{o}{.}\PYG{n}{std}\PYG{p}{(}\PYG{n}{np}\PYG{o}{.}\PYG{n}{log}\PYG{p}{(}\PYG{n}{b}\PYG{p}{)}\PYG{p}{)}
\PYG{g+gp}{\PYGZgt{}\PYGZgt{}\PYGZgt{} }\PYG{n}{mu} \PYG{o}{=} \PYG{n}{np}\PYG{o}{.}\PYG{n}{mean}\PYG{p}{(}\PYG{n}{np}\PYG{o}{.}\PYG{n}{log}\PYG{p}{(}\PYG{n}{b}\PYG{p}{)}\PYG{p}{)}
\end{Verbatim}

\begin{Verbatim}[commandchars=\\\{\}]
\PYG{g+gp}{\PYGZgt{}\PYGZgt{}\PYGZgt{} }\PYG{n}{x} \PYG{o}{=} \PYG{n}{np}\PYG{o}{.}\PYG{n}{linspace}\PYG{p}{(}\PYG{n+nb}{min}\PYG{p}{(}\PYG{n}{bins}\PYG{p}{)}\PYG{p}{,} \PYG{n+nb}{max}\PYG{p}{(}\PYG{n}{bins}\PYG{p}{)}\PYG{p}{,} \PYG{l+m+mi}{10000}\PYG{p}{)}
\PYG{g+gp}{\PYGZgt{}\PYGZgt{}\PYGZgt{} }\PYG{n}{pdf} \PYG{o}{=} \PYG{p}{(}\PYG{n}{np}\PYG{o}{.}\PYG{n}{exp}\PYG{p}{(}\PYG{o}{\PYGZhy{}}\PYG{p}{(}\PYG{n}{np}\PYG{o}{.}\PYG{n}{log}\PYG{p}{(}\PYG{n}{x}\PYG{p}{)} \PYG{o}{\PYGZhy{}} \PYG{n}{mu}\PYG{p}{)}\PYG{o}{*}\PYG{o}{*}\PYG{l+m+mi}{2} \PYG{o}{/} \PYG{p}{(}\PYG{l+m+mi}{2} \PYG{o}{*} \PYG{n}{sigma}\PYG{o}{*}\PYG{o}{*}\PYG{l+m+mi}{2}\PYG{p}{)}\PYG{p}{)}
\PYG{g+gp}{... }       \PYG{o}{/} \PYG{p}{(}\PYG{n}{x} \PYG{o}{*} \PYG{n}{sigma} \PYG{o}{*} \PYG{n}{np}\PYG{o}{.}\PYG{n}{sqrt}\PYG{p}{(}\PYG{l+m+mi}{2} \PYG{o}{*} \PYG{n}{np}\PYG{o}{.}\PYG{n}{pi}\PYG{p}{)}\PYG{p}{)}\PYG{p}{)}
\end{Verbatim}

\begin{Verbatim}[commandchars=\\\{\}]
\PYG{g+gp}{\PYGZgt{}\PYGZgt{}\PYGZgt{} }\PYG{n}{plt}\PYG{o}{.}\PYG{n}{plot}\PYG{p}{(}\PYG{n}{x}\PYG{p}{,} \PYG{n}{pdf}\PYG{p}{,} \PYG{n}{color}\PYG{o}{=}\PYG{l+s}{\PYGZsq{}}\PYG{l+s}{r}\PYG{l+s}{\PYGZsq{}}\PYG{p}{,} \PYG{n}{linewidth}\PYG{o}{=}\PYG{l+m+mi}{2}\PYG{p}{)}
\PYG{g+gp}{\PYGZgt{}\PYGZgt{}\PYGZgt{} }\PYG{n}{plt}\PYG{o}{.}\PYG{n}{show}\PYG{p}{(}\PYG{p}{)}
\end{Verbatim}

\end{fulllineitems}

\index{logseries() (in module lib.IO.writefiles)}

\begin{fulllineitems}
\phantomsection\label{lib.IO:lib.IO.writefiles.logseries}\pysiglinewithargsret{\code{lib.IO.writefiles.}\bfcode{logseries}}{\emph{p}, \emph{size=None}}{}
Draw samples from a Logarithmic Series distribution.

Samples are drawn from a Log Series distribution with specified
parameter, p (probability, 0 \textless{} p \textless{} 1).

loc : float

scale : float \textgreater{} 0.
\begin{description}
\item[{size}] \leavevmode{[}\{tuple, int\}{]}
Output shape.  If the given shape is, e.g., \code{(m, n, k)}, then
\code{m * n * k} samples are drawn.

\end{description}
\begin{description}
\item[{samples}] \leavevmode{[}\{ndarray, scalar\}{]}
where the values are all integers in  {[}0, n{]}.

\end{description}
\begin{description}
\item[{scipy.stats.distributions.logser}] \leavevmode{[}probability density function,{]}
distribution or cumulative density function, etc.

\end{description}

The probability density for the Log Series distribution is
\begin{gather}
\begin{split}P(k) = \frac{-p^k}{k \ln(1-p)},\end{split}\notag
\end{gather}
where p = probability.

The Log Series distribution is frequently used to represent species
richness and occurrence, first proposed by Fisher, Corbet, and
Williams in 1943 {[}2{]}.  It may also be used to model the numbers of
occupants seen in cars {[}3{]}.

Draw samples from the distribution:

\begin{Verbatim}[commandchars=\\\{\}]
\PYG{g+gp}{\PYGZgt{}\PYGZgt{}\PYGZgt{} }\PYG{n}{a} \PYG{o}{=} \PYG{o}{.}\PYG{l+m+mi}{6}
\PYG{g+gp}{\PYGZgt{}\PYGZgt{}\PYGZgt{} }\PYG{n}{s} \PYG{o}{=} \PYG{n}{np}\PYG{o}{.}\PYG{n}{random}\PYG{o}{.}\PYG{n}{logseries}\PYG{p}{(}\PYG{n}{a}\PYG{p}{,} \PYG{l+m+mi}{10000}\PYG{p}{)}
\PYG{g+gp}{\PYGZgt{}\PYGZgt{}\PYGZgt{} }\PYG{n}{count}\PYG{p}{,} \PYG{n}{bins}\PYG{p}{,} \PYG{n}{ignored} \PYG{o}{=} \PYG{n}{plt}\PYG{o}{.}\PYG{n}{hist}\PYG{p}{(}\PYG{n}{s}\PYG{p}{)}
\end{Verbatim}

\#   plot against distribution

\begin{Verbatim}[commandchars=\\\{\}]
\PYG{g+gp}{\PYGZgt{}\PYGZgt{}\PYGZgt{} }\PYG{k}{def} \PYG{n+nf}{logseries}\PYG{p}{(}\PYG{n}{k}\PYG{p}{,} \PYG{n}{p}\PYG{p}{)}\PYG{p}{:}
\PYG{g+gp}{... }    \PYG{k}{return} \PYG{o}{\PYGZhy{}}\PYG{n}{p}\PYG{o}{*}\PYG{o}{*}\PYG{n}{k}\PYG{o}{/}\PYG{p}{(}\PYG{n}{k}\PYG{o}{*}\PYG{n}{log}\PYG{p}{(}\PYG{l+m+mi}{1}\PYG{o}{\PYGZhy{}}\PYG{n}{p}\PYG{p}{)}\PYG{p}{)}
\PYG{g+gp}{\PYGZgt{}\PYGZgt{}\PYGZgt{} }\PYG{n}{plt}\PYG{o}{.}\PYG{n}{plot}\PYG{p}{(}\PYG{n}{bins}\PYG{p}{,} \PYG{n}{logseries}\PYG{p}{(}\PYG{n}{bins}\PYG{p}{,} \PYG{n}{a}\PYG{p}{)}\PYG{o}{*}\PYG{n}{count}\PYG{o}{.}\PYG{n}{max}\PYG{p}{(}\PYG{p}{)}\PYG{o}{/}
\PYG{g+go}{             logseries(bins, a).max(), \PYGZsq{}r\PYGZsq{})}
\PYG{g+gp}{\PYGZgt{}\PYGZgt{}\PYGZgt{} }\PYG{n}{plt}\PYG{o}{.}\PYG{n}{show}\PYG{p}{(}\PYG{p}{)}
\end{Verbatim}

\end{fulllineitems}

\index{multinomial() (in module lib.IO.writefiles)}

\begin{fulllineitems}
\phantomsection\label{lib.IO:lib.IO.writefiles.multinomial}\pysiglinewithargsret{\code{lib.IO.writefiles.}\bfcode{multinomial}}{\emph{n}, \emph{pvals}, \emph{size=None}}{}
Draw samples from a multinomial distribution.

The multinomial distribution is a multivariate generalisation of the
binomial distribution.  Take an experiment with one of \code{p}
possible outcomes.  An example of such an experiment is throwing a dice,
where the outcome can be 1 through 6.  Each sample drawn from the
distribution represents \emph{n} such experiments.  Its values,
\code{X\_i = {[}X\_0, X\_1, ..., X\_p{]}}, represent the number of times the outcome
was \code{i}.
\begin{description}
\item[{n}] \leavevmode{[}int{]}
Number of experiments.

\item[{pvals}] \leavevmode{[}sequence of floats, length p{]}
Probabilities of each of the \code{p} different outcomes.  These
should sum to 1 (however, the last element is always assumed to
account for the remaining probability, as long as
\code{sum(pvals{[}:-1{]}) \textless{}= 1)}.

\item[{size}] \leavevmode{[}tuple of ints{]}
Given a \emph{size} of \code{(M, N, K)}, then \code{M*N*K} samples are drawn,
and the output shape becomes \code{(M, N, K, p)}, since each sample
has shape \code{(p,)}.

\end{description}

Throw a dice 20 times:

\begin{Verbatim}[commandchars=\\\{\}]
\PYG{g+gp}{\PYGZgt{}\PYGZgt{}\PYGZgt{} }\PYG{n}{np}\PYG{o}{.}\PYG{n}{random}\PYG{o}{.}\PYG{n}{multinomial}\PYG{p}{(}\PYG{l+m+mi}{20}\PYG{p}{,} \PYG{p}{[}\PYG{l+m+mi}{1}\PYG{o}{/}\PYG{l+m+mf}{6.}\PYG{p}{]}\PYG{o}{*}\PYG{l+m+mi}{6}\PYG{p}{,} \PYG{n}{size}\PYG{o}{=}\PYG{l+m+mi}{1}\PYG{p}{)}
\PYG{g+go}{array([[4, 1, 7, 5, 2, 1]])}
\end{Verbatim}

It landed 4 times on 1, once on 2, etc.

Now, throw the dice 20 times, and 20 times again:

\begin{Verbatim}[commandchars=\\\{\}]
\PYG{g+gp}{\PYGZgt{}\PYGZgt{}\PYGZgt{} }\PYG{n}{np}\PYG{o}{.}\PYG{n}{random}\PYG{o}{.}\PYG{n}{multinomial}\PYG{p}{(}\PYG{l+m+mi}{20}\PYG{p}{,} \PYG{p}{[}\PYG{l+m+mi}{1}\PYG{o}{/}\PYG{l+m+mf}{6.}\PYG{p}{]}\PYG{o}{*}\PYG{l+m+mi}{6}\PYG{p}{,} \PYG{n}{size}\PYG{o}{=}\PYG{l+m+mi}{2}\PYG{p}{)}
\PYG{g+go}{array([[3, 4, 3, 3, 4, 3],}
\PYG{g+go}{       [2, 4, 3, 4, 0, 7]])}
\end{Verbatim}

For the first run, we threw 3 times 1, 4 times 2, etc.  For the second,
we threw 2 times 1, 4 times 2, etc.

A loaded dice is more likely to land on number 6:

\begin{Verbatim}[commandchars=\\\{\}]
\PYG{g+gp}{\PYGZgt{}\PYGZgt{}\PYGZgt{} }\PYG{n}{np}\PYG{o}{.}\PYG{n}{random}\PYG{o}{.}\PYG{n}{multinomial}\PYG{p}{(}\PYG{l+m+mi}{100}\PYG{p}{,} \PYG{p}{[}\PYG{l+m+mi}{1}\PYG{o}{/}\PYG{l+m+mf}{7.}\PYG{p}{]}\PYG{o}{*}\PYG{l+m+mi}{5}\PYG{p}{)}
\PYG{g+go}{array([13, 16, 13, 16, 42])}
\end{Verbatim}

\end{fulllineitems}

\index{multivariate\_normal() (in module lib.IO.writefiles)}

\begin{fulllineitems}
\phantomsection\label{lib.IO:lib.IO.writefiles.multivariate_normal}\pysiglinewithargsret{\code{lib.IO.writefiles.}\bfcode{multivariate\_normal}}{\emph{mean}, \emph{cov}\optional{, \emph{size}}}{}
Draw random samples from a multivariate normal distribution.

The multivariate normal, multinormal or Gaussian distribution is a
generalization of the one-dimensional normal distribution to higher
dimensions.  Such a distribution is specified by its mean and
covariance matrix.  These parameters are analogous to the mean
(average or ``center'') and variance (standard deviation, or ``width,''
squared) of the one-dimensional normal distribution.
\begin{description}
\item[{mean}] \leavevmode{[}1-D array\_like, of length N{]}
Mean of the N-dimensional distribution.

\item[{cov}] \leavevmode{[}2-D array\_like, of shape (N, N){]}
Covariance matrix of the distribution.  Must be symmetric and
positive semi-definite for ``physically meaningful'' results.

\item[{size}] \leavevmode{[}int or tuple of ints, optional{]}
Given a shape of, for example, \code{(m,n,k)}, \code{m*n*k} samples are
generated, and packed in an \emph{m}-by-\emph{n}-by-\emph{k} arrangement.  Because
each sample is \emph{N}-dimensional, the output shape is \code{(m,n,k,N)}.
If no shape is specified, a single (\emph{N}-D) sample is returned.

\end{description}
\begin{description}
\item[{out}] \leavevmode{[}ndarray{]}
The drawn samples, of shape \emph{size}, if that was provided.  If not,
the shape is \code{(N,)}.

In other words, each entry \code{out{[}i,j,...,:{]}} is an N-dimensional
value drawn from the distribution.

\end{description}

The mean is a coordinate in N-dimensional space, which represents the
location where samples are most likely to be generated.  This is
analogous to the peak of the bell curve for the one-dimensional or
univariate normal distribution.

Covariance indicates the level to which two variables vary together.
From the multivariate normal distribution, we draw N-dimensional
samples, \(X = [x_1, x_2, ... x_N]\).  The covariance matrix
element \(C_{ij}\) is the covariance of \(x_i\) and \(x_j\).
The element \(C_{ii}\) is the variance of \(x_i\) (i.e. its
``spread'').

Instead of specifying the full covariance matrix, popular
approximations include:
\begin{itemize}
\item {} 
Spherical covariance (\emph{cov} is a multiple of the identity matrix)

\item {} 
Diagonal covariance (\emph{cov} has non-negative elements, and only on
the diagonal)

\end{itemize}

This geometrical property can be seen in two dimensions by plotting
generated data-points:

\begin{Verbatim}[commandchars=\\\{\}]
\PYG{g+gp}{\PYGZgt{}\PYGZgt{}\PYGZgt{} }\PYG{n}{mean} \PYG{o}{=} \PYG{p}{[}\PYG{l+m+mi}{0}\PYG{p}{,}\PYG{l+m+mi}{0}\PYG{p}{]}
\PYG{g+gp}{\PYGZgt{}\PYGZgt{}\PYGZgt{} }\PYG{n}{cov} \PYG{o}{=} \PYG{p}{[}\PYG{p}{[}\PYG{l+m+mi}{1}\PYG{p}{,}\PYG{l+m+mi}{0}\PYG{p}{]}\PYG{p}{,}\PYG{p}{[}\PYG{l+m+mi}{0}\PYG{p}{,}\PYG{l+m+mi}{100}\PYG{p}{]}\PYG{p}{]} \PYG{c}{\PYGZsh{} diagonal covariance, points lie on x or y\PYGZhy{}axis}
\end{Verbatim}

\begin{Verbatim}[commandchars=\\\{\}]
\PYG{g+gp}{\PYGZgt{}\PYGZgt{}\PYGZgt{} }\PYG{k+kn}{import} \PYG{n+nn}{matplotlib.pyplot} \PYG{k+kn}{as} \PYG{n+nn}{plt}
\PYG{g+gp}{\PYGZgt{}\PYGZgt{}\PYGZgt{} }\PYG{n}{x}\PYG{p}{,}\PYG{n}{y} \PYG{o}{=} \PYG{n}{np}\PYG{o}{.}\PYG{n}{random}\PYG{o}{.}\PYG{n}{multivariate\PYGZus{}normal}\PYG{p}{(}\PYG{n}{mean}\PYG{p}{,}\PYG{n}{cov}\PYG{p}{,}\PYG{l+m+mi}{5000}\PYG{p}{)}\PYG{o}{.}\PYG{n}{T}
\PYG{g+gp}{\PYGZgt{}\PYGZgt{}\PYGZgt{} }\PYG{n}{plt}\PYG{o}{.}\PYG{n}{plot}\PYG{p}{(}\PYG{n}{x}\PYG{p}{,}\PYG{n}{y}\PYG{p}{,}\PYG{l+s}{\PYGZsq{}}\PYG{l+s}{x}\PYG{l+s}{\PYGZsq{}}\PYG{p}{)}\PYG{p}{;} \PYG{n}{plt}\PYG{o}{.}\PYG{n}{axis}\PYG{p}{(}\PYG{l+s}{\PYGZsq{}}\PYG{l+s}{equal}\PYG{l+s}{\PYGZsq{}}\PYG{p}{)}\PYG{p}{;} \PYG{n}{plt}\PYG{o}{.}\PYG{n}{show}\PYG{p}{(}\PYG{p}{)}
\end{Verbatim}

Note that the covariance matrix must be non-negative definite.

Papoulis, A., \emph{Probability, Random Variables, and Stochastic Processes},
3rd ed., New York: McGraw-Hill, 1991.

Duda, R. O., Hart, P. E., and Stork, D. G., \emph{Pattern Classification},
2nd ed., New York: Wiley, 2001.

\begin{Verbatim}[commandchars=\\\{\}]
\PYG{g+gp}{\PYGZgt{}\PYGZgt{}\PYGZgt{} }\PYG{n}{mean} \PYG{o}{=} \PYG{p}{(}\PYG{l+m+mi}{1}\PYG{p}{,}\PYG{l+m+mi}{2}\PYG{p}{)}
\PYG{g+gp}{\PYGZgt{}\PYGZgt{}\PYGZgt{} }\PYG{n}{cov} \PYG{o}{=} \PYG{p}{[}\PYG{p}{[}\PYG{l+m+mi}{1}\PYG{p}{,}\PYG{l+m+mi}{0}\PYG{p}{]}\PYG{p}{,}\PYG{p}{[}\PYG{l+m+mi}{1}\PYG{p}{,}\PYG{l+m+mi}{0}\PYG{p}{]}\PYG{p}{]}
\PYG{g+gp}{\PYGZgt{}\PYGZgt{}\PYGZgt{} }\PYG{n}{x} \PYG{o}{=} \PYG{n}{np}\PYG{o}{.}\PYG{n}{random}\PYG{o}{.}\PYG{n}{multivariate\PYGZus{}normal}\PYG{p}{(}\PYG{n}{mean}\PYG{p}{,}\PYG{n}{cov}\PYG{p}{,}\PYG{p}{(}\PYG{l+m+mi}{3}\PYG{p}{,}\PYG{l+m+mi}{3}\PYG{p}{)}\PYG{p}{)}
\PYG{g+gp}{\PYGZgt{}\PYGZgt{}\PYGZgt{} }\PYG{n}{x}\PYG{o}{.}\PYG{n}{shape}
\PYG{g+go}{(3, 3, 2)}
\end{Verbatim}

The following is probably true, given that 0.6 is roughly twice the
standard deviation:

\begin{Verbatim}[commandchars=\\\{\}]
\PYG{g+gp}{\PYGZgt{}\PYGZgt{}\PYGZgt{} }\PYG{k}{print} \PYG{n+nb}{list}\PYG{p}{(} \PYG{p}{(}\PYG{n}{x}\PYG{p}{[}\PYG{l+m+mi}{0}\PYG{p}{,}\PYG{l+m+mi}{0}\PYG{p}{,}\PYG{p}{:}\PYG{p}{]} \PYG{o}{\PYGZhy{}} \PYG{n}{mean}\PYG{p}{)} \PYG{o}{\PYGZlt{}} \PYG{l+m+mf}{0.6} \PYG{p}{)}
\PYG{g+go}{[True, True]}
\end{Verbatim}

\end{fulllineitems}

\index{negative\_binomial() (in module lib.IO.writefiles)}

\begin{fulllineitems}
\phantomsection\label{lib.IO:lib.IO.writefiles.negative_binomial}\pysiglinewithargsret{\code{lib.IO.writefiles.}\bfcode{negative\_binomial}}{\emph{n}, \emph{p}, \emph{size=None}}{}
Draw samples from a negative\_binomial distribution.

Samples are drawn from a negative\_Binomial distribution with specified
parameters, \emph{n} trials and \emph{p} probability of success where \emph{n} is an
integer \textgreater{} 0 and \emph{p} is in the interval {[}0, 1{]}.
\begin{description}
\item[{n}] \leavevmode{[}int{]}
Parameter, \textgreater{} 0.

\item[{p}] \leavevmode{[}float{]}
Parameter, \textgreater{}= 0 and \textless{}=1.

\item[{size}] \leavevmode{[}int or tuple of ints{]}
Output shape. If the given shape is, e.g., \code{(m, n, k)}, then
\code{m * n * k} samples are drawn.

\end{description}
\begin{description}
\item[{samples}] \leavevmode{[}int or ndarray of ints{]}
Drawn samples.

\end{description}

The probability density for the Negative Binomial distribution is
\begin{gather}
\begin{split}P(N;n,p) = \binom{N+n-1}{n-1}p^{n}(1-p)^{N},\end{split}\notag
\end{gather}
where \(n-1\) is the number of successes, \(p\) is the probability
of success, and \(N+n-1\) is the number of trials.

The negative binomial distribution gives the probability of n-1 successes
and N failures in N+n-1 trials, and success on the (N+n)th trial.

If one throws a die repeatedly until the third time a ``1'' appears, then the
probability distribution of the number of non-``1''s that appear before the
third ``1'' is a negative binomial distribution.

Draw samples from the distribution:

A real world example. A company drills wild-cat oil exploration wells, each
with an estimated probability of success of 0.1.  What is the probability
of having one success for each successive well, that is what is the
probability of a single success after drilling 5 wells, after 6 wells,
etc.?

\begin{Verbatim}[commandchars=\\\{\}]
\PYG{g+gp}{\PYGZgt{}\PYGZgt{}\PYGZgt{} }\PYG{n}{s} \PYG{o}{=} \PYG{n}{np}\PYG{o}{.}\PYG{n}{random}\PYG{o}{.}\PYG{n}{negative\PYGZus{}binomial}\PYG{p}{(}\PYG{l+m+mi}{1}\PYG{p}{,} \PYG{l+m+mf}{0.1}\PYG{p}{,} \PYG{l+m+mi}{100000}\PYG{p}{)}
\PYG{g+gp}{\PYGZgt{}\PYGZgt{}\PYGZgt{} }\PYG{k}{for} \PYG{n}{i} \PYG{o+ow}{in} \PYG{n+nb}{range}\PYG{p}{(}\PYG{l+m+mi}{1}\PYG{p}{,} \PYG{l+m+mi}{11}\PYG{p}{)}\PYG{p}{:}
\PYG{g+gp}{... }   \PYG{n}{probability} \PYG{o}{=} \PYG{n+nb}{sum}\PYG{p}{(}\PYG{n}{s}\PYG{o}{\PYGZlt{}}\PYG{n}{i}\PYG{p}{)} \PYG{o}{/} \PYG{l+m+mf}{100000.}
\PYG{g+gp}{... }   \PYG{k}{print} \PYG{n}{i}\PYG{p}{,} \PYG{l+s}{\PYGZdq{}}\PYG{l+s}{wells drilled, probability of one success =}\PYG{l+s}{\PYGZdq{}}\PYG{p}{,} \PYG{n}{probability}
\end{Verbatim}

\end{fulllineitems}

\index{noncentral\_chisquare() (in module lib.IO.writefiles)}

\begin{fulllineitems}
\phantomsection\label{lib.IO:lib.IO.writefiles.noncentral_chisquare}\pysiglinewithargsret{\code{lib.IO.writefiles.}\bfcode{noncentral\_chisquare}}{\emph{df}, \emph{nonc}, \emph{size=None}}{}
Draw samples from a noncentral chi-square distribution.

The noncentral \(\chi^2\) distribution is a generalisation of
the \(\chi^2\) distribution.
\begin{description}
\item[{df}] \leavevmode{[}int{]}
Degrees of freedom, should be \textgreater{}= 1.

\item[{nonc}] \leavevmode{[}float{]}
Non-centrality, should be \textgreater{} 0.

\item[{size}] \leavevmode{[}int or tuple of ints{]}
Shape of the output.

\end{description}

The probability density function for the noncentral Chi-square distribution
is
\begin{gather}
\begin{split}P(x;df,nonc) = \sum^{\infty}_{i=0}
\frac{e^{-nonc/2}(nonc/2)^{i}}{i!}P_{Y_{df+2i}}(x),\end{split}\notag
\end{gather}
where \(Y_{q}\) is the Chi-square with q degrees of freedom.

In Delhi (2007), it is noted that the noncentral chi-square is useful in
bombing and coverage problems, the probability of killing the point target
given by the noncentral chi-squared distribution.

Draw values from the distribution and plot the histogram

\begin{Verbatim}[commandchars=\\\{\}]
\PYG{g+gp}{\PYGZgt{}\PYGZgt{}\PYGZgt{} }\PYG{k+kn}{import} \PYG{n+nn}{matplotlib.pyplot} \PYG{k+kn}{as} \PYG{n+nn}{plt}
\PYG{g+gp}{\PYGZgt{}\PYGZgt{}\PYGZgt{} }\PYG{n}{values} \PYG{o}{=} \PYG{n}{plt}\PYG{o}{.}\PYG{n}{hist}\PYG{p}{(}\PYG{n}{np}\PYG{o}{.}\PYG{n}{random}\PYG{o}{.}\PYG{n}{noncentral\PYGZus{}chisquare}\PYG{p}{(}\PYG{l+m+mi}{3}\PYG{p}{,} \PYG{l+m+mi}{20}\PYG{p}{,} \PYG{l+m+mi}{100000}\PYG{p}{)}\PYG{p}{,}
\PYG{g+gp}{... }                  \PYG{n}{bins}\PYG{o}{=}\PYG{l+m+mi}{200}\PYG{p}{,} \PYG{n}{normed}\PYG{o}{=}\PYG{n+nb+bp}{True}\PYG{p}{)}
\PYG{g+gp}{\PYGZgt{}\PYGZgt{}\PYGZgt{} }\PYG{n}{plt}\PYG{o}{.}\PYG{n}{show}\PYG{p}{(}\PYG{p}{)}
\end{Verbatim}

Draw values from a noncentral chisquare with very small noncentrality,
and compare to a chisquare.

\begin{Verbatim}[commandchars=\\\{\}]
\PYG{g+gp}{\PYGZgt{}\PYGZgt{}\PYGZgt{} }\PYG{n}{plt}\PYG{o}{.}\PYG{n}{figure}\PYG{p}{(}\PYG{p}{)}
\PYG{g+gp}{\PYGZgt{}\PYGZgt{}\PYGZgt{} }\PYG{n}{values} \PYG{o}{=} \PYG{n}{plt}\PYG{o}{.}\PYG{n}{hist}\PYG{p}{(}\PYG{n}{np}\PYG{o}{.}\PYG{n}{random}\PYG{o}{.}\PYG{n}{noncentral\PYGZus{}chisquare}\PYG{p}{(}\PYG{l+m+mi}{3}\PYG{p}{,} \PYG{o}{.}\PYG{l+m+mo}{0000001}\PYG{p}{,} \PYG{l+m+mi}{100000}\PYG{p}{)}\PYG{p}{,}
\PYG{g+gp}{... }                  \PYG{n}{bins}\PYG{o}{=}\PYG{n}{np}\PYG{o}{.}\PYG{n}{arange}\PYG{p}{(}\PYG{l+m+mf}{0.}\PYG{p}{,} \PYG{l+m+mi}{25}\PYG{p}{,} \PYG{o}{.}\PYG{l+m+mi}{1}\PYG{p}{)}\PYG{p}{,} \PYG{n}{normed}\PYG{o}{=}\PYG{n+nb+bp}{True}\PYG{p}{)}
\PYG{g+gp}{\PYGZgt{}\PYGZgt{}\PYGZgt{} }\PYG{n}{values2} \PYG{o}{=} \PYG{n}{plt}\PYG{o}{.}\PYG{n}{hist}\PYG{p}{(}\PYG{n}{np}\PYG{o}{.}\PYG{n}{random}\PYG{o}{.}\PYG{n}{chisquare}\PYG{p}{(}\PYG{l+m+mi}{3}\PYG{p}{,} \PYG{l+m+mi}{100000}\PYG{p}{)}\PYG{p}{,}
\PYG{g+gp}{... }                   \PYG{n}{bins}\PYG{o}{=}\PYG{n}{np}\PYG{o}{.}\PYG{n}{arange}\PYG{p}{(}\PYG{l+m+mf}{0.}\PYG{p}{,} \PYG{l+m+mi}{25}\PYG{p}{,} \PYG{o}{.}\PYG{l+m+mi}{1}\PYG{p}{)}\PYG{p}{,} \PYG{n}{normed}\PYG{o}{=}\PYG{n+nb+bp}{True}\PYG{p}{)}
\PYG{g+gp}{\PYGZgt{}\PYGZgt{}\PYGZgt{} }\PYG{n}{plt}\PYG{o}{.}\PYG{n}{plot}\PYG{p}{(}\PYG{n}{values}\PYG{p}{[}\PYG{l+m+mi}{1}\PYG{p}{]}\PYG{p}{[}\PYG{l+m+mi}{0}\PYG{p}{:}\PYG{o}{\PYGZhy{}}\PYG{l+m+mi}{1}\PYG{p}{]}\PYG{p}{,} \PYG{n}{values}\PYG{p}{[}\PYG{l+m+mi}{0}\PYG{p}{]}\PYG{o}{\PYGZhy{}}\PYG{n}{values2}\PYG{p}{[}\PYG{l+m+mi}{0}\PYG{p}{]}\PYG{p}{,} \PYG{l+s}{\PYGZsq{}}\PYG{l+s}{ob}\PYG{l+s}{\PYGZsq{}}\PYG{p}{)}
\PYG{g+gp}{\PYGZgt{}\PYGZgt{}\PYGZgt{} }\PYG{n}{plt}\PYG{o}{.}\PYG{n}{show}\PYG{p}{(}\PYG{p}{)}
\end{Verbatim}

Demonstrate how large values of non-centrality lead to a more symmetric
distribution.

\begin{Verbatim}[commandchars=\\\{\}]
\PYG{g+gp}{\PYGZgt{}\PYGZgt{}\PYGZgt{} }\PYG{n}{plt}\PYG{o}{.}\PYG{n}{figure}\PYG{p}{(}\PYG{p}{)}
\PYG{g+gp}{\PYGZgt{}\PYGZgt{}\PYGZgt{} }\PYG{n}{values} \PYG{o}{=} \PYG{n}{plt}\PYG{o}{.}\PYG{n}{hist}\PYG{p}{(}\PYG{n}{np}\PYG{o}{.}\PYG{n}{random}\PYG{o}{.}\PYG{n}{noncentral\PYGZus{}chisquare}\PYG{p}{(}\PYG{l+m+mi}{3}\PYG{p}{,} \PYG{l+m+mi}{20}\PYG{p}{,} \PYG{l+m+mi}{100000}\PYG{p}{)}\PYG{p}{,}
\PYG{g+gp}{... }                  \PYG{n}{bins}\PYG{o}{=}\PYG{l+m+mi}{200}\PYG{p}{,} \PYG{n}{normed}\PYG{o}{=}\PYG{n+nb+bp}{True}\PYG{p}{)}
\PYG{g+gp}{\PYGZgt{}\PYGZgt{}\PYGZgt{} }\PYG{n}{plt}\PYG{o}{.}\PYG{n}{show}\PYG{p}{(}\PYG{p}{)}
\end{Verbatim}

\end{fulllineitems}

\index{noncentral\_f() (in module lib.IO.writefiles)}

\begin{fulllineitems}
\phantomsection\label{lib.IO:lib.IO.writefiles.noncentral_f}\pysiglinewithargsret{\code{lib.IO.writefiles.}\bfcode{noncentral\_f}}{\emph{dfnum}, \emph{dfden}, \emph{nonc}, \emph{size=None}}{}
Draw samples from the noncentral F distribution.

Samples are drawn from an F distribution with specified parameters,
\emph{dfnum} (degrees of freedom in numerator) and \emph{dfden} (degrees of
freedom in denominator), where both parameters \textgreater{} 1.
\emph{nonc} is the non-centrality parameter.
\begin{description}
\item[{dfnum}] \leavevmode{[}int{]}
Parameter, should be \textgreater{} 1.

\item[{dfden}] \leavevmode{[}int{]}
Parameter, should be \textgreater{} 1.

\item[{nonc}] \leavevmode{[}float{]}
Parameter, should be \textgreater{}= 0.

\item[{size}] \leavevmode{[}int or tuple of ints{]}
Output shape. If the given shape is, e.g., \code{(m, n, k)}, then
\code{m * n * k} samples are drawn.

\end{description}
\begin{description}
\item[{samples}] \leavevmode{[}scalar or ndarray{]}
Drawn samples.

\end{description}

When calculating the power of an experiment (power = probability of
rejecting the null hypothesis when a specific alternative is true) the
non-central F statistic becomes important.  When the null hypothesis is
true, the F statistic follows a central F distribution. When the null
hypothesis is not true, then it follows a non-central F statistic.

Weisstein, Eric W. ``Noncentral F-Distribution.'' From MathWorld--A Wolfram
Web Resource.  \href{http://mathworld.wolfram.com/NoncentralF-Distribution.html}{http://mathworld.wolfram.com/NoncentralF-Distribution.html}

Wikipedia, ``Noncentral F distribution'',
\href{http://en.wikipedia.org/wiki/Noncentral\_F-distribution}{http://en.wikipedia.org/wiki/Noncentral\_F-distribution}

In a study, testing for a specific alternative to the null hypothesis
requires use of the Noncentral F distribution. We need to calculate the
area in the tail of the distribution that exceeds the value of the F
distribution for the null hypothesis.  We'll plot the two probability
distributions for comparison.

\begin{Verbatim}[commandchars=\\\{\}]
\PYG{g+gp}{\PYGZgt{}\PYGZgt{}\PYGZgt{} }\PYG{n}{dfnum} \PYG{o}{=} \PYG{l+m+mi}{3} \PYG{c}{\PYGZsh{} between group deg of freedom}
\PYG{g+gp}{\PYGZgt{}\PYGZgt{}\PYGZgt{} }\PYG{n}{dfden} \PYG{o}{=} \PYG{l+m+mi}{20} \PYG{c}{\PYGZsh{} within groups degrees of freedom}
\PYG{g+gp}{\PYGZgt{}\PYGZgt{}\PYGZgt{} }\PYG{n}{nonc} \PYG{o}{=} \PYG{l+m+mf}{3.0}
\PYG{g+gp}{\PYGZgt{}\PYGZgt{}\PYGZgt{} }\PYG{n}{nc\PYGZus{}vals} \PYG{o}{=} \PYG{n}{np}\PYG{o}{.}\PYG{n}{random}\PYG{o}{.}\PYG{n}{noncentral\PYGZus{}f}\PYG{p}{(}\PYG{n}{dfnum}\PYG{p}{,} \PYG{n}{dfden}\PYG{p}{,} \PYG{n}{nonc}\PYG{p}{,} \PYG{l+m+mi}{1000000}\PYG{p}{)}
\PYG{g+gp}{\PYGZgt{}\PYGZgt{}\PYGZgt{} }\PYG{n}{NF} \PYG{o}{=} \PYG{n}{np}\PYG{o}{.}\PYG{n}{histogram}\PYG{p}{(}\PYG{n}{nc\PYGZus{}vals}\PYG{p}{,} \PYG{n}{bins}\PYG{o}{=}\PYG{l+m+mi}{50}\PYG{p}{,} \PYG{n}{normed}\PYG{o}{=}\PYG{n+nb+bp}{True}\PYG{p}{)}
\PYG{g+gp}{\PYGZgt{}\PYGZgt{}\PYGZgt{} }\PYG{n}{c\PYGZus{}vals} \PYG{o}{=} \PYG{n}{np}\PYG{o}{.}\PYG{n}{random}\PYG{o}{.}\PYG{n}{f}\PYG{p}{(}\PYG{n}{dfnum}\PYG{p}{,} \PYG{n}{dfden}\PYG{p}{,} \PYG{l+m+mi}{1000000}\PYG{p}{)}
\PYG{g+gp}{\PYGZgt{}\PYGZgt{}\PYGZgt{} }\PYG{n}{F} \PYG{o}{=} \PYG{n}{np}\PYG{o}{.}\PYG{n}{histogram}\PYG{p}{(}\PYG{n}{c\PYGZus{}vals}\PYG{p}{,} \PYG{n}{bins}\PYG{o}{=}\PYG{l+m+mi}{50}\PYG{p}{,} \PYG{n}{normed}\PYG{o}{=}\PYG{n+nb+bp}{True}\PYG{p}{)}
\PYG{g+gp}{\PYGZgt{}\PYGZgt{}\PYGZgt{} }\PYG{n}{plt}\PYG{o}{.}\PYG{n}{plot}\PYG{p}{(}\PYG{n}{F}\PYG{p}{[}\PYG{l+m+mi}{1}\PYG{p}{]}\PYG{p}{[}\PYG{l+m+mi}{1}\PYG{p}{:}\PYG{p}{]}\PYG{p}{,} \PYG{n}{F}\PYG{p}{[}\PYG{l+m+mi}{0}\PYG{p}{]}\PYG{p}{)}
\PYG{g+gp}{\PYGZgt{}\PYGZgt{}\PYGZgt{} }\PYG{n}{plt}\PYG{o}{.}\PYG{n}{plot}\PYG{p}{(}\PYG{n}{NF}\PYG{p}{[}\PYG{l+m+mi}{1}\PYG{p}{]}\PYG{p}{[}\PYG{l+m+mi}{1}\PYG{p}{:}\PYG{p}{]}\PYG{p}{,} \PYG{n}{NF}\PYG{p}{[}\PYG{l+m+mi}{0}\PYG{p}{]}\PYG{p}{)}
\PYG{g+gp}{\PYGZgt{}\PYGZgt{}\PYGZgt{} }\PYG{n}{plt}\PYG{o}{.}\PYG{n}{show}\PYG{p}{(}\PYG{p}{)}
\end{Verbatim}

\end{fulllineitems}

\index{normal() (in module lib.IO.writefiles)}

\begin{fulllineitems}
\phantomsection\label{lib.IO:lib.IO.writefiles.normal}\pysiglinewithargsret{\code{lib.IO.writefiles.}\bfcode{normal}}{\emph{loc=0.0}, \emph{scale=1.0}, \emph{size=None}}{}
Draw random samples from a normal (Gaussian) distribution.

The probability density function of the normal distribution, first
derived by De Moivre and 200 years later by both Gauss and Laplace
independently {\color{red}\bfseries{}{[}2{]}\_}, is often called the bell curve because of
its characteristic shape (see the example below).

The normal distributions occurs often in nature.  For example, it
describes the commonly occurring distribution of samples influenced
by a large number of tiny, random disturbances, each with its own
unique distribution {\color{red}\bfseries{}{[}2{]}\_}.
\begin{description}
\item[{loc}] \leavevmode{[}float{]}
Mean (``centre'') of the distribution.

\item[{scale}] \leavevmode{[}float{]}
Standard deviation (spread or ``width'') of the distribution.

\item[{size}] \leavevmode{[}tuple of ints{]}
Output shape.  If the given shape is, e.g., \code{(m, n, k)}, then
\code{m * n * k} samples are drawn.

\end{description}
\begin{description}
\item[{scipy.stats.distributions.norm}] \leavevmode{[}probability density function,{]}
distribution or cumulative density function, etc.

\end{description}

The probability density for the Gaussian distribution is
\begin{gather}
\begin{split}p(x) = \frac{1}{\sqrt{ 2 \pi \sigma^2 }}
e^{ - \frac{ (x - \mu)^2 } {2 \sigma^2} },\end{split}\notag
\end{gather}
where \(\mu\) is the mean and \(\sigma\) the standard deviation.
The square of the standard deviation, \(\sigma^2\), is called the
variance.

The function has its peak at the mean, and its ``spread'' increases with
the standard deviation (the function reaches 0.607 times its maximum at
\(x + \sigma\) and \(x - \sigma\) {\color{red}\bfseries{}{[}2{]}\_}).  This implies that
\emph{numpy.random.normal} is more likely to return samples lying close to the
mean, rather than those far away.

Draw samples from the distribution:

\begin{Verbatim}[commandchars=\\\{\}]
\PYG{g+gp}{\PYGZgt{}\PYGZgt{}\PYGZgt{} }\PYG{n}{mu}\PYG{p}{,} \PYG{n}{sigma} \PYG{o}{=} \PYG{l+m+mi}{0}\PYG{p}{,} \PYG{l+m+mf}{0.1} \PYG{c}{\PYGZsh{} mean and standard deviation}
\PYG{g+gp}{\PYGZgt{}\PYGZgt{}\PYGZgt{} }\PYG{n}{s} \PYG{o}{=} \PYG{n}{np}\PYG{o}{.}\PYG{n}{random}\PYG{o}{.}\PYG{n}{normal}\PYG{p}{(}\PYG{n}{mu}\PYG{p}{,} \PYG{n}{sigma}\PYG{p}{,} \PYG{l+m+mi}{1000}\PYG{p}{)}
\end{Verbatim}

Verify the mean and the variance:

\begin{Verbatim}[commandchars=\\\{\}]
\PYG{g+gp}{\PYGZgt{}\PYGZgt{}\PYGZgt{} }\PYG{n+nb}{abs}\PYG{p}{(}\PYG{n}{mu} \PYG{o}{\PYGZhy{}} \PYG{n}{np}\PYG{o}{.}\PYG{n}{mean}\PYG{p}{(}\PYG{n}{s}\PYG{p}{)}\PYG{p}{)} \PYG{o}{\PYGZlt{}} \PYG{l+m+mf}{0.01}
\PYG{g+go}{True}
\end{Verbatim}

\begin{Verbatim}[commandchars=\\\{\}]
\PYG{g+gp}{\PYGZgt{}\PYGZgt{}\PYGZgt{} }\PYG{n+nb}{abs}\PYG{p}{(}\PYG{n}{sigma} \PYG{o}{\PYGZhy{}} \PYG{n}{np}\PYG{o}{.}\PYG{n}{std}\PYG{p}{(}\PYG{n}{s}\PYG{p}{,} \PYG{n}{ddof}\PYG{o}{=}\PYG{l+m+mi}{1}\PYG{p}{)}\PYG{p}{)} \PYG{o}{\PYGZlt{}} \PYG{l+m+mf}{0.01}
\PYG{g+go}{True}
\end{Verbatim}

Display the histogram of the samples, along with
the probability density function:

\begin{Verbatim}[commandchars=\\\{\}]
\PYG{g+gp}{\PYGZgt{}\PYGZgt{}\PYGZgt{} }\PYG{k+kn}{import} \PYG{n+nn}{matplotlib.pyplot} \PYG{k+kn}{as} \PYG{n+nn}{plt}
\PYG{g+gp}{\PYGZgt{}\PYGZgt{}\PYGZgt{} }\PYG{n}{count}\PYG{p}{,} \PYG{n}{bins}\PYG{p}{,} \PYG{n}{ignored} \PYG{o}{=} \PYG{n}{plt}\PYG{o}{.}\PYG{n}{hist}\PYG{p}{(}\PYG{n}{s}\PYG{p}{,} \PYG{l+m+mi}{30}\PYG{p}{,} \PYG{n}{normed}\PYG{o}{=}\PYG{n+nb+bp}{True}\PYG{p}{)}
\PYG{g+gp}{\PYGZgt{}\PYGZgt{}\PYGZgt{} }\PYG{n}{plt}\PYG{o}{.}\PYG{n}{plot}\PYG{p}{(}\PYG{n}{bins}\PYG{p}{,} \PYG{l+m+mi}{1}\PYG{o}{/}\PYG{p}{(}\PYG{n}{sigma} \PYG{o}{*} \PYG{n}{np}\PYG{o}{.}\PYG{n}{sqrt}\PYG{p}{(}\PYG{l+m+mi}{2} \PYG{o}{*} \PYG{n}{np}\PYG{o}{.}\PYG{n}{pi}\PYG{p}{)}\PYG{p}{)} \PYG{o}{*}
\PYG{g+gp}{... }               \PYG{n}{np}\PYG{o}{.}\PYG{n}{exp}\PYG{p}{(} \PYG{o}{\PYGZhy{}} \PYG{p}{(}\PYG{n}{bins} \PYG{o}{\PYGZhy{}} \PYG{n}{mu}\PYG{p}{)}\PYG{o}{*}\PYG{o}{*}\PYG{l+m+mi}{2} \PYG{o}{/} \PYG{p}{(}\PYG{l+m+mi}{2} \PYG{o}{*} \PYG{n}{sigma}\PYG{o}{*}\PYG{o}{*}\PYG{l+m+mi}{2}\PYG{p}{)} \PYG{p}{)}\PYG{p}{,}
\PYG{g+gp}{... }         \PYG{n}{linewidth}\PYG{o}{=}\PYG{l+m+mi}{2}\PYG{p}{,} \PYG{n}{color}\PYG{o}{=}\PYG{l+s}{\PYGZsq{}}\PYG{l+s}{r}\PYG{l+s}{\PYGZsq{}}\PYG{p}{)}
\PYG{g+gp}{\PYGZgt{}\PYGZgt{}\PYGZgt{} }\PYG{n}{plt}\PYG{o}{.}\PYG{n}{show}\PYG{p}{(}\PYG{p}{)}
\end{Verbatim}

\end{fulllineitems}

\index{pareto() (in module lib.IO.writefiles)}

\begin{fulllineitems}
\phantomsection\label{lib.IO:lib.IO.writefiles.pareto}\pysiglinewithargsret{\code{lib.IO.writefiles.}\bfcode{pareto}}{\emph{a}, \emph{size=None}}{}
Draw samples from a Pareto II or Lomax distribution with specified shape.

The Lomax or Pareto II distribution is a shifted Pareto distribution. The
classical Pareto distribution can be obtained from the Lomax distribution
by adding the location parameter m, see below. The smallest value of the
Lomax distribution is zero while for the classical Pareto distribution it
is m, where the standard Pareto distribution has location m=1.
Lomax can also be considered as a simplified version of the Generalized
Pareto distribution (available in SciPy), with the scale set to one and
the location set to zero.

The Pareto distribution must be greater than zero, and is unbounded above.
It is also known as the ``80-20 rule''.  In this distribution, 80 percent of
the weights are in the lowest 20 percent of the range, while the other 20
percent fill the remaining 80 percent of the range.
\begin{description}
\item[{shape}] \leavevmode{[}float, \textgreater{} 0.{]}
Shape of the distribution.

\item[{size}] \leavevmode{[}tuple of ints{]}
Output shape.  If the given shape is, e.g., \code{(m, n, k)}, then
\code{m * n * k} samples are drawn.

\end{description}
\begin{description}
\item[{scipy.stats.distributions.lomax.pdf}] \leavevmode{[}probability density function,{]}
distribution or cumulative density function, etc.

\item[{scipy.stats.distributions.genpareto.pdf}] \leavevmode{[}probability density function,{]}
distribution or cumulative density function, etc.

\end{description}

The probability density for the Pareto distribution is
\begin{gather}
\begin{split}p(x) = \frac{am^a}{x^{a+1}}\end{split}\notag
\end{gather}
where \(a\) is the shape and \(m\) the location

The Pareto distribution, named after the Italian economist Vilfredo Pareto,
is a power law probability distribution useful in many real world problems.
Outside the field of economics it is generally referred to as the Bradford
distribution. Pareto developed the distribution to describe the
distribution of wealth in an economy.  It has also found use in insurance,
web page access statistics, oil field sizes, and many other problems,
including the download frequency for projects in Sourceforge {[}1{]}.  It is
one of the so-called ``fat-tailed'' distributions.

Draw samples from the distribution:

\begin{Verbatim}[commandchars=\\\{\}]
\PYG{g+gp}{\PYGZgt{}\PYGZgt{}\PYGZgt{} }\PYG{n}{a}\PYG{p}{,} \PYG{n}{m} \PYG{o}{=} \PYG{l+m+mf}{3.}\PYG{p}{,} \PYG{l+m+mf}{1.} \PYG{c}{\PYGZsh{} shape and mode}
\PYG{g+gp}{\PYGZgt{}\PYGZgt{}\PYGZgt{} }\PYG{n}{s} \PYG{o}{=} \PYG{n}{np}\PYG{o}{.}\PYG{n}{random}\PYG{o}{.}\PYG{n}{pareto}\PYG{p}{(}\PYG{n}{a}\PYG{p}{,} \PYG{l+m+mi}{1000}\PYG{p}{)} \PYG{o}{+} \PYG{n}{m}
\end{Verbatim}

Display the histogram of the samples, along with
the probability density function:

\begin{Verbatim}[commandchars=\\\{\}]
\PYG{g+gp}{\PYGZgt{}\PYGZgt{}\PYGZgt{} }\PYG{k+kn}{import} \PYG{n+nn}{matplotlib.pyplot} \PYG{k+kn}{as} \PYG{n+nn}{plt}
\PYG{g+gp}{\PYGZgt{}\PYGZgt{}\PYGZgt{} }\PYG{n}{count}\PYG{p}{,} \PYG{n}{bins}\PYG{p}{,} \PYG{n}{ignored} \PYG{o}{=} \PYG{n}{plt}\PYG{o}{.}\PYG{n}{hist}\PYG{p}{(}\PYG{n}{s}\PYG{p}{,} \PYG{l+m+mi}{100}\PYG{p}{,} \PYG{n}{normed}\PYG{o}{=}\PYG{n+nb+bp}{True}\PYG{p}{,} \PYG{n}{align}\PYG{o}{=}\PYG{l+s}{\PYGZsq{}}\PYG{l+s}{center}\PYG{l+s}{\PYGZsq{}}\PYG{p}{)}
\PYG{g+gp}{\PYGZgt{}\PYGZgt{}\PYGZgt{} }\PYG{n}{fit} \PYG{o}{=} \PYG{n}{a}\PYG{o}{*}\PYG{n}{m}\PYG{o}{*}\PYG{o}{*}\PYG{n}{a}\PYG{o}{/}\PYG{n}{bins}\PYG{o}{*}\PYG{o}{*}\PYG{p}{(}\PYG{n}{a}\PYG{o}{+}\PYG{l+m+mi}{1}\PYG{p}{)}
\PYG{g+gp}{\PYGZgt{}\PYGZgt{}\PYGZgt{} }\PYG{n}{plt}\PYG{o}{.}\PYG{n}{plot}\PYG{p}{(}\PYG{n}{bins}\PYG{p}{,} \PYG{n+nb}{max}\PYG{p}{(}\PYG{n}{count}\PYG{p}{)}\PYG{o}{*}\PYG{n}{fit}\PYG{o}{/}\PYG{n+nb}{max}\PYG{p}{(}\PYG{n}{fit}\PYG{p}{)}\PYG{p}{,}\PYG{n}{linewidth}\PYG{o}{=}\PYG{l+m+mi}{2}\PYG{p}{,} \PYG{n}{color}\PYG{o}{=}\PYG{l+s}{\PYGZsq{}}\PYG{l+s}{r}\PYG{l+s}{\PYGZsq{}}\PYG{p}{)}
\PYG{g+gp}{\PYGZgt{}\PYGZgt{}\PYGZgt{} }\PYG{n}{plt}\PYG{o}{.}\PYG{n}{show}\PYG{p}{(}\PYG{p}{)}
\end{Verbatim}

\end{fulllineitems}

\index{permutation() (in module lib.IO.writefiles)}

\begin{fulllineitems}
\phantomsection\label{lib.IO:lib.IO.writefiles.permutation}\pysiglinewithargsret{\code{lib.IO.writefiles.}\bfcode{permutation}}{\emph{x}}{}
Randomly permute a sequence, or return a permuted range.

If \emph{x} is a multi-dimensional array, it is only shuffled along its
first index.
\begin{description}
\item[{x}] \leavevmode{[}int or array\_like{]}
If \emph{x} is an integer, randomly permute \code{np.arange(x)}.
If \emph{x} is an array, make a copy and shuffle the elements
randomly.

\end{description}
\begin{description}
\item[{out}] \leavevmode{[}ndarray{]}
Permuted sequence or array range.

\end{description}

\begin{Verbatim}[commandchars=\\\{\}]
\PYG{g+gp}{\PYGZgt{}\PYGZgt{}\PYGZgt{} }\PYG{n}{np}\PYG{o}{.}\PYG{n}{random}\PYG{o}{.}\PYG{n}{permutation}\PYG{p}{(}\PYG{l+m+mi}{10}\PYG{p}{)}
\PYG{g+go}{array([1, 7, 4, 3, 0, 9, 2, 5, 8, 6])}
\end{Verbatim}

\begin{Verbatim}[commandchars=\\\{\}]
\PYG{g+gp}{\PYGZgt{}\PYGZgt{}\PYGZgt{} }\PYG{n}{np}\PYG{o}{.}\PYG{n}{random}\PYG{o}{.}\PYG{n}{permutation}\PYG{p}{(}\PYG{p}{[}\PYG{l+m+mi}{1}\PYG{p}{,} \PYG{l+m+mi}{4}\PYG{p}{,} \PYG{l+m+mi}{9}\PYG{p}{,} \PYG{l+m+mi}{12}\PYG{p}{,} \PYG{l+m+mi}{15}\PYG{p}{]}\PYG{p}{)}
\PYG{g+go}{array([15,  1,  9,  4, 12])}
\end{Verbatim}

\begin{Verbatim}[commandchars=\\\{\}]
\PYG{g+gp}{\PYGZgt{}\PYGZgt{}\PYGZgt{} }\PYG{n}{arr} \PYG{o}{=} \PYG{n}{np}\PYG{o}{.}\PYG{n}{arange}\PYG{p}{(}\PYG{l+m+mi}{9}\PYG{p}{)}\PYG{o}{.}\PYG{n}{reshape}\PYG{p}{(}\PYG{p}{(}\PYG{l+m+mi}{3}\PYG{p}{,} \PYG{l+m+mi}{3}\PYG{p}{)}\PYG{p}{)}
\PYG{g+gp}{\PYGZgt{}\PYGZgt{}\PYGZgt{} }\PYG{n}{np}\PYG{o}{.}\PYG{n}{random}\PYG{o}{.}\PYG{n}{permutation}\PYG{p}{(}\PYG{n}{arr}\PYG{p}{)}
\PYG{g+go}{array([[6, 7, 8],}
\PYG{g+go}{       [0, 1, 2],}
\PYG{g+go}{       [3, 4, 5]])}
\end{Verbatim}

\end{fulllineitems}

\index{poisson() (in module lib.IO.writefiles)}

\begin{fulllineitems}
\phantomsection\label{lib.IO:lib.IO.writefiles.poisson}\pysiglinewithargsret{\code{lib.IO.writefiles.}\bfcode{poisson}}{\emph{lam=1.0}, \emph{size=None}}{}
Draw samples from a Poisson distribution.

The Poisson distribution is the limit of the Binomial
distribution for large N.
\begin{description}
\item[{lam}] \leavevmode{[}float{]}
Expectation of interval, should be \textgreater{}= 0.

\item[{size}] \leavevmode{[}int or tuple of ints, optional{]}
Output shape. If the given shape is, e.g., \code{(m, n, k)}, then
\code{m * n * k} samples are drawn.

\end{description}

The Poisson distribution
\begin{gather}
\begin{split}f(k; \lambda)=\frac{\lambda^k e^{-\lambda}}{k!}\end{split}\notag
\end{gather}
For events with an expected separation \(\lambda\) the Poisson
distribution \(f(k; \lambda)\) describes the probability of
\(k\) events occurring within the observed interval \(\lambda\).

Because the output is limited to the range of the C long type, a
ValueError is raised when \emph{lam} is within 10 sigma of the maximum
representable value.

Draw samples from the distribution:

\begin{Verbatim}[commandchars=\\\{\}]
\PYG{g+gp}{\PYGZgt{}\PYGZgt{}\PYGZgt{} }\PYG{k+kn}{import} \PYG{n+nn}{numpy} \PYG{k+kn}{as} \PYG{n+nn}{np}
\PYG{g+gp}{\PYGZgt{}\PYGZgt{}\PYGZgt{} }\PYG{n}{s} \PYG{o}{=} \PYG{n}{np}\PYG{o}{.}\PYG{n}{random}\PYG{o}{.}\PYG{n}{poisson}\PYG{p}{(}\PYG{l+m+mi}{5}\PYG{p}{,} \PYG{l+m+mi}{10000}\PYG{p}{)}
\end{Verbatim}

Display histogram of the sample:

\begin{Verbatim}[commandchars=\\\{\}]
\PYG{g+gp}{\PYGZgt{}\PYGZgt{}\PYGZgt{} }\PYG{k+kn}{import} \PYG{n+nn}{matplotlib.pyplot} \PYG{k+kn}{as} \PYG{n+nn}{plt}
\PYG{g+gp}{\PYGZgt{}\PYGZgt{}\PYGZgt{} }\PYG{n}{count}\PYG{p}{,} \PYG{n}{bins}\PYG{p}{,} \PYG{n}{ignored} \PYG{o}{=} \PYG{n}{plt}\PYG{o}{.}\PYG{n}{hist}\PYG{p}{(}\PYG{n}{s}\PYG{p}{,} \PYG{l+m+mi}{14}\PYG{p}{,} \PYG{n}{normed}\PYG{o}{=}\PYG{n+nb+bp}{True}\PYG{p}{)}
\PYG{g+gp}{\PYGZgt{}\PYGZgt{}\PYGZgt{} }\PYG{n}{plt}\PYG{o}{.}\PYG{n}{show}\PYG{p}{(}\PYG{p}{)}
\end{Verbatim}

\end{fulllineitems}

\index{power() (in module lib.IO.writefiles)}

\begin{fulllineitems}
\phantomsection\label{lib.IO:lib.IO.writefiles.power}\pysiglinewithargsret{\code{lib.IO.writefiles.}\bfcode{power}}{\emph{a}, \emph{size=None}}{}
Draws samples in {[}0, 1{]} from a power distribution with positive
exponent a - 1.

Also known as the power function distribution.
\begin{description}
\item[{a}] \leavevmode{[}float{]}
parameter, \textgreater{} 0

\item[{size}] \leavevmode{[}tuple of ints{]}\begin{description}
\item[{Output shape.  If the given shape is, e.g., \code{(m, n, k)}, then}] \leavevmode
\code{m * n * k} samples are drawn.

\end{description}

\end{description}
\begin{description}
\item[{samples}] \leavevmode{[}\{ndarray, scalar\}{]}
The returned samples lie in {[}0, 1{]}.

\end{description}
\begin{description}
\item[{ValueError}] \leavevmode
If a\textless{}1.

\end{description}

The probability density function is
\begin{gather}
\begin{split}P(x; a) = ax^{a-1}, 0 \le x \le 1, a>0.\end{split}\notag
\end{gather}
The power function distribution is just the inverse of the Pareto
distribution. It may also be seen as a special case of the Beta
distribution.

It is used, for example, in modeling the over-reporting of insurance
claims.

Draw samples from the distribution:

\begin{Verbatim}[commandchars=\\\{\}]
\PYG{g+gp}{\PYGZgt{}\PYGZgt{}\PYGZgt{} }\PYG{n}{a} \PYG{o}{=} \PYG{l+m+mf}{5.} \PYG{c}{\PYGZsh{} shape}
\PYG{g+gp}{\PYGZgt{}\PYGZgt{}\PYGZgt{} }\PYG{n}{samples} \PYG{o}{=} \PYG{l+m+mi}{1000}
\PYG{g+gp}{\PYGZgt{}\PYGZgt{}\PYGZgt{} }\PYG{n}{s} \PYG{o}{=} \PYG{n}{np}\PYG{o}{.}\PYG{n}{random}\PYG{o}{.}\PYG{n}{power}\PYG{p}{(}\PYG{n}{a}\PYG{p}{,} \PYG{n}{samples}\PYG{p}{)}
\end{Verbatim}

Display the histogram of the samples, along with
the probability density function:

\begin{Verbatim}[commandchars=\\\{\}]
\PYG{g+gp}{\PYGZgt{}\PYGZgt{}\PYGZgt{} }\PYG{k+kn}{import} \PYG{n+nn}{matplotlib.pyplot} \PYG{k+kn}{as} \PYG{n+nn}{plt}
\PYG{g+gp}{\PYGZgt{}\PYGZgt{}\PYGZgt{} }\PYG{n}{count}\PYG{p}{,} \PYG{n}{bins}\PYG{p}{,} \PYG{n}{ignored} \PYG{o}{=} \PYG{n}{plt}\PYG{o}{.}\PYG{n}{hist}\PYG{p}{(}\PYG{n}{s}\PYG{p}{,} \PYG{n}{bins}\PYG{o}{=}\PYG{l+m+mi}{30}\PYG{p}{)}
\PYG{g+gp}{\PYGZgt{}\PYGZgt{}\PYGZgt{} }\PYG{n}{x} \PYG{o}{=} \PYG{n}{np}\PYG{o}{.}\PYG{n}{linspace}\PYG{p}{(}\PYG{l+m+mi}{0}\PYG{p}{,} \PYG{l+m+mi}{1}\PYG{p}{,} \PYG{l+m+mi}{100}\PYG{p}{)}
\PYG{g+gp}{\PYGZgt{}\PYGZgt{}\PYGZgt{} }\PYG{n}{y} \PYG{o}{=} \PYG{n}{a}\PYG{o}{*}\PYG{n}{x}\PYG{o}{*}\PYG{o}{*}\PYG{p}{(}\PYG{n}{a}\PYG{o}{\PYGZhy{}}\PYG{l+m+mf}{1.}\PYG{p}{)}
\PYG{g+gp}{\PYGZgt{}\PYGZgt{}\PYGZgt{} }\PYG{n}{normed\PYGZus{}y} \PYG{o}{=} \PYG{n}{samples}\PYG{o}{*}\PYG{n}{np}\PYG{o}{.}\PYG{n}{diff}\PYG{p}{(}\PYG{n}{bins}\PYG{p}{)}\PYG{p}{[}\PYG{l+m+mi}{0}\PYG{p}{]}\PYG{o}{*}\PYG{n}{y}
\PYG{g+gp}{\PYGZgt{}\PYGZgt{}\PYGZgt{} }\PYG{n}{plt}\PYG{o}{.}\PYG{n}{plot}\PYG{p}{(}\PYG{n}{x}\PYG{p}{,} \PYG{n}{normed\PYGZus{}y}\PYG{p}{)}
\PYG{g+gp}{\PYGZgt{}\PYGZgt{}\PYGZgt{} }\PYG{n}{plt}\PYG{o}{.}\PYG{n}{show}\PYG{p}{(}\PYG{p}{)}
\end{Verbatim}

Compare the power function distribution to the inverse of the Pareto.

\begin{Verbatim}[commandchars=\\\{\}]
\PYG{g+gp}{\PYGZgt{}\PYGZgt{}\PYGZgt{} }\PYG{k+kn}{from} \PYG{n+nn}{scipy} \PYG{k+kn}{import} \PYG{n}{stats}
\PYG{g+gp}{\PYGZgt{}\PYGZgt{}\PYGZgt{} }\PYG{n}{rvs} \PYG{o}{=} \PYG{n}{np}\PYG{o}{.}\PYG{n}{random}\PYG{o}{.}\PYG{n}{power}\PYG{p}{(}\PYG{l+m+mi}{5}\PYG{p}{,} \PYG{l+m+mi}{1000000}\PYG{p}{)}
\PYG{g+gp}{\PYGZgt{}\PYGZgt{}\PYGZgt{} }\PYG{n}{rvsp} \PYG{o}{=} \PYG{n}{np}\PYG{o}{.}\PYG{n}{random}\PYG{o}{.}\PYG{n}{pareto}\PYG{p}{(}\PYG{l+m+mi}{5}\PYG{p}{,} \PYG{l+m+mi}{1000000}\PYG{p}{)}
\PYG{g+gp}{\PYGZgt{}\PYGZgt{}\PYGZgt{} }\PYG{n}{xx} \PYG{o}{=} \PYG{n}{np}\PYG{o}{.}\PYG{n}{linspace}\PYG{p}{(}\PYG{l+m+mi}{0}\PYG{p}{,}\PYG{l+m+mi}{1}\PYG{p}{,}\PYG{l+m+mi}{100}\PYG{p}{)}
\PYG{g+gp}{\PYGZgt{}\PYGZgt{}\PYGZgt{} }\PYG{n}{powpdf} \PYG{o}{=} \PYG{n}{stats}\PYG{o}{.}\PYG{n}{powerlaw}\PYG{o}{.}\PYG{n}{pdf}\PYG{p}{(}\PYG{n}{xx}\PYG{p}{,}\PYG{l+m+mi}{5}\PYG{p}{)}
\end{Verbatim}

\begin{Verbatim}[commandchars=\\\{\}]
\PYG{g+gp}{\PYGZgt{}\PYGZgt{}\PYGZgt{} }\PYG{n}{plt}\PYG{o}{.}\PYG{n}{figure}\PYG{p}{(}\PYG{p}{)}
\PYG{g+gp}{\PYGZgt{}\PYGZgt{}\PYGZgt{} }\PYG{n}{plt}\PYG{o}{.}\PYG{n}{hist}\PYG{p}{(}\PYG{n}{rvs}\PYG{p}{,} \PYG{n}{bins}\PYG{o}{=}\PYG{l+m+mi}{50}\PYG{p}{,} \PYG{n}{normed}\PYG{o}{=}\PYG{n+nb+bp}{True}\PYG{p}{)}
\PYG{g+gp}{\PYGZgt{}\PYGZgt{}\PYGZgt{} }\PYG{n}{plt}\PYG{o}{.}\PYG{n}{plot}\PYG{p}{(}\PYG{n}{xx}\PYG{p}{,}\PYG{n}{powpdf}\PYG{p}{,}\PYG{l+s}{\PYGZsq{}}\PYG{l+s}{r\PYGZhy{}}\PYG{l+s}{\PYGZsq{}}\PYG{p}{)}
\PYG{g+gp}{\PYGZgt{}\PYGZgt{}\PYGZgt{} }\PYG{n}{plt}\PYG{o}{.}\PYG{n}{title}\PYG{p}{(}\PYG{l+s}{\PYGZsq{}}\PYG{l+s}{np.random.power(5)}\PYG{l+s}{\PYGZsq{}}\PYG{p}{)}
\end{Verbatim}

\begin{Verbatim}[commandchars=\\\{\}]
\PYG{g+gp}{\PYGZgt{}\PYGZgt{}\PYGZgt{} }\PYG{n}{plt}\PYG{o}{.}\PYG{n}{figure}\PYG{p}{(}\PYG{p}{)}
\PYG{g+gp}{\PYGZgt{}\PYGZgt{}\PYGZgt{} }\PYG{n}{plt}\PYG{o}{.}\PYG{n}{hist}\PYG{p}{(}\PYG{l+m+mf}{1.}\PYG{o}{/}\PYG{p}{(}\PYG{l+m+mf}{1.}\PYG{o}{+}\PYG{n}{rvsp}\PYG{p}{)}\PYG{p}{,} \PYG{n}{bins}\PYG{o}{=}\PYG{l+m+mi}{50}\PYG{p}{,} \PYG{n}{normed}\PYG{o}{=}\PYG{n+nb+bp}{True}\PYG{p}{)}
\PYG{g+gp}{\PYGZgt{}\PYGZgt{}\PYGZgt{} }\PYG{n}{plt}\PYG{o}{.}\PYG{n}{plot}\PYG{p}{(}\PYG{n}{xx}\PYG{p}{,}\PYG{n}{powpdf}\PYG{p}{,}\PYG{l+s}{\PYGZsq{}}\PYG{l+s}{r\PYGZhy{}}\PYG{l+s}{\PYGZsq{}}\PYG{p}{)}
\PYG{g+gp}{\PYGZgt{}\PYGZgt{}\PYGZgt{} }\PYG{n}{plt}\PYG{o}{.}\PYG{n}{title}\PYG{p}{(}\PYG{l+s}{\PYGZsq{}}\PYG{l+s}{inverse of 1 + np.random.pareto(5)}\PYG{l+s}{\PYGZsq{}}\PYG{p}{)}
\end{Verbatim}

\begin{Verbatim}[commandchars=\\\{\}]
\PYG{g+gp}{\PYGZgt{}\PYGZgt{}\PYGZgt{} }\PYG{n}{plt}\PYG{o}{.}\PYG{n}{figure}\PYG{p}{(}\PYG{p}{)}
\PYG{g+gp}{\PYGZgt{}\PYGZgt{}\PYGZgt{} }\PYG{n}{plt}\PYG{o}{.}\PYG{n}{hist}\PYG{p}{(}\PYG{l+m+mf}{1.}\PYG{o}{/}\PYG{p}{(}\PYG{l+m+mf}{1.}\PYG{o}{+}\PYG{n}{rvsp}\PYG{p}{)}\PYG{p}{,} \PYG{n}{bins}\PYG{o}{=}\PYG{l+m+mi}{50}\PYG{p}{,} \PYG{n}{normed}\PYG{o}{=}\PYG{n+nb+bp}{True}\PYG{p}{)}
\PYG{g+gp}{\PYGZgt{}\PYGZgt{}\PYGZgt{} }\PYG{n}{plt}\PYG{o}{.}\PYG{n}{plot}\PYG{p}{(}\PYG{n}{xx}\PYG{p}{,}\PYG{n}{powpdf}\PYG{p}{,}\PYG{l+s}{\PYGZsq{}}\PYG{l+s}{r\PYGZhy{}}\PYG{l+s}{\PYGZsq{}}\PYG{p}{)}
\PYG{g+gp}{\PYGZgt{}\PYGZgt{}\PYGZgt{} }\PYG{n}{plt}\PYG{o}{.}\PYG{n}{title}\PYG{p}{(}\PYG{l+s}{\PYGZsq{}}\PYG{l+s}{inverse of stats.pareto(5)}\PYG{l+s}{\PYGZsq{}}\PYG{p}{)}
\end{Verbatim}

\end{fulllineitems}

\index{rand() (in module lib.IO.writefiles)}

\begin{fulllineitems}
\phantomsection\label{lib.IO:lib.IO.writefiles.rand}\pysiglinewithargsret{\code{lib.IO.writefiles.}\bfcode{rand}}{\emph{d0}, \emph{d1}, \emph{...}, \emph{dn}}{}
Random values in a given shape.

Create an array of the given shape and propagate it with
random samples from a uniform distribution
over \code{{[}0, 1)}.
\begin{description}
\item[{d0, d1, ..., dn}] \leavevmode{[}int, optional{]}
The dimensions of the returned array, should all be positive.
If no argument is given a single Python float is returned.

\end{description}
\begin{description}
\item[{out}] \leavevmode{[}ndarray, shape \code{(d0, d1, ..., dn)}{]}
Random values.

\end{description}

random

This is a convenience function. If you want an interface that
takes a shape-tuple as the first argument, refer to
np.random.random\_sample .

\begin{Verbatim}[commandchars=\\\{\}]
\PYG{g+gp}{\PYGZgt{}\PYGZgt{}\PYGZgt{} }\PYG{n}{np}\PYG{o}{.}\PYG{n}{random}\PYG{o}{.}\PYG{n}{rand}\PYG{p}{(}\PYG{l+m+mi}{3}\PYG{p}{,}\PYG{l+m+mi}{2}\PYG{p}{)}
\PYG{g+go}{array([[ 0.14022471,  0.96360618],  \PYGZsh{}random}
\PYG{g+go}{       [ 0.37601032,  0.25528411],  \PYGZsh{}random}
\PYG{g+go}{       [ 0.49313049,  0.94909878]]) \PYGZsh{}random}
\end{Verbatim}

\end{fulllineitems}

\index{randint() (in module lib.IO.writefiles)}

\begin{fulllineitems}
\phantomsection\label{lib.IO:lib.IO.writefiles.randint}\pysiglinewithargsret{\code{lib.IO.writefiles.}\bfcode{randint}}{\emph{low}, \emph{high=None}, \emph{size=None}}{}
Return random integers from \emph{low} (inclusive) to \emph{high} (exclusive).

Return random integers from the ``discrete uniform'' distribution in the
``half-open'' interval {[}\emph{low}, \emph{high}). If \emph{high} is None (the default),
then results are from {[}0, \emph{low}).
\begin{description}
\item[{low}] \leavevmode{[}int{]}
Lowest (signed) integer to be drawn from the distribution (unless
\code{high=None}, in which case this parameter is the \emph{highest} such
integer).

\item[{high}] \leavevmode{[}int, optional{]}
If provided, one above the largest (signed) integer to be drawn
from the distribution (see above for behavior if \code{high=None}).

\item[{size}] \leavevmode{[}int or tuple of ints, optional{]}
Output shape. Default is None, in which case a single int is
returned.

\end{description}
\begin{description}
\item[{out}] \leavevmode{[}int or ndarray of ints{]}
\emph{size}-shaped array of random integers from the appropriate
distribution, or a single such random int if \emph{size} not provided.

\end{description}
\begin{description}
\item[{random.random\_integers}] \leavevmode{[}similar to \emph{randint}, only for the closed{]}
interval {[}\emph{low}, \emph{high}{]}, and 1 is the lowest value if \emph{high} is
omitted. In particular, this other one is the one to use to generate
uniformly distributed discrete non-integers.

\end{description}

\begin{Verbatim}[commandchars=\\\{\}]
\PYG{g+gp}{\PYGZgt{}\PYGZgt{}\PYGZgt{} }\PYG{n}{np}\PYG{o}{.}\PYG{n}{random}\PYG{o}{.}\PYG{n}{randint}\PYG{p}{(}\PYG{l+m+mi}{2}\PYG{p}{,} \PYG{n}{size}\PYG{o}{=}\PYG{l+m+mi}{10}\PYG{p}{)}
\PYG{g+go}{array([1, 0, 0, 0, 1, 1, 0, 0, 1, 0])}
\PYG{g+gp}{\PYGZgt{}\PYGZgt{}\PYGZgt{} }\PYG{n}{np}\PYG{o}{.}\PYG{n}{random}\PYG{o}{.}\PYG{n}{randint}\PYG{p}{(}\PYG{l+m+mi}{1}\PYG{p}{,} \PYG{n}{size}\PYG{o}{=}\PYG{l+m+mi}{10}\PYG{p}{)}
\PYG{g+go}{array([0, 0, 0, 0, 0, 0, 0, 0, 0, 0])}
\end{Verbatim}

Generate a 2 x 4 array of ints between 0 and 4, inclusive:

\begin{Verbatim}[commandchars=\\\{\}]
\PYG{g+gp}{\PYGZgt{}\PYGZgt{}\PYGZgt{} }\PYG{n}{np}\PYG{o}{.}\PYG{n}{random}\PYG{o}{.}\PYG{n}{randint}\PYG{p}{(}\PYG{l+m+mi}{5}\PYG{p}{,} \PYG{n}{size}\PYG{o}{=}\PYG{p}{(}\PYG{l+m+mi}{2}\PYG{p}{,} \PYG{l+m+mi}{4}\PYG{p}{)}\PYG{p}{)}
\PYG{g+go}{array([[4, 0, 2, 1],}
\PYG{g+go}{       [3, 2, 2, 0]])}
\end{Verbatim}

\end{fulllineitems}

\index{randn() (in module lib.IO.writefiles)}

\begin{fulllineitems}
\phantomsection\label{lib.IO:lib.IO.writefiles.randn}\pysiglinewithargsret{\code{lib.IO.writefiles.}\bfcode{randn}}{\emph{d0}, \emph{d1}, \emph{...}, \emph{dn}}{}
Return a sample (or samples) from the ``standard normal'' distribution.

If positive, int\_like or int-convertible arguments are provided,
\emph{randn} generates an array of shape \code{(d0, d1, ..., dn)}, filled
with random floats sampled from a univariate ``normal'' (Gaussian)
distribution of mean 0 and variance 1 (if any of the \(d_i\) are
floats, they are first converted to integers by truncation). A single
float randomly sampled from the distribution is returned if no
argument is provided.

This is a convenience function.  If you want an interface that takes a
tuple as the first argument, use \emph{numpy.random.standard\_normal} instead.
\begin{description}
\item[{d0, d1, ..., dn}] \leavevmode{[}int, optional{]}
The dimensions of the returned array, should be all positive.
If no argument is given a single Python float is returned.

\end{description}
\begin{description}
\item[{Z}] \leavevmode{[}ndarray or float{]}
A \code{(d0, d1, ..., dn)}-shaped array of floating-point samples from
the standard normal distribution, or a single such float if
no parameters were supplied.

\end{description}

random.standard\_normal : Similar, but takes a tuple as its argument.

For random samples from \(N(\mu, \sigma^2)\), use:

\code{sigma * np.random.randn(...) + mu}

\begin{Verbatim}[commandchars=\\\{\}]
\PYG{g+gp}{\PYGZgt{}\PYGZgt{}\PYGZgt{} }\PYG{n}{np}\PYG{o}{.}\PYG{n}{random}\PYG{o}{.}\PYG{n}{randn}\PYG{p}{(}\PYG{p}{)}
\PYG{g+go}{2.1923875335537315 \PYGZsh{}random}
\end{Verbatim}

Two-by-four array of samples from N(3, 6.25):

\begin{Verbatim}[commandchars=\\\{\}]
\PYG{g+gp}{\PYGZgt{}\PYGZgt{}\PYGZgt{} }\PYG{l+m+mf}{2.5} \PYG{o}{*} \PYG{n}{np}\PYG{o}{.}\PYG{n}{random}\PYG{o}{.}\PYG{n}{randn}\PYG{p}{(}\PYG{l+m+mi}{2}\PYG{p}{,} \PYG{l+m+mi}{4}\PYG{p}{)} \PYG{o}{+} \PYG{l+m+mi}{3}
\PYG{g+go}{array([[\PYGZhy{}4.49401501,  4.00950034, \PYGZhy{}1.81814867,  7.29718677],  \PYGZsh{}random}
\PYG{g+go}{       [ 0.39924804,  4.68456316,  4.99394529,  4.84057254]]) \PYGZsh{}random}
\end{Verbatim}

\end{fulllineitems}

\index{random() (in module lib.IO.writefiles)}

\begin{fulllineitems}
\phantomsection\label{lib.IO:lib.IO.writefiles.random}\pysiglinewithargsret{\code{lib.IO.writefiles.}\bfcode{random}}{}{}
random\_sample(size=None)

Return random floats in the half-open interval {[}0.0, 1.0).

Results are from the ``continuous uniform'' distribution over the
stated interval.  To sample \(Unif[a, b), b > a\) multiply
the output of \emph{random\_sample} by \emph{(b-a)} and add \emph{a}:

\begin{Verbatim}[commandchars=\\\{\}]
\PYG{p}{(}\PYG{n}{b} \PYG{o}{\PYGZhy{}} \PYG{n}{a}\PYG{p}{)} \PYG{o}{*} \PYG{n}{random\PYGZus{}sample}\PYG{p}{(}\PYG{p}{)} \PYG{o}{+} \PYG{n}{a}
\end{Verbatim}
\begin{description}
\item[{size}] \leavevmode{[}int or tuple of ints, optional{]}
Defines the shape of the returned array of random floats. If None
(the default), returns a single float.

\end{description}
\begin{description}
\item[{out}] \leavevmode{[}float or ndarray of floats{]}
Array of random floats of shape \emph{size} (unless \code{size=None}, in which
case a single float is returned).

\end{description}

\begin{Verbatim}[commandchars=\\\{\}]
\PYG{g+gp}{\PYGZgt{}\PYGZgt{}\PYGZgt{} }\PYG{n}{np}\PYG{o}{.}\PYG{n}{random}\PYG{o}{.}\PYG{n}{random\PYGZus{}sample}\PYG{p}{(}\PYG{p}{)}
\PYG{g+go}{0.47108547995356098}
\PYG{g+gp}{\PYGZgt{}\PYGZgt{}\PYGZgt{} }\PYG{n+nb}{type}\PYG{p}{(}\PYG{n}{np}\PYG{o}{.}\PYG{n}{random}\PYG{o}{.}\PYG{n}{random\PYGZus{}sample}\PYG{p}{(}\PYG{p}{)}\PYG{p}{)}
\PYG{g+go}{\PYGZlt{}type \PYGZsq{}float\PYGZsq{}\PYGZgt{}}
\PYG{g+gp}{\PYGZgt{}\PYGZgt{}\PYGZgt{} }\PYG{n}{np}\PYG{o}{.}\PYG{n}{random}\PYG{o}{.}\PYG{n}{random\PYGZus{}sample}\PYG{p}{(}\PYG{p}{(}\PYG{l+m+mi}{5}\PYG{p}{,}\PYG{p}{)}\PYG{p}{)}
\PYG{g+go}{array([ 0.30220482,  0.86820401,  0.1654503 ,  0.11659149,  0.54323428])}
\end{Verbatim}

Three-by-two array of random numbers from {[}-5, 0):

\begin{Verbatim}[commandchars=\\\{\}]
\PYG{g+gp}{\PYGZgt{}\PYGZgt{}\PYGZgt{} }\PYG{l+m+mi}{5} \PYG{o}{*} \PYG{n}{np}\PYG{o}{.}\PYG{n}{random}\PYG{o}{.}\PYG{n}{random\PYGZus{}sample}\PYG{p}{(}\PYG{p}{(}\PYG{l+m+mi}{3}\PYG{p}{,} \PYG{l+m+mi}{2}\PYG{p}{)}\PYG{p}{)} \PYG{o}{\PYGZhy{}} \PYG{l+m+mi}{5}
\PYG{g+go}{array([[\PYGZhy{}3.99149989, \PYGZhy{}0.52338984],}
\PYG{g+go}{       [\PYGZhy{}2.99091858, \PYGZhy{}0.79479508],}
\PYG{g+go}{       [\PYGZhy{}1.23204345, \PYGZhy{}1.75224494]])}
\end{Verbatim}

\end{fulllineitems}

\index{random\_integers() (in module lib.IO.writefiles)}

\begin{fulllineitems}
\phantomsection\label{lib.IO:lib.IO.writefiles.random_integers}\pysiglinewithargsret{\code{lib.IO.writefiles.}\bfcode{random\_integers}}{\emph{low}, \emph{high=None}, \emph{size=None}}{}
Return random integers between \emph{low} and \emph{high}, inclusive.

Return random integers from the ``discrete uniform'' distribution in the
closed interval {[}\emph{low}, \emph{high}{]}.  If \emph{high} is None (the default),
then results are from {[}1, \emph{low}{]}.
\begin{description}
\item[{low}] \leavevmode{[}int{]}
Lowest (signed) integer to be drawn from the distribution (unless
\code{high=None}, in which case this parameter is the \emph{highest} such
integer).

\item[{high}] \leavevmode{[}int, optional{]}
If provided, the largest (signed) integer to be drawn from the
distribution (see above for behavior if \code{high=None}).

\item[{size}] \leavevmode{[}int or tuple of ints, optional{]}
Output shape. Default is None, in which case a single int is returned.

\end{description}
\begin{description}
\item[{out}] \leavevmode{[}int or ndarray of ints{]}
\emph{size}-shaped array of random integers from the appropriate
distribution, or a single such random int if \emph{size} not provided.

\end{description}
\begin{description}
\item[{random.randint}] \leavevmode{[}Similar to \emph{random\_integers}, only for the half-open{]}
interval {[}\emph{low}, \emph{high}), and 0 is the lowest value if \emph{high} is
omitted.

\end{description}

To sample from N evenly spaced floating-point numbers between a and b,
use:

\begin{Verbatim}[commandchars=\\\{\}]
\PYG{n}{a} \PYG{o}{+} \PYG{p}{(}\PYG{n}{b} \PYG{o}{\PYGZhy{}} \PYG{n}{a}\PYG{p}{)} \PYG{o}{*} \PYG{p}{(}\PYG{n}{np}\PYG{o}{.}\PYG{n}{random}\PYG{o}{.}\PYG{n}{random\PYGZus{}integers}\PYG{p}{(}\PYG{n}{N}\PYG{p}{)} \PYG{o}{\PYGZhy{}} \PYG{l+m+mi}{1}\PYG{p}{)} \PYG{o}{/} \PYG{p}{(}\PYG{n}{N} \PYG{o}{\PYGZhy{}} \PYG{l+m+mf}{1.}\PYG{p}{)}
\end{Verbatim}

\begin{Verbatim}[commandchars=\\\{\}]
\PYG{g+gp}{\PYGZgt{}\PYGZgt{}\PYGZgt{} }\PYG{n}{np}\PYG{o}{.}\PYG{n}{random}\PYG{o}{.}\PYG{n}{random\PYGZus{}integers}\PYG{p}{(}\PYG{l+m+mi}{5}\PYG{p}{)}
\PYG{g+go}{4}
\PYG{g+gp}{\PYGZgt{}\PYGZgt{}\PYGZgt{} }\PYG{n+nb}{type}\PYG{p}{(}\PYG{n}{np}\PYG{o}{.}\PYG{n}{random}\PYG{o}{.}\PYG{n}{random\PYGZus{}integers}\PYG{p}{(}\PYG{l+m+mi}{5}\PYG{p}{)}\PYG{p}{)}
\PYG{g+go}{\PYGZlt{}type \PYGZsq{}int\PYGZsq{}\PYGZgt{}}
\PYG{g+gp}{\PYGZgt{}\PYGZgt{}\PYGZgt{} }\PYG{n}{np}\PYG{o}{.}\PYG{n}{random}\PYG{o}{.}\PYG{n}{random\PYGZus{}integers}\PYG{p}{(}\PYG{l+m+mi}{5}\PYG{p}{,} \PYG{n}{size}\PYG{o}{=}\PYG{p}{(}\PYG{l+m+mf}{3.}\PYG{p}{,}\PYG{l+m+mf}{2.}\PYG{p}{)}\PYG{p}{)}
\PYG{g+go}{array([[5, 4],}
\PYG{g+go}{       [3, 3],}
\PYG{g+go}{       [4, 5]])}
\end{Verbatim}

Choose five random numbers from the set of five evenly-spaced
numbers between 0 and 2.5, inclusive (\emph{i.e.}, from the set
\({0, 5/8, 10/8, 15/8, 20/8}\)):

\begin{Verbatim}[commandchars=\\\{\}]
\PYG{g+gp}{\PYGZgt{}\PYGZgt{}\PYGZgt{} }\PYG{l+m+mf}{2.5} \PYG{o}{*} \PYG{p}{(}\PYG{n}{np}\PYG{o}{.}\PYG{n}{random}\PYG{o}{.}\PYG{n}{random\PYGZus{}integers}\PYG{p}{(}\PYG{l+m+mi}{5}\PYG{p}{,} \PYG{n}{size}\PYG{o}{=}\PYG{p}{(}\PYG{l+m+mi}{5}\PYG{p}{,}\PYG{p}{)}\PYG{p}{)} \PYG{o}{\PYGZhy{}} \PYG{l+m+mi}{1}\PYG{p}{)} \PYG{o}{/} \PYG{l+m+mf}{4.}
\PYG{g+go}{array([ 0.625,  1.25 ,  0.625,  0.625,  2.5  ])}
\end{Verbatim}

Roll two six sided dice 1000 times and sum the results:

\begin{Verbatim}[commandchars=\\\{\}]
\PYG{g+gp}{\PYGZgt{}\PYGZgt{}\PYGZgt{} }\PYG{n}{d1} \PYG{o}{=} \PYG{n}{np}\PYG{o}{.}\PYG{n}{random}\PYG{o}{.}\PYG{n}{random\PYGZus{}integers}\PYG{p}{(}\PYG{l+m+mi}{1}\PYG{p}{,} \PYG{l+m+mi}{6}\PYG{p}{,} \PYG{l+m+mi}{1000}\PYG{p}{)}
\PYG{g+gp}{\PYGZgt{}\PYGZgt{}\PYGZgt{} }\PYG{n}{d2} \PYG{o}{=} \PYG{n}{np}\PYG{o}{.}\PYG{n}{random}\PYG{o}{.}\PYG{n}{random\PYGZus{}integers}\PYG{p}{(}\PYG{l+m+mi}{1}\PYG{p}{,} \PYG{l+m+mi}{6}\PYG{p}{,} \PYG{l+m+mi}{1000}\PYG{p}{)}
\PYG{g+gp}{\PYGZgt{}\PYGZgt{}\PYGZgt{} }\PYG{n}{dsums} \PYG{o}{=} \PYG{n}{d1} \PYG{o}{+} \PYG{n}{d2}
\end{Verbatim}

Display results as a histogram:

\begin{Verbatim}[commandchars=\\\{\}]
\PYG{g+gp}{\PYGZgt{}\PYGZgt{}\PYGZgt{} }\PYG{k+kn}{import} \PYG{n+nn}{matplotlib.pyplot} \PYG{k+kn}{as} \PYG{n+nn}{plt}
\PYG{g+gp}{\PYGZgt{}\PYGZgt{}\PYGZgt{} }\PYG{n}{count}\PYG{p}{,} \PYG{n}{bins}\PYG{p}{,} \PYG{n}{ignored} \PYG{o}{=} \PYG{n}{plt}\PYG{o}{.}\PYG{n}{hist}\PYG{p}{(}\PYG{n}{dsums}\PYG{p}{,} \PYG{l+m+mi}{11}\PYG{p}{,} \PYG{n}{normed}\PYG{o}{=}\PYG{n+nb+bp}{True}\PYG{p}{)}
\PYG{g+gp}{\PYGZgt{}\PYGZgt{}\PYGZgt{} }\PYG{n}{plt}\PYG{o}{.}\PYG{n}{show}\PYG{p}{(}\PYG{p}{)}
\end{Verbatim}

\end{fulllineitems}

\index{random\_sample() (in module lib.IO.writefiles)}

\begin{fulllineitems}
\phantomsection\label{lib.IO:lib.IO.writefiles.random_sample}\pysiglinewithargsret{\code{lib.IO.writefiles.}\bfcode{random\_sample}}{\emph{size=None}}{}
Return random floats in the half-open interval {[}0.0, 1.0).

Results are from the ``continuous uniform'' distribution over the
stated interval.  To sample \(Unif[a, b), b > a\) multiply
the output of \emph{random\_sample} by \emph{(b-a)} and add \emph{a}:

\begin{Verbatim}[commandchars=\\\{\}]
\PYG{p}{(}\PYG{n}{b} \PYG{o}{\PYGZhy{}} \PYG{n}{a}\PYG{p}{)} \PYG{o}{*} \PYG{n}{random\PYGZus{}sample}\PYG{p}{(}\PYG{p}{)} \PYG{o}{+} \PYG{n}{a}
\end{Verbatim}
\begin{description}
\item[{size}] \leavevmode{[}int or tuple of ints, optional{]}
Defines the shape of the returned array of random floats. If None
(the default), returns a single float.

\end{description}
\begin{description}
\item[{out}] \leavevmode{[}float or ndarray of floats{]}
Array of random floats of shape \emph{size} (unless \code{size=None}, in which
case a single float is returned).

\end{description}

\begin{Verbatim}[commandchars=\\\{\}]
\PYG{g+gp}{\PYGZgt{}\PYGZgt{}\PYGZgt{} }\PYG{n}{np}\PYG{o}{.}\PYG{n}{random}\PYG{o}{.}\PYG{n}{random\PYGZus{}sample}\PYG{p}{(}\PYG{p}{)}
\PYG{g+go}{0.47108547995356098}
\PYG{g+gp}{\PYGZgt{}\PYGZgt{}\PYGZgt{} }\PYG{n+nb}{type}\PYG{p}{(}\PYG{n}{np}\PYG{o}{.}\PYG{n}{random}\PYG{o}{.}\PYG{n}{random\PYGZus{}sample}\PYG{p}{(}\PYG{p}{)}\PYG{p}{)}
\PYG{g+go}{\PYGZlt{}type \PYGZsq{}float\PYGZsq{}\PYGZgt{}}
\PYG{g+gp}{\PYGZgt{}\PYGZgt{}\PYGZgt{} }\PYG{n}{np}\PYG{o}{.}\PYG{n}{random}\PYG{o}{.}\PYG{n}{random\PYGZus{}sample}\PYG{p}{(}\PYG{p}{(}\PYG{l+m+mi}{5}\PYG{p}{,}\PYG{p}{)}\PYG{p}{)}
\PYG{g+go}{array([ 0.30220482,  0.86820401,  0.1654503 ,  0.11659149,  0.54323428])}
\end{Verbatim}

Three-by-two array of random numbers from {[}-5, 0):

\begin{Verbatim}[commandchars=\\\{\}]
\PYG{g+gp}{\PYGZgt{}\PYGZgt{}\PYGZgt{} }\PYG{l+m+mi}{5} \PYG{o}{*} \PYG{n}{np}\PYG{o}{.}\PYG{n}{random}\PYG{o}{.}\PYG{n}{random\PYGZus{}sample}\PYG{p}{(}\PYG{p}{(}\PYG{l+m+mi}{3}\PYG{p}{,} \PYG{l+m+mi}{2}\PYG{p}{)}\PYG{p}{)} \PYG{o}{\PYGZhy{}} \PYG{l+m+mi}{5}
\PYG{g+go}{array([[\PYGZhy{}3.99149989, \PYGZhy{}0.52338984],}
\PYG{g+go}{       [\PYGZhy{}2.99091858, \PYGZhy{}0.79479508],}
\PYG{g+go}{       [\PYGZhy{}1.23204345, \PYGZhy{}1.75224494]])}
\end{Verbatim}

\end{fulllineitems}

\index{ranf() (in module lib.IO.writefiles)}

\begin{fulllineitems}
\phantomsection\label{lib.IO:lib.IO.writefiles.ranf}\pysiglinewithargsret{\code{lib.IO.writefiles.}\bfcode{ranf}}{}{}
random\_sample(size=None)

Return random floats in the half-open interval {[}0.0, 1.0).

Results are from the ``continuous uniform'' distribution over the
stated interval.  To sample \(Unif[a, b), b > a\) multiply
the output of \emph{random\_sample} by \emph{(b-a)} and add \emph{a}:

\begin{Verbatim}[commandchars=\\\{\}]
\PYG{p}{(}\PYG{n}{b} \PYG{o}{\PYGZhy{}} \PYG{n}{a}\PYG{p}{)} \PYG{o}{*} \PYG{n}{random\PYGZus{}sample}\PYG{p}{(}\PYG{p}{)} \PYG{o}{+} \PYG{n}{a}
\end{Verbatim}
\begin{description}
\item[{size}] \leavevmode{[}int or tuple of ints, optional{]}
Defines the shape of the returned array of random floats. If None
(the default), returns a single float.

\end{description}
\begin{description}
\item[{out}] \leavevmode{[}float or ndarray of floats{]}
Array of random floats of shape \emph{size} (unless \code{size=None}, in which
case a single float is returned).

\end{description}

\begin{Verbatim}[commandchars=\\\{\}]
\PYG{g+gp}{\PYGZgt{}\PYGZgt{}\PYGZgt{} }\PYG{n}{np}\PYG{o}{.}\PYG{n}{random}\PYG{o}{.}\PYG{n}{random\PYGZus{}sample}\PYG{p}{(}\PYG{p}{)}
\PYG{g+go}{0.47108547995356098}
\PYG{g+gp}{\PYGZgt{}\PYGZgt{}\PYGZgt{} }\PYG{n+nb}{type}\PYG{p}{(}\PYG{n}{np}\PYG{o}{.}\PYG{n}{random}\PYG{o}{.}\PYG{n}{random\PYGZus{}sample}\PYG{p}{(}\PYG{p}{)}\PYG{p}{)}
\PYG{g+go}{\PYGZlt{}type \PYGZsq{}float\PYGZsq{}\PYGZgt{}}
\PYG{g+gp}{\PYGZgt{}\PYGZgt{}\PYGZgt{} }\PYG{n}{np}\PYG{o}{.}\PYG{n}{random}\PYG{o}{.}\PYG{n}{random\PYGZus{}sample}\PYG{p}{(}\PYG{p}{(}\PYG{l+m+mi}{5}\PYG{p}{,}\PYG{p}{)}\PYG{p}{)}
\PYG{g+go}{array([ 0.30220482,  0.86820401,  0.1654503 ,  0.11659149,  0.54323428])}
\end{Verbatim}

Three-by-two array of random numbers from {[}-5, 0):

\begin{Verbatim}[commandchars=\\\{\}]
\PYG{g+gp}{\PYGZgt{}\PYGZgt{}\PYGZgt{} }\PYG{l+m+mi}{5} \PYG{o}{*} \PYG{n}{np}\PYG{o}{.}\PYG{n}{random}\PYG{o}{.}\PYG{n}{random\PYGZus{}sample}\PYG{p}{(}\PYG{p}{(}\PYG{l+m+mi}{3}\PYG{p}{,} \PYG{l+m+mi}{2}\PYG{p}{)}\PYG{p}{)} \PYG{o}{\PYGZhy{}} \PYG{l+m+mi}{5}
\PYG{g+go}{array([[\PYGZhy{}3.99149989, \PYGZhy{}0.52338984],}
\PYG{g+go}{       [\PYGZhy{}2.99091858, \PYGZhy{}0.79479508],}
\PYG{g+go}{       [\PYGZhy{}1.23204345, \PYGZhy{}1.75224494]])}
\end{Verbatim}

\end{fulllineitems}

\index{rayleigh() (in module lib.IO.writefiles)}

\begin{fulllineitems}
\phantomsection\label{lib.IO:lib.IO.writefiles.rayleigh}\pysiglinewithargsret{\code{lib.IO.writefiles.}\bfcode{rayleigh}}{\emph{scale=1.0}, \emph{size=None}}{}
Draw samples from a Rayleigh distribution.

The \(\chi\) and Weibull distributions are generalizations of the
Rayleigh.
\begin{description}
\item[{scale}] \leavevmode{[}scalar{]}
Scale, also equals the mode. Should be \textgreater{}= 0.

\item[{size}] \leavevmode{[}int or tuple of ints, optional{]}
Shape of the output. Default is None, in which case a single
value is returned.

\end{description}

The probability density function for the Rayleigh distribution is
\begin{gather}
\begin{split}P(x;scale) = \frac{x}{scale^2}e^{\frac{-x^2}{2 \cdotp scale^2}}\end{split}\notag
\end{gather}
The Rayleigh distribution arises if the wind speed and wind direction are
both gaussian variables, then the vector wind velocity forms a Rayleigh
distribution. The Rayleigh distribution is used to model the expected
output from wind turbines.

Draw values from the distribution and plot the histogram

\begin{Verbatim}[commandchars=\\\{\}]
\PYG{g+gp}{\PYGZgt{}\PYGZgt{}\PYGZgt{} }\PYG{n}{values} \PYG{o}{=} \PYG{n}{hist}\PYG{p}{(}\PYG{n}{np}\PYG{o}{.}\PYG{n}{random}\PYG{o}{.}\PYG{n}{rayleigh}\PYG{p}{(}\PYG{l+m+mi}{3}\PYG{p}{,} \PYG{l+m+mi}{100000}\PYG{p}{)}\PYG{p}{,} \PYG{n}{bins}\PYG{o}{=}\PYG{l+m+mi}{200}\PYG{p}{,} \PYG{n}{normed}\PYG{o}{=}\PYG{n+nb+bp}{True}\PYG{p}{)}
\end{Verbatim}

Wave heights tend to follow a Rayleigh distribution. If the mean wave
height is 1 meter, what fraction of waves are likely to be larger than 3
meters?

\begin{Verbatim}[commandchars=\\\{\}]
\PYG{g+gp}{\PYGZgt{}\PYGZgt{}\PYGZgt{} }\PYG{n}{meanvalue} \PYG{o}{=} \PYG{l+m+mi}{1}
\PYG{g+gp}{\PYGZgt{}\PYGZgt{}\PYGZgt{} }\PYG{n}{modevalue} \PYG{o}{=} \PYG{n}{np}\PYG{o}{.}\PYG{n}{sqrt}\PYG{p}{(}\PYG{l+m+mi}{2} \PYG{o}{/} \PYG{n}{np}\PYG{o}{.}\PYG{n}{pi}\PYG{p}{)} \PYG{o}{*} \PYG{n}{meanvalue}
\PYG{g+gp}{\PYGZgt{}\PYGZgt{}\PYGZgt{} }\PYG{n}{s} \PYG{o}{=} \PYG{n}{np}\PYG{o}{.}\PYG{n}{random}\PYG{o}{.}\PYG{n}{rayleigh}\PYG{p}{(}\PYG{n}{modevalue}\PYG{p}{,} \PYG{l+m+mi}{1000000}\PYG{p}{)}
\end{Verbatim}

The percentage of waves larger than 3 meters is:

\begin{Verbatim}[commandchars=\\\{\}]
\PYG{g+gp}{\PYGZgt{}\PYGZgt{}\PYGZgt{} }\PYG{l+m+mf}{100.}\PYG{o}{*}\PYG{n+nb}{sum}\PYG{p}{(}\PYG{n}{s}\PYG{o}{\PYGZgt{}}\PYG{l+m+mi}{3}\PYG{p}{)}\PYG{o}{/}\PYG{l+m+mf}{1000000.}
\PYG{g+go}{0.087300000000000003}
\end{Verbatim}

\end{fulllineitems}

\index{sample() (in module lib.IO.writefiles)}

\begin{fulllineitems}
\phantomsection\label{lib.IO:lib.IO.writefiles.sample}\pysiglinewithargsret{\code{lib.IO.writefiles.}\bfcode{sample}}{}{}
random\_sample(size=None)

Return random floats in the half-open interval {[}0.0, 1.0).

Results are from the ``continuous uniform'' distribution over the
stated interval.  To sample \(Unif[a, b), b > a\) multiply
the output of \emph{random\_sample} by \emph{(b-a)} and add \emph{a}:

\begin{Verbatim}[commandchars=\\\{\}]
\PYG{p}{(}\PYG{n}{b} \PYG{o}{\PYGZhy{}} \PYG{n}{a}\PYG{p}{)} \PYG{o}{*} \PYG{n}{random\PYGZus{}sample}\PYG{p}{(}\PYG{p}{)} \PYG{o}{+} \PYG{n}{a}
\end{Verbatim}
\begin{description}
\item[{size}] \leavevmode{[}int or tuple of ints, optional{]}
Defines the shape of the returned array of random floats. If None
(the default), returns a single float.

\end{description}
\begin{description}
\item[{out}] \leavevmode{[}float or ndarray of floats{]}
Array of random floats of shape \emph{size} (unless \code{size=None}, in which
case a single float is returned).

\end{description}

\begin{Verbatim}[commandchars=\\\{\}]
\PYG{g+gp}{\PYGZgt{}\PYGZgt{}\PYGZgt{} }\PYG{n}{np}\PYG{o}{.}\PYG{n}{random}\PYG{o}{.}\PYG{n}{random\PYGZus{}sample}\PYG{p}{(}\PYG{p}{)}
\PYG{g+go}{0.47108547995356098}
\PYG{g+gp}{\PYGZgt{}\PYGZgt{}\PYGZgt{} }\PYG{n+nb}{type}\PYG{p}{(}\PYG{n}{np}\PYG{o}{.}\PYG{n}{random}\PYG{o}{.}\PYG{n}{random\PYGZus{}sample}\PYG{p}{(}\PYG{p}{)}\PYG{p}{)}
\PYG{g+go}{\PYGZlt{}type \PYGZsq{}float\PYGZsq{}\PYGZgt{}}
\PYG{g+gp}{\PYGZgt{}\PYGZgt{}\PYGZgt{} }\PYG{n}{np}\PYG{o}{.}\PYG{n}{random}\PYG{o}{.}\PYG{n}{random\PYGZus{}sample}\PYG{p}{(}\PYG{p}{(}\PYG{l+m+mi}{5}\PYG{p}{,}\PYG{p}{)}\PYG{p}{)}
\PYG{g+go}{array([ 0.30220482,  0.86820401,  0.1654503 ,  0.11659149,  0.54323428])}
\end{Verbatim}

Three-by-two array of random numbers from {[}-5, 0):

\begin{Verbatim}[commandchars=\\\{\}]
\PYG{g+gp}{\PYGZgt{}\PYGZgt{}\PYGZgt{} }\PYG{l+m+mi}{5} \PYG{o}{*} \PYG{n}{np}\PYG{o}{.}\PYG{n}{random}\PYG{o}{.}\PYG{n}{random\PYGZus{}sample}\PYG{p}{(}\PYG{p}{(}\PYG{l+m+mi}{3}\PYG{p}{,} \PYG{l+m+mi}{2}\PYG{p}{)}\PYG{p}{)} \PYG{o}{\PYGZhy{}} \PYG{l+m+mi}{5}
\PYG{g+go}{array([[\PYGZhy{}3.99149989, \PYGZhy{}0.52338984],}
\PYG{g+go}{       [\PYGZhy{}2.99091858, \PYGZhy{}0.79479508],}
\PYG{g+go}{       [\PYGZhy{}1.23204345, \PYGZhy{}1.75224494]])}
\end{Verbatim}

\end{fulllineitems}

\index{saveRandomSeed() (in module lib.IO.writefiles)}

\begin{fulllineitems}
\phantomsection\label{lib.IO:lib.IO.writefiles.saveRandomSeed}\pysiglinewithargsret{\code{lib.IO.writefiles.}\bfcode{saveRandomSeed}}{\emph{tmpPath}}{}
Function to save the random seed

\end{fulllineitems}

\index{seed() (in module lib.IO.writefiles)}

\begin{fulllineitems}
\phantomsection\label{lib.IO:lib.IO.writefiles.seed}\pysiglinewithargsret{\code{lib.IO.writefiles.}\bfcode{seed}}{\emph{seed=None}}{}
Seed the generator.

This method is called when \emph{RandomState} is initialized. It can be
called again to re-seed the generator. For details, see \emph{RandomState}.
\begin{description}
\item[{seed}] \leavevmode{[}int or array\_like, optional{]}
Seed for \emph{RandomState}.

\end{description}

RandomState

\end{fulllineitems}

\index{set\_state() (in module lib.IO.writefiles)}

\begin{fulllineitems}
\phantomsection\label{lib.IO:lib.IO.writefiles.set_state}\pysiglinewithargsret{\code{lib.IO.writefiles.}\bfcode{set\_state}}{\emph{state}}{}
Set the internal state of the generator from a tuple.

For use if one has reason to manually (re-)set the internal state of the
``Mersenne Twister''{\color{red}\bfseries{}{[}1{]}\_} pseudo-random number generating algorithm.
\begin{description}
\item[{state}] \leavevmode{[}tuple(str, ndarray of 624 uints, int, int, float){]}
The \emph{state} tuple has the following items:
\begin{enumerate}
\item {} 
the string `MT19937', specifying the Mersenne Twister algorithm.

\item {} 
a 1-D array of 624 unsigned integers \code{keys}.

\item {} 
an integer \code{pos}.

\item {} 
an integer \code{has\_gauss}.

\item {} 
a float \code{cached\_gaussian}.

\end{enumerate}

\end{description}
\begin{description}
\item[{out}] \leavevmode{[}None{]}
Returns `None' on success.

\end{description}

get\_state

\emph{set\_state} and \emph{get\_state} are not needed to work with any of the
random distributions in NumPy. If the internal state is manually altered,
the user should know exactly what he/she is doing.

For backwards compatibility, the form (str, array of 624 uints, int) is
also accepted although it is missing some information about the cached
Gaussian value: \code{state = ('MT19937', keys, pos)}.

\end{fulllineitems}

\index{shuffle() (in module lib.IO.writefiles)}

\begin{fulllineitems}
\phantomsection\label{lib.IO:lib.IO.writefiles.shuffle}\pysiglinewithargsret{\code{lib.IO.writefiles.}\bfcode{shuffle}}{\emph{x}}{}
Modify a sequence in-place by shuffling its contents.
\begin{description}
\item[{x}] \leavevmode{[}array\_like{]}
The array or list to be shuffled.

\end{description}

None

\begin{Verbatim}[commandchars=\\\{\}]
\PYG{g+gp}{\PYGZgt{}\PYGZgt{}\PYGZgt{} }\PYG{n}{arr} \PYG{o}{=} \PYG{n}{np}\PYG{o}{.}\PYG{n}{arange}\PYG{p}{(}\PYG{l+m+mi}{10}\PYG{p}{)}
\PYG{g+gp}{\PYGZgt{}\PYGZgt{}\PYGZgt{} }\PYG{n}{np}\PYG{o}{.}\PYG{n}{random}\PYG{o}{.}\PYG{n}{shuffle}\PYG{p}{(}\PYG{n}{arr}\PYG{p}{)}
\PYG{g+gp}{\PYGZgt{}\PYGZgt{}\PYGZgt{} }\PYG{n}{arr}
\PYG{g+go}{[1 7 5 2 9 4 3 6 0 8]}
\end{Verbatim}

This function only shuffles the array along the first index of a
multi-dimensional array:

\begin{Verbatim}[commandchars=\\\{\}]
\PYG{g+gp}{\PYGZgt{}\PYGZgt{}\PYGZgt{} }\PYG{n}{arr} \PYG{o}{=} \PYG{n}{np}\PYG{o}{.}\PYG{n}{arange}\PYG{p}{(}\PYG{l+m+mi}{9}\PYG{p}{)}\PYG{o}{.}\PYG{n}{reshape}\PYG{p}{(}\PYG{p}{(}\PYG{l+m+mi}{3}\PYG{p}{,} \PYG{l+m+mi}{3}\PYG{p}{)}\PYG{p}{)}
\PYG{g+gp}{\PYGZgt{}\PYGZgt{}\PYGZgt{} }\PYG{n}{np}\PYG{o}{.}\PYG{n}{random}\PYG{o}{.}\PYG{n}{shuffle}\PYG{p}{(}\PYG{n}{arr}\PYG{p}{)}
\PYG{g+gp}{\PYGZgt{}\PYGZgt{}\PYGZgt{} }\PYG{n}{arr}
\PYG{g+go}{array([[3, 4, 5],}
\PYG{g+go}{       [6, 7, 8],}
\PYG{g+go}{       [0, 1, 2]])}
\end{Verbatim}

\end{fulllineitems}

\index{standard\_cauchy() (in module lib.IO.writefiles)}

\begin{fulllineitems}
\phantomsection\label{lib.IO:lib.IO.writefiles.standard_cauchy}\pysiglinewithargsret{\code{lib.IO.writefiles.}\bfcode{standard\_cauchy}}{\emph{size=None}}{}
Standard Cauchy distribution with mode = 0.

Also known as the Lorentz distribution.
\begin{description}
\item[{size}] \leavevmode{[}int or tuple of ints{]}
Shape of the output.

\end{description}
\begin{description}
\item[{samples}] \leavevmode{[}ndarray or scalar{]}
The drawn samples.

\end{description}

The probability density function for the full Cauchy distribution is
\begin{gather}
\begin{split}P(x; x_0, \gamma) = \frac{1}{\pi \gamma \bigl[ 1+
(\frac{x-x_0}{\gamma})^2 \bigr] }\end{split}\notag
\end{gather}
and the Standard Cauchy distribution just sets \(x_0=0\) and
\(\gamma=1\)

The Cauchy distribution arises in the solution to the driven harmonic
oscillator problem, and also describes spectral line broadening. It
also describes the distribution of values at which a line tilted at
a random angle will cut the x axis.

When studying hypothesis tests that assume normality, seeing how the
tests perform on data from a Cauchy distribution is a good indicator of
their sensitivity to a heavy-tailed distribution, since the Cauchy looks
very much like a Gaussian distribution, but with heavier tails.

Draw samples and plot the distribution:

\begin{Verbatim}[commandchars=\\\{\}]
\PYG{g+gp}{\PYGZgt{}\PYGZgt{}\PYGZgt{} }\PYG{n}{s} \PYG{o}{=} \PYG{n}{np}\PYG{o}{.}\PYG{n}{random}\PYG{o}{.}\PYG{n}{standard\PYGZus{}cauchy}\PYG{p}{(}\PYG{l+m+mi}{1000000}\PYG{p}{)}
\PYG{g+gp}{\PYGZgt{}\PYGZgt{}\PYGZgt{} }\PYG{n}{s} \PYG{o}{=} \PYG{n}{s}\PYG{p}{[}\PYG{p}{(}\PYG{n}{s}\PYG{o}{\PYGZgt{}}\PYG{o}{\PYGZhy{}}\PYG{l+m+mi}{25}\PYG{p}{)} \PYG{o}{\PYGZam{}} \PYG{p}{(}\PYG{n}{s}\PYG{o}{\PYGZlt{}}\PYG{l+m+mi}{25}\PYG{p}{)}\PYG{p}{]}  \PYG{c}{\PYGZsh{} truncate distribution so it plots well}
\PYG{g+gp}{\PYGZgt{}\PYGZgt{}\PYGZgt{} }\PYG{n}{plt}\PYG{o}{.}\PYG{n}{hist}\PYG{p}{(}\PYG{n}{s}\PYG{p}{,} \PYG{n}{bins}\PYG{o}{=}\PYG{l+m+mi}{100}\PYG{p}{)}
\PYG{g+gp}{\PYGZgt{}\PYGZgt{}\PYGZgt{} }\PYG{n}{plt}\PYG{o}{.}\PYG{n}{show}\PYG{p}{(}\PYG{p}{)}
\end{Verbatim}

\end{fulllineitems}

\index{standard\_exponential() (in module lib.IO.writefiles)}

\begin{fulllineitems}
\phantomsection\label{lib.IO:lib.IO.writefiles.standard_exponential}\pysiglinewithargsret{\code{lib.IO.writefiles.}\bfcode{standard\_exponential}}{\emph{size=None}}{}
Draw samples from the standard exponential distribution.

\emph{standard\_exponential} is identical to the exponential distribution
with a scale parameter of 1.
\begin{description}
\item[{size}] \leavevmode{[}int or tuple of ints{]}
Shape of the output.

\end{description}
\begin{description}
\item[{out}] \leavevmode{[}float or ndarray{]}
Drawn samples.

\end{description}

Output a 3x8000 array:

\begin{Verbatim}[commandchars=\\\{\}]
\PYG{g+gp}{\PYGZgt{}\PYGZgt{}\PYGZgt{} }\PYG{n}{n} \PYG{o}{=} \PYG{n}{np}\PYG{o}{.}\PYG{n}{random}\PYG{o}{.}\PYG{n}{standard\PYGZus{}exponential}\PYG{p}{(}\PYG{p}{(}\PYG{l+m+mi}{3}\PYG{p}{,} \PYG{l+m+mi}{8000}\PYG{p}{)}\PYG{p}{)}
\end{Verbatim}

\end{fulllineitems}

\index{standard\_gamma() (in module lib.IO.writefiles)}

\begin{fulllineitems}
\phantomsection\label{lib.IO:lib.IO.writefiles.standard_gamma}\pysiglinewithargsret{\code{lib.IO.writefiles.}\bfcode{standard\_gamma}}{\emph{shape}, \emph{size=None}}{}
Draw samples from a Standard Gamma distribution.

Samples are drawn from a Gamma distribution with specified parameters,
shape (sometimes designated ``k'') and scale=1.
\begin{description}
\item[{shape}] \leavevmode{[}float{]}
Parameter, should be \textgreater{} 0.

\item[{size}] \leavevmode{[}int or tuple of ints{]}
Output shape.  If the given shape is, e.g., \code{(m, n, k)}, then
\code{m * n * k} samples are drawn.

\end{description}
\begin{description}
\item[{samples}] \leavevmode{[}ndarray or scalar{]}
The drawn samples.

\end{description}
\begin{description}
\item[{scipy.stats.distributions.gamma}] \leavevmode{[}probability density function,{]}
distribution or cumulative density function, etc.

\end{description}

The probability density for the Gamma distribution is
\begin{gather}
\begin{split}p(x) = x^{k-1}\frac{e^{-x/\theta}}{\theta^k\Gamma(k)},\end{split}\notag
\end{gather}
where \(k\) is the shape and \(\theta\) the scale,
and \(\Gamma\) is the Gamma function.

The Gamma distribution is often used to model the times to failure of
electronic components, and arises naturally in processes for which the
waiting times between Poisson distributed events are relevant.

Draw samples from the distribution:

\begin{Verbatim}[commandchars=\\\{\}]
\PYG{g+gp}{\PYGZgt{}\PYGZgt{}\PYGZgt{} }\PYG{n}{shape}\PYG{p}{,} \PYG{n}{scale} \PYG{o}{=} \PYG{l+m+mf}{2.}\PYG{p}{,} \PYG{l+m+mf}{1.} \PYG{c}{\PYGZsh{} mean and width}
\PYG{g+gp}{\PYGZgt{}\PYGZgt{}\PYGZgt{} }\PYG{n}{s} \PYG{o}{=} \PYG{n}{np}\PYG{o}{.}\PYG{n}{random}\PYG{o}{.}\PYG{n}{standard\PYGZus{}gamma}\PYG{p}{(}\PYG{n}{shape}\PYG{p}{,} \PYG{l+m+mi}{1000000}\PYG{p}{)}
\end{Verbatim}

Display the histogram of the samples, along with
the probability density function:

\begin{Verbatim}[commandchars=\\\{\}]
\PYG{g+gp}{\PYGZgt{}\PYGZgt{}\PYGZgt{} }\PYG{k+kn}{import} \PYG{n+nn}{matplotlib.pyplot} \PYG{k+kn}{as} \PYG{n+nn}{plt}
\PYG{g+gp}{\PYGZgt{}\PYGZgt{}\PYGZgt{} }\PYG{k+kn}{import} \PYG{n+nn}{scipy.special} \PYG{k+kn}{as} \PYG{n+nn}{sps}
\PYG{g+gp}{\PYGZgt{}\PYGZgt{}\PYGZgt{} }\PYG{n}{count}\PYG{p}{,} \PYG{n}{bins}\PYG{p}{,} \PYG{n}{ignored} \PYG{o}{=} \PYG{n}{plt}\PYG{o}{.}\PYG{n}{hist}\PYG{p}{(}\PYG{n}{s}\PYG{p}{,} \PYG{l+m+mi}{50}\PYG{p}{,} \PYG{n}{normed}\PYG{o}{=}\PYG{n+nb+bp}{True}\PYG{p}{)}
\PYG{g+gp}{\PYGZgt{}\PYGZgt{}\PYGZgt{} }\PYG{n}{y} \PYG{o}{=} \PYG{n}{bins}\PYG{o}{*}\PYG{o}{*}\PYG{p}{(}\PYG{n}{shape}\PYG{o}{\PYGZhy{}}\PYG{l+m+mi}{1}\PYG{p}{)} \PYG{o}{*} \PYG{p}{(}\PYG{p}{(}\PYG{n}{np}\PYG{o}{.}\PYG{n}{exp}\PYG{p}{(}\PYG{o}{\PYGZhy{}}\PYG{n}{bins}\PYG{o}{/}\PYG{n}{scale}\PYG{p}{)}\PYG{p}{)}\PYG{o}{/} \PYGZbs{}
\PYG{g+gp}{... }                      \PYG{p}{(}\PYG{n}{sps}\PYG{o}{.}\PYG{n}{gamma}\PYG{p}{(}\PYG{n}{shape}\PYG{p}{)} \PYG{o}{*} \PYG{n}{scale}\PYG{o}{*}\PYG{o}{*}\PYG{n}{shape}\PYG{p}{)}\PYG{p}{)}
\PYG{g+gp}{\PYGZgt{}\PYGZgt{}\PYGZgt{} }\PYG{n}{plt}\PYG{o}{.}\PYG{n}{plot}\PYG{p}{(}\PYG{n}{bins}\PYG{p}{,} \PYG{n}{y}\PYG{p}{,} \PYG{n}{linewidth}\PYG{o}{=}\PYG{l+m+mi}{2}\PYG{p}{,} \PYG{n}{color}\PYG{o}{=}\PYG{l+s}{\PYGZsq{}}\PYG{l+s}{r}\PYG{l+s}{\PYGZsq{}}\PYG{p}{)}
\PYG{g+gp}{\PYGZgt{}\PYGZgt{}\PYGZgt{} }\PYG{n}{plt}\PYG{o}{.}\PYG{n}{show}\PYG{p}{(}\PYG{p}{)}
\end{Verbatim}

\end{fulllineitems}

\index{standard\_normal() (in module lib.IO.writefiles)}

\begin{fulllineitems}
\phantomsection\label{lib.IO:lib.IO.writefiles.standard_normal}\pysiglinewithargsret{\code{lib.IO.writefiles.}\bfcode{standard\_normal}}{\emph{size=None}}{}
Returns samples from a Standard Normal distribution (mean=0, stdev=1).
\begin{description}
\item[{size}] \leavevmode{[}int or tuple of ints, optional{]}
Output shape. Default is None, in which case a single value is
returned.

\end{description}
\begin{description}
\item[{out}] \leavevmode{[}float or ndarray{]}
Drawn samples.

\end{description}

\begin{Verbatim}[commandchars=\\\{\}]
\PYG{g+gp}{\PYGZgt{}\PYGZgt{}\PYGZgt{} }\PYG{n}{s} \PYG{o}{=} \PYG{n}{np}\PYG{o}{.}\PYG{n}{random}\PYG{o}{.}\PYG{n}{standard\PYGZus{}normal}\PYG{p}{(}\PYG{l+m+mi}{8000}\PYG{p}{)}
\PYG{g+gp}{\PYGZgt{}\PYGZgt{}\PYGZgt{} }\PYG{n}{s}
\PYG{g+go}{array([ 0.6888893 ,  0.78096262, \PYGZhy{}0.89086505, ...,  0.49876311, \PYGZsh{}random}
\PYG{g+go}{       \PYGZhy{}0.38672696, \PYGZhy{}0.4685006 ])                               \PYGZsh{}random}
\PYG{g+gp}{\PYGZgt{}\PYGZgt{}\PYGZgt{} }\PYG{n}{s}\PYG{o}{.}\PYG{n}{shape}
\PYG{g+go}{(8000,)}
\PYG{g+gp}{\PYGZgt{}\PYGZgt{}\PYGZgt{} }\PYG{n}{s} \PYG{o}{=} \PYG{n}{np}\PYG{o}{.}\PYG{n}{random}\PYG{o}{.}\PYG{n}{standard\PYGZus{}normal}\PYG{p}{(}\PYG{n}{size}\PYG{o}{=}\PYG{p}{(}\PYG{l+m+mi}{3}\PYG{p}{,} \PYG{l+m+mi}{4}\PYG{p}{,} \PYG{l+m+mi}{2}\PYG{p}{)}\PYG{p}{)}
\PYG{g+gp}{\PYGZgt{}\PYGZgt{}\PYGZgt{} }\PYG{n}{s}\PYG{o}{.}\PYG{n}{shape}
\PYG{g+go}{(3, 4, 2)}
\end{Verbatim}

\end{fulllineitems}

\index{standard\_t() (in module lib.IO.writefiles)}

\begin{fulllineitems}
\phantomsection\label{lib.IO:lib.IO.writefiles.standard_t}\pysiglinewithargsret{\code{lib.IO.writefiles.}\bfcode{standard\_t}}{\emph{df}, \emph{size=None}}{}
Standard Student's t distribution with df degrees of freedom.

A special case of the hyperbolic distribution.
As \emph{df} gets large, the result resembles that of the standard normal
distribution (\emph{standard\_normal}).
\begin{description}
\item[{df}] \leavevmode{[}int{]}
Degrees of freedom, should be \textgreater{} 0.

\item[{size}] \leavevmode{[}int or tuple of ints, optional{]}
Output shape. Default is None, in which case a single value is
returned.

\end{description}
\begin{description}
\item[{samples}] \leavevmode{[}ndarray or scalar{]}
Drawn samples.

\end{description}

The probability density function for the t distribution is
\begin{gather}
\begin{split}P(x, df) = \frac{\Gamma(\frac{df+1}{2})}{\sqrt{\pi df}
\Gamma(\frac{df}{2})}\Bigl( 1+\frac{x^2}{df} \Bigr)^{-(df+1)/2}\end{split}\notag
\end{gather}
The t test is based on an assumption that the data come from a Normal
distribution. The t test provides a way to test whether the sample mean
(that is the mean calculated from the data) is a good estimate of the true
mean.

The derivation of the t-distribution was forst published in 1908 by William
Gisset while working for the Guinness Brewery in Dublin. Due to proprietary
issues, he had to publish under a pseudonym, and so he used the name
Student.

From Dalgaard page 83 {\color{red}\bfseries{}{[}1{]}\_}, suppose the daily energy intake for 11
women in Kj is:

\begin{Verbatim}[commandchars=\\\{\}]
\PYG{g+gp}{\PYGZgt{}\PYGZgt{}\PYGZgt{} }\PYG{n}{intake} \PYG{o}{=} \PYG{n}{np}\PYG{o}{.}\PYG{n}{array}\PYG{p}{(}\PYG{p}{[}\PYG{l+m+mf}{5260.}\PYG{p}{,} \PYG{l+m+mi}{5470}\PYG{p}{,} \PYG{l+m+mi}{5640}\PYG{p}{,} \PYG{l+m+mi}{6180}\PYG{p}{,} \PYG{l+m+mi}{6390}\PYG{p}{,} \PYG{l+m+mi}{6515}\PYG{p}{,} \PYG{l+m+mi}{6805}\PYG{p}{,} \PYG{l+m+mi}{7515}\PYG{p}{,} \PYGZbs{}
\PYG{g+gp}{... }                   \PYG{l+m+mi}{7515}\PYG{p}{,} \PYG{l+m+mi}{8230}\PYG{p}{,} \PYG{l+m+mi}{8770}\PYG{p}{]}\PYG{p}{)}
\end{Verbatim}

Does their energy intake deviate systematically from the recommended
value of 7725 kJ?

We have 10 degrees of freedom, so is the sample mean within 95\% of the
recommended value?

\begin{Verbatim}[commandchars=\\\{\}]
\PYG{g+gp}{\PYGZgt{}\PYGZgt{}\PYGZgt{} }\PYG{n}{s} \PYG{o}{=} \PYG{n}{np}\PYG{o}{.}\PYG{n}{random}\PYG{o}{.}\PYG{n}{standard\PYGZus{}t}\PYG{p}{(}\PYG{l+m+mi}{10}\PYG{p}{,} \PYG{n}{size}\PYG{o}{=}\PYG{l+m+mi}{100000}\PYG{p}{)}
\PYG{g+gp}{\PYGZgt{}\PYGZgt{}\PYGZgt{} }\PYG{n}{np}\PYG{o}{.}\PYG{n}{mean}\PYG{p}{(}\PYG{n}{intake}\PYG{p}{)}
\PYG{g+go}{6753.636363636364}
\PYG{g+gp}{\PYGZgt{}\PYGZgt{}\PYGZgt{} }\PYG{n}{intake}\PYG{o}{.}\PYG{n}{std}\PYG{p}{(}\PYG{n}{ddof}\PYG{o}{=}\PYG{l+m+mi}{1}\PYG{p}{)}
\PYG{g+go}{1142.1232221373727}
\end{Verbatim}

Calculate the t statistic, setting the ddof parameter to the unbiased
value so the divisor in the standard deviation will be degrees of
freedom, N-1.

\begin{Verbatim}[commandchars=\\\{\}]
\PYG{g+gp}{\PYGZgt{}\PYGZgt{}\PYGZgt{} }\PYG{n}{t} \PYG{o}{=} \PYG{p}{(}\PYG{n}{np}\PYG{o}{.}\PYG{n}{mean}\PYG{p}{(}\PYG{n}{intake}\PYG{p}{)}\PYG{o}{\PYGZhy{}}\PYG{l+m+mi}{7725}\PYG{p}{)}\PYG{o}{/}\PYG{p}{(}\PYG{n}{intake}\PYG{o}{.}\PYG{n}{std}\PYG{p}{(}\PYG{n}{ddof}\PYG{o}{=}\PYG{l+m+mi}{1}\PYG{p}{)}\PYG{o}{/}\PYG{n}{np}\PYG{o}{.}\PYG{n}{sqrt}\PYG{p}{(}\PYG{n+nb}{len}\PYG{p}{(}\PYG{n}{intake}\PYG{p}{)}\PYG{p}{)}\PYG{p}{)}
\PYG{g+gp}{\PYGZgt{}\PYGZgt{}\PYGZgt{} }\PYG{k+kn}{import} \PYG{n+nn}{matplotlib.pyplot} \PYG{k+kn}{as} \PYG{n+nn}{plt}
\PYG{g+gp}{\PYGZgt{}\PYGZgt{}\PYGZgt{} }\PYG{n}{h} \PYG{o}{=} \PYG{n}{plt}\PYG{o}{.}\PYG{n}{hist}\PYG{p}{(}\PYG{n}{s}\PYG{p}{,} \PYG{n}{bins}\PYG{o}{=}\PYG{l+m+mi}{100}\PYG{p}{,} \PYG{n}{normed}\PYG{o}{=}\PYG{n+nb+bp}{True}\PYG{p}{)}
\end{Verbatim}

For a one-sided t-test, how far out in the distribution does the t
statistic appear?

\begin{Verbatim}[commandchars=\\\{\}]
\PYG{g+gp}{\PYGZgt{}\PYGZgt{}\PYGZgt{} }\PYG{o}{\PYGZgt{}\PYGZgt{}}\PYG{o}{\PYGZgt{}} \PYG{n}{np}\PYG{o}{.}\PYG{n}{sum}\PYG{p}{(}\PYG{n}{s}\PYG{o}{\PYGZlt{}}\PYG{n}{t}\PYG{p}{)} \PYG{o}{/} \PYG{n+nb}{float}\PYG{p}{(}\PYG{n+nb}{len}\PYG{p}{(}\PYG{n}{s}\PYG{p}{)}\PYG{p}{)}
\PYG{g+go}{0.0090699999999999999  \PYGZsh{}random}
\end{Verbatim}

So the p-value is about 0.009, which says the null hypothesis has a
probability of about 99\% of being true.

\end{fulllineitems}

\index{triangular() (in module lib.IO.writefiles)}

\begin{fulllineitems}
\phantomsection\label{lib.IO:lib.IO.writefiles.triangular}\pysiglinewithargsret{\code{lib.IO.writefiles.}\bfcode{triangular}}{\emph{left}, \emph{mode}, \emph{right}, \emph{size=None}}{}
Draw samples from the triangular distribution.

The triangular distribution is a continuous probability distribution with
lower limit left, peak at mode, and upper limit right. Unlike the other
distributions, these parameters directly define the shape of the pdf.
\begin{description}
\item[{left}] \leavevmode{[}scalar{]}
Lower limit.

\item[{mode}] \leavevmode{[}scalar{]}
The value where the peak of the distribution occurs.
The value should fulfill the condition \code{left \textless{}= mode \textless{}= right}.

\item[{right}] \leavevmode{[}scalar{]}
Upper limit, should be larger than \emph{left}.

\item[{size}] \leavevmode{[}int or tuple of ints, optional{]}
Output shape. Default is None, in which case a single value is
returned.

\end{description}
\begin{description}
\item[{samples}] \leavevmode{[}ndarray or scalar{]}
The returned samples all lie in the interval {[}left, right{]}.

\end{description}

The probability density function for the Triangular distribution is
\begin{gather}
\begin{split}P(x;l, m, r) = \begin{cases}
\frac{2(x-l)}{(r-l)(m-l)}& \text{for $l \leq x \leq m$},\\
\frac{2(m-x)}{(r-l)(r-m)}& \text{for $m \leq x \leq r$},\\
0& \text{otherwise}.
\end{cases}\end{split}\notag
\end{gather}
The triangular distribution is often used in ill-defined problems where the
underlying distribution is not known, but some knowledge of the limits and
mode exists. Often it is used in simulations.

Draw values from the distribution and plot the histogram:

\begin{Verbatim}[commandchars=\\\{\}]
\PYG{g+gp}{\PYGZgt{}\PYGZgt{}\PYGZgt{} }\PYG{k+kn}{import} \PYG{n+nn}{matplotlib.pyplot} \PYG{k+kn}{as} \PYG{n+nn}{plt}
\PYG{g+gp}{\PYGZgt{}\PYGZgt{}\PYGZgt{} }\PYG{n}{h} \PYG{o}{=} \PYG{n}{plt}\PYG{o}{.}\PYG{n}{hist}\PYG{p}{(}\PYG{n}{np}\PYG{o}{.}\PYG{n}{random}\PYG{o}{.}\PYG{n}{triangular}\PYG{p}{(}\PYG{o}{\PYGZhy{}}\PYG{l+m+mi}{3}\PYG{p}{,} \PYG{l+m+mi}{0}\PYG{p}{,} \PYG{l+m+mi}{8}\PYG{p}{,} \PYG{l+m+mi}{100000}\PYG{p}{)}\PYG{p}{,} \PYG{n}{bins}\PYG{o}{=}\PYG{l+m+mi}{200}\PYG{p}{,}
\PYG{g+gp}{... }             \PYG{n}{normed}\PYG{o}{=}\PYG{n+nb+bp}{True}\PYG{p}{)}
\PYG{g+gp}{\PYGZgt{}\PYGZgt{}\PYGZgt{} }\PYG{n}{plt}\PYG{o}{.}\PYG{n}{show}\PYG{p}{(}\PYG{p}{)}
\end{Verbatim}

\end{fulllineitems}

\index{uniform() (in module lib.IO.writefiles)}

\begin{fulllineitems}
\phantomsection\label{lib.IO:lib.IO.writefiles.uniform}\pysiglinewithargsret{\code{lib.IO.writefiles.}\bfcode{uniform}}{\emph{low=0.0}, \emph{high=1.0}, \emph{size=1}}{}
Draw samples from a uniform distribution.

Samples are uniformly distributed over the half-open interval
\code{{[}low, high)} (includes low, but excludes high).  In other words,
any value within the given interval is equally likely to be drawn
by \emph{uniform}.
\begin{description}
\item[{low}] \leavevmode{[}float, optional{]}
Lower boundary of the output interval.  All values generated will be
greater than or equal to low.  The default value is 0.

\item[{high}] \leavevmode{[}float{]}
Upper boundary of the output interval.  All values generated will be
less than high.  The default value is 1.0.

\item[{size}] \leavevmode{[}int or tuple of ints, optional{]}
Shape of output.  If the given size is, for example, (m,n,k),
m*n*k samples are generated.  If no shape is specified, a single sample
is returned.

\end{description}
\begin{description}
\item[{out}] \leavevmode{[}ndarray{]}
Drawn samples, with shape \emph{size}.

\end{description}

randint : Discrete uniform distribution, yielding integers.
random\_integers : Discrete uniform distribution over the closed
\begin{quote}

interval \code{{[}low, high{]}}.
\end{quote}

random\_sample : Floats uniformly distributed over \code{{[}0, 1)}.
random : Alias for \emph{random\_sample}.
rand : Convenience function that accepts dimensions as input, e.g.,
\begin{quote}

\code{rand(2,2)} would generate a 2-by-2 array of floats,
uniformly distributed over \code{{[}0, 1)}.
\end{quote}

The probability density function of the uniform distribution is
\begin{gather}
\begin{split}p(x) = \frac{1}{b - a}\end{split}\notag
\end{gather}
anywhere within the interval \code{{[}a, b)}, and zero elsewhere.

Draw samples from the distribution:

\begin{Verbatim}[commandchars=\\\{\}]
\PYG{g+gp}{\PYGZgt{}\PYGZgt{}\PYGZgt{} }\PYG{n}{s} \PYG{o}{=} \PYG{n}{np}\PYG{o}{.}\PYG{n}{random}\PYG{o}{.}\PYG{n}{uniform}\PYG{p}{(}\PYG{o}{\PYGZhy{}}\PYG{l+m+mi}{1}\PYG{p}{,}\PYG{l+m+mi}{0}\PYG{p}{,}\PYG{l+m+mi}{1000}\PYG{p}{)}
\end{Verbatim}

All values are within the given interval:

\begin{Verbatim}[commandchars=\\\{\}]
\PYG{g+gp}{\PYGZgt{}\PYGZgt{}\PYGZgt{} }\PYG{n}{np}\PYG{o}{.}\PYG{n}{all}\PYG{p}{(}\PYG{n}{s} \PYG{o}{\PYGZgt{}}\PYG{o}{=} \PYG{o}{\PYGZhy{}}\PYG{l+m+mi}{1}\PYG{p}{)}
\PYG{g+go}{True}
\PYG{g+gp}{\PYGZgt{}\PYGZgt{}\PYGZgt{} }\PYG{n}{np}\PYG{o}{.}\PYG{n}{all}\PYG{p}{(}\PYG{n}{s} \PYG{o}{\PYGZlt{}} \PYG{l+m+mi}{0}\PYG{p}{)}
\PYG{g+go}{True}
\end{Verbatim}

Display the histogram of the samples, along with the
probability density function:

\begin{Verbatim}[commandchars=\\\{\}]
\PYG{g+gp}{\PYGZgt{}\PYGZgt{}\PYGZgt{} }\PYG{k+kn}{import} \PYG{n+nn}{matplotlib.pyplot} \PYG{k+kn}{as} \PYG{n+nn}{plt}
\PYG{g+gp}{\PYGZgt{}\PYGZgt{}\PYGZgt{} }\PYG{n}{count}\PYG{p}{,} \PYG{n}{bins}\PYG{p}{,} \PYG{n}{ignored} \PYG{o}{=} \PYG{n}{plt}\PYG{o}{.}\PYG{n}{hist}\PYG{p}{(}\PYG{n}{s}\PYG{p}{,} \PYG{l+m+mi}{15}\PYG{p}{,} \PYG{n}{normed}\PYG{o}{=}\PYG{n+nb+bp}{True}\PYG{p}{)}
\PYG{g+gp}{\PYGZgt{}\PYGZgt{}\PYGZgt{} }\PYG{n}{plt}\PYG{o}{.}\PYG{n}{plot}\PYG{p}{(}\PYG{n}{bins}\PYG{p}{,} \PYG{n}{np}\PYG{o}{.}\PYG{n}{ones\PYGZus{}like}\PYG{p}{(}\PYG{n}{bins}\PYG{p}{)}\PYG{p}{,} \PYG{n}{linewidth}\PYG{o}{=}\PYG{l+m+mi}{2}\PYG{p}{,} \PYG{n}{color}\PYG{o}{=}\PYG{l+s}{\PYGZsq{}}\PYG{l+s}{r}\PYG{l+s}{\PYGZsq{}}\PYG{p}{)}
\PYG{g+gp}{\PYGZgt{}\PYGZgt{}\PYGZgt{} }\PYG{n}{plt}\PYG{o}{.}\PYG{n}{show}\PYG{p}{(}\PYG{p}{)}
\end{Verbatim}

\end{fulllineitems}

\index{vonmises() (in module lib.IO.writefiles)}

\begin{fulllineitems}
\phantomsection\label{lib.IO:lib.IO.writefiles.vonmises}\pysiglinewithargsret{\code{lib.IO.writefiles.}\bfcode{vonmises}}{\emph{mu}, \emph{kappa}, \emph{size=None}}{}
Draw samples from a von Mises distribution.

Samples are drawn from a von Mises distribution with specified mode
(mu) and dispersion (kappa), on the interval {[}-pi, pi{]}.

The von Mises distribution (also known as the circular normal
distribution) is a continuous probability distribution on the unit
circle.  It may be thought of as the circular analogue of the normal
distribution.
\begin{description}
\item[{mu}] \leavevmode{[}float{]}
Mode (``center'') of the distribution.

\item[{kappa}] \leavevmode{[}float{]}
Dispersion of the distribution, has to be \textgreater{}=0.

\item[{size}] \leavevmode{[}int or tuple of int{]}
Output shape.  If the given shape is, e.g., \code{(m, n, k)}, then
\code{m * n * k} samples are drawn.

\end{description}
\begin{description}
\item[{samples}] \leavevmode{[}scalar or ndarray{]}
The returned samples, which are in the interval {[}-pi, pi{]}.

\end{description}
\begin{description}
\item[{scipy.stats.distributions.vonmises}] \leavevmode{[}probability density function,{]}
distribution, or cumulative density function, etc.

\end{description}

The probability density for the von Mises distribution is
\begin{gather}
\begin{split}p(x) = \frac{e^{\kappa cos(x-\mu)}}{2\pi I_0(\kappa)},\end{split}\notag
\end{gather}
where \(\mu\) is the mode and \(\kappa\) the dispersion,
and \(I_0(\kappa)\) is the modified Bessel function of order 0.

The von Mises is named for Richard Edler von Mises, who was born in
Austria-Hungary, in what is now the Ukraine.  He fled to the United
States in 1939 and became a professor at Harvard.  He worked in
probability theory, aerodynamics, fluid mechanics, and philosophy of
science.

Abramowitz, M. and Stegun, I. A. (ed.), \emph{Handbook of Mathematical
Functions}, New York: Dover, 1965.

von Mises, R., \emph{Mathematical Theory of Probability and Statistics},
New York: Academic Press, 1964.

Draw samples from the distribution:

\begin{Verbatim}[commandchars=\\\{\}]
\PYG{g+gp}{\PYGZgt{}\PYGZgt{}\PYGZgt{} }\PYG{n}{mu}\PYG{p}{,} \PYG{n}{kappa} \PYG{o}{=} \PYG{l+m+mf}{0.0}\PYG{p}{,} \PYG{l+m+mf}{4.0} \PYG{c}{\PYGZsh{} mean and dispersion}
\PYG{g+gp}{\PYGZgt{}\PYGZgt{}\PYGZgt{} }\PYG{n}{s} \PYG{o}{=} \PYG{n}{np}\PYG{o}{.}\PYG{n}{random}\PYG{o}{.}\PYG{n}{vonmises}\PYG{p}{(}\PYG{n}{mu}\PYG{p}{,} \PYG{n}{kappa}\PYG{p}{,} \PYG{l+m+mi}{1000}\PYG{p}{)}
\end{Verbatim}

Display the histogram of the samples, along with
the probability density function:

\begin{Verbatim}[commandchars=\\\{\}]
\PYG{g+gp}{\PYGZgt{}\PYGZgt{}\PYGZgt{} }\PYG{k+kn}{import} \PYG{n+nn}{matplotlib.pyplot} \PYG{k+kn}{as} \PYG{n+nn}{plt}
\PYG{g+gp}{\PYGZgt{}\PYGZgt{}\PYGZgt{} }\PYG{k+kn}{import} \PYG{n+nn}{scipy.special} \PYG{k+kn}{as} \PYG{n+nn}{sps}
\PYG{g+gp}{\PYGZgt{}\PYGZgt{}\PYGZgt{} }\PYG{n}{count}\PYG{p}{,} \PYG{n}{bins}\PYG{p}{,} \PYG{n}{ignored} \PYG{o}{=} \PYG{n}{plt}\PYG{o}{.}\PYG{n}{hist}\PYG{p}{(}\PYG{n}{s}\PYG{p}{,} \PYG{l+m+mi}{50}\PYG{p}{,} \PYG{n}{normed}\PYG{o}{=}\PYG{n+nb+bp}{True}\PYG{p}{)}
\PYG{g+gp}{\PYGZgt{}\PYGZgt{}\PYGZgt{} }\PYG{n}{x} \PYG{o}{=} \PYG{n}{np}\PYG{o}{.}\PYG{n}{arange}\PYG{p}{(}\PYG{o}{\PYGZhy{}}\PYG{n}{np}\PYG{o}{.}\PYG{n}{pi}\PYG{p}{,} \PYG{n}{np}\PYG{o}{.}\PYG{n}{pi}\PYG{p}{,} \PYG{l+m+mi}{2}\PYG{o}{*}\PYG{n}{np}\PYG{o}{.}\PYG{n}{pi}\PYG{o}{/}\PYG{l+m+mf}{50.}\PYG{p}{)}
\PYG{g+gp}{\PYGZgt{}\PYGZgt{}\PYGZgt{} }\PYG{n}{y} \PYG{o}{=} \PYG{o}{\PYGZhy{}}\PYG{n}{np}\PYG{o}{.}\PYG{n}{exp}\PYG{p}{(}\PYG{n}{kappa}\PYG{o}{*}\PYG{n}{np}\PYG{o}{.}\PYG{n}{cos}\PYG{p}{(}\PYG{n}{x}\PYG{o}{\PYGZhy{}}\PYG{n}{mu}\PYG{p}{)}\PYG{p}{)}\PYG{o}{/}\PYG{p}{(}\PYG{l+m+mi}{2}\PYG{o}{*}\PYG{n}{np}\PYG{o}{.}\PYG{n}{pi}\PYG{o}{*}\PYG{n}{sps}\PYG{o}{.}\PYG{n}{jn}\PYG{p}{(}\PYG{l+m+mi}{0}\PYG{p}{,}\PYG{n}{kappa}\PYG{p}{)}\PYG{p}{)}
\PYG{g+gp}{\PYGZgt{}\PYGZgt{}\PYGZgt{} }\PYG{n}{plt}\PYG{o}{.}\PYG{n}{plot}\PYG{p}{(}\PYG{n}{x}\PYG{p}{,} \PYG{n}{y}\PYG{o}{/}\PYG{n+nb}{max}\PYG{p}{(}\PYG{n}{y}\PYG{p}{)}\PYG{p}{,} \PYG{n}{linewidth}\PYG{o}{=}\PYG{l+m+mi}{2}\PYG{p}{,} \PYG{n}{color}\PYG{o}{=}\PYG{l+s}{\PYGZsq{}}\PYG{l+s}{r}\PYG{l+s}{\PYGZsq{}}\PYG{p}{)}
\PYG{g+gp}{\PYGZgt{}\PYGZgt{}\PYGZgt{} }\PYG{n}{plt}\PYG{o}{.}\PYG{n}{show}\PYG{p}{(}\PYG{p}{)}
\end{Verbatim}

\end{fulllineitems}

\index{wald() (in module lib.IO.writefiles)}

\begin{fulllineitems}
\phantomsection\label{lib.IO:lib.IO.writefiles.wald}\pysiglinewithargsret{\code{lib.IO.writefiles.}\bfcode{wald}}{\emph{mean}, \emph{scale}, \emph{size=None}}{}
Draw samples from a Wald, or Inverse Gaussian, distribution.

As the scale approaches infinity, the distribution becomes more like a
Gaussian.

Some references claim that the Wald is an Inverse Gaussian with mean=1, but
this is by no means universal.

The Inverse Gaussian distribution was first studied in relationship to
Brownian motion. In 1956 M.C.K. Tweedie used the name Inverse Gaussian
because there is an inverse relationship between the time to cover a unit
distance and distance covered in unit time.
\begin{description}
\item[{mean}] \leavevmode{[}scalar{]}
Distribution mean, should be \textgreater{} 0.

\item[{scale}] \leavevmode{[}scalar{]}
Scale parameter, should be \textgreater{}= 0.

\item[{size}] \leavevmode{[}int or tuple of ints, optional{]}
Output shape. Default is None, in which case a single value is
returned.

\end{description}
\begin{description}
\item[{samples}] \leavevmode{[}ndarray or scalar{]}
Drawn sample, all greater than zero.

\end{description}

The probability density function for the Wald distribution is
\begin{gather}
\begin{split}P(x;mean,scale) = \sqrt{\frac{scale}{2\pi x^3}}e^
\frac{-scale(x-mean)^2}{2\cdotp mean^2x}\end{split}\notag
\end{gather}
As noted above the Inverse Gaussian distribution first arise from attempts
to model Brownian Motion. It is also a competitor to the Weibull for use in
reliability modeling and modeling stock returns and interest rate
processes.

Draw values from the distribution and plot the histogram:

\begin{Verbatim}[commandchars=\\\{\}]
\PYG{g+gp}{\PYGZgt{}\PYGZgt{}\PYGZgt{} }\PYG{k+kn}{import} \PYG{n+nn}{matplotlib.pyplot} \PYG{k+kn}{as} \PYG{n+nn}{plt}
\PYG{g+gp}{\PYGZgt{}\PYGZgt{}\PYGZgt{} }\PYG{n}{h} \PYG{o}{=} \PYG{n}{plt}\PYG{o}{.}\PYG{n}{hist}\PYG{p}{(}\PYG{n}{np}\PYG{o}{.}\PYG{n}{random}\PYG{o}{.}\PYG{n}{wald}\PYG{p}{(}\PYG{l+m+mi}{3}\PYG{p}{,} \PYG{l+m+mi}{2}\PYG{p}{,} \PYG{l+m+mi}{100000}\PYG{p}{)}\PYG{p}{,} \PYG{n}{bins}\PYG{o}{=}\PYG{l+m+mi}{200}\PYG{p}{,} \PYG{n}{normed}\PYG{o}{=}\PYG{n+nb+bp}{True}\PYG{p}{)}
\PYG{g+gp}{\PYGZgt{}\PYGZgt{}\PYGZgt{} }\PYG{n}{plt}\PYG{o}{.}\PYG{n}{show}\PYG{p}{(}\PYG{p}{)}
\end{Verbatim}

\end{fulllineitems}

\index{weibull() (in module lib.IO.writefiles)}

\begin{fulllineitems}
\phantomsection\label{lib.IO:lib.IO.writefiles.weibull}\pysiglinewithargsret{\code{lib.IO.writefiles.}\bfcode{weibull}}{\emph{a}, \emph{size=None}}{}
Weibull distribution.

Draw samples from a 1-parameter Weibull distribution with the given
shape parameter \emph{a}.
\begin{gather}
\begin{split}X = (-ln(U))^{1/a}\end{split}\notag
\end{gather}
Here, U is drawn from the uniform distribution over (0,1{]}.

The more common 2-parameter Weibull, including a scale parameter
\(\lambda\) is just \(X = \lambda(-ln(U))^{1/a}\).
\begin{description}
\item[{a}] \leavevmode{[}float{]}
Shape of the distribution.

\item[{size}] \leavevmode{[}tuple of ints{]}
Output shape.  If the given shape is, e.g., \code{(m, n, k)}, then
\code{m * n * k} samples are drawn.

\end{description}

scipy.stats.distributions.weibull\_max
scipy.stats.distributions.weibull\_min
scipy.stats.distributions.genextreme
gumbel

The Weibull (or Type III asymptotic extreme value distribution for smallest
values, SEV Type III, or Rosin-Rammler distribution) is one of a class of
Generalized Extreme Value (GEV) distributions used in modeling extreme
value problems.  This class includes the Gumbel and Frechet distributions.

The probability density for the Weibull distribution is
\begin{gather}
\begin{split}p(x) = \frac{a}
{\lambda}(\frac{x}{\lambda})^{a-1}e^{-(x/\lambda)^a},\end{split}\notag
\end{gather}
where \(a\) is the shape and \(\lambda\) the scale.

The function has its peak (the mode) at
\(\lambda(\frac{a-1}{a})^{1/a}\).

When \code{a = 1}, the Weibull distribution reduces to the exponential
distribution.

Draw samples from the distribution:

\begin{Verbatim}[commandchars=\\\{\}]
\PYG{g+gp}{\PYGZgt{}\PYGZgt{}\PYGZgt{} }\PYG{n}{a} \PYG{o}{=} \PYG{l+m+mf}{5.} \PYG{c}{\PYGZsh{} shape}
\PYG{g+gp}{\PYGZgt{}\PYGZgt{}\PYGZgt{} }\PYG{n}{s} \PYG{o}{=} \PYG{n}{np}\PYG{o}{.}\PYG{n}{random}\PYG{o}{.}\PYG{n}{weibull}\PYG{p}{(}\PYG{n}{a}\PYG{p}{,} \PYG{l+m+mi}{1000}\PYG{p}{)}
\end{Verbatim}

Display the histogram of the samples, along with
the probability density function:

\begin{Verbatim}[commandchars=\\\{\}]
\PYG{g+gp}{\PYGZgt{}\PYGZgt{}\PYGZgt{} }\PYG{k+kn}{import} \PYG{n+nn}{matplotlib.pyplot} \PYG{k+kn}{as} \PYG{n+nn}{plt}
\PYG{g+gp}{\PYGZgt{}\PYGZgt{}\PYGZgt{} }\PYG{n}{x} \PYG{o}{=} \PYG{n}{np}\PYG{o}{.}\PYG{n}{arange}\PYG{p}{(}\PYG{l+m+mi}{1}\PYG{p}{,}\PYG{l+m+mf}{100.}\PYG{p}{)}\PYG{o}{/}\PYG{l+m+mf}{50.}
\PYG{g+gp}{\PYGZgt{}\PYGZgt{}\PYGZgt{} }\PYG{k}{def} \PYG{n+nf}{weib}\PYG{p}{(}\PYG{n}{x}\PYG{p}{,}\PYG{n}{n}\PYG{p}{,}\PYG{n}{a}\PYG{p}{)}\PYG{p}{:}
\PYG{g+gp}{... }    \PYG{k}{return} \PYG{p}{(}\PYG{n}{a} \PYG{o}{/} \PYG{n}{n}\PYG{p}{)} \PYG{o}{*} \PYG{p}{(}\PYG{n}{x} \PYG{o}{/} \PYG{n}{n}\PYG{p}{)}\PYG{o}{*}\PYG{o}{*}\PYG{p}{(}\PYG{n}{a} \PYG{o}{\PYGZhy{}} \PYG{l+m+mi}{1}\PYG{p}{)} \PYG{o}{*} \PYG{n}{np}\PYG{o}{.}\PYG{n}{exp}\PYG{p}{(}\PYG{o}{\PYGZhy{}}\PYG{p}{(}\PYG{n}{x} \PYG{o}{/} \PYG{n}{n}\PYG{p}{)}\PYG{o}{*}\PYG{o}{*}\PYG{n}{a}\PYG{p}{)}
\end{Verbatim}

\begin{Verbatim}[commandchars=\\\{\}]
\PYG{g+gp}{\PYGZgt{}\PYGZgt{}\PYGZgt{} }\PYG{n}{count}\PYG{p}{,} \PYG{n}{bins}\PYG{p}{,} \PYG{n}{ignored} \PYG{o}{=} \PYG{n}{plt}\PYG{o}{.}\PYG{n}{hist}\PYG{p}{(}\PYG{n}{np}\PYG{o}{.}\PYG{n}{random}\PYG{o}{.}\PYG{n}{weibull}\PYG{p}{(}\PYG{l+m+mf}{5.}\PYG{p}{,}\PYG{l+m+mi}{1000}\PYG{p}{)}\PYG{p}{)}
\PYG{g+gp}{\PYGZgt{}\PYGZgt{}\PYGZgt{} }\PYG{n}{x} \PYG{o}{=} \PYG{n}{np}\PYG{o}{.}\PYG{n}{arange}\PYG{p}{(}\PYG{l+m+mi}{1}\PYG{p}{,}\PYG{l+m+mf}{100.}\PYG{p}{)}\PYG{o}{/}\PYG{l+m+mf}{50.}
\PYG{g+gp}{\PYGZgt{}\PYGZgt{}\PYGZgt{} }\PYG{n}{scale} \PYG{o}{=} \PYG{n}{count}\PYG{o}{.}\PYG{n}{max}\PYG{p}{(}\PYG{p}{)}\PYG{o}{/}\PYG{n}{weib}\PYG{p}{(}\PYG{n}{x}\PYG{p}{,} \PYG{l+m+mf}{1.}\PYG{p}{,} \PYG{l+m+mf}{5.}\PYG{p}{)}\PYG{o}{.}\PYG{n}{max}\PYG{p}{(}\PYG{p}{)}
\PYG{g+gp}{\PYGZgt{}\PYGZgt{}\PYGZgt{} }\PYG{n}{plt}\PYG{o}{.}\PYG{n}{plot}\PYG{p}{(}\PYG{n}{x}\PYG{p}{,} \PYG{n}{weib}\PYG{p}{(}\PYG{n}{x}\PYG{p}{,} \PYG{l+m+mf}{1.}\PYG{p}{,} \PYG{l+m+mf}{5.}\PYG{p}{)}\PYG{o}{*}\PYG{n}{scale}\PYG{p}{)}
\PYG{g+gp}{\PYGZgt{}\PYGZgt{}\PYGZgt{} }\PYG{n}{plt}\PYG{o}{.}\PYG{n}{show}\PYG{p}{(}\PYG{p}{)}
\end{Verbatim}

\end{fulllineitems}

\index{writeAllFilesAndCreateResFolder() (in module lib.IO.writefiles)}

\begin{fulllineitems}
\phantomsection\label{lib.IO:lib.IO.writefiles.writeAllFilesAndCreateResFolder}\pysiglinewithargsret{\code{lib.IO.writefiles.}\bfcode{writeAllFilesAndCreateResFolder}}{\emph{pathFile}, \emph{resFolderName}, \emph{cats}, \emph{rcts}, \emph{food}, \emph{spontRatio=None}, \emph{kspontass=None}, \emph{kspontdiss=None}, \emph{conf=False}}{}
\end{fulllineitems}

\index{write\_acsCatalysis\_file() (in module lib.IO.writefiles)}

\begin{fulllineitems}
\phantomsection\label{lib.IO:lib.IO.writefiles.write_acsCatalysis_file}\pysiglinewithargsret{\code{lib.IO.writefiles.}\bfcode{write\_acsCatalysis\_file}}{\emph{path\_file}, \emph{catStr}}{}
\end{fulllineitems}

\index{write\_acsReactions\_file() (in module lib.IO.writefiles)}

\begin{fulllineitems}
\phantomsection\label{lib.IO:lib.IO.writefiles.write_acsReactions_file}\pysiglinewithargsret{\code{lib.IO.writefiles.}\bfcode{write\_acsReactions\_file}}{\emph{path\_file}, \emph{rctStr}, \emph{spontRatio=None}, \emph{kspontass=None}, \emph{kspontdiss=None}}{}
\end{fulllineitems}

\index{write\_acsms\_file() (in module lib.IO.writefiles)}

\begin{fulllineitems}
\phantomsection\label{lib.IO:lib.IO.writefiles.write_acsms_file}\pysiglinewithargsret{\code{lib.IO.writefiles.}\bfcode{write\_acsms\_file}}{\emph{path\_file}, \emph{nGen=10}, \emph{nSim=1}, \emph{nSec=1000}, \emph{nRct=200000000}, \emph{nH=0}, \emph{nA=0}, \emph{rs=0}, \emph{dl=0}, \emph{tssi=10}, \emph{ftsi=0}, \emph{nspmt=1}, \emph{lfds=13}, \emph{oc=0.0001}, \emph{ecc=0}, \emph{alf='AB'}, \emph{v=1e-18}, \emph{vg=0}, \emph{sd=0}, \emph{nrg=0}, \emph{rse=0}, \emph{ncml=2}, \emph{P=0.00103306}, \emph{cp=0.5}, \emph{mrevrct=0}, \emph{rr=0}, \emph{rrr=0}, \emph{sr=0}, \emph{K\_ass=50}, \emph{K\_diss=25}, \emph{K\_cpx=50}, \emph{K\_cpxDiss=1}, \emph{K\_nrg=0}, \emph{K\_nrg\_decay=0}, \emph{K\_spont\_ass=0}, \emph{K\_spont\_diss=0}, \emph{moleculeDecay\_KineticConstant=0.02}, \emph{diffusion\_contribute=0}, \emph{solubility\_threshold=0}, \emph{influx\_rate=0}, \emph{maxLOut=3}, \emph{fileAmountSaveInterval=10}, \emph{saveRtcInfo=1}, \emph{randInitSpeciesConc=0}, \emph{tmpTheta=0}}{}
\end{fulllineitems}

\index{write\_and\_createInfluxFile() (in module lib.IO.writefiles)}

\begin{fulllineitems}
\phantomsection\label{lib.IO:lib.IO.writefiles.write_and_createInfluxFile}\pysiglinewithargsret{\code{lib.IO.writefiles.}\bfcode{write\_and\_createInfluxFile}}{\emph{path\_file}, \emph{tmpFood}}{}
\end{fulllineitems}

\index{write\_and\_create\_std\_nrgFile() (in module lib.IO.writefiles)}

\begin{fulllineitems}
\phantomsection\label{lib.IO:lib.IO.writefiles.write_and_create_std_nrgFile}\pysiglinewithargsret{\code{lib.IO.writefiles.}\bfcode{write\_and\_create\_std\_nrgFile}}{\emph{path\_file}}{}
\end{fulllineitems}

\index{write\_init\_raf\_all() (in module lib.IO.writefiles)}

\begin{fulllineitems}
\phantomsection\label{lib.IO:lib.IO.writefiles.write_init_raf_all}\pysiglinewithargsret{\code{lib.IO.writefiles.}\bfcode{write\_init\_raf\_all}}{\emph{fid}, \emph{rafinfo}, \emph{folder}, \emph{rcts}, \emph{cats}}{}
\end{fulllineitems}

\index{write\_init\_raf\_list() (in module lib.IO.writefiles)}

\begin{fulllineitems}
\phantomsection\label{lib.IO:lib.IO.writefiles.write_init_raf_list}\pysiglinewithargsret{\code{lib.IO.writefiles.}\bfcode{write\_init\_raf\_list}}{\emph{fid}, \emph{rafinfo}, \emph{folder}}{}
\end{fulllineitems}

\index{zipf() (in module lib.IO.writefiles)}

\begin{fulllineitems}
\phantomsection\label{lib.IO:lib.IO.writefiles.zipf}\pysiglinewithargsret{\code{lib.IO.writefiles.}\bfcode{zipf}}{\emph{a}, \emph{size=None}}{}
Draw samples from a Zipf distribution.

Samples are drawn from a Zipf distribution with specified parameter
\emph{a} \textgreater{} 1.

The Zipf distribution (also known as the zeta distribution) is a
continuous probability distribution that satisfies Zipf's law: the
frequency of an item is inversely proportional to its rank in a
frequency table.
\begin{description}
\item[{a}] \leavevmode{[}float \textgreater{} 1{]}
Distribution parameter.

\item[{size}] \leavevmode{[}int or tuple of int, optional{]}
Output shape.  If the given shape is, e.g., \code{(m, n, k)}, then
\code{m * n * k} samples are drawn; a single integer is equivalent in
its result to providing a mono-tuple, i.e., a 1-D array of length
\emph{size} is returned.  The default is None, in which case a single
scalar is returned.

\end{description}
\begin{description}
\item[{samples}] \leavevmode{[}scalar or ndarray{]}
The returned samples are greater than or equal to one.

\end{description}
\begin{description}
\item[{scipy.stats.distributions.zipf}] \leavevmode{[}probability density function,{]}
distribution, or cumulative density function, etc.

\end{description}

The probability density for the Zipf distribution is
\begin{gather}
\begin{split}p(x) = \frac{x^{-a}}{\zeta(a)},\end{split}\notag
\end{gather}
where \(\zeta\) is the Riemann Zeta function.

It is named for the American linguist George Kingsley Zipf, who noted
that the frequency of any word in a sample of a language is inversely
proportional to its rank in the frequency table.

Zipf, G. K., \emph{Selected Studies of the Principle of Relative Frequency
in Language}, Cambridge, MA: Harvard Univ. Press, 1932.

Draw samples from the distribution:

\begin{Verbatim}[commandchars=\\\{\}]
\PYG{g+gp}{\PYGZgt{}\PYGZgt{}\PYGZgt{} }\PYG{n}{a} \PYG{o}{=} \PYG{l+m+mf}{2.} \PYG{c}{\PYGZsh{} parameter}
\PYG{g+gp}{\PYGZgt{}\PYGZgt{}\PYGZgt{} }\PYG{n}{s} \PYG{o}{=} \PYG{n}{np}\PYG{o}{.}\PYG{n}{random}\PYG{o}{.}\PYG{n}{zipf}\PYG{p}{(}\PYG{n}{a}\PYG{p}{,} \PYG{l+m+mi}{1000}\PYG{p}{)}
\end{Verbatim}

Display the histogram of the samples, along with
the probability density function:

\begin{Verbatim}[commandchars=\\\{\}]
\PYG{g+gp}{\PYGZgt{}\PYGZgt{}\PYGZgt{} }\PYG{k+kn}{import} \PYG{n+nn}{matplotlib.pyplot} \PYG{k+kn}{as} \PYG{n+nn}{plt}
\PYG{g+gp}{\PYGZgt{}\PYGZgt{}\PYGZgt{} }\PYG{k+kn}{import} \PYG{n+nn}{scipy.special} \PYG{k+kn}{as} \PYG{n+nn}{sps}
\PYG{g+go}{Truncate s values at 50 so plot is interesting}
\PYG{g+gp}{\PYGZgt{}\PYGZgt{}\PYGZgt{} }\PYG{n}{count}\PYG{p}{,} \PYG{n}{bins}\PYG{p}{,} \PYG{n}{ignored} \PYG{o}{=} \PYG{n}{plt}\PYG{o}{.}\PYG{n}{hist}\PYG{p}{(}\PYG{n}{s}\PYG{p}{[}\PYG{n}{s}\PYG{o}{\PYGZlt{}}\PYG{l+m+mi}{50}\PYG{p}{]}\PYG{p}{,} \PYG{l+m+mi}{50}\PYG{p}{,} \PYG{n}{normed}\PYG{o}{=}\PYG{n+nb+bp}{True}\PYG{p}{)}
\PYG{g+gp}{\PYGZgt{}\PYGZgt{}\PYGZgt{} }\PYG{n}{x} \PYG{o}{=} \PYG{n}{np}\PYG{o}{.}\PYG{n}{arange}\PYG{p}{(}\PYG{l+m+mf}{1.}\PYG{p}{,} \PYG{l+m+mf}{50.}\PYG{p}{)}
\PYG{g+gp}{\PYGZgt{}\PYGZgt{}\PYGZgt{} }\PYG{n}{y} \PYG{o}{=} \PYG{n}{x}\PYG{o}{*}\PYG{o}{*}\PYG{p}{(}\PYG{o}{\PYGZhy{}}\PYG{n}{a}\PYG{p}{)}\PYG{o}{/}\PYG{n}{sps}\PYG{o}{.}\PYG{n}{zetac}\PYG{p}{(}\PYG{n}{a}\PYG{p}{)}
\PYG{g+gp}{\PYGZgt{}\PYGZgt{}\PYGZgt{} }\PYG{n}{plt}\PYG{o}{.}\PYG{n}{plot}\PYG{p}{(}\PYG{n}{x}\PYG{p}{,} \PYG{n}{y}\PYG{o}{/}\PYG{n+nb}{max}\PYG{p}{(}\PYG{n}{y}\PYG{p}{)}\PYG{p}{,} \PYG{n}{linewidth}\PYG{o}{=}\PYG{l+m+mi}{2}\PYG{p}{,} \PYG{n}{color}\PYG{o}{=}\PYG{l+s}{\PYGZsq{}}\PYG{l+s}{r}\PYG{l+s}{\PYGZsq{}}\PYG{p}{)}
\PYG{g+gp}{\PYGZgt{}\PYGZgt{}\PYGZgt{} }\PYG{n}{plt}\PYG{o}{.}\PYG{n}{show}\PYG{p}{(}\PYG{p}{)}
\end{Verbatim}

\end{fulllineitems}



\subsection{dyn Package}
\label{lib.dyn::doc}\label{lib.dyn:dyn-package}

\subsubsection{\texttt{dynamics} Module}
\label{lib.dyn:module-lib.dyn.dynamics}\label{lib.dyn:dynamics-module}\index{lib.dyn.dynamics (module)}\index{beta() (in module lib.dyn.dynamics)}

\begin{fulllineitems}
\phantomsection\label{lib.dyn:lib.dyn.dynamics.beta}\pysiglinewithargsret{\code{lib.dyn.dynamics.}\bfcode{beta}}{\emph{a}, \emph{b}, \emph{size=None}}{}
The Beta distribution over \code{{[}0, 1{]}}.

The Beta distribution is a special case of the Dirichlet distribution,
and is related to the Gamma distribution.  It has the probability
distribution function
\begin{gather}
\begin{split}f(x; a,b) = \frac{1}{B(\alpha, \beta)} x^{\alpha - 1}
(1 - x)^{\beta - 1},\end{split}\notag
\end{gather}
where the normalisation, B, is the beta function,
\begin{gather}
\begin{split}B(\alpha, \beta) = \int_0^1 t^{\alpha - 1}
(1 - t)^{\beta - 1} dt.\end{split}\notag
\end{gather}
It is often seen in Bayesian inference and order statistics.
\begin{description}
\item[{a}] \leavevmode{[}float{]}
Alpha, non-negative.

\item[{b}] \leavevmode{[}float{]}
Beta, non-negative.

\item[{size}] \leavevmode{[}tuple of ints, optional{]}
The number of samples to draw.  The output is packed according to
the size given.

\end{description}
\begin{description}
\item[{out}] \leavevmode{[}ndarray{]}
Array of the given shape, containing values drawn from a
Beta distribution.

\end{description}

\end{fulllineitems}

\index{binomial() (in module lib.dyn.dynamics)}

\begin{fulllineitems}
\phantomsection\label{lib.dyn:lib.dyn.dynamics.binomial}\pysiglinewithargsret{\code{lib.dyn.dynamics.}\bfcode{binomial}}{\emph{n}, \emph{p}, \emph{size=None}}{}
Draw samples from a binomial distribution.

Samples are drawn from a Binomial distribution with specified
parameters, n trials and p probability of success where
n an integer \textgreater{}= 0 and p is in the interval {[}0,1{]}. (n may be
input as a float, but it is truncated to an integer in use)
\begin{description}
\item[{n}] \leavevmode{[}float (but truncated to an integer){]}
parameter, \textgreater{}= 0.

\item[{p}] \leavevmode{[}float{]}
parameter, \textgreater{}= 0 and \textless{}=1.

\item[{size}] \leavevmode{[}\{tuple, int\}{]}
Output shape.  If the given shape is, e.g., \code{(m, n, k)}, then
\code{m * n * k} samples are drawn.

\end{description}
\begin{description}
\item[{samples}] \leavevmode{[}\{ndarray, scalar\}{]}
where the values are all integers in  {[}0, n{]}.

\end{description}
\begin{description}
\item[{scipy.stats.distributions.binom}] \leavevmode{[}probability density function,{]}
distribution or cumulative density function, etc.

\end{description}

The probability density for the Binomial distribution is
\begin{gather}
\begin{split}P(N) = \binom{n}{N}p^N(1-p)^{n-N},\end{split}\notag
\end{gather}
where \(n\) is the number of trials, \(p\) is the probability
of success, and \(N\) is the number of successes.

When estimating the standard error of a proportion in a population by
using a random sample, the normal distribution works well unless the
product p*n \textless{}=5, where p = population proportion estimate, and n =
number of samples, in which case the binomial distribution is used
instead. For example, a sample of 15 people shows 4 who are left
handed, and 11 who are right handed. Then p = 4/15 = 27\%. 0.27*15 = 4,
so the binomial distribution should be used in this case.

Draw samples from the distribution:

\begin{Verbatim}[commandchars=\\\{\}]
\PYG{g+gp}{\PYGZgt{}\PYGZgt{}\PYGZgt{} }\PYG{n}{n}\PYG{p}{,} \PYG{n}{p} \PYG{o}{=} \PYG{l+m+mi}{10}\PYG{p}{,} \PYG{o}{.}\PYG{l+m+mi}{5} \PYG{c}{\PYGZsh{} number of trials, probability of each trial}
\PYG{g+gp}{\PYGZgt{}\PYGZgt{}\PYGZgt{} }\PYG{n}{s} \PYG{o}{=} \PYG{n}{np}\PYG{o}{.}\PYG{n}{random}\PYG{o}{.}\PYG{n}{binomial}\PYG{p}{(}\PYG{n}{n}\PYG{p}{,} \PYG{n}{p}\PYG{p}{,} \PYG{l+m+mi}{1000}\PYG{p}{)}
\PYG{g+go}{\PYGZsh{} result of flipping a coin 10 times, tested 1000 times.}
\end{Verbatim}

A real world example. A company drills 9 wild-cat oil exploration
wells, each with an estimated probability of success of 0.1. All nine
wells fail. What is the probability of that happening?

Let's do 20,000 trials of the model, and count the number that
generate zero positive results.

\begin{Verbatim}[commandchars=\\\{\}]
\PYG{g+gp}{\PYGZgt{}\PYGZgt{}\PYGZgt{} }\PYG{n+nb}{sum}\PYG{p}{(}\PYG{n}{np}\PYG{o}{.}\PYG{n}{random}\PYG{o}{.}\PYG{n}{binomial}\PYG{p}{(}\PYG{l+m+mi}{9}\PYG{p}{,}\PYG{l+m+mf}{0.1}\PYG{p}{,}\PYG{l+m+mi}{20000}\PYG{p}{)}\PYG{o}{==}\PYG{l+m+mi}{0}\PYG{p}{)}\PYG{o}{/}\PYG{l+m+mf}{20000.}
\PYG{g+go}{answer = 0.38885, or 38\PYGZpc{}.}
\end{Verbatim}

\end{fulllineitems}

\index{chisquare() (in module lib.dyn.dynamics)}

\begin{fulllineitems}
\phantomsection\label{lib.dyn:lib.dyn.dynamics.chisquare}\pysiglinewithargsret{\code{lib.dyn.dynamics.}\bfcode{chisquare}}{\emph{df}, \emph{size=None}}{}
Draw samples from a chi-square distribution.

When \emph{df} independent random variables, each with standard normal
distributions (mean 0, variance 1), are squared and summed, the
resulting distribution is chi-square (see Notes).  This distribution
is often used in hypothesis testing.
\begin{description}
\item[{df}] \leavevmode{[}int{]}
Number of degrees of freedom.

\item[{size}] \leavevmode{[}tuple of ints, int, optional{]}
Size of the returned array.  By default, a scalar is
returned.

\end{description}
\begin{description}
\item[{output}] \leavevmode{[}ndarray{]}
Samples drawn from the distribution, packed in a \emph{size}-shaped
array.

\end{description}
\begin{description}
\item[{ValueError}] \leavevmode
When \emph{df} \textless{}= 0 or when an inappropriate \emph{size} (e.g. \code{size=-1})
is given.

\end{description}

The variable obtained by summing the squares of \emph{df} independent,
standard normally distributed random variables:
\begin{gather}
\begin{split}Q = \sum_{i=0}^{\mathtt{df}} X^2_i\end{split}\notag
\end{gather}
is chi-square distributed, denoted
\begin{gather}
\begin{split}Q \sim \chi^2_k.\end{split}\notag
\end{gather}
The probability density function of the chi-squared distribution is
\begin{gather}
\begin{split}p(x) = \frac{(1/2)^{k/2}}{\Gamma(k/2)}
x^{k/2 - 1} e^{-x/2},\end{split}\notag
\end{gather}
where \(\Gamma\) is the gamma function,
\begin{gather}
\begin{split}\Gamma(x) = \int_0^{-\infty} t^{x - 1} e^{-t} dt.\end{split}\notag
\end{gather}
\href{http://www.itl.nist.gov/div898/handbook/eda/section3/eda3666.htm}{NIST/SEMATECH e-Handbook of Statistical Methods}

\begin{Verbatim}[commandchars=\\\{\}]
\PYG{g+gp}{\PYGZgt{}\PYGZgt{}\PYGZgt{} }\PYG{n}{np}\PYG{o}{.}\PYG{n}{random}\PYG{o}{.}\PYG{n}{chisquare}\PYG{p}{(}\PYG{l+m+mi}{2}\PYG{p}{,}\PYG{l+m+mi}{4}\PYG{p}{)}
\PYG{g+go}{array([ 1.89920014,  9.00867716,  3.13710533,  5.62318272])}
\end{Verbatim}

\end{fulllineitems}

\index{exponential() (in module lib.dyn.dynamics)}

\begin{fulllineitems}
\phantomsection\label{lib.dyn:lib.dyn.dynamics.exponential}\pysiglinewithargsret{\code{lib.dyn.dynamics.}\bfcode{exponential}}{\emph{scale=1.0}, \emph{size=None}}{}
Exponential distribution.

Its probability density function is
\begin{gather}
\begin{split}f(x; \frac{1}{\beta}) = \frac{1}{\beta} \exp(-\frac{x}{\beta}),\end{split}\notag
\end{gather}
for \code{x \textgreater{} 0} and 0 elsewhere. \(\beta\) is the scale parameter,
which is the inverse of the rate parameter \(\lambda = 1/\beta\).
The rate parameter is an alternative, widely used parameterization
of the exponential distribution {\color{red}\bfseries{}{[}3{]}\_}.

The exponential distribution is a continuous analogue of the
geometric distribution.  It describes many common situations, such as
the size of raindrops measured over many rainstorms {\color{red}\bfseries{}{[}1{]}\_}, or the time
between page requests to Wikipedia {\color{red}\bfseries{}{[}2{]}\_}.
\begin{description}
\item[{scale}] \leavevmode{[}float{]}
The scale parameter, \(\beta = 1/\lambda\).

\item[{size}] \leavevmode{[}tuple of ints{]}
Number of samples to draw.  The output is shaped
according to \emph{size}.

\end{description}

\end{fulllineitems}

\index{f() (in module lib.dyn.dynamics)}

\begin{fulllineitems}
\phantomsection\label{lib.dyn:lib.dyn.dynamics.f}\pysiglinewithargsret{\code{lib.dyn.dynamics.}\bfcode{f}}{\emph{dfnum}, \emph{dfden}, \emph{size=None}}{}
Draw samples from a F distribution.

Samples are drawn from an F distribution with specified parameters,
\emph{dfnum} (degrees of freedom in numerator) and \emph{dfden} (degrees of freedom
in denominator), where both parameters should be greater than zero.

The random variate of the F distribution (also known as the
Fisher distribution) is a continuous probability distribution
that arises in ANOVA tests, and is the ratio of two chi-square
variates.
\begin{description}
\item[{dfnum}] \leavevmode{[}float{]}
Degrees of freedom in numerator. Should be greater than zero.

\item[{dfden}] \leavevmode{[}float{]}
Degrees of freedom in denominator. Should be greater than zero.

\item[{size}] \leavevmode{[}\{tuple, int\}, optional{]}
Output shape.  If the given shape is, e.g., \code{(m, n, k)},
then \code{m * n * k} samples are drawn. By default only one sample
is returned.

\end{description}
\begin{description}
\item[{samples}] \leavevmode{[}\{ndarray, scalar\}{]}
Samples from the Fisher distribution.

\end{description}
\begin{description}
\item[{scipy.stats.distributions.f}] \leavevmode{[}probability density function,{]}
distribution or cumulative density function, etc.

\end{description}

The F statistic is used to compare in-group variances to between-group
variances. Calculating the distribution depends on the sampling, and
so it is a function of the respective degrees of freedom in the
problem.  The variable \emph{dfnum} is the number of samples minus one, the
between-groups degrees of freedom, while \emph{dfden} is the within-groups
degrees of freedom, the sum of the number of samples in each group
minus the number of groups.

An example from Glantz{[}1{]}, pp 47-40.
Two groups, children of diabetics (25 people) and children from people
without diabetes (25 controls). Fasting blood glucose was measured,
case group had a mean value of 86.1, controls had a mean value of
82.2. Standard deviations were 2.09 and 2.49 respectively. Are these
data consistent with the null hypothesis that the parents diabetic
status does not affect their children's blood glucose levels?
Calculating the F statistic from the data gives a value of 36.01.

Draw samples from the distribution:

\begin{Verbatim}[commandchars=\\\{\}]
\PYG{g+gp}{\PYGZgt{}\PYGZgt{}\PYGZgt{} }\PYG{n}{dfnum} \PYG{o}{=} \PYG{l+m+mf}{1.} \PYG{c}{\PYGZsh{} between group degrees of freedom}
\PYG{g+gp}{\PYGZgt{}\PYGZgt{}\PYGZgt{} }\PYG{n}{dfden} \PYG{o}{=} \PYG{l+m+mf}{48.} \PYG{c}{\PYGZsh{} within groups degrees of freedom}
\PYG{g+gp}{\PYGZgt{}\PYGZgt{}\PYGZgt{} }\PYG{n}{s} \PYG{o}{=} \PYG{n}{np}\PYG{o}{.}\PYG{n}{random}\PYG{o}{.}\PYG{n}{f}\PYG{p}{(}\PYG{n}{dfnum}\PYG{p}{,} \PYG{n}{dfden}\PYG{p}{,} \PYG{l+m+mi}{1000}\PYG{p}{)}
\end{Verbatim}

The lower bound for the top 1\% of the samples is :

\begin{Verbatim}[commandchars=\\\{\}]
\PYG{g+gp}{\PYGZgt{}\PYGZgt{}\PYGZgt{} }\PYG{n}{sort}\PYG{p}{(}\PYG{n}{s}\PYG{p}{)}\PYG{p}{[}\PYG{o}{\PYGZhy{}}\PYG{l+m+mi}{10}\PYG{p}{]}
\PYG{g+go}{7.61988120985}
\end{Verbatim}

So there is about a 1\% chance that the F statistic will exceed 7.62,
the measured value is 36, so the null hypothesis is rejected at the 1\%
level.

\end{fulllineitems}

\index{fluxAnalysis() (in module lib.dyn.dynamics)}

\begin{fulllineitems}
\phantomsection\label{lib.dyn:lib.dyn.dynamics.fluxAnalysis}\pysiglinewithargsret{\code{lib.dyn.dynamics.}\bfcode{fluxAnalysis}}{\emph{tmpDir}, \emph{resDirPath}, \emph{strZeros}, \emph{ngen}}{}
\end{fulllineitems}

\index{gamma() (in module lib.dyn.dynamics)}

\begin{fulllineitems}
\phantomsection\label{lib.dyn:lib.dyn.dynamics.gamma}\pysiglinewithargsret{\code{lib.dyn.dynamics.}\bfcode{gamma}}{\emph{shape}, \emph{scale=1.0}, \emph{size=None}}{}
Draw samples from a Gamma distribution.

Samples are drawn from a Gamma distribution with specified parameters,
\emph{shape} (sometimes designated ``k'') and \emph{scale} (sometimes designated
``theta''), where both parameters are \textgreater{} 0.
\begin{description}
\item[{shape}] \leavevmode{[}scalar \textgreater{} 0{]}
The shape of the gamma distribution.

\item[{scale}] \leavevmode{[}scalar \textgreater{} 0, optional{]}
The scale of the gamma distribution.  Default is equal to 1.

\item[{size}] \leavevmode{[}shape\_tuple, optional{]}
Output shape.  If the given shape is, e.g., \code{(m, n, k)}, then
\code{m * n * k} samples are drawn.

\end{description}
\begin{description}
\item[{out}] \leavevmode{[}ndarray, float{]}
Returns one sample unless \emph{size} parameter is specified.

\end{description}
\begin{description}
\item[{scipy.stats.distributions.gamma}] \leavevmode{[}probability density function,{]}
distribution or cumulative density function, etc.

\end{description}

The probability density for the Gamma distribution is
\begin{gather}
\begin{split}p(x) = x^{k-1}\frac{e^{-x/\theta}}{\theta^k\Gamma(k)},\end{split}\notag
\end{gather}
where \(k\) is the shape and \(\theta\) the scale,
and \(\Gamma\) is the Gamma function.

The Gamma distribution is often used to model the times to failure of
electronic components, and arises naturally in processes for which the
waiting times between Poisson distributed events are relevant.

Draw samples from the distribution:

\begin{Verbatim}[commandchars=\\\{\}]
\PYG{g+gp}{\PYGZgt{}\PYGZgt{}\PYGZgt{} }\PYG{n}{shape}\PYG{p}{,} \PYG{n}{scale} \PYG{o}{=} \PYG{l+m+mf}{2.}\PYG{p}{,} \PYG{l+m+mf}{2.} \PYG{c}{\PYGZsh{} mean and dispersion}
\PYG{g+gp}{\PYGZgt{}\PYGZgt{}\PYGZgt{} }\PYG{n}{s} \PYG{o}{=} \PYG{n}{np}\PYG{o}{.}\PYG{n}{random}\PYG{o}{.}\PYG{n}{gamma}\PYG{p}{(}\PYG{n}{shape}\PYG{p}{,} \PYG{n}{scale}\PYG{p}{,} \PYG{l+m+mi}{1000}\PYG{p}{)}
\end{Verbatim}

Display the histogram of the samples, along with
the probability density function:

\begin{Verbatim}[commandchars=\\\{\}]
\PYG{g+gp}{\PYGZgt{}\PYGZgt{}\PYGZgt{} }\PYG{k+kn}{import} \PYG{n+nn}{matplotlib.pyplot} \PYG{k+kn}{as} \PYG{n+nn}{plt}
\PYG{g+gp}{\PYGZgt{}\PYGZgt{}\PYGZgt{} }\PYG{k+kn}{import} \PYG{n+nn}{scipy.special} \PYG{k+kn}{as} \PYG{n+nn}{sps}
\PYG{g+gp}{\PYGZgt{}\PYGZgt{}\PYGZgt{} }\PYG{n}{count}\PYG{p}{,} \PYG{n}{bins}\PYG{p}{,} \PYG{n}{ignored} \PYG{o}{=} \PYG{n}{plt}\PYG{o}{.}\PYG{n}{hist}\PYG{p}{(}\PYG{n}{s}\PYG{p}{,} \PYG{l+m+mi}{50}\PYG{p}{,} \PYG{n}{normed}\PYG{o}{=}\PYG{n+nb+bp}{True}\PYG{p}{)}
\PYG{g+gp}{\PYGZgt{}\PYGZgt{}\PYGZgt{} }\PYG{n}{y} \PYG{o}{=} \PYG{n}{bins}\PYG{o}{*}\PYG{o}{*}\PYG{p}{(}\PYG{n}{shape}\PYG{o}{\PYGZhy{}}\PYG{l+m+mi}{1}\PYG{p}{)}\PYG{o}{*}\PYG{p}{(}\PYG{n}{np}\PYG{o}{.}\PYG{n}{exp}\PYG{p}{(}\PYG{o}{\PYGZhy{}}\PYG{n}{bins}\PYG{o}{/}\PYG{n}{scale}\PYG{p}{)} \PYG{o}{/}
\PYG{g+gp}{... }                     \PYG{p}{(}\PYG{n}{sps}\PYG{o}{.}\PYG{n}{gamma}\PYG{p}{(}\PYG{n}{shape}\PYG{p}{)}\PYG{o}{*}\PYG{n}{scale}\PYG{o}{*}\PYG{o}{*}\PYG{n}{shape}\PYG{p}{)}\PYG{p}{)}
\PYG{g+gp}{\PYGZgt{}\PYGZgt{}\PYGZgt{} }\PYG{n}{plt}\PYG{o}{.}\PYG{n}{plot}\PYG{p}{(}\PYG{n}{bins}\PYG{p}{,} \PYG{n}{y}\PYG{p}{,} \PYG{n}{linewidth}\PYG{o}{=}\PYG{l+m+mi}{2}\PYG{p}{,} \PYG{n}{color}\PYG{o}{=}\PYG{l+s}{\PYGZsq{}}\PYG{l+s}{r}\PYG{l+s}{\PYGZsq{}}\PYG{p}{)}
\PYG{g+gp}{\PYGZgt{}\PYGZgt{}\PYGZgt{} }\PYG{n}{plt}\PYG{o}{.}\PYG{n}{show}\PYG{p}{(}\PYG{p}{)}
\end{Verbatim}

\end{fulllineitems}

\index{generateFluxList() (in module lib.dyn.dynamics)}

\begin{fulllineitems}
\phantomsection\label{lib.dyn:lib.dyn.dynamics.generateFluxList}\pysiglinewithargsret{\code{lib.dyn.dynamics.}\bfcode{generateFluxList}}{\emph{tmpPath}, \emph{tmpSysType}, \emph{tmpLastID=None}}{}
\end{fulllineitems}

\index{geometric() (in module lib.dyn.dynamics)}

\begin{fulllineitems}
\phantomsection\label{lib.dyn:lib.dyn.dynamics.geometric}\pysiglinewithargsret{\code{lib.dyn.dynamics.}\bfcode{geometric}}{\emph{p}, \emph{size=None}}{}
Draw samples from the geometric distribution.

Bernoulli trials are experiments with one of two outcomes:
success or failure (an example of such an experiment is flipping
a coin).  The geometric distribution models the number of trials
that must be run in order to achieve success.  It is therefore
supported on the positive integers, \code{k = 1, 2, ...}.

The probability mass function of the geometric distribution is
\begin{gather}
\begin{split}f(k) = (1 - p)^{k - 1} p\end{split}\notag
\end{gather}
where \emph{p} is the probability of success of an individual trial.
\begin{description}
\item[{p}] \leavevmode{[}float{]}
The probability of success of an individual trial.

\item[{size}] \leavevmode{[}tuple of ints{]}
Number of values to draw from the distribution.  The output
is shaped according to \emph{size}.

\end{description}
\begin{description}
\item[{out}] \leavevmode{[}ndarray{]}
Samples from the geometric distribution, shaped according to
\emph{size}.

\end{description}

Draw ten thousand values from the geometric distribution,
with the probability of an individual success equal to 0.35:

\begin{Verbatim}[commandchars=\\\{\}]
\PYG{g+gp}{\PYGZgt{}\PYGZgt{}\PYGZgt{} }\PYG{n}{z} \PYG{o}{=} \PYG{n}{np}\PYG{o}{.}\PYG{n}{random}\PYG{o}{.}\PYG{n}{geometric}\PYG{p}{(}\PYG{n}{p}\PYG{o}{=}\PYG{l+m+mf}{0.35}\PYG{p}{,} \PYG{n}{size}\PYG{o}{=}\PYG{l+m+mi}{10000}\PYG{p}{)}
\end{Verbatim}

How many trials succeeded after a single run?

\begin{Verbatim}[commandchars=\\\{\}]
\PYG{g+gp}{\PYGZgt{}\PYGZgt{}\PYGZgt{} }\PYG{p}{(}\PYG{n}{z} \PYG{o}{==} \PYG{l+m+mi}{1}\PYG{p}{)}\PYG{o}{.}\PYG{n}{sum}\PYG{p}{(}\PYG{p}{)} \PYG{o}{/} \PYG{l+m+mf}{10000.}
\PYG{g+go}{0.34889999999999999 \PYGZsh{}random}
\end{Verbatim}

\end{fulllineitems}

\index{get\_state() (in module lib.dyn.dynamics)}

\begin{fulllineitems}
\phantomsection\label{lib.dyn:lib.dyn.dynamics.get_state}\pysiglinewithargsret{\code{lib.dyn.dynamics.}\bfcode{get\_state}}{}{}
Return a tuple representing the internal state of the generator.

For more details, see \emph{set\_state}.
\begin{description}
\item[{out}] \leavevmode{[}tuple(str, ndarray of 624 uints, int, int, float){]}
The returned tuple has the following items:
\begin{enumerate}
\item {} 
the string `MT19937'.

\item {} 
a 1-D array of 624 unsigned integer keys.

\item {} 
an integer \code{pos}.

\item {} 
an integer \code{has\_gauss}.

\item {} 
a float \code{cached\_gaussian}.

\end{enumerate}

\end{description}

set\_state

\emph{set\_state} and \emph{get\_state} are not needed to work with any of the
random distributions in NumPy. If the internal state is manually altered,
the user should know exactly what he/she is doing.

\end{fulllineitems}

\index{gumbel() (in module lib.dyn.dynamics)}

\begin{fulllineitems}
\phantomsection\label{lib.dyn:lib.dyn.dynamics.gumbel}\pysiglinewithargsret{\code{lib.dyn.dynamics.}\bfcode{gumbel}}{\emph{loc=0.0}, \emph{scale=1.0}, \emph{size=None}}{}
Gumbel distribution.

Draw samples from a Gumbel distribution with specified location and scale.
For more information on the Gumbel distribution, see Notes and References
below.
\begin{description}
\item[{loc}] \leavevmode{[}float{]}
The location of the mode of the distribution.

\item[{scale}] \leavevmode{[}float{]}
The scale parameter of the distribution.

\item[{size}] \leavevmode{[}tuple of ints{]}
Output shape.  If the given shape is, e.g., \code{(m, n, k)}, then
\code{m * n * k} samples are drawn.

\end{description}
\begin{description}
\item[{out}] \leavevmode{[}ndarray{]}
The samples

\end{description}

scipy.stats.gumbel\_l
scipy.stats.gumbel\_r
scipy.stats.genextreme
\begin{quote}

probability density function, distribution, or cumulative density
function, etc. for each of the above
\end{quote}

weibull

The Gumbel (or Smallest Extreme Value (SEV) or the Smallest Extreme Value
Type I) distribution is one of a class of Generalized Extreme Value (GEV)
distributions used in modeling extreme value problems.  The Gumbel is a
special case of the Extreme Value Type I distribution for maximums from
distributions with ``exponential-like'' tails.

The probability density for the Gumbel distribution is
\begin{gather}
\begin{split}p(x) = \frac{e^{-(x - \mu)/ \beta}}{\beta} e^{ -e^{-(x - \mu)/
\beta}},\end{split}\notag
\end{gather}
where \(\mu\) is the mode, a location parameter, and \(\beta\) is
the scale parameter.

The Gumbel (named for German mathematician Emil Julius Gumbel) was used
very early in the hydrology literature, for modeling the occurrence of
flood events. It is also used for modeling maximum wind speed and rainfall
rates.  It is a ``fat-tailed'' distribution - the probability of an event in
the tail of the distribution is larger than if one used a Gaussian, hence
the surprisingly frequent occurrence of 100-year floods. Floods were
initially modeled as a Gaussian process, which underestimated the frequency
of extreme events.

It is one of a class of extreme value distributions, the Generalized
Extreme Value (GEV) distributions, which also includes the Weibull and
Frechet.

The function has a mean of \(\mu + 0.57721\beta\) and a variance of
\(\frac{\pi^2}{6}\beta^2\).

Gumbel, E. J., \emph{Statistics of Extremes}, New York: Columbia University
Press, 1958.

Reiss, R.-D. and Thomas, M., \emph{Statistical Analysis of Extreme Values from
Insurance, Finance, Hydrology and Other Fields}, Basel: Birkhauser Verlag,
2001.

Draw samples from the distribution:

\begin{Verbatim}[commandchars=\\\{\}]
\PYG{g+gp}{\PYGZgt{}\PYGZgt{}\PYGZgt{} }\PYG{n}{mu}\PYG{p}{,} \PYG{n}{beta} \PYG{o}{=} \PYG{l+m+mi}{0}\PYG{p}{,} \PYG{l+m+mf}{0.1} \PYG{c}{\PYGZsh{} location and scale}
\PYG{g+gp}{\PYGZgt{}\PYGZgt{}\PYGZgt{} }\PYG{n}{s} \PYG{o}{=} \PYG{n}{np}\PYG{o}{.}\PYG{n}{random}\PYG{o}{.}\PYG{n}{gumbel}\PYG{p}{(}\PYG{n}{mu}\PYG{p}{,} \PYG{n}{beta}\PYG{p}{,} \PYG{l+m+mi}{1000}\PYG{p}{)}
\end{Verbatim}

Display the histogram of the samples, along with
the probability density function:

\begin{Verbatim}[commandchars=\\\{\}]
\PYG{g+gp}{\PYGZgt{}\PYGZgt{}\PYGZgt{} }\PYG{k+kn}{import} \PYG{n+nn}{matplotlib.pyplot} \PYG{k+kn}{as} \PYG{n+nn}{plt}
\PYG{g+gp}{\PYGZgt{}\PYGZgt{}\PYGZgt{} }\PYG{n}{count}\PYG{p}{,} \PYG{n}{bins}\PYG{p}{,} \PYG{n}{ignored} \PYG{o}{=} \PYG{n}{plt}\PYG{o}{.}\PYG{n}{hist}\PYG{p}{(}\PYG{n}{s}\PYG{p}{,} \PYG{l+m+mi}{30}\PYG{p}{,} \PYG{n}{normed}\PYG{o}{=}\PYG{n+nb+bp}{True}\PYG{p}{)}
\PYG{g+gp}{\PYGZgt{}\PYGZgt{}\PYGZgt{} }\PYG{n}{plt}\PYG{o}{.}\PYG{n}{plot}\PYG{p}{(}\PYG{n}{bins}\PYG{p}{,} \PYG{p}{(}\PYG{l+m+mi}{1}\PYG{o}{/}\PYG{n}{beta}\PYG{p}{)}\PYG{o}{*}\PYG{n}{np}\PYG{o}{.}\PYG{n}{exp}\PYG{p}{(}\PYG{o}{\PYGZhy{}}\PYG{p}{(}\PYG{n}{bins} \PYG{o}{\PYGZhy{}} \PYG{n}{mu}\PYG{p}{)}\PYG{o}{/}\PYG{n}{beta}\PYG{p}{)}
\PYG{g+gp}{... }         \PYG{o}{*} \PYG{n}{np}\PYG{o}{.}\PYG{n}{exp}\PYG{p}{(} \PYG{o}{\PYGZhy{}}\PYG{n}{np}\PYG{o}{.}\PYG{n}{exp}\PYG{p}{(} \PYG{o}{\PYGZhy{}}\PYG{p}{(}\PYG{n}{bins} \PYG{o}{\PYGZhy{}} \PYG{n}{mu}\PYG{p}{)} \PYG{o}{/}\PYG{n}{beta}\PYG{p}{)} \PYG{p}{)}\PYG{p}{,}
\PYG{g+gp}{... }         \PYG{n}{linewidth}\PYG{o}{=}\PYG{l+m+mi}{2}\PYG{p}{,} \PYG{n}{color}\PYG{o}{=}\PYG{l+s}{\PYGZsq{}}\PYG{l+s}{r}\PYG{l+s}{\PYGZsq{}}\PYG{p}{)}
\PYG{g+gp}{\PYGZgt{}\PYGZgt{}\PYGZgt{} }\PYG{n}{plt}\PYG{o}{.}\PYG{n}{show}\PYG{p}{(}\PYG{p}{)}
\end{Verbatim}

Show how an extreme value distribution can arise from a Gaussian process
and compare to a Gaussian:

\begin{Verbatim}[commandchars=\\\{\}]
\PYG{g+gp}{\PYGZgt{}\PYGZgt{}\PYGZgt{} }\PYG{n}{means} \PYG{o}{=} \PYG{p}{[}\PYG{p}{]}
\PYG{g+gp}{\PYGZgt{}\PYGZgt{}\PYGZgt{} }\PYG{n}{maxima} \PYG{o}{=} \PYG{p}{[}\PYG{p}{]}
\PYG{g+gp}{\PYGZgt{}\PYGZgt{}\PYGZgt{} }\PYG{k}{for} \PYG{n}{i} \PYG{o+ow}{in} \PYG{n+nb}{range}\PYG{p}{(}\PYG{l+m+mi}{0}\PYG{p}{,}\PYG{l+m+mi}{1000}\PYG{p}{)} \PYG{p}{:}
\PYG{g+gp}{... }   \PYG{n}{a} \PYG{o}{=} \PYG{n}{np}\PYG{o}{.}\PYG{n}{random}\PYG{o}{.}\PYG{n}{normal}\PYG{p}{(}\PYG{n}{mu}\PYG{p}{,} \PYG{n}{beta}\PYG{p}{,} \PYG{l+m+mi}{1000}\PYG{p}{)}
\PYG{g+gp}{... }   \PYG{n}{means}\PYG{o}{.}\PYG{n}{append}\PYG{p}{(}\PYG{n}{a}\PYG{o}{.}\PYG{n}{mean}\PYG{p}{(}\PYG{p}{)}\PYG{p}{)}
\PYG{g+gp}{... }   \PYG{n}{maxima}\PYG{o}{.}\PYG{n}{append}\PYG{p}{(}\PYG{n}{a}\PYG{o}{.}\PYG{n}{max}\PYG{p}{(}\PYG{p}{)}\PYG{p}{)}
\PYG{g+gp}{\PYGZgt{}\PYGZgt{}\PYGZgt{} }\PYG{n}{count}\PYG{p}{,} \PYG{n}{bins}\PYG{p}{,} \PYG{n}{ignored} \PYG{o}{=} \PYG{n}{plt}\PYG{o}{.}\PYG{n}{hist}\PYG{p}{(}\PYG{n}{maxima}\PYG{p}{,} \PYG{l+m+mi}{30}\PYG{p}{,} \PYG{n}{normed}\PYG{o}{=}\PYG{n+nb+bp}{True}\PYG{p}{)}
\PYG{g+gp}{\PYGZgt{}\PYGZgt{}\PYGZgt{} }\PYG{n}{beta} \PYG{o}{=} \PYG{n}{np}\PYG{o}{.}\PYG{n}{std}\PYG{p}{(}\PYG{n}{maxima}\PYG{p}{)}\PYG{o}{*}\PYG{n}{np}\PYG{o}{.}\PYG{n}{pi}\PYG{o}{/}\PYG{n}{np}\PYG{o}{.}\PYG{n}{sqrt}\PYG{p}{(}\PYG{l+m+mi}{6}\PYG{p}{)}
\PYG{g+gp}{\PYGZgt{}\PYGZgt{}\PYGZgt{} }\PYG{n}{mu} \PYG{o}{=} \PYG{n}{np}\PYG{o}{.}\PYG{n}{mean}\PYG{p}{(}\PYG{n}{maxima}\PYG{p}{)} \PYG{o}{\PYGZhy{}} \PYG{l+m+mf}{0.57721}\PYG{o}{*}\PYG{n}{beta}
\PYG{g+gp}{\PYGZgt{}\PYGZgt{}\PYGZgt{} }\PYG{n}{plt}\PYG{o}{.}\PYG{n}{plot}\PYG{p}{(}\PYG{n}{bins}\PYG{p}{,} \PYG{p}{(}\PYG{l+m+mi}{1}\PYG{o}{/}\PYG{n}{beta}\PYG{p}{)}\PYG{o}{*}\PYG{n}{np}\PYG{o}{.}\PYG{n}{exp}\PYG{p}{(}\PYG{o}{\PYGZhy{}}\PYG{p}{(}\PYG{n}{bins} \PYG{o}{\PYGZhy{}} \PYG{n}{mu}\PYG{p}{)}\PYG{o}{/}\PYG{n}{beta}\PYG{p}{)}
\PYG{g+gp}{... }         \PYG{o}{*} \PYG{n}{np}\PYG{o}{.}\PYG{n}{exp}\PYG{p}{(}\PYG{o}{\PYGZhy{}}\PYG{n}{np}\PYG{o}{.}\PYG{n}{exp}\PYG{p}{(}\PYG{o}{\PYGZhy{}}\PYG{p}{(}\PYG{n}{bins} \PYG{o}{\PYGZhy{}} \PYG{n}{mu}\PYG{p}{)}\PYG{o}{/}\PYG{n}{beta}\PYG{p}{)}\PYG{p}{)}\PYG{p}{,}
\PYG{g+gp}{... }         \PYG{n}{linewidth}\PYG{o}{=}\PYG{l+m+mi}{2}\PYG{p}{,} \PYG{n}{color}\PYG{o}{=}\PYG{l+s}{\PYGZsq{}}\PYG{l+s}{r}\PYG{l+s}{\PYGZsq{}}\PYG{p}{)}
\PYG{g+gp}{\PYGZgt{}\PYGZgt{}\PYGZgt{} }\PYG{n}{plt}\PYG{o}{.}\PYG{n}{plot}\PYG{p}{(}\PYG{n}{bins}\PYG{p}{,} \PYG{l+m+mi}{1}\PYG{o}{/}\PYG{p}{(}\PYG{n}{beta} \PYG{o}{*} \PYG{n}{np}\PYG{o}{.}\PYG{n}{sqrt}\PYG{p}{(}\PYG{l+m+mi}{2} \PYG{o}{*} \PYG{n}{np}\PYG{o}{.}\PYG{n}{pi}\PYG{p}{)}\PYG{p}{)}
\PYG{g+gp}{... }         \PYG{o}{*} \PYG{n}{np}\PYG{o}{.}\PYG{n}{exp}\PYG{p}{(}\PYG{o}{\PYGZhy{}}\PYG{p}{(}\PYG{n}{bins} \PYG{o}{\PYGZhy{}} \PYG{n}{mu}\PYG{p}{)}\PYG{o}{*}\PYG{o}{*}\PYG{l+m+mi}{2} \PYG{o}{/} \PYG{p}{(}\PYG{l+m+mi}{2} \PYG{o}{*} \PYG{n}{beta}\PYG{o}{*}\PYG{o}{*}\PYG{l+m+mi}{2}\PYG{p}{)}\PYG{p}{)}\PYG{p}{,}
\PYG{g+gp}{... }         \PYG{n}{linewidth}\PYG{o}{=}\PYG{l+m+mi}{2}\PYG{p}{,} \PYG{n}{color}\PYG{o}{=}\PYG{l+s}{\PYGZsq{}}\PYG{l+s}{g}\PYG{l+s}{\PYGZsq{}}\PYG{p}{)}
\PYG{g+gp}{\PYGZgt{}\PYGZgt{}\PYGZgt{} }\PYG{n}{plt}\PYG{o}{.}\PYG{n}{show}\PYG{p}{(}\PYG{p}{)}
\end{Verbatim}

\end{fulllineitems}

\index{hypergeometric() (in module lib.dyn.dynamics)}

\begin{fulllineitems}
\phantomsection\label{lib.dyn:lib.dyn.dynamics.hypergeometric}\pysiglinewithargsret{\code{lib.dyn.dynamics.}\bfcode{hypergeometric}}{\emph{ngood}, \emph{nbad}, \emph{nsample}, \emph{size=None}}{}
Draw samples from a Hypergeometric distribution.

Samples are drawn from a Hypergeometric distribution with specified
parameters, ngood (ways to make a good selection), nbad (ways to make
a bad selection), and nsample = number of items sampled, which is less
than or equal to the sum ngood + nbad.
\begin{description}
\item[{ngood}] \leavevmode{[}int or array\_like{]}
Number of ways to make a good selection.  Must be nonnegative.

\item[{nbad}] \leavevmode{[}int or array\_like{]}
Number of ways to make a bad selection.  Must be nonnegative.

\item[{nsample}] \leavevmode{[}int or array\_like{]}
Number of items sampled.  Must be at least 1 and at most
\code{ngood + nbad}.

\item[{size}] \leavevmode{[}int or tuple of int{]}
Output shape.  If the given shape is, e.g., \code{(m, n, k)}, then
\code{m * n * k} samples are drawn.

\end{description}
\begin{description}
\item[{samples}] \leavevmode{[}ndarray or scalar{]}
The values are all integers in  {[}0, n{]}.

\end{description}
\begin{description}
\item[{scipy.stats.distributions.hypergeom}] \leavevmode{[}probability density function,{]}
distribution or cumulative density function, etc.

\end{description}

The probability density for the Hypergeometric distribution is
\begin{gather}
\begin{split}P(x) = \frac{\binom{m}{n}\binom{N-m}{n-x}}{\binom{N}{n}},\end{split}\notag
\end{gather}
where \(0 \le x \le m\) and \(n+m-N \le x \le n\)

for P(x) the probability of x successes, n = ngood, m = nbad, and
N = number of samples.

Consider an urn with black and white marbles in it, ngood of them
black and nbad are white. If you draw nsample balls without
replacement, then the Hypergeometric distribution describes the
distribution of black balls in the drawn sample.

Note that this distribution is very similar to the Binomial
distribution, except that in this case, samples are drawn without
replacement, whereas in the Binomial case samples are drawn with
replacement (or the sample space is infinite). As the sample space
becomes large, this distribution approaches the Binomial.

Draw samples from the distribution:

\begin{Verbatim}[commandchars=\\\{\}]
\PYG{g+gp}{\PYGZgt{}\PYGZgt{}\PYGZgt{} }\PYG{n}{ngood}\PYG{p}{,} \PYG{n}{nbad}\PYG{p}{,} \PYG{n}{nsamp} \PYG{o}{=} \PYG{l+m+mi}{100}\PYG{p}{,} \PYG{l+m+mi}{2}\PYG{p}{,} \PYG{l+m+mi}{10}
\PYG{g+go}{\PYGZsh{} number of good, number of bad, and number of samples}
\PYG{g+gp}{\PYGZgt{}\PYGZgt{}\PYGZgt{} }\PYG{n}{s} \PYG{o}{=} \PYG{n}{np}\PYG{o}{.}\PYG{n}{random}\PYG{o}{.}\PYG{n}{hypergeometric}\PYG{p}{(}\PYG{n}{ngood}\PYG{p}{,} \PYG{n}{nbad}\PYG{p}{,} \PYG{n}{nsamp}\PYG{p}{,} \PYG{l+m+mi}{1000}\PYG{p}{)}
\PYG{g+gp}{\PYGZgt{}\PYGZgt{}\PYGZgt{} }\PYG{n}{hist}\PYG{p}{(}\PYG{n}{s}\PYG{p}{)}
\PYG{g+go}{\PYGZsh{}   note that it is very unlikely to grab both bad items}
\end{Verbatim}

Suppose you have an urn with 15 white and 15 black marbles.
If you pull 15 marbles at random, how likely is it that
12 or more of them are one color?

\begin{Verbatim}[commandchars=\\\{\}]
\PYG{g+gp}{\PYGZgt{}\PYGZgt{}\PYGZgt{} }\PYG{n}{s} \PYG{o}{=} \PYG{n}{np}\PYG{o}{.}\PYG{n}{random}\PYG{o}{.}\PYG{n}{hypergeometric}\PYG{p}{(}\PYG{l+m+mi}{15}\PYG{p}{,} \PYG{l+m+mi}{15}\PYG{p}{,} \PYG{l+m+mi}{15}\PYG{p}{,} \PYG{l+m+mi}{100000}\PYG{p}{)}
\PYG{g+gp}{\PYGZgt{}\PYGZgt{}\PYGZgt{} }\PYG{n+nb}{sum}\PYG{p}{(}\PYG{n}{s}\PYG{o}{\PYGZgt{}}\PYG{o}{=}\PYG{l+m+mi}{12}\PYG{p}{)}\PYG{o}{/}\PYG{l+m+mf}{100000.} \PYG{o}{+} \PYG{n+nb}{sum}\PYG{p}{(}\PYG{n}{s}\PYG{o}{\PYGZlt{}}\PYG{o}{=}\PYG{l+m+mi}{3}\PYG{p}{)}\PYG{o}{/}\PYG{l+m+mf}{100000.}
\PYG{g+go}{\PYGZsh{}   answer = 0.003 ... pretty unlikely!}
\end{Verbatim}

\end{fulllineitems}

\index{laplace() (in module lib.dyn.dynamics)}

\begin{fulllineitems}
\phantomsection\label{lib.dyn:lib.dyn.dynamics.laplace}\pysiglinewithargsret{\code{lib.dyn.dynamics.}\bfcode{laplace}}{\emph{loc=0.0}, \emph{scale=1.0}, \emph{size=None}}{}
Draw samples from the Laplace or double exponential distribution with
specified location (or mean) and scale (decay).

The Laplace distribution is similar to the Gaussian/normal distribution,
but is sharper at the peak and has fatter tails. It represents the
difference between two independent, identically distributed exponential
random variables.
\begin{description}
\item[{loc}] \leavevmode{[}float{]}
The position, \(\mu\), of the distribution peak.

\item[{scale}] \leavevmode{[}float{]}
\(\lambda\), the exponential decay.

\end{description}

It has the probability density function
\begin{gather}
\begin{split}f(x; \mu, \lambda) = \frac{1}{2\lambda}
\exp\left(-\frac{|x - \mu|}{\lambda}\right).\end{split}\notag
\end{gather}
The first law of Laplace, from 1774, states that the frequency of an error
can be expressed as an exponential function of the absolute magnitude of
the error, which leads to the Laplace distribution. For many problems in
Economics and Health sciences, this distribution seems to model the data
better than the standard Gaussian distribution

Draw samples from the distribution

\begin{Verbatim}[commandchars=\\\{\}]
\PYG{g+gp}{\PYGZgt{}\PYGZgt{}\PYGZgt{} }\PYG{n}{loc}\PYG{p}{,} \PYG{n}{scale} \PYG{o}{=} \PYG{l+m+mf}{0.}\PYG{p}{,} \PYG{l+m+mf}{1.}
\PYG{g+gp}{\PYGZgt{}\PYGZgt{}\PYGZgt{} }\PYG{n}{s} \PYG{o}{=} \PYG{n}{np}\PYG{o}{.}\PYG{n}{random}\PYG{o}{.}\PYG{n}{laplace}\PYG{p}{(}\PYG{n}{loc}\PYG{p}{,} \PYG{n}{scale}\PYG{p}{,} \PYG{l+m+mi}{1000}\PYG{p}{)}
\end{Verbatim}

Display the histogram of the samples, along with
the probability density function:

\begin{Verbatim}[commandchars=\\\{\}]
\PYG{g+gp}{\PYGZgt{}\PYGZgt{}\PYGZgt{} }\PYG{k+kn}{import} \PYG{n+nn}{matplotlib.pyplot} \PYG{k+kn}{as} \PYG{n+nn}{plt}
\PYG{g+gp}{\PYGZgt{}\PYGZgt{}\PYGZgt{} }\PYG{n}{count}\PYG{p}{,} \PYG{n}{bins}\PYG{p}{,} \PYG{n}{ignored} \PYG{o}{=} \PYG{n}{plt}\PYG{o}{.}\PYG{n}{hist}\PYG{p}{(}\PYG{n}{s}\PYG{p}{,} \PYG{l+m+mi}{30}\PYG{p}{,} \PYG{n}{normed}\PYG{o}{=}\PYG{n+nb+bp}{True}\PYG{p}{)}
\PYG{g+gp}{\PYGZgt{}\PYGZgt{}\PYGZgt{} }\PYG{n}{x} \PYG{o}{=} \PYG{n}{np}\PYG{o}{.}\PYG{n}{arange}\PYG{p}{(}\PYG{o}{\PYGZhy{}}\PYG{l+m+mf}{8.}\PYG{p}{,} \PYG{l+m+mf}{8.}\PYG{p}{,} \PYG{o}{.}\PYG{l+m+mo}{01}\PYG{p}{)}
\PYG{g+gp}{\PYGZgt{}\PYGZgt{}\PYGZgt{} }\PYG{n}{pdf} \PYG{o}{=} \PYG{n}{np}\PYG{o}{.}\PYG{n}{exp}\PYG{p}{(}\PYG{o}{\PYGZhy{}}\PYG{n+nb}{abs}\PYG{p}{(}\PYG{n}{x}\PYG{o}{\PYGZhy{}}\PYG{n}{loc}\PYG{o}{/}\PYG{n}{scale}\PYG{p}{)}\PYG{p}{)}\PYG{o}{/}\PYG{p}{(}\PYG{l+m+mf}{2.}\PYG{o}{*}\PYG{n}{scale}\PYG{p}{)}
\PYG{g+gp}{\PYGZgt{}\PYGZgt{}\PYGZgt{} }\PYG{n}{plt}\PYG{o}{.}\PYG{n}{plot}\PYG{p}{(}\PYG{n}{x}\PYG{p}{,} \PYG{n}{pdf}\PYG{p}{)}
\end{Verbatim}

Plot Gaussian for comparison:

\begin{Verbatim}[commandchars=\\\{\}]
\PYG{g+gp}{\PYGZgt{}\PYGZgt{}\PYGZgt{} }\PYG{n}{g} \PYG{o}{=} \PYG{p}{(}\PYG{l+m+mi}{1}\PYG{o}{/}\PYG{p}{(}\PYG{n}{scale} \PYG{o}{*} \PYG{n}{np}\PYG{o}{.}\PYG{n}{sqrt}\PYG{p}{(}\PYG{l+m+mi}{2} \PYG{o}{*} \PYG{n}{np}\PYG{o}{.}\PYG{n}{pi}\PYG{p}{)}\PYG{p}{)} \PYG{o}{*} 
\PYG{g+gp}{... }     \PYG{n}{np}\PYG{o}{.}\PYG{n}{exp}\PYG{p}{(} \PYG{o}{\PYGZhy{}} \PYG{p}{(}\PYG{n}{x} \PYG{o}{\PYGZhy{}} \PYG{n}{loc}\PYG{p}{)}\PYG{o}{*}\PYG{o}{*}\PYG{l+m+mi}{2} \PYG{o}{/} \PYG{p}{(}\PYG{l+m+mi}{2} \PYG{o}{*} \PYG{n}{scale}\PYG{o}{*}\PYG{o}{*}\PYG{l+m+mi}{2}\PYG{p}{)} \PYG{p}{)}\PYG{p}{)}
\PYG{g+gp}{\PYGZgt{}\PYGZgt{}\PYGZgt{} }\PYG{n}{plt}\PYG{o}{.}\PYG{n}{plot}\PYG{p}{(}\PYG{n}{x}\PYG{p}{,}\PYG{n}{g}\PYG{p}{)}
\end{Verbatim}

\end{fulllineitems}

\index{logistic() (in module lib.dyn.dynamics)}

\begin{fulllineitems}
\phantomsection\label{lib.dyn:lib.dyn.dynamics.logistic}\pysiglinewithargsret{\code{lib.dyn.dynamics.}\bfcode{logistic}}{\emph{loc=0.0}, \emph{scale=1.0}, \emph{size=None}}{}
Draw samples from a Logistic distribution.

Samples are drawn from a Logistic distribution with specified
parameters, loc (location or mean, also median), and scale (\textgreater{}0).

loc : float

scale : float \textgreater{} 0.
\begin{description}
\item[{size}] \leavevmode{[}\{tuple, int\}{]}
Output shape.  If the given shape is, e.g., \code{(m, n, k)}, then
\code{m * n * k} samples are drawn.

\end{description}
\begin{description}
\item[{samples}] \leavevmode{[}\{ndarray, scalar\}{]}
where the values are all integers in  {[}0, n{]}.

\end{description}
\begin{description}
\item[{scipy.stats.distributions.logistic}] \leavevmode{[}probability density function,{]}
distribution or cumulative density function, etc.

\end{description}

The probability density for the Logistic distribution is
\begin{gather}
\begin{split}P(x) = P(x) = \frac{e^{-(x-\mu)/s}}{s(1+e^{-(x-\mu)/s})^2},\end{split}\notag
\end{gather}
where \(\mu\) = location and \(s\) = scale.

The Logistic distribution is used in Extreme Value problems where it
can act as a mixture of Gumbel distributions, in Epidemiology, and by
the World Chess Federation (FIDE) where it is used in the Elo ranking
system, assuming the performance of each player is a logistically
distributed random variable.

Draw samples from the distribution:

\begin{Verbatim}[commandchars=\\\{\}]
\PYG{g+gp}{\PYGZgt{}\PYGZgt{}\PYGZgt{} }\PYG{n}{loc}\PYG{p}{,} \PYG{n}{scale} \PYG{o}{=} \PYG{l+m+mi}{10}\PYG{p}{,} \PYG{l+m+mi}{1}
\PYG{g+gp}{\PYGZgt{}\PYGZgt{}\PYGZgt{} }\PYG{n}{s} \PYG{o}{=} \PYG{n}{np}\PYG{o}{.}\PYG{n}{random}\PYG{o}{.}\PYG{n}{logistic}\PYG{p}{(}\PYG{n}{loc}\PYG{p}{,} \PYG{n}{scale}\PYG{p}{,} \PYG{l+m+mi}{10000}\PYG{p}{)}
\PYG{g+gp}{\PYGZgt{}\PYGZgt{}\PYGZgt{} }\PYG{n}{count}\PYG{p}{,} \PYG{n}{bins}\PYG{p}{,} \PYG{n}{ignored} \PYG{o}{=} \PYG{n}{plt}\PYG{o}{.}\PYG{n}{hist}\PYG{p}{(}\PYG{n}{s}\PYG{p}{,} \PYG{n}{bins}\PYG{o}{=}\PYG{l+m+mi}{50}\PYG{p}{)}
\end{Verbatim}

\#   plot against distribution

\begin{Verbatim}[commandchars=\\\{\}]
\PYG{g+gp}{\PYGZgt{}\PYGZgt{}\PYGZgt{} }\PYG{k}{def} \PYG{n+nf}{logist}\PYG{p}{(}\PYG{n}{x}\PYG{p}{,} \PYG{n}{loc}\PYG{p}{,} \PYG{n}{scale}\PYG{p}{)}\PYG{p}{:}
\PYG{g+gp}{... }    \PYG{k}{return} \PYG{n}{exp}\PYG{p}{(}\PYG{p}{(}\PYG{n}{loc}\PYG{o}{\PYGZhy{}}\PYG{n}{x}\PYG{p}{)}\PYG{o}{/}\PYG{n}{scale}\PYG{p}{)}\PYG{o}{/}\PYG{p}{(}\PYG{n}{scale}\PYG{o}{*}\PYG{p}{(}\PYG{l+m+mi}{1}\PYG{o}{+}\PYG{n}{exp}\PYG{p}{(}\PYG{p}{(}\PYG{n}{loc}\PYG{o}{\PYGZhy{}}\PYG{n}{x}\PYG{p}{)}\PYG{o}{/}\PYG{n}{scale}\PYG{p}{)}\PYG{p}{)}\PYG{o}{*}\PYG{o}{*}\PYG{l+m+mi}{2}\PYG{p}{)}
\PYG{g+gp}{\PYGZgt{}\PYGZgt{}\PYGZgt{} }\PYG{n}{plt}\PYG{o}{.}\PYG{n}{plot}\PYG{p}{(}\PYG{n}{bins}\PYG{p}{,} \PYG{n}{logist}\PYG{p}{(}\PYG{n}{bins}\PYG{p}{,} \PYG{n}{loc}\PYG{p}{,} \PYG{n}{scale}\PYG{p}{)}\PYG{o}{*}\PYG{n}{count}\PYG{o}{.}\PYG{n}{max}\PYG{p}{(}\PYG{p}{)}\PYG{o}{/}\PYGZbs{}
\PYG{g+gp}{... }\PYG{n}{logist}\PYG{p}{(}\PYG{n}{bins}\PYG{p}{,} \PYG{n}{loc}\PYG{p}{,} \PYG{n}{scale}\PYG{p}{)}\PYG{o}{.}\PYG{n}{max}\PYG{p}{(}\PYG{p}{)}\PYG{p}{)}
\PYG{g+gp}{\PYGZgt{}\PYGZgt{}\PYGZgt{} }\PYG{n}{plt}\PYG{o}{.}\PYG{n}{show}\PYG{p}{(}\PYG{p}{)}
\end{Verbatim}

\end{fulllineitems}

\index{lognormal() (in module lib.dyn.dynamics)}

\begin{fulllineitems}
\phantomsection\label{lib.dyn:lib.dyn.dynamics.lognormal}\pysiglinewithargsret{\code{lib.dyn.dynamics.}\bfcode{lognormal}}{\emph{mean=0.0}, \emph{sigma=1.0}, \emph{size=None}}{}
Return samples drawn from a log-normal distribution.

Draw samples from a log-normal distribution with specified mean,
standard deviation, and array shape.  Note that the mean and standard
deviation are not the values for the distribution itself, but of the
underlying normal distribution it is derived from.
\begin{description}
\item[{mean}] \leavevmode{[}float{]}
Mean value of the underlying normal distribution

\item[{sigma}] \leavevmode{[}float, \textgreater{} 0.{]}
Standard deviation of the underlying normal distribution

\item[{size}] \leavevmode{[}tuple of ints{]}
Output shape.  If the given shape is, e.g., \code{(m, n, k)}, then
\code{m * n * k} samples are drawn.

\end{description}
\begin{description}
\item[{samples}] \leavevmode{[}ndarray or float{]}
The desired samples. An array of the same shape as \emph{size} if given,
if \emph{size} is None a float is returned.

\end{description}
\begin{description}
\item[{scipy.stats.lognorm}] \leavevmode{[}probability density function, distribution,{]}
cumulative density function, etc.

\end{description}

A variable \emph{x} has a log-normal distribution if \emph{log(x)} is normally
distributed.  The probability density function for the log-normal
distribution is:
\begin{gather}
\begin{split}p(x) = \frac{1}{\sigma x \sqrt{2\pi}}
e^{(-\frac{(ln(x)-\mu)^2}{2\sigma^2})}\end{split}\notag
\end{gather}
where \(\mu\) is the mean and \(\sigma\) is the standard
deviation of the normally distributed logarithm of the variable.
A log-normal distribution results if a random variable is the \emph{product}
of a large number of independent, identically-distributed variables in
the same way that a normal distribution results if the variable is the
\emph{sum} of a large number of independent, identically-distributed
variables.

Limpert, E., Stahel, W. A., and Abbt, M., ``Log-normal Distributions
across the Sciences: Keys and Clues,'' \emph{BioScience}, Vol. 51, No. 5,
May, 2001.  \href{http://stat.ethz.ch/~stahel/lognormal/bioscience.pdf}{http://stat.ethz.ch/\textasciitilde{}stahel/lognormal/bioscience.pdf}

Reiss, R.D. and Thomas, M., \emph{Statistical Analysis of Extreme Values},
Basel: Birkhauser Verlag, 2001, pp. 31-32.

Draw samples from the distribution:

\begin{Verbatim}[commandchars=\\\{\}]
\PYG{g+gp}{\PYGZgt{}\PYGZgt{}\PYGZgt{} }\PYG{n}{mu}\PYG{p}{,} \PYG{n}{sigma} \PYG{o}{=} \PYG{l+m+mf}{3.}\PYG{p}{,} \PYG{l+m+mf}{1.} \PYG{c}{\PYGZsh{} mean and standard deviation}
\PYG{g+gp}{\PYGZgt{}\PYGZgt{}\PYGZgt{} }\PYG{n}{s} \PYG{o}{=} \PYG{n}{np}\PYG{o}{.}\PYG{n}{random}\PYG{o}{.}\PYG{n}{lognormal}\PYG{p}{(}\PYG{n}{mu}\PYG{p}{,} \PYG{n}{sigma}\PYG{p}{,} \PYG{l+m+mi}{1000}\PYG{p}{)}
\end{Verbatim}

Display the histogram of the samples, along with
the probability density function:

\begin{Verbatim}[commandchars=\\\{\}]
\PYG{g+gp}{\PYGZgt{}\PYGZgt{}\PYGZgt{} }\PYG{k+kn}{import} \PYG{n+nn}{matplotlib.pyplot} \PYG{k+kn}{as} \PYG{n+nn}{plt}
\PYG{g+gp}{\PYGZgt{}\PYGZgt{}\PYGZgt{} }\PYG{n}{count}\PYG{p}{,} \PYG{n}{bins}\PYG{p}{,} \PYG{n}{ignored} \PYG{o}{=} \PYG{n}{plt}\PYG{o}{.}\PYG{n}{hist}\PYG{p}{(}\PYG{n}{s}\PYG{p}{,} \PYG{l+m+mi}{100}\PYG{p}{,} \PYG{n}{normed}\PYG{o}{=}\PYG{n+nb+bp}{True}\PYG{p}{,} \PYG{n}{align}\PYG{o}{=}\PYG{l+s}{\PYGZsq{}}\PYG{l+s}{mid}\PYG{l+s}{\PYGZsq{}}\PYG{p}{)}
\end{Verbatim}

\begin{Verbatim}[commandchars=\\\{\}]
\PYG{g+gp}{\PYGZgt{}\PYGZgt{}\PYGZgt{} }\PYG{n}{x} \PYG{o}{=} \PYG{n}{np}\PYG{o}{.}\PYG{n}{linspace}\PYG{p}{(}\PYG{n+nb}{min}\PYG{p}{(}\PYG{n}{bins}\PYG{p}{)}\PYG{p}{,} \PYG{n+nb}{max}\PYG{p}{(}\PYG{n}{bins}\PYG{p}{)}\PYG{p}{,} \PYG{l+m+mi}{10000}\PYG{p}{)}
\PYG{g+gp}{\PYGZgt{}\PYGZgt{}\PYGZgt{} }\PYG{n}{pdf} \PYG{o}{=} \PYG{p}{(}\PYG{n}{np}\PYG{o}{.}\PYG{n}{exp}\PYG{p}{(}\PYG{o}{\PYGZhy{}}\PYG{p}{(}\PYG{n}{np}\PYG{o}{.}\PYG{n}{log}\PYG{p}{(}\PYG{n}{x}\PYG{p}{)} \PYG{o}{\PYGZhy{}} \PYG{n}{mu}\PYG{p}{)}\PYG{o}{*}\PYG{o}{*}\PYG{l+m+mi}{2} \PYG{o}{/} \PYG{p}{(}\PYG{l+m+mi}{2} \PYG{o}{*} \PYG{n}{sigma}\PYG{o}{*}\PYG{o}{*}\PYG{l+m+mi}{2}\PYG{p}{)}\PYG{p}{)}
\PYG{g+gp}{... }       \PYG{o}{/} \PYG{p}{(}\PYG{n}{x} \PYG{o}{*} \PYG{n}{sigma} \PYG{o}{*} \PYG{n}{np}\PYG{o}{.}\PYG{n}{sqrt}\PYG{p}{(}\PYG{l+m+mi}{2} \PYG{o}{*} \PYG{n}{np}\PYG{o}{.}\PYG{n}{pi}\PYG{p}{)}\PYG{p}{)}\PYG{p}{)}
\end{Verbatim}

\begin{Verbatim}[commandchars=\\\{\}]
\PYG{g+gp}{\PYGZgt{}\PYGZgt{}\PYGZgt{} }\PYG{n}{plt}\PYG{o}{.}\PYG{n}{plot}\PYG{p}{(}\PYG{n}{x}\PYG{p}{,} \PYG{n}{pdf}\PYG{p}{,} \PYG{n}{linewidth}\PYG{o}{=}\PYG{l+m+mi}{2}\PYG{p}{,} \PYG{n}{color}\PYG{o}{=}\PYG{l+s}{\PYGZsq{}}\PYG{l+s}{r}\PYG{l+s}{\PYGZsq{}}\PYG{p}{)}
\PYG{g+gp}{\PYGZgt{}\PYGZgt{}\PYGZgt{} }\PYG{n}{plt}\PYG{o}{.}\PYG{n}{axis}\PYG{p}{(}\PYG{l+s}{\PYGZsq{}}\PYG{l+s}{tight}\PYG{l+s}{\PYGZsq{}}\PYG{p}{)}
\PYG{g+gp}{\PYGZgt{}\PYGZgt{}\PYGZgt{} }\PYG{n}{plt}\PYG{o}{.}\PYG{n}{show}\PYG{p}{(}\PYG{p}{)}
\end{Verbatim}

Demonstrate that taking the products of random samples from a uniform
distribution can be fit well by a log-normal probability density function.

\begin{Verbatim}[commandchars=\\\{\}]
\PYG{g+gp}{\PYGZgt{}\PYGZgt{}\PYGZgt{} }\PYG{c}{\PYGZsh{} Generate a thousand samples: each is the product of 100 random}
\PYG{g+gp}{\PYGZgt{}\PYGZgt{}\PYGZgt{} }\PYG{c}{\PYGZsh{} values, drawn from a normal distribution.}
\PYG{g+gp}{\PYGZgt{}\PYGZgt{}\PYGZgt{} }\PYG{n}{b} \PYG{o}{=} \PYG{p}{[}\PYG{p}{]}
\PYG{g+gp}{\PYGZgt{}\PYGZgt{}\PYGZgt{} }\PYG{k}{for} \PYG{n}{i} \PYG{o+ow}{in} \PYG{n+nb}{range}\PYG{p}{(}\PYG{l+m+mi}{1000}\PYG{p}{)}\PYG{p}{:}
\PYG{g+gp}{... }   \PYG{n}{a} \PYG{o}{=} \PYG{l+m+mf}{10.} \PYG{o}{+} \PYG{n}{np}\PYG{o}{.}\PYG{n}{random}\PYG{o}{.}\PYG{n}{random}\PYG{p}{(}\PYG{l+m+mi}{100}\PYG{p}{)}
\PYG{g+gp}{... }   \PYG{n}{b}\PYG{o}{.}\PYG{n}{append}\PYG{p}{(}\PYG{n}{np}\PYG{o}{.}\PYG{n}{product}\PYG{p}{(}\PYG{n}{a}\PYG{p}{)}\PYG{p}{)}
\end{Verbatim}

\begin{Verbatim}[commandchars=\\\{\}]
\PYG{g+gp}{\PYGZgt{}\PYGZgt{}\PYGZgt{} }\PYG{n}{b} \PYG{o}{=} \PYG{n}{np}\PYG{o}{.}\PYG{n}{array}\PYG{p}{(}\PYG{n}{b}\PYG{p}{)} \PYG{o}{/} \PYG{n}{np}\PYG{o}{.}\PYG{n}{min}\PYG{p}{(}\PYG{n}{b}\PYG{p}{)} \PYG{c}{\PYGZsh{} scale values to be positive}
\PYG{g+gp}{\PYGZgt{}\PYGZgt{}\PYGZgt{} }\PYG{n}{count}\PYG{p}{,} \PYG{n}{bins}\PYG{p}{,} \PYG{n}{ignored} \PYG{o}{=} \PYG{n}{plt}\PYG{o}{.}\PYG{n}{hist}\PYG{p}{(}\PYG{n}{b}\PYG{p}{,} \PYG{l+m+mi}{100}\PYG{p}{,} \PYG{n}{normed}\PYG{o}{=}\PYG{n+nb+bp}{True}\PYG{p}{,} \PYG{n}{align}\PYG{o}{=}\PYG{l+s}{\PYGZsq{}}\PYG{l+s}{center}\PYG{l+s}{\PYGZsq{}}\PYG{p}{)}
\PYG{g+gp}{\PYGZgt{}\PYGZgt{}\PYGZgt{} }\PYG{n}{sigma} \PYG{o}{=} \PYG{n}{np}\PYG{o}{.}\PYG{n}{std}\PYG{p}{(}\PYG{n}{np}\PYG{o}{.}\PYG{n}{log}\PYG{p}{(}\PYG{n}{b}\PYG{p}{)}\PYG{p}{)}
\PYG{g+gp}{\PYGZgt{}\PYGZgt{}\PYGZgt{} }\PYG{n}{mu} \PYG{o}{=} \PYG{n}{np}\PYG{o}{.}\PYG{n}{mean}\PYG{p}{(}\PYG{n}{np}\PYG{o}{.}\PYG{n}{log}\PYG{p}{(}\PYG{n}{b}\PYG{p}{)}\PYG{p}{)}
\end{Verbatim}

\begin{Verbatim}[commandchars=\\\{\}]
\PYG{g+gp}{\PYGZgt{}\PYGZgt{}\PYGZgt{} }\PYG{n}{x} \PYG{o}{=} \PYG{n}{np}\PYG{o}{.}\PYG{n}{linspace}\PYG{p}{(}\PYG{n+nb}{min}\PYG{p}{(}\PYG{n}{bins}\PYG{p}{)}\PYG{p}{,} \PYG{n+nb}{max}\PYG{p}{(}\PYG{n}{bins}\PYG{p}{)}\PYG{p}{,} \PYG{l+m+mi}{10000}\PYG{p}{)}
\PYG{g+gp}{\PYGZgt{}\PYGZgt{}\PYGZgt{} }\PYG{n}{pdf} \PYG{o}{=} \PYG{p}{(}\PYG{n}{np}\PYG{o}{.}\PYG{n}{exp}\PYG{p}{(}\PYG{o}{\PYGZhy{}}\PYG{p}{(}\PYG{n}{np}\PYG{o}{.}\PYG{n}{log}\PYG{p}{(}\PYG{n}{x}\PYG{p}{)} \PYG{o}{\PYGZhy{}} \PYG{n}{mu}\PYG{p}{)}\PYG{o}{*}\PYG{o}{*}\PYG{l+m+mi}{2} \PYG{o}{/} \PYG{p}{(}\PYG{l+m+mi}{2} \PYG{o}{*} \PYG{n}{sigma}\PYG{o}{*}\PYG{o}{*}\PYG{l+m+mi}{2}\PYG{p}{)}\PYG{p}{)}
\PYG{g+gp}{... }       \PYG{o}{/} \PYG{p}{(}\PYG{n}{x} \PYG{o}{*} \PYG{n}{sigma} \PYG{o}{*} \PYG{n}{np}\PYG{o}{.}\PYG{n}{sqrt}\PYG{p}{(}\PYG{l+m+mi}{2} \PYG{o}{*} \PYG{n}{np}\PYG{o}{.}\PYG{n}{pi}\PYG{p}{)}\PYG{p}{)}\PYG{p}{)}
\end{Verbatim}

\begin{Verbatim}[commandchars=\\\{\}]
\PYG{g+gp}{\PYGZgt{}\PYGZgt{}\PYGZgt{} }\PYG{n}{plt}\PYG{o}{.}\PYG{n}{plot}\PYG{p}{(}\PYG{n}{x}\PYG{p}{,} \PYG{n}{pdf}\PYG{p}{,} \PYG{n}{color}\PYG{o}{=}\PYG{l+s}{\PYGZsq{}}\PYG{l+s}{r}\PYG{l+s}{\PYGZsq{}}\PYG{p}{,} \PYG{n}{linewidth}\PYG{o}{=}\PYG{l+m+mi}{2}\PYG{p}{)}
\PYG{g+gp}{\PYGZgt{}\PYGZgt{}\PYGZgt{} }\PYG{n}{plt}\PYG{o}{.}\PYG{n}{show}\PYG{p}{(}\PYG{p}{)}
\end{Verbatim}

\end{fulllineitems}

\index{logseries() (in module lib.dyn.dynamics)}

\begin{fulllineitems}
\phantomsection\label{lib.dyn:lib.dyn.dynamics.logseries}\pysiglinewithargsret{\code{lib.dyn.dynamics.}\bfcode{logseries}}{\emph{p}, \emph{size=None}}{}
Draw samples from a Logarithmic Series distribution.

Samples are drawn from a Log Series distribution with specified
parameter, p (probability, 0 \textless{} p \textless{} 1).

loc : float

scale : float \textgreater{} 0.
\begin{description}
\item[{size}] \leavevmode{[}\{tuple, int\}{]}
Output shape.  If the given shape is, e.g., \code{(m, n, k)}, then
\code{m * n * k} samples are drawn.

\end{description}
\begin{description}
\item[{samples}] \leavevmode{[}\{ndarray, scalar\}{]}
where the values are all integers in  {[}0, n{]}.

\end{description}
\begin{description}
\item[{scipy.stats.distributions.logser}] \leavevmode{[}probability density function,{]}
distribution or cumulative density function, etc.

\end{description}

The probability density for the Log Series distribution is
\begin{gather}
\begin{split}P(k) = \frac{-p^k}{k \ln(1-p)},\end{split}\notag
\end{gather}
where p = probability.

The Log Series distribution is frequently used to represent species
richness and occurrence, first proposed by Fisher, Corbet, and
Williams in 1943 {[}2{]}.  It may also be used to model the numbers of
occupants seen in cars {[}3{]}.

Draw samples from the distribution:

\begin{Verbatim}[commandchars=\\\{\}]
\PYG{g+gp}{\PYGZgt{}\PYGZgt{}\PYGZgt{} }\PYG{n}{a} \PYG{o}{=} \PYG{o}{.}\PYG{l+m+mi}{6}
\PYG{g+gp}{\PYGZgt{}\PYGZgt{}\PYGZgt{} }\PYG{n}{s} \PYG{o}{=} \PYG{n}{np}\PYG{o}{.}\PYG{n}{random}\PYG{o}{.}\PYG{n}{logseries}\PYG{p}{(}\PYG{n}{a}\PYG{p}{,} \PYG{l+m+mi}{10000}\PYG{p}{)}
\PYG{g+gp}{\PYGZgt{}\PYGZgt{}\PYGZgt{} }\PYG{n}{count}\PYG{p}{,} \PYG{n}{bins}\PYG{p}{,} \PYG{n}{ignored} \PYG{o}{=} \PYG{n}{plt}\PYG{o}{.}\PYG{n}{hist}\PYG{p}{(}\PYG{n}{s}\PYG{p}{)}
\end{Verbatim}

\#   plot against distribution

\begin{Verbatim}[commandchars=\\\{\}]
\PYG{g+gp}{\PYGZgt{}\PYGZgt{}\PYGZgt{} }\PYG{k}{def} \PYG{n+nf}{logseries}\PYG{p}{(}\PYG{n}{k}\PYG{p}{,} \PYG{n}{p}\PYG{p}{)}\PYG{p}{:}
\PYG{g+gp}{... }    \PYG{k}{return} \PYG{o}{\PYGZhy{}}\PYG{n}{p}\PYG{o}{*}\PYG{o}{*}\PYG{n}{k}\PYG{o}{/}\PYG{p}{(}\PYG{n}{k}\PYG{o}{*}\PYG{n}{log}\PYG{p}{(}\PYG{l+m+mi}{1}\PYG{o}{\PYGZhy{}}\PYG{n}{p}\PYG{p}{)}\PYG{p}{)}
\PYG{g+gp}{\PYGZgt{}\PYGZgt{}\PYGZgt{} }\PYG{n}{plt}\PYG{o}{.}\PYG{n}{plot}\PYG{p}{(}\PYG{n}{bins}\PYG{p}{,} \PYG{n}{logseries}\PYG{p}{(}\PYG{n}{bins}\PYG{p}{,} \PYG{n}{a}\PYG{p}{)}\PYG{o}{*}\PYG{n}{count}\PYG{o}{.}\PYG{n}{max}\PYG{p}{(}\PYG{p}{)}\PYG{o}{/}
\PYG{g+go}{             logseries(bins, a).max(), \PYGZsq{}r\PYGZsq{})}
\PYG{g+gp}{\PYGZgt{}\PYGZgt{}\PYGZgt{} }\PYG{n}{plt}\PYG{o}{.}\PYG{n}{show}\PYG{p}{(}\PYG{p}{)}
\end{Verbatim}

\end{fulllineitems}

\index{multinomial() (in module lib.dyn.dynamics)}

\begin{fulllineitems}
\phantomsection\label{lib.dyn:lib.dyn.dynamics.multinomial}\pysiglinewithargsret{\code{lib.dyn.dynamics.}\bfcode{multinomial}}{\emph{n}, \emph{pvals}, \emph{size=None}}{}
Draw samples from a multinomial distribution.

The multinomial distribution is a multivariate generalisation of the
binomial distribution.  Take an experiment with one of \code{p}
possible outcomes.  An example of such an experiment is throwing a dice,
where the outcome can be 1 through 6.  Each sample drawn from the
distribution represents \emph{n} such experiments.  Its values,
\code{X\_i = {[}X\_0, X\_1, ..., X\_p{]}}, represent the number of times the outcome
was \code{i}.
\begin{description}
\item[{n}] \leavevmode{[}int{]}
Number of experiments.

\item[{pvals}] \leavevmode{[}sequence of floats, length p{]}
Probabilities of each of the \code{p} different outcomes.  These
should sum to 1 (however, the last element is always assumed to
account for the remaining probability, as long as
\code{sum(pvals{[}:-1{]}) \textless{}= 1)}.

\item[{size}] \leavevmode{[}tuple of ints{]}
Given a \emph{size} of \code{(M, N, K)}, then \code{M*N*K} samples are drawn,
and the output shape becomes \code{(M, N, K, p)}, since each sample
has shape \code{(p,)}.

\end{description}

Throw a dice 20 times:

\begin{Verbatim}[commandchars=\\\{\}]
\PYG{g+gp}{\PYGZgt{}\PYGZgt{}\PYGZgt{} }\PYG{n}{np}\PYG{o}{.}\PYG{n}{random}\PYG{o}{.}\PYG{n}{multinomial}\PYG{p}{(}\PYG{l+m+mi}{20}\PYG{p}{,} \PYG{p}{[}\PYG{l+m+mi}{1}\PYG{o}{/}\PYG{l+m+mf}{6.}\PYG{p}{]}\PYG{o}{*}\PYG{l+m+mi}{6}\PYG{p}{,} \PYG{n}{size}\PYG{o}{=}\PYG{l+m+mi}{1}\PYG{p}{)}
\PYG{g+go}{array([[4, 1, 7, 5, 2, 1]])}
\end{Verbatim}

It landed 4 times on 1, once on 2, etc.

Now, throw the dice 20 times, and 20 times again:

\begin{Verbatim}[commandchars=\\\{\}]
\PYG{g+gp}{\PYGZgt{}\PYGZgt{}\PYGZgt{} }\PYG{n}{np}\PYG{o}{.}\PYG{n}{random}\PYG{o}{.}\PYG{n}{multinomial}\PYG{p}{(}\PYG{l+m+mi}{20}\PYG{p}{,} \PYG{p}{[}\PYG{l+m+mi}{1}\PYG{o}{/}\PYG{l+m+mf}{6.}\PYG{p}{]}\PYG{o}{*}\PYG{l+m+mi}{6}\PYG{p}{,} \PYG{n}{size}\PYG{o}{=}\PYG{l+m+mi}{2}\PYG{p}{)}
\PYG{g+go}{array([[3, 4, 3, 3, 4, 3],}
\PYG{g+go}{       [2, 4, 3, 4, 0, 7]])}
\end{Verbatim}

For the first run, we threw 3 times 1, 4 times 2, etc.  For the second,
we threw 2 times 1, 4 times 2, etc.

A loaded dice is more likely to land on number 6:

\begin{Verbatim}[commandchars=\\\{\}]
\PYG{g+gp}{\PYGZgt{}\PYGZgt{}\PYGZgt{} }\PYG{n}{np}\PYG{o}{.}\PYG{n}{random}\PYG{o}{.}\PYG{n}{multinomial}\PYG{p}{(}\PYG{l+m+mi}{100}\PYG{p}{,} \PYG{p}{[}\PYG{l+m+mi}{1}\PYG{o}{/}\PYG{l+m+mf}{7.}\PYG{p}{]}\PYG{o}{*}\PYG{l+m+mi}{5}\PYG{p}{)}
\PYG{g+go}{array([13, 16, 13, 16, 42])}
\end{Verbatim}

\end{fulllineitems}

\index{multivariate\_normal() (in module lib.dyn.dynamics)}

\begin{fulllineitems}
\phantomsection\label{lib.dyn:lib.dyn.dynamics.multivariate_normal}\pysiglinewithargsret{\code{lib.dyn.dynamics.}\bfcode{multivariate\_normal}}{\emph{mean}, \emph{cov}\optional{, \emph{size}}}{}
Draw random samples from a multivariate normal distribution.

The multivariate normal, multinormal or Gaussian distribution is a
generalization of the one-dimensional normal distribution to higher
dimensions.  Such a distribution is specified by its mean and
covariance matrix.  These parameters are analogous to the mean
(average or ``center'') and variance (standard deviation, or ``width,''
squared) of the one-dimensional normal distribution.
\begin{description}
\item[{mean}] \leavevmode{[}1-D array\_like, of length N{]}
Mean of the N-dimensional distribution.

\item[{cov}] \leavevmode{[}2-D array\_like, of shape (N, N){]}
Covariance matrix of the distribution.  Must be symmetric and
positive semi-definite for ``physically meaningful'' results.

\item[{size}] \leavevmode{[}int or tuple of ints, optional{]}
Given a shape of, for example, \code{(m,n,k)}, \code{m*n*k} samples are
generated, and packed in an \emph{m}-by-\emph{n}-by-\emph{k} arrangement.  Because
each sample is \emph{N}-dimensional, the output shape is \code{(m,n,k,N)}.
If no shape is specified, a single (\emph{N}-D) sample is returned.

\end{description}
\begin{description}
\item[{out}] \leavevmode{[}ndarray{]}
The drawn samples, of shape \emph{size}, if that was provided.  If not,
the shape is \code{(N,)}.

In other words, each entry \code{out{[}i,j,...,:{]}} is an N-dimensional
value drawn from the distribution.

\end{description}

The mean is a coordinate in N-dimensional space, which represents the
location where samples are most likely to be generated.  This is
analogous to the peak of the bell curve for the one-dimensional or
univariate normal distribution.

Covariance indicates the level to which two variables vary together.
From the multivariate normal distribution, we draw N-dimensional
samples, \(X = [x_1, x_2, ... x_N]\).  The covariance matrix
element \(C_{ij}\) is the covariance of \(x_i\) and \(x_j\).
The element \(C_{ii}\) is the variance of \(x_i\) (i.e. its
``spread'').

Instead of specifying the full covariance matrix, popular
approximations include:
\begin{itemize}
\item {} 
Spherical covariance (\emph{cov} is a multiple of the identity matrix)

\item {} 
Diagonal covariance (\emph{cov} has non-negative elements, and only on
the diagonal)

\end{itemize}

This geometrical property can be seen in two dimensions by plotting
generated data-points:

\begin{Verbatim}[commandchars=\\\{\}]
\PYG{g+gp}{\PYGZgt{}\PYGZgt{}\PYGZgt{} }\PYG{n}{mean} \PYG{o}{=} \PYG{p}{[}\PYG{l+m+mi}{0}\PYG{p}{,}\PYG{l+m+mi}{0}\PYG{p}{]}
\PYG{g+gp}{\PYGZgt{}\PYGZgt{}\PYGZgt{} }\PYG{n}{cov} \PYG{o}{=} \PYG{p}{[}\PYG{p}{[}\PYG{l+m+mi}{1}\PYG{p}{,}\PYG{l+m+mi}{0}\PYG{p}{]}\PYG{p}{,}\PYG{p}{[}\PYG{l+m+mi}{0}\PYG{p}{,}\PYG{l+m+mi}{100}\PYG{p}{]}\PYG{p}{]} \PYG{c}{\PYGZsh{} diagonal covariance, points lie on x or y\PYGZhy{}axis}
\end{Verbatim}

\begin{Verbatim}[commandchars=\\\{\}]
\PYG{g+gp}{\PYGZgt{}\PYGZgt{}\PYGZgt{} }\PYG{k+kn}{import} \PYG{n+nn}{matplotlib.pyplot} \PYG{k+kn}{as} \PYG{n+nn}{plt}
\PYG{g+gp}{\PYGZgt{}\PYGZgt{}\PYGZgt{} }\PYG{n}{x}\PYG{p}{,}\PYG{n}{y} \PYG{o}{=} \PYG{n}{np}\PYG{o}{.}\PYG{n}{random}\PYG{o}{.}\PYG{n}{multivariate\PYGZus{}normal}\PYG{p}{(}\PYG{n}{mean}\PYG{p}{,}\PYG{n}{cov}\PYG{p}{,}\PYG{l+m+mi}{5000}\PYG{p}{)}\PYG{o}{.}\PYG{n}{T}
\PYG{g+gp}{\PYGZgt{}\PYGZgt{}\PYGZgt{} }\PYG{n}{plt}\PYG{o}{.}\PYG{n}{plot}\PYG{p}{(}\PYG{n}{x}\PYG{p}{,}\PYG{n}{y}\PYG{p}{,}\PYG{l+s}{\PYGZsq{}}\PYG{l+s}{x}\PYG{l+s}{\PYGZsq{}}\PYG{p}{)}\PYG{p}{;} \PYG{n}{plt}\PYG{o}{.}\PYG{n}{axis}\PYG{p}{(}\PYG{l+s}{\PYGZsq{}}\PYG{l+s}{equal}\PYG{l+s}{\PYGZsq{}}\PYG{p}{)}\PYG{p}{;} \PYG{n}{plt}\PYG{o}{.}\PYG{n}{show}\PYG{p}{(}\PYG{p}{)}
\end{Verbatim}

Note that the covariance matrix must be non-negative definite.

Papoulis, A., \emph{Probability, Random Variables, and Stochastic Processes},
3rd ed., New York: McGraw-Hill, 1991.

Duda, R. O., Hart, P. E., and Stork, D. G., \emph{Pattern Classification},
2nd ed., New York: Wiley, 2001.

\begin{Verbatim}[commandchars=\\\{\}]
\PYG{g+gp}{\PYGZgt{}\PYGZgt{}\PYGZgt{} }\PYG{n}{mean} \PYG{o}{=} \PYG{p}{(}\PYG{l+m+mi}{1}\PYG{p}{,}\PYG{l+m+mi}{2}\PYG{p}{)}
\PYG{g+gp}{\PYGZgt{}\PYGZgt{}\PYGZgt{} }\PYG{n}{cov} \PYG{o}{=} \PYG{p}{[}\PYG{p}{[}\PYG{l+m+mi}{1}\PYG{p}{,}\PYG{l+m+mi}{0}\PYG{p}{]}\PYG{p}{,}\PYG{p}{[}\PYG{l+m+mi}{1}\PYG{p}{,}\PYG{l+m+mi}{0}\PYG{p}{]}\PYG{p}{]}
\PYG{g+gp}{\PYGZgt{}\PYGZgt{}\PYGZgt{} }\PYG{n}{x} \PYG{o}{=} \PYG{n}{np}\PYG{o}{.}\PYG{n}{random}\PYG{o}{.}\PYG{n}{multivariate\PYGZus{}normal}\PYG{p}{(}\PYG{n}{mean}\PYG{p}{,}\PYG{n}{cov}\PYG{p}{,}\PYG{p}{(}\PYG{l+m+mi}{3}\PYG{p}{,}\PYG{l+m+mi}{3}\PYG{p}{)}\PYG{p}{)}
\PYG{g+gp}{\PYGZgt{}\PYGZgt{}\PYGZgt{} }\PYG{n}{x}\PYG{o}{.}\PYG{n}{shape}
\PYG{g+go}{(3, 3, 2)}
\end{Verbatim}

The following is probably true, given that 0.6 is roughly twice the
standard deviation:

\begin{Verbatim}[commandchars=\\\{\}]
\PYG{g+gp}{\PYGZgt{}\PYGZgt{}\PYGZgt{} }\PYG{k}{print} \PYG{n+nb}{list}\PYG{p}{(} \PYG{p}{(}\PYG{n}{x}\PYG{p}{[}\PYG{l+m+mi}{0}\PYG{p}{,}\PYG{l+m+mi}{0}\PYG{p}{,}\PYG{p}{:}\PYG{p}{]} \PYG{o}{\PYGZhy{}} \PYG{n}{mean}\PYG{p}{)} \PYG{o}{\PYGZlt{}} \PYG{l+m+mf}{0.6} \PYG{p}{)}
\PYG{g+go}{[True, True]}
\end{Verbatim}

\end{fulllineitems}

\index{negative\_binomial() (in module lib.dyn.dynamics)}

\begin{fulllineitems}
\phantomsection\label{lib.dyn:lib.dyn.dynamics.negative_binomial}\pysiglinewithargsret{\code{lib.dyn.dynamics.}\bfcode{negative\_binomial}}{\emph{n}, \emph{p}, \emph{size=None}}{}
Draw samples from a negative\_binomial distribution.

Samples are drawn from a negative\_Binomial distribution with specified
parameters, \emph{n} trials and \emph{p} probability of success where \emph{n} is an
integer \textgreater{} 0 and \emph{p} is in the interval {[}0, 1{]}.
\begin{description}
\item[{n}] \leavevmode{[}int{]}
Parameter, \textgreater{} 0.

\item[{p}] \leavevmode{[}float{]}
Parameter, \textgreater{}= 0 and \textless{}=1.

\item[{size}] \leavevmode{[}int or tuple of ints{]}
Output shape. If the given shape is, e.g., \code{(m, n, k)}, then
\code{m * n * k} samples are drawn.

\end{description}
\begin{description}
\item[{samples}] \leavevmode{[}int or ndarray of ints{]}
Drawn samples.

\end{description}

The probability density for the Negative Binomial distribution is
\begin{gather}
\begin{split}P(N;n,p) = \binom{N+n-1}{n-1}p^{n}(1-p)^{N},\end{split}\notag
\end{gather}
where \(n-1\) is the number of successes, \(p\) is the probability
of success, and \(N+n-1\) is the number of trials.

The negative binomial distribution gives the probability of n-1 successes
and N failures in N+n-1 trials, and success on the (N+n)th trial.

If one throws a die repeatedly until the third time a ``1'' appears, then the
probability distribution of the number of non-``1''s that appear before the
third ``1'' is a negative binomial distribution.

Draw samples from the distribution:

A real world example. A company drills wild-cat oil exploration wells, each
with an estimated probability of success of 0.1.  What is the probability
of having one success for each successive well, that is what is the
probability of a single success after drilling 5 wells, after 6 wells,
etc.?

\begin{Verbatim}[commandchars=\\\{\}]
\PYG{g+gp}{\PYGZgt{}\PYGZgt{}\PYGZgt{} }\PYG{n}{s} \PYG{o}{=} \PYG{n}{np}\PYG{o}{.}\PYG{n}{random}\PYG{o}{.}\PYG{n}{negative\PYGZus{}binomial}\PYG{p}{(}\PYG{l+m+mi}{1}\PYG{p}{,} \PYG{l+m+mf}{0.1}\PYG{p}{,} \PYG{l+m+mi}{100000}\PYG{p}{)}
\PYG{g+gp}{\PYGZgt{}\PYGZgt{}\PYGZgt{} }\PYG{k}{for} \PYG{n}{i} \PYG{o+ow}{in} \PYG{n+nb}{range}\PYG{p}{(}\PYG{l+m+mi}{1}\PYG{p}{,} \PYG{l+m+mi}{11}\PYG{p}{)}\PYG{p}{:}
\PYG{g+gp}{... }   \PYG{n}{probability} \PYG{o}{=} \PYG{n+nb}{sum}\PYG{p}{(}\PYG{n}{s}\PYG{o}{\PYGZlt{}}\PYG{n}{i}\PYG{p}{)} \PYG{o}{/} \PYG{l+m+mf}{100000.}
\PYG{g+gp}{... }   \PYG{k}{print} \PYG{n}{i}\PYG{p}{,} \PYG{l+s}{\PYGZdq{}}\PYG{l+s}{wells drilled, probability of one success =}\PYG{l+s}{\PYGZdq{}}\PYG{p}{,} \PYG{n}{probability}
\end{Verbatim}

\end{fulllineitems}

\index{noncentral\_chisquare() (in module lib.dyn.dynamics)}

\begin{fulllineitems}
\phantomsection\label{lib.dyn:lib.dyn.dynamics.noncentral_chisquare}\pysiglinewithargsret{\code{lib.dyn.dynamics.}\bfcode{noncentral\_chisquare}}{\emph{df}, \emph{nonc}, \emph{size=None}}{}
Draw samples from a noncentral chi-square distribution.

The noncentral \(\chi^2\) distribution is a generalisation of
the \(\chi^2\) distribution.
\begin{description}
\item[{df}] \leavevmode{[}int{]}
Degrees of freedom, should be \textgreater{}= 1.

\item[{nonc}] \leavevmode{[}float{]}
Non-centrality, should be \textgreater{} 0.

\item[{size}] \leavevmode{[}int or tuple of ints{]}
Shape of the output.

\end{description}

The probability density function for the noncentral Chi-square distribution
is
\begin{gather}
\begin{split}P(x;df,nonc) = \sum^{\infty}_{i=0}
\frac{e^{-nonc/2}(nonc/2)^{i}}{i!}P_{Y_{df+2i}}(x),\end{split}\notag
\end{gather}
where \(Y_{q}\) is the Chi-square with q degrees of freedom.

In Delhi (2007), it is noted that the noncentral chi-square is useful in
bombing and coverage problems, the probability of killing the point target
given by the noncentral chi-squared distribution.

Draw values from the distribution and plot the histogram

\begin{Verbatim}[commandchars=\\\{\}]
\PYG{g+gp}{\PYGZgt{}\PYGZgt{}\PYGZgt{} }\PYG{k+kn}{import} \PYG{n+nn}{matplotlib.pyplot} \PYG{k+kn}{as} \PYG{n+nn}{plt}
\PYG{g+gp}{\PYGZgt{}\PYGZgt{}\PYGZgt{} }\PYG{n}{values} \PYG{o}{=} \PYG{n}{plt}\PYG{o}{.}\PYG{n}{hist}\PYG{p}{(}\PYG{n}{np}\PYG{o}{.}\PYG{n}{random}\PYG{o}{.}\PYG{n}{noncentral\PYGZus{}chisquare}\PYG{p}{(}\PYG{l+m+mi}{3}\PYG{p}{,} \PYG{l+m+mi}{20}\PYG{p}{,} \PYG{l+m+mi}{100000}\PYG{p}{)}\PYG{p}{,}
\PYG{g+gp}{... }                  \PYG{n}{bins}\PYG{o}{=}\PYG{l+m+mi}{200}\PYG{p}{,} \PYG{n}{normed}\PYG{o}{=}\PYG{n+nb+bp}{True}\PYG{p}{)}
\PYG{g+gp}{\PYGZgt{}\PYGZgt{}\PYGZgt{} }\PYG{n}{plt}\PYG{o}{.}\PYG{n}{show}\PYG{p}{(}\PYG{p}{)}
\end{Verbatim}

Draw values from a noncentral chisquare with very small noncentrality,
and compare to a chisquare.

\begin{Verbatim}[commandchars=\\\{\}]
\PYG{g+gp}{\PYGZgt{}\PYGZgt{}\PYGZgt{} }\PYG{n}{plt}\PYG{o}{.}\PYG{n}{figure}\PYG{p}{(}\PYG{p}{)}
\PYG{g+gp}{\PYGZgt{}\PYGZgt{}\PYGZgt{} }\PYG{n}{values} \PYG{o}{=} \PYG{n}{plt}\PYG{o}{.}\PYG{n}{hist}\PYG{p}{(}\PYG{n}{np}\PYG{o}{.}\PYG{n}{random}\PYG{o}{.}\PYG{n}{noncentral\PYGZus{}chisquare}\PYG{p}{(}\PYG{l+m+mi}{3}\PYG{p}{,} \PYG{o}{.}\PYG{l+m+mo}{0000001}\PYG{p}{,} \PYG{l+m+mi}{100000}\PYG{p}{)}\PYG{p}{,}
\PYG{g+gp}{... }                  \PYG{n}{bins}\PYG{o}{=}\PYG{n}{np}\PYG{o}{.}\PYG{n}{arange}\PYG{p}{(}\PYG{l+m+mf}{0.}\PYG{p}{,} \PYG{l+m+mi}{25}\PYG{p}{,} \PYG{o}{.}\PYG{l+m+mi}{1}\PYG{p}{)}\PYG{p}{,} \PYG{n}{normed}\PYG{o}{=}\PYG{n+nb+bp}{True}\PYG{p}{)}
\PYG{g+gp}{\PYGZgt{}\PYGZgt{}\PYGZgt{} }\PYG{n}{values2} \PYG{o}{=} \PYG{n}{plt}\PYG{o}{.}\PYG{n}{hist}\PYG{p}{(}\PYG{n}{np}\PYG{o}{.}\PYG{n}{random}\PYG{o}{.}\PYG{n}{chisquare}\PYG{p}{(}\PYG{l+m+mi}{3}\PYG{p}{,} \PYG{l+m+mi}{100000}\PYG{p}{)}\PYG{p}{,}
\PYG{g+gp}{... }                   \PYG{n}{bins}\PYG{o}{=}\PYG{n}{np}\PYG{o}{.}\PYG{n}{arange}\PYG{p}{(}\PYG{l+m+mf}{0.}\PYG{p}{,} \PYG{l+m+mi}{25}\PYG{p}{,} \PYG{o}{.}\PYG{l+m+mi}{1}\PYG{p}{)}\PYG{p}{,} \PYG{n}{normed}\PYG{o}{=}\PYG{n+nb+bp}{True}\PYG{p}{)}
\PYG{g+gp}{\PYGZgt{}\PYGZgt{}\PYGZgt{} }\PYG{n}{plt}\PYG{o}{.}\PYG{n}{plot}\PYG{p}{(}\PYG{n}{values}\PYG{p}{[}\PYG{l+m+mi}{1}\PYG{p}{]}\PYG{p}{[}\PYG{l+m+mi}{0}\PYG{p}{:}\PYG{o}{\PYGZhy{}}\PYG{l+m+mi}{1}\PYG{p}{]}\PYG{p}{,} \PYG{n}{values}\PYG{p}{[}\PYG{l+m+mi}{0}\PYG{p}{]}\PYG{o}{\PYGZhy{}}\PYG{n}{values2}\PYG{p}{[}\PYG{l+m+mi}{0}\PYG{p}{]}\PYG{p}{,} \PYG{l+s}{\PYGZsq{}}\PYG{l+s}{ob}\PYG{l+s}{\PYGZsq{}}\PYG{p}{)}
\PYG{g+gp}{\PYGZgt{}\PYGZgt{}\PYGZgt{} }\PYG{n}{plt}\PYG{o}{.}\PYG{n}{show}\PYG{p}{(}\PYG{p}{)}
\end{Verbatim}

Demonstrate how large values of non-centrality lead to a more symmetric
distribution.

\begin{Verbatim}[commandchars=\\\{\}]
\PYG{g+gp}{\PYGZgt{}\PYGZgt{}\PYGZgt{} }\PYG{n}{plt}\PYG{o}{.}\PYG{n}{figure}\PYG{p}{(}\PYG{p}{)}
\PYG{g+gp}{\PYGZgt{}\PYGZgt{}\PYGZgt{} }\PYG{n}{values} \PYG{o}{=} \PYG{n}{plt}\PYG{o}{.}\PYG{n}{hist}\PYG{p}{(}\PYG{n}{np}\PYG{o}{.}\PYG{n}{random}\PYG{o}{.}\PYG{n}{noncentral\PYGZus{}chisquare}\PYG{p}{(}\PYG{l+m+mi}{3}\PYG{p}{,} \PYG{l+m+mi}{20}\PYG{p}{,} \PYG{l+m+mi}{100000}\PYG{p}{)}\PYG{p}{,}
\PYG{g+gp}{... }                  \PYG{n}{bins}\PYG{o}{=}\PYG{l+m+mi}{200}\PYG{p}{,} \PYG{n}{normed}\PYG{o}{=}\PYG{n+nb+bp}{True}\PYG{p}{)}
\PYG{g+gp}{\PYGZgt{}\PYGZgt{}\PYGZgt{} }\PYG{n}{plt}\PYG{o}{.}\PYG{n}{show}\PYG{p}{(}\PYG{p}{)}
\end{Verbatim}

\end{fulllineitems}

\index{noncentral\_f() (in module lib.dyn.dynamics)}

\begin{fulllineitems}
\phantomsection\label{lib.dyn:lib.dyn.dynamics.noncentral_f}\pysiglinewithargsret{\code{lib.dyn.dynamics.}\bfcode{noncentral\_f}}{\emph{dfnum}, \emph{dfden}, \emph{nonc}, \emph{size=None}}{}
Draw samples from the noncentral F distribution.

Samples are drawn from an F distribution with specified parameters,
\emph{dfnum} (degrees of freedom in numerator) and \emph{dfden} (degrees of
freedom in denominator), where both parameters \textgreater{} 1.
\emph{nonc} is the non-centrality parameter.
\begin{description}
\item[{dfnum}] \leavevmode{[}int{]}
Parameter, should be \textgreater{} 1.

\item[{dfden}] \leavevmode{[}int{]}
Parameter, should be \textgreater{} 1.

\item[{nonc}] \leavevmode{[}float{]}
Parameter, should be \textgreater{}= 0.

\item[{size}] \leavevmode{[}int or tuple of ints{]}
Output shape. If the given shape is, e.g., \code{(m, n, k)}, then
\code{m * n * k} samples are drawn.

\end{description}
\begin{description}
\item[{samples}] \leavevmode{[}scalar or ndarray{]}
Drawn samples.

\end{description}

When calculating the power of an experiment (power = probability of
rejecting the null hypothesis when a specific alternative is true) the
non-central F statistic becomes important.  When the null hypothesis is
true, the F statistic follows a central F distribution. When the null
hypothesis is not true, then it follows a non-central F statistic.

Weisstein, Eric W. ``Noncentral F-Distribution.'' From MathWorld--A Wolfram
Web Resource.  \href{http://mathworld.wolfram.com/NoncentralF-Distribution.html}{http://mathworld.wolfram.com/NoncentralF-Distribution.html}

Wikipedia, ``Noncentral F distribution'',
\href{http://en.wikipedia.org/wiki/Noncentral\_F-distribution}{http://en.wikipedia.org/wiki/Noncentral\_F-distribution}

In a study, testing for a specific alternative to the null hypothesis
requires use of the Noncentral F distribution. We need to calculate the
area in the tail of the distribution that exceeds the value of the F
distribution for the null hypothesis.  We'll plot the two probability
distributions for comparison.

\begin{Verbatim}[commandchars=\\\{\}]
\PYG{g+gp}{\PYGZgt{}\PYGZgt{}\PYGZgt{} }\PYG{n}{dfnum} \PYG{o}{=} \PYG{l+m+mi}{3} \PYG{c}{\PYGZsh{} between group deg of freedom}
\PYG{g+gp}{\PYGZgt{}\PYGZgt{}\PYGZgt{} }\PYG{n}{dfden} \PYG{o}{=} \PYG{l+m+mi}{20} \PYG{c}{\PYGZsh{} within groups degrees of freedom}
\PYG{g+gp}{\PYGZgt{}\PYGZgt{}\PYGZgt{} }\PYG{n}{nonc} \PYG{o}{=} \PYG{l+m+mf}{3.0}
\PYG{g+gp}{\PYGZgt{}\PYGZgt{}\PYGZgt{} }\PYG{n}{nc\PYGZus{}vals} \PYG{o}{=} \PYG{n}{np}\PYG{o}{.}\PYG{n}{random}\PYG{o}{.}\PYG{n}{noncentral\PYGZus{}f}\PYG{p}{(}\PYG{n}{dfnum}\PYG{p}{,} \PYG{n}{dfden}\PYG{p}{,} \PYG{n}{nonc}\PYG{p}{,} \PYG{l+m+mi}{1000000}\PYG{p}{)}
\PYG{g+gp}{\PYGZgt{}\PYGZgt{}\PYGZgt{} }\PYG{n}{NF} \PYG{o}{=} \PYG{n}{np}\PYG{o}{.}\PYG{n}{histogram}\PYG{p}{(}\PYG{n}{nc\PYGZus{}vals}\PYG{p}{,} \PYG{n}{bins}\PYG{o}{=}\PYG{l+m+mi}{50}\PYG{p}{,} \PYG{n}{normed}\PYG{o}{=}\PYG{n+nb+bp}{True}\PYG{p}{)}
\PYG{g+gp}{\PYGZgt{}\PYGZgt{}\PYGZgt{} }\PYG{n}{c\PYGZus{}vals} \PYG{o}{=} \PYG{n}{np}\PYG{o}{.}\PYG{n}{random}\PYG{o}{.}\PYG{n}{f}\PYG{p}{(}\PYG{n}{dfnum}\PYG{p}{,} \PYG{n}{dfden}\PYG{p}{,} \PYG{l+m+mi}{1000000}\PYG{p}{)}
\PYG{g+gp}{\PYGZgt{}\PYGZgt{}\PYGZgt{} }\PYG{n}{F} \PYG{o}{=} \PYG{n}{np}\PYG{o}{.}\PYG{n}{histogram}\PYG{p}{(}\PYG{n}{c\PYGZus{}vals}\PYG{p}{,} \PYG{n}{bins}\PYG{o}{=}\PYG{l+m+mi}{50}\PYG{p}{,} \PYG{n}{normed}\PYG{o}{=}\PYG{n+nb+bp}{True}\PYG{p}{)}
\PYG{g+gp}{\PYGZgt{}\PYGZgt{}\PYGZgt{} }\PYG{n}{plt}\PYG{o}{.}\PYG{n}{plot}\PYG{p}{(}\PYG{n}{F}\PYG{p}{[}\PYG{l+m+mi}{1}\PYG{p}{]}\PYG{p}{[}\PYG{l+m+mi}{1}\PYG{p}{:}\PYG{p}{]}\PYG{p}{,} \PYG{n}{F}\PYG{p}{[}\PYG{l+m+mi}{0}\PYG{p}{]}\PYG{p}{)}
\PYG{g+gp}{\PYGZgt{}\PYGZgt{}\PYGZgt{} }\PYG{n}{plt}\PYG{o}{.}\PYG{n}{plot}\PYG{p}{(}\PYG{n}{NF}\PYG{p}{[}\PYG{l+m+mi}{1}\PYG{p}{]}\PYG{p}{[}\PYG{l+m+mi}{1}\PYG{p}{:}\PYG{p}{]}\PYG{p}{,} \PYG{n}{NF}\PYG{p}{[}\PYG{l+m+mi}{0}\PYG{p}{]}\PYG{p}{)}
\PYG{g+gp}{\PYGZgt{}\PYGZgt{}\PYGZgt{} }\PYG{n}{plt}\PYG{o}{.}\PYG{n}{show}\PYG{p}{(}\PYG{p}{)}
\end{Verbatim}

\end{fulllineitems}

\index{normal() (in module lib.dyn.dynamics)}

\begin{fulllineitems}
\phantomsection\label{lib.dyn:lib.dyn.dynamics.normal}\pysiglinewithargsret{\code{lib.dyn.dynamics.}\bfcode{normal}}{\emph{loc=0.0}, \emph{scale=1.0}, \emph{size=None}}{}
Draw random samples from a normal (Gaussian) distribution.

The probability density function of the normal distribution, first
derived by De Moivre and 200 years later by both Gauss and Laplace
independently {\color{red}\bfseries{}{[}2{]}\_}, is often called the bell curve because of
its characteristic shape (see the example below).

The normal distributions occurs often in nature.  For example, it
describes the commonly occurring distribution of samples influenced
by a large number of tiny, random disturbances, each with its own
unique distribution {\color{red}\bfseries{}{[}2{]}\_}.
\begin{description}
\item[{loc}] \leavevmode{[}float{]}
Mean (``centre'') of the distribution.

\item[{scale}] \leavevmode{[}float{]}
Standard deviation (spread or ``width'') of the distribution.

\item[{size}] \leavevmode{[}tuple of ints{]}
Output shape.  If the given shape is, e.g., \code{(m, n, k)}, then
\code{m * n * k} samples are drawn.

\end{description}
\begin{description}
\item[{scipy.stats.distributions.norm}] \leavevmode{[}probability density function,{]}
distribution or cumulative density function, etc.

\end{description}

The probability density for the Gaussian distribution is
\begin{gather}
\begin{split}p(x) = \frac{1}{\sqrt{ 2 \pi \sigma^2 }}
e^{ - \frac{ (x - \mu)^2 } {2 \sigma^2} },\end{split}\notag
\end{gather}
where \(\mu\) is the mean and \(\sigma\) the standard deviation.
The square of the standard deviation, \(\sigma^2\), is called the
variance.

The function has its peak at the mean, and its ``spread'' increases with
the standard deviation (the function reaches 0.607 times its maximum at
\(x + \sigma\) and \(x - \sigma\) {\color{red}\bfseries{}{[}2{]}\_}).  This implies that
\emph{numpy.random.normal} is more likely to return samples lying close to the
mean, rather than those far away.

Draw samples from the distribution:

\begin{Verbatim}[commandchars=\\\{\}]
\PYG{g+gp}{\PYGZgt{}\PYGZgt{}\PYGZgt{} }\PYG{n}{mu}\PYG{p}{,} \PYG{n}{sigma} \PYG{o}{=} \PYG{l+m+mi}{0}\PYG{p}{,} \PYG{l+m+mf}{0.1} \PYG{c}{\PYGZsh{} mean and standard deviation}
\PYG{g+gp}{\PYGZgt{}\PYGZgt{}\PYGZgt{} }\PYG{n}{s} \PYG{o}{=} \PYG{n}{np}\PYG{o}{.}\PYG{n}{random}\PYG{o}{.}\PYG{n}{normal}\PYG{p}{(}\PYG{n}{mu}\PYG{p}{,} \PYG{n}{sigma}\PYG{p}{,} \PYG{l+m+mi}{1000}\PYG{p}{)}
\end{Verbatim}

Verify the mean and the variance:

\begin{Verbatim}[commandchars=\\\{\}]
\PYG{g+gp}{\PYGZgt{}\PYGZgt{}\PYGZgt{} }\PYG{n+nb}{abs}\PYG{p}{(}\PYG{n}{mu} \PYG{o}{\PYGZhy{}} \PYG{n}{np}\PYG{o}{.}\PYG{n}{mean}\PYG{p}{(}\PYG{n}{s}\PYG{p}{)}\PYG{p}{)} \PYG{o}{\PYGZlt{}} \PYG{l+m+mf}{0.01}
\PYG{g+go}{True}
\end{Verbatim}

\begin{Verbatim}[commandchars=\\\{\}]
\PYG{g+gp}{\PYGZgt{}\PYGZgt{}\PYGZgt{} }\PYG{n+nb}{abs}\PYG{p}{(}\PYG{n}{sigma} \PYG{o}{\PYGZhy{}} \PYG{n}{np}\PYG{o}{.}\PYG{n}{std}\PYG{p}{(}\PYG{n}{s}\PYG{p}{,} \PYG{n}{ddof}\PYG{o}{=}\PYG{l+m+mi}{1}\PYG{p}{)}\PYG{p}{)} \PYG{o}{\PYGZlt{}} \PYG{l+m+mf}{0.01}
\PYG{g+go}{True}
\end{Verbatim}

Display the histogram of the samples, along with
the probability density function:

\begin{Verbatim}[commandchars=\\\{\}]
\PYG{g+gp}{\PYGZgt{}\PYGZgt{}\PYGZgt{} }\PYG{k+kn}{import} \PYG{n+nn}{matplotlib.pyplot} \PYG{k+kn}{as} \PYG{n+nn}{plt}
\PYG{g+gp}{\PYGZgt{}\PYGZgt{}\PYGZgt{} }\PYG{n}{count}\PYG{p}{,} \PYG{n}{bins}\PYG{p}{,} \PYG{n}{ignored} \PYG{o}{=} \PYG{n}{plt}\PYG{o}{.}\PYG{n}{hist}\PYG{p}{(}\PYG{n}{s}\PYG{p}{,} \PYG{l+m+mi}{30}\PYG{p}{,} \PYG{n}{normed}\PYG{o}{=}\PYG{n+nb+bp}{True}\PYG{p}{)}
\PYG{g+gp}{\PYGZgt{}\PYGZgt{}\PYGZgt{} }\PYG{n}{plt}\PYG{o}{.}\PYG{n}{plot}\PYG{p}{(}\PYG{n}{bins}\PYG{p}{,} \PYG{l+m+mi}{1}\PYG{o}{/}\PYG{p}{(}\PYG{n}{sigma} \PYG{o}{*} \PYG{n}{np}\PYG{o}{.}\PYG{n}{sqrt}\PYG{p}{(}\PYG{l+m+mi}{2} \PYG{o}{*} \PYG{n}{np}\PYG{o}{.}\PYG{n}{pi}\PYG{p}{)}\PYG{p}{)} \PYG{o}{*}
\PYG{g+gp}{... }               \PYG{n}{np}\PYG{o}{.}\PYG{n}{exp}\PYG{p}{(} \PYG{o}{\PYGZhy{}} \PYG{p}{(}\PYG{n}{bins} \PYG{o}{\PYGZhy{}} \PYG{n}{mu}\PYG{p}{)}\PYG{o}{*}\PYG{o}{*}\PYG{l+m+mi}{2} \PYG{o}{/} \PYG{p}{(}\PYG{l+m+mi}{2} \PYG{o}{*} \PYG{n}{sigma}\PYG{o}{*}\PYG{o}{*}\PYG{l+m+mi}{2}\PYG{p}{)} \PYG{p}{)}\PYG{p}{,}
\PYG{g+gp}{... }         \PYG{n}{linewidth}\PYG{o}{=}\PYG{l+m+mi}{2}\PYG{p}{,} \PYG{n}{color}\PYG{o}{=}\PYG{l+s}{\PYGZsq{}}\PYG{l+s}{r}\PYG{l+s}{\PYGZsq{}}\PYG{p}{)}
\PYG{g+gp}{\PYGZgt{}\PYGZgt{}\PYGZgt{} }\PYG{n}{plt}\PYG{o}{.}\PYG{n}{show}\PYG{p}{(}\PYG{p}{)}
\end{Verbatim}

\end{fulllineitems}

\index{pareto() (in module lib.dyn.dynamics)}

\begin{fulllineitems}
\phantomsection\label{lib.dyn:lib.dyn.dynamics.pareto}\pysiglinewithargsret{\code{lib.dyn.dynamics.}\bfcode{pareto}}{\emph{a}, \emph{size=None}}{}
Draw samples from a Pareto II or Lomax distribution with specified shape.

The Lomax or Pareto II distribution is a shifted Pareto distribution. The
classical Pareto distribution can be obtained from the Lomax distribution
by adding the location parameter m, see below. The smallest value of the
Lomax distribution is zero while for the classical Pareto distribution it
is m, where the standard Pareto distribution has location m=1.
Lomax can also be considered as a simplified version of the Generalized
Pareto distribution (available in SciPy), with the scale set to one and
the location set to zero.

The Pareto distribution must be greater than zero, and is unbounded above.
It is also known as the ``80-20 rule''.  In this distribution, 80 percent of
the weights are in the lowest 20 percent of the range, while the other 20
percent fill the remaining 80 percent of the range.
\begin{description}
\item[{shape}] \leavevmode{[}float, \textgreater{} 0.{]}
Shape of the distribution.

\item[{size}] \leavevmode{[}tuple of ints{]}
Output shape.  If the given shape is, e.g., \code{(m, n, k)}, then
\code{m * n * k} samples are drawn.

\end{description}
\begin{description}
\item[{scipy.stats.distributions.lomax.pdf}] \leavevmode{[}probability density function,{]}
distribution or cumulative density function, etc.

\item[{scipy.stats.distributions.genpareto.pdf}] \leavevmode{[}probability density function,{]}
distribution or cumulative density function, etc.

\end{description}

The probability density for the Pareto distribution is
\begin{gather}
\begin{split}p(x) = \frac{am^a}{x^{a+1}}\end{split}\notag
\end{gather}
where \(a\) is the shape and \(m\) the location

The Pareto distribution, named after the Italian economist Vilfredo Pareto,
is a power law probability distribution useful in many real world problems.
Outside the field of economics it is generally referred to as the Bradford
distribution. Pareto developed the distribution to describe the
distribution of wealth in an economy.  It has also found use in insurance,
web page access statistics, oil field sizes, and many other problems,
including the download frequency for projects in Sourceforge {[}1{]}.  It is
one of the so-called ``fat-tailed'' distributions.

Draw samples from the distribution:

\begin{Verbatim}[commandchars=\\\{\}]
\PYG{g+gp}{\PYGZgt{}\PYGZgt{}\PYGZgt{} }\PYG{n}{a}\PYG{p}{,} \PYG{n}{m} \PYG{o}{=} \PYG{l+m+mf}{3.}\PYG{p}{,} \PYG{l+m+mf}{1.} \PYG{c}{\PYGZsh{} shape and mode}
\PYG{g+gp}{\PYGZgt{}\PYGZgt{}\PYGZgt{} }\PYG{n}{s} \PYG{o}{=} \PYG{n}{np}\PYG{o}{.}\PYG{n}{random}\PYG{o}{.}\PYG{n}{pareto}\PYG{p}{(}\PYG{n}{a}\PYG{p}{,} \PYG{l+m+mi}{1000}\PYG{p}{)} \PYG{o}{+} \PYG{n}{m}
\end{Verbatim}

Display the histogram of the samples, along with
the probability density function:

\begin{Verbatim}[commandchars=\\\{\}]
\PYG{g+gp}{\PYGZgt{}\PYGZgt{}\PYGZgt{} }\PYG{k+kn}{import} \PYG{n+nn}{matplotlib.pyplot} \PYG{k+kn}{as} \PYG{n+nn}{plt}
\PYG{g+gp}{\PYGZgt{}\PYGZgt{}\PYGZgt{} }\PYG{n}{count}\PYG{p}{,} \PYG{n}{bins}\PYG{p}{,} \PYG{n}{ignored} \PYG{o}{=} \PYG{n}{plt}\PYG{o}{.}\PYG{n}{hist}\PYG{p}{(}\PYG{n}{s}\PYG{p}{,} \PYG{l+m+mi}{100}\PYG{p}{,} \PYG{n}{normed}\PYG{o}{=}\PYG{n+nb+bp}{True}\PYG{p}{,} \PYG{n}{align}\PYG{o}{=}\PYG{l+s}{\PYGZsq{}}\PYG{l+s}{center}\PYG{l+s}{\PYGZsq{}}\PYG{p}{)}
\PYG{g+gp}{\PYGZgt{}\PYGZgt{}\PYGZgt{} }\PYG{n}{fit} \PYG{o}{=} \PYG{n}{a}\PYG{o}{*}\PYG{n}{m}\PYG{o}{*}\PYG{o}{*}\PYG{n}{a}\PYG{o}{/}\PYG{n}{bins}\PYG{o}{*}\PYG{o}{*}\PYG{p}{(}\PYG{n}{a}\PYG{o}{+}\PYG{l+m+mi}{1}\PYG{p}{)}
\PYG{g+gp}{\PYGZgt{}\PYGZgt{}\PYGZgt{} }\PYG{n}{plt}\PYG{o}{.}\PYG{n}{plot}\PYG{p}{(}\PYG{n}{bins}\PYG{p}{,} \PYG{n+nb}{max}\PYG{p}{(}\PYG{n}{count}\PYG{p}{)}\PYG{o}{*}\PYG{n}{fit}\PYG{o}{/}\PYG{n+nb}{max}\PYG{p}{(}\PYG{n}{fit}\PYG{p}{)}\PYG{p}{,}\PYG{n}{linewidth}\PYG{o}{=}\PYG{l+m+mi}{2}\PYG{p}{,} \PYG{n}{color}\PYG{o}{=}\PYG{l+s}{\PYGZsq{}}\PYG{l+s}{r}\PYG{l+s}{\PYGZsq{}}\PYG{p}{)}
\PYG{g+gp}{\PYGZgt{}\PYGZgt{}\PYGZgt{} }\PYG{n}{plt}\PYG{o}{.}\PYG{n}{show}\PYG{p}{(}\PYG{p}{)}
\end{Verbatim}

\end{fulllineitems}

\index{permutation() (in module lib.dyn.dynamics)}

\begin{fulllineitems}
\phantomsection\label{lib.dyn:lib.dyn.dynamics.permutation}\pysiglinewithargsret{\code{lib.dyn.dynamics.}\bfcode{permutation}}{\emph{x}}{}
Randomly permute a sequence, or return a permuted range.

If \emph{x} is a multi-dimensional array, it is only shuffled along its
first index.
\begin{description}
\item[{x}] \leavevmode{[}int or array\_like{]}
If \emph{x} is an integer, randomly permute \code{np.arange(x)}.
If \emph{x} is an array, make a copy and shuffle the elements
randomly.

\end{description}
\begin{description}
\item[{out}] \leavevmode{[}ndarray{]}
Permuted sequence or array range.

\end{description}

\begin{Verbatim}[commandchars=\\\{\}]
\PYG{g+gp}{\PYGZgt{}\PYGZgt{}\PYGZgt{} }\PYG{n}{np}\PYG{o}{.}\PYG{n}{random}\PYG{o}{.}\PYG{n}{permutation}\PYG{p}{(}\PYG{l+m+mi}{10}\PYG{p}{)}
\PYG{g+go}{array([1, 7, 4, 3, 0, 9, 2, 5, 8, 6])}
\end{Verbatim}

\begin{Verbatim}[commandchars=\\\{\}]
\PYG{g+gp}{\PYGZgt{}\PYGZgt{}\PYGZgt{} }\PYG{n}{np}\PYG{o}{.}\PYG{n}{random}\PYG{o}{.}\PYG{n}{permutation}\PYG{p}{(}\PYG{p}{[}\PYG{l+m+mi}{1}\PYG{p}{,} \PYG{l+m+mi}{4}\PYG{p}{,} \PYG{l+m+mi}{9}\PYG{p}{,} \PYG{l+m+mi}{12}\PYG{p}{,} \PYG{l+m+mi}{15}\PYG{p}{]}\PYG{p}{)}
\PYG{g+go}{array([15,  1,  9,  4, 12])}
\end{Verbatim}

\begin{Verbatim}[commandchars=\\\{\}]
\PYG{g+gp}{\PYGZgt{}\PYGZgt{}\PYGZgt{} }\PYG{n}{arr} \PYG{o}{=} \PYG{n}{np}\PYG{o}{.}\PYG{n}{arange}\PYG{p}{(}\PYG{l+m+mi}{9}\PYG{p}{)}\PYG{o}{.}\PYG{n}{reshape}\PYG{p}{(}\PYG{p}{(}\PYG{l+m+mi}{3}\PYG{p}{,} \PYG{l+m+mi}{3}\PYG{p}{)}\PYG{p}{)}
\PYG{g+gp}{\PYGZgt{}\PYGZgt{}\PYGZgt{} }\PYG{n}{np}\PYG{o}{.}\PYG{n}{random}\PYG{o}{.}\PYG{n}{permutation}\PYG{p}{(}\PYG{n}{arr}\PYG{p}{)}
\PYG{g+go}{array([[6, 7, 8],}
\PYG{g+go}{       [0, 1, 2],}
\PYG{g+go}{       [3, 4, 5]])}
\end{Verbatim}

\end{fulllineitems}

\index{poisson() (in module lib.dyn.dynamics)}

\begin{fulllineitems}
\phantomsection\label{lib.dyn:lib.dyn.dynamics.poisson}\pysiglinewithargsret{\code{lib.dyn.dynamics.}\bfcode{poisson}}{\emph{lam=1.0}, \emph{size=None}}{}
Draw samples from a Poisson distribution.

The Poisson distribution is the limit of the Binomial
distribution for large N.
\begin{description}
\item[{lam}] \leavevmode{[}float{]}
Expectation of interval, should be \textgreater{}= 0.

\item[{size}] \leavevmode{[}int or tuple of ints, optional{]}
Output shape. If the given shape is, e.g., \code{(m, n, k)}, then
\code{m * n * k} samples are drawn.

\end{description}

The Poisson distribution
\begin{gather}
\begin{split}f(k; \lambda)=\frac{\lambda^k e^{-\lambda}}{k!}\end{split}\notag
\end{gather}
For events with an expected separation \(\lambda\) the Poisson
distribution \(f(k; \lambda)\) describes the probability of
\(k\) events occurring within the observed interval \(\lambda\).

Because the output is limited to the range of the C long type, a
ValueError is raised when \emph{lam} is within 10 sigma of the maximum
representable value.

Draw samples from the distribution:

\begin{Verbatim}[commandchars=\\\{\}]
\PYG{g+gp}{\PYGZgt{}\PYGZgt{}\PYGZgt{} }\PYG{k+kn}{import} \PYG{n+nn}{numpy} \PYG{k+kn}{as} \PYG{n+nn}{np}
\PYG{g+gp}{\PYGZgt{}\PYGZgt{}\PYGZgt{} }\PYG{n}{s} \PYG{o}{=} \PYG{n}{np}\PYG{o}{.}\PYG{n}{random}\PYG{o}{.}\PYG{n}{poisson}\PYG{p}{(}\PYG{l+m+mi}{5}\PYG{p}{,} \PYG{l+m+mi}{10000}\PYG{p}{)}
\end{Verbatim}

Display histogram of the sample:

\begin{Verbatim}[commandchars=\\\{\}]
\PYG{g+gp}{\PYGZgt{}\PYGZgt{}\PYGZgt{} }\PYG{k+kn}{import} \PYG{n+nn}{matplotlib.pyplot} \PYG{k+kn}{as} \PYG{n+nn}{plt}
\PYG{g+gp}{\PYGZgt{}\PYGZgt{}\PYGZgt{} }\PYG{n}{count}\PYG{p}{,} \PYG{n}{bins}\PYG{p}{,} \PYG{n}{ignored} \PYG{o}{=} \PYG{n}{plt}\PYG{o}{.}\PYG{n}{hist}\PYG{p}{(}\PYG{n}{s}\PYG{p}{,} \PYG{l+m+mi}{14}\PYG{p}{,} \PYG{n}{normed}\PYG{o}{=}\PYG{n+nb+bp}{True}\PYG{p}{)}
\PYG{g+gp}{\PYGZgt{}\PYGZgt{}\PYGZgt{} }\PYG{n}{plt}\PYG{o}{.}\PYG{n}{show}\PYG{p}{(}\PYG{p}{)}
\end{Verbatim}

\end{fulllineitems}

\index{power() (in module lib.dyn.dynamics)}

\begin{fulllineitems}
\phantomsection\label{lib.dyn:lib.dyn.dynamics.power}\pysiglinewithargsret{\code{lib.dyn.dynamics.}\bfcode{power}}{\emph{a}, \emph{size=None}}{}
Draws samples in {[}0, 1{]} from a power distribution with positive
exponent a - 1.

Also known as the power function distribution.
\begin{description}
\item[{a}] \leavevmode{[}float{]}
parameter, \textgreater{} 0

\item[{size}] \leavevmode{[}tuple of ints{]}\begin{description}
\item[{Output shape.  If the given shape is, e.g., \code{(m, n, k)}, then}] \leavevmode
\code{m * n * k} samples are drawn.

\end{description}

\end{description}
\begin{description}
\item[{samples}] \leavevmode{[}\{ndarray, scalar\}{]}
The returned samples lie in {[}0, 1{]}.

\end{description}
\begin{description}
\item[{ValueError}] \leavevmode
If a\textless{}1.

\end{description}

The probability density function is
\begin{gather}
\begin{split}P(x; a) = ax^{a-1}, 0 \le x \le 1, a>0.\end{split}\notag
\end{gather}
The power function distribution is just the inverse of the Pareto
distribution. It may also be seen as a special case of the Beta
distribution.

It is used, for example, in modeling the over-reporting of insurance
claims.

Draw samples from the distribution:

\begin{Verbatim}[commandchars=\\\{\}]
\PYG{g+gp}{\PYGZgt{}\PYGZgt{}\PYGZgt{} }\PYG{n}{a} \PYG{o}{=} \PYG{l+m+mf}{5.} \PYG{c}{\PYGZsh{} shape}
\PYG{g+gp}{\PYGZgt{}\PYGZgt{}\PYGZgt{} }\PYG{n}{samples} \PYG{o}{=} \PYG{l+m+mi}{1000}
\PYG{g+gp}{\PYGZgt{}\PYGZgt{}\PYGZgt{} }\PYG{n}{s} \PYG{o}{=} \PYG{n}{np}\PYG{o}{.}\PYG{n}{random}\PYG{o}{.}\PYG{n}{power}\PYG{p}{(}\PYG{n}{a}\PYG{p}{,} \PYG{n}{samples}\PYG{p}{)}
\end{Verbatim}

Display the histogram of the samples, along with
the probability density function:

\begin{Verbatim}[commandchars=\\\{\}]
\PYG{g+gp}{\PYGZgt{}\PYGZgt{}\PYGZgt{} }\PYG{k+kn}{import} \PYG{n+nn}{matplotlib.pyplot} \PYG{k+kn}{as} \PYG{n+nn}{plt}
\PYG{g+gp}{\PYGZgt{}\PYGZgt{}\PYGZgt{} }\PYG{n}{count}\PYG{p}{,} \PYG{n}{bins}\PYG{p}{,} \PYG{n}{ignored} \PYG{o}{=} \PYG{n}{plt}\PYG{o}{.}\PYG{n}{hist}\PYG{p}{(}\PYG{n}{s}\PYG{p}{,} \PYG{n}{bins}\PYG{o}{=}\PYG{l+m+mi}{30}\PYG{p}{)}
\PYG{g+gp}{\PYGZgt{}\PYGZgt{}\PYGZgt{} }\PYG{n}{x} \PYG{o}{=} \PYG{n}{np}\PYG{o}{.}\PYG{n}{linspace}\PYG{p}{(}\PYG{l+m+mi}{0}\PYG{p}{,} \PYG{l+m+mi}{1}\PYG{p}{,} \PYG{l+m+mi}{100}\PYG{p}{)}
\PYG{g+gp}{\PYGZgt{}\PYGZgt{}\PYGZgt{} }\PYG{n}{y} \PYG{o}{=} \PYG{n}{a}\PYG{o}{*}\PYG{n}{x}\PYG{o}{*}\PYG{o}{*}\PYG{p}{(}\PYG{n}{a}\PYG{o}{\PYGZhy{}}\PYG{l+m+mf}{1.}\PYG{p}{)}
\PYG{g+gp}{\PYGZgt{}\PYGZgt{}\PYGZgt{} }\PYG{n}{normed\PYGZus{}y} \PYG{o}{=} \PYG{n}{samples}\PYG{o}{*}\PYG{n}{np}\PYG{o}{.}\PYG{n}{diff}\PYG{p}{(}\PYG{n}{bins}\PYG{p}{)}\PYG{p}{[}\PYG{l+m+mi}{0}\PYG{p}{]}\PYG{o}{*}\PYG{n}{y}
\PYG{g+gp}{\PYGZgt{}\PYGZgt{}\PYGZgt{} }\PYG{n}{plt}\PYG{o}{.}\PYG{n}{plot}\PYG{p}{(}\PYG{n}{x}\PYG{p}{,} \PYG{n}{normed\PYGZus{}y}\PYG{p}{)}
\PYG{g+gp}{\PYGZgt{}\PYGZgt{}\PYGZgt{} }\PYG{n}{plt}\PYG{o}{.}\PYG{n}{show}\PYG{p}{(}\PYG{p}{)}
\end{Verbatim}

Compare the power function distribution to the inverse of the Pareto.

\begin{Verbatim}[commandchars=\\\{\}]
\PYG{g+gp}{\PYGZgt{}\PYGZgt{}\PYGZgt{} }\PYG{k+kn}{from} \PYG{n+nn}{scipy} \PYG{k+kn}{import} \PYG{n}{stats}
\PYG{g+gp}{\PYGZgt{}\PYGZgt{}\PYGZgt{} }\PYG{n}{rvs} \PYG{o}{=} \PYG{n}{np}\PYG{o}{.}\PYG{n}{random}\PYG{o}{.}\PYG{n}{power}\PYG{p}{(}\PYG{l+m+mi}{5}\PYG{p}{,} \PYG{l+m+mi}{1000000}\PYG{p}{)}
\PYG{g+gp}{\PYGZgt{}\PYGZgt{}\PYGZgt{} }\PYG{n}{rvsp} \PYG{o}{=} \PYG{n}{np}\PYG{o}{.}\PYG{n}{random}\PYG{o}{.}\PYG{n}{pareto}\PYG{p}{(}\PYG{l+m+mi}{5}\PYG{p}{,} \PYG{l+m+mi}{1000000}\PYG{p}{)}
\PYG{g+gp}{\PYGZgt{}\PYGZgt{}\PYGZgt{} }\PYG{n}{xx} \PYG{o}{=} \PYG{n}{np}\PYG{o}{.}\PYG{n}{linspace}\PYG{p}{(}\PYG{l+m+mi}{0}\PYG{p}{,}\PYG{l+m+mi}{1}\PYG{p}{,}\PYG{l+m+mi}{100}\PYG{p}{)}
\PYG{g+gp}{\PYGZgt{}\PYGZgt{}\PYGZgt{} }\PYG{n}{powpdf} \PYG{o}{=} \PYG{n}{stats}\PYG{o}{.}\PYG{n}{powerlaw}\PYG{o}{.}\PYG{n}{pdf}\PYG{p}{(}\PYG{n}{xx}\PYG{p}{,}\PYG{l+m+mi}{5}\PYG{p}{)}
\end{Verbatim}

\begin{Verbatim}[commandchars=\\\{\}]
\PYG{g+gp}{\PYGZgt{}\PYGZgt{}\PYGZgt{} }\PYG{n}{plt}\PYG{o}{.}\PYG{n}{figure}\PYG{p}{(}\PYG{p}{)}
\PYG{g+gp}{\PYGZgt{}\PYGZgt{}\PYGZgt{} }\PYG{n}{plt}\PYG{o}{.}\PYG{n}{hist}\PYG{p}{(}\PYG{n}{rvs}\PYG{p}{,} \PYG{n}{bins}\PYG{o}{=}\PYG{l+m+mi}{50}\PYG{p}{,} \PYG{n}{normed}\PYG{o}{=}\PYG{n+nb+bp}{True}\PYG{p}{)}
\PYG{g+gp}{\PYGZgt{}\PYGZgt{}\PYGZgt{} }\PYG{n}{plt}\PYG{o}{.}\PYG{n}{plot}\PYG{p}{(}\PYG{n}{xx}\PYG{p}{,}\PYG{n}{powpdf}\PYG{p}{,}\PYG{l+s}{\PYGZsq{}}\PYG{l+s}{r\PYGZhy{}}\PYG{l+s}{\PYGZsq{}}\PYG{p}{)}
\PYG{g+gp}{\PYGZgt{}\PYGZgt{}\PYGZgt{} }\PYG{n}{plt}\PYG{o}{.}\PYG{n}{title}\PYG{p}{(}\PYG{l+s}{\PYGZsq{}}\PYG{l+s}{np.random.power(5)}\PYG{l+s}{\PYGZsq{}}\PYG{p}{)}
\end{Verbatim}

\begin{Verbatim}[commandchars=\\\{\}]
\PYG{g+gp}{\PYGZgt{}\PYGZgt{}\PYGZgt{} }\PYG{n}{plt}\PYG{o}{.}\PYG{n}{figure}\PYG{p}{(}\PYG{p}{)}
\PYG{g+gp}{\PYGZgt{}\PYGZgt{}\PYGZgt{} }\PYG{n}{plt}\PYG{o}{.}\PYG{n}{hist}\PYG{p}{(}\PYG{l+m+mf}{1.}\PYG{o}{/}\PYG{p}{(}\PYG{l+m+mf}{1.}\PYG{o}{+}\PYG{n}{rvsp}\PYG{p}{)}\PYG{p}{,} \PYG{n}{bins}\PYG{o}{=}\PYG{l+m+mi}{50}\PYG{p}{,} \PYG{n}{normed}\PYG{o}{=}\PYG{n+nb+bp}{True}\PYG{p}{)}
\PYG{g+gp}{\PYGZgt{}\PYGZgt{}\PYGZgt{} }\PYG{n}{plt}\PYG{o}{.}\PYG{n}{plot}\PYG{p}{(}\PYG{n}{xx}\PYG{p}{,}\PYG{n}{powpdf}\PYG{p}{,}\PYG{l+s}{\PYGZsq{}}\PYG{l+s}{r\PYGZhy{}}\PYG{l+s}{\PYGZsq{}}\PYG{p}{)}
\PYG{g+gp}{\PYGZgt{}\PYGZgt{}\PYGZgt{} }\PYG{n}{plt}\PYG{o}{.}\PYG{n}{title}\PYG{p}{(}\PYG{l+s}{\PYGZsq{}}\PYG{l+s}{inverse of 1 + np.random.pareto(5)}\PYG{l+s}{\PYGZsq{}}\PYG{p}{)}
\end{Verbatim}

\begin{Verbatim}[commandchars=\\\{\}]
\PYG{g+gp}{\PYGZgt{}\PYGZgt{}\PYGZgt{} }\PYG{n}{plt}\PYG{o}{.}\PYG{n}{figure}\PYG{p}{(}\PYG{p}{)}
\PYG{g+gp}{\PYGZgt{}\PYGZgt{}\PYGZgt{} }\PYG{n}{plt}\PYG{o}{.}\PYG{n}{hist}\PYG{p}{(}\PYG{l+m+mf}{1.}\PYG{o}{/}\PYG{p}{(}\PYG{l+m+mf}{1.}\PYG{o}{+}\PYG{n}{rvsp}\PYG{p}{)}\PYG{p}{,} \PYG{n}{bins}\PYG{o}{=}\PYG{l+m+mi}{50}\PYG{p}{,} \PYG{n}{normed}\PYG{o}{=}\PYG{n+nb+bp}{True}\PYG{p}{)}
\PYG{g+gp}{\PYGZgt{}\PYGZgt{}\PYGZgt{} }\PYG{n}{plt}\PYG{o}{.}\PYG{n}{plot}\PYG{p}{(}\PYG{n}{xx}\PYG{p}{,}\PYG{n}{powpdf}\PYG{p}{,}\PYG{l+s}{\PYGZsq{}}\PYG{l+s}{r\PYGZhy{}}\PYG{l+s}{\PYGZsq{}}\PYG{p}{)}
\PYG{g+gp}{\PYGZgt{}\PYGZgt{}\PYGZgt{} }\PYG{n}{plt}\PYG{o}{.}\PYG{n}{title}\PYG{p}{(}\PYG{l+s}{\PYGZsq{}}\PYG{l+s}{inverse of stats.pareto(5)}\PYG{l+s}{\PYGZsq{}}\PYG{p}{)}
\end{Verbatim}

\end{fulllineitems}

\index{rand() (in module lib.dyn.dynamics)}

\begin{fulllineitems}
\phantomsection\label{lib.dyn:lib.dyn.dynamics.rand}\pysiglinewithargsret{\code{lib.dyn.dynamics.}\bfcode{rand}}{\emph{d0}, \emph{d1}, \emph{...}, \emph{dn}}{}
Random values in a given shape.

Create an array of the given shape and propagate it with
random samples from a uniform distribution
over \code{{[}0, 1)}.
\begin{description}
\item[{d0, d1, ..., dn}] \leavevmode{[}int, optional{]}
The dimensions of the returned array, should all be positive.
If no argument is given a single Python float is returned.

\end{description}
\begin{description}
\item[{out}] \leavevmode{[}ndarray, shape \code{(d0, d1, ..., dn)}{]}
Random values.

\end{description}

random

This is a convenience function. If you want an interface that
takes a shape-tuple as the first argument, refer to
np.random.random\_sample .

\begin{Verbatim}[commandchars=\\\{\}]
\PYG{g+gp}{\PYGZgt{}\PYGZgt{}\PYGZgt{} }\PYG{n}{np}\PYG{o}{.}\PYG{n}{random}\PYG{o}{.}\PYG{n}{rand}\PYG{p}{(}\PYG{l+m+mi}{3}\PYG{p}{,}\PYG{l+m+mi}{2}\PYG{p}{)}
\PYG{g+go}{array([[ 0.14022471,  0.96360618],  \PYGZsh{}random}
\PYG{g+go}{       [ 0.37601032,  0.25528411],  \PYGZsh{}random}
\PYG{g+go}{       [ 0.49313049,  0.94909878]]) \PYGZsh{}random}
\end{Verbatim}

\end{fulllineitems}

\index{randint() (in module lib.dyn.dynamics)}

\begin{fulllineitems}
\phantomsection\label{lib.dyn:lib.dyn.dynamics.randint}\pysiglinewithargsret{\code{lib.dyn.dynamics.}\bfcode{randint}}{\emph{low}, \emph{high=None}, \emph{size=None}}{}
Return random integers from \emph{low} (inclusive) to \emph{high} (exclusive).

Return random integers from the ``discrete uniform'' distribution in the
``half-open'' interval {[}\emph{low}, \emph{high}). If \emph{high} is None (the default),
then results are from {[}0, \emph{low}).
\begin{description}
\item[{low}] \leavevmode{[}int{]}
Lowest (signed) integer to be drawn from the distribution (unless
\code{high=None}, in which case this parameter is the \emph{highest} such
integer).

\item[{high}] \leavevmode{[}int, optional{]}
If provided, one above the largest (signed) integer to be drawn
from the distribution (see above for behavior if \code{high=None}).

\item[{size}] \leavevmode{[}int or tuple of ints, optional{]}
Output shape. Default is None, in which case a single int is
returned.

\end{description}
\begin{description}
\item[{out}] \leavevmode{[}int or ndarray of ints{]}
\emph{size}-shaped array of random integers from the appropriate
distribution, or a single such random int if \emph{size} not provided.

\end{description}
\begin{description}
\item[{random.random\_integers}] \leavevmode{[}similar to \emph{randint}, only for the closed{]}
interval {[}\emph{low}, \emph{high}{]}, and 1 is the lowest value if \emph{high} is
omitted. In particular, this other one is the one to use to generate
uniformly distributed discrete non-integers.

\end{description}

\begin{Verbatim}[commandchars=\\\{\}]
\PYG{g+gp}{\PYGZgt{}\PYGZgt{}\PYGZgt{} }\PYG{n}{np}\PYG{o}{.}\PYG{n}{random}\PYG{o}{.}\PYG{n}{randint}\PYG{p}{(}\PYG{l+m+mi}{2}\PYG{p}{,} \PYG{n}{size}\PYG{o}{=}\PYG{l+m+mi}{10}\PYG{p}{)}
\PYG{g+go}{array([1, 0, 0, 0, 1, 1, 0, 0, 1, 0])}
\PYG{g+gp}{\PYGZgt{}\PYGZgt{}\PYGZgt{} }\PYG{n}{np}\PYG{o}{.}\PYG{n}{random}\PYG{o}{.}\PYG{n}{randint}\PYG{p}{(}\PYG{l+m+mi}{1}\PYG{p}{,} \PYG{n}{size}\PYG{o}{=}\PYG{l+m+mi}{10}\PYG{p}{)}
\PYG{g+go}{array([0, 0, 0, 0, 0, 0, 0, 0, 0, 0])}
\end{Verbatim}

Generate a 2 x 4 array of ints between 0 and 4, inclusive:

\begin{Verbatim}[commandchars=\\\{\}]
\PYG{g+gp}{\PYGZgt{}\PYGZgt{}\PYGZgt{} }\PYG{n}{np}\PYG{o}{.}\PYG{n}{random}\PYG{o}{.}\PYG{n}{randint}\PYG{p}{(}\PYG{l+m+mi}{5}\PYG{p}{,} \PYG{n}{size}\PYG{o}{=}\PYG{p}{(}\PYG{l+m+mi}{2}\PYG{p}{,} \PYG{l+m+mi}{4}\PYG{p}{)}\PYG{p}{)}
\PYG{g+go}{array([[4, 0, 2, 1],}
\PYG{g+go}{       [3, 2, 2, 0]])}
\end{Verbatim}

\end{fulllineitems}

\index{randn() (in module lib.dyn.dynamics)}

\begin{fulllineitems}
\phantomsection\label{lib.dyn:lib.dyn.dynamics.randn}\pysiglinewithargsret{\code{lib.dyn.dynamics.}\bfcode{randn}}{\emph{d0}, \emph{d1}, \emph{...}, \emph{dn}}{}
Return a sample (or samples) from the ``standard normal'' distribution.

If positive, int\_like or int-convertible arguments are provided,
\emph{randn} generates an array of shape \code{(d0, d1, ..., dn)}, filled
with random floats sampled from a univariate ``normal'' (Gaussian)
distribution of mean 0 and variance 1 (if any of the \(d_i\) are
floats, they are first converted to integers by truncation). A single
float randomly sampled from the distribution is returned if no
argument is provided.

This is a convenience function.  If you want an interface that takes a
tuple as the first argument, use \emph{numpy.random.standard\_normal} instead.
\begin{description}
\item[{d0, d1, ..., dn}] \leavevmode{[}int, optional{]}
The dimensions of the returned array, should be all positive.
If no argument is given a single Python float is returned.

\end{description}
\begin{description}
\item[{Z}] \leavevmode{[}ndarray or float{]}
A \code{(d0, d1, ..., dn)}-shaped array of floating-point samples from
the standard normal distribution, or a single such float if
no parameters were supplied.

\end{description}

random.standard\_normal : Similar, but takes a tuple as its argument.

For random samples from \(N(\mu, \sigma^2)\), use:

\code{sigma * np.random.randn(...) + mu}

\begin{Verbatim}[commandchars=\\\{\}]
\PYG{g+gp}{\PYGZgt{}\PYGZgt{}\PYGZgt{} }\PYG{n}{np}\PYG{o}{.}\PYG{n}{random}\PYG{o}{.}\PYG{n}{randn}\PYG{p}{(}\PYG{p}{)}
\PYG{g+go}{2.1923875335537315 \PYGZsh{}random}
\end{Verbatim}

Two-by-four array of samples from N(3, 6.25):

\begin{Verbatim}[commandchars=\\\{\}]
\PYG{g+gp}{\PYGZgt{}\PYGZgt{}\PYGZgt{} }\PYG{l+m+mf}{2.5} \PYG{o}{*} \PYG{n}{np}\PYG{o}{.}\PYG{n}{random}\PYG{o}{.}\PYG{n}{randn}\PYG{p}{(}\PYG{l+m+mi}{2}\PYG{p}{,} \PYG{l+m+mi}{4}\PYG{p}{)} \PYG{o}{+} \PYG{l+m+mi}{3}
\PYG{g+go}{array([[\PYGZhy{}4.49401501,  4.00950034, \PYGZhy{}1.81814867,  7.29718677],  \PYGZsh{}random}
\PYG{g+go}{       [ 0.39924804,  4.68456316,  4.99394529,  4.84057254]]) \PYGZsh{}random}
\end{Verbatim}

\end{fulllineitems}

\index{random() (in module lib.dyn.dynamics)}

\begin{fulllineitems}
\phantomsection\label{lib.dyn:lib.dyn.dynamics.random}\pysiglinewithargsret{\code{lib.dyn.dynamics.}\bfcode{random}}{}{}
random\_sample(size=None)

Return random floats in the half-open interval {[}0.0, 1.0).

Results are from the ``continuous uniform'' distribution over the
stated interval.  To sample \(Unif[a, b), b > a\) multiply
the output of \emph{random\_sample} by \emph{(b-a)} and add \emph{a}:

\begin{Verbatim}[commandchars=\\\{\}]
\PYG{p}{(}\PYG{n}{b} \PYG{o}{\PYGZhy{}} \PYG{n}{a}\PYG{p}{)} \PYG{o}{*} \PYG{n}{random\PYGZus{}sample}\PYG{p}{(}\PYG{p}{)} \PYG{o}{+} \PYG{n}{a}
\end{Verbatim}
\begin{description}
\item[{size}] \leavevmode{[}int or tuple of ints, optional{]}
Defines the shape of the returned array of random floats. If None
(the default), returns a single float.

\end{description}
\begin{description}
\item[{out}] \leavevmode{[}float or ndarray of floats{]}
Array of random floats of shape \emph{size} (unless \code{size=None}, in which
case a single float is returned).

\end{description}

\begin{Verbatim}[commandchars=\\\{\}]
\PYG{g+gp}{\PYGZgt{}\PYGZgt{}\PYGZgt{} }\PYG{n}{np}\PYG{o}{.}\PYG{n}{random}\PYG{o}{.}\PYG{n}{random\PYGZus{}sample}\PYG{p}{(}\PYG{p}{)}
\PYG{g+go}{0.47108547995356098}
\PYG{g+gp}{\PYGZgt{}\PYGZgt{}\PYGZgt{} }\PYG{n+nb}{type}\PYG{p}{(}\PYG{n}{np}\PYG{o}{.}\PYG{n}{random}\PYG{o}{.}\PYG{n}{random\PYGZus{}sample}\PYG{p}{(}\PYG{p}{)}\PYG{p}{)}
\PYG{g+go}{\PYGZlt{}type \PYGZsq{}float\PYGZsq{}\PYGZgt{}}
\PYG{g+gp}{\PYGZgt{}\PYGZgt{}\PYGZgt{} }\PYG{n}{np}\PYG{o}{.}\PYG{n}{random}\PYG{o}{.}\PYG{n}{random\PYGZus{}sample}\PYG{p}{(}\PYG{p}{(}\PYG{l+m+mi}{5}\PYG{p}{,}\PYG{p}{)}\PYG{p}{)}
\PYG{g+go}{array([ 0.30220482,  0.86820401,  0.1654503 ,  0.11659149,  0.54323428])}
\end{Verbatim}

Three-by-two array of random numbers from {[}-5, 0):

\begin{Verbatim}[commandchars=\\\{\}]
\PYG{g+gp}{\PYGZgt{}\PYGZgt{}\PYGZgt{} }\PYG{l+m+mi}{5} \PYG{o}{*} \PYG{n}{np}\PYG{o}{.}\PYG{n}{random}\PYG{o}{.}\PYG{n}{random\PYGZus{}sample}\PYG{p}{(}\PYG{p}{(}\PYG{l+m+mi}{3}\PYG{p}{,} \PYG{l+m+mi}{2}\PYG{p}{)}\PYG{p}{)} \PYG{o}{\PYGZhy{}} \PYG{l+m+mi}{5}
\PYG{g+go}{array([[\PYGZhy{}3.99149989, \PYGZhy{}0.52338984],}
\PYG{g+go}{       [\PYGZhy{}2.99091858, \PYGZhy{}0.79479508],}
\PYG{g+go}{       [\PYGZhy{}1.23204345, \PYGZhy{}1.75224494]])}
\end{Verbatim}

\end{fulllineitems}

\index{random\_integers() (in module lib.dyn.dynamics)}

\begin{fulllineitems}
\phantomsection\label{lib.dyn:lib.dyn.dynamics.random_integers}\pysiglinewithargsret{\code{lib.dyn.dynamics.}\bfcode{random\_integers}}{\emph{low}, \emph{high=None}, \emph{size=None}}{}
Return random integers between \emph{low} and \emph{high}, inclusive.

Return random integers from the ``discrete uniform'' distribution in the
closed interval {[}\emph{low}, \emph{high}{]}.  If \emph{high} is None (the default),
then results are from {[}1, \emph{low}{]}.
\begin{description}
\item[{low}] \leavevmode{[}int{]}
Lowest (signed) integer to be drawn from the distribution (unless
\code{high=None}, in which case this parameter is the \emph{highest} such
integer).

\item[{high}] \leavevmode{[}int, optional{]}
If provided, the largest (signed) integer to be drawn from the
distribution (see above for behavior if \code{high=None}).

\item[{size}] \leavevmode{[}int or tuple of ints, optional{]}
Output shape. Default is None, in which case a single int is returned.

\end{description}
\begin{description}
\item[{out}] \leavevmode{[}int or ndarray of ints{]}
\emph{size}-shaped array of random integers from the appropriate
distribution, or a single such random int if \emph{size} not provided.

\end{description}
\begin{description}
\item[{random.randint}] \leavevmode{[}Similar to \emph{random\_integers}, only for the half-open{]}
interval {[}\emph{low}, \emph{high}), and 0 is the lowest value if \emph{high} is
omitted.

\end{description}

To sample from N evenly spaced floating-point numbers between a and b,
use:

\begin{Verbatim}[commandchars=\\\{\}]
\PYG{n}{a} \PYG{o}{+} \PYG{p}{(}\PYG{n}{b} \PYG{o}{\PYGZhy{}} \PYG{n}{a}\PYG{p}{)} \PYG{o}{*} \PYG{p}{(}\PYG{n}{np}\PYG{o}{.}\PYG{n}{random}\PYG{o}{.}\PYG{n}{random\PYGZus{}integers}\PYG{p}{(}\PYG{n}{N}\PYG{p}{)} \PYG{o}{\PYGZhy{}} \PYG{l+m+mi}{1}\PYG{p}{)} \PYG{o}{/} \PYG{p}{(}\PYG{n}{N} \PYG{o}{\PYGZhy{}} \PYG{l+m+mf}{1.}\PYG{p}{)}
\end{Verbatim}

\begin{Verbatim}[commandchars=\\\{\}]
\PYG{g+gp}{\PYGZgt{}\PYGZgt{}\PYGZgt{} }\PYG{n}{np}\PYG{o}{.}\PYG{n}{random}\PYG{o}{.}\PYG{n}{random\PYGZus{}integers}\PYG{p}{(}\PYG{l+m+mi}{5}\PYG{p}{)}
\PYG{g+go}{4}
\PYG{g+gp}{\PYGZgt{}\PYGZgt{}\PYGZgt{} }\PYG{n+nb}{type}\PYG{p}{(}\PYG{n}{np}\PYG{o}{.}\PYG{n}{random}\PYG{o}{.}\PYG{n}{random\PYGZus{}integers}\PYG{p}{(}\PYG{l+m+mi}{5}\PYG{p}{)}\PYG{p}{)}
\PYG{g+go}{\PYGZlt{}type \PYGZsq{}int\PYGZsq{}\PYGZgt{}}
\PYG{g+gp}{\PYGZgt{}\PYGZgt{}\PYGZgt{} }\PYG{n}{np}\PYG{o}{.}\PYG{n}{random}\PYG{o}{.}\PYG{n}{random\PYGZus{}integers}\PYG{p}{(}\PYG{l+m+mi}{5}\PYG{p}{,} \PYG{n}{size}\PYG{o}{=}\PYG{p}{(}\PYG{l+m+mf}{3.}\PYG{p}{,}\PYG{l+m+mf}{2.}\PYG{p}{)}\PYG{p}{)}
\PYG{g+go}{array([[5, 4],}
\PYG{g+go}{       [3, 3],}
\PYG{g+go}{       [4, 5]])}
\end{Verbatim}

Choose five random numbers from the set of five evenly-spaced
numbers between 0 and 2.5, inclusive (\emph{i.e.}, from the set
\({0, 5/8, 10/8, 15/8, 20/8}\)):

\begin{Verbatim}[commandchars=\\\{\}]
\PYG{g+gp}{\PYGZgt{}\PYGZgt{}\PYGZgt{} }\PYG{l+m+mf}{2.5} \PYG{o}{*} \PYG{p}{(}\PYG{n}{np}\PYG{o}{.}\PYG{n}{random}\PYG{o}{.}\PYG{n}{random\PYGZus{}integers}\PYG{p}{(}\PYG{l+m+mi}{5}\PYG{p}{,} \PYG{n}{size}\PYG{o}{=}\PYG{p}{(}\PYG{l+m+mi}{5}\PYG{p}{,}\PYG{p}{)}\PYG{p}{)} \PYG{o}{\PYGZhy{}} \PYG{l+m+mi}{1}\PYG{p}{)} \PYG{o}{/} \PYG{l+m+mf}{4.}
\PYG{g+go}{array([ 0.625,  1.25 ,  0.625,  0.625,  2.5  ])}
\end{Verbatim}

Roll two six sided dice 1000 times and sum the results:

\begin{Verbatim}[commandchars=\\\{\}]
\PYG{g+gp}{\PYGZgt{}\PYGZgt{}\PYGZgt{} }\PYG{n}{d1} \PYG{o}{=} \PYG{n}{np}\PYG{o}{.}\PYG{n}{random}\PYG{o}{.}\PYG{n}{random\PYGZus{}integers}\PYG{p}{(}\PYG{l+m+mi}{1}\PYG{p}{,} \PYG{l+m+mi}{6}\PYG{p}{,} \PYG{l+m+mi}{1000}\PYG{p}{)}
\PYG{g+gp}{\PYGZgt{}\PYGZgt{}\PYGZgt{} }\PYG{n}{d2} \PYG{o}{=} \PYG{n}{np}\PYG{o}{.}\PYG{n}{random}\PYG{o}{.}\PYG{n}{random\PYGZus{}integers}\PYG{p}{(}\PYG{l+m+mi}{1}\PYG{p}{,} \PYG{l+m+mi}{6}\PYG{p}{,} \PYG{l+m+mi}{1000}\PYG{p}{)}
\PYG{g+gp}{\PYGZgt{}\PYGZgt{}\PYGZgt{} }\PYG{n}{dsums} \PYG{o}{=} \PYG{n}{d1} \PYG{o}{+} \PYG{n}{d2}
\end{Verbatim}

Display results as a histogram:

\begin{Verbatim}[commandchars=\\\{\}]
\PYG{g+gp}{\PYGZgt{}\PYGZgt{}\PYGZgt{} }\PYG{k+kn}{import} \PYG{n+nn}{matplotlib.pyplot} \PYG{k+kn}{as} \PYG{n+nn}{plt}
\PYG{g+gp}{\PYGZgt{}\PYGZgt{}\PYGZgt{} }\PYG{n}{count}\PYG{p}{,} \PYG{n}{bins}\PYG{p}{,} \PYG{n}{ignored} \PYG{o}{=} \PYG{n}{plt}\PYG{o}{.}\PYG{n}{hist}\PYG{p}{(}\PYG{n}{dsums}\PYG{p}{,} \PYG{l+m+mi}{11}\PYG{p}{,} \PYG{n}{normed}\PYG{o}{=}\PYG{n+nb+bp}{True}\PYG{p}{)}
\PYG{g+gp}{\PYGZgt{}\PYGZgt{}\PYGZgt{} }\PYG{n}{plt}\PYG{o}{.}\PYG{n}{show}\PYG{p}{(}\PYG{p}{)}
\end{Verbatim}

\end{fulllineitems}

\index{random\_sample() (in module lib.dyn.dynamics)}

\begin{fulllineitems}
\phantomsection\label{lib.dyn:lib.dyn.dynamics.random_sample}\pysiglinewithargsret{\code{lib.dyn.dynamics.}\bfcode{random\_sample}}{\emph{size=None}}{}
Return random floats in the half-open interval {[}0.0, 1.0).

Results are from the ``continuous uniform'' distribution over the
stated interval.  To sample \(Unif[a, b), b > a\) multiply
the output of \emph{random\_sample} by \emph{(b-a)} and add \emph{a}:

\begin{Verbatim}[commandchars=\\\{\}]
\PYG{p}{(}\PYG{n}{b} \PYG{o}{\PYGZhy{}} \PYG{n}{a}\PYG{p}{)} \PYG{o}{*} \PYG{n}{random\PYGZus{}sample}\PYG{p}{(}\PYG{p}{)} \PYG{o}{+} \PYG{n}{a}
\end{Verbatim}
\begin{description}
\item[{size}] \leavevmode{[}int or tuple of ints, optional{]}
Defines the shape of the returned array of random floats. If None
(the default), returns a single float.

\end{description}
\begin{description}
\item[{out}] \leavevmode{[}float or ndarray of floats{]}
Array of random floats of shape \emph{size} (unless \code{size=None}, in which
case a single float is returned).

\end{description}

\begin{Verbatim}[commandchars=\\\{\}]
\PYG{g+gp}{\PYGZgt{}\PYGZgt{}\PYGZgt{} }\PYG{n}{np}\PYG{o}{.}\PYG{n}{random}\PYG{o}{.}\PYG{n}{random\PYGZus{}sample}\PYG{p}{(}\PYG{p}{)}
\PYG{g+go}{0.47108547995356098}
\PYG{g+gp}{\PYGZgt{}\PYGZgt{}\PYGZgt{} }\PYG{n+nb}{type}\PYG{p}{(}\PYG{n}{np}\PYG{o}{.}\PYG{n}{random}\PYG{o}{.}\PYG{n}{random\PYGZus{}sample}\PYG{p}{(}\PYG{p}{)}\PYG{p}{)}
\PYG{g+go}{\PYGZlt{}type \PYGZsq{}float\PYGZsq{}\PYGZgt{}}
\PYG{g+gp}{\PYGZgt{}\PYGZgt{}\PYGZgt{} }\PYG{n}{np}\PYG{o}{.}\PYG{n}{random}\PYG{o}{.}\PYG{n}{random\PYGZus{}sample}\PYG{p}{(}\PYG{p}{(}\PYG{l+m+mi}{5}\PYG{p}{,}\PYG{p}{)}\PYG{p}{)}
\PYG{g+go}{array([ 0.30220482,  0.86820401,  0.1654503 ,  0.11659149,  0.54323428])}
\end{Verbatim}

Three-by-two array of random numbers from {[}-5, 0):

\begin{Verbatim}[commandchars=\\\{\}]
\PYG{g+gp}{\PYGZgt{}\PYGZgt{}\PYGZgt{} }\PYG{l+m+mi}{5} \PYG{o}{*} \PYG{n}{np}\PYG{o}{.}\PYG{n}{random}\PYG{o}{.}\PYG{n}{random\PYGZus{}sample}\PYG{p}{(}\PYG{p}{(}\PYG{l+m+mi}{3}\PYG{p}{,} \PYG{l+m+mi}{2}\PYG{p}{)}\PYG{p}{)} \PYG{o}{\PYGZhy{}} \PYG{l+m+mi}{5}
\PYG{g+go}{array([[\PYGZhy{}3.99149989, \PYGZhy{}0.52338984],}
\PYG{g+go}{       [\PYGZhy{}2.99091858, \PYGZhy{}0.79479508],}
\PYG{g+go}{       [\PYGZhy{}1.23204345, \PYGZhy{}1.75224494]])}
\end{Verbatim}

\end{fulllineitems}

\index{ranf() (in module lib.dyn.dynamics)}

\begin{fulllineitems}
\phantomsection\label{lib.dyn:lib.dyn.dynamics.ranf}\pysiglinewithargsret{\code{lib.dyn.dynamics.}\bfcode{ranf}}{}{}
random\_sample(size=None)

Return random floats in the half-open interval {[}0.0, 1.0).

Results are from the ``continuous uniform'' distribution over the
stated interval.  To sample \(Unif[a, b), b > a\) multiply
the output of \emph{random\_sample} by \emph{(b-a)} and add \emph{a}:

\begin{Verbatim}[commandchars=\\\{\}]
\PYG{p}{(}\PYG{n}{b} \PYG{o}{\PYGZhy{}} \PYG{n}{a}\PYG{p}{)} \PYG{o}{*} \PYG{n}{random\PYGZus{}sample}\PYG{p}{(}\PYG{p}{)} \PYG{o}{+} \PYG{n}{a}
\end{Verbatim}
\begin{description}
\item[{size}] \leavevmode{[}int or tuple of ints, optional{]}
Defines the shape of the returned array of random floats. If None
(the default), returns a single float.

\end{description}
\begin{description}
\item[{out}] \leavevmode{[}float or ndarray of floats{]}
Array of random floats of shape \emph{size} (unless \code{size=None}, in which
case a single float is returned).

\end{description}

\begin{Verbatim}[commandchars=\\\{\}]
\PYG{g+gp}{\PYGZgt{}\PYGZgt{}\PYGZgt{} }\PYG{n}{np}\PYG{o}{.}\PYG{n}{random}\PYG{o}{.}\PYG{n}{random\PYGZus{}sample}\PYG{p}{(}\PYG{p}{)}
\PYG{g+go}{0.47108547995356098}
\PYG{g+gp}{\PYGZgt{}\PYGZgt{}\PYGZgt{} }\PYG{n+nb}{type}\PYG{p}{(}\PYG{n}{np}\PYG{o}{.}\PYG{n}{random}\PYG{o}{.}\PYG{n}{random\PYGZus{}sample}\PYG{p}{(}\PYG{p}{)}\PYG{p}{)}
\PYG{g+go}{\PYGZlt{}type \PYGZsq{}float\PYGZsq{}\PYGZgt{}}
\PYG{g+gp}{\PYGZgt{}\PYGZgt{}\PYGZgt{} }\PYG{n}{np}\PYG{o}{.}\PYG{n}{random}\PYG{o}{.}\PYG{n}{random\PYGZus{}sample}\PYG{p}{(}\PYG{p}{(}\PYG{l+m+mi}{5}\PYG{p}{,}\PYG{p}{)}\PYG{p}{)}
\PYG{g+go}{array([ 0.30220482,  0.86820401,  0.1654503 ,  0.11659149,  0.54323428])}
\end{Verbatim}

Three-by-two array of random numbers from {[}-5, 0):

\begin{Verbatim}[commandchars=\\\{\}]
\PYG{g+gp}{\PYGZgt{}\PYGZgt{}\PYGZgt{} }\PYG{l+m+mi}{5} \PYG{o}{*} \PYG{n}{np}\PYG{o}{.}\PYG{n}{random}\PYG{o}{.}\PYG{n}{random\PYGZus{}sample}\PYG{p}{(}\PYG{p}{(}\PYG{l+m+mi}{3}\PYG{p}{,} \PYG{l+m+mi}{2}\PYG{p}{)}\PYG{p}{)} \PYG{o}{\PYGZhy{}} \PYG{l+m+mi}{5}
\PYG{g+go}{array([[\PYGZhy{}3.99149989, \PYGZhy{}0.52338984],}
\PYG{g+go}{       [\PYGZhy{}2.99091858, \PYGZhy{}0.79479508],}
\PYG{g+go}{       [\PYGZhy{}1.23204345, \PYGZhy{}1.75224494]])}
\end{Verbatim}

\end{fulllineitems}

\index{rangeFloat() (in module lib.dyn.dynamics)}

\begin{fulllineitems}
\phantomsection\label{lib.dyn:lib.dyn.dynamics.rangeFloat}\pysiglinewithargsret{\code{lib.dyn.dynamics.}\bfcode{rangeFloat}}{\emph{start}, \emph{step}, \emph{stop}}{}
\end{fulllineitems}

\index{rayleigh() (in module lib.dyn.dynamics)}

\begin{fulllineitems}
\phantomsection\label{lib.dyn:lib.dyn.dynamics.rayleigh}\pysiglinewithargsret{\code{lib.dyn.dynamics.}\bfcode{rayleigh}}{\emph{scale=1.0}, \emph{size=None}}{}
Draw samples from a Rayleigh distribution.

The \(\chi\) and Weibull distributions are generalizations of the
Rayleigh.
\begin{description}
\item[{scale}] \leavevmode{[}scalar{]}
Scale, also equals the mode. Should be \textgreater{}= 0.

\item[{size}] \leavevmode{[}int or tuple of ints, optional{]}
Shape of the output. Default is None, in which case a single
value is returned.

\end{description}

The probability density function for the Rayleigh distribution is
\begin{gather}
\begin{split}P(x;scale) = \frac{x}{scale^2}e^{\frac{-x^2}{2 \cdotp scale^2}}\end{split}\notag
\end{gather}
The Rayleigh distribution arises if the wind speed and wind direction are
both gaussian variables, then the vector wind velocity forms a Rayleigh
distribution. The Rayleigh distribution is used to model the expected
output from wind turbines.

Draw values from the distribution and plot the histogram

\begin{Verbatim}[commandchars=\\\{\}]
\PYG{g+gp}{\PYGZgt{}\PYGZgt{}\PYGZgt{} }\PYG{n}{values} \PYG{o}{=} \PYG{n}{hist}\PYG{p}{(}\PYG{n}{np}\PYG{o}{.}\PYG{n}{random}\PYG{o}{.}\PYG{n}{rayleigh}\PYG{p}{(}\PYG{l+m+mi}{3}\PYG{p}{,} \PYG{l+m+mi}{100000}\PYG{p}{)}\PYG{p}{,} \PYG{n}{bins}\PYG{o}{=}\PYG{l+m+mi}{200}\PYG{p}{,} \PYG{n}{normed}\PYG{o}{=}\PYG{n+nb+bp}{True}\PYG{p}{)}
\end{Verbatim}

Wave heights tend to follow a Rayleigh distribution. If the mean wave
height is 1 meter, what fraction of waves are likely to be larger than 3
meters?

\begin{Verbatim}[commandchars=\\\{\}]
\PYG{g+gp}{\PYGZgt{}\PYGZgt{}\PYGZgt{} }\PYG{n}{meanvalue} \PYG{o}{=} \PYG{l+m+mi}{1}
\PYG{g+gp}{\PYGZgt{}\PYGZgt{}\PYGZgt{} }\PYG{n}{modevalue} \PYG{o}{=} \PYG{n}{np}\PYG{o}{.}\PYG{n}{sqrt}\PYG{p}{(}\PYG{l+m+mi}{2} \PYG{o}{/} \PYG{n}{np}\PYG{o}{.}\PYG{n}{pi}\PYG{p}{)} \PYG{o}{*} \PYG{n}{meanvalue}
\PYG{g+gp}{\PYGZgt{}\PYGZgt{}\PYGZgt{} }\PYG{n}{s} \PYG{o}{=} \PYG{n}{np}\PYG{o}{.}\PYG{n}{random}\PYG{o}{.}\PYG{n}{rayleigh}\PYG{p}{(}\PYG{n}{modevalue}\PYG{p}{,} \PYG{l+m+mi}{1000000}\PYG{p}{)}
\end{Verbatim}

The percentage of waves larger than 3 meters is:

\begin{Verbatim}[commandchars=\\\{\}]
\PYG{g+gp}{\PYGZgt{}\PYGZgt{}\PYGZgt{} }\PYG{l+m+mf}{100.}\PYG{o}{*}\PYG{n+nb}{sum}\PYG{p}{(}\PYG{n}{s}\PYG{o}{\PYGZgt{}}\PYG{l+m+mi}{3}\PYG{p}{)}\PYG{o}{/}\PYG{l+m+mf}{1000000.}
\PYG{g+go}{0.087300000000000003}
\end{Verbatim}

\end{fulllineitems}

\index{sample() (in module lib.dyn.dynamics)}

\begin{fulllineitems}
\phantomsection\label{lib.dyn:lib.dyn.dynamics.sample}\pysiglinewithargsret{\code{lib.dyn.dynamics.}\bfcode{sample}}{}{}
random\_sample(size=None)

Return random floats in the half-open interval {[}0.0, 1.0).

Results are from the ``continuous uniform'' distribution over the
stated interval.  To sample \(Unif[a, b), b > a\) multiply
the output of \emph{random\_sample} by \emph{(b-a)} and add \emph{a}:

\begin{Verbatim}[commandchars=\\\{\}]
\PYG{p}{(}\PYG{n}{b} \PYG{o}{\PYGZhy{}} \PYG{n}{a}\PYG{p}{)} \PYG{o}{*} \PYG{n}{random\PYGZus{}sample}\PYG{p}{(}\PYG{p}{)} \PYG{o}{+} \PYG{n}{a}
\end{Verbatim}
\begin{description}
\item[{size}] \leavevmode{[}int or tuple of ints, optional{]}
Defines the shape of the returned array of random floats. If None
(the default), returns a single float.

\end{description}
\begin{description}
\item[{out}] \leavevmode{[}float or ndarray of floats{]}
Array of random floats of shape \emph{size} (unless \code{size=None}, in which
case a single float is returned).

\end{description}

\begin{Verbatim}[commandchars=\\\{\}]
\PYG{g+gp}{\PYGZgt{}\PYGZgt{}\PYGZgt{} }\PYG{n}{np}\PYG{o}{.}\PYG{n}{random}\PYG{o}{.}\PYG{n}{random\PYGZus{}sample}\PYG{p}{(}\PYG{p}{)}
\PYG{g+go}{0.47108547995356098}
\PYG{g+gp}{\PYGZgt{}\PYGZgt{}\PYGZgt{} }\PYG{n+nb}{type}\PYG{p}{(}\PYG{n}{np}\PYG{o}{.}\PYG{n}{random}\PYG{o}{.}\PYG{n}{random\PYGZus{}sample}\PYG{p}{(}\PYG{p}{)}\PYG{p}{)}
\PYG{g+go}{\PYGZlt{}type \PYGZsq{}float\PYGZsq{}\PYGZgt{}}
\PYG{g+gp}{\PYGZgt{}\PYGZgt{}\PYGZgt{} }\PYG{n}{np}\PYG{o}{.}\PYG{n}{random}\PYG{o}{.}\PYG{n}{random\PYGZus{}sample}\PYG{p}{(}\PYG{p}{(}\PYG{l+m+mi}{5}\PYG{p}{,}\PYG{p}{)}\PYG{p}{)}
\PYG{g+go}{array([ 0.30220482,  0.86820401,  0.1654503 ,  0.11659149,  0.54323428])}
\end{Verbatim}

Three-by-two array of random numbers from {[}-5, 0):

\begin{Verbatim}[commandchars=\\\{\}]
\PYG{g+gp}{\PYGZgt{}\PYGZgt{}\PYGZgt{} }\PYG{l+m+mi}{5} \PYG{o}{*} \PYG{n}{np}\PYG{o}{.}\PYG{n}{random}\PYG{o}{.}\PYG{n}{random\PYGZus{}sample}\PYG{p}{(}\PYG{p}{(}\PYG{l+m+mi}{3}\PYG{p}{,} \PYG{l+m+mi}{2}\PYG{p}{)}\PYG{p}{)} \PYG{o}{\PYGZhy{}} \PYG{l+m+mi}{5}
\PYG{g+go}{array([[\PYGZhy{}3.99149989, \PYGZhy{}0.52338984],}
\PYG{g+go}{       [\PYGZhy{}2.99091858, \PYGZhy{}0.79479508],}
\PYG{g+go}{       [\PYGZhy{}1.23204345, \PYGZhy{}1.75224494]])}
\end{Verbatim}

\end{fulllineitems}

\index{seed() (in module lib.dyn.dynamics)}

\begin{fulllineitems}
\phantomsection\label{lib.dyn:lib.dyn.dynamics.seed}\pysiglinewithargsret{\code{lib.dyn.dynamics.}\bfcode{seed}}{\emph{seed=None}}{}
Seed the generator.

This method is called when \emph{RandomState} is initialized. It can be
called again to re-seed the generator. For details, see \emph{RandomState}.
\begin{description}
\item[{seed}] \leavevmode{[}int or array\_like, optional{]}
Seed for \emph{RandomState}.

\end{description}

RandomState

\end{fulllineitems}

\index{set\_state() (in module lib.dyn.dynamics)}

\begin{fulllineitems}
\phantomsection\label{lib.dyn:lib.dyn.dynamics.set_state}\pysiglinewithargsret{\code{lib.dyn.dynamics.}\bfcode{set\_state}}{\emph{state}}{}
Set the internal state of the generator from a tuple.

For use if one has reason to manually (re-)set the internal state of the
``Mersenne Twister''{\color{red}\bfseries{}{[}1{]}\_} pseudo-random number generating algorithm.
\begin{description}
\item[{state}] \leavevmode{[}tuple(str, ndarray of 624 uints, int, int, float){]}
The \emph{state} tuple has the following items:
\begin{enumerate}
\item {} 
the string `MT19937', specifying the Mersenne Twister algorithm.

\item {} 
a 1-D array of 624 unsigned integers \code{keys}.

\item {} 
an integer \code{pos}.

\item {} 
an integer \code{has\_gauss}.

\item {} 
a float \code{cached\_gaussian}.

\end{enumerate}

\end{description}
\begin{description}
\item[{out}] \leavevmode{[}None{]}
Returns `None' on success.

\end{description}

get\_state

\emph{set\_state} and \emph{get\_state} are not needed to work with any of the
random distributions in NumPy. If the internal state is manually altered,
the user should know exactly what he/she is doing.

For backwards compatibility, the form (str, array of 624 uints, int) is
also accepted although it is missing some information about the cached
Gaussian value: \code{state = ('MT19937', keys, pos)}.

\end{fulllineitems}

\index{shuffle() (in module lib.dyn.dynamics)}

\begin{fulllineitems}
\phantomsection\label{lib.dyn:lib.dyn.dynamics.shuffle}\pysiglinewithargsret{\code{lib.dyn.dynamics.}\bfcode{shuffle}}{\emph{x}}{}
Modify a sequence in-place by shuffling its contents.
\begin{description}
\item[{x}] \leavevmode{[}array\_like{]}
The array or list to be shuffled.

\end{description}

None

\begin{Verbatim}[commandchars=\\\{\}]
\PYG{g+gp}{\PYGZgt{}\PYGZgt{}\PYGZgt{} }\PYG{n}{arr} \PYG{o}{=} \PYG{n}{np}\PYG{o}{.}\PYG{n}{arange}\PYG{p}{(}\PYG{l+m+mi}{10}\PYG{p}{)}
\PYG{g+gp}{\PYGZgt{}\PYGZgt{}\PYGZgt{} }\PYG{n}{np}\PYG{o}{.}\PYG{n}{random}\PYG{o}{.}\PYG{n}{shuffle}\PYG{p}{(}\PYG{n}{arr}\PYG{p}{)}
\PYG{g+gp}{\PYGZgt{}\PYGZgt{}\PYGZgt{} }\PYG{n}{arr}
\PYG{g+go}{[1 7 5 2 9 4 3 6 0 8]}
\end{Verbatim}

This function only shuffles the array along the first index of a
multi-dimensional array:

\begin{Verbatim}[commandchars=\\\{\}]
\PYG{g+gp}{\PYGZgt{}\PYGZgt{}\PYGZgt{} }\PYG{n}{arr} \PYG{o}{=} \PYG{n}{np}\PYG{o}{.}\PYG{n}{arange}\PYG{p}{(}\PYG{l+m+mi}{9}\PYG{p}{)}\PYG{o}{.}\PYG{n}{reshape}\PYG{p}{(}\PYG{p}{(}\PYG{l+m+mi}{3}\PYG{p}{,} \PYG{l+m+mi}{3}\PYG{p}{)}\PYG{p}{)}
\PYG{g+gp}{\PYGZgt{}\PYGZgt{}\PYGZgt{} }\PYG{n}{np}\PYG{o}{.}\PYG{n}{random}\PYG{o}{.}\PYG{n}{shuffle}\PYG{p}{(}\PYG{n}{arr}\PYG{p}{)}
\PYG{g+gp}{\PYGZgt{}\PYGZgt{}\PYGZgt{} }\PYG{n}{arr}
\PYG{g+go}{array([[3, 4, 5],}
\PYG{g+go}{       [6, 7, 8],}
\PYG{g+go}{       [0, 1, 2]])}
\end{Verbatim}

\end{fulllineitems}

\index{standard\_cauchy() (in module lib.dyn.dynamics)}

\begin{fulllineitems}
\phantomsection\label{lib.dyn:lib.dyn.dynamics.standard_cauchy}\pysiglinewithargsret{\code{lib.dyn.dynamics.}\bfcode{standard\_cauchy}}{\emph{size=None}}{}
Standard Cauchy distribution with mode = 0.

Also known as the Lorentz distribution.
\begin{description}
\item[{size}] \leavevmode{[}int or tuple of ints{]}
Shape of the output.

\end{description}
\begin{description}
\item[{samples}] \leavevmode{[}ndarray or scalar{]}
The drawn samples.

\end{description}

The probability density function for the full Cauchy distribution is
\begin{gather}
\begin{split}P(x; x_0, \gamma) = \frac{1}{\pi \gamma \bigl[ 1+
(\frac{x-x_0}{\gamma})^2 \bigr] }\end{split}\notag
\end{gather}
and the Standard Cauchy distribution just sets \(x_0=0\) and
\(\gamma=1\)

The Cauchy distribution arises in the solution to the driven harmonic
oscillator problem, and also describes spectral line broadening. It
also describes the distribution of values at which a line tilted at
a random angle will cut the x axis.

When studying hypothesis tests that assume normality, seeing how the
tests perform on data from a Cauchy distribution is a good indicator of
their sensitivity to a heavy-tailed distribution, since the Cauchy looks
very much like a Gaussian distribution, but with heavier tails.

Draw samples and plot the distribution:

\begin{Verbatim}[commandchars=\\\{\}]
\PYG{g+gp}{\PYGZgt{}\PYGZgt{}\PYGZgt{} }\PYG{n}{s} \PYG{o}{=} \PYG{n}{np}\PYG{o}{.}\PYG{n}{random}\PYG{o}{.}\PYG{n}{standard\PYGZus{}cauchy}\PYG{p}{(}\PYG{l+m+mi}{1000000}\PYG{p}{)}
\PYG{g+gp}{\PYGZgt{}\PYGZgt{}\PYGZgt{} }\PYG{n}{s} \PYG{o}{=} \PYG{n}{s}\PYG{p}{[}\PYG{p}{(}\PYG{n}{s}\PYG{o}{\PYGZgt{}}\PYG{o}{\PYGZhy{}}\PYG{l+m+mi}{25}\PYG{p}{)} \PYG{o}{\PYGZam{}} \PYG{p}{(}\PYG{n}{s}\PYG{o}{\PYGZlt{}}\PYG{l+m+mi}{25}\PYG{p}{)}\PYG{p}{]}  \PYG{c}{\PYGZsh{} truncate distribution so it plots well}
\PYG{g+gp}{\PYGZgt{}\PYGZgt{}\PYGZgt{} }\PYG{n}{plt}\PYG{o}{.}\PYG{n}{hist}\PYG{p}{(}\PYG{n}{s}\PYG{p}{,} \PYG{n}{bins}\PYG{o}{=}\PYG{l+m+mi}{100}\PYG{p}{)}
\PYG{g+gp}{\PYGZgt{}\PYGZgt{}\PYGZgt{} }\PYG{n}{plt}\PYG{o}{.}\PYG{n}{show}\PYG{p}{(}\PYG{p}{)}
\end{Verbatim}

\end{fulllineitems}

\index{standard\_exponential() (in module lib.dyn.dynamics)}

\begin{fulllineitems}
\phantomsection\label{lib.dyn:lib.dyn.dynamics.standard_exponential}\pysiglinewithargsret{\code{lib.dyn.dynamics.}\bfcode{standard\_exponential}}{\emph{size=None}}{}
Draw samples from the standard exponential distribution.

\emph{standard\_exponential} is identical to the exponential distribution
with a scale parameter of 1.
\begin{description}
\item[{size}] \leavevmode{[}int or tuple of ints{]}
Shape of the output.

\end{description}
\begin{description}
\item[{out}] \leavevmode{[}float or ndarray{]}
Drawn samples.

\end{description}

Output a 3x8000 array:

\begin{Verbatim}[commandchars=\\\{\}]
\PYG{g+gp}{\PYGZgt{}\PYGZgt{}\PYGZgt{} }\PYG{n}{n} \PYG{o}{=} \PYG{n}{np}\PYG{o}{.}\PYG{n}{random}\PYG{o}{.}\PYG{n}{standard\PYGZus{}exponential}\PYG{p}{(}\PYG{p}{(}\PYG{l+m+mi}{3}\PYG{p}{,} \PYG{l+m+mi}{8000}\PYG{p}{)}\PYG{p}{)}
\end{Verbatim}

\end{fulllineitems}

\index{standard\_gamma() (in module lib.dyn.dynamics)}

\begin{fulllineitems}
\phantomsection\label{lib.dyn:lib.dyn.dynamics.standard_gamma}\pysiglinewithargsret{\code{lib.dyn.dynamics.}\bfcode{standard\_gamma}}{\emph{shape}, \emph{size=None}}{}
Draw samples from a Standard Gamma distribution.

Samples are drawn from a Gamma distribution with specified parameters,
shape (sometimes designated ``k'') and scale=1.
\begin{description}
\item[{shape}] \leavevmode{[}float{]}
Parameter, should be \textgreater{} 0.

\item[{size}] \leavevmode{[}int or tuple of ints{]}
Output shape.  If the given shape is, e.g., \code{(m, n, k)}, then
\code{m * n * k} samples are drawn.

\end{description}
\begin{description}
\item[{samples}] \leavevmode{[}ndarray or scalar{]}
The drawn samples.

\end{description}
\begin{description}
\item[{scipy.stats.distributions.gamma}] \leavevmode{[}probability density function,{]}
distribution or cumulative density function, etc.

\end{description}

The probability density for the Gamma distribution is
\begin{gather}
\begin{split}p(x) = x^{k-1}\frac{e^{-x/\theta}}{\theta^k\Gamma(k)},\end{split}\notag
\end{gather}
where \(k\) is the shape and \(\theta\) the scale,
and \(\Gamma\) is the Gamma function.

The Gamma distribution is often used to model the times to failure of
electronic components, and arises naturally in processes for which the
waiting times between Poisson distributed events are relevant.

Draw samples from the distribution:

\begin{Verbatim}[commandchars=\\\{\}]
\PYG{g+gp}{\PYGZgt{}\PYGZgt{}\PYGZgt{} }\PYG{n}{shape}\PYG{p}{,} \PYG{n}{scale} \PYG{o}{=} \PYG{l+m+mf}{2.}\PYG{p}{,} \PYG{l+m+mf}{1.} \PYG{c}{\PYGZsh{} mean and width}
\PYG{g+gp}{\PYGZgt{}\PYGZgt{}\PYGZgt{} }\PYG{n}{s} \PYG{o}{=} \PYG{n}{np}\PYG{o}{.}\PYG{n}{random}\PYG{o}{.}\PYG{n}{standard\PYGZus{}gamma}\PYG{p}{(}\PYG{n}{shape}\PYG{p}{,} \PYG{l+m+mi}{1000000}\PYG{p}{)}
\end{Verbatim}

Display the histogram of the samples, along with
the probability density function:

\begin{Verbatim}[commandchars=\\\{\}]
\PYG{g+gp}{\PYGZgt{}\PYGZgt{}\PYGZgt{} }\PYG{k+kn}{import} \PYG{n+nn}{matplotlib.pyplot} \PYG{k+kn}{as} \PYG{n+nn}{plt}
\PYG{g+gp}{\PYGZgt{}\PYGZgt{}\PYGZgt{} }\PYG{k+kn}{import} \PYG{n+nn}{scipy.special} \PYG{k+kn}{as} \PYG{n+nn}{sps}
\PYG{g+gp}{\PYGZgt{}\PYGZgt{}\PYGZgt{} }\PYG{n}{count}\PYG{p}{,} \PYG{n}{bins}\PYG{p}{,} \PYG{n}{ignored} \PYG{o}{=} \PYG{n}{plt}\PYG{o}{.}\PYG{n}{hist}\PYG{p}{(}\PYG{n}{s}\PYG{p}{,} \PYG{l+m+mi}{50}\PYG{p}{,} \PYG{n}{normed}\PYG{o}{=}\PYG{n+nb+bp}{True}\PYG{p}{)}
\PYG{g+gp}{\PYGZgt{}\PYGZgt{}\PYGZgt{} }\PYG{n}{y} \PYG{o}{=} \PYG{n}{bins}\PYG{o}{*}\PYG{o}{*}\PYG{p}{(}\PYG{n}{shape}\PYG{o}{\PYGZhy{}}\PYG{l+m+mi}{1}\PYG{p}{)} \PYG{o}{*} \PYG{p}{(}\PYG{p}{(}\PYG{n}{np}\PYG{o}{.}\PYG{n}{exp}\PYG{p}{(}\PYG{o}{\PYGZhy{}}\PYG{n}{bins}\PYG{o}{/}\PYG{n}{scale}\PYG{p}{)}\PYG{p}{)}\PYG{o}{/} \PYGZbs{}
\PYG{g+gp}{... }                      \PYG{p}{(}\PYG{n}{sps}\PYG{o}{.}\PYG{n}{gamma}\PYG{p}{(}\PYG{n}{shape}\PYG{p}{)} \PYG{o}{*} \PYG{n}{scale}\PYG{o}{*}\PYG{o}{*}\PYG{n}{shape}\PYG{p}{)}\PYG{p}{)}
\PYG{g+gp}{\PYGZgt{}\PYGZgt{}\PYGZgt{} }\PYG{n}{plt}\PYG{o}{.}\PYG{n}{plot}\PYG{p}{(}\PYG{n}{bins}\PYG{p}{,} \PYG{n}{y}\PYG{p}{,} \PYG{n}{linewidth}\PYG{o}{=}\PYG{l+m+mi}{2}\PYG{p}{,} \PYG{n}{color}\PYG{o}{=}\PYG{l+s}{\PYGZsq{}}\PYG{l+s}{r}\PYG{l+s}{\PYGZsq{}}\PYG{p}{)}
\PYG{g+gp}{\PYGZgt{}\PYGZgt{}\PYGZgt{} }\PYG{n}{plt}\PYG{o}{.}\PYG{n}{show}\PYG{p}{(}\PYG{p}{)}
\end{Verbatim}

\end{fulllineitems}

\index{standard\_normal() (in module lib.dyn.dynamics)}

\begin{fulllineitems}
\phantomsection\label{lib.dyn:lib.dyn.dynamics.standard_normal}\pysiglinewithargsret{\code{lib.dyn.dynamics.}\bfcode{standard\_normal}}{\emph{size=None}}{}
Returns samples from a Standard Normal distribution (mean=0, stdev=1).
\begin{description}
\item[{size}] \leavevmode{[}int or tuple of ints, optional{]}
Output shape. Default is None, in which case a single value is
returned.

\end{description}
\begin{description}
\item[{out}] \leavevmode{[}float or ndarray{]}
Drawn samples.

\end{description}

\begin{Verbatim}[commandchars=\\\{\}]
\PYG{g+gp}{\PYGZgt{}\PYGZgt{}\PYGZgt{} }\PYG{n}{s} \PYG{o}{=} \PYG{n}{np}\PYG{o}{.}\PYG{n}{random}\PYG{o}{.}\PYG{n}{standard\PYGZus{}normal}\PYG{p}{(}\PYG{l+m+mi}{8000}\PYG{p}{)}
\PYG{g+gp}{\PYGZgt{}\PYGZgt{}\PYGZgt{} }\PYG{n}{s}
\PYG{g+go}{array([ 0.6888893 ,  0.78096262, \PYGZhy{}0.89086505, ...,  0.49876311, \PYGZsh{}random}
\PYG{g+go}{       \PYGZhy{}0.38672696, \PYGZhy{}0.4685006 ])                               \PYGZsh{}random}
\PYG{g+gp}{\PYGZgt{}\PYGZgt{}\PYGZgt{} }\PYG{n}{s}\PYG{o}{.}\PYG{n}{shape}
\PYG{g+go}{(8000,)}
\PYG{g+gp}{\PYGZgt{}\PYGZgt{}\PYGZgt{} }\PYG{n}{s} \PYG{o}{=} \PYG{n}{np}\PYG{o}{.}\PYG{n}{random}\PYG{o}{.}\PYG{n}{standard\PYGZus{}normal}\PYG{p}{(}\PYG{n}{size}\PYG{o}{=}\PYG{p}{(}\PYG{l+m+mi}{3}\PYG{p}{,} \PYG{l+m+mi}{4}\PYG{p}{,} \PYG{l+m+mi}{2}\PYG{p}{)}\PYG{p}{)}
\PYG{g+gp}{\PYGZgt{}\PYGZgt{}\PYGZgt{} }\PYG{n}{s}\PYG{o}{.}\PYG{n}{shape}
\PYG{g+go}{(3, 4, 2)}
\end{Verbatim}

\end{fulllineitems}

\index{standard\_t() (in module lib.dyn.dynamics)}

\begin{fulllineitems}
\phantomsection\label{lib.dyn:lib.dyn.dynamics.standard_t}\pysiglinewithargsret{\code{lib.dyn.dynamics.}\bfcode{standard\_t}}{\emph{df}, \emph{size=None}}{}
Standard Student's t distribution with df degrees of freedom.

A special case of the hyperbolic distribution.
As \emph{df} gets large, the result resembles that of the standard normal
distribution (\emph{standard\_normal}).
\begin{description}
\item[{df}] \leavevmode{[}int{]}
Degrees of freedom, should be \textgreater{} 0.

\item[{size}] \leavevmode{[}int or tuple of ints, optional{]}
Output shape. Default is None, in which case a single value is
returned.

\end{description}
\begin{description}
\item[{samples}] \leavevmode{[}ndarray or scalar{]}
Drawn samples.

\end{description}

The probability density function for the t distribution is
\begin{gather}
\begin{split}P(x, df) = \frac{\Gamma(\frac{df+1}{2})}{\sqrt{\pi df}
\Gamma(\frac{df}{2})}\Bigl( 1+\frac{x^2}{df} \Bigr)^{-(df+1)/2}\end{split}\notag
\end{gather}
The t test is based on an assumption that the data come from a Normal
distribution. The t test provides a way to test whether the sample mean
(that is the mean calculated from the data) is a good estimate of the true
mean.

The derivation of the t-distribution was forst published in 1908 by William
Gisset while working for the Guinness Brewery in Dublin. Due to proprietary
issues, he had to publish under a pseudonym, and so he used the name
Student.

From Dalgaard page 83 {\color{red}\bfseries{}{[}1{]}\_}, suppose the daily energy intake for 11
women in Kj is:

\begin{Verbatim}[commandchars=\\\{\}]
\PYG{g+gp}{\PYGZgt{}\PYGZgt{}\PYGZgt{} }\PYG{n}{intake} \PYG{o}{=} \PYG{n}{np}\PYG{o}{.}\PYG{n}{array}\PYG{p}{(}\PYG{p}{[}\PYG{l+m+mf}{5260.}\PYG{p}{,} \PYG{l+m+mi}{5470}\PYG{p}{,} \PYG{l+m+mi}{5640}\PYG{p}{,} \PYG{l+m+mi}{6180}\PYG{p}{,} \PYG{l+m+mi}{6390}\PYG{p}{,} \PYG{l+m+mi}{6515}\PYG{p}{,} \PYG{l+m+mi}{6805}\PYG{p}{,} \PYG{l+m+mi}{7515}\PYG{p}{,} \PYGZbs{}
\PYG{g+gp}{... }                   \PYG{l+m+mi}{7515}\PYG{p}{,} \PYG{l+m+mi}{8230}\PYG{p}{,} \PYG{l+m+mi}{8770}\PYG{p}{]}\PYG{p}{)}
\end{Verbatim}

Does their energy intake deviate systematically from the recommended
value of 7725 kJ?

We have 10 degrees of freedom, so is the sample mean within 95\% of the
recommended value?

\begin{Verbatim}[commandchars=\\\{\}]
\PYG{g+gp}{\PYGZgt{}\PYGZgt{}\PYGZgt{} }\PYG{n}{s} \PYG{o}{=} \PYG{n}{np}\PYG{o}{.}\PYG{n}{random}\PYG{o}{.}\PYG{n}{standard\PYGZus{}t}\PYG{p}{(}\PYG{l+m+mi}{10}\PYG{p}{,} \PYG{n}{size}\PYG{o}{=}\PYG{l+m+mi}{100000}\PYG{p}{)}
\PYG{g+gp}{\PYGZgt{}\PYGZgt{}\PYGZgt{} }\PYG{n}{np}\PYG{o}{.}\PYG{n}{mean}\PYG{p}{(}\PYG{n}{intake}\PYG{p}{)}
\PYG{g+go}{6753.636363636364}
\PYG{g+gp}{\PYGZgt{}\PYGZgt{}\PYGZgt{} }\PYG{n}{intake}\PYG{o}{.}\PYG{n}{std}\PYG{p}{(}\PYG{n}{ddof}\PYG{o}{=}\PYG{l+m+mi}{1}\PYG{p}{)}
\PYG{g+go}{1142.1232221373727}
\end{Verbatim}

Calculate the t statistic, setting the ddof parameter to the unbiased
value so the divisor in the standard deviation will be degrees of
freedom, N-1.

\begin{Verbatim}[commandchars=\\\{\}]
\PYG{g+gp}{\PYGZgt{}\PYGZgt{}\PYGZgt{} }\PYG{n}{t} \PYG{o}{=} \PYG{p}{(}\PYG{n}{np}\PYG{o}{.}\PYG{n}{mean}\PYG{p}{(}\PYG{n}{intake}\PYG{p}{)}\PYG{o}{\PYGZhy{}}\PYG{l+m+mi}{7725}\PYG{p}{)}\PYG{o}{/}\PYG{p}{(}\PYG{n}{intake}\PYG{o}{.}\PYG{n}{std}\PYG{p}{(}\PYG{n}{ddof}\PYG{o}{=}\PYG{l+m+mi}{1}\PYG{p}{)}\PYG{o}{/}\PYG{n}{np}\PYG{o}{.}\PYG{n}{sqrt}\PYG{p}{(}\PYG{n+nb}{len}\PYG{p}{(}\PYG{n}{intake}\PYG{p}{)}\PYG{p}{)}\PYG{p}{)}
\PYG{g+gp}{\PYGZgt{}\PYGZgt{}\PYGZgt{} }\PYG{k+kn}{import} \PYG{n+nn}{matplotlib.pyplot} \PYG{k+kn}{as} \PYG{n+nn}{plt}
\PYG{g+gp}{\PYGZgt{}\PYGZgt{}\PYGZgt{} }\PYG{n}{h} \PYG{o}{=} \PYG{n}{plt}\PYG{o}{.}\PYG{n}{hist}\PYG{p}{(}\PYG{n}{s}\PYG{p}{,} \PYG{n}{bins}\PYG{o}{=}\PYG{l+m+mi}{100}\PYG{p}{,} \PYG{n}{normed}\PYG{o}{=}\PYG{n+nb+bp}{True}\PYG{p}{)}
\end{Verbatim}

For a one-sided t-test, how far out in the distribution does the t
statistic appear?

\begin{Verbatim}[commandchars=\\\{\}]
\PYG{g+gp}{\PYGZgt{}\PYGZgt{}\PYGZgt{} }\PYG{o}{\PYGZgt{}\PYGZgt{}}\PYG{o}{\PYGZgt{}} \PYG{n}{np}\PYG{o}{.}\PYG{n}{sum}\PYG{p}{(}\PYG{n}{s}\PYG{o}{\PYGZlt{}}\PYG{n}{t}\PYG{p}{)} \PYG{o}{/} \PYG{n+nb}{float}\PYG{p}{(}\PYG{n+nb}{len}\PYG{p}{(}\PYG{n}{s}\PYG{p}{)}\PYG{p}{)}
\PYG{g+go}{0.0090699999999999999  \PYGZsh{}random}
\end{Verbatim}

So the p-value is about 0.009, which says the null hypothesis has a
probability of about 99\% of being true.

\end{fulllineitems}

\index{triangular() (in module lib.dyn.dynamics)}

\begin{fulllineitems}
\phantomsection\label{lib.dyn:lib.dyn.dynamics.triangular}\pysiglinewithargsret{\code{lib.dyn.dynamics.}\bfcode{triangular}}{\emph{left}, \emph{mode}, \emph{right}, \emph{size=None}}{}
Draw samples from the triangular distribution.

The triangular distribution is a continuous probability distribution with
lower limit left, peak at mode, and upper limit right. Unlike the other
distributions, these parameters directly define the shape of the pdf.
\begin{description}
\item[{left}] \leavevmode{[}scalar{]}
Lower limit.

\item[{mode}] \leavevmode{[}scalar{]}
The value where the peak of the distribution occurs.
The value should fulfill the condition \code{left \textless{}= mode \textless{}= right}.

\item[{right}] \leavevmode{[}scalar{]}
Upper limit, should be larger than \emph{left}.

\item[{size}] \leavevmode{[}int or tuple of ints, optional{]}
Output shape. Default is None, in which case a single value is
returned.

\end{description}
\begin{description}
\item[{samples}] \leavevmode{[}ndarray or scalar{]}
The returned samples all lie in the interval {[}left, right{]}.

\end{description}

The probability density function for the Triangular distribution is
\begin{gather}
\begin{split}P(x;l, m, r) = \begin{cases}
\frac{2(x-l)}{(r-l)(m-l)}& \text{for $l \leq x \leq m$},\\
\frac{2(m-x)}{(r-l)(r-m)}& \text{for $m \leq x \leq r$},\\
0& \text{otherwise}.
\end{cases}\end{split}\notag
\end{gather}
The triangular distribution is often used in ill-defined problems where the
underlying distribution is not known, but some knowledge of the limits and
mode exists. Often it is used in simulations.

Draw values from the distribution and plot the histogram:

\begin{Verbatim}[commandchars=\\\{\}]
\PYG{g+gp}{\PYGZgt{}\PYGZgt{}\PYGZgt{} }\PYG{k+kn}{import} \PYG{n+nn}{matplotlib.pyplot} \PYG{k+kn}{as} \PYG{n+nn}{plt}
\PYG{g+gp}{\PYGZgt{}\PYGZgt{}\PYGZgt{} }\PYG{n}{h} \PYG{o}{=} \PYG{n}{plt}\PYG{o}{.}\PYG{n}{hist}\PYG{p}{(}\PYG{n}{np}\PYG{o}{.}\PYG{n}{random}\PYG{o}{.}\PYG{n}{triangular}\PYG{p}{(}\PYG{o}{\PYGZhy{}}\PYG{l+m+mi}{3}\PYG{p}{,} \PYG{l+m+mi}{0}\PYG{p}{,} \PYG{l+m+mi}{8}\PYG{p}{,} \PYG{l+m+mi}{100000}\PYG{p}{)}\PYG{p}{,} \PYG{n}{bins}\PYG{o}{=}\PYG{l+m+mi}{200}\PYG{p}{,}
\PYG{g+gp}{... }             \PYG{n}{normed}\PYG{o}{=}\PYG{n+nb+bp}{True}\PYG{p}{)}
\PYG{g+gp}{\PYGZgt{}\PYGZgt{}\PYGZgt{} }\PYG{n}{plt}\PYG{o}{.}\PYG{n}{show}\PYG{p}{(}\PYG{p}{)}
\end{Verbatim}

\end{fulllineitems}

\index{uniform() (in module lib.dyn.dynamics)}

\begin{fulllineitems}
\phantomsection\label{lib.dyn:lib.dyn.dynamics.uniform}\pysiglinewithargsret{\code{lib.dyn.dynamics.}\bfcode{uniform}}{\emph{low=0.0}, \emph{high=1.0}, \emph{size=1}}{}
Draw samples from a uniform distribution.

Samples are uniformly distributed over the half-open interval
\code{{[}low, high)} (includes low, but excludes high).  In other words,
any value within the given interval is equally likely to be drawn
by \emph{uniform}.
\begin{description}
\item[{low}] \leavevmode{[}float, optional{]}
Lower boundary of the output interval.  All values generated will be
greater than or equal to low.  The default value is 0.

\item[{high}] \leavevmode{[}float{]}
Upper boundary of the output interval.  All values generated will be
less than high.  The default value is 1.0.

\item[{size}] \leavevmode{[}int or tuple of ints, optional{]}
Shape of output.  If the given size is, for example, (m,n,k),
m*n*k samples are generated.  If no shape is specified, a single sample
is returned.

\end{description}
\begin{description}
\item[{out}] \leavevmode{[}ndarray{]}
Drawn samples, with shape \emph{size}.

\end{description}

randint : Discrete uniform distribution, yielding integers.
random\_integers : Discrete uniform distribution over the closed
\begin{quote}

interval \code{{[}low, high{]}}.
\end{quote}

random\_sample : Floats uniformly distributed over \code{{[}0, 1)}.
random : Alias for \emph{random\_sample}.
rand : Convenience function that accepts dimensions as input, e.g.,
\begin{quote}

\code{rand(2,2)} would generate a 2-by-2 array of floats,
uniformly distributed over \code{{[}0, 1)}.
\end{quote}

The probability density function of the uniform distribution is
\begin{gather}
\begin{split}p(x) = \frac{1}{b - a}\end{split}\notag
\end{gather}
anywhere within the interval \code{{[}a, b)}, and zero elsewhere.

Draw samples from the distribution:

\begin{Verbatim}[commandchars=\\\{\}]
\PYG{g+gp}{\PYGZgt{}\PYGZgt{}\PYGZgt{} }\PYG{n}{s} \PYG{o}{=} \PYG{n}{np}\PYG{o}{.}\PYG{n}{random}\PYG{o}{.}\PYG{n}{uniform}\PYG{p}{(}\PYG{o}{\PYGZhy{}}\PYG{l+m+mi}{1}\PYG{p}{,}\PYG{l+m+mi}{0}\PYG{p}{,}\PYG{l+m+mi}{1000}\PYG{p}{)}
\end{Verbatim}

All values are within the given interval:

\begin{Verbatim}[commandchars=\\\{\}]
\PYG{g+gp}{\PYGZgt{}\PYGZgt{}\PYGZgt{} }\PYG{n}{np}\PYG{o}{.}\PYG{n}{all}\PYG{p}{(}\PYG{n}{s} \PYG{o}{\PYGZgt{}}\PYG{o}{=} \PYG{o}{\PYGZhy{}}\PYG{l+m+mi}{1}\PYG{p}{)}
\PYG{g+go}{True}
\PYG{g+gp}{\PYGZgt{}\PYGZgt{}\PYGZgt{} }\PYG{n}{np}\PYG{o}{.}\PYG{n}{all}\PYG{p}{(}\PYG{n}{s} \PYG{o}{\PYGZlt{}} \PYG{l+m+mi}{0}\PYG{p}{)}
\PYG{g+go}{True}
\end{Verbatim}

Display the histogram of the samples, along with the
probability density function:

\begin{Verbatim}[commandchars=\\\{\}]
\PYG{g+gp}{\PYGZgt{}\PYGZgt{}\PYGZgt{} }\PYG{k+kn}{import} \PYG{n+nn}{matplotlib.pyplot} \PYG{k+kn}{as} \PYG{n+nn}{plt}
\PYG{g+gp}{\PYGZgt{}\PYGZgt{}\PYGZgt{} }\PYG{n}{count}\PYG{p}{,} \PYG{n}{bins}\PYG{p}{,} \PYG{n}{ignored} \PYG{o}{=} \PYG{n}{plt}\PYG{o}{.}\PYG{n}{hist}\PYG{p}{(}\PYG{n}{s}\PYG{p}{,} \PYG{l+m+mi}{15}\PYG{p}{,} \PYG{n}{normed}\PYG{o}{=}\PYG{n+nb+bp}{True}\PYG{p}{)}
\PYG{g+gp}{\PYGZgt{}\PYGZgt{}\PYGZgt{} }\PYG{n}{plt}\PYG{o}{.}\PYG{n}{plot}\PYG{p}{(}\PYG{n}{bins}\PYG{p}{,} \PYG{n}{np}\PYG{o}{.}\PYG{n}{ones\PYGZus{}like}\PYG{p}{(}\PYG{n}{bins}\PYG{p}{)}\PYG{p}{,} \PYG{n}{linewidth}\PYG{o}{=}\PYG{l+m+mi}{2}\PYG{p}{,} \PYG{n}{color}\PYG{o}{=}\PYG{l+s}{\PYGZsq{}}\PYG{l+s}{r}\PYG{l+s}{\PYGZsq{}}\PYG{p}{)}
\PYG{g+gp}{\PYGZgt{}\PYGZgt{}\PYGZgt{} }\PYG{n}{plt}\PYG{o}{.}\PYG{n}{show}\PYG{p}{(}\PYG{p}{)}
\end{Verbatim}

\end{fulllineitems}

\index{vonmises() (in module lib.dyn.dynamics)}

\begin{fulllineitems}
\phantomsection\label{lib.dyn:lib.dyn.dynamics.vonmises}\pysiglinewithargsret{\code{lib.dyn.dynamics.}\bfcode{vonmises}}{\emph{mu}, \emph{kappa}, \emph{size=None}}{}
Draw samples from a von Mises distribution.

Samples are drawn from a von Mises distribution with specified mode
(mu) and dispersion (kappa), on the interval {[}-pi, pi{]}.

The von Mises distribution (also known as the circular normal
distribution) is a continuous probability distribution on the unit
circle.  It may be thought of as the circular analogue of the normal
distribution.
\begin{description}
\item[{mu}] \leavevmode{[}float{]}
Mode (``center'') of the distribution.

\item[{kappa}] \leavevmode{[}float{]}
Dispersion of the distribution, has to be \textgreater{}=0.

\item[{size}] \leavevmode{[}int or tuple of int{]}
Output shape.  If the given shape is, e.g., \code{(m, n, k)}, then
\code{m * n * k} samples are drawn.

\end{description}
\begin{description}
\item[{samples}] \leavevmode{[}scalar or ndarray{]}
The returned samples, which are in the interval {[}-pi, pi{]}.

\end{description}
\begin{description}
\item[{scipy.stats.distributions.vonmises}] \leavevmode{[}probability density function,{]}
distribution, or cumulative density function, etc.

\end{description}

The probability density for the von Mises distribution is
\begin{gather}
\begin{split}p(x) = \frac{e^{\kappa cos(x-\mu)}}{2\pi I_0(\kappa)},\end{split}\notag
\end{gather}
where \(\mu\) is the mode and \(\kappa\) the dispersion,
and \(I_0(\kappa)\) is the modified Bessel function of order 0.

The von Mises is named for Richard Edler von Mises, who was born in
Austria-Hungary, in what is now the Ukraine.  He fled to the United
States in 1939 and became a professor at Harvard.  He worked in
probability theory, aerodynamics, fluid mechanics, and philosophy of
science.

Abramowitz, M. and Stegun, I. A. (ed.), \emph{Handbook of Mathematical
Functions}, New York: Dover, 1965.

von Mises, R., \emph{Mathematical Theory of Probability and Statistics},
New York: Academic Press, 1964.

Draw samples from the distribution:

\begin{Verbatim}[commandchars=\\\{\}]
\PYG{g+gp}{\PYGZgt{}\PYGZgt{}\PYGZgt{} }\PYG{n}{mu}\PYG{p}{,} \PYG{n}{kappa} \PYG{o}{=} \PYG{l+m+mf}{0.0}\PYG{p}{,} \PYG{l+m+mf}{4.0} \PYG{c}{\PYGZsh{} mean and dispersion}
\PYG{g+gp}{\PYGZgt{}\PYGZgt{}\PYGZgt{} }\PYG{n}{s} \PYG{o}{=} \PYG{n}{np}\PYG{o}{.}\PYG{n}{random}\PYG{o}{.}\PYG{n}{vonmises}\PYG{p}{(}\PYG{n}{mu}\PYG{p}{,} \PYG{n}{kappa}\PYG{p}{,} \PYG{l+m+mi}{1000}\PYG{p}{)}
\end{Verbatim}

Display the histogram of the samples, along with
the probability density function:

\begin{Verbatim}[commandchars=\\\{\}]
\PYG{g+gp}{\PYGZgt{}\PYGZgt{}\PYGZgt{} }\PYG{k+kn}{import} \PYG{n+nn}{matplotlib.pyplot} \PYG{k+kn}{as} \PYG{n+nn}{plt}
\PYG{g+gp}{\PYGZgt{}\PYGZgt{}\PYGZgt{} }\PYG{k+kn}{import} \PYG{n+nn}{scipy.special} \PYG{k+kn}{as} \PYG{n+nn}{sps}
\PYG{g+gp}{\PYGZgt{}\PYGZgt{}\PYGZgt{} }\PYG{n}{count}\PYG{p}{,} \PYG{n}{bins}\PYG{p}{,} \PYG{n}{ignored} \PYG{o}{=} \PYG{n}{plt}\PYG{o}{.}\PYG{n}{hist}\PYG{p}{(}\PYG{n}{s}\PYG{p}{,} \PYG{l+m+mi}{50}\PYG{p}{,} \PYG{n}{normed}\PYG{o}{=}\PYG{n+nb+bp}{True}\PYG{p}{)}
\PYG{g+gp}{\PYGZgt{}\PYGZgt{}\PYGZgt{} }\PYG{n}{x} \PYG{o}{=} \PYG{n}{np}\PYG{o}{.}\PYG{n}{arange}\PYG{p}{(}\PYG{o}{\PYGZhy{}}\PYG{n}{np}\PYG{o}{.}\PYG{n}{pi}\PYG{p}{,} \PYG{n}{np}\PYG{o}{.}\PYG{n}{pi}\PYG{p}{,} \PYG{l+m+mi}{2}\PYG{o}{*}\PYG{n}{np}\PYG{o}{.}\PYG{n}{pi}\PYG{o}{/}\PYG{l+m+mf}{50.}\PYG{p}{)}
\PYG{g+gp}{\PYGZgt{}\PYGZgt{}\PYGZgt{} }\PYG{n}{y} \PYG{o}{=} \PYG{o}{\PYGZhy{}}\PYG{n}{np}\PYG{o}{.}\PYG{n}{exp}\PYG{p}{(}\PYG{n}{kappa}\PYG{o}{*}\PYG{n}{np}\PYG{o}{.}\PYG{n}{cos}\PYG{p}{(}\PYG{n}{x}\PYG{o}{\PYGZhy{}}\PYG{n}{mu}\PYG{p}{)}\PYG{p}{)}\PYG{o}{/}\PYG{p}{(}\PYG{l+m+mi}{2}\PYG{o}{*}\PYG{n}{np}\PYG{o}{.}\PYG{n}{pi}\PYG{o}{*}\PYG{n}{sps}\PYG{o}{.}\PYG{n}{jn}\PYG{p}{(}\PYG{l+m+mi}{0}\PYG{p}{,}\PYG{n}{kappa}\PYG{p}{)}\PYG{p}{)}
\PYG{g+gp}{\PYGZgt{}\PYGZgt{}\PYGZgt{} }\PYG{n}{plt}\PYG{o}{.}\PYG{n}{plot}\PYG{p}{(}\PYG{n}{x}\PYG{p}{,} \PYG{n}{y}\PYG{o}{/}\PYG{n+nb}{max}\PYG{p}{(}\PYG{n}{y}\PYG{p}{)}\PYG{p}{,} \PYG{n}{linewidth}\PYG{o}{=}\PYG{l+m+mi}{2}\PYG{p}{,} \PYG{n}{color}\PYG{o}{=}\PYG{l+s}{\PYGZsq{}}\PYG{l+s}{r}\PYG{l+s}{\PYGZsq{}}\PYG{p}{)}
\PYG{g+gp}{\PYGZgt{}\PYGZgt{}\PYGZgt{} }\PYG{n}{plt}\PYG{o}{.}\PYG{n}{show}\PYG{p}{(}\PYG{p}{)}
\end{Verbatim}

\end{fulllineitems}

\index{wald() (in module lib.dyn.dynamics)}

\begin{fulllineitems}
\phantomsection\label{lib.dyn:lib.dyn.dynamics.wald}\pysiglinewithargsret{\code{lib.dyn.dynamics.}\bfcode{wald}}{\emph{mean}, \emph{scale}, \emph{size=None}}{}
Draw samples from a Wald, or Inverse Gaussian, distribution.

As the scale approaches infinity, the distribution becomes more like a
Gaussian.

Some references claim that the Wald is an Inverse Gaussian with mean=1, but
this is by no means universal.

The Inverse Gaussian distribution was first studied in relationship to
Brownian motion. In 1956 M.C.K. Tweedie used the name Inverse Gaussian
because there is an inverse relationship between the time to cover a unit
distance and distance covered in unit time.
\begin{description}
\item[{mean}] \leavevmode{[}scalar{]}
Distribution mean, should be \textgreater{} 0.

\item[{scale}] \leavevmode{[}scalar{]}
Scale parameter, should be \textgreater{}= 0.

\item[{size}] \leavevmode{[}int or tuple of ints, optional{]}
Output shape. Default is None, in which case a single value is
returned.

\end{description}
\begin{description}
\item[{samples}] \leavevmode{[}ndarray or scalar{]}
Drawn sample, all greater than zero.

\end{description}

The probability density function for the Wald distribution is
\begin{gather}
\begin{split}P(x;mean,scale) = \sqrt{\frac{scale}{2\pi x^3}}e^
\frac{-scale(x-mean)^2}{2\cdotp mean^2x}\end{split}\notag
\end{gather}
As noted above the Inverse Gaussian distribution first arise from attempts
to model Brownian Motion. It is also a competitor to the Weibull for use in
reliability modeling and modeling stock returns and interest rate
processes.

Draw values from the distribution and plot the histogram:

\begin{Verbatim}[commandchars=\\\{\}]
\PYG{g+gp}{\PYGZgt{}\PYGZgt{}\PYGZgt{} }\PYG{k+kn}{import} \PYG{n+nn}{matplotlib.pyplot} \PYG{k+kn}{as} \PYG{n+nn}{plt}
\PYG{g+gp}{\PYGZgt{}\PYGZgt{}\PYGZgt{} }\PYG{n}{h} \PYG{o}{=} \PYG{n}{plt}\PYG{o}{.}\PYG{n}{hist}\PYG{p}{(}\PYG{n}{np}\PYG{o}{.}\PYG{n}{random}\PYG{o}{.}\PYG{n}{wald}\PYG{p}{(}\PYG{l+m+mi}{3}\PYG{p}{,} \PYG{l+m+mi}{2}\PYG{p}{,} \PYG{l+m+mi}{100000}\PYG{p}{)}\PYG{p}{,} \PYG{n}{bins}\PYG{o}{=}\PYG{l+m+mi}{200}\PYG{p}{,} \PYG{n}{normed}\PYG{o}{=}\PYG{n+nb+bp}{True}\PYG{p}{)}
\PYG{g+gp}{\PYGZgt{}\PYGZgt{}\PYGZgt{} }\PYG{n}{plt}\PYG{o}{.}\PYG{n}{show}\PYG{p}{(}\PYG{p}{)}
\end{Verbatim}

\end{fulllineitems}

\index{weibull() (in module lib.dyn.dynamics)}

\begin{fulllineitems}
\phantomsection\label{lib.dyn:lib.dyn.dynamics.weibull}\pysiglinewithargsret{\code{lib.dyn.dynamics.}\bfcode{weibull}}{\emph{a}, \emph{size=None}}{}
Weibull distribution.

Draw samples from a 1-parameter Weibull distribution with the given
shape parameter \emph{a}.
\begin{gather}
\begin{split}X = (-ln(U))^{1/a}\end{split}\notag
\end{gather}
Here, U is drawn from the uniform distribution over (0,1{]}.

The more common 2-parameter Weibull, including a scale parameter
\(\lambda\) is just \(X = \lambda(-ln(U))^{1/a}\).
\begin{description}
\item[{a}] \leavevmode{[}float{]}
Shape of the distribution.

\item[{size}] \leavevmode{[}tuple of ints{]}
Output shape.  If the given shape is, e.g., \code{(m, n, k)}, then
\code{m * n * k} samples are drawn.

\end{description}

scipy.stats.distributions.weibull\_max
scipy.stats.distributions.weibull\_min
scipy.stats.distributions.genextreme
gumbel

The Weibull (or Type III asymptotic extreme value distribution for smallest
values, SEV Type III, or Rosin-Rammler distribution) is one of a class of
Generalized Extreme Value (GEV) distributions used in modeling extreme
value problems.  This class includes the Gumbel and Frechet distributions.

The probability density for the Weibull distribution is
\begin{gather}
\begin{split}p(x) = \frac{a}
{\lambda}(\frac{x}{\lambda})^{a-1}e^{-(x/\lambda)^a},\end{split}\notag
\end{gather}
where \(a\) is the shape and \(\lambda\) the scale.

The function has its peak (the mode) at
\(\lambda(\frac{a-1}{a})^{1/a}\).

When \code{a = 1}, the Weibull distribution reduces to the exponential
distribution.

Draw samples from the distribution:

\begin{Verbatim}[commandchars=\\\{\}]
\PYG{g+gp}{\PYGZgt{}\PYGZgt{}\PYGZgt{} }\PYG{n}{a} \PYG{o}{=} \PYG{l+m+mf}{5.} \PYG{c}{\PYGZsh{} shape}
\PYG{g+gp}{\PYGZgt{}\PYGZgt{}\PYGZgt{} }\PYG{n}{s} \PYG{o}{=} \PYG{n}{np}\PYG{o}{.}\PYG{n}{random}\PYG{o}{.}\PYG{n}{weibull}\PYG{p}{(}\PYG{n}{a}\PYG{p}{,} \PYG{l+m+mi}{1000}\PYG{p}{)}
\end{Verbatim}

Display the histogram of the samples, along with
the probability density function:

\begin{Verbatim}[commandchars=\\\{\}]
\PYG{g+gp}{\PYGZgt{}\PYGZgt{}\PYGZgt{} }\PYG{k+kn}{import} \PYG{n+nn}{matplotlib.pyplot} \PYG{k+kn}{as} \PYG{n+nn}{plt}
\PYG{g+gp}{\PYGZgt{}\PYGZgt{}\PYGZgt{} }\PYG{n}{x} \PYG{o}{=} \PYG{n}{np}\PYG{o}{.}\PYG{n}{arange}\PYG{p}{(}\PYG{l+m+mi}{1}\PYG{p}{,}\PYG{l+m+mf}{100.}\PYG{p}{)}\PYG{o}{/}\PYG{l+m+mf}{50.}
\PYG{g+gp}{\PYGZgt{}\PYGZgt{}\PYGZgt{} }\PYG{k}{def} \PYG{n+nf}{weib}\PYG{p}{(}\PYG{n}{x}\PYG{p}{,}\PYG{n}{n}\PYG{p}{,}\PYG{n}{a}\PYG{p}{)}\PYG{p}{:}
\PYG{g+gp}{... }    \PYG{k}{return} \PYG{p}{(}\PYG{n}{a} \PYG{o}{/} \PYG{n}{n}\PYG{p}{)} \PYG{o}{*} \PYG{p}{(}\PYG{n}{x} \PYG{o}{/} \PYG{n}{n}\PYG{p}{)}\PYG{o}{*}\PYG{o}{*}\PYG{p}{(}\PYG{n}{a} \PYG{o}{\PYGZhy{}} \PYG{l+m+mi}{1}\PYG{p}{)} \PYG{o}{*} \PYG{n}{np}\PYG{o}{.}\PYG{n}{exp}\PYG{p}{(}\PYG{o}{\PYGZhy{}}\PYG{p}{(}\PYG{n}{x} \PYG{o}{/} \PYG{n}{n}\PYG{p}{)}\PYG{o}{*}\PYG{o}{*}\PYG{n}{a}\PYG{p}{)}
\end{Verbatim}

\begin{Verbatim}[commandchars=\\\{\}]
\PYG{g+gp}{\PYGZgt{}\PYGZgt{}\PYGZgt{} }\PYG{n}{count}\PYG{p}{,} \PYG{n}{bins}\PYG{p}{,} \PYG{n}{ignored} \PYG{o}{=} \PYG{n}{plt}\PYG{o}{.}\PYG{n}{hist}\PYG{p}{(}\PYG{n}{np}\PYG{o}{.}\PYG{n}{random}\PYG{o}{.}\PYG{n}{weibull}\PYG{p}{(}\PYG{l+m+mf}{5.}\PYG{p}{,}\PYG{l+m+mi}{1000}\PYG{p}{)}\PYG{p}{)}
\PYG{g+gp}{\PYGZgt{}\PYGZgt{}\PYGZgt{} }\PYG{n}{x} \PYG{o}{=} \PYG{n}{np}\PYG{o}{.}\PYG{n}{arange}\PYG{p}{(}\PYG{l+m+mi}{1}\PYG{p}{,}\PYG{l+m+mf}{100.}\PYG{p}{)}\PYG{o}{/}\PYG{l+m+mf}{50.}
\PYG{g+gp}{\PYGZgt{}\PYGZgt{}\PYGZgt{} }\PYG{n}{scale} \PYG{o}{=} \PYG{n}{count}\PYG{o}{.}\PYG{n}{max}\PYG{p}{(}\PYG{p}{)}\PYG{o}{/}\PYG{n}{weib}\PYG{p}{(}\PYG{n}{x}\PYG{p}{,} \PYG{l+m+mf}{1.}\PYG{p}{,} \PYG{l+m+mf}{5.}\PYG{p}{)}\PYG{o}{.}\PYG{n}{max}\PYG{p}{(}\PYG{p}{)}
\PYG{g+gp}{\PYGZgt{}\PYGZgt{}\PYGZgt{} }\PYG{n}{plt}\PYG{o}{.}\PYG{n}{plot}\PYG{p}{(}\PYG{n}{x}\PYG{p}{,} \PYG{n}{weib}\PYG{p}{(}\PYG{n}{x}\PYG{p}{,} \PYG{l+m+mf}{1.}\PYG{p}{,} \PYG{l+m+mf}{5.}\PYG{p}{)}\PYG{o}{*}\PYG{n}{scale}\PYG{p}{)}
\PYG{g+gp}{\PYGZgt{}\PYGZgt{}\PYGZgt{} }\PYG{n}{plt}\PYG{o}{.}\PYG{n}{show}\PYG{p}{(}\PYG{p}{)}
\end{Verbatim}

\end{fulllineitems}

\index{zipf() (in module lib.dyn.dynamics)}

\begin{fulllineitems}
\phantomsection\label{lib.dyn:lib.dyn.dynamics.zipf}\pysiglinewithargsret{\code{lib.dyn.dynamics.}\bfcode{zipf}}{\emph{a}, \emph{size=None}}{}
Draw samples from a Zipf distribution.

Samples are drawn from a Zipf distribution with specified parameter
\emph{a} \textgreater{} 1.

The Zipf distribution (also known as the zeta distribution) is a
continuous probability distribution that satisfies Zipf's law: the
frequency of an item is inversely proportional to its rank in a
frequency table.
\begin{description}
\item[{a}] \leavevmode{[}float \textgreater{} 1{]}
Distribution parameter.

\item[{size}] \leavevmode{[}int or tuple of int, optional{]}
Output shape.  If the given shape is, e.g., \code{(m, n, k)}, then
\code{m * n * k} samples are drawn; a single integer is equivalent in
its result to providing a mono-tuple, i.e., a 1-D array of length
\emph{size} is returned.  The default is None, in which case a single
scalar is returned.

\end{description}
\begin{description}
\item[{samples}] \leavevmode{[}scalar or ndarray{]}
The returned samples are greater than or equal to one.

\end{description}
\begin{description}
\item[{scipy.stats.distributions.zipf}] \leavevmode{[}probability density function,{]}
distribution, or cumulative density function, etc.

\end{description}

The probability density for the Zipf distribution is
\begin{gather}
\begin{split}p(x) = \frac{x^{-a}}{\zeta(a)},\end{split}\notag
\end{gather}
where \(\zeta\) is the Riemann Zeta function.

It is named for the American linguist George Kingsley Zipf, who noted
that the frequency of any word in a sample of a language is inversely
proportional to its rank in the frequency table.

Zipf, G. K., \emph{Selected Studies of the Principle of Relative Frequency
in Language}, Cambridge, MA: Harvard Univ. Press, 1932.

Draw samples from the distribution:

\begin{Verbatim}[commandchars=\\\{\}]
\PYG{g+gp}{\PYGZgt{}\PYGZgt{}\PYGZgt{} }\PYG{n}{a} \PYG{o}{=} \PYG{l+m+mf}{2.} \PYG{c}{\PYGZsh{} parameter}
\PYG{g+gp}{\PYGZgt{}\PYGZgt{}\PYGZgt{} }\PYG{n}{s} \PYG{o}{=} \PYG{n}{np}\PYG{o}{.}\PYG{n}{random}\PYG{o}{.}\PYG{n}{zipf}\PYG{p}{(}\PYG{n}{a}\PYG{p}{,} \PYG{l+m+mi}{1000}\PYG{p}{)}
\end{Verbatim}

Display the histogram of the samples, along with
the probability density function:

\begin{Verbatim}[commandchars=\\\{\}]
\PYG{g+gp}{\PYGZgt{}\PYGZgt{}\PYGZgt{} }\PYG{k+kn}{import} \PYG{n+nn}{matplotlib.pyplot} \PYG{k+kn}{as} \PYG{n+nn}{plt}
\PYG{g+gp}{\PYGZgt{}\PYGZgt{}\PYGZgt{} }\PYG{k+kn}{import} \PYG{n+nn}{scipy.special} \PYG{k+kn}{as} \PYG{n+nn}{sps}
\PYG{g+go}{Truncate s values at 50 so plot is interesting}
\PYG{g+gp}{\PYGZgt{}\PYGZgt{}\PYGZgt{} }\PYG{n}{count}\PYG{p}{,} \PYG{n}{bins}\PYG{p}{,} \PYG{n}{ignored} \PYG{o}{=} \PYG{n}{plt}\PYG{o}{.}\PYG{n}{hist}\PYG{p}{(}\PYG{n}{s}\PYG{p}{[}\PYG{n}{s}\PYG{o}{\PYGZlt{}}\PYG{l+m+mi}{50}\PYG{p}{]}\PYG{p}{,} \PYG{l+m+mi}{50}\PYG{p}{,} \PYG{n}{normed}\PYG{o}{=}\PYG{n+nb+bp}{True}\PYG{p}{)}
\PYG{g+gp}{\PYGZgt{}\PYGZgt{}\PYGZgt{} }\PYG{n}{x} \PYG{o}{=} \PYG{n}{np}\PYG{o}{.}\PYG{n}{arange}\PYG{p}{(}\PYG{l+m+mf}{1.}\PYG{p}{,} \PYG{l+m+mf}{50.}\PYG{p}{)}
\PYG{g+gp}{\PYGZgt{}\PYGZgt{}\PYGZgt{} }\PYG{n}{y} \PYG{o}{=} \PYG{n}{x}\PYG{o}{*}\PYG{o}{*}\PYG{p}{(}\PYG{o}{\PYGZhy{}}\PYG{n}{a}\PYG{p}{)}\PYG{o}{/}\PYG{n}{sps}\PYG{o}{.}\PYG{n}{zetac}\PYG{p}{(}\PYG{n}{a}\PYG{p}{)}
\PYG{g+gp}{\PYGZgt{}\PYGZgt{}\PYGZgt{} }\PYG{n}{plt}\PYG{o}{.}\PYG{n}{plot}\PYG{p}{(}\PYG{n}{x}\PYG{p}{,} \PYG{n}{y}\PYG{o}{/}\PYG{n+nb}{max}\PYG{p}{(}\PYG{n}{y}\PYG{p}{)}\PYG{p}{,} \PYG{n}{linewidth}\PYG{o}{=}\PYG{l+m+mi}{2}\PYG{p}{,} \PYG{n}{color}\PYG{o}{=}\PYG{l+s}{\PYGZsq{}}\PYG{l+s}{r}\PYG{l+s}{\PYGZsq{}}\PYG{p}{)}
\PYG{g+gp}{\PYGZgt{}\PYGZgt{}\PYGZgt{} }\PYG{n}{plt}\PYG{o}{.}\PYG{n}{show}\PYG{p}{(}\PYG{p}{)}
\end{Verbatim}

\end{fulllineitems}



\subsection{graph Package}
\label{lib.graph::doc}\label{lib.graph:graph-package}

\subsubsection{\texttt{network} Module}
\label{lib.graph:module-lib.graph.network}\label{lib.graph:network-module}\index{lib.graph.network (module)}\index{beta() (in module lib.graph.network)}

\begin{fulllineitems}
\phantomsection\label{lib.graph:lib.graph.network.beta}\pysiglinewithargsret{\code{lib.graph.network.}\bfcode{beta}}{\emph{a}, \emph{b}, \emph{size=None}}{}
The Beta distribution over \code{{[}0, 1{]}}.

The Beta distribution is a special case of the Dirichlet distribution,
and is related to the Gamma distribution.  It has the probability
distribution function
\begin{gather}
\begin{split}f(x; a,b) = \frac{1}{B(\alpha, \beta)} x^{\alpha - 1}
(1 - x)^{\beta - 1},\end{split}\notag
\end{gather}
where the normalisation, B, is the beta function,
\begin{gather}
\begin{split}B(\alpha, \beta) = \int_0^1 t^{\alpha - 1}
(1 - t)^{\beta - 1} dt.\end{split}\notag
\end{gather}
It is often seen in Bayesian inference and order statistics.
\begin{description}
\item[{a}] \leavevmode{[}float{]}
Alpha, non-negative.

\item[{b}] \leavevmode{[}float{]}
Beta, non-negative.

\item[{size}] \leavevmode{[}tuple of ints, optional{]}
The number of samples to draw.  The output is packed according to
the size given.

\end{description}
\begin{description}
\item[{out}] \leavevmode{[}ndarray{]}
Array of the given shape, containing values drawn from a
Beta distribution.

\end{description}

\end{fulllineitems}

\index{binomial() (in module lib.graph.network)}

\begin{fulllineitems}
\phantomsection\label{lib.graph:lib.graph.network.binomial}\pysiglinewithargsret{\code{lib.graph.network.}\bfcode{binomial}}{\emph{n}, \emph{p}, \emph{size=None}}{}
Draw samples from a binomial distribution.

Samples are drawn from a Binomial distribution with specified
parameters, n trials and p probability of success where
n an integer \textgreater{}= 0 and p is in the interval {[}0,1{]}. (n may be
input as a float, but it is truncated to an integer in use)
\begin{description}
\item[{n}] \leavevmode{[}float (but truncated to an integer){]}
parameter, \textgreater{}= 0.

\item[{p}] \leavevmode{[}float{]}
parameter, \textgreater{}= 0 and \textless{}=1.

\item[{size}] \leavevmode{[}\{tuple, int\}{]}
Output shape.  If the given shape is, e.g., \code{(m, n, k)}, then
\code{m * n * k} samples are drawn.

\end{description}
\begin{description}
\item[{samples}] \leavevmode{[}\{ndarray, scalar\}{]}
where the values are all integers in  {[}0, n{]}.

\end{description}
\begin{description}
\item[{scipy.stats.distributions.binom}] \leavevmode{[}probability density function,{]}
distribution or cumulative density function, etc.

\end{description}

The probability density for the Binomial distribution is
\begin{gather}
\begin{split}P(N) = \binom{n}{N}p^N(1-p)^{n-N},\end{split}\notag
\end{gather}
where \(n\) is the number of trials, \(p\) is the probability
of success, and \(N\) is the number of successes.

When estimating the standard error of a proportion in a population by
using a random sample, the normal distribution works well unless the
product p*n \textless{}=5, where p = population proportion estimate, and n =
number of samples, in which case the binomial distribution is used
instead. For example, a sample of 15 people shows 4 who are left
handed, and 11 who are right handed. Then p = 4/15 = 27\%. 0.27*15 = 4,
so the binomial distribution should be used in this case.

Draw samples from the distribution:

\begin{Verbatim}[commandchars=\\\{\}]
\PYG{g+gp}{\PYGZgt{}\PYGZgt{}\PYGZgt{} }\PYG{n}{n}\PYG{p}{,} \PYG{n}{p} \PYG{o}{=} \PYG{l+m+mi}{10}\PYG{p}{,} \PYG{o}{.}\PYG{l+m+mi}{5} \PYG{c}{\PYGZsh{} number of trials, probability of each trial}
\PYG{g+gp}{\PYGZgt{}\PYGZgt{}\PYGZgt{} }\PYG{n}{s} \PYG{o}{=} \PYG{n}{np}\PYG{o}{.}\PYG{n}{random}\PYG{o}{.}\PYG{n}{binomial}\PYG{p}{(}\PYG{n}{n}\PYG{p}{,} \PYG{n}{p}\PYG{p}{,} \PYG{l+m+mi}{1000}\PYG{p}{)}
\PYG{g+go}{\PYGZsh{} result of flipping a coin 10 times, tested 1000 times.}
\end{Verbatim}

A real world example. A company drills 9 wild-cat oil exploration
wells, each with an estimated probability of success of 0.1. All nine
wells fail. What is the probability of that happening?

Let's do 20,000 trials of the model, and count the number that
generate zero positive results.

\begin{Verbatim}[commandchars=\\\{\}]
\PYG{g+gp}{\PYGZgt{}\PYGZgt{}\PYGZgt{} }\PYG{n+nb}{sum}\PYG{p}{(}\PYG{n}{np}\PYG{o}{.}\PYG{n}{random}\PYG{o}{.}\PYG{n}{binomial}\PYG{p}{(}\PYG{l+m+mi}{9}\PYG{p}{,}\PYG{l+m+mf}{0.1}\PYG{p}{,}\PYG{l+m+mi}{20000}\PYG{p}{)}\PYG{o}{==}\PYG{l+m+mi}{0}\PYG{p}{)}\PYG{o}{/}\PYG{l+m+mf}{20000.}
\PYG{g+go}{answer = 0.38885, or 38\PYGZpc{}.}
\end{Verbatim}

\end{fulllineitems}

\index{chisquare() (in module lib.graph.network)}

\begin{fulllineitems}
\phantomsection\label{lib.graph:lib.graph.network.chisquare}\pysiglinewithargsret{\code{lib.graph.network.}\bfcode{chisquare}}{\emph{df}, \emph{size=None}}{}
Draw samples from a chi-square distribution.

When \emph{df} independent random variables, each with standard normal
distributions (mean 0, variance 1), are squared and summed, the
resulting distribution is chi-square (see Notes).  This distribution
is often used in hypothesis testing.
\begin{description}
\item[{df}] \leavevmode{[}int{]}
Number of degrees of freedom.

\item[{size}] \leavevmode{[}tuple of ints, int, optional{]}
Size of the returned array.  By default, a scalar is
returned.

\end{description}
\begin{description}
\item[{output}] \leavevmode{[}ndarray{]}
Samples drawn from the distribution, packed in a \emph{size}-shaped
array.

\end{description}
\begin{description}
\item[{ValueError}] \leavevmode
When \emph{df} \textless{}= 0 or when an inappropriate \emph{size} (e.g. \code{size=-1})
is given.

\end{description}

The variable obtained by summing the squares of \emph{df} independent,
standard normally distributed random variables:
\begin{gather}
\begin{split}Q = \sum_{i=0}^{\mathtt{df}} X^2_i\end{split}\notag
\end{gather}
is chi-square distributed, denoted
\begin{gather}
\begin{split}Q \sim \chi^2_k.\end{split}\notag
\end{gather}
The probability density function of the chi-squared distribution is
\begin{gather}
\begin{split}p(x) = \frac{(1/2)^{k/2}}{\Gamma(k/2)}
x^{k/2 - 1} e^{-x/2},\end{split}\notag
\end{gather}
where \(\Gamma\) is the gamma function,
\begin{gather}
\begin{split}\Gamma(x) = \int_0^{-\infty} t^{x - 1} e^{-t} dt.\end{split}\notag
\end{gather}
\href{http://www.itl.nist.gov/div898/handbook/eda/section3/eda3666.htm}{NIST/SEMATECH e-Handbook of Statistical Methods}

\begin{Verbatim}[commandchars=\\\{\}]
\PYG{g+gp}{\PYGZgt{}\PYGZgt{}\PYGZgt{} }\PYG{n}{np}\PYG{o}{.}\PYG{n}{random}\PYG{o}{.}\PYG{n}{chisquare}\PYG{p}{(}\PYG{l+m+mi}{2}\PYG{p}{,}\PYG{l+m+mi}{4}\PYG{p}{)}
\PYG{g+go}{array([ 1.89920014,  9.00867716,  3.13710533,  5.62318272])}
\end{Verbatim}

\end{fulllineitems}

\index{create\_chemistry() (in module lib.graph.network)}

\begin{fulllineitems}
\phantomsection\label{lib.graph:lib.graph.network.create_chemistry}\pysiglinewithargsret{\code{lib.graph.network.}\bfcode{create\_chemistry}}{\emph{args}, \emph{originalSpeciesList}, \emph{parameters}, \emph{rctToCat}, \emph{totCleavage}, \emph{totCond}, \emph{tmpac}, \emph{autocat=True}}{}
\end{fulllineitems}

\index{exponential() (in module lib.graph.network)}

\begin{fulllineitems}
\phantomsection\label{lib.graph:lib.graph.network.exponential}\pysiglinewithargsret{\code{lib.graph.network.}\bfcode{exponential}}{\emph{scale=1.0}, \emph{size=None}}{}
Exponential distribution.

Its probability density function is
\begin{gather}
\begin{split}f(x; \frac{1}{\beta}) = \frac{1}{\beta} \exp(-\frac{x}{\beta}),\end{split}\notag
\end{gather}
for \code{x \textgreater{} 0} and 0 elsewhere. \(\beta\) is the scale parameter,
which is the inverse of the rate parameter \(\lambda = 1/\beta\).
The rate parameter is an alternative, widely used parameterization
of the exponential distribution {\color{red}\bfseries{}{[}3{]}\_}.

The exponential distribution is a continuous analogue of the
geometric distribution.  It describes many common situations, such as
the size of raindrops measured over many rainstorms {\color{red}\bfseries{}{[}1{]}\_}, or the time
between page requests to Wikipedia {\color{red}\bfseries{}{[}2{]}\_}.
\begin{description}
\item[{scale}] \leavevmode{[}float{]}
The scale parameter, \(\beta = 1/\lambda\).

\item[{size}] \leavevmode{[}tuple of ints{]}
Number of samples to draw.  The output is shaped
according to \emph{size}.

\end{description}

\end{fulllineitems}

\index{f() (in module lib.graph.network)}

\begin{fulllineitems}
\phantomsection\label{lib.graph:lib.graph.network.f}\pysiglinewithargsret{\code{lib.graph.network.}\bfcode{f}}{\emph{dfnum}, \emph{dfden}, \emph{size=None}}{}
Draw samples from a F distribution.

Samples are drawn from an F distribution with specified parameters,
\emph{dfnum} (degrees of freedom in numerator) and \emph{dfden} (degrees of freedom
in denominator), where both parameters should be greater than zero.

The random variate of the F distribution (also known as the
Fisher distribution) is a continuous probability distribution
that arises in ANOVA tests, and is the ratio of two chi-square
variates.
\begin{description}
\item[{dfnum}] \leavevmode{[}float{]}
Degrees of freedom in numerator. Should be greater than zero.

\item[{dfden}] \leavevmode{[}float{]}
Degrees of freedom in denominator. Should be greater than zero.

\item[{size}] \leavevmode{[}\{tuple, int\}, optional{]}
Output shape.  If the given shape is, e.g., \code{(m, n, k)},
then \code{m * n * k} samples are drawn. By default only one sample
is returned.

\end{description}
\begin{description}
\item[{samples}] \leavevmode{[}\{ndarray, scalar\}{]}
Samples from the Fisher distribution.

\end{description}
\begin{description}
\item[{scipy.stats.distributions.f}] \leavevmode{[}probability density function,{]}
distribution or cumulative density function, etc.

\end{description}

The F statistic is used to compare in-group variances to between-group
variances. Calculating the distribution depends on the sampling, and
so it is a function of the respective degrees of freedom in the
problem.  The variable \emph{dfnum} is the number of samples minus one, the
between-groups degrees of freedom, while \emph{dfden} is the within-groups
degrees of freedom, the sum of the number of samples in each group
minus the number of groups.

An example from Glantz{[}1{]}, pp 47-40.
Two groups, children of diabetics (25 people) and children from people
without diabetes (25 controls). Fasting blood glucose was measured,
case group had a mean value of 86.1, controls had a mean value of
82.2. Standard deviations were 2.09 and 2.49 respectively. Are these
data consistent with the null hypothesis that the parents diabetic
status does not affect their children's blood glucose levels?
Calculating the F statistic from the data gives a value of 36.01.

Draw samples from the distribution:

\begin{Verbatim}[commandchars=\\\{\}]
\PYG{g+gp}{\PYGZgt{}\PYGZgt{}\PYGZgt{} }\PYG{n}{dfnum} \PYG{o}{=} \PYG{l+m+mf}{1.} \PYG{c}{\PYGZsh{} between group degrees of freedom}
\PYG{g+gp}{\PYGZgt{}\PYGZgt{}\PYGZgt{} }\PYG{n}{dfden} \PYG{o}{=} \PYG{l+m+mf}{48.} \PYG{c}{\PYGZsh{} within groups degrees of freedom}
\PYG{g+gp}{\PYGZgt{}\PYGZgt{}\PYGZgt{} }\PYG{n}{s} \PYG{o}{=} \PYG{n}{np}\PYG{o}{.}\PYG{n}{random}\PYG{o}{.}\PYG{n}{f}\PYG{p}{(}\PYG{n}{dfnum}\PYG{p}{,} \PYG{n}{dfden}\PYG{p}{,} \PYG{l+m+mi}{1000}\PYG{p}{)}
\end{Verbatim}

The lower bound for the top 1\% of the samples is :

\begin{Verbatim}[commandchars=\\\{\}]
\PYG{g+gp}{\PYGZgt{}\PYGZgt{}\PYGZgt{} }\PYG{n}{sort}\PYG{p}{(}\PYG{n}{s}\PYG{p}{)}\PYG{p}{[}\PYG{o}{\PYGZhy{}}\PYG{l+m+mi}{10}\PYG{p}{]}
\PYG{g+go}{7.61988120985}
\end{Verbatim}

So there is about a 1\% chance that the F statistic will exceed 7.62,
the measured value is 36, so the null hypothesis is rejected at the 1\%
level.

\end{fulllineitems}

\index{fixCondensationReaction() (in module lib.graph.network)}

\begin{fulllineitems}
\phantomsection\label{lib.graph:lib.graph.network.fixCondensationReaction}\pysiglinewithargsret{\code{lib.graph.network.}\bfcode{fixCondensationReaction}}{\emph{m1}, \emph{m2}, \emph{m3}, \emph{rcts}}{}
\end{fulllineitems}

\index{gamma() (in module lib.graph.network)}

\begin{fulllineitems}
\phantomsection\label{lib.graph:lib.graph.network.gamma}\pysiglinewithargsret{\code{lib.graph.network.}\bfcode{gamma}}{\emph{shape}, \emph{scale=1.0}, \emph{size=None}}{}
Draw samples from a Gamma distribution.

Samples are drawn from a Gamma distribution with specified parameters,
\emph{shape} (sometimes designated ``k'') and \emph{scale} (sometimes designated
``theta''), where both parameters are \textgreater{} 0.
\begin{description}
\item[{shape}] \leavevmode{[}scalar \textgreater{} 0{]}
The shape of the gamma distribution.

\item[{scale}] \leavevmode{[}scalar \textgreater{} 0, optional{]}
The scale of the gamma distribution.  Default is equal to 1.

\item[{size}] \leavevmode{[}shape\_tuple, optional{]}
Output shape.  If the given shape is, e.g., \code{(m, n, k)}, then
\code{m * n * k} samples are drawn.

\end{description}
\begin{description}
\item[{out}] \leavevmode{[}ndarray, float{]}
Returns one sample unless \emph{size} parameter is specified.

\end{description}
\begin{description}
\item[{scipy.stats.distributions.gamma}] \leavevmode{[}probability density function,{]}
distribution or cumulative density function, etc.

\end{description}

The probability density for the Gamma distribution is
\begin{gather}
\begin{split}p(x) = x^{k-1}\frac{e^{-x/\theta}}{\theta^k\Gamma(k)},\end{split}\notag
\end{gather}
where \(k\) is the shape and \(\theta\) the scale,
and \(\Gamma\) is the Gamma function.

The Gamma distribution is often used to model the times to failure of
electronic components, and arises naturally in processes for which the
waiting times between Poisson distributed events are relevant.

Draw samples from the distribution:

\begin{Verbatim}[commandchars=\\\{\}]
\PYG{g+gp}{\PYGZgt{}\PYGZgt{}\PYGZgt{} }\PYG{n}{shape}\PYG{p}{,} \PYG{n}{scale} \PYG{o}{=} \PYG{l+m+mf}{2.}\PYG{p}{,} \PYG{l+m+mf}{2.} \PYG{c}{\PYGZsh{} mean and dispersion}
\PYG{g+gp}{\PYGZgt{}\PYGZgt{}\PYGZgt{} }\PYG{n}{s} \PYG{o}{=} \PYG{n}{np}\PYG{o}{.}\PYG{n}{random}\PYG{o}{.}\PYG{n}{gamma}\PYG{p}{(}\PYG{n}{shape}\PYG{p}{,} \PYG{n}{scale}\PYG{p}{,} \PYG{l+m+mi}{1000}\PYG{p}{)}
\end{Verbatim}

Display the histogram of the samples, along with
the probability density function:

\begin{Verbatim}[commandchars=\\\{\}]
\PYG{g+gp}{\PYGZgt{}\PYGZgt{}\PYGZgt{} }\PYG{k+kn}{import} \PYG{n+nn}{matplotlib.pyplot} \PYG{k+kn}{as} \PYG{n+nn}{plt}
\PYG{g+gp}{\PYGZgt{}\PYGZgt{}\PYGZgt{} }\PYG{k+kn}{import} \PYG{n+nn}{scipy.special} \PYG{k+kn}{as} \PYG{n+nn}{sps}
\PYG{g+gp}{\PYGZgt{}\PYGZgt{}\PYGZgt{} }\PYG{n}{count}\PYG{p}{,} \PYG{n}{bins}\PYG{p}{,} \PYG{n}{ignored} \PYG{o}{=} \PYG{n}{plt}\PYG{o}{.}\PYG{n}{hist}\PYG{p}{(}\PYG{n}{s}\PYG{p}{,} \PYG{l+m+mi}{50}\PYG{p}{,} \PYG{n}{normed}\PYG{o}{=}\PYG{n+nb+bp}{True}\PYG{p}{)}
\PYG{g+gp}{\PYGZgt{}\PYGZgt{}\PYGZgt{} }\PYG{n}{y} \PYG{o}{=} \PYG{n}{bins}\PYG{o}{*}\PYG{o}{*}\PYG{p}{(}\PYG{n}{shape}\PYG{o}{\PYGZhy{}}\PYG{l+m+mi}{1}\PYG{p}{)}\PYG{o}{*}\PYG{p}{(}\PYG{n}{np}\PYG{o}{.}\PYG{n}{exp}\PYG{p}{(}\PYG{o}{\PYGZhy{}}\PYG{n}{bins}\PYG{o}{/}\PYG{n}{scale}\PYG{p}{)} \PYG{o}{/}
\PYG{g+gp}{... }                     \PYG{p}{(}\PYG{n}{sps}\PYG{o}{.}\PYG{n}{gamma}\PYG{p}{(}\PYG{n}{shape}\PYG{p}{)}\PYG{o}{*}\PYG{n}{scale}\PYG{o}{*}\PYG{o}{*}\PYG{n}{shape}\PYG{p}{)}\PYG{p}{)}
\PYG{g+gp}{\PYGZgt{}\PYGZgt{}\PYGZgt{} }\PYG{n}{plt}\PYG{o}{.}\PYG{n}{plot}\PYG{p}{(}\PYG{n}{bins}\PYG{p}{,} \PYG{n}{y}\PYG{p}{,} \PYG{n}{linewidth}\PYG{o}{=}\PYG{l+m+mi}{2}\PYG{p}{,} \PYG{n}{color}\PYG{o}{=}\PYG{l+s}{\PYGZsq{}}\PYG{l+s}{r}\PYG{l+s}{\PYGZsq{}}\PYG{p}{)}
\PYG{g+gp}{\PYGZgt{}\PYGZgt{}\PYGZgt{} }\PYG{n}{plt}\PYG{o}{.}\PYG{n}{show}\PYG{p}{(}\PYG{p}{)}
\end{Verbatim}

\end{fulllineitems}

\index{geometric() (in module lib.graph.network)}

\begin{fulllineitems}
\phantomsection\label{lib.graph:lib.graph.network.geometric}\pysiglinewithargsret{\code{lib.graph.network.}\bfcode{geometric}}{\emph{p}, \emph{size=None}}{}
Draw samples from the geometric distribution.

Bernoulli trials are experiments with one of two outcomes:
success or failure (an example of such an experiment is flipping
a coin).  The geometric distribution models the number of trials
that must be run in order to achieve success.  It is therefore
supported on the positive integers, \code{k = 1, 2, ...}.

The probability mass function of the geometric distribution is
\begin{gather}
\begin{split}f(k) = (1 - p)^{k - 1} p\end{split}\notag
\end{gather}
where \emph{p} is the probability of success of an individual trial.
\begin{description}
\item[{p}] \leavevmode{[}float{]}
The probability of success of an individual trial.

\item[{size}] \leavevmode{[}tuple of ints{]}
Number of values to draw from the distribution.  The output
is shaped according to \emph{size}.

\end{description}
\begin{description}
\item[{out}] \leavevmode{[}ndarray{]}
Samples from the geometric distribution, shaped according to
\emph{size}.

\end{description}

Draw ten thousand values from the geometric distribution,
with the probability of an individual success equal to 0.35:

\begin{Verbatim}[commandchars=\\\{\}]
\PYG{g+gp}{\PYGZgt{}\PYGZgt{}\PYGZgt{} }\PYG{n}{z} \PYG{o}{=} \PYG{n}{np}\PYG{o}{.}\PYG{n}{random}\PYG{o}{.}\PYG{n}{geometric}\PYG{p}{(}\PYG{n}{p}\PYG{o}{=}\PYG{l+m+mf}{0.35}\PYG{p}{,} \PYG{n}{size}\PYG{o}{=}\PYG{l+m+mi}{10000}\PYG{p}{)}
\end{Verbatim}

How many trials succeeded after a single run?

\begin{Verbatim}[commandchars=\\\{\}]
\PYG{g+gp}{\PYGZgt{}\PYGZgt{}\PYGZgt{} }\PYG{p}{(}\PYG{n}{z} \PYG{o}{==} \PYG{l+m+mi}{1}\PYG{p}{)}\PYG{o}{.}\PYG{n}{sum}\PYG{p}{(}\PYG{p}{)} \PYG{o}{/} \PYG{l+m+mf}{10000.}
\PYG{g+go}{0.34889999999999999 \PYGZsh{}random}
\end{Verbatim}

\end{fulllineitems}

\index{get\_state() (in module lib.graph.network)}

\begin{fulllineitems}
\phantomsection\label{lib.graph:lib.graph.network.get_state}\pysiglinewithargsret{\code{lib.graph.network.}\bfcode{get\_state}}{}{}
Return a tuple representing the internal state of the generator.

For more details, see \emph{set\_state}.
\begin{description}
\item[{out}] \leavevmode{[}tuple(str, ndarray of 624 uints, int, int, float){]}
The returned tuple has the following items:
\begin{enumerate}
\item {} 
the string `MT19937'.

\item {} 
a 1-D array of 624 unsigned integer keys.

\item {} 
an integer \code{pos}.

\item {} 
an integer \code{has\_gauss}.

\item {} 
a float \code{cached\_gaussian}.

\end{enumerate}

\end{description}

set\_state

\emph{set\_state} and \emph{get\_state} are not needed to work with any of the
random distributions in NumPy. If the internal state is manually altered,
the user should know exactly what he/she is doing.

\end{fulllineitems}

\index{gumbel() (in module lib.graph.network)}

\begin{fulllineitems}
\phantomsection\label{lib.graph:lib.graph.network.gumbel}\pysiglinewithargsret{\code{lib.graph.network.}\bfcode{gumbel}}{\emph{loc=0.0}, \emph{scale=1.0}, \emph{size=None}}{}
Gumbel distribution.

Draw samples from a Gumbel distribution with specified location and scale.
For more information on the Gumbel distribution, see Notes and References
below.
\begin{description}
\item[{loc}] \leavevmode{[}float{]}
The location of the mode of the distribution.

\item[{scale}] \leavevmode{[}float{]}
The scale parameter of the distribution.

\item[{size}] \leavevmode{[}tuple of ints{]}
Output shape.  If the given shape is, e.g., \code{(m, n, k)}, then
\code{m * n * k} samples are drawn.

\end{description}
\begin{description}
\item[{out}] \leavevmode{[}ndarray{]}
The samples

\end{description}

scipy.stats.gumbel\_l
scipy.stats.gumbel\_r
scipy.stats.genextreme
\begin{quote}

probability density function, distribution, or cumulative density
function, etc. for each of the above
\end{quote}

weibull

The Gumbel (or Smallest Extreme Value (SEV) or the Smallest Extreme Value
Type I) distribution is one of a class of Generalized Extreme Value (GEV)
distributions used in modeling extreme value problems.  The Gumbel is a
special case of the Extreme Value Type I distribution for maximums from
distributions with ``exponential-like'' tails.

The probability density for the Gumbel distribution is
\begin{gather}
\begin{split}p(x) = \frac{e^{-(x - \mu)/ \beta}}{\beta} e^{ -e^{-(x - \mu)/
\beta}},\end{split}\notag
\end{gather}
where \(\mu\) is the mode, a location parameter, and \(\beta\) is
the scale parameter.

The Gumbel (named for German mathematician Emil Julius Gumbel) was used
very early in the hydrology literature, for modeling the occurrence of
flood events. It is also used for modeling maximum wind speed and rainfall
rates.  It is a ``fat-tailed'' distribution - the probability of an event in
the tail of the distribution is larger than if one used a Gaussian, hence
the surprisingly frequent occurrence of 100-year floods. Floods were
initially modeled as a Gaussian process, which underestimated the frequency
of extreme events.

It is one of a class of extreme value distributions, the Generalized
Extreme Value (GEV) distributions, which also includes the Weibull and
Frechet.

The function has a mean of \(\mu + 0.57721\beta\) and a variance of
\(\frac{\pi^2}{6}\beta^2\).

Gumbel, E. J., \emph{Statistics of Extremes}, New York: Columbia University
Press, 1958.

Reiss, R.-D. and Thomas, M., \emph{Statistical Analysis of Extreme Values from
Insurance, Finance, Hydrology and Other Fields}, Basel: Birkhauser Verlag,
2001.

Draw samples from the distribution:

\begin{Verbatim}[commandchars=\\\{\}]
\PYG{g+gp}{\PYGZgt{}\PYGZgt{}\PYGZgt{} }\PYG{n}{mu}\PYG{p}{,} \PYG{n}{beta} \PYG{o}{=} \PYG{l+m+mi}{0}\PYG{p}{,} \PYG{l+m+mf}{0.1} \PYG{c}{\PYGZsh{} location and scale}
\PYG{g+gp}{\PYGZgt{}\PYGZgt{}\PYGZgt{} }\PYG{n}{s} \PYG{o}{=} \PYG{n}{np}\PYG{o}{.}\PYG{n}{random}\PYG{o}{.}\PYG{n}{gumbel}\PYG{p}{(}\PYG{n}{mu}\PYG{p}{,} \PYG{n}{beta}\PYG{p}{,} \PYG{l+m+mi}{1000}\PYG{p}{)}
\end{Verbatim}

Display the histogram of the samples, along with
the probability density function:

\begin{Verbatim}[commandchars=\\\{\}]
\PYG{g+gp}{\PYGZgt{}\PYGZgt{}\PYGZgt{} }\PYG{k+kn}{import} \PYG{n+nn}{matplotlib.pyplot} \PYG{k+kn}{as} \PYG{n+nn}{plt}
\PYG{g+gp}{\PYGZgt{}\PYGZgt{}\PYGZgt{} }\PYG{n}{count}\PYG{p}{,} \PYG{n}{bins}\PYG{p}{,} \PYG{n}{ignored} \PYG{o}{=} \PYG{n}{plt}\PYG{o}{.}\PYG{n}{hist}\PYG{p}{(}\PYG{n}{s}\PYG{p}{,} \PYG{l+m+mi}{30}\PYG{p}{,} \PYG{n}{normed}\PYG{o}{=}\PYG{n+nb+bp}{True}\PYG{p}{)}
\PYG{g+gp}{\PYGZgt{}\PYGZgt{}\PYGZgt{} }\PYG{n}{plt}\PYG{o}{.}\PYG{n}{plot}\PYG{p}{(}\PYG{n}{bins}\PYG{p}{,} \PYG{p}{(}\PYG{l+m+mi}{1}\PYG{o}{/}\PYG{n}{beta}\PYG{p}{)}\PYG{o}{*}\PYG{n}{np}\PYG{o}{.}\PYG{n}{exp}\PYG{p}{(}\PYG{o}{\PYGZhy{}}\PYG{p}{(}\PYG{n}{bins} \PYG{o}{\PYGZhy{}} \PYG{n}{mu}\PYG{p}{)}\PYG{o}{/}\PYG{n}{beta}\PYG{p}{)}
\PYG{g+gp}{... }         \PYG{o}{*} \PYG{n}{np}\PYG{o}{.}\PYG{n}{exp}\PYG{p}{(} \PYG{o}{\PYGZhy{}}\PYG{n}{np}\PYG{o}{.}\PYG{n}{exp}\PYG{p}{(} \PYG{o}{\PYGZhy{}}\PYG{p}{(}\PYG{n}{bins} \PYG{o}{\PYGZhy{}} \PYG{n}{mu}\PYG{p}{)} \PYG{o}{/}\PYG{n}{beta}\PYG{p}{)} \PYG{p}{)}\PYG{p}{,}
\PYG{g+gp}{... }         \PYG{n}{linewidth}\PYG{o}{=}\PYG{l+m+mi}{2}\PYG{p}{,} \PYG{n}{color}\PYG{o}{=}\PYG{l+s}{\PYGZsq{}}\PYG{l+s}{r}\PYG{l+s}{\PYGZsq{}}\PYG{p}{)}
\PYG{g+gp}{\PYGZgt{}\PYGZgt{}\PYGZgt{} }\PYG{n}{plt}\PYG{o}{.}\PYG{n}{show}\PYG{p}{(}\PYG{p}{)}
\end{Verbatim}

Show how an extreme value distribution can arise from a Gaussian process
and compare to a Gaussian:

\begin{Verbatim}[commandchars=\\\{\}]
\PYG{g+gp}{\PYGZgt{}\PYGZgt{}\PYGZgt{} }\PYG{n}{means} \PYG{o}{=} \PYG{p}{[}\PYG{p}{]}
\PYG{g+gp}{\PYGZgt{}\PYGZgt{}\PYGZgt{} }\PYG{n}{maxima} \PYG{o}{=} \PYG{p}{[}\PYG{p}{]}
\PYG{g+gp}{\PYGZgt{}\PYGZgt{}\PYGZgt{} }\PYG{k}{for} \PYG{n}{i} \PYG{o+ow}{in} \PYG{n+nb}{range}\PYG{p}{(}\PYG{l+m+mi}{0}\PYG{p}{,}\PYG{l+m+mi}{1000}\PYG{p}{)} \PYG{p}{:}
\PYG{g+gp}{... }   \PYG{n}{a} \PYG{o}{=} \PYG{n}{np}\PYG{o}{.}\PYG{n}{random}\PYG{o}{.}\PYG{n}{normal}\PYG{p}{(}\PYG{n}{mu}\PYG{p}{,} \PYG{n}{beta}\PYG{p}{,} \PYG{l+m+mi}{1000}\PYG{p}{)}
\PYG{g+gp}{... }   \PYG{n}{means}\PYG{o}{.}\PYG{n}{append}\PYG{p}{(}\PYG{n}{a}\PYG{o}{.}\PYG{n}{mean}\PYG{p}{(}\PYG{p}{)}\PYG{p}{)}
\PYG{g+gp}{... }   \PYG{n}{maxima}\PYG{o}{.}\PYG{n}{append}\PYG{p}{(}\PYG{n}{a}\PYG{o}{.}\PYG{n}{max}\PYG{p}{(}\PYG{p}{)}\PYG{p}{)}
\PYG{g+gp}{\PYGZgt{}\PYGZgt{}\PYGZgt{} }\PYG{n}{count}\PYG{p}{,} \PYG{n}{bins}\PYG{p}{,} \PYG{n}{ignored} \PYG{o}{=} \PYG{n}{plt}\PYG{o}{.}\PYG{n}{hist}\PYG{p}{(}\PYG{n}{maxima}\PYG{p}{,} \PYG{l+m+mi}{30}\PYG{p}{,} \PYG{n}{normed}\PYG{o}{=}\PYG{n+nb+bp}{True}\PYG{p}{)}
\PYG{g+gp}{\PYGZgt{}\PYGZgt{}\PYGZgt{} }\PYG{n}{beta} \PYG{o}{=} \PYG{n}{np}\PYG{o}{.}\PYG{n}{std}\PYG{p}{(}\PYG{n}{maxima}\PYG{p}{)}\PYG{o}{*}\PYG{n}{np}\PYG{o}{.}\PYG{n}{pi}\PYG{o}{/}\PYG{n}{np}\PYG{o}{.}\PYG{n}{sqrt}\PYG{p}{(}\PYG{l+m+mi}{6}\PYG{p}{)}
\PYG{g+gp}{\PYGZgt{}\PYGZgt{}\PYGZgt{} }\PYG{n}{mu} \PYG{o}{=} \PYG{n}{np}\PYG{o}{.}\PYG{n}{mean}\PYG{p}{(}\PYG{n}{maxima}\PYG{p}{)} \PYG{o}{\PYGZhy{}} \PYG{l+m+mf}{0.57721}\PYG{o}{*}\PYG{n}{beta}
\PYG{g+gp}{\PYGZgt{}\PYGZgt{}\PYGZgt{} }\PYG{n}{plt}\PYG{o}{.}\PYG{n}{plot}\PYG{p}{(}\PYG{n}{bins}\PYG{p}{,} \PYG{p}{(}\PYG{l+m+mi}{1}\PYG{o}{/}\PYG{n}{beta}\PYG{p}{)}\PYG{o}{*}\PYG{n}{np}\PYG{o}{.}\PYG{n}{exp}\PYG{p}{(}\PYG{o}{\PYGZhy{}}\PYG{p}{(}\PYG{n}{bins} \PYG{o}{\PYGZhy{}} \PYG{n}{mu}\PYG{p}{)}\PYG{o}{/}\PYG{n}{beta}\PYG{p}{)}
\PYG{g+gp}{... }         \PYG{o}{*} \PYG{n}{np}\PYG{o}{.}\PYG{n}{exp}\PYG{p}{(}\PYG{o}{\PYGZhy{}}\PYG{n}{np}\PYG{o}{.}\PYG{n}{exp}\PYG{p}{(}\PYG{o}{\PYGZhy{}}\PYG{p}{(}\PYG{n}{bins} \PYG{o}{\PYGZhy{}} \PYG{n}{mu}\PYG{p}{)}\PYG{o}{/}\PYG{n}{beta}\PYG{p}{)}\PYG{p}{)}\PYG{p}{,}
\PYG{g+gp}{... }         \PYG{n}{linewidth}\PYG{o}{=}\PYG{l+m+mi}{2}\PYG{p}{,} \PYG{n}{color}\PYG{o}{=}\PYG{l+s}{\PYGZsq{}}\PYG{l+s}{r}\PYG{l+s}{\PYGZsq{}}\PYG{p}{)}
\PYG{g+gp}{\PYGZgt{}\PYGZgt{}\PYGZgt{} }\PYG{n}{plt}\PYG{o}{.}\PYG{n}{plot}\PYG{p}{(}\PYG{n}{bins}\PYG{p}{,} \PYG{l+m+mi}{1}\PYG{o}{/}\PYG{p}{(}\PYG{n}{beta} \PYG{o}{*} \PYG{n}{np}\PYG{o}{.}\PYG{n}{sqrt}\PYG{p}{(}\PYG{l+m+mi}{2} \PYG{o}{*} \PYG{n}{np}\PYG{o}{.}\PYG{n}{pi}\PYG{p}{)}\PYG{p}{)}
\PYG{g+gp}{... }         \PYG{o}{*} \PYG{n}{np}\PYG{o}{.}\PYG{n}{exp}\PYG{p}{(}\PYG{o}{\PYGZhy{}}\PYG{p}{(}\PYG{n}{bins} \PYG{o}{\PYGZhy{}} \PYG{n}{mu}\PYG{p}{)}\PYG{o}{*}\PYG{o}{*}\PYG{l+m+mi}{2} \PYG{o}{/} \PYG{p}{(}\PYG{l+m+mi}{2} \PYG{o}{*} \PYG{n}{beta}\PYG{o}{*}\PYG{o}{*}\PYG{l+m+mi}{2}\PYG{p}{)}\PYG{p}{)}\PYG{p}{,}
\PYG{g+gp}{... }         \PYG{n}{linewidth}\PYG{o}{=}\PYG{l+m+mi}{2}\PYG{p}{,} \PYG{n}{color}\PYG{o}{=}\PYG{l+s}{\PYGZsq{}}\PYG{l+s}{g}\PYG{l+s}{\PYGZsq{}}\PYG{p}{)}
\PYG{g+gp}{\PYGZgt{}\PYGZgt{}\PYGZgt{} }\PYG{n}{plt}\PYG{o}{.}\PYG{n}{show}\PYG{p}{(}\PYG{p}{)}
\end{Verbatim}

\end{fulllineitems}

\index{hypergeometric() (in module lib.graph.network)}

\begin{fulllineitems}
\phantomsection\label{lib.graph:lib.graph.network.hypergeometric}\pysiglinewithargsret{\code{lib.graph.network.}\bfcode{hypergeometric}}{\emph{ngood}, \emph{nbad}, \emph{nsample}, \emph{size=None}}{}
Draw samples from a Hypergeometric distribution.

Samples are drawn from a Hypergeometric distribution with specified
parameters, ngood (ways to make a good selection), nbad (ways to make
a bad selection), and nsample = number of items sampled, which is less
than or equal to the sum ngood + nbad.
\begin{description}
\item[{ngood}] \leavevmode{[}int or array\_like{]}
Number of ways to make a good selection.  Must be nonnegative.

\item[{nbad}] \leavevmode{[}int or array\_like{]}
Number of ways to make a bad selection.  Must be nonnegative.

\item[{nsample}] \leavevmode{[}int or array\_like{]}
Number of items sampled.  Must be at least 1 and at most
\code{ngood + nbad}.

\item[{size}] \leavevmode{[}int or tuple of int{]}
Output shape.  If the given shape is, e.g., \code{(m, n, k)}, then
\code{m * n * k} samples are drawn.

\end{description}
\begin{description}
\item[{samples}] \leavevmode{[}ndarray or scalar{]}
The values are all integers in  {[}0, n{]}.

\end{description}
\begin{description}
\item[{scipy.stats.distributions.hypergeom}] \leavevmode{[}probability density function,{]}
distribution or cumulative density function, etc.

\end{description}

The probability density for the Hypergeometric distribution is
\begin{gather}
\begin{split}P(x) = \frac{\binom{m}{n}\binom{N-m}{n-x}}{\binom{N}{n}},\end{split}\notag
\end{gather}
where \(0 \le x \le m\) and \(n+m-N \le x \le n\)

for P(x) the probability of x successes, n = ngood, m = nbad, and
N = number of samples.

Consider an urn with black and white marbles in it, ngood of them
black and nbad are white. If you draw nsample balls without
replacement, then the Hypergeometric distribution describes the
distribution of black balls in the drawn sample.

Note that this distribution is very similar to the Binomial
distribution, except that in this case, samples are drawn without
replacement, whereas in the Binomial case samples are drawn with
replacement (or the sample space is infinite). As the sample space
becomes large, this distribution approaches the Binomial.

Draw samples from the distribution:

\begin{Verbatim}[commandchars=\\\{\}]
\PYG{g+gp}{\PYGZgt{}\PYGZgt{}\PYGZgt{} }\PYG{n}{ngood}\PYG{p}{,} \PYG{n}{nbad}\PYG{p}{,} \PYG{n}{nsamp} \PYG{o}{=} \PYG{l+m+mi}{100}\PYG{p}{,} \PYG{l+m+mi}{2}\PYG{p}{,} \PYG{l+m+mi}{10}
\PYG{g+go}{\PYGZsh{} number of good, number of bad, and number of samples}
\PYG{g+gp}{\PYGZgt{}\PYGZgt{}\PYGZgt{} }\PYG{n}{s} \PYG{o}{=} \PYG{n}{np}\PYG{o}{.}\PYG{n}{random}\PYG{o}{.}\PYG{n}{hypergeometric}\PYG{p}{(}\PYG{n}{ngood}\PYG{p}{,} \PYG{n}{nbad}\PYG{p}{,} \PYG{n}{nsamp}\PYG{p}{,} \PYG{l+m+mi}{1000}\PYG{p}{)}
\PYG{g+gp}{\PYGZgt{}\PYGZgt{}\PYGZgt{} }\PYG{n}{hist}\PYG{p}{(}\PYG{n}{s}\PYG{p}{)}
\PYG{g+go}{\PYGZsh{}   note that it is very unlikely to grab both bad items}
\end{Verbatim}

Suppose you have an urn with 15 white and 15 black marbles.
If you pull 15 marbles at random, how likely is it that
12 or more of them are one color?

\begin{Verbatim}[commandchars=\\\{\}]
\PYG{g+gp}{\PYGZgt{}\PYGZgt{}\PYGZgt{} }\PYG{n}{s} \PYG{o}{=} \PYG{n}{np}\PYG{o}{.}\PYG{n}{random}\PYG{o}{.}\PYG{n}{hypergeometric}\PYG{p}{(}\PYG{l+m+mi}{15}\PYG{p}{,} \PYG{l+m+mi}{15}\PYG{p}{,} \PYG{l+m+mi}{15}\PYG{p}{,} \PYG{l+m+mi}{100000}\PYG{p}{)}
\PYG{g+gp}{\PYGZgt{}\PYGZgt{}\PYGZgt{} }\PYG{n+nb}{sum}\PYG{p}{(}\PYG{n}{s}\PYG{o}{\PYGZgt{}}\PYG{o}{=}\PYG{l+m+mi}{12}\PYG{p}{)}\PYG{o}{/}\PYG{l+m+mf}{100000.} \PYG{o}{+} \PYG{n+nb}{sum}\PYG{p}{(}\PYG{n}{s}\PYG{o}{\PYGZlt{}}\PYG{o}{=}\PYG{l+m+mi}{3}\PYG{p}{)}\PYG{o}{/}\PYG{l+m+mf}{100000.}
\PYG{g+go}{\PYGZsh{}   answer = 0.003 ... pretty unlikely!}
\end{Verbatim}

\end{fulllineitems}

\index{laplace() (in module lib.graph.network)}

\begin{fulllineitems}
\phantomsection\label{lib.graph:lib.graph.network.laplace}\pysiglinewithargsret{\code{lib.graph.network.}\bfcode{laplace}}{\emph{loc=0.0}, \emph{scale=1.0}, \emph{size=None}}{}
Draw samples from the Laplace or double exponential distribution with
specified location (or mean) and scale (decay).

The Laplace distribution is similar to the Gaussian/normal distribution,
but is sharper at the peak and has fatter tails. It represents the
difference between two independent, identically distributed exponential
random variables.
\begin{description}
\item[{loc}] \leavevmode{[}float{]}
The position, \(\mu\), of the distribution peak.

\item[{scale}] \leavevmode{[}float{]}
\(\lambda\), the exponential decay.

\end{description}

It has the probability density function
\begin{gather}
\begin{split}f(x; \mu, \lambda) = \frac{1}{2\lambda}
\exp\left(-\frac{|x - \mu|}{\lambda}\right).\end{split}\notag
\end{gather}
The first law of Laplace, from 1774, states that the frequency of an error
can be expressed as an exponential function of the absolute magnitude of
the error, which leads to the Laplace distribution. For many problems in
Economics and Health sciences, this distribution seems to model the data
better than the standard Gaussian distribution

Draw samples from the distribution

\begin{Verbatim}[commandchars=\\\{\}]
\PYG{g+gp}{\PYGZgt{}\PYGZgt{}\PYGZgt{} }\PYG{n}{loc}\PYG{p}{,} \PYG{n}{scale} \PYG{o}{=} \PYG{l+m+mf}{0.}\PYG{p}{,} \PYG{l+m+mf}{1.}
\PYG{g+gp}{\PYGZgt{}\PYGZgt{}\PYGZgt{} }\PYG{n}{s} \PYG{o}{=} \PYG{n}{np}\PYG{o}{.}\PYG{n}{random}\PYG{o}{.}\PYG{n}{laplace}\PYG{p}{(}\PYG{n}{loc}\PYG{p}{,} \PYG{n}{scale}\PYG{p}{,} \PYG{l+m+mi}{1000}\PYG{p}{)}
\end{Verbatim}

Display the histogram of the samples, along with
the probability density function:

\begin{Verbatim}[commandchars=\\\{\}]
\PYG{g+gp}{\PYGZgt{}\PYGZgt{}\PYGZgt{} }\PYG{k+kn}{import} \PYG{n+nn}{matplotlib.pyplot} \PYG{k+kn}{as} \PYG{n+nn}{plt}
\PYG{g+gp}{\PYGZgt{}\PYGZgt{}\PYGZgt{} }\PYG{n}{count}\PYG{p}{,} \PYG{n}{bins}\PYG{p}{,} \PYG{n}{ignored} \PYG{o}{=} \PYG{n}{plt}\PYG{o}{.}\PYG{n}{hist}\PYG{p}{(}\PYG{n}{s}\PYG{p}{,} \PYG{l+m+mi}{30}\PYG{p}{,} \PYG{n}{normed}\PYG{o}{=}\PYG{n+nb+bp}{True}\PYG{p}{)}
\PYG{g+gp}{\PYGZgt{}\PYGZgt{}\PYGZgt{} }\PYG{n}{x} \PYG{o}{=} \PYG{n}{np}\PYG{o}{.}\PYG{n}{arange}\PYG{p}{(}\PYG{o}{\PYGZhy{}}\PYG{l+m+mf}{8.}\PYG{p}{,} \PYG{l+m+mf}{8.}\PYG{p}{,} \PYG{o}{.}\PYG{l+m+mo}{01}\PYG{p}{)}
\PYG{g+gp}{\PYGZgt{}\PYGZgt{}\PYGZgt{} }\PYG{n}{pdf} \PYG{o}{=} \PYG{n}{np}\PYG{o}{.}\PYG{n}{exp}\PYG{p}{(}\PYG{o}{\PYGZhy{}}\PYG{n+nb}{abs}\PYG{p}{(}\PYG{n}{x}\PYG{o}{\PYGZhy{}}\PYG{n}{loc}\PYG{o}{/}\PYG{n}{scale}\PYG{p}{)}\PYG{p}{)}\PYG{o}{/}\PYG{p}{(}\PYG{l+m+mf}{2.}\PYG{o}{*}\PYG{n}{scale}\PYG{p}{)}
\PYG{g+gp}{\PYGZgt{}\PYGZgt{}\PYGZgt{} }\PYG{n}{plt}\PYG{o}{.}\PYG{n}{plot}\PYG{p}{(}\PYG{n}{x}\PYG{p}{,} \PYG{n}{pdf}\PYG{p}{)}
\end{Verbatim}

Plot Gaussian for comparison:

\begin{Verbatim}[commandchars=\\\{\}]
\PYG{g+gp}{\PYGZgt{}\PYGZgt{}\PYGZgt{} }\PYG{n}{g} \PYG{o}{=} \PYG{p}{(}\PYG{l+m+mi}{1}\PYG{o}{/}\PYG{p}{(}\PYG{n}{scale} \PYG{o}{*} \PYG{n}{np}\PYG{o}{.}\PYG{n}{sqrt}\PYG{p}{(}\PYG{l+m+mi}{2} \PYG{o}{*} \PYG{n}{np}\PYG{o}{.}\PYG{n}{pi}\PYG{p}{)}\PYG{p}{)} \PYG{o}{*} 
\PYG{g+gp}{... }     \PYG{n}{np}\PYG{o}{.}\PYG{n}{exp}\PYG{p}{(} \PYG{o}{\PYGZhy{}} \PYG{p}{(}\PYG{n}{x} \PYG{o}{\PYGZhy{}} \PYG{n}{loc}\PYG{p}{)}\PYG{o}{*}\PYG{o}{*}\PYG{l+m+mi}{2} \PYG{o}{/} \PYG{p}{(}\PYG{l+m+mi}{2} \PYG{o}{*} \PYG{n}{scale}\PYG{o}{*}\PYG{o}{*}\PYG{l+m+mi}{2}\PYG{p}{)} \PYG{p}{)}\PYG{p}{)}
\PYG{g+gp}{\PYGZgt{}\PYGZgt{}\PYGZgt{} }\PYG{n}{plt}\PYG{o}{.}\PYG{n}{plot}\PYG{p}{(}\PYG{n}{x}\PYG{p}{,}\PYG{n}{g}\PYG{p}{)}
\end{Verbatim}

\end{fulllineitems}

\index{logistic() (in module lib.graph.network)}

\begin{fulllineitems}
\phantomsection\label{lib.graph:lib.graph.network.logistic}\pysiglinewithargsret{\code{lib.graph.network.}\bfcode{logistic}}{\emph{loc=0.0}, \emph{scale=1.0}, \emph{size=None}}{}
Draw samples from a Logistic distribution.

Samples are drawn from a Logistic distribution with specified
parameters, loc (location or mean, also median), and scale (\textgreater{}0).

loc : float

scale : float \textgreater{} 0.
\begin{description}
\item[{size}] \leavevmode{[}\{tuple, int\}{]}
Output shape.  If the given shape is, e.g., \code{(m, n, k)}, then
\code{m * n * k} samples are drawn.

\end{description}
\begin{description}
\item[{samples}] \leavevmode{[}\{ndarray, scalar\}{]}
where the values are all integers in  {[}0, n{]}.

\end{description}
\begin{description}
\item[{scipy.stats.distributions.logistic}] \leavevmode{[}probability density function,{]}
distribution or cumulative density function, etc.

\end{description}

The probability density for the Logistic distribution is
\begin{gather}
\begin{split}P(x) = P(x) = \frac{e^{-(x-\mu)/s}}{s(1+e^{-(x-\mu)/s})^2},\end{split}\notag
\end{gather}
where \(\mu\) = location and \(s\) = scale.

The Logistic distribution is used in Extreme Value problems where it
can act as a mixture of Gumbel distributions, in Epidemiology, and by
the World Chess Federation (FIDE) where it is used in the Elo ranking
system, assuming the performance of each player is a logistically
distributed random variable.

Draw samples from the distribution:

\begin{Verbatim}[commandchars=\\\{\}]
\PYG{g+gp}{\PYGZgt{}\PYGZgt{}\PYGZgt{} }\PYG{n}{loc}\PYG{p}{,} \PYG{n}{scale} \PYG{o}{=} \PYG{l+m+mi}{10}\PYG{p}{,} \PYG{l+m+mi}{1}
\PYG{g+gp}{\PYGZgt{}\PYGZgt{}\PYGZgt{} }\PYG{n}{s} \PYG{o}{=} \PYG{n}{np}\PYG{o}{.}\PYG{n}{random}\PYG{o}{.}\PYG{n}{logistic}\PYG{p}{(}\PYG{n}{loc}\PYG{p}{,} \PYG{n}{scale}\PYG{p}{,} \PYG{l+m+mi}{10000}\PYG{p}{)}
\PYG{g+gp}{\PYGZgt{}\PYGZgt{}\PYGZgt{} }\PYG{n}{count}\PYG{p}{,} \PYG{n}{bins}\PYG{p}{,} \PYG{n}{ignored} \PYG{o}{=} \PYG{n}{plt}\PYG{o}{.}\PYG{n}{hist}\PYG{p}{(}\PYG{n}{s}\PYG{p}{,} \PYG{n}{bins}\PYG{o}{=}\PYG{l+m+mi}{50}\PYG{p}{)}
\end{Verbatim}

\#   plot against distribution

\begin{Verbatim}[commandchars=\\\{\}]
\PYG{g+gp}{\PYGZgt{}\PYGZgt{}\PYGZgt{} }\PYG{k}{def} \PYG{n+nf}{logist}\PYG{p}{(}\PYG{n}{x}\PYG{p}{,} \PYG{n}{loc}\PYG{p}{,} \PYG{n}{scale}\PYG{p}{)}\PYG{p}{:}
\PYG{g+gp}{... }    \PYG{k}{return} \PYG{n}{exp}\PYG{p}{(}\PYG{p}{(}\PYG{n}{loc}\PYG{o}{\PYGZhy{}}\PYG{n}{x}\PYG{p}{)}\PYG{o}{/}\PYG{n}{scale}\PYG{p}{)}\PYG{o}{/}\PYG{p}{(}\PYG{n}{scale}\PYG{o}{*}\PYG{p}{(}\PYG{l+m+mi}{1}\PYG{o}{+}\PYG{n}{exp}\PYG{p}{(}\PYG{p}{(}\PYG{n}{loc}\PYG{o}{\PYGZhy{}}\PYG{n}{x}\PYG{p}{)}\PYG{o}{/}\PYG{n}{scale}\PYG{p}{)}\PYG{p}{)}\PYG{o}{*}\PYG{o}{*}\PYG{l+m+mi}{2}\PYG{p}{)}
\PYG{g+gp}{\PYGZgt{}\PYGZgt{}\PYGZgt{} }\PYG{n}{plt}\PYG{o}{.}\PYG{n}{plot}\PYG{p}{(}\PYG{n}{bins}\PYG{p}{,} \PYG{n}{logist}\PYG{p}{(}\PYG{n}{bins}\PYG{p}{,} \PYG{n}{loc}\PYG{p}{,} \PYG{n}{scale}\PYG{p}{)}\PYG{o}{*}\PYG{n}{count}\PYG{o}{.}\PYG{n}{max}\PYG{p}{(}\PYG{p}{)}\PYG{o}{/}\PYGZbs{}
\PYG{g+gp}{... }\PYG{n}{logist}\PYG{p}{(}\PYG{n}{bins}\PYG{p}{,} \PYG{n}{loc}\PYG{p}{,} \PYG{n}{scale}\PYG{p}{)}\PYG{o}{.}\PYG{n}{max}\PYG{p}{(}\PYG{p}{)}\PYG{p}{)}
\PYG{g+gp}{\PYGZgt{}\PYGZgt{}\PYGZgt{} }\PYG{n}{plt}\PYG{o}{.}\PYG{n}{show}\PYG{p}{(}\PYG{p}{)}
\end{Verbatim}

\end{fulllineitems}

\index{lognormal() (in module lib.graph.network)}

\begin{fulllineitems}
\phantomsection\label{lib.graph:lib.graph.network.lognormal}\pysiglinewithargsret{\code{lib.graph.network.}\bfcode{lognormal}}{\emph{mean=0.0}, \emph{sigma=1.0}, \emph{size=None}}{}
Return samples drawn from a log-normal distribution.

Draw samples from a log-normal distribution with specified mean,
standard deviation, and array shape.  Note that the mean and standard
deviation are not the values for the distribution itself, but of the
underlying normal distribution it is derived from.
\begin{description}
\item[{mean}] \leavevmode{[}float{]}
Mean value of the underlying normal distribution

\item[{sigma}] \leavevmode{[}float, \textgreater{} 0.{]}
Standard deviation of the underlying normal distribution

\item[{size}] \leavevmode{[}tuple of ints{]}
Output shape.  If the given shape is, e.g., \code{(m, n, k)}, then
\code{m * n * k} samples are drawn.

\end{description}
\begin{description}
\item[{samples}] \leavevmode{[}ndarray or float{]}
The desired samples. An array of the same shape as \emph{size} if given,
if \emph{size} is None a float is returned.

\end{description}
\begin{description}
\item[{scipy.stats.lognorm}] \leavevmode{[}probability density function, distribution,{]}
cumulative density function, etc.

\end{description}

A variable \emph{x} has a log-normal distribution if \emph{log(x)} is normally
distributed.  The probability density function for the log-normal
distribution is:
\begin{gather}
\begin{split}p(x) = \frac{1}{\sigma x \sqrt{2\pi}}
e^{(-\frac{(ln(x)-\mu)^2}{2\sigma^2})}\end{split}\notag
\end{gather}
where \(\mu\) is the mean and \(\sigma\) is the standard
deviation of the normally distributed logarithm of the variable.
A log-normal distribution results if a random variable is the \emph{product}
of a large number of independent, identically-distributed variables in
the same way that a normal distribution results if the variable is the
\emph{sum} of a large number of independent, identically-distributed
variables.

Limpert, E., Stahel, W. A., and Abbt, M., ``Log-normal Distributions
across the Sciences: Keys and Clues,'' \emph{BioScience}, Vol. 51, No. 5,
May, 2001.  \href{http://stat.ethz.ch/~stahel/lognormal/bioscience.pdf}{http://stat.ethz.ch/\textasciitilde{}stahel/lognormal/bioscience.pdf}

Reiss, R.D. and Thomas, M., \emph{Statistical Analysis of Extreme Values},
Basel: Birkhauser Verlag, 2001, pp. 31-32.

Draw samples from the distribution:

\begin{Verbatim}[commandchars=\\\{\}]
\PYG{g+gp}{\PYGZgt{}\PYGZgt{}\PYGZgt{} }\PYG{n}{mu}\PYG{p}{,} \PYG{n}{sigma} \PYG{o}{=} \PYG{l+m+mf}{3.}\PYG{p}{,} \PYG{l+m+mf}{1.} \PYG{c}{\PYGZsh{} mean and standard deviation}
\PYG{g+gp}{\PYGZgt{}\PYGZgt{}\PYGZgt{} }\PYG{n}{s} \PYG{o}{=} \PYG{n}{np}\PYG{o}{.}\PYG{n}{random}\PYG{o}{.}\PYG{n}{lognormal}\PYG{p}{(}\PYG{n}{mu}\PYG{p}{,} \PYG{n}{sigma}\PYG{p}{,} \PYG{l+m+mi}{1000}\PYG{p}{)}
\end{Verbatim}

Display the histogram of the samples, along with
the probability density function:

\begin{Verbatim}[commandchars=\\\{\}]
\PYG{g+gp}{\PYGZgt{}\PYGZgt{}\PYGZgt{} }\PYG{k+kn}{import} \PYG{n+nn}{matplotlib.pyplot} \PYG{k+kn}{as} \PYG{n+nn}{plt}
\PYG{g+gp}{\PYGZgt{}\PYGZgt{}\PYGZgt{} }\PYG{n}{count}\PYG{p}{,} \PYG{n}{bins}\PYG{p}{,} \PYG{n}{ignored} \PYG{o}{=} \PYG{n}{plt}\PYG{o}{.}\PYG{n}{hist}\PYG{p}{(}\PYG{n}{s}\PYG{p}{,} \PYG{l+m+mi}{100}\PYG{p}{,} \PYG{n}{normed}\PYG{o}{=}\PYG{n+nb+bp}{True}\PYG{p}{,} \PYG{n}{align}\PYG{o}{=}\PYG{l+s}{\PYGZsq{}}\PYG{l+s}{mid}\PYG{l+s}{\PYGZsq{}}\PYG{p}{)}
\end{Verbatim}

\begin{Verbatim}[commandchars=\\\{\}]
\PYG{g+gp}{\PYGZgt{}\PYGZgt{}\PYGZgt{} }\PYG{n}{x} \PYG{o}{=} \PYG{n}{np}\PYG{o}{.}\PYG{n}{linspace}\PYG{p}{(}\PYG{n+nb}{min}\PYG{p}{(}\PYG{n}{bins}\PYG{p}{)}\PYG{p}{,} \PYG{n+nb}{max}\PYG{p}{(}\PYG{n}{bins}\PYG{p}{)}\PYG{p}{,} \PYG{l+m+mi}{10000}\PYG{p}{)}
\PYG{g+gp}{\PYGZgt{}\PYGZgt{}\PYGZgt{} }\PYG{n}{pdf} \PYG{o}{=} \PYG{p}{(}\PYG{n}{np}\PYG{o}{.}\PYG{n}{exp}\PYG{p}{(}\PYG{o}{\PYGZhy{}}\PYG{p}{(}\PYG{n}{np}\PYG{o}{.}\PYG{n}{log}\PYG{p}{(}\PYG{n}{x}\PYG{p}{)} \PYG{o}{\PYGZhy{}} \PYG{n}{mu}\PYG{p}{)}\PYG{o}{*}\PYG{o}{*}\PYG{l+m+mi}{2} \PYG{o}{/} \PYG{p}{(}\PYG{l+m+mi}{2} \PYG{o}{*} \PYG{n}{sigma}\PYG{o}{*}\PYG{o}{*}\PYG{l+m+mi}{2}\PYG{p}{)}\PYG{p}{)}
\PYG{g+gp}{... }       \PYG{o}{/} \PYG{p}{(}\PYG{n}{x} \PYG{o}{*} \PYG{n}{sigma} \PYG{o}{*} \PYG{n}{np}\PYG{o}{.}\PYG{n}{sqrt}\PYG{p}{(}\PYG{l+m+mi}{2} \PYG{o}{*} \PYG{n}{np}\PYG{o}{.}\PYG{n}{pi}\PYG{p}{)}\PYG{p}{)}\PYG{p}{)}
\end{Verbatim}

\begin{Verbatim}[commandchars=\\\{\}]
\PYG{g+gp}{\PYGZgt{}\PYGZgt{}\PYGZgt{} }\PYG{n}{plt}\PYG{o}{.}\PYG{n}{plot}\PYG{p}{(}\PYG{n}{x}\PYG{p}{,} \PYG{n}{pdf}\PYG{p}{,} \PYG{n}{linewidth}\PYG{o}{=}\PYG{l+m+mi}{2}\PYG{p}{,} \PYG{n}{color}\PYG{o}{=}\PYG{l+s}{\PYGZsq{}}\PYG{l+s}{r}\PYG{l+s}{\PYGZsq{}}\PYG{p}{)}
\PYG{g+gp}{\PYGZgt{}\PYGZgt{}\PYGZgt{} }\PYG{n}{plt}\PYG{o}{.}\PYG{n}{axis}\PYG{p}{(}\PYG{l+s}{\PYGZsq{}}\PYG{l+s}{tight}\PYG{l+s}{\PYGZsq{}}\PYG{p}{)}
\PYG{g+gp}{\PYGZgt{}\PYGZgt{}\PYGZgt{} }\PYG{n}{plt}\PYG{o}{.}\PYG{n}{show}\PYG{p}{(}\PYG{p}{)}
\end{Verbatim}

Demonstrate that taking the products of random samples from a uniform
distribution can be fit well by a log-normal probability density function.

\begin{Verbatim}[commandchars=\\\{\}]
\PYG{g+gp}{\PYGZgt{}\PYGZgt{}\PYGZgt{} }\PYG{c}{\PYGZsh{} Generate a thousand samples: each is the product of 100 random}
\PYG{g+gp}{\PYGZgt{}\PYGZgt{}\PYGZgt{} }\PYG{c}{\PYGZsh{} values, drawn from a normal distribution.}
\PYG{g+gp}{\PYGZgt{}\PYGZgt{}\PYGZgt{} }\PYG{n}{b} \PYG{o}{=} \PYG{p}{[}\PYG{p}{]}
\PYG{g+gp}{\PYGZgt{}\PYGZgt{}\PYGZgt{} }\PYG{k}{for} \PYG{n}{i} \PYG{o+ow}{in} \PYG{n+nb}{range}\PYG{p}{(}\PYG{l+m+mi}{1000}\PYG{p}{)}\PYG{p}{:}
\PYG{g+gp}{... }   \PYG{n}{a} \PYG{o}{=} \PYG{l+m+mf}{10.} \PYG{o}{+} \PYG{n}{np}\PYG{o}{.}\PYG{n}{random}\PYG{o}{.}\PYG{n}{random}\PYG{p}{(}\PYG{l+m+mi}{100}\PYG{p}{)}
\PYG{g+gp}{... }   \PYG{n}{b}\PYG{o}{.}\PYG{n}{append}\PYG{p}{(}\PYG{n}{np}\PYG{o}{.}\PYG{n}{product}\PYG{p}{(}\PYG{n}{a}\PYG{p}{)}\PYG{p}{)}
\end{Verbatim}

\begin{Verbatim}[commandchars=\\\{\}]
\PYG{g+gp}{\PYGZgt{}\PYGZgt{}\PYGZgt{} }\PYG{n}{b} \PYG{o}{=} \PYG{n}{np}\PYG{o}{.}\PYG{n}{array}\PYG{p}{(}\PYG{n}{b}\PYG{p}{)} \PYG{o}{/} \PYG{n}{np}\PYG{o}{.}\PYG{n}{min}\PYG{p}{(}\PYG{n}{b}\PYG{p}{)} \PYG{c}{\PYGZsh{} scale values to be positive}
\PYG{g+gp}{\PYGZgt{}\PYGZgt{}\PYGZgt{} }\PYG{n}{count}\PYG{p}{,} \PYG{n}{bins}\PYG{p}{,} \PYG{n}{ignored} \PYG{o}{=} \PYG{n}{plt}\PYG{o}{.}\PYG{n}{hist}\PYG{p}{(}\PYG{n}{b}\PYG{p}{,} \PYG{l+m+mi}{100}\PYG{p}{,} \PYG{n}{normed}\PYG{o}{=}\PYG{n+nb+bp}{True}\PYG{p}{,} \PYG{n}{align}\PYG{o}{=}\PYG{l+s}{\PYGZsq{}}\PYG{l+s}{center}\PYG{l+s}{\PYGZsq{}}\PYG{p}{)}
\PYG{g+gp}{\PYGZgt{}\PYGZgt{}\PYGZgt{} }\PYG{n}{sigma} \PYG{o}{=} \PYG{n}{np}\PYG{o}{.}\PYG{n}{std}\PYG{p}{(}\PYG{n}{np}\PYG{o}{.}\PYG{n}{log}\PYG{p}{(}\PYG{n}{b}\PYG{p}{)}\PYG{p}{)}
\PYG{g+gp}{\PYGZgt{}\PYGZgt{}\PYGZgt{} }\PYG{n}{mu} \PYG{o}{=} \PYG{n}{np}\PYG{o}{.}\PYG{n}{mean}\PYG{p}{(}\PYG{n}{np}\PYG{o}{.}\PYG{n}{log}\PYG{p}{(}\PYG{n}{b}\PYG{p}{)}\PYG{p}{)}
\end{Verbatim}

\begin{Verbatim}[commandchars=\\\{\}]
\PYG{g+gp}{\PYGZgt{}\PYGZgt{}\PYGZgt{} }\PYG{n}{x} \PYG{o}{=} \PYG{n}{np}\PYG{o}{.}\PYG{n}{linspace}\PYG{p}{(}\PYG{n+nb}{min}\PYG{p}{(}\PYG{n}{bins}\PYG{p}{)}\PYG{p}{,} \PYG{n+nb}{max}\PYG{p}{(}\PYG{n}{bins}\PYG{p}{)}\PYG{p}{,} \PYG{l+m+mi}{10000}\PYG{p}{)}
\PYG{g+gp}{\PYGZgt{}\PYGZgt{}\PYGZgt{} }\PYG{n}{pdf} \PYG{o}{=} \PYG{p}{(}\PYG{n}{np}\PYG{o}{.}\PYG{n}{exp}\PYG{p}{(}\PYG{o}{\PYGZhy{}}\PYG{p}{(}\PYG{n}{np}\PYG{o}{.}\PYG{n}{log}\PYG{p}{(}\PYG{n}{x}\PYG{p}{)} \PYG{o}{\PYGZhy{}} \PYG{n}{mu}\PYG{p}{)}\PYG{o}{*}\PYG{o}{*}\PYG{l+m+mi}{2} \PYG{o}{/} \PYG{p}{(}\PYG{l+m+mi}{2} \PYG{o}{*} \PYG{n}{sigma}\PYG{o}{*}\PYG{o}{*}\PYG{l+m+mi}{2}\PYG{p}{)}\PYG{p}{)}
\PYG{g+gp}{... }       \PYG{o}{/} \PYG{p}{(}\PYG{n}{x} \PYG{o}{*} \PYG{n}{sigma} \PYG{o}{*} \PYG{n}{np}\PYG{o}{.}\PYG{n}{sqrt}\PYG{p}{(}\PYG{l+m+mi}{2} \PYG{o}{*} \PYG{n}{np}\PYG{o}{.}\PYG{n}{pi}\PYG{p}{)}\PYG{p}{)}\PYG{p}{)}
\end{Verbatim}

\begin{Verbatim}[commandchars=\\\{\}]
\PYG{g+gp}{\PYGZgt{}\PYGZgt{}\PYGZgt{} }\PYG{n}{plt}\PYG{o}{.}\PYG{n}{plot}\PYG{p}{(}\PYG{n}{x}\PYG{p}{,} \PYG{n}{pdf}\PYG{p}{,} \PYG{n}{color}\PYG{o}{=}\PYG{l+s}{\PYGZsq{}}\PYG{l+s}{r}\PYG{l+s}{\PYGZsq{}}\PYG{p}{,} \PYG{n}{linewidth}\PYG{o}{=}\PYG{l+m+mi}{2}\PYG{p}{)}
\PYG{g+gp}{\PYGZgt{}\PYGZgt{}\PYGZgt{} }\PYG{n}{plt}\PYG{o}{.}\PYG{n}{show}\PYG{p}{(}\PYG{p}{)}
\end{Verbatim}

\end{fulllineitems}

\index{logseries() (in module lib.graph.network)}

\begin{fulllineitems}
\phantomsection\label{lib.graph:lib.graph.network.logseries}\pysiglinewithargsret{\code{lib.graph.network.}\bfcode{logseries}}{\emph{p}, \emph{size=None}}{}
Draw samples from a Logarithmic Series distribution.

Samples are drawn from a Log Series distribution with specified
parameter, p (probability, 0 \textless{} p \textless{} 1).

loc : float

scale : float \textgreater{} 0.
\begin{description}
\item[{size}] \leavevmode{[}\{tuple, int\}{]}
Output shape.  If the given shape is, e.g., \code{(m, n, k)}, then
\code{m * n * k} samples are drawn.

\end{description}
\begin{description}
\item[{samples}] \leavevmode{[}\{ndarray, scalar\}{]}
where the values are all integers in  {[}0, n{]}.

\end{description}
\begin{description}
\item[{scipy.stats.distributions.logser}] \leavevmode{[}probability density function,{]}
distribution or cumulative density function, etc.

\end{description}

The probability density for the Log Series distribution is
\begin{gather}
\begin{split}P(k) = \frac{-p^k}{k \ln(1-p)},\end{split}\notag
\end{gather}
where p = probability.

The Log Series distribution is frequently used to represent species
richness and occurrence, first proposed by Fisher, Corbet, and
Williams in 1943 {[}2{]}.  It may also be used to model the numbers of
occupants seen in cars {[}3{]}.

Draw samples from the distribution:

\begin{Verbatim}[commandchars=\\\{\}]
\PYG{g+gp}{\PYGZgt{}\PYGZgt{}\PYGZgt{} }\PYG{n}{a} \PYG{o}{=} \PYG{o}{.}\PYG{l+m+mi}{6}
\PYG{g+gp}{\PYGZgt{}\PYGZgt{}\PYGZgt{} }\PYG{n}{s} \PYG{o}{=} \PYG{n}{np}\PYG{o}{.}\PYG{n}{random}\PYG{o}{.}\PYG{n}{logseries}\PYG{p}{(}\PYG{n}{a}\PYG{p}{,} \PYG{l+m+mi}{10000}\PYG{p}{)}
\PYG{g+gp}{\PYGZgt{}\PYGZgt{}\PYGZgt{} }\PYG{n}{count}\PYG{p}{,} \PYG{n}{bins}\PYG{p}{,} \PYG{n}{ignored} \PYG{o}{=} \PYG{n}{plt}\PYG{o}{.}\PYG{n}{hist}\PYG{p}{(}\PYG{n}{s}\PYG{p}{)}
\end{Verbatim}

\#   plot against distribution

\begin{Verbatim}[commandchars=\\\{\}]
\PYG{g+gp}{\PYGZgt{}\PYGZgt{}\PYGZgt{} }\PYG{k}{def} \PYG{n+nf}{logseries}\PYG{p}{(}\PYG{n}{k}\PYG{p}{,} \PYG{n}{p}\PYG{p}{)}\PYG{p}{:}
\PYG{g+gp}{... }    \PYG{k}{return} \PYG{o}{\PYGZhy{}}\PYG{n}{p}\PYG{o}{*}\PYG{o}{*}\PYG{n}{k}\PYG{o}{/}\PYG{p}{(}\PYG{n}{k}\PYG{o}{*}\PYG{n}{log}\PYG{p}{(}\PYG{l+m+mi}{1}\PYG{o}{\PYGZhy{}}\PYG{n}{p}\PYG{p}{)}\PYG{p}{)}
\PYG{g+gp}{\PYGZgt{}\PYGZgt{}\PYGZgt{} }\PYG{n}{plt}\PYG{o}{.}\PYG{n}{plot}\PYG{p}{(}\PYG{n}{bins}\PYG{p}{,} \PYG{n}{logseries}\PYG{p}{(}\PYG{n}{bins}\PYG{p}{,} \PYG{n}{a}\PYG{p}{)}\PYG{o}{*}\PYG{n}{count}\PYG{o}{.}\PYG{n}{max}\PYG{p}{(}\PYG{p}{)}\PYG{o}{/}
\PYG{g+go}{             logseries(bins, a).max(), \PYGZsq{}r\PYGZsq{})}
\PYG{g+gp}{\PYGZgt{}\PYGZgt{}\PYGZgt{} }\PYG{n}{plt}\PYG{o}{.}\PYG{n}{show}\PYG{p}{(}\PYG{p}{)}
\end{Verbatim}

\end{fulllineitems}

\index{multinomial() (in module lib.graph.network)}

\begin{fulllineitems}
\phantomsection\label{lib.graph:lib.graph.network.multinomial}\pysiglinewithargsret{\code{lib.graph.network.}\bfcode{multinomial}}{\emph{n}, \emph{pvals}, \emph{size=None}}{}
Draw samples from a multinomial distribution.

The multinomial distribution is a multivariate generalisation of the
binomial distribution.  Take an experiment with one of \code{p}
possible outcomes.  An example of such an experiment is throwing a dice,
where the outcome can be 1 through 6.  Each sample drawn from the
distribution represents \emph{n} such experiments.  Its values,
\code{X\_i = {[}X\_0, X\_1, ..., X\_p{]}}, represent the number of times the outcome
was \code{i}.
\begin{description}
\item[{n}] \leavevmode{[}int{]}
Number of experiments.

\item[{pvals}] \leavevmode{[}sequence of floats, length p{]}
Probabilities of each of the \code{p} different outcomes.  These
should sum to 1 (however, the last element is always assumed to
account for the remaining probability, as long as
\code{sum(pvals{[}:-1{]}) \textless{}= 1)}.

\item[{size}] \leavevmode{[}tuple of ints{]}
Given a \emph{size} of \code{(M, N, K)}, then \code{M*N*K} samples are drawn,
and the output shape becomes \code{(M, N, K, p)}, since each sample
has shape \code{(p,)}.

\end{description}

Throw a dice 20 times:

\begin{Verbatim}[commandchars=\\\{\}]
\PYG{g+gp}{\PYGZgt{}\PYGZgt{}\PYGZgt{} }\PYG{n}{np}\PYG{o}{.}\PYG{n}{random}\PYG{o}{.}\PYG{n}{multinomial}\PYG{p}{(}\PYG{l+m+mi}{20}\PYG{p}{,} \PYG{p}{[}\PYG{l+m+mi}{1}\PYG{o}{/}\PYG{l+m+mf}{6.}\PYG{p}{]}\PYG{o}{*}\PYG{l+m+mi}{6}\PYG{p}{,} \PYG{n}{size}\PYG{o}{=}\PYG{l+m+mi}{1}\PYG{p}{)}
\PYG{g+go}{array([[4, 1, 7, 5, 2, 1]])}
\end{Verbatim}

It landed 4 times on 1, once on 2, etc.

Now, throw the dice 20 times, and 20 times again:

\begin{Verbatim}[commandchars=\\\{\}]
\PYG{g+gp}{\PYGZgt{}\PYGZgt{}\PYGZgt{} }\PYG{n}{np}\PYG{o}{.}\PYG{n}{random}\PYG{o}{.}\PYG{n}{multinomial}\PYG{p}{(}\PYG{l+m+mi}{20}\PYG{p}{,} \PYG{p}{[}\PYG{l+m+mi}{1}\PYG{o}{/}\PYG{l+m+mf}{6.}\PYG{p}{]}\PYG{o}{*}\PYG{l+m+mi}{6}\PYG{p}{,} \PYG{n}{size}\PYG{o}{=}\PYG{l+m+mi}{2}\PYG{p}{)}
\PYG{g+go}{array([[3, 4, 3, 3, 4, 3],}
\PYG{g+go}{       [2, 4, 3, 4, 0, 7]])}
\end{Verbatim}

For the first run, we threw 3 times 1, 4 times 2, etc.  For the second,
we threw 2 times 1, 4 times 2, etc.

A loaded dice is more likely to land on number 6:

\begin{Verbatim}[commandchars=\\\{\}]
\PYG{g+gp}{\PYGZgt{}\PYGZgt{}\PYGZgt{} }\PYG{n}{np}\PYG{o}{.}\PYG{n}{random}\PYG{o}{.}\PYG{n}{multinomial}\PYG{p}{(}\PYG{l+m+mi}{100}\PYG{p}{,} \PYG{p}{[}\PYG{l+m+mi}{1}\PYG{o}{/}\PYG{l+m+mf}{7.}\PYG{p}{]}\PYG{o}{*}\PYG{l+m+mi}{5}\PYG{p}{)}
\PYG{g+go}{array([13, 16, 13, 16, 42])}
\end{Verbatim}

\end{fulllineitems}

\index{multivariate\_normal() (in module lib.graph.network)}

\begin{fulllineitems}
\phantomsection\label{lib.graph:lib.graph.network.multivariate_normal}\pysiglinewithargsret{\code{lib.graph.network.}\bfcode{multivariate\_normal}}{\emph{mean}, \emph{cov}\optional{, \emph{size}}}{}
Draw random samples from a multivariate normal distribution.

The multivariate normal, multinormal or Gaussian distribution is a
generalization of the one-dimensional normal distribution to higher
dimensions.  Such a distribution is specified by its mean and
covariance matrix.  These parameters are analogous to the mean
(average or ``center'') and variance (standard deviation, or ``width,''
squared) of the one-dimensional normal distribution.
\begin{description}
\item[{mean}] \leavevmode{[}1-D array\_like, of length N{]}
Mean of the N-dimensional distribution.

\item[{cov}] \leavevmode{[}2-D array\_like, of shape (N, N){]}
Covariance matrix of the distribution.  Must be symmetric and
positive semi-definite for ``physically meaningful'' results.

\item[{size}] \leavevmode{[}int or tuple of ints, optional{]}
Given a shape of, for example, \code{(m,n,k)}, \code{m*n*k} samples are
generated, and packed in an \emph{m}-by-\emph{n}-by-\emph{k} arrangement.  Because
each sample is \emph{N}-dimensional, the output shape is \code{(m,n,k,N)}.
If no shape is specified, a single (\emph{N}-D) sample is returned.

\end{description}
\begin{description}
\item[{out}] \leavevmode{[}ndarray{]}
The drawn samples, of shape \emph{size}, if that was provided.  If not,
the shape is \code{(N,)}.

In other words, each entry \code{out{[}i,j,...,:{]}} is an N-dimensional
value drawn from the distribution.

\end{description}

The mean is a coordinate in N-dimensional space, which represents the
location where samples are most likely to be generated.  This is
analogous to the peak of the bell curve for the one-dimensional or
univariate normal distribution.

Covariance indicates the level to which two variables vary together.
From the multivariate normal distribution, we draw N-dimensional
samples, \(X = [x_1, x_2, ... x_N]\).  The covariance matrix
element \(C_{ij}\) is the covariance of \(x_i\) and \(x_j\).
The element \(C_{ii}\) is the variance of \(x_i\) (i.e. its
``spread'').

Instead of specifying the full covariance matrix, popular
approximations include:
\begin{itemize}
\item {} 
Spherical covariance (\emph{cov} is a multiple of the identity matrix)

\item {} 
Diagonal covariance (\emph{cov} has non-negative elements, and only on
the diagonal)

\end{itemize}

This geometrical property can be seen in two dimensions by plotting
generated data-points:

\begin{Verbatim}[commandchars=\\\{\}]
\PYG{g+gp}{\PYGZgt{}\PYGZgt{}\PYGZgt{} }\PYG{n}{mean} \PYG{o}{=} \PYG{p}{[}\PYG{l+m+mi}{0}\PYG{p}{,}\PYG{l+m+mi}{0}\PYG{p}{]}
\PYG{g+gp}{\PYGZgt{}\PYGZgt{}\PYGZgt{} }\PYG{n}{cov} \PYG{o}{=} \PYG{p}{[}\PYG{p}{[}\PYG{l+m+mi}{1}\PYG{p}{,}\PYG{l+m+mi}{0}\PYG{p}{]}\PYG{p}{,}\PYG{p}{[}\PYG{l+m+mi}{0}\PYG{p}{,}\PYG{l+m+mi}{100}\PYG{p}{]}\PYG{p}{]} \PYG{c}{\PYGZsh{} diagonal covariance, points lie on x or y\PYGZhy{}axis}
\end{Verbatim}

\begin{Verbatim}[commandchars=\\\{\}]
\PYG{g+gp}{\PYGZgt{}\PYGZgt{}\PYGZgt{} }\PYG{k+kn}{import} \PYG{n+nn}{matplotlib.pyplot} \PYG{k+kn}{as} \PYG{n+nn}{plt}
\PYG{g+gp}{\PYGZgt{}\PYGZgt{}\PYGZgt{} }\PYG{n}{x}\PYG{p}{,}\PYG{n}{y} \PYG{o}{=} \PYG{n}{np}\PYG{o}{.}\PYG{n}{random}\PYG{o}{.}\PYG{n}{multivariate\PYGZus{}normal}\PYG{p}{(}\PYG{n}{mean}\PYG{p}{,}\PYG{n}{cov}\PYG{p}{,}\PYG{l+m+mi}{5000}\PYG{p}{)}\PYG{o}{.}\PYG{n}{T}
\PYG{g+gp}{\PYGZgt{}\PYGZgt{}\PYGZgt{} }\PYG{n}{plt}\PYG{o}{.}\PYG{n}{plot}\PYG{p}{(}\PYG{n}{x}\PYG{p}{,}\PYG{n}{y}\PYG{p}{,}\PYG{l+s}{\PYGZsq{}}\PYG{l+s}{x}\PYG{l+s}{\PYGZsq{}}\PYG{p}{)}\PYG{p}{;} \PYG{n}{plt}\PYG{o}{.}\PYG{n}{axis}\PYG{p}{(}\PYG{l+s}{\PYGZsq{}}\PYG{l+s}{equal}\PYG{l+s}{\PYGZsq{}}\PYG{p}{)}\PYG{p}{;} \PYG{n}{plt}\PYG{o}{.}\PYG{n}{show}\PYG{p}{(}\PYG{p}{)}
\end{Verbatim}

Note that the covariance matrix must be non-negative definite.

Papoulis, A., \emph{Probability, Random Variables, and Stochastic Processes},
3rd ed., New York: McGraw-Hill, 1991.

Duda, R. O., Hart, P. E., and Stork, D. G., \emph{Pattern Classification},
2nd ed., New York: Wiley, 2001.

\begin{Verbatim}[commandchars=\\\{\}]
\PYG{g+gp}{\PYGZgt{}\PYGZgt{}\PYGZgt{} }\PYG{n}{mean} \PYG{o}{=} \PYG{p}{(}\PYG{l+m+mi}{1}\PYG{p}{,}\PYG{l+m+mi}{2}\PYG{p}{)}
\PYG{g+gp}{\PYGZgt{}\PYGZgt{}\PYGZgt{} }\PYG{n}{cov} \PYG{o}{=} \PYG{p}{[}\PYG{p}{[}\PYG{l+m+mi}{1}\PYG{p}{,}\PYG{l+m+mi}{0}\PYG{p}{]}\PYG{p}{,}\PYG{p}{[}\PYG{l+m+mi}{1}\PYG{p}{,}\PYG{l+m+mi}{0}\PYG{p}{]}\PYG{p}{]}
\PYG{g+gp}{\PYGZgt{}\PYGZgt{}\PYGZgt{} }\PYG{n}{x} \PYG{o}{=} \PYG{n}{np}\PYG{o}{.}\PYG{n}{random}\PYG{o}{.}\PYG{n}{multivariate\PYGZus{}normal}\PYG{p}{(}\PYG{n}{mean}\PYG{p}{,}\PYG{n}{cov}\PYG{p}{,}\PYG{p}{(}\PYG{l+m+mi}{3}\PYG{p}{,}\PYG{l+m+mi}{3}\PYG{p}{)}\PYG{p}{)}
\PYG{g+gp}{\PYGZgt{}\PYGZgt{}\PYGZgt{} }\PYG{n}{x}\PYG{o}{.}\PYG{n}{shape}
\PYG{g+go}{(3, 3, 2)}
\end{Verbatim}

The following is probably true, given that 0.6 is roughly twice the
standard deviation:

\begin{Verbatim}[commandchars=\\\{\}]
\PYG{g+gp}{\PYGZgt{}\PYGZgt{}\PYGZgt{} }\PYG{k}{print} \PYG{n+nb}{list}\PYG{p}{(} \PYG{p}{(}\PYG{n}{x}\PYG{p}{[}\PYG{l+m+mi}{0}\PYG{p}{,}\PYG{l+m+mi}{0}\PYG{p}{,}\PYG{p}{:}\PYG{p}{]} \PYG{o}{\PYGZhy{}} \PYG{n}{mean}\PYG{p}{)} \PYG{o}{\PYGZlt{}} \PYG{l+m+mf}{0.6} \PYG{p}{)}
\PYG{g+go}{[True, True]}
\end{Verbatim}

\end{fulllineitems}

\index{negative\_binomial() (in module lib.graph.network)}

\begin{fulllineitems}
\phantomsection\label{lib.graph:lib.graph.network.negative_binomial}\pysiglinewithargsret{\code{lib.graph.network.}\bfcode{negative\_binomial}}{\emph{n}, \emph{p}, \emph{size=None}}{}
Draw samples from a negative\_binomial distribution.

Samples are drawn from a negative\_Binomial distribution with specified
parameters, \emph{n} trials and \emph{p} probability of success where \emph{n} is an
integer \textgreater{} 0 and \emph{p} is in the interval {[}0, 1{]}.
\begin{description}
\item[{n}] \leavevmode{[}int{]}
Parameter, \textgreater{} 0.

\item[{p}] \leavevmode{[}float{]}
Parameter, \textgreater{}= 0 and \textless{}=1.

\item[{size}] \leavevmode{[}int or tuple of ints{]}
Output shape. If the given shape is, e.g., \code{(m, n, k)}, then
\code{m * n * k} samples are drawn.

\end{description}
\begin{description}
\item[{samples}] \leavevmode{[}int or ndarray of ints{]}
Drawn samples.

\end{description}

The probability density for the Negative Binomial distribution is
\begin{gather}
\begin{split}P(N;n,p) = \binom{N+n-1}{n-1}p^{n}(1-p)^{N},\end{split}\notag
\end{gather}
where \(n-1\) is the number of successes, \(p\) is the probability
of success, and \(N+n-1\) is the number of trials.

The negative binomial distribution gives the probability of n-1 successes
and N failures in N+n-1 trials, and success on the (N+n)th trial.

If one throws a die repeatedly until the third time a ``1'' appears, then the
probability distribution of the number of non-``1''s that appear before the
third ``1'' is a negative binomial distribution.

Draw samples from the distribution:

A real world example. A company drills wild-cat oil exploration wells, each
with an estimated probability of success of 0.1.  What is the probability
of having one success for each successive well, that is what is the
probability of a single success after drilling 5 wells, after 6 wells,
etc.?

\begin{Verbatim}[commandchars=\\\{\}]
\PYG{g+gp}{\PYGZgt{}\PYGZgt{}\PYGZgt{} }\PYG{n}{s} \PYG{o}{=} \PYG{n}{np}\PYG{o}{.}\PYG{n}{random}\PYG{o}{.}\PYG{n}{negative\PYGZus{}binomial}\PYG{p}{(}\PYG{l+m+mi}{1}\PYG{p}{,} \PYG{l+m+mf}{0.1}\PYG{p}{,} \PYG{l+m+mi}{100000}\PYG{p}{)}
\PYG{g+gp}{\PYGZgt{}\PYGZgt{}\PYGZgt{} }\PYG{k}{for} \PYG{n}{i} \PYG{o+ow}{in} \PYG{n+nb}{range}\PYG{p}{(}\PYG{l+m+mi}{1}\PYG{p}{,} \PYG{l+m+mi}{11}\PYG{p}{)}\PYG{p}{:}
\PYG{g+gp}{... }   \PYG{n}{probability} \PYG{o}{=} \PYG{n+nb}{sum}\PYG{p}{(}\PYG{n}{s}\PYG{o}{\PYGZlt{}}\PYG{n}{i}\PYG{p}{)} \PYG{o}{/} \PYG{l+m+mf}{100000.}
\PYG{g+gp}{... }   \PYG{k}{print} \PYG{n}{i}\PYG{p}{,} \PYG{l+s}{\PYGZdq{}}\PYG{l+s}{wells drilled, probability of one success =}\PYG{l+s}{\PYGZdq{}}\PYG{p}{,} \PYG{n}{probability}
\end{Verbatim}

\end{fulllineitems}

\index{net\_analysis\_of\_dynamic\_graphs() (in module lib.graph.network)}

\begin{fulllineitems}
\phantomsection\label{lib.graph:lib.graph.network.net_analysis_of_dynamic_graphs}\pysiglinewithargsret{\code{lib.graph.network.}\bfcode{net\_analysis\_of\_dynamic\_graphs}}{\emph{fid\_dynRafRes}, \emph{tmpTime}, \emph{rcts}, \emph{cats}, \emph{foodList}, \emph{growth=False}, \emph{rctsALL=None}, \emph{catsALL=None}, \emph{completeRCTS=None}, \emph{debug=False}}{}
\end{fulllineitems}

\index{net\_analysis\_of\_static\_graphs() (in module lib.graph.network)}

\begin{fulllineitems}
\phantomsection\label{lib.graph:lib.graph.network.net_analysis_of_static_graphs}\pysiglinewithargsret{\code{lib.graph.network.}\bfcode{net\_analysis\_of\_static\_graphs}}{\emph{fid\_initRafRes}, \emph{fid\_initRafResALL}, \emph{fid\_initRafResLIST}, \emph{tmpDir}, \emph{rctProb}, \emph{avgCon}, \emph{rcts}, \emph{cats}, \emph{foodList}, \emph{maxDim}, \emph{debug=False}}{}
\end{fulllineitems}

\index{noncentral\_chisquare() (in module lib.graph.network)}

\begin{fulllineitems}
\phantomsection\label{lib.graph:lib.graph.network.noncentral_chisquare}\pysiglinewithargsret{\code{lib.graph.network.}\bfcode{noncentral\_chisquare}}{\emph{df}, \emph{nonc}, \emph{size=None}}{}
Draw samples from a noncentral chi-square distribution.

The noncentral \(\chi^2\) distribution is a generalisation of
the \(\chi^2\) distribution.
\begin{description}
\item[{df}] \leavevmode{[}int{]}
Degrees of freedom, should be \textgreater{}= 1.

\item[{nonc}] \leavevmode{[}float{]}
Non-centrality, should be \textgreater{} 0.

\item[{size}] \leavevmode{[}int or tuple of ints{]}
Shape of the output.

\end{description}

The probability density function for the noncentral Chi-square distribution
is
\begin{gather}
\begin{split}P(x;df,nonc) = \sum^{\infty}_{i=0}
\frac{e^{-nonc/2}(nonc/2)^{i}}{i!}P_{Y_{df+2i}}(x),\end{split}\notag
\end{gather}
where \(Y_{q}\) is the Chi-square with q degrees of freedom.

In Delhi (2007), it is noted that the noncentral chi-square is useful in
bombing and coverage problems, the probability of killing the point target
given by the noncentral chi-squared distribution.

Draw values from the distribution and plot the histogram

\begin{Verbatim}[commandchars=\\\{\}]
\PYG{g+gp}{\PYGZgt{}\PYGZgt{}\PYGZgt{} }\PYG{k+kn}{import} \PYG{n+nn}{matplotlib.pyplot} \PYG{k+kn}{as} \PYG{n+nn}{plt}
\PYG{g+gp}{\PYGZgt{}\PYGZgt{}\PYGZgt{} }\PYG{n}{values} \PYG{o}{=} \PYG{n}{plt}\PYG{o}{.}\PYG{n}{hist}\PYG{p}{(}\PYG{n}{np}\PYG{o}{.}\PYG{n}{random}\PYG{o}{.}\PYG{n}{noncentral\PYGZus{}chisquare}\PYG{p}{(}\PYG{l+m+mi}{3}\PYG{p}{,} \PYG{l+m+mi}{20}\PYG{p}{,} \PYG{l+m+mi}{100000}\PYG{p}{)}\PYG{p}{,}
\PYG{g+gp}{... }                  \PYG{n}{bins}\PYG{o}{=}\PYG{l+m+mi}{200}\PYG{p}{,} \PYG{n}{normed}\PYG{o}{=}\PYG{n+nb+bp}{True}\PYG{p}{)}
\PYG{g+gp}{\PYGZgt{}\PYGZgt{}\PYGZgt{} }\PYG{n}{plt}\PYG{o}{.}\PYG{n}{show}\PYG{p}{(}\PYG{p}{)}
\end{Verbatim}

Draw values from a noncentral chisquare with very small noncentrality,
and compare to a chisquare.

\begin{Verbatim}[commandchars=\\\{\}]
\PYG{g+gp}{\PYGZgt{}\PYGZgt{}\PYGZgt{} }\PYG{n}{plt}\PYG{o}{.}\PYG{n}{figure}\PYG{p}{(}\PYG{p}{)}
\PYG{g+gp}{\PYGZgt{}\PYGZgt{}\PYGZgt{} }\PYG{n}{values} \PYG{o}{=} \PYG{n}{plt}\PYG{o}{.}\PYG{n}{hist}\PYG{p}{(}\PYG{n}{np}\PYG{o}{.}\PYG{n}{random}\PYG{o}{.}\PYG{n}{noncentral\PYGZus{}chisquare}\PYG{p}{(}\PYG{l+m+mi}{3}\PYG{p}{,} \PYG{o}{.}\PYG{l+m+mo}{0000001}\PYG{p}{,} \PYG{l+m+mi}{100000}\PYG{p}{)}\PYG{p}{,}
\PYG{g+gp}{... }                  \PYG{n}{bins}\PYG{o}{=}\PYG{n}{np}\PYG{o}{.}\PYG{n}{arange}\PYG{p}{(}\PYG{l+m+mf}{0.}\PYG{p}{,} \PYG{l+m+mi}{25}\PYG{p}{,} \PYG{o}{.}\PYG{l+m+mi}{1}\PYG{p}{)}\PYG{p}{,} \PYG{n}{normed}\PYG{o}{=}\PYG{n+nb+bp}{True}\PYG{p}{)}
\PYG{g+gp}{\PYGZgt{}\PYGZgt{}\PYGZgt{} }\PYG{n}{values2} \PYG{o}{=} \PYG{n}{plt}\PYG{o}{.}\PYG{n}{hist}\PYG{p}{(}\PYG{n}{np}\PYG{o}{.}\PYG{n}{random}\PYG{o}{.}\PYG{n}{chisquare}\PYG{p}{(}\PYG{l+m+mi}{3}\PYG{p}{,} \PYG{l+m+mi}{100000}\PYG{p}{)}\PYG{p}{,}
\PYG{g+gp}{... }                   \PYG{n}{bins}\PYG{o}{=}\PYG{n}{np}\PYG{o}{.}\PYG{n}{arange}\PYG{p}{(}\PYG{l+m+mf}{0.}\PYG{p}{,} \PYG{l+m+mi}{25}\PYG{p}{,} \PYG{o}{.}\PYG{l+m+mi}{1}\PYG{p}{)}\PYG{p}{,} \PYG{n}{normed}\PYG{o}{=}\PYG{n+nb+bp}{True}\PYG{p}{)}
\PYG{g+gp}{\PYGZgt{}\PYGZgt{}\PYGZgt{} }\PYG{n}{plt}\PYG{o}{.}\PYG{n}{plot}\PYG{p}{(}\PYG{n}{values}\PYG{p}{[}\PYG{l+m+mi}{1}\PYG{p}{]}\PYG{p}{[}\PYG{l+m+mi}{0}\PYG{p}{:}\PYG{o}{\PYGZhy{}}\PYG{l+m+mi}{1}\PYG{p}{]}\PYG{p}{,} \PYG{n}{values}\PYG{p}{[}\PYG{l+m+mi}{0}\PYG{p}{]}\PYG{o}{\PYGZhy{}}\PYG{n}{values2}\PYG{p}{[}\PYG{l+m+mi}{0}\PYG{p}{]}\PYG{p}{,} \PYG{l+s}{\PYGZsq{}}\PYG{l+s}{ob}\PYG{l+s}{\PYGZsq{}}\PYG{p}{)}
\PYG{g+gp}{\PYGZgt{}\PYGZgt{}\PYGZgt{} }\PYG{n}{plt}\PYG{o}{.}\PYG{n}{show}\PYG{p}{(}\PYG{p}{)}
\end{Verbatim}

Demonstrate how large values of non-centrality lead to a more symmetric
distribution.

\begin{Verbatim}[commandchars=\\\{\}]
\PYG{g+gp}{\PYGZgt{}\PYGZgt{}\PYGZgt{} }\PYG{n}{plt}\PYG{o}{.}\PYG{n}{figure}\PYG{p}{(}\PYG{p}{)}
\PYG{g+gp}{\PYGZgt{}\PYGZgt{}\PYGZgt{} }\PYG{n}{values} \PYG{o}{=} \PYG{n}{plt}\PYG{o}{.}\PYG{n}{hist}\PYG{p}{(}\PYG{n}{np}\PYG{o}{.}\PYG{n}{random}\PYG{o}{.}\PYG{n}{noncentral\PYGZus{}chisquare}\PYG{p}{(}\PYG{l+m+mi}{3}\PYG{p}{,} \PYG{l+m+mi}{20}\PYG{p}{,} \PYG{l+m+mi}{100000}\PYG{p}{)}\PYG{p}{,}
\PYG{g+gp}{... }                  \PYG{n}{bins}\PYG{o}{=}\PYG{l+m+mi}{200}\PYG{p}{,} \PYG{n}{normed}\PYG{o}{=}\PYG{n+nb+bp}{True}\PYG{p}{)}
\PYG{g+gp}{\PYGZgt{}\PYGZgt{}\PYGZgt{} }\PYG{n}{plt}\PYG{o}{.}\PYG{n}{show}\PYG{p}{(}\PYG{p}{)}
\end{Verbatim}

\end{fulllineitems}

\index{noncentral\_f() (in module lib.graph.network)}

\begin{fulllineitems}
\phantomsection\label{lib.graph:lib.graph.network.noncentral_f}\pysiglinewithargsret{\code{lib.graph.network.}\bfcode{noncentral\_f}}{\emph{dfnum}, \emph{dfden}, \emph{nonc}, \emph{size=None}}{}
Draw samples from the noncentral F distribution.

Samples are drawn from an F distribution with specified parameters,
\emph{dfnum} (degrees of freedom in numerator) and \emph{dfden} (degrees of
freedom in denominator), where both parameters \textgreater{} 1.
\emph{nonc} is the non-centrality parameter.
\begin{description}
\item[{dfnum}] \leavevmode{[}int{]}
Parameter, should be \textgreater{} 1.

\item[{dfden}] \leavevmode{[}int{]}
Parameter, should be \textgreater{} 1.

\item[{nonc}] \leavevmode{[}float{]}
Parameter, should be \textgreater{}= 0.

\item[{size}] \leavevmode{[}int or tuple of ints{]}
Output shape. If the given shape is, e.g., \code{(m, n, k)}, then
\code{m * n * k} samples are drawn.

\end{description}
\begin{description}
\item[{samples}] \leavevmode{[}scalar or ndarray{]}
Drawn samples.

\end{description}

When calculating the power of an experiment (power = probability of
rejecting the null hypothesis when a specific alternative is true) the
non-central F statistic becomes important.  When the null hypothesis is
true, the F statistic follows a central F distribution. When the null
hypothesis is not true, then it follows a non-central F statistic.

Weisstein, Eric W. ``Noncentral F-Distribution.'' From MathWorld--A Wolfram
Web Resource.  \href{http://mathworld.wolfram.com/NoncentralF-Distribution.html}{http://mathworld.wolfram.com/NoncentralF-Distribution.html}

Wikipedia, ``Noncentral F distribution'',
\href{http://en.wikipedia.org/wiki/Noncentral\_F-distribution}{http://en.wikipedia.org/wiki/Noncentral\_F-distribution}

In a study, testing for a specific alternative to the null hypothesis
requires use of the Noncentral F distribution. We need to calculate the
area in the tail of the distribution that exceeds the value of the F
distribution for the null hypothesis.  We'll plot the two probability
distributions for comparison.

\begin{Verbatim}[commandchars=\\\{\}]
\PYG{g+gp}{\PYGZgt{}\PYGZgt{}\PYGZgt{} }\PYG{n}{dfnum} \PYG{o}{=} \PYG{l+m+mi}{3} \PYG{c}{\PYGZsh{} between group deg of freedom}
\PYG{g+gp}{\PYGZgt{}\PYGZgt{}\PYGZgt{} }\PYG{n}{dfden} \PYG{o}{=} \PYG{l+m+mi}{20} \PYG{c}{\PYGZsh{} within groups degrees of freedom}
\PYG{g+gp}{\PYGZgt{}\PYGZgt{}\PYGZgt{} }\PYG{n}{nonc} \PYG{o}{=} \PYG{l+m+mf}{3.0}
\PYG{g+gp}{\PYGZgt{}\PYGZgt{}\PYGZgt{} }\PYG{n}{nc\PYGZus{}vals} \PYG{o}{=} \PYG{n}{np}\PYG{o}{.}\PYG{n}{random}\PYG{o}{.}\PYG{n}{noncentral\PYGZus{}f}\PYG{p}{(}\PYG{n}{dfnum}\PYG{p}{,} \PYG{n}{dfden}\PYG{p}{,} \PYG{n}{nonc}\PYG{p}{,} \PYG{l+m+mi}{1000000}\PYG{p}{)}
\PYG{g+gp}{\PYGZgt{}\PYGZgt{}\PYGZgt{} }\PYG{n}{NF} \PYG{o}{=} \PYG{n}{np}\PYG{o}{.}\PYG{n}{histogram}\PYG{p}{(}\PYG{n}{nc\PYGZus{}vals}\PYG{p}{,} \PYG{n}{bins}\PYG{o}{=}\PYG{l+m+mi}{50}\PYG{p}{,} \PYG{n}{normed}\PYG{o}{=}\PYG{n+nb+bp}{True}\PYG{p}{)}
\PYG{g+gp}{\PYGZgt{}\PYGZgt{}\PYGZgt{} }\PYG{n}{c\PYGZus{}vals} \PYG{o}{=} \PYG{n}{np}\PYG{o}{.}\PYG{n}{random}\PYG{o}{.}\PYG{n}{f}\PYG{p}{(}\PYG{n}{dfnum}\PYG{p}{,} \PYG{n}{dfden}\PYG{p}{,} \PYG{l+m+mi}{1000000}\PYG{p}{)}
\PYG{g+gp}{\PYGZgt{}\PYGZgt{}\PYGZgt{} }\PYG{n}{F} \PYG{o}{=} \PYG{n}{np}\PYG{o}{.}\PYG{n}{histogram}\PYG{p}{(}\PYG{n}{c\PYGZus{}vals}\PYG{p}{,} \PYG{n}{bins}\PYG{o}{=}\PYG{l+m+mi}{50}\PYG{p}{,} \PYG{n}{normed}\PYG{o}{=}\PYG{n+nb+bp}{True}\PYG{p}{)}
\PYG{g+gp}{\PYGZgt{}\PYGZgt{}\PYGZgt{} }\PYG{n}{plt}\PYG{o}{.}\PYG{n}{plot}\PYG{p}{(}\PYG{n}{F}\PYG{p}{[}\PYG{l+m+mi}{1}\PYG{p}{]}\PYG{p}{[}\PYG{l+m+mi}{1}\PYG{p}{:}\PYG{p}{]}\PYG{p}{,} \PYG{n}{F}\PYG{p}{[}\PYG{l+m+mi}{0}\PYG{p}{]}\PYG{p}{)}
\PYG{g+gp}{\PYGZgt{}\PYGZgt{}\PYGZgt{} }\PYG{n}{plt}\PYG{o}{.}\PYG{n}{plot}\PYG{p}{(}\PYG{n}{NF}\PYG{p}{[}\PYG{l+m+mi}{1}\PYG{p}{]}\PYG{p}{[}\PYG{l+m+mi}{1}\PYG{p}{:}\PYG{p}{]}\PYG{p}{,} \PYG{n}{NF}\PYG{p}{[}\PYG{l+m+mi}{0}\PYG{p}{]}\PYG{p}{)}
\PYG{g+gp}{\PYGZgt{}\PYGZgt{}\PYGZgt{} }\PYG{n}{plt}\PYG{o}{.}\PYG{n}{show}\PYG{p}{(}\PYG{p}{)}
\end{Verbatim}

\end{fulllineitems}

\index{normal() (in module lib.graph.network)}

\begin{fulllineitems}
\phantomsection\label{lib.graph:lib.graph.network.normal}\pysiglinewithargsret{\code{lib.graph.network.}\bfcode{normal}}{\emph{loc=0.0}, \emph{scale=1.0}, \emph{size=None}}{}
Draw random samples from a normal (Gaussian) distribution.

The probability density function of the normal distribution, first
derived by De Moivre and 200 years later by both Gauss and Laplace
independently {\color{red}\bfseries{}{[}2{]}\_}, is often called the bell curve because of
its characteristic shape (see the example below).

The normal distributions occurs often in nature.  For example, it
describes the commonly occurring distribution of samples influenced
by a large number of tiny, random disturbances, each with its own
unique distribution {\color{red}\bfseries{}{[}2{]}\_}.
\begin{description}
\item[{loc}] \leavevmode{[}float{]}
Mean (``centre'') of the distribution.

\item[{scale}] \leavevmode{[}float{]}
Standard deviation (spread or ``width'') of the distribution.

\item[{size}] \leavevmode{[}tuple of ints{]}
Output shape.  If the given shape is, e.g., \code{(m, n, k)}, then
\code{m * n * k} samples are drawn.

\end{description}
\begin{description}
\item[{scipy.stats.distributions.norm}] \leavevmode{[}probability density function,{]}
distribution or cumulative density function, etc.

\end{description}

The probability density for the Gaussian distribution is
\begin{gather}
\begin{split}p(x) = \frac{1}{\sqrt{ 2 \pi \sigma^2 }}
e^{ - \frac{ (x - \mu)^2 } {2 \sigma^2} },\end{split}\notag
\end{gather}
where \(\mu\) is the mean and \(\sigma\) the standard deviation.
The square of the standard deviation, \(\sigma^2\), is called the
variance.

The function has its peak at the mean, and its ``spread'' increases with
the standard deviation (the function reaches 0.607 times its maximum at
\(x + \sigma\) and \(x - \sigma\) {\color{red}\bfseries{}{[}2{]}\_}).  This implies that
\emph{numpy.random.normal} is more likely to return samples lying close to the
mean, rather than those far away.

Draw samples from the distribution:

\begin{Verbatim}[commandchars=\\\{\}]
\PYG{g+gp}{\PYGZgt{}\PYGZgt{}\PYGZgt{} }\PYG{n}{mu}\PYG{p}{,} \PYG{n}{sigma} \PYG{o}{=} \PYG{l+m+mi}{0}\PYG{p}{,} \PYG{l+m+mf}{0.1} \PYG{c}{\PYGZsh{} mean and standard deviation}
\PYG{g+gp}{\PYGZgt{}\PYGZgt{}\PYGZgt{} }\PYG{n}{s} \PYG{o}{=} \PYG{n}{np}\PYG{o}{.}\PYG{n}{random}\PYG{o}{.}\PYG{n}{normal}\PYG{p}{(}\PYG{n}{mu}\PYG{p}{,} \PYG{n}{sigma}\PYG{p}{,} \PYG{l+m+mi}{1000}\PYG{p}{)}
\end{Verbatim}

Verify the mean and the variance:

\begin{Verbatim}[commandchars=\\\{\}]
\PYG{g+gp}{\PYGZgt{}\PYGZgt{}\PYGZgt{} }\PYG{n+nb}{abs}\PYG{p}{(}\PYG{n}{mu} \PYG{o}{\PYGZhy{}} \PYG{n}{np}\PYG{o}{.}\PYG{n}{mean}\PYG{p}{(}\PYG{n}{s}\PYG{p}{)}\PYG{p}{)} \PYG{o}{\PYGZlt{}} \PYG{l+m+mf}{0.01}
\PYG{g+go}{True}
\end{Verbatim}

\begin{Verbatim}[commandchars=\\\{\}]
\PYG{g+gp}{\PYGZgt{}\PYGZgt{}\PYGZgt{} }\PYG{n+nb}{abs}\PYG{p}{(}\PYG{n}{sigma} \PYG{o}{\PYGZhy{}} \PYG{n}{np}\PYG{o}{.}\PYG{n}{std}\PYG{p}{(}\PYG{n}{s}\PYG{p}{,} \PYG{n}{ddof}\PYG{o}{=}\PYG{l+m+mi}{1}\PYG{p}{)}\PYG{p}{)} \PYG{o}{\PYGZlt{}} \PYG{l+m+mf}{0.01}
\PYG{g+go}{True}
\end{Verbatim}

Display the histogram of the samples, along with
the probability density function:

\begin{Verbatim}[commandchars=\\\{\}]
\PYG{g+gp}{\PYGZgt{}\PYGZgt{}\PYGZgt{} }\PYG{k+kn}{import} \PYG{n+nn}{matplotlib.pyplot} \PYG{k+kn}{as} \PYG{n+nn}{plt}
\PYG{g+gp}{\PYGZgt{}\PYGZgt{}\PYGZgt{} }\PYG{n}{count}\PYG{p}{,} \PYG{n}{bins}\PYG{p}{,} \PYG{n}{ignored} \PYG{o}{=} \PYG{n}{plt}\PYG{o}{.}\PYG{n}{hist}\PYG{p}{(}\PYG{n}{s}\PYG{p}{,} \PYG{l+m+mi}{30}\PYG{p}{,} \PYG{n}{normed}\PYG{o}{=}\PYG{n+nb+bp}{True}\PYG{p}{)}
\PYG{g+gp}{\PYGZgt{}\PYGZgt{}\PYGZgt{} }\PYG{n}{plt}\PYG{o}{.}\PYG{n}{plot}\PYG{p}{(}\PYG{n}{bins}\PYG{p}{,} \PYG{l+m+mi}{1}\PYG{o}{/}\PYG{p}{(}\PYG{n}{sigma} \PYG{o}{*} \PYG{n}{np}\PYG{o}{.}\PYG{n}{sqrt}\PYG{p}{(}\PYG{l+m+mi}{2} \PYG{o}{*} \PYG{n}{np}\PYG{o}{.}\PYG{n}{pi}\PYG{p}{)}\PYG{p}{)} \PYG{o}{*}
\PYG{g+gp}{... }               \PYG{n}{np}\PYG{o}{.}\PYG{n}{exp}\PYG{p}{(} \PYG{o}{\PYGZhy{}} \PYG{p}{(}\PYG{n}{bins} \PYG{o}{\PYGZhy{}} \PYG{n}{mu}\PYG{p}{)}\PYG{o}{*}\PYG{o}{*}\PYG{l+m+mi}{2} \PYG{o}{/} \PYG{p}{(}\PYG{l+m+mi}{2} \PYG{o}{*} \PYG{n}{sigma}\PYG{o}{*}\PYG{o}{*}\PYG{l+m+mi}{2}\PYG{p}{)} \PYG{p}{)}\PYG{p}{,}
\PYG{g+gp}{... }         \PYG{n}{linewidth}\PYG{o}{=}\PYG{l+m+mi}{2}\PYG{p}{,} \PYG{n}{color}\PYG{o}{=}\PYG{l+s}{\PYGZsq{}}\PYG{l+s}{r}\PYG{l+s}{\PYGZsq{}}\PYG{p}{)}
\PYG{g+gp}{\PYGZgt{}\PYGZgt{}\PYGZgt{} }\PYG{n}{plt}\PYG{o}{.}\PYG{n}{show}\PYG{p}{(}\PYG{p}{)}
\end{Verbatim}

\end{fulllineitems}

\index{pareto() (in module lib.graph.network)}

\begin{fulllineitems}
\phantomsection\label{lib.graph:lib.graph.network.pareto}\pysiglinewithargsret{\code{lib.graph.network.}\bfcode{pareto}}{\emph{a}, \emph{size=None}}{}
Draw samples from a Pareto II or Lomax distribution with specified shape.

The Lomax or Pareto II distribution is a shifted Pareto distribution. The
classical Pareto distribution can be obtained from the Lomax distribution
by adding the location parameter m, see below. The smallest value of the
Lomax distribution is zero while for the classical Pareto distribution it
is m, where the standard Pareto distribution has location m=1.
Lomax can also be considered as a simplified version of the Generalized
Pareto distribution (available in SciPy), with the scale set to one and
the location set to zero.

The Pareto distribution must be greater than zero, and is unbounded above.
It is also known as the ``80-20 rule''.  In this distribution, 80 percent of
the weights are in the lowest 20 percent of the range, while the other 20
percent fill the remaining 80 percent of the range.
\begin{description}
\item[{shape}] \leavevmode{[}float, \textgreater{} 0.{]}
Shape of the distribution.

\item[{size}] \leavevmode{[}tuple of ints{]}
Output shape.  If the given shape is, e.g., \code{(m, n, k)}, then
\code{m * n * k} samples are drawn.

\end{description}
\begin{description}
\item[{scipy.stats.distributions.lomax.pdf}] \leavevmode{[}probability density function,{]}
distribution or cumulative density function, etc.

\item[{scipy.stats.distributions.genpareto.pdf}] \leavevmode{[}probability density function,{]}
distribution or cumulative density function, etc.

\end{description}

The probability density for the Pareto distribution is
\begin{gather}
\begin{split}p(x) = \frac{am^a}{x^{a+1}}\end{split}\notag
\end{gather}
where \(a\) is the shape and \(m\) the location

The Pareto distribution, named after the Italian economist Vilfredo Pareto,
is a power law probability distribution useful in many real world problems.
Outside the field of economics it is generally referred to as the Bradford
distribution. Pareto developed the distribution to describe the
distribution of wealth in an economy.  It has also found use in insurance,
web page access statistics, oil field sizes, and many other problems,
including the download frequency for projects in Sourceforge {[}1{]}.  It is
one of the so-called ``fat-tailed'' distributions.

Draw samples from the distribution:

\begin{Verbatim}[commandchars=\\\{\}]
\PYG{g+gp}{\PYGZgt{}\PYGZgt{}\PYGZgt{} }\PYG{n}{a}\PYG{p}{,} \PYG{n}{m} \PYG{o}{=} \PYG{l+m+mf}{3.}\PYG{p}{,} \PYG{l+m+mf}{1.} \PYG{c}{\PYGZsh{} shape and mode}
\PYG{g+gp}{\PYGZgt{}\PYGZgt{}\PYGZgt{} }\PYG{n}{s} \PYG{o}{=} \PYG{n}{np}\PYG{o}{.}\PYG{n}{random}\PYG{o}{.}\PYG{n}{pareto}\PYG{p}{(}\PYG{n}{a}\PYG{p}{,} \PYG{l+m+mi}{1000}\PYG{p}{)} \PYG{o}{+} \PYG{n}{m}
\end{Verbatim}

Display the histogram of the samples, along with
the probability density function:

\begin{Verbatim}[commandchars=\\\{\}]
\PYG{g+gp}{\PYGZgt{}\PYGZgt{}\PYGZgt{} }\PYG{k+kn}{import} \PYG{n+nn}{matplotlib.pyplot} \PYG{k+kn}{as} \PYG{n+nn}{plt}
\PYG{g+gp}{\PYGZgt{}\PYGZgt{}\PYGZgt{} }\PYG{n}{count}\PYG{p}{,} \PYG{n}{bins}\PYG{p}{,} \PYG{n}{ignored} \PYG{o}{=} \PYG{n}{plt}\PYG{o}{.}\PYG{n}{hist}\PYG{p}{(}\PYG{n}{s}\PYG{p}{,} \PYG{l+m+mi}{100}\PYG{p}{,} \PYG{n}{normed}\PYG{o}{=}\PYG{n+nb+bp}{True}\PYG{p}{,} \PYG{n}{align}\PYG{o}{=}\PYG{l+s}{\PYGZsq{}}\PYG{l+s}{center}\PYG{l+s}{\PYGZsq{}}\PYG{p}{)}
\PYG{g+gp}{\PYGZgt{}\PYGZgt{}\PYGZgt{} }\PYG{n}{fit} \PYG{o}{=} \PYG{n}{a}\PYG{o}{*}\PYG{n}{m}\PYG{o}{*}\PYG{o}{*}\PYG{n}{a}\PYG{o}{/}\PYG{n}{bins}\PYG{o}{*}\PYG{o}{*}\PYG{p}{(}\PYG{n}{a}\PYG{o}{+}\PYG{l+m+mi}{1}\PYG{p}{)}
\PYG{g+gp}{\PYGZgt{}\PYGZgt{}\PYGZgt{} }\PYG{n}{plt}\PYG{o}{.}\PYG{n}{plot}\PYG{p}{(}\PYG{n}{bins}\PYG{p}{,} \PYG{n+nb}{max}\PYG{p}{(}\PYG{n}{count}\PYG{p}{)}\PYG{o}{*}\PYG{n}{fit}\PYG{o}{/}\PYG{n+nb}{max}\PYG{p}{(}\PYG{n}{fit}\PYG{p}{)}\PYG{p}{,}\PYG{n}{linewidth}\PYG{o}{=}\PYG{l+m+mi}{2}\PYG{p}{,} \PYG{n}{color}\PYG{o}{=}\PYG{l+s}{\PYGZsq{}}\PYG{l+s}{r}\PYG{l+s}{\PYGZsq{}}\PYG{p}{)}
\PYG{g+gp}{\PYGZgt{}\PYGZgt{}\PYGZgt{} }\PYG{n}{plt}\PYG{o}{.}\PYG{n}{show}\PYG{p}{(}\PYG{p}{)}
\end{Verbatim}

\end{fulllineitems}

\index{permutation() (in module lib.graph.network)}

\begin{fulllineitems}
\phantomsection\label{lib.graph:lib.graph.network.permutation}\pysiglinewithargsret{\code{lib.graph.network.}\bfcode{permutation}}{\emph{x}}{}
Randomly permute a sequence, or return a permuted range.

If \emph{x} is a multi-dimensional array, it is only shuffled along its
first index.
\begin{description}
\item[{x}] \leavevmode{[}int or array\_like{]}
If \emph{x} is an integer, randomly permute \code{np.arange(x)}.
If \emph{x} is an array, make a copy and shuffle the elements
randomly.

\end{description}
\begin{description}
\item[{out}] \leavevmode{[}ndarray{]}
Permuted sequence or array range.

\end{description}

\begin{Verbatim}[commandchars=\\\{\}]
\PYG{g+gp}{\PYGZgt{}\PYGZgt{}\PYGZgt{} }\PYG{n}{np}\PYG{o}{.}\PYG{n}{random}\PYG{o}{.}\PYG{n}{permutation}\PYG{p}{(}\PYG{l+m+mi}{10}\PYG{p}{)}
\PYG{g+go}{array([1, 7, 4, 3, 0, 9, 2, 5, 8, 6])}
\end{Verbatim}

\begin{Verbatim}[commandchars=\\\{\}]
\PYG{g+gp}{\PYGZgt{}\PYGZgt{}\PYGZgt{} }\PYG{n}{np}\PYG{o}{.}\PYG{n}{random}\PYG{o}{.}\PYG{n}{permutation}\PYG{p}{(}\PYG{p}{[}\PYG{l+m+mi}{1}\PYG{p}{,} \PYG{l+m+mi}{4}\PYG{p}{,} \PYG{l+m+mi}{9}\PYG{p}{,} \PYG{l+m+mi}{12}\PYG{p}{,} \PYG{l+m+mi}{15}\PYG{p}{]}\PYG{p}{)}
\PYG{g+go}{array([15,  1,  9,  4, 12])}
\end{Verbatim}

\begin{Verbatim}[commandchars=\\\{\}]
\PYG{g+gp}{\PYGZgt{}\PYGZgt{}\PYGZgt{} }\PYG{n}{arr} \PYG{o}{=} \PYG{n}{np}\PYG{o}{.}\PYG{n}{arange}\PYG{p}{(}\PYG{l+m+mi}{9}\PYG{p}{)}\PYG{o}{.}\PYG{n}{reshape}\PYG{p}{(}\PYG{p}{(}\PYG{l+m+mi}{3}\PYG{p}{,} \PYG{l+m+mi}{3}\PYG{p}{)}\PYG{p}{)}
\PYG{g+gp}{\PYGZgt{}\PYGZgt{}\PYGZgt{} }\PYG{n}{np}\PYG{o}{.}\PYG{n}{random}\PYG{o}{.}\PYG{n}{permutation}\PYG{p}{(}\PYG{n}{arr}\PYG{p}{)}
\PYG{g+go}{array([[6, 7, 8],}
\PYG{g+go}{       [0, 1, 2],}
\PYG{g+go}{       [3, 4, 5]])}
\end{Verbatim}

\end{fulllineitems}

\index{poisson() (in module lib.graph.network)}

\begin{fulllineitems}
\phantomsection\label{lib.graph:lib.graph.network.poisson}\pysiglinewithargsret{\code{lib.graph.network.}\bfcode{poisson}}{\emph{lam=1.0}, \emph{size=None}}{}
Draw samples from a Poisson distribution.

The Poisson distribution is the limit of the Binomial
distribution for large N.
\begin{description}
\item[{lam}] \leavevmode{[}float{]}
Expectation of interval, should be \textgreater{}= 0.

\item[{size}] \leavevmode{[}int or tuple of ints, optional{]}
Output shape. If the given shape is, e.g., \code{(m, n, k)}, then
\code{m * n * k} samples are drawn.

\end{description}

The Poisson distribution
\begin{gather}
\begin{split}f(k; \lambda)=\frac{\lambda^k e^{-\lambda}}{k!}\end{split}\notag
\end{gather}
For events with an expected separation \(\lambda\) the Poisson
distribution \(f(k; \lambda)\) describes the probability of
\(k\) events occurring within the observed interval \(\lambda\).

Because the output is limited to the range of the C long type, a
ValueError is raised when \emph{lam} is within 10 sigma of the maximum
representable value.

Draw samples from the distribution:

\begin{Verbatim}[commandchars=\\\{\}]
\PYG{g+gp}{\PYGZgt{}\PYGZgt{}\PYGZgt{} }\PYG{k+kn}{import} \PYG{n+nn}{numpy} \PYG{k+kn}{as} \PYG{n+nn}{np}
\PYG{g+gp}{\PYGZgt{}\PYGZgt{}\PYGZgt{} }\PYG{n}{s} \PYG{o}{=} \PYG{n}{np}\PYG{o}{.}\PYG{n}{random}\PYG{o}{.}\PYG{n}{poisson}\PYG{p}{(}\PYG{l+m+mi}{5}\PYG{p}{,} \PYG{l+m+mi}{10000}\PYG{p}{)}
\end{Verbatim}

Display histogram of the sample:

\begin{Verbatim}[commandchars=\\\{\}]
\PYG{g+gp}{\PYGZgt{}\PYGZgt{}\PYGZgt{} }\PYG{k+kn}{import} \PYG{n+nn}{matplotlib.pyplot} \PYG{k+kn}{as} \PYG{n+nn}{plt}
\PYG{g+gp}{\PYGZgt{}\PYGZgt{}\PYGZgt{} }\PYG{n}{count}\PYG{p}{,} \PYG{n}{bins}\PYG{p}{,} \PYG{n}{ignored} \PYG{o}{=} \PYG{n}{plt}\PYG{o}{.}\PYG{n}{hist}\PYG{p}{(}\PYG{n}{s}\PYG{p}{,} \PYG{l+m+mi}{14}\PYG{p}{,} \PYG{n}{normed}\PYG{o}{=}\PYG{n+nb+bp}{True}\PYG{p}{)}
\PYG{g+gp}{\PYGZgt{}\PYGZgt{}\PYGZgt{} }\PYG{n}{plt}\PYG{o}{.}\PYG{n}{show}\PYG{p}{(}\PYG{p}{)}
\end{Verbatim}

\end{fulllineitems}

\index{power() (in module lib.graph.network)}

\begin{fulllineitems}
\phantomsection\label{lib.graph:lib.graph.network.power}\pysiglinewithargsret{\code{lib.graph.network.}\bfcode{power}}{\emph{a}, \emph{size=None}}{}
Draws samples in {[}0, 1{]} from a power distribution with positive
exponent a - 1.

Also known as the power function distribution.
\begin{description}
\item[{a}] \leavevmode{[}float{]}
parameter, \textgreater{} 0

\item[{size}] \leavevmode{[}tuple of ints{]}\begin{description}
\item[{Output shape.  If the given shape is, e.g., \code{(m, n, k)}, then}] \leavevmode
\code{m * n * k} samples are drawn.

\end{description}

\end{description}
\begin{description}
\item[{samples}] \leavevmode{[}\{ndarray, scalar\}{]}
The returned samples lie in {[}0, 1{]}.

\end{description}
\begin{description}
\item[{ValueError}] \leavevmode
If a\textless{}1.

\end{description}

The probability density function is
\begin{gather}
\begin{split}P(x; a) = ax^{a-1}, 0 \le x \le 1, a>0.\end{split}\notag
\end{gather}
The power function distribution is just the inverse of the Pareto
distribution. It may also be seen as a special case of the Beta
distribution.

It is used, for example, in modeling the over-reporting of insurance
claims.

Draw samples from the distribution:

\begin{Verbatim}[commandchars=\\\{\}]
\PYG{g+gp}{\PYGZgt{}\PYGZgt{}\PYGZgt{} }\PYG{n}{a} \PYG{o}{=} \PYG{l+m+mf}{5.} \PYG{c}{\PYGZsh{} shape}
\PYG{g+gp}{\PYGZgt{}\PYGZgt{}\PYGZgt{} }\PYG{n}{samples} \PYG{o}{=} \PYG{l+m+mi}{1000}
\PYG{g+gp}{\PYGZgt{}\PYGZgt{}\PYGZgt{} }\PYG{n}{s} \PYG{o}{=} \PYG{n}{np}\PYG{o}{.}\PYG{n}{random}\PYG{o}{.}\PYG{n}{power}\PYG{p}{(}\PYG{n}{a}\PYG{p}{,} \PYG{n}{samples}\PYG{p}{)}
\end{Verbatim}

Display the histogram of the samples, along with
the probability density function:

\begin{Verbatim}[commandchars=\\\{\}]
\PYG{g+gp}{\PYGZgt{}\PYGZgt{}\PYGZgt{} }\PYG{k+kn}{import} \PYG{n+nn}{matplotlib.pyplot} \PYG{k+kn}{as} \PYG{n+nn}{plt}
\PYG{g+gp}{\PYGZgt{}\PYGZgt{}\PYGZgt{} }\PYG{n}{count}\PYG{p}{,} \PYG{n}{bins}\PYG{p}{,} \PYG{n}{ignored} \PYG{o}{=} \PYG{n}{plt}\PYG{o}{.}\PYG{n}{hist}\PYG{p}{(}\PYG{n}{s}\PYG{p}{,} \PYG{n}{bins}\PYG{o}{=}\PYG{l+m+mi}{30}\PYG{p}{)}
\PYG{g+gp}{\PYGZgt{}\PYGZgt{}\PYGZgt{} }\PYG{n}{x} \PYG{o}{=} \PYG{n}{np}\PYG{o}{.}\PYG{n}{linspace}\PYG{p}{(}\PYG{l+m+mi}{0}\PYG{p}{,} \PYG{l+m+mi}{1}\PYG{p}{,} \PYG{l+m+mi}{100}\PYG{p}{)}
\PYG{g+gp}{\PYGZgt{}\PYGZgt{}\PYGZgt{} }\PYG{n}{y} \PYG{o}{=} \PYG{n}{a}\PYG{o}{*}\PYG{n}{x}\PYG{o}{*}\PYG{o}{*}\PYG{p}{(}\PYG{n}{a}\PYG{o}{\PYGZhy{}}\PYG{l+m+mf}{1.}\PYG{p}{)}
\PYG{g+gp}{\PYGZgt{}\PYGZgt{}\PYGZgt{} }\PYG{n}{normed\PYGZus{}y} \PYG{o}{=} \PYG{n}{samples}\PYG{o}{*}\PYG{n}{np}\PYG{o}{.}\PYG{n}{diff}\PYG{p}{(}\PYG{n}{bins}\PYG{p}{)}\PYG{p}{[}\PYG{l+m+mi}{0}\PYG{p}{]}\PYG{o}{*}\PYG{n}{y}
\PYG{g+gp}{\PYGZgt{}\PYGZgt{}\PYGZgt{} }\PYG{n}{plt}\PYG{o}{.}\PYG{n}{plot}\PYG{p}{(}\PYG{n}{x}\PYG{p}{,} \PYG{n}{normed\PYGZus{}y}\PYG{p}{)}
\PYG{g+gp}{\PYGZgt{}\PYGZgt{}\PYGZgt{} }\PYG{n}{plt}\PYG{o}{.}\PYG{n}{show}\PYG{p}{(}\PYG{p}{)}
\end{Verbatim}

Compare the power function distribution to the inverse of the Pareto.

\begin{Verbatim}[commandchars=\\\{\}]
\PYG{g+gp}{\PYGZgt{}\PYGZgt{}\PYGZgt{} }\PYG{k+kn}{from} \PYG{n+nn}{scipy} \PYG{k+kn}{import} \PYG{n}{stats}
\PYG{g+gp}{\PYGZgt{}\PYGZgt{}\PYGZgt{} }\PYG{n}{rvs} \PYG{o}{=} \PYG{n}{np}\PYG{o}{.}\PYG{n}{random}\PYG{o}{.}\PYG{n}{power}\PYG{p}{(}\PYG{l+m+mi}{5}\PYG{p}{,} \PYG{l+m+mi}{1000000}\PYG{p}{)}
\PYG{g+gp}{\PYGZgt{}\PYGZgt{}\PYGZgt{} }\PYG{n}{rvsp} \PYG{o}{=} \PYG{n}{np}\PYG{o}{.}\PYG{n}{random}\PYG{o}{.}\PYG{n}{pareto}\PYG{p}{(}\PYG{l+m+mi}{5}\PYG{p}{,} \PYG{l+m+mi}{1000000}\PYG{p}{)}
\PYG{g+gp}{\PYGZgt{}\PYGZgt{}\PYGZgt{} }\PYG{n}{xx} \PYG{o}{=} \PYG{n}{np}\PYG{o}{.}\PYG{n}{linspace}\PYG{p}{(}\PYG{l+m+mi}{0}\PYG{p}{,}\PYG{l+m+mi}{1}\PYG{p}{,}\PYG{l+m+mi}{100}\PYG{p}{)}
\PYG{g+gp}{\PYGZgt{}\PYGZgt{}\PYGZgt{} }\PYG{n}{powpdf} \PYG{o}{=} \PYG{n}{stats}\PYG{o}{.}\PYG{n}{powerlaw}\PYG{o}{.}\PYG{n}{pdf}\PYG{p}{(}\PYG{n}{xx}\PYG{p}{,}\PYG{l+m+mi}{5}\PYG{p}{)}
\end{Verbatim}

\begin{Verbatim}[commandchars=\\\{\}]
\PYG{g+gp}{\PYGZgt{}\PYGZgt{}\PYGZgt{} }\PYG{n}{plt}\PYG{o}{.}\PYG{n}{figure}\PYG{p}{(}\PYG{p}{)}
\PYG{g+gp}{\PYGZgt{}\PYGZgt{}\PYGZgt{} }\PYG{n}{plt}\PYG{o}{.}\PYG{n}{hist}\PYG{p}{(}\PYG{n}{rvs}\PYG{p}{,} \PYG{n}{bins}\PYG{o}{=}\PYG{l+m+mi}{50}\PYG{p}{,} \PYG{n}{normed}\PYG{o}{=}\PYG{n+nb+bp}{True}\PYG{p}{)}
\PYG{g+gp}{\PYGZgt{}\PYGZgt{}\PYGZgt{} }\PYG{n}{plt}\PYG{o}{.}\PYG{n}{plot}\PYG{p}{(}\PYG{n}{xx}\PYG{p}{,}\PYG{n}{powpdf}\PYG{p}{,}\PYG{l+s}{\PYGZsq{}}\PYG{l+s}{r\PYGZhy{}}\PYG{l+s}{\PYGZsq{}}\PYG{p}{)}
\PYG{g+gp}{\PYGZgt{}\PYGZgt{}\PYGZgt{} }\PYG{n}{plt}\PYG{o}{.}\PYG{n}{title}\PYG{p}{(}\PYG{l+s}{\PYGZsq{}}\PYG{l+s}{np.random.power(5)}\PYG{l+s}{\PYGZsq{}}\PYG{p}{)}
\end{Verbatim}

\begin{Verbatim}[commandchars=\\\{\}]
\PYG{g+gp}{\PYGZgt{}\PYGZgt{}\PYGZgt{} }\PYG{n}{plt}\PYG{o}{.}\PYG{n}{figure}\PYG{p}{(}\PYG{p}{)}
\PYG{g+gp}{\PYGZgt{}\PYGZgt{}\PYGZgt{} }\PYG{n}{plt}\PYG{o}{.}\PYG{n}{hist}\PYG{p}{(}\PYG{l+m+mf}{1.}\PYG{o}{/}\PYG{p}{(}\PYG{l+m+mf}{1.}\PYG{o}{+}\PYG{n}{rvsp}\PYG{p}{)}\PYG{p}{,} \PYG{n}{bins}\PYG{o}{=}\PYG{l+m+mi}{50}\PYG{p}{,} \PYG{n}{normed}\PYG{o}{=}\PYG{n+nb+bp}{True}\PYG{p}{)}
\PYG{g+gp}{\PYGZgt{}\PYGZgt{}\PYGZgt{} }\PYG{n}{plt}\PYG{o}{.}\PYG{n}{plot}\PYG{p}{(}\PYG{n}{xx}\PYG{p}{,}\PYG{n}{powpdf}\PYG{p}{,}\PYG{l+s}{\PYGZsq{}}\PYG{l+s}{r\PYGZhy{}}\PYG{l+s}{\PYGZsq{}}\PYG{p}{)}
\PYG{g+gp}{\PYGZgt{}\PYGZgt{}\PYGZgt{} }\PYG{n}{plt}\PYG{o}{.}\PYG{n}{title}\PYG{p}{(}\PYG{l+s}{\PYGZsq{}}\PYG{l+s}{inverse of 1 + np.random.pareto(5)}\PYG{l+s}{\PYGZsq{}}\PYG{p}{)}
\end{Verbatim}

\begin{Verbatim}[commandchars=\\\{\}]
\PYG{g+gp}{\PYGZgt{}\PYGZgt{}\PYGZgt{} }\PYG{n}{plt}\PYG{o}{.}\PYG{n}{figure}\PYG{p}{(}\PYG{p}{)}
\PYG{g+gp}{\PYGZgt{}\PYGZgt{}\PYGZgt{} }\PYG{n}{plt}\PYG{o}{.}\PYG{n}{hist}\PYG{p}{(}\PYG{l+m+mf}{1.}\PYG{o}{/}\PYG{p}{(}\PYG{l+m+mf}{1.}\PYG{o}{+}\PYG{n}{rvsp}\PYG{p}{)}\PYG{p}{,} \PYG{n}{bins}\PYG{o}{=}\PYG{l+m+mi}{50}\PYG{p}{,} \PYG{n}{normed}\PYG{o}{=}\PYG{n+nb+bp}{True}\PYG{p}{)}
\PYG{g+gp}{\PYGZgt{}\PYGZgt{}\PYGZgt{} }\PYG{n}{plt}\PYG{o}{.}\PYG{n}{plot}\PYG{p}{(}\PYG{n}{xx}\PYG{p}{,}\PYG{n}{powpdf}\PYG{p}{,}\PYG{l+s}{\PYGZsq{}}\PYG{l+s}{r\PYGZhy{}}\PYG{l+s}{\PYGZsq{}}\PYG{p}{)}
\PYG{g+gp}{\PYGZgt{}\PYGZgt{}\PYGZgt{} }\PYG{n}{plt}\PYG{o}{.}\PYG{n}{title}\PYG{p}{(}\PYG{l+s}{\PYGZsq{}}\PYG{l+s}{inverse of stats.pareto(5)}\PYG{l+s}{\PYGZsq{}}\PYG{p}{)}
\end{Verbatim}

\end{fulllineitems}

\index{rand() (in module lib.graph.network)}

\begin{fulllineitems}
\phantomsection\label{lib.graph:lib.graph.network.rand}\pysiglinewithargsret{\code{lib.graph.network.}\bfcode{rand}}{\emph{d0}, \emph{d1}, \emph{...}, \emph{dn}}{}
Random values in a given shape.

Create an array of the given shape and propagate it with
random samples from a uniform distribution
over \code{{[}0, 1)}.
\begin{description}
\item[{d0, d1, ..., dn}] \leavevmode{[}int, optional{]}
The dimensions of the returned array, should all be positive.
If no argument is given a single Python float is returned.

\end{description}
\begin{description}
\item[{out}] \leavevmode{[}ndarray, shape \code{(d0, d1, ..., dn)}{]}
Random values.

\end{description}

random

This is a convenience function. If you want an interface that
takes a shape-tuple as the first argument, refer to
np.random.random\_sample .

\begin{Verbatim}[commandchars=\\\{\}]
\PYG{g+gp}{\PYGZgt{}\PYGZgt{}\PYGZgt{} }\PYG{n}{np}\PYG{o}{.}\PYG{n}{random}\PYG{o}{.}\PYG{n}{rand}\PYG{p}{(}\PYG{l+m+mi}{3}\PYG{p}{,}\PYG{l+m+mi}{2}\PYG{p}{)}
\PYG{g+go}{array([[ 0.14022471,  0.96360618],  \PYGZsh{}random}
\PYG{g+go}{       [ 0.37601032,  0.25528411],  \PYGZsh{}random}
\PYG{g+go}{       [ 0.49313049,  0.94909878]]) \PYGZsh{}random}
\end{Verbatim}

\end{fulllineitems}

\index{randint() (in module lib.graph.network)}

\begin{fulllineitems}
\phantomsection\label{lib.graph:lib.graph.network.randint}\pysiglinewithargsret{\code{lib.graph.network.}\bfcode{randint}}{\emph{low}, \emph{high=None}, \emph{size=None}}{}
Return random integers from \emph{low} (inclusive) to \emph{high} (exclusive).

Return random integers from the ``discrete uniform'' distribution in the
``half-open'' interval {[}\emph{low}, \emph{high}). If \emph{high} is None (the default),
then results are from {[}0, \emph{low}).
\begin{description}
\item[{low}] \leavevmode{[}int{]}
Lowest (signed) integer to be drawn from the distribution (unless
\code{high=None}, in which case this parameter is the \emph{highest} such
integer).

\item[{high}] \leavevmode{[}int, optional{]}
If provided, one above the largest (signed) integer to be drawn
from the distribution (see above for behavior if \code{high=None}).

\item[{size}] \leavevmode{[}int or tuple of ints, optional{]}
Output shape. Default is None, in which case a single int is
returned.

\end{description}
\begin{description}
\item[{out}] \leavevmode{[}int or ndarray of ints{]}
\emph{size}-shaped array of random integers from the appropriate
distribution, or a single such random int if \emph{size} not provided.

\end{description}
\begin{description}
\item[{random.random\_integers}] \leavevmode{[}similar to \emph{randint}, only for the closed{]}
interval {[}\emph{low}, \emph{high}{]}, and 1 is the lowest value if \emph{high} is
omitted. In particular, this other one is the one to use to generate
uniformly distributed discrete non-integers.

\end{description}

\begin{Verbatim}[commandchars=\\\{\}]
\PYG{g+gp}{\PYGZgt{}\PYGZgt{}\PYGZgt{} }\PYG{n}{np}\PYG{o}{.}\PYG{n}{random}\PYG{o}{.}\PYG{n}{randint}\PYG{p}{(}\PYG{l+m+mi}{2}\PYG{p}{,} \PYG{n}{size}\PYG{o}{=}\PYG{l+m+mi}{10}\PYG{p}{)}
\PYG{g+go}{array([1, 0, 0, 0, 1, 1, 0, 0, 1, 0])}
\PYG{g+gp}{\PYGZgt{}\PYGZgt{}\PYGZgt{} }\PYG{n}{np}\PYG{o}{.}\PYG{n}{random}\PYG{o}{.}\PYG{n}{randint}\PYG{p}{(}\PYG{l+m+mi}{1}\PYG{p}{,} \PYG{n}{size}\PYG{o}{=}\PYG{l+m+mi}{10}\PYG{p}{)}
\PYG{g+go}{array([0, 0, 0, 0, 0, 0, 0, 0, 0, 0])}
\end{Verbatim}

Generate a 2 x 4 array of ints between 0 and 4, inclusive:

\begin{Verbatim}[commandchars=\\\{\}]
\PYG{g+gp}{\PYGZgt{}\PYGZgt{}\PYGZgt{} }\PYG{n}{np}\PYG{o}{.}\PYG{n}{random}\PYG{o}{.}\PYG{n}{randint}\PYG{p}{(}\PYG{l+m+mi}{5}\PYG{p}{,} \PYG{n}{size}\PYG{o}{=}\PYG{p}{(}\PYG{l+m+mi}{2}\PYG{p}{,} \PYG{l+m+mi}{4}\PYG{p}{)}\PYG{p}{)}
\PYG{g+go}{array([[4, 0, 2, 1],}
\PYG{g+go}{       [3, 2, 2, 0]])}
\end{Verbatim}

\end{fulllineitems}

\index{randn() (in module lib.graph.network)}

\begin{fulllineitems}
\phantomsection\label{lib.graph:lib.graph.network.randn}\pysiglinewithargsret{\code{lib.graph.network.}\bfcode{randn}}{\emph{d0}, \emph{d1}, \emph{...}, \emph{dn}}{}
Return a sample (or samples) from the ``standard normal'' distribution.

If positive, int\_like or int-convertible arguments are provided,
\emph{randn} generates an array of shape \code{(d0, d1, ..., dn)}, filled
with random floats sampled from a univariate ``normal'' (Gaussian)
distribution of mean 0 and variance 1 (if any of the \(d_i\) are
floats, they are first converted to integers by truncation). A single
float randomly sampled from the distribution is returned if no
argument is provided.

This is a convenience function.  If you want an interface that takes a
tuple as the first argument, use \emph{numpy.random.standard\_normal} instead.
\begin{description}
\item[{d0, d1, ..., dn}] \leavevmode{[}int, optional{]}
The dimensions of the returned array, should be all positive.
If no argument is given a single Python float is returned.

\end{description}
\begin{description}
\item[{Z}] \leavevmode{[}ndarray or float{]}
A \code{(d0, d1, ..., dn)}-shaped array of floating-point samples from
the standard normal distribution, or a single such float if
no parameters were supplied.

\end{description}

random.standard\_normal : Similar, but takes a tuple as its argument.

For random samples from \(N(\mu, \sigma^2)\), use:

\code{sigma * np.random.randn(...) + mu}

\begin{Verbatim}[commandchars=\\\{\}]
\PYG{g+gp}{\PYGZgt{}\PYGZgt{}\PYGZgt{} }\PYG{n}{np}\PYG{o}{.}\PYG{n}{random}\PYG{o}{.}\PYG{n}{randn}\PYG{p}{(}\PYG{p}{)}
\PYG{g+go}{2.1923875335537315 \PYGZsh{}random}
\end{Verbatim}

Two-by-four array of samples from N(3, 6.25):

\begin{Verbatim}[commandchars=\\\{\}]
\PYG{g+gp}{\PYGZgt{}\PYGZgt{}\PYGZgt{} }\PYG{l+m+mf}{2.5} \PYG{o}{*} \PYG{n}{np}\PYG{o}{.}\PYG{n}{random}\PYG{o}{.}\PYG{n}{randn}\PYG{p}{(}\PYG{l+m+mi}{2}\PYG{p}{,} \PYG{l+m+mi}{4}\PYG{p}{)} \PYG{o}{+} \PYG{l+m+mi}{3}
\PYG{g+go}{array([[\PYGZhy{}4.49401501,  4.00950034, \PYGZhy{}1.81814867,  7.29718677],  \PYGZsh{}random}
\PYG{g+go}{       [ 0.39924804,  4.68456316,  4.99394529,  4.84057254]]) \PYGZsh{}random}
\end{Verbatim}

\end{fulllineitems}

\index{random() (in module lib.graph.network)}

\begin{fulllineitems}
\phantomsection\label{lib.graph:lib.graph.network.random}\pysiglinewithargsret{\code{lib.graph.network.}\bfcode{random}}{}{}
random\_sample(size=None)

Return random floats in the half-open interval {[}0.0, 1.0).

Results are from the ``continuous uniform'' distribution over the
stated interval.  To sample \(Unif[a, b), b > a\) multiply
the output of \emph{random\_sample} by \emph{(b-a)} and add \emph{a}:

\begin{Verbatim}[commandchars=\\\{\}]
\PYG{p}{(}\PYG{n}{b} \PYG{o}{\PYGZhy{}} \PYG{n}{a}\PYG{p}{)} \PYG{o}{*} \PYG{n}{random\PYGZus{}sample}\PYG{p}{(}\PYG{p}{)} \PYG{o}{+} \PYG{n}{a}
\end{Verbatim}
\begin{description}
\item[{size}] \leavevmode{[}int or tuple of ints, optional{]}
Defines the shape of the returned array of random floats. If None
(the default), returns a single float.

\end{description}
\begin{description}
\item[{out}] \leavevmode{[}float or ndarray of floats{]}
Array of random floats of shape \emph{size} (unless \code{size=None}, in which
case a single float is returned).

\end{description}

\begin{Verbatim}[commandchars=\\\{\}]
\PYG{g+gp}{\PYGZgt{}\PYGZgt{}\PYGZgt{} }\PYG{n}{np}\PYG{o}{.}\PYG{n}{random}\PYG{o}{.}\PYG{n}{random\PYGZus{}sample}\PYG{p}{(}\PYG{p}{)}
\PYG{g+go}{0.47108547995356098}
\PYG{g+gp}{\PYGZgt{}\PYGZgt{}\PYGZgt{} }\PYG{n+nb}{type}\PYG{p}{(}\PYG{n}{np}\PYG{o}{.}\PYG{n}{random}\PYG{o}{.}\PYG{n}{random\PYGZus{}sample}\PYG{p}{(}\PYG{p}{)}\PYG{p}{)}
\PYG{g+go}{\PYGZlt{}type \PYGZsq{}float\PYGZsq{}\PYGZgt{}}
\PYG{g+gp}{\PYGZgt{}\PYGZgt{}\PYGZgt{} }\PYG{n}{np}\PYG{o}{.}\PYG{n}{random}\PYG{o}{.}\PYG{n}{random\PYGZus{}sample}\PYG{p}{(}\PYG{p}{(}\PYG{l+m+mi}{5}\PYG{p}{,}\PYG{p}{)}\PYG{p}{)}
\PYG{g+go}{array([ 0.30220482,  0.86820401,  0.1654503 ,  0.11659149,  0.54323428])}
\end{Verbatim}

Three-by-two array of random numbers from {[}-5, 0):

\begin{Verbatim}[commandchars=\\\{\}]
\PYG{g+gp}{\PYGZgt{}\PYGZgt{}\PYGZgt{} }\PYG{l+m+mi}{5} \PYG{o}{*} \PYG{n}{np}\PYG{o}{.}\PYG{n}{random}\PYG{o}{.}\PYG{n}{random\PYGZus{}sample}\PYG{p}{(}\PYG{p}{(}\PYG{l+m+mi}{3}\PYG{p}{,} \PYG{l+m+mi}{2}\PYG{p}{)}\PYG{p}{)} \PYG{o}{\PYGZhy{}} \PYG{l+m+mi}{5}
\PYG{g+go}{array([[\PYGZhy{}3.99149989, \PYGZhy{}0.52338984],}
\PYG{g+go}{       [\PYGZhy{}2.99091858, \PYGZhy{}0.79479508],}
\PYG{g+go}{       [\PYGZhy{}1.23204345, \PYGZhy{}1.75224494]])}
\end{Verbatim}

\end{fulllineitems}

\index{random\_integers() (in module lib.graph.network)}

\begin{fulllineitems}
\phantomsection\label{lib.graph:lib.graph.network.random_integers}\pysiglinewithargsret{\code{lib.graph.network.}\bfcode{random\_integers}}{\emph{low}, \emph{high=None}, \emph{size=None}}{}
Return random integers between \emph{low} and \emph{high}, inclusive.

Return random integers from the ``discrete uniform'' distribution in the
closed interval {[}\emph{low}, \emph{high}{]}.  If \emph{high} is None (the default),
then results are from {[}1, \emph{low}{]}.
\begin{description}
\item[{low}] \leavevmode{[}int{]}
Lowest (signed) integer to be drawn from the distribution (unless
\code{high=None}, in which case this parameter is the \emph{highest} such
integer).

\item[{high}] \leavevmode{[}int, optional{]}
If provided, the largest (signed) integer to be drawn from the
distribution (see above for behavior if \code{high=None}).

\item[{size}] \leavevmode{[}int or tuple of ints, optional{]}
Output shape. Default is None, in which case a single int is returned.

\end{description}
\begin{description}
\item[{out}] \leavevmode{[}int or ndarray of ints{]}
\emph{size}-shaped array of random integers from the appropriate
distribution, or a single such random int if \emph{size} not provided.

\end{description}
\begin{description}
\item[{random.randint}] \leavevmode{[}Similar to \emph{random\_integers}, only for the half-open{]}
interval {[}\emph{low}, \emph{high}), and 0 is the lowest value if \emph{high} is
omitted.

\end{description}

To sample from N evenly spaced floating-point numbers between a and b,
use:

\begin{Verbatim}[commandchars=\\\{\}]
\PYG{n}{a} \PYG{o}{+} \PYG{p}{(}\PYG{n}{b} \PYG{o}{\PYGZhy{}} \PYG{n}{a}\PYG{p}{)} \PYG{o}{*} \PYG{p}{(}\PYG{n}{np}\PYG{o}{.}\PYG{n}{random}\PYG{o}{.}\PYG{n}{random\PYGZus{}integers}\PYG{p}{(}\PYG{n}{N}\PYG{p}{)} \PYG{o}{\PYGZhy{}} \PYG{l+m+mi}{1}\PYG{p}{)} \PYG{o}{/} \PYG{p}{(}\PYG{n}{N} \PYG{o}{\PYGZhy{}} \PYG{l+m+mf}{1.}\PYG{p}{)}
\end{Verbatim}

\begin{Verbatim}[commandchars=\\\{\}]
\PYG{g+gp}{\PYGZgt{}\PYGZgt{}\PYGZgt{} }\PYG{n}{np}\PYG{o}{.}\PYG{n}{random}\PYG{o}{.}\PYG{n}{random\PYGZus{}integers}\PYG{p}{(}\PYG{l+m+mi}{5}\PYG{p}{)}
\PYG{g+go}{4}
\PYG{g+gp}{\PYGZgt{}\PYGZgt{}\PYGZgt{} }\PYG{n+nb}{type}\PYG{p}{(}\PYG{n}{np}\PYG{o}{.}\PYG{n}{random}\PYG{o}{.}\PYG{n}{random\PYGZus{}integers}\PYG{p}{(}\PYG{l+m+mi}{5}\PYG{p}{)}\PYG{p}{)}
\PYG{g+go}{\PYGZlt{}type \PYGZsq{}int\PYGZsq{}\PYGZgt{}}
\PYG{g+gp}{\PYGZgt{}\PYGZgt{}\PYGZgt{} }\PYG{n}{np}\PYG{o}{.}\PYG{n}{random}\PYG{o}{.}\PYG{n}{random\PYGZus{}integers}\PYG{p}{(}\PYG{l+m+mi}{5}\PYG{p}{,} \PYG{n}{size}\PYG{o}{=}\PYG{p}{(}\PYG{l+m+mf}{3.}\PYG{p}{,}\PYG{l+m+mf}{2.}\PYG{p}{)}\PYG{p}{)}
\PYG{g+go}{array([[5, 4],}
\PYG{g+go}{       [3, 3],}
\PYG{g+go}{       [4, 5]])}
\end{Verbatim}

Choose five random numbers from the set of five evenly-spaced
numbers between 0 and 2.5, inclusive (\emph{i.e.}, from the set
\({0, 5/8, 10/8, 15/8, 20/8}\)):

\begin{Verbatim}[commandchars=\\\{\}]
\PYG{g+gp}{\PYGZgt{}\PYGZgt{}\PYGZgt{} }\PYG{l+m+mf}{2.5} \PYG{o}{*} \PYG{p}{(}\PYG{n}{np}\PYG{o}{.}\PYG{n}{random}\PYG{o}{.}\PYG{n}{random\PYGZus{}integers}\PYG{p}{(}\PYG{l+m+mi}{5}\PYG{p}{,} \PYG{n}{size}\PYG{o}{=}\PYG{p}{(}\PYG{l+m+mi}{5}\PYG{p}{,}\PYG{p}{)}\PYG{p}{)} \PYG{o}{\PYGZhy{}} \PYG{l+m+mi}{1}\PYG{p}{)} \PYG{o}{/} \PYG{l+m+mf}{4.}
\PYG{g+go}{array([ 0.625,  1.25 ,  0.625,  0.625,  2.5  ])}
\end{Verbatim}

Roll two six sided dice 1000 times and sum the results:

\begin{Verbatim}[commandchars=\\\{\}]
\PYG{g+gp}{\PYGZgt{}\PYGZgt{}\PYGZgt{} }\PYG{n}{d1} \PYG{o}{=} \PYG{n}{np}\PYG{o}{.}\PYG{n}{random}\PYG{o}{.}\PYG{n}{random\PYGZus{}integers}\PYG{p}{(}\PYG{l+m+mi}{1}\PYG{p}{,} \PYG{l+m+mi}{6}\PYG{p}{,} \PYG{l+m+mi}{1000}\PYG{p}{)}
\PYG{g+gp}{\PYGZgt{}\PYGZgt{}\PYGZgt{} }\PYG{n}{d2} \PYG{o}{=} \PYG{n}{np}\PYG{o}{.}\PYG{n}{random}\PYG{o}{.}\PYG{n}{random\PYGZus{}integers}\PYG{p}{(}\PYG{l+m+mi}{1}\PYG{p}{,} \PYG{l+m+mi}{6}\PYG{p}{,} \PYG{l+m+mi}{1000}\PYG{p}{)}
\PYG{g+gp}{\PYGZgt{}\PYGZgt{}\PYGZgt{} }\PYG{n}{dsums} \PYG{o}{=} \PYG{n}{d1} \PYG{o}{+} \PYG{n}{d2}
\end{Verbatim}

Display results as a histogram:

\begin{Verbatim}[commandchars=\\\{\}]
\PYG{g+gp}{\PYGZgt{}\PYGZgt{}\PYGZgt{} }\PYG{k+kn}{import} \PYG{n+nn}{matplotlib.pyplot} \PYG{k+kn}{as} \PYG{n+nn}{plt}
\PYG{g+gp}{\PYGZgt{}\PYGZgt{}\PYGZgt{} }\PYG{n}{count}\PYG{p}{,} \PYG{n}{bins}\PYG{p}{,} \PYG{n}{ignored} \PYG{o}{=} \PYG{n}{plt}\PYG{o}{.}\PYG{n}{hist}\PYG{p}{(}\PYG{n}{dsums}\PYG{p}{,} \PYG{l+m+mi}{11}\PYG{p}{,} \PYG{n}{normed}\PYG{o}{=}\PYG{n+nb+bp}{True}\PYG{p}{)}
\PYG{g+gp}{\PYGZgt{}\PYGZgt{}\PYGZgt{} }\PYG{n}{plt}\PYG{o}{.}\PYG{n}{show}\PYG{p}{(}\PYG{p}{)}
\end{Verbatim}

\end{fulllineitems}

\index{random\_sample() (in module lib.graph.network)}

\begin{fulllineitems}
\phantomsection\label{lib.graph:lib.graph.network.random_sample}\pysiglinewithargsret{\code{lib.graph.network.}\bfcode{random\_sample}}{\emph{size=None}}{}
Return random floats in the half-open interval {[}0.0, 1.0).

Results are from the ``continuous uniform'' distribution over the
stated interval.  To sample \(Unif[a, b), b > a\) multiply
the output of \emph{random\_sample} by \emph{(b-a)} and add \emph{a}:

\begin{Verbatim}[commandchars=\\\{\}]
\PYG{p}{(}\PYG{n}{b} \PYG{o}{\PYGZhy{}} \PYG{n}{a}\PYG{p}{)} \PYG{o}{*} \PYG{n}{random\PYGZus{}sample}\PYG{p}{(}\PYG{p}{)} \PYG{o}{+} \PYG{n}{a}
\end{Verbatim}
\begin{description}
\item[{size}] \leavevmode{[}int or tuple of ints, optional{]}
Defines the shape of the returned array of random floats. If None
(the default), returns a single float.

\end{description}
\begin{description}
\item[{out}] \leavevmode{[}float or ndarray of floats{]}
Array of random floats of shape \emph{size} (unless \code{size=None}, in which
case a single float is returned).

\end{description}

\begin{Verbatim}[commandchars=\\\{\}]
\PYG{g+gp}{\PYGZgt{}\PYGZgt{}\PYGZgt{} }\PYG{n}{np}\PYG{o}{.}\PYG{n}{random}\PYG{o}{.}\PYG{n}{random\PYGZus{}sample}\PYG{p}{(}\PYG{p}{)}
\PYG{g+go}{0.47108547995356098}
\PYG{g+gp}{\PYGZgt{}\PYGZgt{}\PYGZgt{} }\PYG{n+nb}{type}\PYG{p}{(}\PYG{n}{np}\PYG{o}{.}\PYG{n}{random}\PYG{o}{.}\PYG{n}{random\PYGZus{}sample}\PYG{p}{(}\PYG{p}{)}\PYG{p}{)}
\PYG{g+go}{\PYGZlt{}type \PYGZsq{}float\PYGZsq{}\PYGZgt{}}
\PYG{g+gp}{\PYGZgt{}\PYGZgt{}\PYGZgt{} }\PYG{n}{np}\PYG{o}{.}\PYG{n}{random}\PYG{o}{.}\PYG{n}{random\PYGZus{}sample}\PYG{p}{(}\PYG{p}{(}\PYG{l+m+mi}{5}\PYG{p}{,}\PYG{p}{)}\PYG{p}{)}
\PYG{g+go}{array([ 0.30220482,  0.86820401,  0.1654503 ,  0.11659149,  0.54323428])}
\end{Verbatim}

Three-by-two array of random numbers from {[}-5, 0):

\begin{Verbatim}[commandchars=\\\{\}]
\PYG{g+gp}{\PYGZgt{}\PYGZgt{}\PYGZgt{} }\PYG{l+m+mi}{5} \PYG{o}{*} \PYG{n}{np}\PYG{o}{.}\PYG{n}{random}\PYG{o}{.}\PYG{n}{random\PYGZus{}sample}\PYG{p}{(}\PYG{p}{(}\PYG{l+m+mi}{3}\PYG{p}{,} \PYG{l+m+mi}{2}\PYG{p}{)}\PYG{p}{)} \PYG{o}{\PYGZhy{}} \PYG{l+m+mi}{5}
\PYG{g+go}{array([[\PYGZhy{}3.99149989, \PYGZhy{}0.52338984],}
\PYG{g+go}{       [\PYGZhy{}2.99091858, \PYGZhy{}0.79479508],}
\PYG{g+go}{       [\PYGZhy{}1.23204345, \PYGZhy{}1.75224494]])}
\end{Verbatim}

\end{fulllineitems}

\index{ranf() (in module lib.graph.network)}

\begin{fulllineitems}
\phantomsection\label{lib.graph:lib.graph.network.ranf}\pysiglinewithargsret{\code{lib.graph.network.}\bfcode{ranf}}{}{}
random\_sample(size=None)

Return random floats in the half-open interval {[}0.0, 1.0).

Results are from the ``continuous uniform'' distribution over the
stated interval.  To sample \(Unif[a, b), b > a\) multiply
the output of \emph{random\_sample} by \emph{(b-a)} and add \emph{a}:

\begin{Verbatim}[commandchars=\\\{\}]
\PYG{p}{(}\PYG{n}{b} \PYG{o}{\PYGZhy{}} \PYG{n}{a}\PYG{p}{)} \PYG{o}{*} \PYG{n}{random\PYGZus{}sample}\PYG{p}{(}\PYG{p}{)} \PYG{o}{+} \PYG{n}{a}
\end{Verbatim}
\begin{description}
\item[{size}] \leavevmode{[}int or tuple of ints, optional{]}
Defines the shape of the returned array of random floats. If None
(the default), returns a single float.

\end{description}
\begin{description}
\item[{out}] \leavevmode{[}float or ndarray of floats{]}
Array of random floats of shape \emph{size} (unless \code{size=None}, in which
case a single float is returned).

\end{description}

\begin{Verbatim}[commandchars=\\\{\}]
\PYG{g+gp}{\PYGZgt{}\PYGZgt{}\PYGZgt{} }\PYG{n}{np}\PYG{o}{.}\PYG{n}{random}\PYG{o}{.}\PYG{n}{random\PYGZus{}sample}\PYG{p}{(}\PYG{p}{)}
\PYG{g+go}{0.47108547995356098}
\PYG{g+gp}{\PYGZgt{}\PYGZgt{}\PYGZgt{} }\PYG{n+nb}{type}\PYG{p}{(}\PYG{n}{np}\PYG{o}{.}\PYG{n}{random}\PYG{o}{.}\PYG{n}{random\PYGZus{}sample}\PYG{p}{(}\PYG{p}{)}\PYG{p}{)}
\PYG{g+go}{\PYGZlt{}type \PYGZsq{}float\PYGZsq{}\PYGZgt{}}
\PYG{g+gp}{\PYGZgt{}\PYGZgt{}\PYGZgt{} }\PYG{n}{np}\PYG{o}{.}\PYG{n}{random}\PYG{o}{.}\PYG{n}{random\PYGZus{}sample}\PYG{p}{(}\PYG{p}{(}\PYG{l+m+mi}{5}\PYG{p}{,}\PYG{p}{)}\PYG{p}{)}
\PYG{g+go}{array([ 0.30220482,  0.86820401,  0.1654503 ,  0.11659149,  0.54323428])}
\end{Verbatim}

Three-by-two array of random numbers from {[}-5, 0):

\begin{Verbatim}[commandchars=\\\{\}]
\PYG{g+gp}{\PYGZgt{}\PYGZgt{}\PYGZgt{} }\PYG{l+m+mi}{5} \PYG{o}{*} \PYG{n}{np}\PYG{o}{.}\PYG{n}{random}\PYG{o}{.}\PYG{n}{random\PYGZus{}sample}\PYG{p}{(}\PYG{p}{(}\PYG{l+m+mi}{3}\PYG{p}{,} \PYG{l+m+mi}{2}\PYG{p}{)}\PYG{p}{)} \PYG{o}{\PYGZhy{}} \PYG{l+m+mi}{5}
\PYG{g+go}{array([[\PYGZhy{}3.99149989, \PYGZhy{}0.52338984],}
\PYG{g+go}{       [\PYGZhy{}2.99091858, \PYGZhy{}0.79479508],}
\PYG{g+go}{       [\PYGZhy{}1.23204345, \PYGZhy{}1.75224494]])}
\end{Verbatim}

\end{fulllineitems}

\index{rayleigh() (in module lib.graph.network)}

\begin{fulllineitems}
\phantomsection\label{lib.graph:lib.graph.network.rayleigh}\pysiglinewithargsret{\code{lib.graph.network.}\bfcode{rayleigh}}{\emph{scale=1.0}, \emph{size=None}}{}
Draw samples from a Rayleigh distribution.

The \(\chi\) and Weibull distributions are generalizations of the
Rayleigh.
\begin{description}
\item[{scale}] \leavevmode{[}scalar{]}
Scale, also equals the mode. Should be \textgreater{}= 0.

\item[{size}] \leavevmode{[}int or tuple of ints, optional{]}
Shape of the output. Default is None, in which case a single
value is returned.

\end{description}

The probability density function for the Rayleigh distribution is
\begin{gather}
\begin{split}P(x;scale) = \frac{x}{scale^2}e^{\frac{-x^2}{2 \cdotp scale^2}}\end{split}\notag
\end{gather}
The Rayleigh distribution arises if the wind speed and wind direction are
both gaussian variables, then the vector wind velocity forms a Rayleigh
distribution. The Rayleigh distribution is used to model the expected
output from wind turbines.

Draw values from the distribution and plot the histogram

\begin{Verbatim}[commandchars=\\\{\}]
\PYG{g+gp}{\PYGZgt{}\PYGZgt{}\PYGZgt{} }\PYG{n}{values} \PYG{o}{=} \PYG{n}{hist}\PYG{p}{(}\PYG{n}{np}\PYG{o}{.}\PYG{n}{random}\PYG{o}{.}\PYG{n}{rayleigh}\PYG{p}{(}\PYG{l+m+mi}{3}\PYG{p}{,} \PYG{l+m+mi}{100000}\PYG{p}{)}\PYG{p}{,} \PYG{n}{bins}\PYG{o}{=}\PYG{l+m+mi}{200}\PYG{p}{,} \PYG{n}{normed}\PYG{o}{=}\PYG{n+nb+bp}{True}\PYG{p}{)}
\end{Verbatim}

Wave heights tend to follow a Rayleigh distribution. If the mean wave
height is 1 meter, what fraction of waves are likely to be larger than 3
meters?

\begin{Verbatim}[commandchars=\\\{\}]
\PYG{g+gp}{\PYGZgt{}\PYGZgt{}\PYGZgt{} }\PYG{n}{meanvalue} \PYG{o}{=} \PYG{l+m+mi}{1}
\PYG{g+gp}{\PYGZgt{}\PYGZgt{}\PYGZgt{} }\PYG{n}{modevalue} \PYG{o}{=} \PYG{n}{np}\PYG{o}{.}\PYG{n}{sqrt}\PYG{p}{(}\PYG{l+m+mi}{2} \PYG{o}{/} \PYG{n}{np}\PYG{o}{.}\PYG{n}{pi}\PYG{p}{)} \PYG{o}{*} \PYG{n}{meanvalue}
\PYG{g+gp}{\PYGZgt{}\PYGZgt{}\PYGZgt{} }\PYG{n}{s} \PYG{o}{=} \PYG{n}{np}\PYG{o}{.}\PYG{n}{random}\PYG{o}{.}\PYG{n}{rayleigh}\PYG{p}{(}\PYG{n}{modevalue}\PYG{p}{,} \PYG{l+m+mi}{1000000}\PYG{p}{)}
\end{Verbatim}

The percentage of waves larger than 3 meters is:

\begin{Verbatim}[commandchars=\\\{\}]
\PYG{g+gp}{\PYGZgt{}\PYGZgt{}\PYGZgt{} }\PYG{l+m+mf}{100.}\PYG{o}{*}\PYG{n+nb}{sum}\PYG{p}{(}\PYG{n}{s}\PYG{o}{\PYGZgt{}}\PYG{l+m+mi}{3}\PYG{p}{)}\PYG{o}{/}\PYG{l+m+mf}{1000000.}
\PYG{g+go}{0.087300000000000003}
\end{Verbatim}

\end{fulllineitems}

\index{removeRareRcts() (in module lib.graph.network)}

\begin{fulllineitems}
\phantomsection\label{lib.graph:lib.graph.network.removeRareRcts}\pysiglinewithargsret{\code{lib.graph.network.}\bfcode{removeRareRcts}}{\emph{graph}, \emph{dt}, \emph{life}, \emph{nrg}, \emph{deltat}}{}
\end{fulllineitems}

\index{return\_scc\_in\_raf() (in module lib.graph.network)}

\begin{fulllineitems}
\phantomsection\label{lib.graph:lib.graph.network.return_scc_in_raf}\pysiglinewithargsret{\code{lib.graph.network.}\bfcode{return\_scc\_in\_raf}}{\emph{tmpRAF}, \emph{tmpClosure}, \emph{tmpCats}}{}
\end{fulllineitems}

\index{sample() (in module lib.graph.network)}

\begin{fulllineitems}
\phantomsection\label{lib.graph:lib.graph.network.sample}\pysiglinewithargsret{\code{lib.graph.network.}\bfcode{sample}}{}{}
random\_sample(size=None)

Return random floats in the half-open interval {[}0.0, 1.0).

Results are from the ``continuous uniform'' distribution over the
stated interval.  To sample \(Unif[a, b), b > a\) multiply
the output of \emph{random\_sample} by \emph{(b-a)} and add \emph{a}:

\begin{Verbatim}[commandchars=\\\{\}]
\PYG{p}{(}\PYG{n}{b} \PYG{o}{\PYGZhy{}} \PYG{n}{a}\PYG{p}{)} \PYG{o}{*} \PYG{n}{random\PYGZus{}sample}\PYG{p}{(}\PYG{p}{)} \PYG{o}{+} \PYG{n}{a}
\end{Verbatim}
\begin{description}
\item[{size}] \leavevmode{[}int or tuple of ints, optional{]}
Defines the shape of the returned array of random floats. If None
(the default), returns a single float.

\end{description}
\begin{description}
\item[{out}] \leavevmode{[}float or ndarray of floats{]}
Array of random floats of shape \emph{size} (unless \code{size=None}, in which
case a single float is returned).

\end{description}

\begin{Verbatim}[commandchars=\\\{\}]
\PYG{g+gp}{\PYGZgt{}\PYGZgt{}\PYGZgt{} }\PYG{n}{np}\PYG{o}{.}\PYG{n}{random}\PYG{o}{.}\PYG{n}{random\PYGZus{}sample}\PYG{p}{(}\PYG{p}{)}
\PYG{g+go}{0.47108547995356098}
\PYG{g+gp}{\PYGZgt{}\PYGZgt{}\PYGZgt{} }\PYG{n+nb}{type}\PYG{p}{(}\PYG{n}{np}\PYG{o}{.}\PYG{n}{random}\PYG{o}{.}\PYG{n}{random\PYGZus{}sample}\PYG{p}{(}\PYG{p}{)}\PYG{p}{)}
\PYG{g+go}{\PYGZlt{}type \PYGZsq{}float\PYGZsq{}\PYGZgt{}}
\PYG{g+gp}{\PYGZgt{}\PYGZgt{}\PYGZgt{} }\PYG{n}{np}\PYG{o}{.}\PYG{n}{random}\PYG{o}{.}\PYG{n}{random\PYGZus{}sample}\PYG{p}{(}\PYG{p}{(}\PYG{l+m+mi}{5}\PYG{p}{,}\PYG{p}{)}\PYG{p}{)}
\PYG{g+go}{array([ 0.30220482,  0.86820401,  0.1654503 ,  0.11659149,  0.54323428])}
\end{Verbatim}

Three-by-two array of random numbers from {[}-5, 0):

\begin{Verbatim}[commandchars=\\\{\}]
\PYG{g+gp}{\PYGZgt{}\PYGZgt{}\PYGZgt{} }\PYG{l+m+mi}{5} \PYG{o}{*} \PYG{n}{np}\PYG{o}{.}\PYG{n}{random}\PYG{o}{.}\PYG{n}{random\PYGZus{}sample}\PYG{p}{(}\PYG{p}{(}\PYG{l+m+mi}{3}\PYG{p}{,} \PYG{l+m+mi}{2}\PYG{p}{)}\PYG{p}{)} \PYG{o}{\PYGZhy{}} \PYG{l+m+mi}{5}
\PYG{g+go}{array([[\PYGZhy{}3.99149989, \PYGZhy{}0.52338984],}
\PYG{g+go}{       [\PYGZhy{}2.99091858, \PYGZhy{}0.79479508],}
\PYG{g+go}{       [\PYGZhy{}1.23204345, \PYGZhy{}1.75224494]])}
\end{Verbatim}

\end{fulllineitems}

\index{seed() (in module lib.graph.network)}

\begin{fulllineitems}
\phantomsection\label{lib.graph:lib.graph.network.seed}\pysiglinewithargsret{\code{lib.graph.network.}\bfcode{seed}}{\emph{seed=None}}{}
Seed the generator.

This method is called when \emph{RandomState} is initialized. It can be
called again to re-seed the generator. For details, see \emph{RandomState}.
\begin{description}
\item[{seed}] \leavevmode{[}int or array\_like, optional{]}
Seed for \emph{RandomState}.

\end{description}

RandomState

\end{fulllineitems}

\index{set\_state() (in module lib.graph.network)}

\begin{fulllineitems}
\phantomsection\label{lib.graph:lib.graph.network.set_state}\pysiglinewithargsret{\code{lib.graph.network.}\bfcode{set\_state}}{\emph{state}}{}
Set the internal state of the generator from a tuple.

For use if one has reason to manually (re-)set the internal state of the
``Mersenne Twister''{\color{red}\bfseries{}{[}1{]}\_} pseudo-random number generating algorithm.
\begin{description}
\item[{state}] \leavevmode{[}tuple(str, ndarray of 624 uints, int, int, float){]}
The \emph{state} tuple has the following items:
\begin{enumerate}
\item {} 
the string `MT19937', specifying the Mersenne Twister algorithm.

\item {} 
a 1-D array of 624 unsigned integers \code{keys}.

\item {} 
an integer \code{pos}.

\item {} 
an integer \code{has\_gauss}.

\item {} 
a float \code{cached\_gaussian}.

\end{enumerate}

\end{description}
\begin{description}
\item[{out}] \leavevmode{[}None{]}
Returns `None' on success.

\end{description}

get\_state

\emph{set\_state} and \emph{get\_state} are not needed to work with any of the
random distributions in NumPy. If the internal state is manually altered,
the user should know exactly what he/she is doing.

For backwards compatibility, the form (str, array of 624 uints, int) is
also accepted although it is missing some information about the cached
Gaussian value: \code{state = ('MT19937', keys, pos)}.

\end{fulllineitems}

\index{shuffle() (in module lib.graph.network)}

\begin{fulllineitems}
\phantomsection\label{lib.graph:lib.graph.network.shuffle}\pysiglinewithargsret{\code{lib.graph.network.}\bfcode{shuffle}}{\emph{x}}{}
Modify a sequence in-place by shuffling its contents.
\begin{description}
\item[{x}] \leavevmode{[}array\_like{]}
The array or list to be shuffled.

\end{description}

None

\begin{Verbatim}[commandchars=\\\{\}]
\PYG{g+gp}{\PYGZgt{}\PYGZgt{}\PYGZgt{} }\PYG{n}{arr} \PYG{o}{=} \PYG{n}{np}\PYG{o}{.}\PYG{n}{arange}\PYG{p}{(}\PYG{l+m+mi}{10}\PYG{p}{)}
\PYG{g+gp}{\PYGZgt{}\PYGZgt{}\PYGZgt{} }\PYG{n}{np}\PYG{o}{.}\PYG{n}{random}\PYG{o}{.}\PYG{n}{shuffle}\PYG{p}{(}\PYG{n}{arr}\PYG{p}{)}
\PYG{g+gp}{\PYGZgt{}\PYGZgt{}\PYGZgt{} }\PYG{n}{arr}
\PYG{g+go}{[1 7 5 2 9 4 3 6 0 8]}
\end{Verbatim}

This function only shuffles the array along the first index of a
multi-dimensional array:

\begin{Verbatim}[commandchars=\\\{\}]
\PYG{g+gp}{\PYGZgt{}\PYGZgt{}\PYGZgt{} }\PYG{n}{arr} \PYG{o}{=} \PYG{n}{np}\PYG{o}{.}\PYG{n}{arange}\PYG{p}{(}\PYG{l+m+mi}{9}\PYG{p}{)}\PYG{o}{.}\PYG{n}{reshape}\PYG{p}{(}\PYG{p}{(}\PYG{l+m+mi}{3}\PYG{p}{,} \PYG{l+m+mi}{3}\PYG{p}{)}\PYG{p}{)}
\PYG{g+gp}{\PYGZgt{}\PYGZgt{}\PYGZgt{} }\PYG{n}{np}\PYG{o}{.}\PYG{n}{random}\PYG{o}{.}\PYG{n}{shuffle}\PYG{p}{(}\PYG{n}{arr}\PYG{p}{)}
\PYG{g+gp}{\PYGZgt{}\PYGZgt{}\PYGZgt{} }\PYG{n}{arr}
\PYG{g+go}{array([[3, 4, 5],}
\PYG{g+go}{       [6, 7, 8],}
\PYG{g+go}{       [0, 1, 2]])}
\end{Verbatim}

\end{fulllineitems}

\index{standard\_cauchy() (in module lib.graph.network)}

\begin{fulllineitems}
\phantomsection\label{lib.graph:lib.graph.network.standard_cauchy}\pysiglinewithargsret{\code{lib.graph.network.}\bfcode{standard\_cauchy}}{\emph{size=None}}{}
Standard Cauchy distribution with mode = 0.

Also known as the Lorentz distribution.
\begin{description}
\item[{size}] \leavevmode{[}int or tuple of ints{]}
Shape of the output.

\end{description}
\begin{description}
\item[{samples}] \leavevmode{[}ndarray or scalar{]}
The drawn samples.

\end{description}

The probability density function for the full Cauchy distribution is
\begin{gather}
\begin{split}P(x; x_0, \gamma) = \frac{1}{\pi \gamma \bigl[ 1+
(\frac{x-x_0}{\gamma})^2 \bigr] }\end{split}\notag
\end{gather}
and the Standard Cauchy distribution just sets \(x_0=0\) and
\(\gamma=1\)

The Cauchy distribution arises in the solution to the driven harmonic
oscillator problem, and also describes spectral line broadening. It
also describes the distribution of values at which a line tilted at
a random angle will cut the x axis.

When studying hypothesis tests that assume normality, seeing how the
tests perform on data from a Cauchy distribution is a good indicator of
their sensitivity to a heavy-tailed distribution, since the Cauchy looks
very much like a Gaussian distribution, but with heavier tails.

Draw samples and plot the distribution:

\begin{Verbatim}[commandchars=\\\{\}]
\PYG{g+gp}{\PYGZgt{}\PYGZgt{}\PYGZgt{} }\PYG{n}{s} \PYG{o}{=} \PYG{n}{np}\PYG{o}{.}\PYG{n}{random}\PYG{o}{.}\PYG{n}{standard\PYGZus{}cauchy}\PYG{p}{(}\PYG{l+m+mi}{1000000}\PYG{p}{)}
\PYG{g+gp}{\PYGZgt{}\PYGZgt{}\PYGZgt{} }\PYG{n}{s} \PYG{o}{=} \PYG{n}{s}\PYG{p}{[}\PYG{p}{(}\PYG{n}{s}\PYG{o}{\PYGZgt{}}\PYG{o}{\PYGZhy{}}\PYG{l+m+mi}{25}\PYG{p}{)} \PYG{o}{\PYGZam{}} \PYG{p}{(}\PYG{n}{s}\PYG{o}{\PYGZlt{}}\PYG{l+m+mi}{25}\PYG{p}{)}\PYG{p}{]}  \PYG{c}{\PYGZsh{} truncate distribution so it plots well}
\PYG{g+gp}{\PYGZgt{}\PYGZgt{}\PYGZgt{} }\PYG{n}{plt}\PYG{o}{.}\PYG{n}{hist}\PYG{p}{(}\PYG{n}{s}\PYG{p}{,} \PYG{n}{bins}\PYG{o}{=}\PYG{l+m+mi}{100}\PYG{p}{)}
\PYG{g+gp}{\PYGZgt{}\PYGZgt{}\PYGZgt{} }\PYG{n}{plt}\PYG{o}{.}\PYG{n}{show}\PYG{p}{(}\PYG{p}{)}
\end{Verbatim}

\end{fulllineitems}

\index{standard\_exponential() (in module lib.graph.network)}

\begin{fulllineitems}
\phantomsection\label{lib.graph:lib.graph.network.standard_exponential}\pysiglinewithargsret{\code{lib.graph.network.}\bfcode{standard\_exponential}}{\emph{size=None}}{}
Draw samples from the standard exponential distribution.

\emph{standard\_exponential} is identical to the exponential distribution
with a scale parameter of 1.
\begin{description}
\item[{size}] \leavevmode{[}int or tuple of ints{]}
Shape of the output.

\end{description}
\begin{description}
\item[{out}] \leavevmode{[}float or ndarray{]}
Drawn samples.

\end{description}

Output a 3x8000 array:

\begin{Verbatim}[commandchars=\\\{\}]
\PYG{g+gp}{\PYGZgt{}\PYGZgt{}\PYGZgt{} }\PYG{n}{n} \PYG{o}{=} \PYG{n}{np}\PYG{o}{.}\PYG{n}{random}\PYG{o}{.}\PYG{n}{standard\PYGZus{}exponential}\PYG{p}{(}\PYG{p}{(}\PYG{l+m+mi}{3}\PYG{p}{,} \PYG{l+m+mi}{8000}\PYG{p}{)}\PYG{p}{)}
\end{Verbatim}

\end{fulllineitems}

\index{standard\_gamma() (in module lib.graph.network)}

\begin{fulllineitems}
\phantomsection\label{lib.graph:lib.graph.network.standard_gamma}\pysiglinewithargsret{\code{lib.graph.network.}\bfcode{standard\_gamma}}{\emph{shape}, \emph{size=None}}{}
Draw samples from a Standard Gamma distribution.

Samples are drawn from a Gamma distribution with specified parameters,
shape (sometimes designated ``k'') and scale=1.
\begin{description}
\item[{shape}] \leavevmode{[}float{]}
Parameter, should be \textgreater{} 0.

\item[{size}] \leavevmode{[}int or tuple of ints{]}
Output shape.  If the given shape is, e.g., \code{(m, n, k)}, then
\code{m * n * k} samples are drawn.

\end{description}
\begin{description}
\item[{samples}] \leavevmode{[}ndarray or scalar{]}
The drawn samples.

\end{description}
\begin{description}
\item[{scipy.stats.distributions.gamma}] \leavevmode{[}probability density function,{]}
distribution or cumulative density function, etc.

\end{description}

The probability density for the Gamma distribution is
\begin{gather}
\begin{split}p(x) = x^{k-1}\frac{e^{-x/\theta}}{\theta^k\Gamma(k)},\end{split}\notag
\end{gather}
where \(k\) is the shape and \(\theta\) the scale,
and \(\Gamma\) is the Gamma function.

The Gamma distribution is often used to model the times to failure of
electronic components, and arises naturally in processes for which the
waiting times between Poisson distributed events are relevant.

Draw samples from the distribution:

\begin{Verbatim}[commandchars=\\\{\}]
\PYG{g+gp}{\PYGZgt{}\PYGZgt{}\PYGZgt{} }\PYG{n}{shape}\PYG{p}{,} \PYG{n}{scale} \PYG{o}{=} \PYG{l+m+mf}{2.}\PYG{p}{,} \PYG{l+m+mf}{1.} \PYG{c}{\PYGZsh{} mean and width}
\PYG{g+gp}{\PYGZgt{}\PYGZgt{}\PYGZgt{} }\PYG{n}{s} \PYG{o}{=} \PYG{n}{np}\PYG{o}{.}\PYG{n}{random}\PYG{o}{.}\PYG{n}{standard\PYGZus{}gamma}\PYG{p}{(}\PYG{n}{shape}\PYG{p}{,} \PYG{l+m+mi}{1000000}\PYG{p}{)}
\end{Verbatim}

Display the histogram of the samples, along with
the probability density function:

\begin{Verbatim}[commandchars=\\\{\}]
\PYG{g+gp}{\PYGZgt{}\PYGZgt{}\PYGZgt{} }\PYG{k+kn}{import} \PYG{n+nn}{matplotlib.pyplot} \PYG{k+kn}{as} \PYG{n+nn}{plt}
\PYG{g+gp}{\PYGZgt{}\PYGZgt{}\PYGZgt{} }\PYG{k+kn}{import} \PYG{n+nn}{scipy.special} \PYG{k+kn}{as} \PYG{n+nn}{sps}
\PYG{g+gp}{\PYGZgt{}\PYGZgt{}\PYGZgt{} }\PYG{n}{count}\PYG{p}{,} \PYG{n}{bins}\PYG{p}{,} \PYG{n}{ignored} \PYG{o}{=} \PYG{n}{plt}\PYG{o}{.}\PYG{n}{hist}\PYG{p}{(}\PYG{n}{s}\PYG{p}{,} \PYG{l+m+mi}{50}\PYG{p}{,} \PYG{n}{normed}\PYG{o}{=}\PYG{n+nb+bp}{True}\PYG{p}{)}
\PYG{g+gp}{\PYGZgt{}\PYGZgt{}\PYGZgt{} }\PYG{n}{y} \PYG{o}{=} \PYG{n}{bins}\PYG{o}{*}\PYG{o}{*}\PYG{p}{(}\PYG{n}{shape}\PYG{o}{\PYGZhy{}}\PYG{l+m+mi}{1}\PYG{p}{)} \PYG{o}{*} \PYG{p}{(}\PYG{p}{(}\PYG{n}{np}\PYG{o}{.}\PYG{n}{exp}\PYG{p}{(}\PYG{o}{\PYGZhy{}}\PYG{n}{bins}\PYG{o}{/}\PYG{n}{scale}\PYG{p}{)}\PYG{p}{)}\PYG{o}{/} \PYGZbs{}
\PYG{g+gp}{... }                      \PYG{p}{(}\PYG{n}{sps}\PYG{o}{.}\PYG{n}{gamma}\PYG{p}{(}\PYG{n}{shape}\PYG{p}{)} \PYG{o}{*} \PYG{n}{scale}\PYG{o}{*}\PYG{o}{*}\PYG{n}{shape}\PYG{p}{)}\PYG{p}{)}
\PYG{g+gp}{\PYGZgt{}\PYGZgt{}\PYGZgt{} }\PYG{n}{plt}\PYG{o}{.}\PYG{n}{plot}\PYG{p}{(}\PYG{n}{bins}\PYG{p}{,} \PYG{n}{y}\PYG{p}{,} \PYG{n}{linewidth}\PYG{o}{=}\PYG{l+m+mi}{2}\PYG{p}{,} \PYG{n}{color}\PYG{o}{=}\PYG{l+s}{\PYGZsq{}}\PYG{l+s}{r}\PYG{l+s}{\PYGZsq{}}\PYG{p}{)}
\PYG{g+gp}{\PYGZgt{}\PYGZgt{}\PYGZgt{} }\PYG{n}{plt}\PYG{o}{.}\PYG{n}{show}\PYG{p}{(}\PYG{p}{)}
\end{Verbatim}

\end{fulllineitems}

\index{standard\_normal() (in module lib.graph.network)}

\begin{fulllineitems}
\phantomsection\label{lib.graph:lib.graph.network.standard_normal}\pysiglinewithargsret{\code{lib.graph.network.}\bfcode{standard\_normal}}{\emph{size=None}}{}
Returns samples from a Standard Normal distribution (mean=0, stdev=1).
\begin{description}
\item[{size}] \leavevmode{[}int or tuple of ints, optional{]}
Output shape. Default is None, in which case a single value is
returned.

\end{description}
\begin{description}
\item[{out}] \leavevmode{[}float or ndarray{]}
Drawn samples.

\end{description}

\begin{Verbatim}[commandchars=\\\{\}]
\PYG{g+gp}{\PYGZgt{}\PYGZgt{}\PYGZgt{} }\PYG{n}{s} \PYG{o}{=} \PYG{n}{np}\PYG{o}{.}\PYG{n}{random}\PYG{o}{.}\PYG{n}{standard\PYGZus{}normal}\PYG{p}{(}\PYG{l+m+mi}{8000}\PYG{p}{)}
\PYG{g+gp}{\PYGZgt{}\PYGZgt{}\PYGZgt{} }\PYG{n}{s}
\PYG{g+go}{array([ 0.6888893 ,  0.78096262, \PYGZhy{}0.89086505, ...,  0.49876311, \PYGZsh{}random}
\PYG{g+go}{       \PYGZhy{}0.38672696, \PYGZhy{}0.4685006 ])                               \PYGZsh{}random}
\PYG{g+gp}{\PYGZgt{}\PYGZgt{}\PYGZgt{} }\PYG{n}{s}\PYG{o}{.}\PYG{n}{shape}
\PYG{g+go}{(8000,)}
\PYG{g+gp}{\PYGZgt{}\PYGZgt{}\PYGZgt{} }\PYG{n}{s} \PYG{o}{=} \PYG{n}{np}\PYG{o}{.}\PYG{n}{random}\PYG{o}{.}\PYG{n}{standard\PYGZus{}normal}\PYG{p}{(}\PYG{n}{size}\PYG{o}{=}\PYG{p}{(}\PYG{l+m+mi}{3}\PYG{p}{,} \PYG{l+m+mi}{4}\PYG{p}{,} \PYG{l+m+mi}{2}\PYG{p}{)}\PYG{p}{)}
\PYG{g+gp}{\PYGZgt{}\PYGZgt{}\PYGZgt{} }\PYG{n}{s}\PYG{o}{.}\PYG{n}{shape}
\PYG{g+go}{(3, 4, 2)}
\end{Verbatim}

\end{fulllineitems}

\index{standard\_t() (in module lib.graph.network)}

\begin{fulllineitems}
\phantomsection\label{lib.graph:lib.graph.network.standard_t}\pysiglinewithargsret{\code{lib.graph.network.}\bfcode{standard\_t}}{\emph{df}, \emph{size=None}}{}
Standard Student's t distribution with df degrees of freedom.

A special case of the hyperbolic distribution.
As \emph{df} gets large, the result resembles that of the standard normal
distribution (\emph{standard\_normal}).
\begin{description}
\item[{df}] \leavevmode{[}int{]}
Degrees of freedom, should be \textgreater{} 0.

\item[{size}] \leavevmode{[}int or tuple of ints, optional{]}
Output shape. Default is None, in which case a single value is
returned.

\end{description}
\begin{description}
\item[{samples}] \leavevmode{[}ndarray or scalar{]}
Drawn samples.

\end{description}

The probability density function for the t distribution is
\begin{gather}
\begin{split}P(x, df) = \frac{\Gamma(\frac{df+1}{2})}{\sqrt{\pi df}
\Gamma(\frac{df}{2})}\Bigl( 1+\frac{x^2}{df} \Bigr)^{-(df+1)/2}\end{split}\notag
\end{gather}
The t test is based on an assumption that the data come from a Normal
distribution. The t test provides a way to test whether the sample mean
(that is the mean calculated from the data) is a good estimate of the true
mean.

The derivation of the t-distribution was forst published in 1908 by William
Gisset while working for the Guinness Brewery in Dublin. Due to proprietary
issues, he had to publish under a pseudonym, and so he used the name
Student.

From Dalgaard page 83 {\color{red}\bfseries{}{[}1{]}\_}, suppose the daily energy intake for 11
women in Kj is:

\begin{Verbatim}[commandchars=\\\{\}]
\PYG{g+gp}{\PYGZgt{}\PYGZgt{}\PYGZgt{} }\PYG{n}{intake} \PYG{o}{=} \PYG{n}{np}\PYG{o}{.}\PYG{n}{array}\PYG{p}{(}\PYG{p}{[}\PYG{l+m+mf}{5260.}\PYG{p}{,} \PYG{l+m+mi}{5470}\PYG{p}{,} \PYG{l+m+mi}{5640}\PYG{p}{,} \PYG{l+m+mi}{6180}\PYG{p}{,} \PYG{l+m+mi}{6390}\PYG{p}{,} \PYG{l+m+mi}{6515}\PYG{p}{,} \PYG{l+m+mi}{6805}\PYG{p}{,} \PYG{l+m+mi}{7515}\PYG{p}{,} \PYGZbs{}
\PYG{g+gp}{... }                   \PYG{l+m+mi}{7515}\PYG{p}{,} \PYG{l+m+mi}{8230}\PYG{p}{,} \PYG{l+m+mi}{8770}\PYG{p}{]}\PYG{p}{)}
\end{Verbatim}

Does their energy intake deviate systematically from the recommended
value of 7725 kJ?

We have 10 degrees of freedom, so is the sample mean within 95\% of the
recommended value?

\begin{Verbatim}[commandchars=\\\{\}]
\PYG{g+gp}{\PYGZgt{}\PYGZgt{}\PYGZgt{} }\PYG{n}{s} \PYG{o}{=} \PYG{n}{np}\PYG{o}{.}\PYG{n}{random}\PYG{o}{.}\PYG{n}{standard\PYGZus{}t}\PYG{p}{(}\PYG{l+m+mi}{10}\PYG{p}{,} \PYG{n}{size}\PYG{o}{=}\PYG{l+m+mi}{100000}\PYG{p}{)}
\PYG{g+gp}{\PYGZgt{}\PYGZgt{}\PYGZgt{} }\PYG{n}{np}\PYG{o}{.}\PYG{n}{mean}\PYG{p}{(}\PYG{n}{intake}\PYG{p}{)}
\PYG{g+go}{6753.636363636364}
\PYG{g+gp}{\PYGZgt{}\PYGZgt{}\PYGZgt{} }\PYG{n}{intake}\PYG{o}{.}\PYG{n}{std}\PYG{p}{(}\PYG{n}{ddof}\PYG{o}{=}\PYG{l+m+mi}{1}\PYG{p}{)}
\PYG{g+go}{1142.1232221373727}
\end{Verbatim}

Calculate the t statistic, setting the ddof parameter to the unbiased
value so the divisor in the standard deviation will be degrees of
freedom, N-1.

\begin{Verbatim}[commandchars=\\\{\}]
\PYG{g+gp}{\PYGZgt{}\PYGZgt{}\PYGZgt{} }\PYG{n}{t} \PYG{o}{=} \PYG{p}{(}\PYG{n}{np}\PYG{o}{.}\PYG{n}{mean}\PYG{p}{(}\PYG{n}{intake}\PYG{p}{)}\PYG{o}{\PYGZhy{}}\PYG{l+m+mi}{7725}\PYG{p}{)}\PYG{o}{/}\PYG{p}{(}\PYG{n}{intake}\PYG{o}{.}\PYG{n}{std}\PYG{p}{(}\PYG{n}{ddof}\PYG{o}{=}\PYG{l+m+mi}{1}\PYG{p}{)}\PYG{o}{/}\PYG{n}{np}\PYG{o}{.}\PYG{n}{sqrt}\PYG{p}{(}\PYG{n+nb}{len}\PYG{p}{(}\PYG{n}{intake}\PYG{p}{)}\PYG{p}{)}\PYG{p}{)}
\PYG{g+gp}{\PYGZgt{}\PYGZgt{}\PYGZgt{} }\PYG{k+kn}{import} \PYG{n+nn}{matplotlib.pyplot} \PYG{k+kn}{as} \PYG{n+nn}{plt}
\PYG{g+gp}{\PYGZgt{}\PYGZgt{}\PYGZgt{} }\PYG{n}{h} \PYG{o}{=} \PYG{n}{plt}\PYG{o}{.}\PYG{n}{hist}\PYG{p}{(}\PYG{n}{s}\PYG{p}{,} \PYG{n}{bins}\PYG{o}{=}\PYG{l+m+mi}{100}\PYG{p}{,} \PYG{n}{normed}\PYG{o}{=}\PYG{n+nb+bp}{True}\PYG{p}{)}
\end{Verbatim}

For a one-sided t-test, how far out in the distribution does the t
statistic appear?

\begin{Verbatim}[commandchars=\\\{\}]
\PYG{g+gp}{\PYGZgt{}\PYGZgt{}\PYGZgt{} }\PYG{o}{\PYGZgt{}\PYGZgt{}}\PYG{o}{\PYGZgt{}} \PYG{n}{np}\PYG{o}{.}\PYG{n}{sum}\PYG{p}{(}\PYG{n}{s}\PYG{o}{\PYGZlt{}}\PYG{n}{t}\PYG{p}{)} \PYG{o}{/} \PYG{n+nb}{float}\PYG{p}{(}\PYG{n+nb}{len}\PYG{p}{(}\PYG{n}{s}\PYG{p}{)}\PYG{p}{)}
\PYG{g+go}{0.0090699999999999999  \PYGZsh{}random}
\end{Verbatim}

So the p-value is about 0.009, which says the null hypothesis has a
probability of about 99\% of being true.

\end{fulllineitems}

\index{triangular() (in module lib.graph.network)}

\begin{fulllineitems}
\phantomsection\label{lib.graph:lib.graph.network.triangular}\pysiglinewithargsret{\code{lib.graph.network.}\bfcode{triangular}}{\emph{left}, \emph{mode}, \emph{right}, \emph{size=None}}{}
Draw samples from the triangular distribution.

The triangular distribution is a continuous probability distribution with
lower limit left, peak at mode, and upper limit right. Unlike the other
distributions, these parameters directly define the shape of the pdf.
\begin{description}
\item[{left}] \leavevmode{[}scalar{]}
Lower limit.

\item[{mode}] \leavevmode{[}scalar{]}
The value where the peak of the distribution occurs.
The value should fulfill the condition \code{left \textless{}= mode \textless{}= right}.

\item[{right}] \leavevmode{[}scalar{]}
Upper limit, should be larger than \emph{left}.

\item[{size}] \leavevmode{[}int or tuple of ints, optional{]}
Output shape. Default is None, in which case a single value is
returned.

\end{description}
\begin{description}
\item[{samples}] \leavevmode{[}ndarray or scalar{]}
The returned samples all lie in the interval {[}left, right{]}.

\end{description}

The probability density function for the Triangular distribution is
\begin{gather}
\begin{split}P(x;l, m, r) = \begin{cases}
\frac{2(x-l)}{(r-l)(m-l)}& \text{for $l \leq x \leq m$},\\
\frac{2(m-x)}{(r-l)(r-m)}& \text{for $m \leq x \leq r$},\\
0& \text{otherwise}.
\end{cases}\end{split}\notag
\end{gather}
The triangular distribution is often used in ill-defined problems where the
underlying distribution is not known, but some knowledge of the limits and
mode exists. Often it is used in simulations.

Draw values from the distribution and plot the histogram:

\begin{Verbatim}[commandchars=\\\{\}]
\PYG{g+gp}{\PYGZgt{}\PYGZgt{}\PYGZgt{} }\PYG{k+kn}{import} \PYG{n+nn}{matplotlib.pyplot} \PYG{k+kn}{as} \PYG{n+nn}{plt}
\PYG{g+gp}{\PYGZgt{}\PYGZgt{}\PYGZgt{} }\PYG{n}{h} \PYG{o}{=} \PYG{n}{plt}\PYG{o}{.}\PYG{n}{hist}\PYG{p}{(}\PYG{n}{np}\PYG{o}{.}\PYG{n}{random}\PYG{o}{.}\PYG{n}{triangular}\PYG{p}{(}\PYG{o}{\PYGZhy{}}\PYG{l+m+mi}{3}\PYG{p}{,} \PYG{l+m+mi}{0}\PYG{p}{,} \PYG{l+m+mi}{8}\PYG{p}{,} \PYG{l+m+mi}{100000}\PYG{p}{)}\PYG{p}{,} \PYG{n}{bins}\PYG{o}{=}\PYG{l+m+mi}{200}\PYG{p}{,}
\PYG{g+gp}{... }             \PYG{n}{normed}\PYG{o}{=}\PYG{n+nb+bp}{True}\PYG{p}{)}
\PYG{g+gp}{\PYGZgt{}\PYGZgt{}\PYGZgt{} }\PYG{n}{plt}\PYG{o}{.}\PYG{n}{show}\PYG{p}{(}\PYG{p}{)}
\end{Verbatim}

\end{fulllineitems}

\index{uniform() (in module lib.graph.network)}

\begin{fulllineitems}
\phantomsection\label{lib.graph:lib.graph.network.uniform}\pysiglinewithargsret{\code{lib.graph.network.}\bfcode{uniform}}{\emph{low=0.0}, \emph{high=1.0}, \emph{size=1}}{}
Draw samples from a uniform distribution.

Samples are uniformly distributed over the half-open interval
\code{{[}low, high)} (includes low, but excludes high).  In other words,
any value within the given interval is equally likely to be drawn
by \emph{uniform}.
\begin{description}
\item[{low}] \leavevmode{[}float, optional{]}
Lower boundary of the output interval.  All values generated will be
greater than or equal to low.  The default value is 0.

\item[{high}] \leavevmode{[}float{]}
Upper boundary of the output interval.  All values generated will be
less than high.  The default value is 1.0.

\item[{size}] \leavevmode{[}int or tuple of ints, optional{]}
Shape of output.  If the given size is, for example, (m,n,k),
m*n*k samples are generated.  If no shape is specified, a single sample
is returned.

\end{description}
\begin{description}
\item[{out}] \leavevmode{[}ndarray{]}
Drawn samples, with shape \emph{size}.

\end{description}

randint : Discrete uniform distribution, yielding integers.
random\_integers : Discrete uniform distribution over the closed
\begin{quote}

interval \code{{[}low, high{]}}.
\end{quote}

random\_sample : Floats uniformly distributed over \code{{[}0, 1)}.
random : Alias for \emph{random\_sample}.
rand : Convenience function that accepts dimensions as input, e.g.,
\begin{quote}

\code{rand(2,2)} would generate a 2-by-2 array of floats,
uniformly distributed over \code{{[}0, 1)}.
\end{quote}

The probability density function of the uniform distribution is
\begin{gather}
\begin{split}p(x) = \frac{1}{b - a}\end{split}\notag
\end{gather}
anywhere within the interval \code{{[}a, b)}, and zero elsewhere.

Draw samples from the distribution:

\begin{Verbatim}[commandchars=\\\{\}]
\PYG{g+gp}{\PYGZgt{}\PYGZgt{}\PYGZgt{} }\PYG{n}{s} \PYG{o}{=} \PYG{n}{np}\PYG{o}{.}\PYG{n}{random}\PYG{o}{.}\PYG{n}{uniform}\PYG{p}{(}\PYG{o}{\PYGZhy{}}\PYG{l+m+mi}{1}\PYG{p}{,}\PYG{l+m+mi}{0}\PYG{p}{,}\PYG{l+m+mi}{1000}\PYG{p}{)}
\end{Verbatim}

All values are within the given interval:

\begin{Verbatim}[commandchars=\\\{\}]
\PYG{g+gp}{\PYGZgt{}\PYGZgt{}\PYGZgt{} }\PYG{n}{np}\PYG{o}{.}\PYG{n}{all}\PYG{p}{(}\PYG{n}{s} \PYG{o}{\PYGZgt{}}\PYG{o}{=} \PYG{o}{\PYGZhy{}}\PYG{l+m+mi}{1}\PYG{p}{)}
\PYG{g+go}{True}
\PYG{g+gp}{\PYGZgt{}\PYGZgt{}\PYGZgt{} }\PYG{n}{np}\PYG{o}{.}\PYG{n}{all}\PYG{p}{(}\PYG{n}{s} \PYG{o}{\PYGZlt{}} \PYG{l+m+mi}{0}\PYG{p}{)}
\PYG{g+go}{True}
\end{Verbatim}

Display the histogram of the samples, along with the
probability density function:

\begin{Verbatim}[commandchars=\\\{\}]
\PYG{g+gp}{\PYGZgt{}\PYGZgt{}\PYGZgt{} }\PYG{k+kn}{import} \PYG{n+nn}{matplotlib.pyplot} \PYG{k+kn}{as} \PYG{n+nn}{plt}
\PYG{g+gp}{\PYGZgt{}\PYGZgt{}\PYGZgt{} }\PYG{n}{count}\PYG{p}{,} \PYG{n}{bins}\PYG{p}{,} \PYG{n}{ignored} \PYG{o}{=} \PYG{n}{plt}\PYG{o}{.}\PYG{n}{hist}\PYG{p}{(}\PYG{n}{s}\PYG{p}{,} \PYG{l+m+mi}{15}\PYG{p}{,} \PYG{n}{normed}\PYG{o}{=}\PYG{n+nb+bp}{True}\PYG{p}{)}
\PYG{g+gp}{\PYGZgt{}\PYGZgt{}\PYGZgt{} }\PYG{n}{plt}\PYG{o}{.}\PYG{n}{plot}\PYG{p}{(}\PYG{n}{bins}\PYG{p}{,} \PYG{n}{np}\PYG{o}{.}\PYG{n}{ones\PYGZus{}like}\PYG{p}{(}\PYG{n}{bins}\PYG{p}{)}\PYG{p}{,} \PYG{n}{linewidth}\PYG{o}{=}\PYG{l+m+mi}{2}\PYG{p}{,} \PYG{n}{color}\PYG{o}{=}\PYG{l+s}{\PYGZsq{}}\PYG{l+s}{r}\PYG{l+s}{\PYGZsq{}}\PYG{p}{)}
\PYG{g+gp}{\PYGZgt{}\PYGZgt{}\PYGZgt{} }\PYG{n}{plt}\PYG{o}{.}\PYG{n}{show}\PYG{p}{(}\PYG{p}{)}
\end{Verbatim}

\end{fulllineitems}

\index{vonmises() (in module lib.graph.network)}

\begin{fulllineitems}
\phantomsection\label{lib.graph:lib.graph.network.vonmises}\pysiglinewithargsret{\code{lib.graph.network.}\bfcode{vonmises}}{\emph{mu}, \emph{kappa}, \emph{size=None}}{}
Draw samples from a von Mises distribution.

Samples are drawn from a von Mises distribution with specified mode
(mu) and dispersion (kappa), on the interval {[}-pi, pi{]}.

The von Mises distribution (also known as the circular normal
distribution) is a continuous probability distribution on the unit
circle.  It may be thought of as the circular analogue of the normal
distribution.
\begin{description}
\item[{mu}] \leavevmode{[}float{]}
Mode (``center'') of the distribution.

\item[{kappa}] \leavevmode{[}float{]}
Dispersion of the distribution, has to be \textgreater{}=0.

\item[{size}] \leavevmode{[}int or tuple of int{]}
Output shape.  If the given shape is, e.g., \code{(m, n, k)}, then
\code{m * n * k} samples are drawn.

\end{description}
\begin{description}
\item[{samples}] \leavevmode{[}scalar or ndarray{]}
The returned samples, which are in the interval {[}-pi, pi{]}.

\end{description}
\begin{description}
\item[{scipy.stats.distributions.vonmises}] \leavevmode{[}probability density function,{]}
distribution, or cumulative density function, etc.

\end{description}

The probability density for the von Mises distribution is
\begin{gather}
\begin{split}p(x) = \frac{e^{\kappa cos(x-\mu)}}{2\pi I_0(\kappa)},\end{split}\notag
\end{gather}
where \(\mu\) is the mode and \(\kappa\) the dispersion,
and \(I_0(\kappa)\) is the modified Bessel function of order 0.

The von Mises is named for Richard Edler von Mises, who was born in
Austria-Hungary, in what is now the Ukraine.  He fled to the United
States in 1939 and became a professor at Harvard.  He worked in
probability theory, aerodynamics, fluid mechanics, and philosophy of
science.

Abramowitz, M. and Stegun, I. A. (ed.), \emph{Handbook of Mathematical
Functions}, New York: Dover, 1965.

von Mises, R., \emph{Mathematical Theory of Probability and Statistics},
New York: Academic Press, 1964.

Draw samples from the distribution:

\begin{Verbatim}[commandchars=\\\{\}]
\PYG{g+gp}{\PYGZgt{}\PYGZgt{}\PYGZgt{} }\PYG{n}{mu}\PYG{p}{,} \PYG{n}{kappa} \PYG{o}{=} \PYG{l+m+mf}{0.0}\PYG{p}{,} \PYG{l+m+mf}{4.0} \PYG{c}{\PYGZsh{} mean and dispersion}
\PYG{g+gp}{\PYGZgt{}\PYGZgt{}\PYGZgt{} }\PYG{n}{s} \PYG{o}{=} \PYG{n}{np}\PYG{o}{.}\PYG{n}{random}\PYG{o}{.}\PYG{n}{vonmises}\PYG{p}{(}\PYG{n}{mu}\PYG{p}{,} \PYG{n}{kappa}\PYG{p}{,} \PYG{l+m+mi}{1000}\PYG{p}{)}
\end{Verbatim}

Display the histogram of the samples, along with
the probability density function:

\begin{Verbatim}[commandchars=\\\{\}]
\PYG{g+gp}{\PYGZgt{}\PYGZgt{}\PYGZgt{} }\PYG{k+kn}{import} \PYG{n+nn}{matplotlib.pyplot} \PYG{k+kn}{as} \PYG{n+nn}{plt}
\PYG{g+gp}{\PYGZgt{}\PYGZgt{}\PYGZgt{} }\PYG{k+kn}{import} \PYG{n+nn}{scipy.special} \PYG{k+kn}{as} \PYG{n+nn}{sps}
\PYG{g+gp}{\PYGZgt{}\PYGZgt{}\PYGZgt{} }\PYG{n}{count}\PYG{p}{,} \PYG{n}{bins}\PYG{p}{,} \PYG{n}{ignored} \PYG{o}{=} \PYG{n}{plt}\PYG{o}{.}\PYG{n}{hist}\PYG{p}{(}\PYG{n}{s}\PYG{p}{,} \PYG{l+m+mi}{50}\PYG{p}{,} \PYG{n}{normed}\PYG{o}{=}\PYG{n+nb+bp}{True}\PYG{p}{)}
\PYG{g+gp}{\PYGZgt{}\PYGZgt{}\PYGZgt{} }\PYG{n}{x} \PYG{o}{=} \PYG{n}{np}\PYG{o}{.}\PYG{n}{arange}\PYG{p}{(}\PYG{o}{\PYGZhy{}}\PYG{n}{np}\PYG{o}{.}\PYG{n}{pi}\PYG{p}{,} \PYG{n}{np}\PYG{o}{.}\PYG{n}{pi}\PYG{p}{,} \PYG{l+m+mi}{2}\PYG{o}{*}\PYG{n}{np}\PYG{o}{.}\PYG{n}{pi}\PYG{o}{/}\PYG{l+m+mf}{50.}\PYG{p}{)}
\PYG{g+gp}{\PYGZgt{}\PYGZgt{}\PYGZgt{} }\PYG{n}{y} \PYG{o}{=} \PYG{o}{\PYGZhy{}}\PYG{n}{np}\PYG{o}{.}\PYG{n}{exp}\PYG{p}{(}\PYG{n}{kappa}\PYG{o}{*}\PYG{n}{np}\PYG{o}{.}\PYG{n}{cos}\PYG{p}{(}\PYG{n}{x}\PYG{o}{\PYGZhy{}}\PYG{n}{mu}\PYG{p}{)}\PYG{p}{)}\PYG{o}{/}\PYG{p}{(}\PYG{l+m+mi}{2}\PYG{o}{*}\PYG{n}{np}\PYG{o}{.}\PYG{n}{pi}\PYG{o}{*}\PYG{n}{sps}\PYG{o}{.}\PYG{n}{jn}\PYG{p}{(}\PYG{l+m+mi}{0}\PYG{p}{,}\PYG{n}{kappa}\PYG{p}{)}\PYG{p}{)}
\PYG{g+gp}{\PYGZgt{}\PYGZgt{}\PYGZgt{} }\PYG{n}{plt}\PYG{o}{.}\PYG{n}{plot}\PYG{p}{(}\PYG{n}{x}\PYG{p}{,} \PYG{n}{y}\PYG{o}{/}\PYG{n+nb}{max}\PYG{p}{(}\PYG{n}{y}\PYG{p}{)}\PYG{p}{,} \PYG{n}{linewidth}\PYG{o}{=}\PYG{l+m+mi}{2}\PYG{p}{,} \PYG{n}{color}\PYG{o}{=}\PYG{l+s}{\PYGZsq{}}\PYG{l+s}{r}\PYG{l+s}{\PYGZsq{}}\PYG{p}{)}
\PYG{g+gp}{\PYGZgt{}\PYGZgt{}\PYGZgt{} }\PYG{n}{plt}\PYG{o}{.}\PYG{n}{show}\PYG{p}{(}\PYG{p}{)}
\end{Verbatim}

\end{fulllineitems}

\index{wald() (in module lib.graph.network)}

\begin{fulllineitems}
\phantomsection\label{lib.graph:lib.graph.network.wald}\pysiglinewithargsret{\code{lib.graph.network.}\bfcode{wald}}{\emph{mean}, \emph{scale}, \emph{size=None}}{}
Draw samples from a Wald, or Inverse Gaussian, distribution.

As the scale approaches infinity, the distribution becomes more like a
Gaussian.

Some references claim that the Wald is an Inverse Gaussian with mean=1, but
this is by no means universal.

The Inverse Gaussian distribution was first studied in relationship to
Brownian motion. In 1956 M.C.K. Tweedie used the name Inverse Gaussian
because there is an inverse relationship between the time to cover a unit
distance and distance covered in unit time.
\begin{description}
\item[{mean}] \leavevmode{[}scalar{]}
Distribution mean, should be \textgreater{} 0.

\item[{scale}] \leavevmode{[}scalar{]}
Scale parameter, should be \textgreater{}= 0.

\item[{size}] \leavevmode{[}int or tuple of ints, optional{]}
Output shape. Default is None, in which case a single value is
returned.

\end{description}
\begin{description}
\item[{samples}] \leavevmode{[}ndarray or scalar{]}
Drawn sample, all greater than zero.

\end{description}

The probability density function for the Wald distribution is
\begin{gather}
\begin{split}P(x;mean,scale) = \sqrt{\frac{scale}{2\pi x^3}}e^
\frac{-scale(x-mean)^2}{2\cdotp mean^2x}\end{split}\notag
\end{gather}
As noted above the Inverse Gaussian distribution first arise from attempts
to model Brownian Motion. It is also a competitor to the Weibull for use in
reliability modeling and modeling stock returns and interest rate
processes.

Draw values from the distribution and plot the histogram:

\begin{Verbatim}[commandchars=\\\{\}]
\PYG{g+gp}{\PYGZgt{}\PYGZgt{}\PYGZgt{} }\PYG{k+kn}{import} \PYG{n+nn}{matplotlib.pyplot} \PYG{k+kn}{as} \PYG{n+nn}{plt}
\PYG{g+gp}{\PYGZgt{}\PYGZgt{}\PYGZgt{} }\PYG{n}{h} \PYG{o}{=} \PYG{n}{plt}\PYG{o}{.}\PYG{n}{hist}\PYG{p}{(}\PYG{n}{np}\PYG{o}{.}\PYG{n}{random}\PYG{o}{.}\PYG{n}{wald}\PYG{p}{(}\PYG{l+m+mi}{3}\PYG{p}{,} \PYG{l+m+mi}{2}\PYG{p}{,} \PYG{l+m+mi}{100000}\PYG{p}{)}\PYG{p}{,} \PYG{n}{bins}\PYG{o}{=}\PYG{l+m+mi}{200}\PYG{p}{,} \PYG{n}{normed}\PYG{o}{=}\PYG{n+nb+bp}{True}\PYG{p}{)}
\PYG{g+gp}{\PYGZgt{}\PYGZgt{}\PYGZgt{} }\PYG{n}{plt}\PYG{o}{.}\PYG{n}{show}\PYG{p}{(}\PYG{p}{)}
\end{Verbatim}

\end{fulllineitems}

\index{weibull() (in module lib.graph.network)}

\begin{fulllineitems}
\phantomsection\label{lib.graph:lib.graph.network.weibull}\pysiglinewithargsret{\code{lib.graph.network.}\bfcode{weibull}}{\emph{a}, \emph{size=None}}{}
Weibull distribution.

Draw samples from a 1-parameter Weibull distribution with the given
shape parameter \emph{a}.
\begin{gather}
\begin{split}X = (-ln(U))^{1/a}\end{split}\notag
\end{gather}
Here, U is drawn from the uniform distribution over (0,1{]}.

The more common 2-parameter Weibull, including a scale parameter
\(\lambda\) is just \(X = \lambda(-ln(U))^{1/a}\).
\begin{description}
\item[{a}] \leavevmode{[}float{]}
Shape of the distribution.

\item[{size}] \leavevmode{[}tuple of ints{]}
Output shape.  If the given shape is, e.g., \code{(m, n, k)}, then
\code{m * n * k} samples are drawn.

\end{description}

scipy.stats.distributions.weibull\_max
scipy.stats.distributions.weibull\_min
scipy.stats.distributions.genextreme
gumbel

The Weibull (or Type III asymptotic extreme value distribution for smallest
values, SEV Type III, or Rosin-Rammler distribution) is one of a class of
Generalized Extreme Value (GEV) distributions used in modeling extreme
value problems.  This class includes the Gumbel and Frechet distributions.

The probability density for the Weibull distribution is
\begin{gather}
\begin{split}p(x) = \frac{a}
{\lambda}(\frac{x}{\lambda})^{a-1}e^{-(x/\lambda)^a},\end{split}\notag
\end{gather}
where \(a\) is the shape and \(\lambda\) the scale.

The function has its peak (the mode) at
\(\lambda(\frac{a-1}{a})^{1/a}\).

When \code{a = 1}, the Weibull distribution reduces to the exponential
distribution.

Draw samples from the distribution:

\begin{Verbatim}[commandchars=\\\{\}]
\PYG{g+gp}{\PYGZgt{}\PYGZgt{}\PYGZgt{} }\PYG{n}{a} \PYG{o}{=} \PYG{l+m+mf}{5.} \PYG{c}{\PYGZsh{} shape}
\PYG{g+gp}{\PYGZgt{}\PYGZgt{}\PYGZgt{} }\PYG{n}{s} \PYG{o}{=} \PYG{n}{np}\PYG{o}{.}\PYG{n}{random}\PYG{o}{.}\PYG{n}{weibull}\PYG{p}{(}\PYG{n}{a}\PYG{p}{,} \PYG{l+m+mi}{1000}\PYG{p}{)}
\end{Verbatim}

Display the histogram of the samples, along with
the probability density function:

\begin{Verbatim}[commandchars=\\\{\}]
\PYG{g+gp}{\PYGZgt{}\PYGZgt{}\PYGZgt{} }\PYG{k+kn}{import} \PYG{n+nn}{matplotlib.pyplot} \PYG{k+kn}{as} \PYG{n+nn}{plt}
\PYG{g+gp}{\PYGZgt{}\PYGZgt{}\PYGZgt{} }\PYG{n}{x} \PYG{o}{=} \PYG{n}{np}\PYG{o}{.}\PYG{n}{arange}\PYG{p}{(}\PYG{l+m+mi}{1}\PYG{p}{,}\PYG{l+m+mf}{100.}\PYG{p}{)}\PYG{o}{/}\PYG{l+m+mf}{50.}
\PYG{g+gp}{\PYGZgt{}\PYGZgt{}\PYGZgt{} }\PYG{k}{def} \PYG{n+nf}{weib}\PYG{p}{(}\PYG{n}{x}\PYG{p}{,}\PYG{n}{n}\PYG{p}{,}\PYG{n}{a}\PYG{p}{)}\PYG{p}{:}
\PYG{g+gp}{... }    \PYG{k}{return} \PYG{p}{(}\PYG{n}{a} \PYG{o}{/} \PYG{n}{n}\PYG{p}{)} \PYG{o}{*} \PYG{p}{(}\PYG{n}{x} \PYG{o}{/} \PYG{n}{n}\PYG{p}{)}\PYG{o}{*}\PYG{o}{*}\PYG{p}{(}\PYG{n}{a} \PYG{o}{\PYGZhy{}} \PYG{l+m+mi}{1}\PYG{p}{)} \PYG{o}{*} \PYG{n}{np}\PYG{o}{.}\PYG{n}{exp}\PYG{p}{(}\PYG{o}{\PYGZhy{}}\PYG{p}{(}\PYG{n}{x} \PYG{o}{/} \PYG{n}{n}\PYG{p}{)}\PYG{o}{*}\PYG{o}{*}\PYG{n}{a}\PYG{p}{)}
\end{Verbatim}

\begin{Verbatim}[commandchars=\\\{\}]
\PYG{g+gp}{\PYGZgt{}\PYGZgt{}\PYGZgt{} }\PYG{n}{count}\PYG{p}{,} \PYG{n}{bins}\PYG{p}{,} \PYG{n}{ignored} \PYG{o}{=} \PYG{n}{plt}\PYG{o}{.}\PYG{n}{hist}\PYG{p}{(}\PYG{n}{np}\PYG{o}{.}\PYG{n}{random}\PYG{o}{.}\PYG{n}{weibull}\PYG{p}{(}\PYG{l+m+mf}{5.}\PYG{p}{,}\PYG{l+m+mi}{1000}\PYG{p}{)}\PYG{p}{)}
\PYG{g+gp}{\PYGZgt{}\PYGZgt{}\PYGZgt{} }\PYG{n}{x} \PYG{o}{=} \PYG{n}{np}\PYG{o}{.}\PYG{n}{arange}\PYG{p}{(}\PYG{l+m+mi}{1}\PYG{p}{,}\PYG{l+m+mf}{100.}\PYG{p}{)}\PYG{o}{/}\PYG{l+m+mf}{50.}
\PYG{g+gp}{\PYGZgt{}\PYGZgt{}\PYGZgt{} }\PYG{n}{scale} \PYG{o}{=} \PYG{n}{count}\PYG{o}{.}\PYG{n}{max}\PYG{p}{(}\PYG{p}{)}\PYG{o}{/}\PYG{n}{weib}\PYG{p}{(}\PYG{n}{x}\PYG{p}{,} \PYG{l+m+mf}{1.}\PYG{p}{,} \PYG{l+m+mf}{5.}\PYG{p}{)}\PYG{o}{.}\PYG{n}{max}\PYG{p}{(}\PYG{p}{)}
\PYG{g+gp}{\PYGZgt{}\PYGZgt{}\PYGZgt{} }\PYG{n}{plt}\PYG{o}{.}\PYG{n}{plot}\PYG{p}{(}\PYG{n}{x}\PYG{p}{,} \PYG{n}{weib}\PYG{p}{(}\PYG{n}{x}\PYG{p}{,} \PYG{l+m+mf}{1.}\PYG{p}{,} \PYG{l+m+mf}{5.}\PYG{p}{)}\PYG{o}{*}\PYG{n}{scale}\PYG{p}{)}
\PYG{g+gp}{\PYGZgt{}\PYGZgt{}\PYGZgt{} }\PYG{n}{plt}\PYG{o}{.}\PYG{n}{show}\PYG{p}{(}\PYG{p}{)}
\end{Verbatim}

\end{fulllineitems}

\index{zipf() (in module lib.graph.network)}

\begin{fulllineitems}
\phantomsection\label{lib.graph:lib.graph.network.zipf}\pysiglinewithargsret{\code{lib.graph.network.}\bfcode{zipf}}{\emph{a}, \emph{size=None}}{}
Draw samples from a Zipf distribution.

Samples are drawn from a Zipf distribution with specified parameter
\emph{a} \textgreater{} 1.

The Zipf distribution (also known as the zeta distribution) is a
continuous probability distribution that satisfies Zipf's law: the
frequency of an item is inversely proportional to its rank in a
frequency table.
\begin{description}
\item[{a}] \leavevmode{[}float \textgreater{} 1{]}
Distribution parameter.

\item[{size}] \leavevmode{[}int or tuple of int, optional{]}
Output shape.  If the given shape is, e.g., \code{(m, n, k)}, then
\code{m * n * k} samples are drawn; a single integer is equivalent in
its result to providing a mono-tuple, i.e., a 1-D array of length
\emph{size} is returned.  The default is None, in which case a single
scalar is returned.

\end{description}
\begin{description}
\item[{samples}] \leavevmode{[}scalar or ndarray{]}
The returned samples are greater than or equal to one.

\end{description}
\begin{description}
\item[{scipy.stats.distributions.zipf}] \leavevmode{[}probability density function,{]}
distribution, or cumulative density function, etc.

\end{description}

The probability density for the Zipf distribution is
\begin{gather}
\begin{split}p(x) = \frac{x^{-a}}{\zeta(a)},\end{split}\notag
\end{gather}
where \(\zeta\) is the Riemann Zeta function.

It is named for the American linguist George Kingsley Zipf, who noted
that the frequency of any word in a sample of a language is inversely
proportional to its rank in the frequency table.

Zipf, G. K., \emph{Selected Studies of the Principle of Relative Frequency
in Language}, Cambridge, MA: Harvard Univ. Press, 1932.

Draw samples from the distribution:

\begin{Verbatim}[commandchars=\\\{\}]
\PYG{g+gp}{\PYGZgt{}\PYGZgt{}\PYGZgt{} }\PYG{n}{a} \PYG{o}{=} \PYG{l+m+mf}{2.} \PYG{c}{\PYGZsh{} parameter}
\PYG{g+gp}{\PYGZgt{}\PYGZgt{}\PYGZgt{} }\PYG{n}{s} \PYG{o}{=} \PYG{n}{np}\PYG{o}{.}\PYG{n}{random}\PYG{o}{.}\PYG{n}{zipf}\PYG{p}{(}\PYG{n}{a}\PYG{p}{,} \PYG{l+m+mi}{1000}\PYG{p}{)}
\end{Verbatim}

Display the histogram of the samples, along with
the probability density function:

\begin{Verbatim}[commandchars=\\\{\}]
\PYG{g+gp}{\PYGZgt{}\PYGZgt{}\PYGZgt{} }\PYG{k+kn}{import} \PYG{n+nn}{matplotlib.pyplot} \PYG{k+kn}{as} \PYG{n+nn}{plt}
\PYG{g+gp}{\PYGZgt{}\PYGZgt{}\PYGZgt{} }\PYG{k+kn}{import} \PYG{n+nn}{scipy.special} \PYG{k+kn}{as} \PYG{n+nn}{sps}
\PYG{g+go}{Truncate s values at 50 so plot is interesting}
\PYG{g+gp}{\PYGZgt{}\PYGZgt{}\PYGZgt{} }\PYG{n}{count}\PYG{p}{,} \PYG{n}{bins}\PYG{p}{,} \PYG{n}{ignored} \PYG{o}{=} \PYG{n}{plt}\PYG{o}{.}\PYG{n}{hist}\PYG{p}{(}\PYG{n}{s}\PYG{p}{[}\PYG{n}{s}\PYG{o}{\PYGZlt{}}\PYG{l+m+mi}{50}\PYG{p}{]}\PYG{p}{,} \PYG{l+m+mi}{50}\PYG{p}{,} \PYG{n}{normed}\PYG{o}{=}\PYG{n+nb+bp}{True}\PYG{p}{)}
\PYG{g+gp}{\PYGZgt{}\PYGZgt{}\PYGZgt{} }\PYG{n}{x} \PYG{o}{=} \PYG{n}{np}\PYG{o}{.}\PYG{n}{arange}\PYG{p}{(}\PYG{l+m+mf}{1.}\PYG{p}{,} \PYG{l+m+mf}{50.}\PYG{p}{)}
\PYG{g+gp}{\PYGZgt{}\PYGZgt{}\PYGZgt{} }\PYG{n}{y} \PYG{o}{=} \PYG{n}{x}\PYG{o}{*}\PYG{o}{*}\PYG{p}{(}\PYG{o}{\PYGZhy{}}\PYG{n}{a}\PYG{p}{)}\PYG{o}{/}\PYG{n}{sps}\PYG{o}{.}\PYG{n}{zetac}\PYG{p}{(}\PYG{n}{a}\PYG{p}{)}
\PYG{g+gp}{\PYGZgt{}\PYGZgt{}\PYGZgt{} }\PYG{n}{plt}\PYG{o}{.}\PYG{n}{plot}\PYG{p}{(}\PYG{n}{x}\PYG{p}{,} \PYG{n}{y}\PYG{o}{/}\PYG{n+nb}{max}\PYG{p}{(}\PYG{n}{y}\PYG{p}{)}\PYG{p}{,} \PYG{n}{linewidth}\PYG{o}{=}\PYG{l+m+mi}{2}\PYG{p}{,} \PYG{n}{color}\PYG{o}{=}\PYG{l+s}{\PYGZsq{}}\PYG{l+s}{r}\PYG{l+s}{\PYGZsq{}}\PYG{p}{)}
\PYG{g+gp}{\PYGZgt{}\PYGZgt{}\PYGZgt{} }\PYG{n}{plt}\PYG{o}{.}\PYG{n}{show}\PYG{p}{(}\PYG{p}{)}
\end{Verbatim}

\end{fulllineitems}



\subsubsection{\texttt{raf} Module}
\label{lib.graph:module-lib.graph.raf}\label{lib.graph:raf-module}\index{lib.graph.raf (module)}\index{Fcondition() (in module lib.graph.raf)}

\begin{fulllineitems}
\phantomsection\label{lib.graph:lib.graph.raf.Fcondition}\pysiglinewithargsret{\code{lib.graph.raf.}\bfcode{Fcondition}}{\emph{tmpCL}, \emph{tmpRA}, \emph{rcts}, \emph{debug=False}}{}
\end{fulllineitems}

\index{RAcondition() (in module lib.graph.raf)}

\begin{fulllineitems}
\phantomsection\label{lib.graph:lib.graph.raf.RAcondition}\pysiglinewithargsret{\code{lib.graph.raf.}\bfcode{RAcondition}}{\emph{tmpCL}, \emph{rcts}, \emph{cats}, \emph{debug=False}}{}
\end{fulllineitems}

\index{beta() (in module lib.graph.raf)}

\begin{fulllineitems}
\phantomsection\label{lib.graph:lib.graph.raf.beta}\pysiglinewithargsret{\code{lib.graph.raf.}\bfcode{beta}}{\emph{a}, \emph{b}, \emph{size=None}}{}
The Beta distribution over \code{{[}0, 1{]}}.

The Beta distribution is a special case of the Dirichlet distribution,
and is related to the Gamma distribution.  It has the probability
distribution function
\begin{gather}
\begin{split}f(x; a,b) = \frac{1}{B(\alpha, \beta)} x^{\alpha - 1}
(1 - x)^{\beta - 1},\end{split}\notag
\end{gather}
where the normalisation, B, is the beta function,
\begin{gather}
\begin{split}B(\alpha, \beta) = \int_0^1 t^{\alpha - 1}
(1 - t)^{\beta - 1} dt.\end{split}\notag
\end{gather}
It is often seen in Bayesian inference and order statistics.
\begin{description}
\item[{a}] \leavevmode{[}float{]}
Alpha, non-negative.

\item[{b}] \leavevmode{[}float{]}
Beta, non-negative.

\item[{size}] \leavevmode{[}tuple of ints, optional{]}
The number of samples to draw.  The output is packed according to
the size given.

\end{description}
\begin{description}
\item[{out}] \leavevmode{[}ndarray{]}
Array of the given shape, containing values drawn from a
Beta distribution.

\end{description}

\end{fulllineitems}

\index{binomial() (in module lib.graph.raf)}

\begin{fulllineitems}
\phantomsection\label{lib.graph:lib.graph.raf.binomial}\pysiglinewithargsret{\code{lib.graph.raf.}\bfcode{binomial}}{\emph{n}, \emph{p}, \emph{size=None}}{}
Draw samples from a binomial distribution.

Samples are drawn from a Binomial distribution with specified
parameters, n trials and p probability of success where
n an integer \textgreater{}= 0 and p is in the interval {[}0,1{]}. (n may be
input as a float, but it is truncated to an integer in use)
\begin{description}
\item[{n}] \leavevmode{[}float (but truncated to an integer){]}
parameter, \textgreater{}= 0.

\item[{p}] \leavevmode{[}float{]}
parameter, \textgreater{}= 0 and \textless{}=1.

\item[{size}] \leavevmode{[}\{tuple, int\}{]}
Output shape.  If the given shape is, e.g., \code{(m, n, k)}, then
\code{m * n * k} samples are drawn.

\end{description}
\begin{description}
\item[{samples}] \leavevmode{[}\{ndarray, scalar\}{]}
where the values are all integers in  {[}0, n{]}.

\end{description}
\begin{description}
\item[{scipy.stats.distributions.binom}] \leavevmode{[}probability density function,{]}
distribution or cumulative density function, etc.

\end{description}

The probability density for the Binomial distribution is
\begin{gather}
\begin{split}P(N) = \binom{n}{N}p^N(1-p)^{n-N},\end{split}\notag
\end{gather}
where \(n\) is the number of trials, \(p\) is the probability
of success, and \(N\) is the number of successes.

When estimating the standard error of a proportion in a population by
using a random sample, the normal distribution works well unless the
product p*n \textless{}=5, where p = population proportion estimate, and n =
number of samples, in which case the binomial distribution is used
instead. For example, a sample of 15 people shows 4 who are left
handed, and 11 who are right handed. Then p = 4/15 = 27\%. 0.27*15 = 4,
so the binomial distribution should be used in this case.

Draw samples from the distribution:

\begin{Verbatim}[commandchars=\\\{\}]
\PYG{g+gp}{\PYGZgt{}\PYGZgt{}\PYGZgt{} }\PYG{n}{n}\PYG{p}{,} \PYG{n}{p} \PYG{o}{=} \PYG{l+m+mi}{10}\PYG{p}{,} \PYG{o}{.}\PYG{l+m+mi}{5} \PYG{c}{\PYGZsh{} number of trials, probability of each trial}
\PYG{g+gp}{\PYGZgt{}\PYGZgt{}\PYGZgt{} }\PYG{n}{s} \PYG{o}{=} \PYG{n}{np}\PYG{o}{.}\PYG{n}{random}\PYG{o}{.}\PYG{n}{binomial}\PYG{p}{(}\PYG{n}{n}\PYG{p}{,} \PYG{n}{p}\PYG{p}{,} \PYG{l+m+mi}{1000}\PYG{p}{)}
\PYG{g+go}{\PYGZsh{} result of flipping a coin 10 times, tested 1000 times.}
\end{Verbatim}

A real world example. A company drills 9 wild-cat oil exploration
wells, each with an estimated probability of success of 0.1. All nine
wells fail. What is the probability of that happening?

Let's do 20,000 trials of the model, and count the number that
generate zero positive results.

\begin{Verbatim}[commandchars=\\\{\}]
\PYG{g+gp}{\PYGZgt{}\PYGZgt{}\PYGZgt{} }\PYG{n+nb}{sum}\PYG{p}{(}\PYG{n}{np}\PYG{o}{.}\PYG{n}{random}\PYG{o}{.}\PYG{n}{binomial}\PYG{p}{(}\PYG{l+m+mi}{9}\PYG{p}{,}\PYG{l+m+mf}{0.1}\PYG{p}{,}\PYG{l+m+mi}{20000}\PYG{p}{)}\PYG{o}{==}\PYG{l+m+mi}{0}\PYG{p}{)}\PYG{o}{/}\PYG{l+m+mf}{20000.}
\PYG{g+go}{answer = 0.38885, or 38\PYGZpc{}.}
\end{Verbatim}

\end{fulllineitems}

\index{chisquare() (in module lib.graph.raf)}

\begin{fulllineitems}
\phantomsection\label{lib.graph:lib.graph.raf.chisquare}\pysiglinewithargsret{\code{lib.graph.raf.}\bfcode{chisquare}}{\emph{df}, \emph{size=None}}{}
Draw samples from a chi-square distribution.

When \emph{df} independent random variables, each with standard normal
distributions (mean 0, variance 1), are squared and summed, the
resulting distribution is chi-square (see Notes).  This distribution
is often used in hypothesis testing.
\begin{description}
\item[{df}] \leavevmode{[}int{]}
Number of degrees of freedom.

\item[{size}] \leavevmode{[}tuple of ints, int, optional{]}
Size of the returned array.  By default, a scalar is
returned.

\end{description}
\begin{description}
\item[{output}] \leavevmode{[}ndarray{]}
Samples drawn from the distribution, packed in a \emph{size}-shaped
array.

\end{description}
\begin{description}
\item[{ValueError}] \leavevmode
When \emph{df} \textless{}= 0 or when an inappropriate \emph{size} (e.g. \code{size=-1})
is given.

\end{description}

The variable obtained by summing the squares of \emph{df} independent,
standard normally distributed random variables:
\begin{gather}
\begin{split}Q = \sum_{i=0}^{\mathtt{df}} X^2_i\end{split}\notag
\end{gather}
is chi-square distributed, denoted
\begin{gather}
\begin{split}Q \sim \chi^2_k.\end{split}\notag
\end{gather}
The probability density function of the chi-squared distribution is
\begin{gather}
\begin{split}p(x) = \frac{(1/2)^{k/2}}{\Gamma(k/2)}
x^{k/2 - 1} e^{-x/2},\end{split}\notag
\end{gather}
where \(\Gamma\) is the gamma function,
\begin{gather}
\begin{split}\Gamma(x) = \int_0^{-\infty} t^{x - 1} e^{-t} dt.\end{split}\notag
\end{gather}
\href{http://www.itl.nist.gov/div898/handbook/eda/section3/eda3666.htm}{NIST/SEMATECH e-Handbook of Statistical Methods}

\begin{Verbatim}[commandchars=\\\{\}]
\PYG{g+gp}{\PYGZgt{}\PYGZgt{}\PYGZgt{} }\PYG{n}{np}\PYG{o}{.}\PYG{n}{random}\PYG{o}{.}\PYG{n}{chisquare}\PYG{p}{(}\PYG{l+m+mi}{2}\PYG{p}{,}\PYG{l+m+mi}{4}\PYG{p}{)}
\PYG{g+go}{array([ 1.89920014,  9.00867716,  3.13710533,  5.62318272])}
\end{Verbatim}

\end{fulllineitems}

\index{exponential() (in module lib.graph.raf)}

\begin{fulllineitems}
\phantomsection\label{lib.graph:lib.graph.raf.exponential}\pysiglinewithargsret{\code{lib.graph.raf.}\bfcode{exponential}}{\emph{scale=1.0}, \emph{size=None}}{}
Exponential distribution.

Its probability density function is
\begin{gather}
\begin{split}f(x; \frac{1}{\beta}) = \frac{1}{\beta} \exp(-\frac{x}{\beta}),\end{split}\notag
\end{gather}
for \code{x \textgreater{} 0} and 0 elsewhere. \(\beta\) is the scale parameter,
which is the inverse of the rate parameter \(\lambda = 1/\beta\).
The rate parameter is an alternative, widely used parameterization
of the exponential distribution {\color{red}\bfseries{}{[}3{]}\_}.

The exponential distribution is a continuous analogue of the
geometric distribution.  It describes many common situations, such as
the size of raindrops measured over many rainstorms {\color{red}\bfseries{}{[}1{]}\_}, or the time
between page requests to Wikipedia {\color{red}\bfseries{}{[}2{]}\_}.
\begin{description}
\item[{scale}] \leavevmode{[}float{]}
The scale parameter, \(\beta = 1/\lambda\).

\item[{size}] \leavevmode{[}tuple of ints{]}
Number of samples to draw.  The output is shaped
according to \emph{size}.

\end{description}

\end{fulllineitems}

\index{f() (in module lib.graph.raf)}

\begin{fulllineitems}
\phantomsection\label{lib.graph:lib.graph.raf.f}\pysiglinewithargsret{\code{lib.graph.raf.}\bfcode{f}}{\emph{dfnum}, \emph{dfden}, \emph{size=None}}{}
Draw samples from a F distribution.

Samples are drawn from an F distribution with specified parameters,
\emph{dfnum} (degrees of freedom in numerator) and \emph{dfden} (degrees of freedom
in denominator), where both parameters should be greater than zero.

The random variate of the F distribution (also known as the
Fisher distribution) is a continuous probability distribution
that arises in ANOVA tests, and is the ratio of two chi-square
variates.
\begin{description}
\item[{dfnum}] \leavevmode{[}float{]}
Degrees of freedom in numerator. Should be greater than zero.

\item[{dfden}] \leavevmode{[}float{]}
Degrees of freedom in denominator. Should be greater than zero.

\item[{size}] \leavevmode{[}\{tuple, int\}, optional{]}
Output shape.  If the given shape is, e.g., \code{(m, n, k)},
then \code{m * n * k} samples are drawn. By default only one sample
is returned.

\end{description}
\begin{description}
\item[{samples}] \leavevmode{[}\{ndarray, scalar\}{]}
Samples from the Fisher distribution.

\end{description}
\begin{description}
\item[{scipy.stats.distributions.f}] \leavevmode{[}probability density function,{]}
distribution or cumulative density function, etc.

\end{description}

The F statistic is used to compare in-group variances to between-group
variances. Calculating the distribution depends on the sampling, and
so it is a function of the respective degrees of freedom in the
problem.  The variable \emph{dfnum} is the number of samples minus one, the
between-groups degrees of freedom, while \emph{dfden} is the within-groups
degrees of freedom, the sum of the number of samples in each group
minus the number of groups.

An example from Glantz{[}1{]}, pp 47-40.
Two groups, children of diabetics (25 people) and children from people
without diabetes (25 controls). Fasting blood glucose was measured,
case group had a mean value of 86.1, controls had a mean value of
82.2. Standard deviations were 2.09 and 2.49 respectively. Are these
data consistent with the null hypothesis that the parents diabetic
status does not affect their children's blood glucose levels?
Calculating the F statistic from the data gives a value of 36.01.

Draw samples from the distribution:

\begin{Verbatim}[commandchars=\\\{\}]
\PYG{g+gp}{\PYGZgt{}\PYGZgt{}\PYGZgt{} }\PYG{n}{dfnum} \PYG{o}{=} \PYG{l+m+mf}{1.} \PYG{c}{\PYGZsh{} between group degrees of freedom}
\PYG{g+gp}{\PYGZgt{}\PYGZgt{}\PYGZgt{} }\PYG{n}{dfden} \PYG{o}{=} \PYG{l+m+mf}{48.} \PYG{c}{\PYGZsh{} within groups degrees of freedom}
\PYG{g+gp}{\PYGZgt{}\PYGZgt{}\PYGZgt{} }\PYG{n}{s} \PYG{o}{=} \PYG{n}{np}\PYG{o}{.}\PYG{n}{random}\PYG{o}{.}\PYG{n}{f}\PYG{p}{(}\PYG{n}{dfnum}\PYG{p}{,} \PYG{n}{dfden}\PYG{p}{,} \PYG{l+m+mi}{1000}\PYG{p}{)}
\end{Verbatim}

The lower bound for the top 1\% of the samples is :

\begin{Verbatim}[commandchars=\\\{\}]
\PYG{g+gp}{\PYGZgt{}\PYGZgt{}\PYGZgt{} }\PYG{n}{sort}\PYG{p}{(}\PYG{n}{s}\PYG{p}{)}\PYG{p}{[}\PYG{o}{\PYGZhy{}}\PYG{l+m+mi}{10}\PYG{p}{]}
\PYG{g+go}{7.61988120985}
\end{Verbatim}

So there is about a 1\% chance that the F statistic will exceed 7.62,
the measured value is 36, so the null hypothesis is rejected at the 1\%
level.

\end{fulllineitems}

\index{findCatforRAF() (in module lib.graph.raf)}

\begin{fulllineitems}
\phantomsection\label{lib.graph:lib.graph.raf.findCatforRAF}\pysiglinewithargsret{\code{lib.graph.raf.}\bfcode{findCatforRAF}}{\emph{tmpCat}, \emph{tmpRAF}, \emph{tmpClosure}, \emph{debug=False}}{}
\end{fulllineitems}

\index{findRAFrcts() (in module lib.graph.raf)}

\begin{fulllineitems}
\phantomsection\label{lib.graph:lib.graph.raf.findRAFrcts}\pysiglinewithargsret{\code{lib.graph.raf.}\bfcode{findRAFrcts}}{\emph{RAF}, \emph{rcts}, \emph{actrcts}}{}
\end{fulllineitems}

\index{gamma() (in module lib.graph.raf)}

\begin{fulllineitems}
\phantomsection\label{lib.graph:lib.graph.raf.gamma}\pysiglinewithargsret{\code{lib.graph.raf.}\bfcode{gamma}}{\emph{shape}, \emph{scale=1.0}, \emph{size=None}}{}
Draw samples from a Gamma distribution.

Samples are drawn from a Gamma distribution with specified parameters,
\emph{shape} (sometimes designated ``k'') and \emph{scale} (sometimes designated
``theta''), where both parameters are \textgreater{} 0.
\begin{description}
\item[{shape}] \leavevmode{[}scalar \textgreater{} 0{]}
The shape of the gamma distribution.

\item[{scale}] \leavevmode{[}scalar \textgreater{} 0, optional{]}
The scale of the gamma distribution.  Default is equal to 1.

\item[{size}] \leavevmode{[}shape\_tuple, optional{]}
Output shape.  If the given shape is, e.g., \code{(m, n, k)}, then
\code{m * n * k} samples are drawn.

\end{description}
\begin{description}
\item[{out}] \leavevmode{[}ndarray, float{]}
Returns one sample unless \emph{size} parameter is specified.

\end{description}
\begin{description}
\item[{scipy.stats.distributions.gamma}] \leavevmode{[}probability density function,{]}
distribution or cumulative density function, etc.

\end{description}

The probability density for the Gamma distribution is
\begin{gather}
\begin{split}p(x) = x^{k-1}\frac{e^{-x/\theta}}{\theta^k\Gamma(k)},\end{split}\notag
\end{gather}
where \(k\) is the shape and \(\theta\) the scale,
and \(\Gamma\) is the Gamma function.

The Gamma distribution is often used to model the times to failure of
electronic components, and arises naturally in processes for which the
waiting times between Poisson distributed events are relevant.

Draw samples from the distribution:

\begin{Verbatim}[commandchars=\\\{\}]
\PYG{g+gp}{\PYGZgt{}\PYGZgt{}\PYGZgt{} }\PYG{n}{shape}\PYG{p}{,} \PYG{n}{scale} \PYG{o}{=} \PYG{l+m+mf}{2.}\PYG{p}{,} \PYG{l+m+mf}{2.} \PYG{c}{\PYGZsh{} mean and dispersion}
\PYG{g+gp}{\PYGZgt{}\PYGZgt{}\PYGZgt{} }\PYG{n}{s} \PYG{o}{=} \PYG{n}{np}\PYG{o}{.}\PYG{n}{random}\PYG{o}{.}\PYG{n}{gamma}\PYG{p}{(}\PYG{n}{shape}\PYG{p}{,} \PYG{n}{scale}\PYG{p}{,} \PYG{l+m+mi}{1000}\PYG{p}{)}
\end{Verbatim}

Display the histogram of the samples, along with
the probability density function:

\begin{Verbatim}[commandchars=\\\{\}]
\PYG{g+gp}{\PYGZgt{}\PYGZgt{}\PYGZgt{} }\PYG{k+kn}{import} \PYG{n+nn}{matplotlib.pyplot} \PYG{k+kn}{as} \PYG{n+nn}{plt}
\PYG{g+gp}{\PYGZgt{}\PYGZgt{}\PYGZgt{} }\PYG{k+kn}{import} \PYG{n+nn}{scipy.special} \PYG{k+kn}{as} \PYG{n+nn}{sps}
\PYG{g+gp}{\PYGZgt{}\PYGZgt{}\PYGZgt{} }\PYG{n}{count}\PYG{p}{,} \PYG{n}{bins}\PYG{p}{,} \PYG{n}{ignored} \PYG{o}{=} \PYG{n}{plt}\PYG{o}{.}\PYG{n}{hist}\PYG{p}{(}\PYG{n}{s}\PYG{p}{,} \PYG{l+m+mi}{50}\PYG{p}{,} \PYG{n}{normed}\PYG{o}{=}\PYG{n+nb+bp}{True}\PYG{p}{)}
\PYG{g+gp}{\PYGZgt{}\PYGZgt{}\PYGZgt{} }\PYG{n}{y} \PYG{o}{=} \PYG{n}{bins}\PYG{o}{*}\PYG{o}{*}\PYG{p}{(}\PYG{n}{shape}\PYG{o}{\PYGZhy{}}\PYG{l+m+mi}{1}\PYG{p}{)}\PYG{o}{*}\PYG{p}{(}\PYG{n}{np}\PYG{o}{.}\PYG{n}{exp}\PYG{p}{(}\PYG{o}{\PYGZhy{}}\PYG{n}{bins}\PYG{o}{/}\PYG{n}{scale}\PYG{p}{)} \PYG{o}{/}
\PYG{g+gp}{... }                     \PYG{p}{(}\PYG{n}{sps}\PYG{o}{.}\PYG{n}{gamma}\PYG{p}{(}\PYG{n}{shape}\PYG{p}{)}\PYG{o}{*}\PYG{n}{scale}\PYG{o}{*}\PYG{o}{*}\PYG{n}{shape}\PYG{p}{)}\PYG{p}{)}
\PYG{g+gp}{\PYGZgt{}\PYGZgt{}\PYGZgt{} }\PYG{n}{plt}\PYG{o}{.}\PYG{n}{plot}\PYG{p}{(}\PYG{n}{bins}\PYG{p}{,} \PYG{n}{y}\PYG{p}{,} \PYG{n}{linewidth}\PYG{o}{=}\PYG{l+m+mi}{2}\PYG{p}{,} \PYG{n}{color}\PYG{o}{=}\PYG{l+s}{\PYGZsq{}}\PYG{l+s}{r}\PYG{l+s}{\PYGZsq{}}\PYG{p}{)}
\PYG{g+gp}{\PYGZgt{}\PYGZgt{}\PYGZgt{} }\PYG{n}{plt}\PYG{o}{.}\PYG{n}{show}\PYG{p}{(}\PYG{p}{)}
\end{Verbatim}

\end{fulllineitems}

\index{generateClosure() (in module lib.graph.raf)}

\begin{fulllineitems}
\phantomsection\label{lib.graph:lib.graph.raf.generateClosure}\pysiglinewithargsret{\code{lib.graph.raf.}\bfcode{generateClosure}}{\emph{tmpF}, \emph{rcts}, \emph{debug=False}}{}
\end{fulllineitems}

\index{geometric() (in module lib.graph.raf)}

\begin{fulllineitems}
\phantomsection\label{lib.graph:lib.graph.raf.geometric}\pysiglinewithargsret{\code{lib.graph.raf.}\bfcode{geometric}}{\emph{p}, \emph{size=None}}{}
Draw samples from the geometric distribution.

Bernoulli trials are experiments with one of two outcomes:
success or failure (an example of such an experiment is flipping
a coin).  The geometric distribution models the number of trials
that must be run in order to achieve success.  It is therefore
supported on the positive integers, \code{k = 1, 2, ...}.

The probability mass function of the geometric distribution is
\begin{gather}
\begin{split}f(k) = (1 - p)^{k - 1} p\end{split}\notag
\end{gather}
where \emph{p} is the probability of success of an individual trial.
\begin{description}
\item[{p}] \leavevmode{[}float{]}
The probability of success of an individual trial.

\item[{size}] \leavevmode{[}tuple of ints{]}
Number of values to draw from the distribution.  The output
is shaped according to \emph{size}.

\end{description}
\begin{description}
\item[{out}] \leavevmode{[}ndarray{]}
Samples from the geometric distribution, shaped according to
\emph{size}.

\end{description}

Draw ten thousand values from the geometric distribution,
with the probability of an individual success equal to 0.35:

\begin{Verbatim}[commandchars=\\\{\}]
\PYG{g+gp}{\PYGZgt{}\PYGZgt{}\PYGZgt{} }\PYG{n}{z} \PYG{o}{=} \PYG{n}{np}\PYG{o}{.}\PYG{n}{random}\PYG{o}{.}\PYG{n}{geometric}\PYG{p}{(}\PYG{n}{p}\PYG{o}{=}\PYG{l+m+mf}{0.35}\PYG{p}{,} \PYG{n}{size}\PYG{o}{=}\PYG{l+m+mi}{10000}\PYG{p}{)}
\end{Verbatim}

How many trials succeeded after a single run?

\begin{Verbatim}[commandchars=\\\{\}]
\PYG{g+gp}{\PYGZgt{}\PYGZgt{}\PYGZgt{} }\PYG{p}{(}\PYG{n}{z} \PYG{o}{==} \PYG{l+m+mi}{1}\PYG{p}{)}\PYG{o}{.}\PYG{n}{sum}\PYG{p}{(}\PYG{p}{)} \PYG{o}{/} \PYG{l+m+mf}{10000.}
\PYG{g+go}{0.34889999999999999 \PYGZsh{}random}
\end{Verbatim}

\end{fulllineitems}

\index{get\_state() (in module lib.graph.raf)}

\begin{fulllineitems}
\phantomsection\label{lib.graph:lib.graph.raf.get_state}\pysiglinewithargsret{\code{lib.graph.raf.}\bfcode{get\_state}}{}{}
Return a tuple representing the internal state of the generator.

For more details, see \emph{set\_state}.
\begin{description}
\item[{out}] \leavevmode{[}tuple(str, ndarray of 624 uints, int, int, float){]}
The returned tuple has the following items:
\begin{enumerate}
\item {} 
the string `MT19937'.

\item {} 
a 1-D array of 624 unsigned integer keys.

\item {} 
an integer \code{pos}.

\item {} 
an integer \code{has\_gauss}.

\item {} 
a float \code{cached\_gaussian}.

\end{enumerate}

\end{description}

set\_state

\emph{set\_state} and \emph{get\_state} are not needed to work with any of the
random distributions in NumPy. If the internal state is manually altered,
the user should know exactly what he/she is doing.

\end{fulllineitems}

\index{gumbel() (in module lib.graph.raf)}

\begin{fulllineitems}
\phantomsection\label{lib.graph:lib.graph.raf.gumbel}\pysiglinewithargsret{\code{lib.graph.raf.}\bfcode{gumbel}}{\emph{loc=0.0}, \emph{scale=1.0}, \emph{size=None}}{}
Gumbel distribution.

Draw samples from a Gumbel distribution with specified location and scale.
For more information on the Gumbel distribution, see Notes and References
below.
\begin{description}
\item[{loc}] \leavevmode{[}float{]}
The location of the mode of the distribution.

\item[{scale}] \leavevmode{[}float{]}
The scale parameter of the distribution.

\item[{size}] \leavevmode{[}tuple of ints{]}
Output shape.  If the given shape is, e.g., \code{(m, n, k)}, then
\code{m * n * k} samples are drawn.

\end{description}
\begin{description}
\item[{out}] \leavevmode{[}ndarray{]}
The samples

\end{description}

scipy.stats.gumbel\_l
scipy.stats.gumbel\_r
scipy.stats.genextreme
\begin{quote}

probability density function, distribution, or cumulative density
function, etc. for each of the above
\end{quote}

weibull

The Gumbel (or Smallest Extreme Value (SEV) or the Smallest Extreme Value
Type I) distribution is one of a class of Generalized Extreme Value (GEV)
distributions used in modeling extreme value problems.  The Gumbel is a
special case of the Extreme Value Type I distribution for maximums from
distributions with ``exponential-like'' tails.

The probability density for the Gumbel distribution is
\begin{gather}
\begin{split}p(x) = \frac{e^{-(x - \mu)/ \beta}}{\beta} e^{ -e^{-(x - \mu)/
\beta}},\end{split}\notag
\end{gather}
where \(\mu\) is the mode, a location parameter, and \(\beta\) is
the scale parameter.

The Gumbel (named for German mathematician Emil Julius Gumbel) was used
very early in the hydrology literature, for modeling the occurrence of
flood events. It is also used for modeling maximum wind speed and rainfall
rates.  It is a ``fat-tailed'' distribution - the probability of an event in
the tail of the distribution is larger than if one used a Gaussian, hence
the surprisingly frequent occurrence of 100-year floods. Floods were
initially modeled as a Gaussian process, which underestimated the frequency
of extreme events.

It is one of a class of extreme value distributions, the Generalized
Extreme Value (GEV) distributions, which also includes the Weibull and
Frechet.

The function has a mean of \(\mu + 0.57721\beta\) and a variance of
\(\frac{\pi^2}{6}\beta^2\).

Gumbel, E. J., \emph{Statistics of Extremes}, New York: Columbia University
Press, 1958.

Reiss, R.-D. and Thomas, M., \emph{Statistical Analysis of Extreme Values from
Insurance, Finance, Hydrology and Other Fields}, Basel: Birkhauser Verlag,
2001.

Draw samples from the distribution:

\begin{Verbatim}[commandchars=\\\{\}]
\PYG{g+gp}{\PYGZgt{}\PYGZgt{}\PYGZgt{} }\PYG{n}{mu}\PYG{p}{,} \PYG{n}{beta} \PYG{o}{=} \PYG{l+m+mi}{0}\PYG{p}{,} \PYG{l+m+mf}{0.1} \PYG{c}{\PYGZsh{} location and scale}
\PYG{g+gp}{\PYGZgt{}\PYGZgt{}\PYGZgt{} }\PYG{n}{s} \PYG{o}{=} \PYG{n}{np}\PYG{o}{.}\PYG{n}{random}\PYG{o}{.}\PYG{n}{gumbel}\PYG{p}{(}\PYG{n}{mu}\PYG{p}{,} \PYG{n}{beta}\PYG{p}{,} \PYG{l+m+mi}{1000}\PYG{p}{)}
\end{Verbatim}

Display the histogram of the samples, along with
the probability density function:

\begin{Verbatim}[commandchars=\\\{\}]
\PYG{g+gp}{\PYGZgt{}\PYGZgt{}\PYGZgt{} }\PYG{k+kn}{import} \PYG{n+nn}{matplotlib.pyplot} \PYG{k+kn}{as} \PYG{n+nn}{plt}
\PYG{g+gp}{\PYGZgt{}\PYGZgt{}\PYGZgt{} }\PYG{n}{count}\PYG{p}{,} \PYG{n}{bins}\PYG{p}{,} \PYG{n}{ignored} \PYG{o}{=} \PYG{n}{plt}\PYG{o}{.}\PYG{n}{hist}\PYG{p}{(}\PYG{n}{s}\PYG{p}{,} \PYG{l+m+mi}{30}\PYG{p}{,} \PYG{n}{normed}\PYG{o}{=}\PYG{n+nb+bp}{True}\PYG{p}{)}
\PYG{g+gp}{\PYGZgt{}\PYGZgt{}\PYGZgt{} }\PYG{n}{plt}\PYG{o}{.}\PYG{n}{plot}\PYG{p}{(}\PYG{n}{bins}\PYG{p}{,} \PYG{p}{(}\PYG{l+m+mi}{1}\PYG{o}{/}\PYG{n}{beta}\PYG{p}{)}\PYG{o}{*}\PYG{n}{np}\PYG{o}{.}\PYG{n}{exp}\PYG{p}{(}\PYG{o}{\PYGZhy{}}\PYG{p}{(}\PYG{n}{bins} \PYG{o}{\PYGZhy{}} \PYG{n}{mu}\PYG{p}{)}\PYG{o}{/}\PYG{n}{beta}\PYG{p}{)}
\PYG{g+gp}{... }         \PYG{o}{*} \PYG{n}{np}\PYG{o}{.}\PYG{n}{exp}\PYG{p}{(} \PYG{o}{\PYGZhy{}}\PYG{n}{np}\PYG{o}{.}\PYG{n}{exp}\PYG{p}{(} \PYG{o}{\PYGZhy{}}\PYG{p}{(}\PYG{n}{bins} \PYG{o}{\PYGZhy{}} \PYG{n}{mu}\PYG{p}{)} \PYG{o}{/}\PYG{n}{beta}\PYG{p}{)} \PYG{p}{)}\PYG{p}{,}
\PYG{g+gp}{... }         \PYG{n}{linewidth}\PYG{o}{=}\PYG{l+m+mi}{2}\PYG{p}{,} \PYG{n}{color}\PYG{o}{=}\PYG{l+s}{\PYGZsq{}}\PYG{l+s}{r}\PYG{l+s}{\PYGZsq{}}\PYG{p}{)}
\PYG{g+gp}{\PYGZgt{}\PYGZgt{}\PYGZgt{} }\PYG{n}{plt}\PYG{o}{.}\PYG{n}{show}\PYG{p}{(}\PYG{p}{)}
\end{Verbatim}

Show how an extreme value distribution can arise from a Gaussian process
and compare to a Gaussian:

\begin{Verbatim}[commandchars=\\\{\}]
\PYG{g+gp}{\PYGZgt{}\PYGZgt{}\PYGZgt{} }\PYG{n}{means} \PYG{o}{=} \PYG{p}{[}\PYG{p}{]}
\PYG{g+gp}{\PYGZgt{}\PYGZgt{}\PYGZgt{} }\PYG{n}{maxima} \PYG{o}{=} \PYG{p}{[}\PYG{p}{]}
\PYG{g+gp}{\PYGZgt{}\PYGZgt{}\PYGZgt{} }\PYG{k}{for} \PYG{n}{i} \PYG{o+ow}{in} \PYG{n+nb}{range}\PYG{p}{(}\PYG{l+m+mi}{0}\PYG{p}{,}\PYG{l+m+mi}{1000}\PYG{p}{)} \PYG{p}{:}
\PYG{g+gp}{... }   \PYG{n}{a} \PYG{o}{=} \PYG{n}{np}\PYG{o}{.}\PYG{n}{random}\PYG{o}{.}\PYG{n}{normal}\PYG{p}{(}\PYG{n}{mu}\PYG{p}{,} \PYG{n}{beta}\PYG{p}{,} \PYG{l+m+mi}{1000}\PYG{p}{)}
\PYG{g+gp}{... }   \PYG{n}{means}\PYG{o}{.}\PYG{n}{append}\PYG{p}{(}\PYG{n}{a}\PYG{o}{.}\PYG{n}{mean}\PYG{p}{(}\PYG{p}{)}\PYG{p}{)}
\PYG{g+gp}{... }   \PYG{n}{maxima}\PYG{o}{.}\PYG{n}{append}\PYG{p}{(}\PYG{n}{a}\PYG{o}{.}\PYG{n}{max}\PYG{p}{(}\PYG{p}{)}\PYG{p}{)}
\PYG{g+gp}{\PYGZgt{}\PYGZgt{}\PYGZgt{} }\PYG{n}{count}\PYG{p}{,} \PYG{n}{bins}\PYG{p}{,} \PYG{n}{ignored} \PYG{o}{=} \PYG{n}{plt}\PYG{o}{.}\PYG{n}{hist}\PYG{p}{(}\PYG{n}{maxima}\PYG{p}{,} \PYG{l+m+mi}{30}\PYG{p}{,} \PYG{n}{normed}\PYG{o}{=}\PYG{n+nb+bp}{True}\PYG{p}{)}
\PYG{g+gp}{\PYGZgt{}\PYGZgt{}\PYGZgt{} }\PYG{n}{beta} \PYG{o}{=} \PYG{n}{np}\PYG{o}{.}\PYG{n}{std}\PYG{p}{(}\PYG{n}{maxima}\PYG{p}{)}\PYG{o}{*}\PYG{n}{np}\PYG{o}{.}\PYG{n}{pi}\PYG{o}{/}\PYG{n}{np}\PYG{o}{.}\PYG{n}{sqrt}\PYG{p}{(}\PYG{l+m+mi}{6}\PYG{p}{)}
\PYG{g+gp}{\PYGZgt{}\PYGZgt{}\PYGZgt{} }\PYG{n}{mu} \PYG{o}{=} \PYG{n}{np}\PYG{o}{.}\PYG{n}{mean}\PYG{p}{(}\PYG{n}{maxima}\PYG{p}{)} \PYG{o}{\PYGZhy{}} \PYG{l+m+mf}{0.57721}\PYG{o}{*}\PYG{n}{beta}
\PYG{g+gp}{\PYGZgt{}\PYGZgt{}\PYGZgt{} }\PYG{n}{plt}\PYG{o}{.}\PYG{n}{plot}\PYG{p}{(}\PYG{n}{bins}\PYG{p}{,} \PYG{p}{(}\PYG{l+m+mi}{1}\PYG{o}{/}\PYG{n}{beta}\PYG{p}{)}\PYG{o}{*}\PYG{n}{np}\PYG{o}{.}\PYG{n}{exp}\PYG{p}{(}\PYG{o}{\PYGZhy{}}\PYG{p}{(}\PYG{n}{bins} \PYG{o}{\PYGZhy{}} \PYG{n}{mu}\PYG{p}{)}\PYG{o}{/}\PYG{n}{beta}\PYG{p}{)}
\PYG{g+gp}{... }         \PYG{o}{*} \PYG{n}{np}\PYG{o}{.}\PYG{n}{exp}\PYG{p}{(}\PYG{o}{\PYGZhy{}}\PYG{n}{np}\PYG{o}{.}\PYG{n}{exp}\PYG{p}{(}\PYG{o}{\PYGZhy{}}\PYG{p}{(}\PYG{n}{bins} \PYG{o}{\PYGZhy{}} \PYG{n}{mu}\PYG{p}{)}\PYG{o}{/}\PYG{n}{beta}\PYG{p}{)}\PYG{p}{)}\PYG{p}{,}
\PYG{g+gp}{... }         \PYG{n}{linewidth}\PYG{o}{=}\PYG{l+m+mi}{2}\PYG{p}{,} \PYG{n}{color}\PYG{o}{=}\PYG{l+s}{\PYGZsq{}}\PYG{l+s}{r}\PYG{l+s}{\PYGZsq{}}\PYG{p}{)}
\PYG{g+gp}{\PYGZgt{}\PYGZgt{}\PYGZgt{} }\PYG{n}{plt}\PYG{o}{.}\PYG{n}{plot}\PYG{p}{(}\PYG{n}{bins}\PYG{p}{,} \PYG{l+m+mi}{1}\PYG{o}{/}\PYG{p}{(}\PYG{n}{beta} \PYG{o}{*} \PYG{n}{np}\PYG{o}{.}\PYG{n}{sqrt}\PYG{p}{(}\PYG{l+m+mi}{2} \PYG{o}{*} \PYG{n}{np}\PYG{o}{.}\PYG{n}{pi}\PYG{p}{)}\PYG{p}{)}
\PYG{g+gp}{... }         \PYG{o}{*} \PYG{n}{np}\PYG{o}{.}\PYG{n}{exp}\PYG{p}{(}\PYG{o}{\PYGZhy{}}\PYG{p}{(}\PYG{n}{bins} \PYG{o}{\PYGZhy{}} \PYG{n}{mu}\PYG{p}{)}\PYG{o}{*}\PYG{o}{*}\PYG{l+m+mi}{2} \PYG{o}{/} \PYG{p}{(}\PYG{l+m+mi}{2} \PYG{o}{*} \PYG{n}{beta}\PYG{o}{*}\PYG{o}{*}\PYG{l+m+mi}{2}\PYG{p}{)}\PYG{p}{)}\PYG{p}{,}
\PYG{g+gp}{... }         \PYG{n}{linewidth}\PYG{o}{=}\PYG{l+m+mi}{2}\PYG{p}{,} \PYG{n}{color}\PYG{o}{=}\PYG{l+s}{\PYGZsq{}}\PYG{l+s}{g}\PYG{l+s}{\PYGZsq{}}\PYG{p}{)}
\PYG{g+gp}{\PYGZgt{}\PYGZgt{}\PYGZgt{} }\PYG{n}{plt}\PYG{o}{.}\PYG{n}{show}\PYG{p}{(}\PYG{p}{)}
\end{Verbatim}

\end{fulllineitems}

\index{hypergeometric() (in module lib.graph.raf)}

\begin{fulllineitems}
\phantomsection\label{lib.graph:lib.graph.raf.hypergeometric}\pysiglinewithargsret{\code{lib.graph.raf.}\bfcode{hypergeometric}}{\emph{ngood}, \emph{nbad}, \emph{nsample}, \emph{size=None}}{}
Draw samples from a Hypergeometric distribution.

Samples are drawn from a Hypergeometric distribution with specified
parameters, ngood (ways to make a good selection), nbad (ways to make
a bad selection), and nsample = number of items sampled, which is less
than or equal to the sum ngood + nbad.
\begin{description}
\item[{ngood}] \leavevmode{[}int or array\_like{]}
Number of ways to make a good selection.  Must be nonnegative.

\item[{nbad}] \leavevmode{[}int or array\_like{]}
Number of ways to make a bad selection.  Must be nonnegative.

\item[{nsample}] \leavevmode{[}int or array\_like{]}
Number of items sampled.  Must be at least 1 and at most
\code{ngood + nbad}.

\item[{size}] \leavevmode{[}int or tuple of int{]}
Output shape.  If the given shape is, e.g., \code{(m, n, k)}, then
\code{m * n * k} samples are drawn.

\end{description}
\begin{description}
\item[{samples}] \leavevmode{[}ndarray or scalar{]}
The values are all integers in  {[}0, n{]}.

\end{description}
\begin{description}
\item[{scipy.stats.distributions.hypergeom}] \leavevmode{[}probability density function,{]}
distribution or cumulative density function, etc.

\end{description}

The probability density for the Hypergeometric distribution is
\begin{gather}
\begin{split}P(x) = \frac{\binom{m}{n}\binom{N-m}{n-x}}{\binom{N}{n}},\end{split}\notag
\end{gather}
where \(0 \le x \le m\) and \(n+m-N \le x \le n\)

for P(x) the probability of x successes, n = ngood, m = nbad, and
N = number of samples.

Consider an urn with black and white marbles in it, ngood of them
black and nbad are white. If you draw nsample balls without
replacement, then the Hypergeometric distribution describes the
distribution of black balls in the drawn sample.

Note that this distribution is very similar to the Binomial
distribution, except that in this case, samples are drawn without
replacement, whereas in the Binomial case samples are drawn with
replacement (or the sample space is infinite). As the sample space
becomes large, this distribution approaches the Binomial.

Draw samples from the distribution:

\begin{Verbatim}[commandchars=\\\{\}]
\PYG{g+gp}{\PYGZgt{}\PYGZgt{}\PYGZgt{} }\PYG{n}{ngood}\PYG{p}{,} \PYG{n}{nbad}\PYG{p}{,} \PYG{n}{nsamp} \PYG{o}{=} \PYG{l+m+mi}{100}\PYG{p}{,} \PYG{l+m+mi}{2}\PYG{p}{,} \PYG{l+m+mi}{10}
\PYG{g+go}{\PYGZsh{} number of good, number of bad, and number of samples}
\PYG{g+gp}{\PYGZgt{}\PYGZgt{}\PYGZgt{} }\PYG{n}{s} \PYG{o}{=} \PYG{n}{np}\PYG{o}{.}\PYG{n}{random}\PYG{o}{.}\PYG{n}{hypergeometric}\PYG{p}{(}\PYG{n}{ngood}\PYG{p}{,} \PYG{n}{nbad}\PYG{p}{,} \PYG{n}{nsamp}\PYG{p}{,} \PYG{l+m+mi}{1000}\PYG{p}{)}
\PYG{g+gp}{\PYGZgt{}\PYGZgt{}\PYGZgt{} }\PYG{n}{hist}\PYG{p}{(}\PYG{n}{s}\PYG{p}{)}
\PYG{g+go}{\PYGZsh{}   note that it is very unlikely to grab both bad items}
\end{Verbatim}

Suppose you have an urn with 15 white and 15 black marbles.
If you pull 15 marbles at random, how likely is it that
12 or more of them are one color?

\begin{Verbatim}[commandchars=\\\{\}]
\PYG{g+gp}{\PYGZgt{}\PYGZgt{}\PYGZgt{} }\PYG{n}{s} \PYG{o}{=} \PYG{n}{np}\PYG{o}{.}\PYG{n}{random}\PYG{o}{.}\PYG{n}{hypergeometric}\PYG{p}{(}\PYG{l+m+mi}{15}\PYG{p}{,} \PYG{l+m+mi}{15}\PYG{p}{,} \PYG{l+m+mi}{15}\PYG{p}{,} \PYG{l+m+mi}{100000}\PYG{p}{)}
\PYG{g+gp}{\PYGZgt{}\PYGZgt{}\PYGZgt{} }\PYG{n+nb}{sum}\PYG{p}{(}\PYG{n}{s}\PYG{o}{\PYGZgt{}}\PYG{o}{=}\PYG{l+m+mi}{12}\PYG{p}{)}\PYG{o}{/}\PYG{l+m+mf}{100000.} \PYG{o}{+} \PYG{n+nb}{sum}\PYG{p}{(}\PYG{n}{s}\PYG{o}{\PYGZlt{}}\PYG{o}{=}\PYG{l+m+mi}{3}\PYG{p}{)}\PYG{o}{/}\PYG{l+m+mf}{100000.}
\PYG{g+go}{\PYGZsh{}   answer = 0.003 ... pretty unlikely!}
\end{Verbatim}

\end{fulllineitems}

\index{laplace() (in module lib.graph.raf)}

\begin{fulllineitems}
\phantomsection\label{lib.graph:lib.graph.raf.laplace}\pysiglinewithargsret{\code{lib.graph.raf.}\bfcode{laplace}}{\emph{loc=0.0}, \emph{scale=1.0}, \emph{size=None}}{}
Draw samples from the Laplace or double exponential distribution with
specified location (or mean) and scale (decay).

The Laplace distribution is similar to the Gaussian/normal distribution,
but is sharper at the peak and has fatter tails. It represents the
difference between two independent, identically distributed exponential
random variables.
\begin{description}
\item[{loc}] \leavevmode{[}float{]}
The position, \(\mu\), of the distribution peak.

\item[{scale}] \leavevmode{[}float{]}
\(\lambda\), the exponential decay.

\end{description}

It has the probability density function
\begin{gather}
\begin{split}f(x; \mu, \lambda) = \frac{1}{2\lambda}
\exp\left(-\frac{|x - \mu|}{\lambda}\right).\end{split}\notag
\end{gather}
The first law of Laplace, from 1774, states that the frequency of an error
can be expressed as an exponential function of the absolute magnitude of
the error, which leads to the Laplace distribution. For many problems in
Economics and Health sciences, this distribution seems to model the data
better than the standard Gaussian distribution

Draw samples from the distribution

\begin{Verbatim}[commandchars=\\\{\}]
\PYG{g+gp}{\PYGZgt{}\PYGZgt{}\PYGZgt{} }\PYG{n}{loc}\PYG{p}{,} \PYG{n}{scale} \PYG{o}{=} \PYG{l+m+mf}{0.}\PYG{p}{,} \PYG{l+m+mf}{1.}
\PYG{g+gp}{\PYGZgt{}\PYGZgt{}\PYGZgt{} }\PYG{n}{s} \PYG{o}{=} \PYG{n}{np}\PYG{o}{.}\PYG{n}{random}\PYG{o}{.}\PYG{n}{laplace}\PYG{p}{(}\PYG{n}{loc}\PYG{p}{,} \PYG{n}{scale}\PYG{p}{,} \PYG{l+m+mi}{1000}\PYG{p}{)}
\end{Verbatim}

Display the histogram of the samples, along with
the probability density function:

\begin{Verbatim}[commandchars=\\\{\}]
\PYG{g+gp}{\PYGZgt{}\PYGZgt{}\PYGZgt{} }\PYG{k+kn}{import} \PYG{n+nn}{matplotlib.pyplot} \PYG{k+kn}{as} \PYG{n+nn}{plt}
\PYG{g+gp}{\PYGZgt{}\PYGZgt{}\PYGZgt{} }\PYG{n}{count}\PYG{p}{,} \PYG{n}{bins}\PYG{p}{,} \PYG{n}{ignored} \PYG{o}{=} \PYG{n}{plt}\PYG{o}{.}\PYG{n}{hist}\PYG{p}{(}\PYG{n}{s}\PYG{p}{,} \PYG{l+m+mi}{30}\PYG{p}{,} \PYG{n}{normed}\PYG{o}{=}\PYG{n+nb+bp}{True}\PYG{p}{)}
\PYG{g+gp}{\PYGZgt{}\PYGZgt{}\PYGZgt{} }\PYG{n}{x} \PYG{o}{=} \PYG{n}{np}\PYG{o}{.}\PYG{n}{arange}\PYG{p}{(}\PYG{o}{\PYGZhy{}}\PYG{l+m+mf}{8.}\PYG{p}{,} \PYG{l+m+mf}{8.}\PYG{p}{,} \PYG{o}{.}\PYG{l+m+mo}{01}\PYG{p}{)}
\PYG{g+gp}{\PYGZgt{}\PYGZgt{}\PYGZgt{} }\PYG{n}{pdf} \PYG{o}{=} \PYG{n}{np}\PYG{o}{.}\PYG{n}{exp}\PYG{p}{(}\PYG{o}{\PYGZhy{}}\PYG{n+nb}{abs}\PYG{p}{(}\PYG{n}{x}\PYG{o}{\PYGZhy{}}\PYG{n}{loc}\PYG{o}{/}\PYG{n}{scale}\PYG{p}{)}\PYG{p}{)}\PYG{o}{/}\PYG{p}{(}\PYG{l+m+mf}{2.}\PYG{o}{*}\PYG{n}{scale}\PYG{p}{)}
\PYG{g+gp}{\PYGZgt{}\PYGZgt{}\PYGZgt{} }\PYG{n}{plt}\PYG{o}{.}\PYG{n}{plot}\PYG{p}{(}\PYG{n}{x}\PYG{p}{,} \PYG{n}{pdf}\PYG{p}{)}
\end{Verbatim}

Plot Gaussian for comparison:

\begin{Verbatim}[commandchars=\\\{\}]
\PYG{g+gp}{\PYGZgt{}\PYGZgt{}\PYGZgt{} }\PYG{n}{g} \PYG{o}{=} \PYG{p}{(}\PYG{l+m+mi}{1}\PYG{o}{/}\PYG{p}{(}\PYG{n}{scale} \PYG{o}{*} \PYG{n}{np}\PYG{o}{.}\PYG{n}{sqrt}\PYG{p}{(}\PYG{l+m+mi}{2} \PYG{o}{*} \PYG{n}{np}\PYG{o}{.}\PYG{n}{pi}\PYG{p}{)}\PYG{p}{)} \PYG{o}{*} 
\PYG{g+gp}{... }     \PYG{n}{np}\PYG{o}{.}\PYG{n}{exp}\PYG{p}{(} \PYG{o}{\PYGZhy{}} \PYG{p}{(}\PYG{n}{x} \PYG{o}{\PYGZhy{}} \PYG{n}{loc}\PYG{p}{)}\PYG{o}{*}\PYG{o}{*}\PYG{l+m+mi}{2} \PYG{o}{/} \PYG{p}{(}\PYG{l+m+mi}{2} \PYG{o}{*} \PYG{n}{scale}\PYG{o}{*}\PYG{o}{*}\PYG{l+m+mi}{2}\PYG{p}{)} \PYG{p}{)}\PYG{p}{)}
\PYG{g+gp}{\PYGZgt{}\PYGZgt{}\PYGZgt{} }\PYG{n}{plt}\PYG{o}{.}\PYG{n}{plot}\PYG{p}{(}\PYG{n}{x}\PYG{p}{,}\PYG{n}{g}\PYG{p}{)}
\end{Verbatim}

\end{fulllineitems}

\index{logistic() (in module lib.graph.raf)}

\begin{fulllineitems}
\phantomsection\label{lib.graph:lib.graph.raf.logistic}\pysiglinewithargsret{\code{lib.graph.raf.}\bfcode{logistic}}{\emph{loc=0.0}, \emph{scale=1.0}, \emph{size=None}}{}
Draw samples from a Logistic distribution.

Samples are drawn from a Logistic distribution with specified
parameters, loc (location or mean, also median), and scale (\textgreater{}0).

loc : float

scale : float \textgreater{} 0.
\begin{description}
\item[{size}] \leavevmode{[}\{tuple, int\}{]}
Output shape.  If the given shape is, e.g., \code{(m, n, k)}, then
\code{m * n * k} samples are drawn.

\end{description}
\begin{description}
\item[{samples}] \leavevmode{[}\{ndarray, scalar\}{]}
where the values are all integers in  {[}0, n{]}.

\end{description}
\begin{description}
\item[{scipy.stats.distributions.logistic}] \leavevmode{[}probability density function,{]}
distribution or cumulative density function, etc.

\end{description}

The probability density for the Logistic distribution is
\begin{gather}
\begin{split}P(x) = P(x) = \frac{e^{-(x-\mu)/s}}{s(1+e^{-(x-\mu)/s})^2},\end{split}\notag
\end{gather}
where \(\mu\) = location and \(s\) = scale.

The Logistic distribution is used in Extreme Value problems where it
can act as a mixture of Gumbel distributions, in Epidemiology, and by
the World Chess Federation (FIDE) where it is used in the Elo ranking
system, assuming the performance of each player is a logistically
distributed random variable.

Draw samples from the distribution:

\begin{Verbatim}[commandchars=\\\{\}]
\PYG{g+gp}{\PYGZgt{}\PYGZgt{}\PYGZgt{} }\PYG{n}{loc}\PYG{p}{,} \PYG{n}{scale} \PYG{o}{=} \PYG{l+m+mi}{10}\PYG{p}{,} \PYG{l+m+mi}{1}
\PYG{g+gp}{\PYGZgt{}\PYGZgt{}\PYGZgt{} }\PYG{n}{s} \PYG{o}{=} \PYG{n}{np}\PYG{o}{.}\PYG{n}{random}\PYG{o}{.}\PYG{n}{logistic}\PYG{p}{(}\PYG{n}{loc}\PYG{p}{,} \PYG{n}{scale}\PYG{p}{,} \PYG{l+m+mi}{10000}\PYG{p}{)}
\PYG{g+gp}{\PYGZgt{}\PYGZgt{}\PYGZgt{} }\PYG{n}{count}\PYG{p}{,} \PYG{n}{bins}\PYG{p}{,} \PYG{n}{ignored} \PYG{o}{=} \PYG{n}{plt}\PYG{o}{.}\PYG{n}{hist}\PYG{p}{(}\PYG{n}{s}\PYG{p}{,} \PYG{n}{bins}\PYG{o}{=}\PYG{l+m+mi}{50}\PYG{p}{)}
\end{Verbatim}

\#   plot against distribution

\begin{Verbatim}[commandchars=\\\{\}]
\PYG{g+gp}{\PYGZgt{}\PYGZgt{}\PYGZgt{} }\PYG{k}{def} \PYG{n+nf}{logist}\PYG{p}{(}\PYG{n}{x}\PYG{p}{,} \PYG{n}{loc}\PYG{p}{,} \PYG{n}{scale}\PYG{p}{)}\PYG{p}{:}
\PYG{g+gp}{... }    \PYG{k}{return} \PYG{n}{exp}\PYG{p}{(}\PYG{p}{(}\PYG{n}{loc}\PYG{o}{\PYGZhy{}}\PYG{n}{x}\PYG{p}{)}\PYG{o}{/}\PYG{n}{scale}\PYG{p}{)}\PYG{o}{/}\PYG{p}{(}\PYG{n}{scale}\PYG{o}{*}\PYG{p}{(}\PYG{l+m+mi}{1}\PYG{o}{+}\PYG{n}{exp}\PYG{p}{(}\PYG{p}{(}\PYG{n}{loc}\PYG{o}{\PYGZhy{}}\PYG{n}{x}\PYG{p}{)}\PYG{o}{/}\PYG{n}{scale}\PYG{p}{)}\PYG{p}{)}\PYG{o}{*}\PYG{o}{*}\PYG{l+m+mi}{2}\PYG{p}{)}
\PYG{g+gp}{\PYGZgt{}\PYGZgt{}\PYGZgt{} }\PYG{n}{plt}\PYG{o}{.}\PYG{n}{plot}\PYG{p}{(}\PYG{n}{bins}\PYG{p}{,} \PYG{n}{logist}\PYG{p}{(}\PYG{n}{bins}\PYG{p}{,} \PYG{n}{loc}\PYG{p}{,} \PYG{n}{scale}\PYG{p}{)}\PYG{o}{*}\PYG{n}{count}\PYG{o}{.}\PYG{n}{max}\PYG{p}{(}\PYG{p}{)}\PYG{o}{/}\PYGZbs{}
\PYG{g+gp}{... }\PYG{n}{logist}\PYG{p}{(}\PYG{n}{bins}\PYG{p}{,} \PYG{n}{loc}\PYG{p}{,} \PYG{n}{scale}\PYG{p}{)}\PYG{o}{.}\PYG{n}{max}\PYG{p}{(}\PYG{p}{)}\PYG{p}{)}
\PYG{g+gp}{\PYGZgt{}\PYGZgt{}\PYGZgt{} }\PYG{n}{plt}\PYG{o}{.}\PYG{n}{show}\PYG{p}{(}\PYG{p}{)}
\end{Verbatim}

\end{fulllineitems}

\index{lognormal() (in module lib.graph.raf)}

\begin{fulllineitems}
\phantomsection\label{lib.graph:lib.graph.raf.lognormal}\pysiglinewithargsret{\code{lib.graph.raf.}\bfcode{lognormal}}{\emph{mean=0.0}, \emph{sigma=1.0}, \emph{size=None}}{}
Return samples drawn from a log-normal distribution.

Draw samples from a log-normal distribution with specified mean,
standard deviation, and array shape.  Note that the mean and standard
deviation are not the values for the distribution itself, but of the
underlying normal distribution it is derived from.
\begin{description}
\item[{mean}] \leavevmode{[}float{]}
Mean value of the underlying normal distribution

\item[{sigma}] \leavevmode{[}float, \textgreater{} 0.{]}
Standard deviation of the underlying normal distribution

\item[{size}] \leavevmode{[}tuple of ints{]}
Output shape.  If the given shape is, e.g., \code{(m, n, k)}, then
\code{m * n * k} samples are drawn.

\end{description}
\begin{description}
\item[{samples}] \leavevmode{[}ndarray or float{]}
The desired samples. An array of the same shape as \emph{size} if given,
if \emph{size} is None a float is returned.

\end{description}
\begin{description}
\item[{scipy.stats.lognorm}] \leavevmode{[}probability density function, distribution,{]}
cumulative density function, etc.

\end{description}

A variable \emph{x} has a log-normal distribution if \emph{log(x)} is normally
distributed.  The probability density function for the log-normal
distribution is:
\begin{gather}
\begin{split}p(x) = \frac{1}{\sigma x \sqrt{2\pi}}
e^{(-\frac{(ln(x)-\mu)^2}{2\sigma^2})}\end{split}\notag
\end{gather}
where \(\mu\) is the mean and \(\sigma\) is the standard
deviation of the normally distributed logarithm of the variable.
A log-normal distribution results if a random variable is the \emph{product}
of a large number of independent, identically-distributed variables in
the same way that a normal distribution results if the variable is the
\emph{sum} of a large number of independent, identically-distributed
variables.

Limpert, E., Stahel, W. A., and Abbt, M., ``Log-normal Distributions
across the Sciences: Keys and Clues,'' \emph{BioScience}, Vol. 51, No. 5,
May, 2001.  \href{http://stat.ethz.ch/~stahel/lognormal/bioscience.pdf}{http://stat.ethz.ch/\textasciitilde{}stahel/lognormal/bioscience.pdf}

Reiss, R.D. and Thomas, M., \emph{Statistical Analysis of Extreme Values},
Basel: Birkhauser Verlag, 2001, pp. 31-32.

Draw samples from the distribution:

\begin{Verbatim}[commandchars=\\\{\}]
\PYG{g+gp}{\PYGZgt{}\PYGZgt{}\PYGZgt{} }\PYG{n}{mu}\PYG{p}{,} \PYG{n}{sigma} \PYG{o}{=} \PYG{l+m+mf}{3.}\PYG{p}{,} \PYG{l+m+mf}{1.} \PYG{c}{\PYGZsh{} mean and standard deviation}
\PYG{g+gp}{\PYGZgt{}\PYGZgt{}\PYGZgt{} }\PYG{n}{s} \PYG{o}{=} \PYG{n}{np}\PYG{o}{.}\PYG{n}{random}\PYG{o}{.}\PYG{n}{lognormal}\PYG{p}{(}\PYG{n}{mu}\PYG{p}{,} \PYG{n}{sigma}\PYG{p}{,} \PYG{l+m+mi}{1000}\PYG{p}{)}
\end{Verbatim}

Display the histogram of the samples, along with
the probability density function:

\begin{Verbatim}[commandchars=\\\{\}]
\PYG{g+gp}{\PYGZgt{}\PYGZgt{}\PYGZgt{} }\PYG{k+kn}{import} \PYG{n+nn}{matplotlib.pyplot} \PYG{k+kn}{as} \PYG{n+nn}{plt}
\PYG{g+gp}{\PYGZgt{}\PYGZgt{}\PYGZgt{} }\PYG{n}{count}\PYG{p}{,} \PYG{n}{bins}\PYG{p}{,} \PYG{n}{ignored} \PYG{o}{=} \PYG{n}{plt}\PYG{o}{.}\PYG{n}{hist}\PYG{p}{(}\PYG{n}{s}\PYG{p}{,} \PYG{l+m+mi}{100}\PYG{p}{,} \PYG{n}{normed}\PYG{o}{=}\PYG{n+nb+bp}{True}\PYG{p}{,} \PYG{n}{align}\PYG{o}{=}\PYG{l+s}{\PYGZsq{}}\PYG{l+s}{mid}\PYG{l+s}{\PYGZsq{}}\PYG{p}{)}
\end{Verbatim}

\begin{Verbatim}[commandchars=\\\{\}]
\PYG{g+gp}{\PYGZgt{}\PYGZgt{}\PYGZgt{} }\PYG{n}{x} \PYG{o}{=} \PYG{n}{np}\PYG{o}{.}\PYG{n}{linspace}\PYG{p}{(}\PYG{n+nb}{min}\PYG{p}{(}\PYG{n}{bins}\PYG{p}{)}\PYG{p}{,} \PYG{n+nb}{max}\PYG{p}{(}\PYG{n}{bins}\PYG{p}{)}\PYG{p}{,} \PYG{l+m+mi}{10000}\PYG{p}{)}
\PYG{g+gp}{\PYGZgt{}\PYGZgt{}\PYGZgt{} }\PYG{n}{pdf} \PYG{o}{=} \PYG{p}{(}\PYG{n}{np}\PYG{o}{.}\PYG{n}{exp}\PYG{p}{(}\PYG{o}{\PYGZhy{}}\PYG{p}{(}\PYG{n}{np}\PYG{o}{.}\PYG{n}{log}\PYG{p}{(}\PYG{n}{x}\PYG{p}{)} \PYG{o}{\PYGZhy{}} \PYG{n}{mu}\PYG{p}{)}\PYG{o}{*}\PYG{o}{*}\PYG{l+m+mi}{2} \PYG{o}{/} \PYG{p}{(}\PYG{l+m+mi}{2} \PYG{o}{*} \PYG{n}{sigma}\PYG{o}{*}\PYG{o}{*}\PYG{l+m+mi}{2}\PYG{p}{)}\PYG{p}{)}
\PYG{g+gp}{... }       \PYG{o}{/} \PYG{p}{(}\PYG{n}{x} \PYG{o}{*} \PYG{n}{sigma} \PYG{o}{*} \PYG{n}{np}\PYG{o}{.}\PYG{n}{sqrt}\PYG{p}{(}\PYG{l+m+mi}{2} \PYG{o}{*} \PYG{n}{np}\PYG{o}{.}\PYG{n}{pi}\PYG{p}{)}\PYG{p}{)}\PYG{p}{)}
\end{Verbatim}

\begin{Verbatim}[commandchars=\\\{\}]
\PYG{g+gp}{\PYGZgt{}\PYGZgt{}\PYGZgt{} }\PYG{n}{plt}\PYG{o}{.}\PYG{n}{plot}\PYG{p}{(}\PYG{n}{x}\PYG{p}{,} \PYG{n}{pdf}\PYG{p}{,} \PYG{n}{linewidth}\PYG{o}{=}\PYG{l+m+mi}{2}\PYG{p}{,} \PYG{n}{color}\PYG{o}{=}\PYG{l+s}{\PYGZsq{}}\PYG{l+s}{r}\PYG{l+s}{\PYGZsq{}}\PYG{p}{)}
\PYG{g+gp}{\PYGZgt{}\PYGZgt{}\PYGZgt{} }\PYG{n}{plt}\PYG{o}{.}\PYG{n}{axis}\PYG{p}{(}\PYG{l+s}{\PYGZsq{}}\PYG{l+s}{tight}\PYG{l+s}{\PYGZsq{}}\PYG{p}{)}
\PYG{g+gp}{\PYGZgt{}\PYGZgt{}\PYGZgt{} }\PYG{n}{plt}\PYG{o}{.}\PYG{n}{show}\PYG{p}{(}\PYG{p}{)}
\end{Verbatim}

Demonstrate that taking the products of random samples from a uniform
distribution can be fit well by a log-normal probability density function.

\begin{Verbatim}[commandchars=\\\{\}]
\PYG{g+gp}{\PYGZgt{}\PYGZgt{}\PYGZgt{} }\PYG{c}{\PYGZsh{} Generate a thousand samples: each is the product of 100 random}
\PYG{g+gp}{\PYGZgt{}\PYGZgt{}\PYGZgt{} }\PYG{c}{\PYGZsh{} values, drawn from a normal distribution.}
\PYG{g+gp}{\PYGZgt{}\PYGZgt{}\PYGZgt{} }\PYG{n}{b} \PYG{o}{=} \PYG{p}{[}\PYG{p}{]}
\PYG{g+gp}{\PYGZgt{}\PYGZgt{}\PYGZgt{} }\PYG{k}{for} \PYG{n}{i} \PYG{o+ow}{in} \PYG{n+nb}{range}\PYG{p}{(}\PYG{l+m+mi}{1000}\PYG{p}{)}\PYG{p}{:}
\PYG{g+gp}{... }   \PYG{n}{a} \PYG{o}{=} \PYG{l+m+mf}{10.} \PYG{o}{+} \PYG{n}{np}\PYG{o}{.}\PYG{n}{random}\PYG{o}{.}\PYG{n}{random}\PYG{p}{(}\PYG{l+m+mi}{100}\PYG{p}{)}
\PYG{g+gp}{... }   \PYG{n}{b}\PYG{o}{.}\PYG{n}{append}\PYG{p}{(}\PYG{n}{np}\PYG{o}{.}\PYG{n}{product}\PYG{p}{(}\PYG{n}{a}\PYG{p}{)}\PYG{p}{)}
\end{Verbatim}

\begin{Verbatim}[commandchars=\\\{\}]
\PYG{g+gp}{\PYGZgt{}\PYGZgt{}\PYGZgt{} }\PYG{n}{b} \PYG{o}{=} \PYG{n}{np}\PYG{o}{.}\PYG{n}{array}\PYG{p}{(}\PYG{n}{b}\PYG{p}{)} \PYG{o}{/} \PYG{n}{np}\PYG{o}{.}\PYG{n}{min}\PYG{p}{(}\PYG{n}{b}\PYG{p}{)} \PYG{c}{\PYGZsh{} scale values to be positive}
\PYG{g+gp}{\PYGZgt{}\PYGZgt{}\PYGZgt{} }\PYG{n}{count}\PYG{p}{,} \PYG{n}{bins}\PYG{p}{,} \PYG{n}{ignored} \PYG{o}{=} \PYG{n}{plt}\PYG{o}{.}\PYG{n}{hist}\PYG{p}{(}\PYG{n}{b}\PYG{p}{,} \PYG{l+m+mi}{100}\PYG{p}{,} \PYG{n}{normed}\PYG{o}{=}\PYG{n+nb+bp}{True}\PYG{p}{,} \PYG{n}{align}\PYG{o}{=}\PYG{l+s}{\PYGZsq{}}\PYG{l+s}{center}\PYG{l+s}{\PYGZsq{}}\PYG{p}{)}
\PYG{g+gp}{\PYGZgt{}\PYGZgt{}\PYGZgt{} }\PYG{n}{sigma} \PYG{o}{=} \PYG{n}{np}\PYG{o}{.}\PYG{n}{std}\PYG{p}{(}\PYG{n}{np}\PYG{o}{.}\PYG{n}{log}\PYG{p}{(}\PYG{n}{b}\PYG{p}{)}\PYG{p}{)}
\PYG{g+gp}{\PYGZgt{}\PYGZgt{}\PYGZgt{} }\PYG{n}{mu} \PYG{o}{=} \PYG{n}{np}\PYG{o}{.}\PYG{n}{mean}\PYG{p}{(}\PYG{n}{np}\PYG{o}{.}\PYG{n}{log}\PYG{p}{(}\PYG{n}{b}\PYG{p}{)}\PYG{p}{)}
\end{Verbatim}

\begin{Verbatim}[commandchars=\\\{\}]
\PYG{g+gp}{\PYGZgt{}\PYGZgt{}\PYGZgt{} }\PYG{n}{x} \PYG{o}{=} \PYG{n}{np}\PYG{o}{.}\PYG{n}{linspace}\PYG{p}{(}\PYG{n+nb}{min}\PYG{p}{(}\PYG{n}{bins}\PYG{p}{)}\PYG{p}{,} \PYG{n+nb}{max}\PYG{p}{(}\PYG{n}{bins}\PYG{p}{)}\PYG{p}{,} \PYG{l+m+mi}{10000}\PYG{p}{)}
\PYG{g+gp}{\PYGZgt{}\PYGZgt{}\PYGZgt{} }\PYG{n}{pdf} \PYG{o}{=} \PYG{p}{(}\PYG{n}{np}\PYG{o}{.}\PYG{n}{exp}\PYG{p}{(}\PYG{o}{\PYGZhy{}}\PYG{p}{(}\PYG{n}{np}\PYG{o}{.}\PYG{n}{log}\PYG{p}{(}\PYG{n}{x}\PYG{p}{)} \PYG{o}{\PYGZhy{}} \PYG{n}{mu}\PYG{p}{)}\PYG{o}{*}\PYG{o}{*}\PYG{l+m+mi}{2} \PYG{o}{/} \PYG{p}{(}\PYG{l+m+mi}{2} \PYG{o}{*} \PYG{n}{sigma}\PYG{o}{*}\PYG{o}{*}\PYG{l+m+mi}{2}\PYG{p}{)}\PYG{p}{)}
\PYG{g+gp}{... }       \PYG{o}{/} \PYG{p}{(}\PYG{n}{x} \PYG{o}{*} \PYG{n}{sigma} \PYG{o}{*} \PYG{n}{np}\PYG{o}{.}\PYG{n}{sqrt}\PYG{p}{(}\PYG{l+m+mi}{2} \PYG{o}{*} \PYG{n}{np}\PYG{o}{.}\PYG{n}{pi}\PYG{p}{)}\PYG{p}{)}\PYG{p}{)}
\end{Verbatim}

\begin{Verbatim}[commandchars=\\\{\}]
\PYG{g+gp}{\PYGZgt{}\PYGZgt{}\PYGZgt{} }\PYG{n}{plt}\PYG{o}{.}\PYG{n}{plot}\PYG{p}{(}\PYG{n}{x}\PYG{p}{,} \PYG{n}{pdf}\PYG{p}{,} \PYG{n}{color}\PYG{o}{=}\PYG{l+s}{\PYGZsq{}}\PYG{l+s}{r}\PYG{l+s}{\PYGZsq{}}\PYG{p}{,} \PYG{n}{linewidth}\PYG{o}{=}\PYG{l+m+mi}{2}\PYG{p}{)}
\PYG{g+gp}{\PYGZgt{}\PYGZgt{}\PYGZgt{} }\PYG{n}{plt}\PYG{o}{.}\PYG{n}{show}\PYG{p}{(}\PYG{p}{)}
\end{Verbatim}

\end{fulllineitems}

\index{logseries() (in module lib.graph.raf)}

\begin{fulllineitems}
\phantomsection\label{lib.graph:lib.graph.raf.logseries}\pysiglinewithargsret{\code{lib.graph.raf.}\bfcode{logseries}}{\emph{p}, \emph{size=None}}{}
Draw samples from a Logarithmic Series distribution.

Samples are drawn from a Log Series distribution with specified
parameter, p (probability, 0 \textless{} p \textless{} 1).

loc : float

scale : float \textgreater{} 0.
\begin{description}
\item[{size}] \leavevmode{[}\{tuple, int\}{]}
Output shape.  If the given shape is, e.g., \code{(m, n, k)}, then
\code{m * n * k} samples are drawn.

\end{description}
\begin{description}
\item[{samples}] \leavevmode{[}\{ndarray, scalar\}{]}
where the values are all integers in  {[}0, n{]}.

\end{description}
\begin{description}
\item[{scipy.stats.distributions.logser}] \leavevmode{[}probability density function,{]}
distribution or cumulative density function, etc.

\end{description}

The probability density for the Log Series distribution is
\begin{gather}
\begin{split}P(k) = \frac{-p^k}{k \ln(1-p)},\end{split}\notag
\end{gather}
where p = probability.

The Log Series distribution is frequently used to represent species
richness and occurrence, first proposed by Fisher, Corbet, and
Williams in 1943 {[}2{]}.  It may also be used to model the numbers of
occupants seen in cars {[}3{]}.

Draw samples from the distribution:

\begin{Verbatim}[commandchars=\\\{\}]
\PYG{g+gp}{\PYGZgt{}\PYGZgt{}\PYGZgt{} }\PYG{n}{a} \PYG{o}{=} \PYG{o}{.}\PYG{l+m+mi}{6}
\PYG{g+gp}{\PYGZgt{}\PYGZgt{}\PYGZgt{} }\PYG{n}{s} \PYG{o}{=} \PYG{n}{np}\PYG{o}{.}\PYG{n}{random}\PYG{o}{.}\PYG{n}{logseries}\PYG{p}{(}\PYG{n}{a}\PYG{p}{,} \PYG{l+m+mi}{10000}\PYG{p}{)}
\PYG{g+gp}{\PYGZgt{}\PYGZgt{}\PYGZgt{} }\PYG{n}{count}\PYG{p}{,} \PYG{n}{bins}\PYG{p}{,} \PYG{n}{ignored} \PYG{o}{=} \PYG{n}{plt}\PYG{o}{.}\PYG{n}{hist}\PYG{p}{(}\PYG{n}{s}\PYG{p}{)}
\end{Verbatim}

\#   plot against distribution

\begin{Verbatim}[commandchars=\\\{\}]
\PYG{g+gp}{\PYGZgt{}\PYGZgt{}\PYGZgt{} }\PYG{k}{def} \PYG{n+nf}{logseries}\PYG{p}{(}\PYG{n}{k}\PYG{p}{,} \PYG{n}{p}\PYG{p}{)}\PYG{p}{:}
\PYG{g+gp}{... }    \PYG{k}{return} \PYG{o}{\PYGZhy{}}\PYG{n}{p}\PYG{o}{*}\PYG{o}{*}\PYG{n}{k}\PYG{o}{/}\PYG{p}{(}\PYG{n}{k}\PYG{o}{*}\PYG{n}{log}\PYG{p}{(}\PYG{l+m+mi}{1}\PYG{o}{\PYGZhy{}}\PYG{n}{p}\PYG{p}{)}\PYG{p}{)}
\PYG{g+gp}{\PYGZgt{}\PYGZgt{}\PYGZgt{} }\PYG{n}{plt}\PYG{o}{.}\PYG{n}{plot}\PYG{p}{(}\PYG{n}{bins}\PYG{p}{,} \PYG{n}{logseries}\PYG{p}{(}\PYG{n}{bins}\PYG{p}{,} \PYG{n}{a}\PYG{p}{)}\PYG{o}{*}\PYG{n}{count}\PYG{o}{.}\PYG{n}{max}\PYG{p}{(}\PYG{p}{)}\PYG{o}{/}
\PYG{g+go}{             logseries(bins, a).max(), \PYGZsq{}r\PYGZsq{})}
\PYG{g+gp}{\PYGZgt{}\PYGZgt{}\PYGZgt{} }\PYG{n}{plt}\PYG{o}{.}\PYG{n}{show}\PYG{p}{(}\PYG{p}{)}
\end{Verbatim}

\end{fulllineitems}

\index{multinomial() (in module lib.graph.raf)}

\begin{fulllineitems}
\phantomsection\label{lib.graph:lib.graph.raf.multinomial}\pysiglinewithargsret{\code{lib.graph.raf.}\bfcode{multinomial}}{\emph{n}, \emph{pvals}, \emph{size=None}}{}
Draw samples from a multinomial distribution.

The multinomial distribution is a multivariate generalisation of the
binomial distribution.  Take an experiment with one of \code{p}
possible outcomes.  An example of such an experiment is throwing a dice,
where the outcome can be 1 through 6.  Each sample drawn from the
distribution represents \emph{n} such experiments.  Its values,
\code{X\_i = {[}X\_0, X\_1, ..., X\_p{]}}, represent the number of times the outcome
was \code{i}.
\begin{description}
\item[{n}] \leavevmode{[}int{]}
Number of experiments.

\item[{pvals}] \leavevmode{[}sequence of floats, length p{]}
Probabilities of each of the \code{p} different outcomes.  These
should sum to 1 (however, the last element is always assumed to
account for the remaining probability, as long as
\code{sum(pvals{[}:-1{]}) \textless{}= 1)}.

\item[{size}] \leavevmode{[}tuple of ints{]}
Given a \emph{size} of \code{(M, N, K)}, then \code{M*N*K} samples are drawn,
and the output shape becomes \code{(M, N, K, p)}, since each sample
has shape \code{(p,)}.

\end{description}

Throw a dice 20 times:

\begin{Verbatim}[commandchars=\\\{\}]
\PYG{g+gp}{\PYGZgt{}\PYGZgt{}\PYGZgt{} }\PYG{n}{np}\PYG{o}{.}\PYG{n}{random}\PYG{o}{.}\PYG{n}{multinomial}\PYG{p}{(}\PYG{l+m+mi}{20}\PYG{p}{,} \PYG{p}{[}\PYG{l+m+mi}{1}\PYG{o}{/}\PYG{l+m+mf}{6.}\PYG{p}{]}\PYG{o}{*}\PYG{l+m+mi}{6}\PYG{p}{,} \PYG{n}{size}\PYG{o}{=}\PYG{l+m+mi}{1}\PYG{p}{)}
\PYG{g+go}{array([[4, 1, 7, 5, 2, 1]])}
\end{Verbatim}

It landed 4 times on 1, once on 2, etc.

Now, throw the dice 20 times, and 20 times again:

\begin{Verbatim}[commandchars=\\\{\}]
\PYG{g+gp}{\PYGZgt{}\PYGZgt{}\PYGZgt{} }\PYG{n}{np}\PYG{o}{.}\PYG{n}{random}\PYG{o}{.}\PYG{n}{multinomial}\PYG{p}{(}\PYG{l+m+mi}{20}\PYG{p}{,} \PYG{p}{[}\PYG{l+m+mi}{1}\PYG{o}{/}\PYG{l+m+mf}{6.}\PYG{p}{]}\PYG{o}{*}\PYG{l+m+mi}{6}\PYG{p}{,} \PYG{n}{size}\PYG{o}{=}\PYG{l+m+mi}{2}\PYG{p}{)}
\PYG{g+go}{array([[3, 4, 3, 3, 4, 3],}
\PYG{g+go}{       [2, 4, 3, 4, 0, 7]])}
\end{Verbatim}

For the first run, we threw 3 times 1, 4 times 2, etc.  For the second,
we threw 2 times 1, 4 times 2, etc.

A loaded dice is more likely to land on number 6:

\begin{Verbatim}[commandchars=\\\{\}]
\PYG{g+gp}{\PYGZgt{}\PYGZgt{}\PYGZgt{} }\PYG{n}{np}\PYG{o}{.}\PYG{n}{random}\PYG{o}{.}\PYG{n}{multinomial}\PYG{p}{(}\PYG{l+m+mi}{100}\PYG{p}{,} \PYG{p}{[}\PYG{l+m+mi}{1}\PYG{o}{/}\PYG{l+m+mf}{7.}\PYG{p}{]}\PYG{o}{*}\PYG{l+m+mi}{5}\PYG{p}{)}
\PYG{g+go}{array([13, 16, 13, 16, 42])}
\end{Verbatim}

\end{fulllineitems}

\index{multivariate\_normal() (in module lib.graph.raf)}

\begin{fulllineitems}
\phantomsection\label{lib.graph:lib.graph.raf.multivariate_normal}\pysiglinewithargsret{\code{lib.graph.raf.}\bfcode{multivariate\_normal}}{\emph{mean}, \emph{cov}\optional{, \emph{size}}}{}
Draw random samples from a multivariate normal distribution.

The multivariate normal, multinormal or Gaussian distribution is a
generalization of the one-dimensional normal distribution to higher
dimensions.  Such a distribution is specified by its mean and
covariance matrix.  These parameters are analogous to the mean
(average or ``center'') and variance (standard deviation, or ``width,''
squared) of the one-dimensional normal distribution.
\begin{description}
\item[{mean}] \leavevmode{[}1-D array\_like, of length N{]}
Mean of the N-dimensional distribution.

\item[{cov}] \leavevmode{[}2-D array\_like, of shape (N, N){]}
Covariance matrix of the distribution.  Must be symmetric and
positive semi-definite for ``physically meaningful'' results.

\item[{size}] \leavevmode{[}int or tuple of ints, optional{]}
Given a shape of, for example, \code{(m,n,k)}, \code{m*n*k} samples are
generated, and packed in an \emph{m}-by-\emph{n}-by-\emph{k} arrangement.  Because
each sample is \emph{N}-dimensional, the output shape is \code{(m,n,k,N)}.
If no shape is specified, a single (\emph{N}-D) sample is returned.

\end{description}
\begin{description}
\item[{out}] \leavevmode{[}ndarray{]}
The drawn samples, of shape \emph{size}, if that was provided.  If not,
the shape is \code{(N,)}.

In other words, each entry \code{out{[}i,j,...,:{]}} is an N-dimensional
value drawn from the distribution.

\end{description}

The mean is a coordinate in N-dimensional space, which represents the
location where samples are most likely to be generated.  This is
analogous to the peak of the bell curve for the one-dimensional or
univariate normal distribution.

Covariance indicates the level to which two variables vary together.
From the multivariate normal distribution, we draw N-dimensional
samples, \(X = [x_1, x_2, ... x_N]\).  The covariance matrix
element \(C_{ij}\) is the covariance of \(x_i\) and \(x_j\).
The element \(C_{ii}\) is the variance of \(x_i\) (i.e. its
``spread'').

Instead of specifying the full covariance matrix, popular
approximations include:
\begin{itemize}
\item {} 
Spherical covariance (\emph{cov} is a multiple of the identity matrix)

\item {} 
Diagonal covariance (\emph{cov} has non-negative elements, and only on
the diagonal)

\end{itemize}

This geometrical property can be seen in two dimensions by plotting
generated data-points:

\begin{Verbatim}[commandchars=\\\{\}]
\PYG{g+gp}{\PYGZgt{}\PYGZgt{}\PYGZgt{} }\PYG{n}{mean} \PYG{o}{=} \PYG{p}{[}\PYG{l+m+mi}{0}\PYG{p}{,}\PYG{l+m+mi}{0}\PYG{p}{]}
\PYG{g+gp}{\PYGZgt{}\PYGZgt{}\PYGZgt{} }\PYG{n}{cov} \PYG{o}{=} \PYG{p}{[}\PYG{p}{[}\PYG{l+m+mi}{1}\PYG{p}{,}\PYG{l+m+mi}{0}\PYG{p}{]}\PYG{p}{,}\PYG{p}{[}\PYG{l+m+mi}{0}\PYG{p}{,}\PYG{l+m+mi}{100}\PYG{p}{]}\PYG{p}{]} \PYG{c}{\PYGZsh{} diagonal covariance, points lie on x or y\PYGZhy{}axis}
\end{Verbatim}

\begin{Verbatim}[commandchars=\\\{\}]
\PYG{g+gp}{\PYGZgt{}\PYGZgt{}\PYGZgt{} }\PYG{k+kn}{import} \PYG{n+nn}{matplotlib.pyplot} \PYG{k+kn}{as} \PYG{n+nn}{plt}
\PYG{g+gp}{\PYGZgt{}\PYGZgt{}\PYGZgt{} }\PYG{n}{x}\PYG{p}{,}\PYG{n}{y} \PYG{o}{=} \PYG{n}{np}\PYG{o}{.}\PYG{n}{random}\PYG{o}{.}\PYG{n}{multivariate\PYGZus{}normal}\PYG{p}{(}\PYG{n}{mean}\PYG{p}{,}\PYG{n}{cov}\PYG{p}{,}\PYG{l+m+mi}{5000}\PYG{p}{)}\PYG{o}{.}\PYG{n}{T}
\PYG{g+gp}{\PYGZgt{}\PYGZgt{}\PYGZgt{} }\PYG{n}{plt}\PYG{o}{.}\PYG{n}{plot}\PYG{p}{(}\PYG{n}{x}\PYG{p}{,}\PYG{n}{y}\PYG{p}{,}\PYG{l+s}{\PYGZsq{}}\PYG{l+s}{x}\PYG{l+s}{\PYGZsq{}}\PYG{p}{)}\PYG{p}{;} \PYG{n}{plt}\PYG{o}{.}\PYG{n}{axis}\PYG{p}{(}\PYG{l+s}{\PYGZsq{}}\PYG{l+s}{equal}\PYG{l+s}{\PYGZsq{}}\PYG{p}{)}\PYG{p}{;} \PYG{n}{plt}\PYG{o}{.}\PYG{n}{show}\PYG{p}{(}\PYG{p}{)}
\end{Verbatim}

Note that the covariance matrix must be non-negative definite.

Papoulis, A., \emph{Probability, Random Variables, and Stochastic Processes},
3rd ed., New York: McGraw-Hill, 1991.

Duda, R. O., Hart, P. E., and Stork, D. G., \emph{Pattern Classification},
2nd ed., New York: Wiley, 2001.

\begin{Verbatim}[commandchars=\\\{\}]
\PYG{g+gp}{\PYGZgt{}\PYGZgt{}\PYGZgt{} }\PYG{n}{mean} \PYG{o}{=} \PYG{p}{(}\PYG{l+m+mi}{1}\PYG{p}{,}\PYG{l+m+mi}{2}\PYG{p}{)}
\PYG{g+gp}{\PYGZgt{}\PYGZgt{}\PYGZgt{} }\PYG{n}{cov} \PYG{o}{=} \PYG{p}{[}\PYG{p}{[}\PYG{l+m+mi}{1}\PYG{p}{,}\PYG{l+m+mi}{0}\PYG{p}{]}\PYG{p}{,}\PYG{p}{[}\PYG{l+m+mi}{1}\PYG{p}{,}\PYG{l+m+mi}{0}\PYG{p}{]}\PYG{p}{]}
\PYG{g+gp}{\PYGZgt{}\PYGZgt{}\PYGZgt{} }\PYG{n}{x} \PYG{o}{=} \PYG{n}{np}\PYG{o}{.}\PYG{n}{random}\PYG{o}{.}\PYG{n}{multivariate\PYGZus{}normal}\PYG{p}{(}\PYG{n}{mean}\PYG{p}{,}\PYG{n}{cov}\PYG{p}{,}\PYG{p}{(}\PYG{l+m+mi}{3}\PYG{p}{,}\PYG{l+m+mi}{3}\PYG{p}{)}\PYG{p}{)}
\PYG{g+gp}{\PYGZgt{}\PYGZgt{}\PYGZgt{} }\PYG{n}{x}\PYG{o}{.}\PYG{n}{shape}
\PYG{g+go}{(3, 3, 2)}
\end{Verbatim}

The following is probably true, given that 0.6 is roughly twice the
standard deviation:

\begin{Verbatim}[commandchars=\\\{\}]
\PYG{g+gp}{\PYGZgt{}\PYGZgt{}\PYGZgt{} }\PYG{k}{print} \PYG{n+nb}{list}\PYG{p}{(} \PYG{p}{(}\PYG{n}{x}\PYG{p}{[}\PYG{l+m+mi}{0}\PYG{p}{,}\PYG{l+m+mi}{0}\PYG{p}{,}\PYG{p}{:}\PYG{p}{]} \PYG{o}{\PYGZhy{}} \PYG{n}{mean}\PYG{p}{)} \PYG{o}{\PYGZlt{}} \PYG{l+m+mf}{0.6} \PYG{p}{)}
\PYG{g+go}{[True, True]}
\end{Verbatim}

\end{fulllineitems}

\index{negative\_binomial() (in module lib.graph.raf)}

\begin{fulllineitems}
\phantomsection\label{lib.graph:lib.graph.raf.negative_binomial}\pysiglinewithargsret{\code{lib.graph.raf.}\bfcode{negative\_binomial}}{\emph{n}, \emph{p}, \emph{size=None}}{}
Draw samples from a negative\_binomial distribution.

Samples are drawn from a negative\_Binomial distribution with specified
parameters, \emph{n} trials and \emph{p} probability of success where \emph{n} is an
integer \textgreater{} 0 and \emph{p} is in the interval {[}0, 1{]}.
\begin{description}
\item[{n}] \leavevmode{[}int{]}
Parameter, \textgreater{} 0.

\item[{p}] \leavevmode{[}float{]}
Parameter, \textgreater{}= 0 and \textless{}=1.

\item[{size}] \leavevmode{[}int or tuple of ints{]}
Output shape. If the given shape is, e.g., \code{(m, n, k)}, then
\code{m * n * k} samples are drawn.

\end{description}
\begin{description}
\item[{samples}] \leavevmode{[}int or ndarray of ints{]}
Drawn samples.

\end{description}

The probability density for the Negative Binomial distribution is
\begin{gather}
\begin{split}P(N;n,p) = \binom{N+n-1}{n-1}p^{n}(1-p)^{N},\end{split}\notag
\end{gather}
where \(n-1\) is the number of successes, \(p\) is the probability
of success, and \(N+n-1\) is the number of trials.

The negative binomial distribution gives the probability of n-1 successes
and N failures in N+n-1 trials, and success on the (N+n)th trial.

If one throws a die repeatedly until the third time a ``1'' appears, then the
probability distribution of the number of non-``1''s that appear before the
third ``1'' is a negative binomial distribution.

Draw samples from the distribution:

A real world example. A company drills wild-cat oil exploration wells, each
with an estimated probability of success of 0.1.  What is the probability
of having one success for each successive well, that is what is the
probability of a single success after drilling 5 wells, after 6 wells,
etc.?

\begin{Verbatim}[commandchars=\\\{\}]
\PYG{g+gp}{\PYGZgt{}\PYGZgt{}\PYGZgt{} }\PYG{n}{s} \PYG{o}{=} \PYG{n}{np}\PYG{o}{.}\PYG{n}{random}\PYG{o}{.}\PYG{n}{negative\PYGZus{}binomial}\PYG{p}{(}\PYG{l+m+mi}{1}\PYG{p}{,} \PYG{l+m+mf}{0.1}\PYG{p}{,} \PYG{l+m+mi}{100000}\PYG{p}{)}
\PYG{g+gp}{\PYGZgt{}\PYGZgt{}\PYGZgt{} }\PYG{k}{for} \PYG{n}{i} \PYG{o+ow}{in} \PYG{n+nb}{range}\PYG{p}{(}\PYG{l+m+mi}{1}\PYG{p}{,} \PYG{l+m+mi}{11}\PYG{p}{)}\PYG{p}{:}
\PYG{g+gp}{... }   \PYG{n}{probability} \PYG{o}{=} \PYG{n+nb}{sum}\PYG{p}{(}\PYG{n}{s}\PYG{o}{\PYGZlt{}}\PYG{n}{i}\PYG{p}{)} \PYG{o}{/} \PYG{l+m+mf}{100000.}
\PYG{g+gp}{... }   \PYG{k}{print} \PYG{n}{i}\PYG{p}{,} \PYG{l+s}{\PYGZdq{}}\PYG{l+s}{wells drilled, probability of one success =}\PYG{l+s}{\PYGZdq{}}\PYG{p}{,} \PYG{n}{probability}
\end{Verbatim}

\end{fulllineitems}

\index{noncentral\_chisquare() (in module lib.graph.raf)}

\begin{fulllineitems}
\phantomsection\label{lib.graph:lib.graph.raf.noncentral_chisquare}\pysiglinewithargsret{\code{lib.graph.raf.}\bfcode{noncentral\_chisquare}}{\emph{df}, \emph{nonc}, \emph{size=None}}{}
Draw samples from a noncentral chi-square distribution.

The noncentral \(\chi^2\) distribution is a generalisation of
the \(\chi^2\) distribution.
\begin{description}
\item[{df}] \leavevmode{[}int{]}
Degrees of freedom, should be \textgreater{}= 1.

\item[{nonc}] \leavevmode{[}float{]}
Non-centrality, should be \textgreater{} 0.

\item[{size}] \leavevmode{[}int or tuple of ints{]}
Shape of the output.

\end{description}

The probability density function for the noncentral Chi-square distribution
is
\begin{gather}
\begin{split}P(x;df,nonc) = \sum^{\infty}_{i=0}
\frac{e^{-nonc/2}(nonc/2)^{i}}{i!}P_{Y_{df+2i}}(x),\end{split}\notag
\end{gather}
where \(Y_{q}\) is the Chi-square with q degrees of freedom.

In Delhi (2007), it is noted that the noncentral chi-square is useful in
bombing and coverage problems, the probability of killing the point target
given by the noncentral chi-squared distribution.

Draw values from the distribution and plot the histogram

\begin{Verbatim}[commandchars=\\\{\}]
\PYG{g+gp}{\PYGZgt{}\PYGZgt{}\PYGZgt{} }\PYG{k+kn}{import} \PYG{n+nn}{matplotlib.pyplot} \PYG{k+kn}{as} \PYG{n+nn}{plt}
\PYG{g+gp}{\PYGZgt{}\PYGZgt{}\PYGZgt{} }\PYG{n}{values} \PYG{o}{=} \PYG{n}{plt}\PYG{o}{.}\PYG{n}{hist}\PYG{p}{(}\PYG{n}{np}\PYG{o}{.}\PYG{n}{random}\PYG{o}{.}\PYG{n}{noncentral\PYGZus{}chisquare}\PYG{p}{(}\PYG{l+m+mi}{3}\PYG{p}{,} \PYG{l+m+mi}{20}\PYG{p}{,} \PYG{l+m+mi}{100000}\PYG{p}{)}\PYG{p}{,}
\PYG{g+gp}{... }                  \PYG{n}{bins}\PYG{o}{=}\PYG{l+m+mi}{200}\PYG{p}{,} \PYG{n}{normed}\PYG{o}{=}\PYG{n+nb+bp}{True}\PYG{p}{)}
\PYG{g+gp}{\PYGZgt{}\PYGZgt{}\PYGZgt{} }\PYG{n}{plt}\PYG{o}{.}\PYG{n}{show}\PYG{p}{(}\PYG{p}{)}
\end{Verbatim}

Draw values from a noncentral chisquare with very small noncentrality,
and compare to a chisquare.

\begin{Verbatim}[commandchars=\\\{\}]
\PYG{g+gp}{\PYGZgt{}\PYGZgt{}\PYGZgt{} }\PYG{n}{plt}\PYG{o}{.}\PYG{n}{figure}\PYG{p}{(}\PYG{p}{)}
\PYG{g+gp}{\PYGZgt{}\PYGZgt{}\PYGZgt{} }\PYG{n}{values} \PYG{o}{=} \PYG{n}{plt}\PYG{o}{.}\PYG{n}{hist}\PYG{p}{(}\PYG{n}{np}\PYG{o}{.}\PYG{n}{random}\PYG{o}{.}\PYG{n}{noncentral\PYGZus{}chisquare}\PYG{p}{(}\PYG{l+m+mi}{3}\PYG{p}{,} \PYG{o}{.}\PYG{l+m+mo}{0000001}\PYG{p}{,} \PYG{l+m+mi}{100000}\PYG{p}{)}\PYG{p}{,}
\PYG{g+gp}{... }                  \PYG{n}{bins}\PYG{o}{=}\PYG{n}{np}\PYG{o}{.}\PYG{n}{arange}\PYG{p}{(}\PYG{l+m+mf}{0.}\PYG{p}{,} \PYG{l+m+mi}{25}\PYG{p}{,} \PYG{o}{.}\PYG{l+m+mi}{1}\PYG{p}{)}\PYG{p}{,} \PYG{n}{normed}\PYG{o}{=}\PYG{n+nb+bp}{True}\PYG{p}{)}
\PYG{g+gp}{\PYGZgt{}\PYGZgt{}\PYGZgt{} }\PYG{n}{values2} \PYG{o}{=} \PYG{n}{plt}\PYG{o}{.}\PYG{n}{hist}\PYG{p}{(}\PYG{n}{np}\PYG{o}{.}\PYG{n}{random}\PYG{o}{.}\PYG{n}{chisquare}\PYG{p}{(}\PYG{l+m+mi}{3}\PYG{p}{,} \PYG{l+m+mi}{100000}\PYG{p}{)}\PYG{p}{,}
\PYG{g+gp}{... }                   \PYG{n}{bins}\PYG{o}{=}\PYG{n}{np}\PYG{o}{.}\PYG{n}{arange}\PYG{p}{(}\PYG{l+m+mf}{0.}\PYG{p}{,} \PYG{l+m+mi}{25}\PYG{p}{,} \PYG{o}{.}\PYG{l+m+mi}{1}\PYG{p}{)}\PYG{p}{,} \PYG{n}{normed}\PYG{o}{=}\PYG{n+nb+bp}{True}\PYG{p}{)}
\PYG{g+gp}{\PYGZgt{}\PYGZgt{}\PYGZgt{} }\PYG{n}{plt}\PYG{o}{.}\PYG{n}{plot}\PYG{p}{(}\PYG{n}{values}\PYG{p}{[}\PYG{l+m+mi}{1}\PYG{p}{]}\PYG{p}{[}\PYG{l+m+mi}{0}\PYG{p}{:}\PYG{o}{\PYGZhy{}}\PYG{l+m+mi}{1}\PYG{p}{]}\PYG{p}{,} \PYG{n}{values}\PYG{p}{[}\PYG{l+m+mi}{0}\PYG{p}{]}\PYG{o}{\PYGZhy{}}\PYG{n}{values2}\PYG{p}{[}\PYG{l+m+mi}{0}\PYG{p}{]}\PYG{p}{,} \PYG{l+s}{\PYGZsq{}}\PYG{l+s}{ob}\PYG{l+s}{\PYGZsq{}}\PYG{p}{)}
\PYG{g+gp}{\PYGZgt{}\PYGZgt{}\PYGZgt{} }\PYG{n}{plt}\PYG{o}{.}\PYG{n}{show}\PYG{p}{(}\PYG{p}{)}
\end{Verbatim}

Demonstrate how large values of non-centrality lead to a more symmetric
distribution.

\begin{Verbatim}[commandchars=\\\{\}]
\PYG{g+gp}{\PYGZgt{}\PYGZgt{}\PYGZgt{} }\PYG{n}{plt}\PYG{o}{.}\PYG{n}{figure}\PYG{p}{(}\PYG{p}{)}
\PYG{g+gp}{\PYGZgt{}\PYGZgt{}\PYGZgt{} }\PYG{n}{values} \PYG{o}{=} \PYG{n}{plt}\PYG{o}{.}\PYG{n}{hist}\PYG{p}{(}\PYG{n}{np}\PYG{o}{.}\PYG{n}{random}\PYG{o}{.}\PYG{n}{noncentral\PYGZus{}chisquare}\PYG{p}{(}\PYG{l+m+mi}{3}\PYG{p}{,} \PYG{l+m+mi}{20}\PYG{p}{,} \PYG{l+m+mi}{100000}\PYG{p}{)}\PYG{p}{,}
\PYG{g+gp}{... }                  \PYG{n}{bins}\PYG{o}{=}\PYG{l+m+mi}{200}\PYG{p}{,} \PYG{n}{normed}\PYG{o}{=}\PYG{n+nb+bp}{True}\PYG{p}{)}
\PYG{g+gp}{\PYGZgt{}\PYGZgt{}\PYGZgt{} }\PYG{n}{plt}\PYG{o}{.}\PYG{n}{show}\PYG{p}{(}\PYG{p}{)}
\end{Verbatim}

\end{fulllineitems}

\index{noncentral\_f() (in module lib.graph.raf)}

\begin{fulllineitems}
\phantomsection\label{lib.graph:lib.graph.raf.noncentral_f}\pysiglinewithargsret{\code{lib.graph.raf.}\bfcode{noncentral\_f}}{\emph{dfnum}, \emph{dfden}, \emph{nonc}, \emph{size=None}}{}
Draw samples from the noncentral F distribution.

Samples are drawn from an F distribution with specified parameters,
\emph{dfnum} (degrees of freedom in numerator) and \emph{dfden} (degrees of
freedom in denominator), where both parameters \textgreater{} 1.
\emph{nonc} is the non-centrality parameter.
\begin{description}
\item[{dfnum}] \leavevmode{[}int{]}
Parameter, should be \textgreater{} 1.

\item[{dfden}] \leavevmode{[}int{]}
Parameter, should be \textgreater{} 1.

\item[{nonc}] \leavevmode{[}float{]}
Parameter, should be \textgreater{}= 0.

\item[{size}] \leavevmode{[}int or tuple of ints{]}
Output shape. If the given shape is, e.g., \code{(m, n, k)}, then
\code{m * n * k} samples are drawn.

\end{description}
\begin{description}
\item[{samples}] \leavevmode{[}scalar or ndarray{]}
Drawn samples.

\end{description}

When calculating the power of an experiment (power = probability of
rejecting the null hypothesis when a specific alternative is true) the
non-central F statistic becomes important.  When the null hypothesis is
true, the F statistic follows a central F distribution. When the null
hypothesis is not true, then it follows a non-central F statistic.

Weisstein, Eric W. ``Noncentral F-Distribution.'' From MathWorld--A Wolfram
Web Resource.  \href{http://mathworld.wolfram.com/NoncentralF-Distribution.html}{http://mathworld.wolfram.com/NoncentralF-Distribution.html}

Wikipedia, ``Noncentral F distribution'',
\href{http://en.wikipedia.org/wiki/Noncentral\_F-distribution}{http://en.wikipedia.org/wiki/Noncentral\_F-distribution}

In a study, testing for a specific alternative to the null hypothesis
requires use of the Noncentral F distribution. We need to calculate the
area in the tail of the distribution that exceeds the value of the F
distribution for the null hypothesis.  We'll plot the two probability
distributions for comparison.

\begin{Verbatim}[commandchars=\\\{\}]
\PYG{g+gp}{\PYGZgt{}\PYGZgt{}\PYGZgt{} }\PYG{n}{dfnum} \PYG{o}{=} \PYG{l+m+mi}{3} \PYG{c}{\PYGZsh{} between group deg of freedom}
\PYG{g+gp}{\PYGZgt{}\PYGZgt{}\PYGZgt{} }\PYG{n}{dfden} \PYG{o}{=} \PYG{l+m+mi}{20} \PYG{c}{\PYGZsh{} within groups degrees of freedom}
\PYG{g+gp}{\PYGZgt{}\PYGZgt{}\PYGZgt{} }\PYG{n}{nonc} \PYG{o}{=} \PYG{l+m+mf}{3.0}
\PYG{g+gp}{\PYGZgt{}\PYGZgt{}\PYGZgt{} }\PYG{n}{nc\PYGZus{}vals} \PYG{o}{=} \PYG{n}{np}\PYG{o}{.}\PYG{n}{random}\PYG{o}{.}\PYG{n}{noncentral\PYGZus{}f}\PYG{p}{(}\PYG{n}{dfnum}\PYG{p}{,} \PYG{n}{dfden}\PYG{p}{,} \PYG{n}{nonc}\PYG{p}{,} \PYG{l+m+mi}{1000000}\PYG{p}{)}
\PYG{g+gp}{\PYGZgt{}\PYGZgt{}\PYGZgt{} }\PYG{n}{NF} \PYG{o}{=} \PYG{n}{np}\PYG{o}{.}\PYG{n}{histogram}\PYG{p}{(}\PYG{n}{nc\PYGZus{}vals}\PYG{p}{,} \PYG{n}{bins}\PYG{o}{=}\PYG{l+m+mi}{50}\PYG{p}{,} \PYG{n}{normed}\PYG{o}{=}\PYG{n+nb+bp}{True}\PYG{p}{)}
\PYG{g+gp}{\PYGZgt{}\PYGZgt{}\PYGZgt{} }\PYG{n}{c\PYGZus{}vals} \PYG{o}{=} \PYG{n}{np}\PYG{o}{.}\PYG{n}{random}\PYG{o}{.}\PYG{n}{f}\PYG{p}{(}\PYG{n}{dfnum}\PYG{p}{,} \PYG{n}{dfden}\PYG{p}{,} \PYG{l+m+mi}{1000000}\PYG{p}{)}
\PYG{g+gp}{\PYGZgt{}\PYGZgt{}\PYGZgt{} }\PYG{n}{F} \PYG{o}{=} \PYG{n}{np}\PYG{o}{.}\PYG{n}{histogram}\PYG{p}{(}\PYG{n}{c\PYGZus{}vals}\PYG{p}{,} \PYG{n}{bins}\PYG{o}{=}\PYG{l+m+mi}{50}\PYG{p}{,} \PYG{n}{normed}\PYG{o}{=}\PYG{n+nb+bp}{True}\PYG{p}{)}
\PYG{g+gp}{\PYGZgt{}\PYGZgt{}\PYGZgt{} }\PYG{n}{plt}\PYG{o}{.}\PYG{n}{plot}\PYG{p}{(}\PYG{n}{F}\PYG{p}{[}\PYG{l+m+mi}{1}\PYG{p}{]}\PYG{p}{[}\PYG{l+m+mi}{1}\PYG{p}{:}\PYG{p}{]}\PYG{p}{,} \PYG{n}{F}\PYG{p}{[}\PYG{l+m+mi}{0}\PYG{p}{]}\PYG{p}{)}
\PYG{g+gp}{\PYGZgt{}\PYGZgt{}\PYGZgt{} }\PYG{n}{plt}\PYG{o}{.}\PYG{n}{plot}\PYG{p}{(}\PYG{n}{NF}\PYG{p}{[}\PYG{l+m+mi}{1}\PYG{p}{]}\PYG{p}{[}\PYG{l+m+mi}{1}\PYG{p}{:}\PYG{p}{]}\PYG{p}{,} \PYG{n}{NF}\PYG{p}{[}\PYG{l+m+mi}{0}\PYG{p}{]}\PYG{p}{)}
\PYG{g+gp}{\PYGZgt{}\PYGZgt{}\PYGZgt{} }\PYG{n}{plt}\PYG{o}{.}\PYG{n}{show}\PYG{p}{(}\PYG{p}{)}
\end{Verbatim}

\end{fulllineitems}

\index{normal() (in module lib.graph.raf)}

\begin{fulllineitems}
\phantomsection\label{lib.graph:lib.graph.raf.normal}\pysiglinewithargsret{\code{lib.graph.raf.}\bfcode{normal}}{\emph{loc=0.0}, \emph{scale=1.0}, \emph{size=None}}{}
Draw random samples from a normal (Gaussian) distribution.

The probability density function of the normal distribution, first
derived by De Moivre and 200 years later by both Gauss and Laplace
independently {\color{red}\bfseries{}{[}2{]}\_}, is often called the bell curve because of
its characteristic shape (see the example below).

The normal distributions occurs often in nature.  For example, it
describes the commonly occurring distribution of samples influenced
by a large number of tiny, random disturbances, each with its own
unique distribution {\color{red}\bfseries{}{[}2{]}\_}.
\begin{description}
\item[{loc}] \leavevmode{[}float{]}
Mean (``centre'') of the distribution.

\item[{scale}] \leavevmode{[}float{]}
Standard deviation (spread or ``width'') of the distribution.

\item[{size}] \leavevmode{[}tuple of ints{]}
Output shape.  If the given shape is, e.g., \code{(m, n, k)}, then
\code{m * n * k} samples are drawn.

\end{description}
\begin{description}
\item[{scipy.stats.distributions.norm}] \leavevmode{[}probability density function,{]}
distribution or cumulative density function, etc.

\end{description}

The probability density for the Gaussian distribution is
\begin{gather}
\begin{split}p(x) = \frac{1}{\sqrt{ 2 \pi \sigma^2 }}
e^{ - \frac{ (x - \mu)^2 } {2 \sigma^2} },\end{split}\notag
\end{gather}
where \(\mu\) is the mean and \(\sigma\) the standard deviation.
The square of the standard deviation, \(\sigma^2\), is called the
variance.

The function has its peak at the mean, and its ``spread'' increases with
the standard deviation (the function reaches 0.607 times its maximum at
\(x + \sigma\) and \(x - \sigma\) {\color{red}\bfseries{}{[}2{]}\_}).  This implies that
\emph{numpy.random.normal} is more likely to return samples lying close to the
mean, rather than those far away.

Draw samples from the distribution:

\begin{Verbatim}[commandchars=\\\{\}]
\PYG{g+gp}{\PYGZgt{}\PYGZgt{}\PYGZgt{} }\PYG{n}{mu}\PYG{p}{,} \PYG{n}{sigma} \PYG{o}{=} \PYG{l+m+mi}{0}\PYG{p}{,} \PYG{l+m+mf}{0.1} \PYG{c}{\PYGZsh{} mean and standard deviation}
\PYG{g+gp}{\PYGZgt{}\PYGZgt{}\PYGZgt{} }\PYG{n}{s} \PYG{o}{=} \PYG{n}{np}\PYG{o}{.}\PYG{n}{random}\PYG{o}{.}\PYG{n}{normal}\PYG{p}{(}\PYG{n}{mu}\PYG{p}{,} \PYG{n}{sigma}\PYG{p}{,} \PYG{l+m+mi}{1000}\PYG{p}{)}
\end{Verbatim}

Verify the mean and the variance:

\begin{Verbatim}[commandchars=\\\{\}]
\PYG{g+gp}{\PYGZgt{}\PYGZgt{}\PYGZgt{} }\PYG{n+nb}{abs}\PYG{p}{(}\PYG{n}{mu} \PYG{o}{\PYGZhy{}} \PYG{n}{np}\PYG{o}{.}\PYG{n}{mean}\PYG{p}{(}\PYG{n}{s}\PYG{p}{)}\PYG{p}{)} \PYG{o}{\PYGZlt{}} \PYG{l+m+mf}{0.01}
\PYG{g+go}{True}
\end{Verbatim}

\begin{Verbatim}[commandchars=\\\{\}]
\PYG{g+gp}{\PYGZgt{}\PYGZgt{}\PYGZgt{} }\PYG{n+nb}{abs}\PYG{p}{(}\PYG{n}{sigma} \PYG{o}{\PYGZhy{}} \PYG{n}{np}\PYG{o}{.}\PYG{n}{std}\PYG{p}{(}\PYG{n}{s}\PYG{p}{,} \PYG{n}{ddof}\PYG{o}{=}\PYG{l+m+mi}{1}\PYG{p}{)}\PYG{p}{)} \PYG{o}{\PYGZlt{}} \PYG{l+m+mf}{0.01}
\PYG{g+go}{True}
\end{Verbatim}

Display the histogram of the samples, along with
the probability density function:

\begin{Verbatim}[commandchars=\\\{\}]
\PYG{g+gp}{\PYGZgt{}\PYGZgt{}\PYGZgt{} }\PYG{k+kn}{import} \PYG{n+nn}{matplotlib.pyplot} \PYG{k+kn}{as} \PYG{n+nn}{plt}
\PYG{g+gp}{\PYGZgt{}\PYGZgt{}\PYGZgt{} }\PYG{n}{count}\PYG{p}{,} \PYG{n}{bins}\PYG{p}{,} \PYG{n}{ignored} \PYG{o}{=} \PYG{n}{plt}\PYG{o}{.}\PYG{n}{hist}\PYG{p}{(}\PYG{n}{s}\PYG{p}{,} \PYG{l+m+mi}{30}\PYG{p}{,} \PYG{n}{normed}\PYG{o}{=}\PYG{n+nb+bp}{True}\PYG{p}{)}
\PYG{g+gp}{\PYGZgt{}\PYGZgt{}\PYGZgt{} }\PYG{n}{plt}\PYG{o}{.}\PYG{n}{plot}\PYG{p}{(}\PYG{n}{bins}\PYG{p}{,} \PYG{l+m+mi}{1}\PYG{o}{/}\PYG{p}{(}\PYG{n}{sigma} \PYG{o}{*} \PYG{n}{np}\PYG{o}{.}\PYG{n}{sqrt}\PYG{p}{(}\PYG{l+m+mi}{2} \PYG{o}{*} \PYG{n}{np}\PYG{o}{.}\PYG{n}{pi}\PYG{p}{)}\PYG{p}{)} \PYG{o}{*}
\PYG{g+gp}{... }               \PYG{n}{np}\PYG{o}{.}\PYG{n}{exp}\PYG{p}{(} \PYG{o}{\PYGZhy{}} \PYG{p}{(}\PYG{n}{bins} \PYG{o}{\PYGZhy{}} \PYG{n}{mu}\PYG{p}{)}\PYG{o}{*}\PYG{o}{*}\PYG{l+m+mi}{2} \PYG{o}{/} \PYG{p}{(}\PYG{l+m+mi}{2} \PYG{o}{*} \PYG{n}{sigma}\PYG{o}{*}\PYG{o}{*}\PYG{l+m+mi}{2}\PYG{p}{)} \PYG{p}{)}\PYG{p}{,}
\PYG{g+gp}{... }         \PYG{n}{linewidth}\PYG{o}{=}\PYG{l+m+mi}{2}\PYG{p}{,} \PYG{n}{color}\PYG{o}{=}\PYG{l+s}{\PYGZsq{}}\PYG{l+s}{r}\PYG{l+s}{\PYGZsq{}}\PYG{p}{)}
\PYG{g+gp}{\PYGZgt{}\PYGZgt{}\PYGZgt{} }\PYG{n}{plt}\PYG{o}{.}\PYG{n}{show}\PYG{p}{(}\PYG{p}{)}
\end{Verbatim}

\end{fulllineitems}

\index{pareto() (in module lib.graph.raf)}

\begin{fulllineitems}
\phantomsection\label{lib.graph:lib.graph.raf.pareto}\pysiglinewithargsret{\code{lib.graph.raf.}\bfcode{pareto}}{\emph{a}, \emph{size=None}}{}
Draw samples from a Pareto II or Lomax distribution with specified shape.

The Lomax or Pareto II distribution is a shifted Pareto distribution. The
classical Pareto distribution can be obtained from the Lomax distribution
by adding the location parameter m, see below. The smallest value of the
Lomax distribution is zero while for the classical Pareto distribution it
is m, where the standard Pareto distribution has location m=1.
Lomax can also be considered as a simplified version of the Generalized
Pareto distribution (available in SciPy), with the scale set to one and
the location set to zero.

The Pareto distribution must be greater than zero, and is unbounded above.
It is also known as the ``80-20 rule''.  In this distribution, 80 percent of
the weights are in the lowest 20 percent of the range, while the other 20
percent fill the remaining 80 percent of the range.
\begin{description}
\item[{shape}] \leavevmode{[}float, \textgreater{} 0.{]}
Shape of the distribution.

\item[{size}] \leavevmode{[}tuple of ints{]}
Output shape.  If the given shape is, e.g., \code{(m, n, k)}, then
\code{m * n * k} samples are drawn.

\end{description}
\begin{description}
\item[{scipy.stats.distributions.lomax.pdf}] \leavevmode{[}probability density function,{]}
distribution or cumulative density function, etc.

\item[{scipy.stats.distributions.genpareto.pdf}] \leavevmode{[}probability density function,{]}
distribution or cumulative density function, etc.

\end{description}

The probability density for the Pareto distribution is
\begin{gather}
\begin{split}p(x) = \frac{am^a}{x^{a+1}}\end{split}\notag
\end{gather}
where \(a\) is the shape and \(m\) the location

The Pareto distribution, named after the Italian economist Vilfredo Pareto,
is a power law probability distribution useful in many real world problems.
Outside the field of economics it is generally referred to as the Bradford
distribution. Pareto developed the distribution to describe the
distribution of wealth in an economy.  It has also found use in insurance,
web page access statistics, oil field sizes, and many other problems,
including the download frequency for projects in Sourceforge {[}1{]}.  It is
one of the so-called ``fat-tailed'' distributions.

Draw samples from the distribution:

\begin{Verbatim}[commandchars=\\\{\}]
\PYG{g+gp}{\PYGZgt{}\PYGZgt{}\PYGZgt{} }\PYG{n}{a}\PYG{p}{,} \PYG{n}{m} \PYG{o}{=} \PYG{l+m+mf}{3.}\PYG{p}{,} \PYG{l+m+mf}{1.} \PYG{c}{\PYGZsh{} shape and mode}
\PYG{g+gp}{\PYGZgt{}\PYGZgt{}\PYGZgt{} }\PYG{n}{s} \PYG{o}{=} \PYG{n}{np}\PYG{o}{.}\PYG{n}{random}\PYG{o}{.}\PYG{n}{pareto}\PYG{p}{(}\PYG{n}{a}\PYG{p}{,} \PYG{l+m+mi}{1000}\PYG{p}{)} \PYG{o}{+} \PYG{n}{m}
\end{Verbatim}

Display the histogram of the samples, along with
the probability density function:

\begin{Verbatim}[commandchars=\\\{\}]
\PYG{g+gp}{\PYGZgt{}\PYGZgt{}\PYGZgt{} }\PYG{k+kn}{import} \PYG{n+nn}{matplotlib.pyplot} \PYG{k+kn}{as} \PYG{n+nn}{plt}
\PYG{g+gp}{\PYGZgt{}\PYGZgt{}\PYGZgt{} }\PYG{n}{count}\PYG{p}{,} \PYG{n}{bins}\PYG{p}{,} \PYG{n}{ignored} \PYG{o}{=} \PYG{n}{plt}\PYG{o}{.}\PYG{n}{hist}\PYG{p}{(}\PYG{n}{s}\PYG{p}{,} \PYG{l+m+mi}{100}\PYG{p}{,} \PYG{n}{normed}\PYG{o}{=}\PYG{n+nb+bp}{True}\PYG{p}{,} \PYG{n}{align}\PYG{o}{=}\PYG{l+s}{\PYGZsq{}}\PYG{l+s}{center}\PYG{l+s}{\PYGZsq{}}\PYG{p}{)}
\PYG{g+gp}{\PYGZgt{}\PYGZgt{}\PYGZgt{} }\PYG{n}{fit} \PYG{o}{=} \PYG{n}{a}\PYG{o}{*}\PYG{n}{m}\PYG{o}{*}\PYG{o}{*}\PYG{n}{a}\PYG{o}{/}\PYG{n}{bins}\PYG{o}{*}\PYG{o}{*}\PYG{p}{(}\PYG{n}{a}\PYG{o}{+}\PYG{l+m+mi}{1}\PYG{p}{)}
\PYG{g+gp}{\PYGZgt{}\PYGZgt{}\PYGZgt{} }\PYG{n}{plt}\PYG{o}{.}\PYG{n}{plot}\PYG{p}{(}\PYG{n}{bins}\PYG{p}{,} \PYG{n+nb}{max}\PYG{p}{(}\PYG{n}{count}\PYG{p}{)}\PYG{o}{*}\PYG{n}{fit}\PYG{o}{/}\PYG{n+nb}{max}\PYG{p}{(}\PYG{n}{fit}\PYG{p}{)}\PYG{p}{,}\PYG{n}{linewidth}\PYG{o}{=}\PYG{l+m+mi}{2}\PYG{p}{,} \PYG{n}{color}\PYG{o}{=}\PYG{l+s}{\PYGZsq{}}\PYG{l+s}{r}\PYG{l+s}{\PYGZsq{}}\PYG{p}{)}
\PYG{g+gp}{\PYGZgt{}\PYGZgt{}\PYGZgt{} }\PYG{n}{plt}\PYG{o}{.}\PYG{n}{show}\PYG{p}{(}\PYG{p}{)}
\end{Verbatim}

\end{fulllineitems}

\index{permutation() (in module lib.graph.raf)}

\begin{fulllineitems}
\phantomsection\label{lib.graph:lib.graph.raf.permutation}\pysiglinewithargsret{\code{lib.graph.raf.}\bfcode{permutation}}{\emph{x}}{}
Randomly permute a sequence, or return a permuted range.

If \emph{x} is a multi-dimensional array, it is only shuffled along its
first index.
\begin{description}
\item[{x}] \leavevmode{[}int or array\_like{]}
If \emph{x} is an integer, randomly permute \code{np.arange(x)}.
If \emph{x} is an array, make a copy and shuffle the elements
randomly.

\end{description}
\begin{description}
\item[{out}] \leavevmode{[}ndarray{]}
Permuted sequence or array range.

\end{description}

\begin{Verbatim}[commandchars=\\\{\}]
\PYG{g+gp}{\PYGZgt{}\PYGZgt{}\PYGZgt{} }\PYG{n}{np}\PYG{o}{.}\PYG{n}{random}\PYG{o}{.}\PYG{n}{permutation}\PYG{p}{(}\PYG{l+m+mi}{10}\PYG{p}{)}
\PYG{g+go}{array([1, 7, 4, 3, 0, 9, 2, 5, 8, 6])}
\end{Verbatim}

\begin{Verbatim}[commandchars=\\\{\}]
\PYG{g+gp}{\PYGZgt{}\PYGZgt{}\PYGZgt{} }\PYG{n}{np}\PYG{o}{.}\PYG{n}{random}\PYG{o}{.}\PYG{n}{permutation}\PYG{p}{(}\PYG{p}{[}\PYG{l+m+mi}{1}\PYG{p}{,} \PYG{l+m+mi}{4}\PYG{p}{,} \PYG{l+m+mi}{9}\PYG{p}{,} \PYG{l+m+mi}{12}\PYG{p}{,} \PYG{l+m+mi}{15}\PYG{p}{]}\PYG{p}{)}
\PYG{g+go}{array([15,  1,  9,  4, 12])}
\end{Verbatim}

\begin{Verbatim}[commandchars=\\\{\}]
\PYG{g+gp}{\PYGZgt{}\PYGZgt{}\PYGZgt{} }\PYG{n}{arr} \PYG{o}{=} \PYG{n}{np}\PYG{o}{.}\PYG{n}{arange}\PYG{p}{(}\PYG{l+m+mi}{9}\PYG{p}{)}\PYG{o}{.}\PYG{n}{reshape}\PYG{p}{(}\PYG{p}{(}\PYG{l+m+mi}{3}\PYG{p}{,} \PYG{l+m+mi}{3}\PYG{p}{)}\PYG{p}{)}
\PYG{g+gp}{\PYGZgt{}\PYGZgt{}\PYGZgt{} }\PYG{n}{np}\PYG{o}{.}\PYG{n}{random}\PYG{o}{.}\PYG{n}{permutation}\PYG{p}{(}\PYG{n}{arr}\PYG{p}{)}
\PYG{g+go}{array([[6, 7, 8],}
\PYG{g+go}{       [0, 1, 2],}
\PYG{g+go}{       [3, 4, 5]])}
\end{Verbatim}

\end{fulllineitems}

\index{poisson() (in module lib.graph.raf)}

\begin{fulllineitems}
\phantomsection\label{lib.graph:lib.graph.raf.poisson}\pysiglinewithargsret{\code{lib.graph.raf.}\bfcode{poisson}}{\emph{lam=1.0}, \emph{size=None}}{}
Draw samples from a Poisson distribution.

The Poisson distribution is the limit of the Binomial
distribution for large N.
\begin{description}
\item[{lam}] \leavevmode{[}float{]}
Expectation of interval, should be \textgreater{}= 0.

\item[{size}] \leavevmode{[}int or tuple of ints, optional{]}
Output shape. If the given shape is, e.g., \code{(m, n, k)}, then
\code{m * n * k} samples are drawn.

\end{description}

The Poisson distribution
\begin{gather}
\begin{split}f(k; \lambda)=\frac{\lambda^k e^{-\lambda}}{k!}\end{split}\notag
\end{gather}
For events with an expected separation \(\lambda\) the Poisson
distribution \(f(k; \lambda)\) describes the probability of
\(k\) events occurring within the observed interval \(\lambda\).

Because the output is limited to the range of the C long type, a
ValueError is raised when \emph{lam} is within 10 sigma of the maximum
representable value.

Draw samples from the distribution:

\begin{Verbatim}[commandchars=\\\{\}]
\PYG{g+gp}{\PYGZgt{}\PYGZgt{}\PYGZgt{} }\PYG{k+kn}{import} \PYG{n+nn}{numpy} \PYG{k+kn}{as} \PYG{n+nn}{np}
\PYG{g+gp}{\PYGZgt{}\PYGZgt{}\PYGZgt{} }\PYG{n}{s} \PYG{o}{=} \PYG{n}{np}\PYG{o}{.}\PYG{n}{random}\PYG{o}{.}\PYG{n}{poisson}\PYG{p}{(}\PYG{l+m+mi}{5}\PYG{p}{,} \PYG{l+m+mi}{10000}\PYG{p}{)}
\end{Verbatim}

Display histogram of the sample:

\begin{Verbatim}[commandchars=\\\{\}]
\PYG{g+gp}{\PYGZgt{}\PYGZgt{}\PYGZgt{} }\PYG{k+kn}{import} \PYG{n+nn}{matplotlib.pyplot} \PYG{k+kn}{as} \PYG{n+nn}{plt}
\PYG{g+gp}{\PYGZgt{}\PYGZgt{}\PYGZgt{} }\PYG{n}{count}\PYG{p}{,} \PYG{n}{bins}\PYG{p}{,} \PYG{n}{ignored} \PYG{o}{=} \PYG{n}{plt}\PYG{o}{.}\PYG{n}{hist}\PYG{p}{(}\PYG{n}{s}\PYG{p}{,} \PYG{l+m+mi}{14}\PYG{p}{,} \PYG{n}{normed}\PYG{o}{=}\PYG{n+nb+bp}{True}\PYG{p}{)}
\PYG{g+gp}{\PYGZgt{}\PYGZgt{}\PYGZgt{} }\PYG{n}{plt}\PYG{o}{.}\PYG{n}{show}\PYG{p}{(}\PYG{p}{)}
\end{Verbatim}

\end{fulllineitems}

\index{power() (in module lib.graph.raf)}

\begin{fulllineitems}
\phantomsection\label{lib.graph:lib.graph.raf.power}\pysiglinewithargsret{\code{lib.graph.raf.}\bfcode{power}}{\emph{a}, \emph{size=None}}{}
Draws samples in {[}0, 1{]} from a power distribution with positive
exponent a - 1.

Also known as the power function distribution.
\begin{description}
\item[{a}] \leavevmode{[}float{]}
parameter, \textgreater{} 0

\item[{size}] \leavevmode{[}tuple of ints{]}\begin{description}
\item[{Output shape.  If the given shape is, e.g., \code{(m, n, k)}, then}] \leavevmode
\code{m * n * k} samples are drawn.

\end{description}

\end{description}
\begin{description}
\item[{samples}] \leavevmode{[}\{ndarray, scalar\}{]}
The returned samples lie in {[}0, 1{]}.

\end{description}
\begin{description}
\item[{ValueError}] \leavevmode
If a\textless{}1.

\end{description}

The probability density function is
\begin{gather}
\begin{split}P(x; a) = ax^{a-1}, 0 \le x \le 1, a>0.\end{split}\notag
\end{gather}
The power function distribution is just the inverse of the Pareto
distribution. It may also be seen as a special case of the Beta
distribution.

It is used, for example, in modeling the over-reporting of insurance
claims.

Draw samples from the distribution:

\begin{Verbatim}[commandchars=\\\{\}]
\PYG{g+gp}{\PYGZgt{}\PYGZgt{}\PYGZgt{} }\PYG{n}{a} \PYG{o}{=} \PYG{l+m+mf}{5.} \PYG{c}{\PYGZsh{} shape}
\PYG{g+gp}{\PYGZgt{}\PYGZgt{}\PYGZgt{} }\PYG{n}{samples} \PYG{o}{=} \PYG{l+m+mi}{1000}
\PYG{g+gp}{\PYGZgt{}\PYGZgt{}\PYGZgt{} }\PYG{n}{s} \PYG{o}{=} \PYG{n}{np}\PYG{o}{.}\PYG{n}{random}\PYG{o}{.}\PYG{n}{power}\PYG{p}{(}\PYG{n}{a}\PYG{p}{,} \PYG{n}{samples}\PYG{p}{)}
\end{Verbatim}

Display the histogram of the samples, along with
the probability density function:

\begin{Verbatim}[commandchars=\\\{\}]
\PYG{g+gp}{\PYGZgt{}\PYGZgt{}\PYGZgt{} }\PYG{k+kn}{import} \PYG{n+nn}{matplotlib.pyplot} \PYG{k+kn}{as} \PYG{n+nn}{plt}
\PYG{g+gp}{\PYGZgt{}\PYGZgt{}\PYGZgt{} }\PYG{n}{count}\PYG{p}{,} \PYG{n}{bins}\PYG{p}{,} \PYG{n}{ignored} \PYG{o}{=} \PYG{n}{plt}\PYG{o}{.}\PYG{n}{hist}\PYG{p}{(}\PYG{n}{s}\PYG{p}{,} \PYG{n}{bins}\PYG{o}{=}\PYG{l+m+mi}{30}\PYG{p}{)}
\PYG{g+gp}{\PYGZgt{}\PYGZgt{}\PYGZgt{} }\PYG{n}{x} \PYG{o}{=} \PYG{n}{np}\PYG{o}{.}\PYG{n}{linspace}\PYG{p}{(}\PYG{l+m+mi}{0}\PYG{p}{,} \PYG{l+m+mi}{1}\PYG{p}{,} \PYG{l+m+mi}{100}\PYG{p}{)}
\PYG{g+gp}{\PYGZgt{}\PYGZgt{}\PYGZgt{} }\PYG{n}{y} \PYG{o}{=} \PYG{n}{a}\PYG{o}{*}\PYG{n}{x}\PYG{o}{*}\PYG{o}{*}\PYG{p}{(}\PYG{n}{a}\PYG{o}{\PYGZhy{}}\PYG{l+m+mf}{1.}\PYG{p}{)}
\PYG{g+gp}{\PYGZgt{}\PYGZgt{}\PYGZgt{} }\PYG{n}{normed\PYGZus{}y} \PYG{o}{=} \PYG{n}{samples}\PYG{o}{*}\PYG{n}{np}\PYG{o}{.}\PYG{n}{diff}\PYG{p}{(}\PYG{n}{bins}\PYG{p}{)}\PYG{p}{[}\PYG{l+m+mi}{0}\PYG{p}{]}\PYG{o}{*}\PYG{n}{y}
\PYG{g+gp}{\PYGZgt{}\PYGZgt{}\PYGZgt{} }\PYG{n}{plt}\PYG{o}{.}\PYG{n}{plot}\PYG{p}{(}\PYG{n}{x}\PYG{p}{,} \PYG{n}{normed\PYGZus{}y}\PYG{p}{)}
\PYG{g+gp}{\PYGZgt{}\PYGZgt{}\PYGZgt{} }\PYG{n}{plt}\PYG{o}{.}\PYG{n}{show}\PYG{p}{(}\PYG{p}{)}
\end{Verbatim}

Compare the power function distribution to the inverse of the Pareto.

\begin{Verbatim}[commandchars=\\\{\}]
\PYG{g+gp}{\PYGZgt{}\PYGZgt{}\PYGZgt{} }\PYG{k+kn}{from} \PYG{n+nn}{scipy} \PYG{k+kn}{import} \PYG{n}{stats}
\PYG{g+gp}{\PYGZgt{}\PYGZgt{}\PYGZgt{} }\PYG{n}{rvs} \PYG{o}{=} \PYG{n}{np}\PYG{o}{.}\PYG{n}{random}\PYG{o}{.}\PYG{n}{power}\PYG{p}{(}\PYG{l+m+mi}{5}\PYG{p}{,} \PYG{l+m+mi}{1000000}\PYG{p}{)}
\PYG{g+gp}{\PYGZgt{}\PYGZgt{}\PYGZgt{} }\PYG{n}{rvsp} \PYG{o}{=} \PYG{n}{np}\PYG{o}{.}\PYG{n}{random}\PYG{o}{.}\PYG{n}{pareto}\PYG{p}{(}\PYG{l+m+mi}{5}\PYG{p}{,} \PYG{l+m+mi}{1000000}\PYG{p}{)}
\PYG{g+gp}{\PYGZgt{}\PYGZgt{}\PYGZgt{} }\PYG{n}{xx} \PYG{o}{=} \PYG{n}{np}\PYG{o}{.}\PYG{n}{linspace}\PYG{p}{(}\PYG{l+m+mi}{0}\PYG{p}{,}\PYG{l+m+mi}{1}\PYG{p}{,}\PYG{l+m+mi}{100}\PYG{p}{)}
\PYG{g+gp}{\PYGZgt{}\PYGZgt{}\PYGZgt{} }\PYG{n}{powpdf} \PYG{o}{=} \PYG{n}{stats}\PYG{o}{.}\PYG{n}{powerlaw}\PYG{o}{.}\PYG{n}{pdf}\PYG{p}{(}\PYG{n}{xx}\PYG{p}{,}\PYG{l+m+mi}{5}\PYG{p}{)}
\end{Verbatim}

\begin{Verbatim}[commandchars=\\\{\}]
\PYG{g+gp}{\PYGZgt{}\PYGZgt{}\PYGZgt{} }\PYG{n}{plt}\PYG{o}{.}\PYG{n}{figure}\PYG{p}{(}\PYG{p}{)}
\PYG{g+gp}{\PYGZgt{}\PYGZgt{}\PYGZgt{} }\PYG{n}{plt}\PYG{o}{.}\PYG{n}{hist}\PYG{p}{(}\PYG{n}{rvs}\PYG{p}{,} \PYG{n}{bins}\PYG{o}{=}\PYG{l+m+mi}{50}\PYG{p}{,} \PYG{n}{normed}\PYG{o}{=}\PYG{n+nb+bp}{True}\PYG{p}{)}
\PYG{g+gp}{\PYGZgt{}\PYGZgt{}\PYGZgt{} }\PYG{n}{plt}\PYG{o}{.}\PYG{n}{plot}\PYG{p}{(}\PYG{n}{xx}\PYG{p}{,}\PYG{n}{powpdf}\PYG{p}{,}\PYG{l+s}{\PYGZsq{}}\PYG{l+s}{r\PYGZhy{}}\PYG{l+s}{\PYGZsq{}}\PYG{p}{)}
\PYG{g+gp}{\PYGZgt{}\PYGZgt{}\PYGZgt{} }\PYG{n}{plt}\PYG{o}{.}\PYG{n}{title}\PYG{p}{(}\PYG{l+s}{\PYGZsq{}}\PYG{l+s}{np.random.power(5)}\PYG{l+s}{\PYGZsq{}}\PYG{p}{)}
\end{Verbatim}

\begin{Verbatim}[commandchars=\\\{\}]
\PYG{g+gp}{\PYGZgt{}\PYGZgt{}\PYGZgt{} }\PYG{n}{plt}\PYG{o}{.}\PYG{n}{figure}\PYG{p}{(}\PYG{p}{)}
\PYG{g+gp}{\PYGZgt{}\PYGZgt{}\PYGZgt{} }\PYG{n}{plt}\PYG{o}{.}\PYG{n}{hist}\PYG{p}{(}\PYG{l+m+mf}{1.}\PYG{o}{/}\PYG{p}{(}\PYG{l+m+mf}{1.}\PYG{o}{+}\PYG{n}{rvsp}\PYG{p}{)}\PYG{p}{,} \PYG{n}{bins}\PYG{o}{=}\PYG{l+m+mi}{50}\PYG{p}{,} \PYG{n}{normed}\PYG{o}{=}\PYG{n+nb+bp}{True}\PYG{p}{)}
\PYG{g+gp}{\PYGZgt{}\PYGZgt{}\PYGZgt{} }\PYG{n}{plt}\PYG{o}{.}\PYG{n}{plot}\PYG{p}{(}\PYG{n}{xx}\PYG{p}{,}\PYG{n}{powpdf}\PYG{p}{,}\PYG{l+s}{\PYGZsq{}}\PYG{l+s}{r\PYGZhy{}}\PYG{l+s}{\PYGZsq{}}\PYG{p}{)}
\PYG{g+gp}{\PYGZgt{}\PYGZgt{}\PYGZgt{} }\PYG{n}{plt}\PYG{o}{.}\PYG{n}{title}\PYG{p}{(}\PYG{l+s}{\PYGZsq{}}\PYG{l+s}{inverse of 1 + np.random.pareto(5)}\PYG{l+s}{\PYGZsq{}}\PYG{p}{)}
\end{Verbatim}

\begin{Verbatim}[commandchars=\\\{\}]
\PYG{g+gp}{\PYGZgt{}\PYGZgt{}\PYGZgt{} }\PYG{n}{plt}\PYG{o}{.}\PYG{n}{figure}\PYG{p}{(}\PYG{p}{)}
\PYG{g+gp}{\PYGZgt{}\PYGZgt{}\PYGZgt{} }\PYG{n}{plt}\PYG{o}{.}\PYG{n}{hist}\PYG{p}{(}\PYG{l+m+mf}{1.}\PYG{o}{/}\PYG{p}{(}\PYG{l+m+mf}{1.}\PYG{o}{+}\PYG{n}{rvsp}\PYG{p}{)}\PYG{p}{,} \PYG{n}{bins}\PYG{o}{=}\PYG{l+m+mi}{50}\PYG{p}{,} \PYG{n}{normed}\PYG{o}{=}\PYG{n+nb+bp}{True}\PYG{p}{)}
\PYG{g+gp}{\PYGZgt{}\PYGZgt{}\PYGZgt{} }\PYG{n}{plt}\PYG{o}{.}\PYG{n}{plot}\PYG{p}{(}\PYG{n}{xx}\PYG{p}{,}\PYG{n}{powpdf}\PYG{p}{,}\PYG{l+s}{\PYGZsq{}}\PYG{l+s}{r\PYGZhy{}}\PYG{l+s}{\PYGZsq{}}\PYG{p}{)}
\PYG{g+gp}{\PYGZgt{}\PYGZgt{}\PYGZgt{} }\PYG{n}{plt}\PYG{o}{.}\PYG{n}{title}\PYG{p}{(}\PYG{l+s}{\PYGZsq{}}\PYG{l+s}{inverse of stats.pareto(5)}\PYG{l+s}{\PYGZsq{}}\PYG{p}{)}
\end{Verbatim}

\end{fulllineitems}

\index{rafComputation() (in module lib.graph.raf)}

\begin{fulllineitems}
\phantomsection\label{lib.graph:lib.graph.raf.rafComputation}\pysiglinewithargsret{\code{lib.graph.raf.}\bfcode{rafComputation}}{\emph{fid\_initRafRes}, \emph{fid\_initRafResALL}, \emph{fid\_initRafResLIST}, \emph{tmpDir}, \emph{rctProb}, \emph{avgCon}, \emph{rcts}, \emph{cats}, \emph{foodList}, \emph{maxDim}, \emph{debug=False}}{}
\end{fulllineitems}

\index{rafDynamicComputation() (in module lib.graph.raf)}

\begin{fulllineitems}
\phantomsection\label{lib.graph:lib.graph.raf.rafDynamicComputation}\pysiglinewithargsret{\code{lib.graph.raf.}\bfcode{rafDynamicComputation}}{\emph{fid\_dynRafRes}, \emph{tmpTime}, \emph{rcts}, \emph{cats}, \emph{foodList}, \emph{growth=False}, \emph{rctsALL=None}, \emph{catsALL=None}, \emph{completeRCTS=None}, \emph{debug=False}}{}
\end{fulllineitems}

\index{rafsearch() (in module lib.graph.raf)}

\begin{fulllineitems}
\phantomsection\label{lib.graph:lib.graph.raf.rafsearch}\pysiglinewithargsret{\code{lib.graph.raf.}\bfcode{rafsearch}}{\emph{rcts}, \emph{cats}, \emph{closure}, \emph{debug=False}}{}
\end{fulllineitems}

\index{rand() (in module lib.graph.raf)}

\begin{fulllineitems}
\phantomsection\label{lib.graph:lib.graph.raf.rand}\pysiglinewithargsret{\code{lib.graph.raf.}\bfcode{rand}}{\emph{d0}, \emph{d1}, \emph{...}, \emph{dn}}{}
Random values in a given shape.

Create an array of the given shape and propagate it with
random samples from a uniform distribution
over \code{{[}0, 1)}.
\begin{description}
\item[{d0, d1, ..., dn}] \leavevmode{[}int, optional{]}
The dimensions of the returned array, should all be positive.
If no argument is given a single Python float is returned.

\end{description}
\begin{description}
\item[{out}] \leavevmode{[}ndarray, shape \code{(d0, d1, ..., dn)}{]}
Random values.

\end{description}

random

This is a convenience function. If you want an interface that
takes a shape-tuple as the first argument, refer to
np.random.random\_sample .

\begin{Verbatim}[commandchars=\\\{\}]
\PYG{g+gp}{\PYGZgt{}\PYGZgt{}\PYGZgt{} }\PYG{n}{np}\PYG{o}{.}\PYG{n}{random}\PYG{o}{.}\PYG{n}{rand}\PYG{p}{(}\PYG{l+m+mi}{3}\PYG{p}{,}\PYG{l+m+mi}{2}\PYG{p}{)}
\PYG{g+go}{array([[ 0.14022471,  0.96360618],  \PYGZsh{}random}
\PYG{g+go}{       [ 0.37601032,  0.25528411],  \PYGZsh{}random}
\PYG{g+go}{       [ 0.49313049,  0.94909878]]) \PYGZsh{}random}
\end{Verbatim}

\end{fulllineitems}

\index{randint() (in module lib.graph.raf)}

\begin{fulllineitems}
\phantomsection\label{lib.graph:lib.graph.raf.randint}\pysiglinewithargsret{\code{lib.graph.raf.}\bfcode{randint}}{\emph{low}, \emph{high=None}, \emph{size=None}}{}
Return random integers from \emph{low} (inclusive) to \emph{high} (exclusive).

Return random integers from the ``discrete uniform'' distribution in the
``half-open'' interval {[}\emph{low}, \emph{high}). If \emph{high} is None (the default),
then results are from {[}0, \emph{low}).
\begin{description}
\item[{low}] \leavevmode{[}int{]}
Lowest (signed) integer to be drawn from the distribution (unless
\code{high=None}, in which case this parameter is the \emph{highest} such
integer).

\item[{high}] \leavevmode{[}int, optional{]}
If provided, one above the largest (signed) integer to be drawn
from the distribution (see above for behavior if \code{high=None}).

\item[{size}] \leavevmode{[}int or tuple of ints, optional{]}
Output shape. Default is None, in which case a single int is
returned.

\end{description}
\begin{description}
\item[{out}] \leavevmode{[}int or ndarray of ints{]}
\emph{size}-shaped array of random integers from the appropriate
distribution, or a single such random int if \emph{size} not provided.

\end{description}
\begin{description}
\item[{random.random\_integers}] \leavevmode{[}similar to \emph{randint}, only for the closed{]}
interval {[}\emph{low}, \emph{high}{]}, and 1 is the lowest value if \emph{high} is
omitted. In particular, this other one is the one to use to generate
uniformly distributed discrete non-integers.

\end{description}

\begin{Verbatim}[commandchars=\\\{\}]
\PYG{g+gp}{\PYGZgt{}\PYGZgt{}\PYGZgt{} }\PYG{n}{np}\PYG{o}{.}\PYG{n}{random}\PYG{o}{.}\PYG{n}{randint}\PYG{p}{(}\PYG{l+m+mi}{2}\PYG{p}{,} \PYG{n}{size}\PYG{o}{=}\PYG{l+m+mi}{10}\PYG{p}{)}
\PYG{g+go}{array([1, 0, 0, 0, 1, 1, 0, 0, 1, 0])}
\PYG{g+gp}{\PYGZgt{}\PYGZgt{}\PYGZgt{} }\PYG{n}{np}\PYG{o}{.}\PYG{n}{random}\PYG{o}{.}\PYG{n}{randint}\PYG{p}{(}\PYG{l+m+mi}{1}\PYG{p}{,} \PYG{n}{size}\PYG{o}{=}\PYG{l+m+mi}{10}\PYG{p}{)}
\PYG{g+go}{array([0, 0, 0, 0, 0, 0, 0, 0, 0, 0])}
\end{Verbatim}

Generate a 2 x 4 array of ints between 0 and 4, inclusive:

\begin{Verbatim}[commandchars=\\\{\}]
\PYG{g+gp}{\PYGZgt{}\PYGZgt{}\PYGZgt{} }\PYG{n}{np}\PYG{o}{.}\PYG{n}{random}\PYG{o}{.}\PYG{n}{randint}\PYG{p}{(}\PYG{l+m+mi}{5}\PYG{p}{,} \PYG{n}{size}\PYG{o}{=}\PYG{p}{(}\PYG{l+m+mi}{2}\PYG{p}{,} \PYG{l+m+mi}{4}\PYG{p}{)}\PYG{p}{)}
\PYG{g+go}{array([[4, 0, 2, 1],}
\PYG{g+go}{       [3, 2, 2, 0]])}
\end{Verbatim}

\end{fulllineitems}

\index{randn() (in module lib.graph.raf)}

\begin{fulllineitems}
\phantomsection\label{lib.graph:lib.graph.raf.randn}\pysiglinewithargsret{\code{lib.graph.raf.}\bfcode{randn}}{\emph{d0}, \emph{d1}, \emph{...}, \emph{dn}}{}
Return a sample (or samples) from the ``standard normal'' distribution.

If positive, int\_like or int-convertible arguments are provided,
\emph{randn} generates an array of shape \code{(d0, d1, ..., dn)}, filled
with random floats sampled from a univariate ``normal'' (Gaussian)
distribution of mean 0 and variance 1 (if any of the \(d_i\) are
floats, they are first converted to integers by truncation). A single
float randomly sampled from the distribution is returned if no
argument is provided.

This is a convenience function.  If you want an interface that takes a
tuple as the first argument, use \emph{numpy.random.standard\_normal} instead.
\begin{description}
\item[{d0, d1, ..., dn}] \leavevmode{[}int, optional{]}
The dimensions of the returned array, should be all positive.
If no argument is given a single Python float is returned.

\end{description}
\begin{description}
\item[{Z}] \leavevmode{[}ndarray or float{]}
A \code{(d0, d1, ..., dn)}-shaped array of floating-point samples from
the standard normal distribution, or a single such float if
no parameters were supplied.

\end{description}

random.standard\_normal : Similar, but takes a tuple as its argument.

For random samples from \(N(\mu, \sigma^2)\), use:

\code{sigma * np.random.randn(...) + mu}

\begin{Verbatim}[commandchars=\\\{\}]
\PYG{g+gp}{\PYGZgt{}\PYGZgt{}\PYGZgt{} }\PYG{n}{np}\PYG{o}{.}\PYG{n}{random}\PYG{o}{.}\PYG{n}{randn}\PYG{p}{(}\PYG{p}{)}
\PYG{g+go}{2.1923875335537315 \PYGZsh{}random}
\end{Verbatim}

Two-by-four array of samples from N(3, 6.25):

\begin{Verbatim}[commandchars=\\\{\}]
\PYG{g+gp}{\PYGZgt{}\PYGZgt{}\PYGZgt{} }\PYG{l+m+mf}{2.5} \PYG{o}{*} \PYG{n}{np}\PYG{o}{.}\PYG{n}{random}\PYG{o}{.}\PYG{n}{randn}\PYG{p}{(}\PYG{l+m+mi}{2}\PYG{p}{,} \PYG{l+m+mi}{4}\PYG{p}{)} \PYG{o}{+} \PYG{l+m+mi}{3}
\PYG{g+go}{array([[\PYGZhy{}4.49401501,  4.00950034, \PYGZhy{}1.81814867,  7.29718677],  \PYGZsh{}random}
\PYG{g+go}{       [ 0.39924804,  4.68456316,  4.99394529,  4.84057254]]) \PYGZsh{}random}
\end{Verbatim}

\end{fulllineitems}

\index{random() (in module lib.graph.raf)}

\begin{fulllineitems}
\phantomsection\label{lib.graph:lib.graph.raf.random}\pysiglinewithargsret{\code{lib.graph.raf.}\bfcode{random}}{}{}
random\_sample(size=None)

Return random floats in the half-open interval {[}0.0, 1.0).

Results are from the ``continuous uniform'' distribution over the
stated interval.  To sample \(Unif[a, b), b > a\) multiply
the output of \emph{random\_sample} by \emph{(b-a)} and add \emph{a}:

\begin{Verbatim}[commandchars=\\\{\}]
\PYG{p}{(}\PYG{n}{b} \PYG{o}{\PYGZhy{}} \PYG{n}{a}\PYG{p}{)} \PYG{o}{*} \PYG{n}{random\PYGZus{}sample}\PYG{p}{(}\PYG{p}{)} \PYG{o}{+} \PYG{n}{a}
\end{Verbatim}
\begin{description}
\item[{size}] \leavevmode{[}int or tuple of ints, optional{]}
Defines the shape of the returned array of random floats. If None
(the default), returns a single float.

\end{description}
\begin{description}
\item[{out}] \leavevmode{[}float or ndarray of floats{]}
Array of random floats of shape \emph{size} (unless \code{size=None}, in which
case a single float is returned).

\end{description}

\begin{Verbatim}[commandchars=\\\{\}]
\PYG{g+gp}{\PYGZgt{}\PYGZgt{}\PYGZgt{} }\PYG{n}{np}\PYG{o}{.}\PYG{n}{random}\PYG{o}{.}\PYG{n}{random\PYGZus{}sample}\PYG{p}{(}\PYG{p}{)}
\PYG{g+go}{0.47108547995356098}
\PYG{g+gp}{\PYGZgt{}\PYGZgt{}\PYGZgt{} }\PYG{n+nb}{type}\PYG{p}{(}\PYG{n}{np}\PYG{o}{.}\PYG{n}{random}\PYG{o}{.}\PYG{n}{random\PYGZus{}sample}\PYG{p}{(}\PYG{p}{)}\PYG{p}{)}
\PYG{g+go}{\PYGZlt{}type \PYGZsq{}float\PYGZsq{}\PYGZgt{}}
\PYG{g+gp}{\PYGZgt{}\PYGZgt{}\PYGZgt{} }\PYG{n}{np}\PYG{o}{.}\PYG{n}{random}\PYG{o}{.}\PYG{n}{random\PYGZus{}sample}\PYG{p}{(}\PYG{p}{(}\PYG{l+m+mi}{5}\PYG{p}{,}\PYG{p}{)}\PYG{p}{)}
\PYG{g+go}{array([ 0.30220482,  0.86820401,  0.1654503 ,  0.11659149,  0.54323428])}
\end{Verbatim}

Three-by-two array of random numbers from {[}-5, 0):

\begin{Verbatim}[commandchars=\\\{\}]
\PYG{g+gp}{\PYGZgt{}\PYGZgt{}\PYGZgt{} }\PYG{l+m+mi}{5} \PYG{o}{*} \PYG{n}{np}\PYG{o}{.}\PYG{n}{random}\PYG{o}{.}\PYG{n}{random\PYGZus{}sample}\PYG{p}{(}\PYG{p}{(}\PYG{l+m+mi}{3}\PYG{p}{,} \PYG{l+m+mi}{2}\PYG{p}{)}\PYG{p}{)} \PYG{o}{\PYGZhy{}} \PYG{l+m+mi}{5}
\PYG{g+go}{array([[\PYGZhy{}3.99149989, \PYGZhy{}0.52338984],}
\PYG{g+go}{       [\PYGZhy{}2.99091858, \PYGZhy{}0.79479508],}
\PYG{g+go}{       [\PYGZhy{}1.23204345, \PYGZhy{}1.75224494]])}
\end{Verbatim}

\end{fulllineitems}

\index{random\_integers() (in module lib.graph.raf)}

\begin{fulllineitems}
\phantomsection\label{lib.graph:lib.graph.raf.random_integers}\pysiglinewithargsret{\code{lib.graph.raf.}\bfcode{random\_integers}}{\emph{low}, \emph{high=None}, \emph{size=None}}{}
Return random integers between \emph{low} and \emph{high}, inclusive.

Return random integers from the ``discrete uniform'' distribution in the
closed interval {[}\emph{low}, \emph{high}{]}.  If \emph{high} is None (the default),
then results are from {[}1, \emph{low}{]}.
\begin{description}
\item[{low}] \leavevmode{[}int{]}
Lowest (signed) integer to be drawn from the distribution (unless
\code{high=None}, in which case this parameter is the \emph{highest} such
integer).

\item[{high}] \leavevmode{[}int, optional{]}
If provided, the largest (signed) integer to be drawn from the
distribution (see above for behavior if \code{high=None}).

\item[{size}] \leavevmode{[}int or tuple of ints, optional{]}
Output shape. Default is None, in which case a single int is returned.

\end{description}
\begin{description}
\item[{out}] \leavevmode{[}int or ndarray of ints{]}
\emph{size}-shaped array of random integers from the appropriate
distribution, or a single such random int if \emph{size} not provided.

\end{description}
\begin{description}
\item[{random.randint}] \leavevmode{[}Similar to \emph{random\_integers}, only for the half-open{]}
interval {[}\emph{low}, \emph{high}), and 0 is the lowest value if \emph{high} is
omitted.

\end{description}

To sample from N evenly spaced floating-point numbers between a and b,
use:

\begin{Verbatim}[commandchars=\\\{\}]
\PYG{n}{a} \PYG{o}{+} \PYG{p}{(}\PYG{n}{b} \PYG{o}{\PYGZhy{}} \PYG{n}{a}\PYG{p}{)} \PYG{o}{*} \PYG{p}{(}\PYG{n}{np}\PYG{o}{.}\PYG{n}{random}\PYG{o}{.}\PYG{n}{random\PYGZus{}integers}\PYG{p}{(}\PYG{n}{N}\PYG{p}{)} \PYG{o}{\PYGZhy{}} \PYG{l+m+mi}{1}\PYG{p}{)} \PYG{o}{/} \PYG{p}{(}\PYG{n}{N} \PYG{o}{\PYGZhy{}} \PYG{l+m+mf}{1.}\PYG{p}{)}
\end{Verbatim}

\begin{Verbatim}[commandchars=\\\{\}]
\PYG{g+gp}{\PYGZgt{}\PYGZgt{}\PYGZgt{} }\PYG{n}{np}\PYG{o}{.}\PYG{n}{random}\PYG{o}{.}\PYG{n}{random\PYGZus{}integers}\PYG{p}{(}\PYG{l+m+mi}{5}\PYG{p}{)}
\PYG{g+go}{4}
\PYG{g+gp}{\PYGZgt{}\PYGZgt{}\PYGZgt{} }\PYG{n+nb}{type}\PYG{p}{(}\PYG{n}{np}\PYG{o}{.}\PYG{n}{random}\PYG{o}{.}\PYG{n}{random\PYGZus{}integers}\PYG{p}{(}\PYG{l+m+mi}{5}\PYG{p}{)}\PYG{p}{)}
\PYG{g+go}{\PYGZlt{}type \PYGZsq{}int\PYGZsq{}\PYGZgt{}}
\PYG{g+gp}{\PYGZgt{}\PYGZgt{}\PYGZgt{} }\PYG{n}{np}\PYG{o}{.}\PYG{n}{random}\PYG{o}{.}\PYG{n}{random\PYGZus{}integers}\PYG{p}{(}\PYG{l+m+mi}{5}\PYG{p}{,} \PYG{n}{size}\PYG{o}{=}\PYG{p}{(}\PYG{l+m+mf}{3.}\PYG{p}{,}\PYG{l+m+mf}{2.}\PYG{p}{)}\PYG{p}{)}
\PYG{g+go}{array([[5, 4],}
\PYG{g+go}{       [3, 3],}
\PYG{g+go}{       [4, 5]])}
\end{Verbatim}

Choose five random numbers from the set of five evenly-spaced
numbers between 0 and 2.5, inclusive (\emph{i.e.}, from the set
\({0, 5/8, 10/8, 15/8, 20/8}\)):

\begin{Verbatim}[commandchars=\\\{\}]
\PYG{g+gp}{\PYGZgt{}\PYGZgt{}\PYGZgt{} }\PYG{l+m+mf}{2.5} \PYG{o}{*} \PYG{p}{(}\PYG{n}{np}\PYG{o}{.}\PYG{n}{random}\PYG{o}{.}\PYG{n}{random\PYGZus{}integers}\PYG{p}{(}\PYG{l+m+mi}{5}\PYG{p}{,} \PYG{n}{size}\PYG{o}{=}\PYG{p}{(}\PYG{l+m+mi}{5}\PYG{p}{,}\PYG{p}{)}\PYG{p}{)} \PYG{o}{\PYGZhy{}} \PYG{l+m+mi}{1}\PYG{p}{)} \PYG{o}{/} \PYG{l+m+mf}{4.}
\PYG{g+go}{array([ 0.625,  1.25 ,  0.625,  0.625,  2.5  ])}
\end{Verbatim}

Roll two six sided dice 1000 times and sum the results:

\begin{Verbatim}[commandchars=\\\{\}]
\PYG{g+gp}{\PYGZgt{}\PYGZgt{}\PYGZgt{} }\PYG{n}{d1} \PYG{o}{=} \PYG{n}{np}\PYG{o}{.}\PYG{n}{random}\PYG{o}{.}\PYG{n}{random\PYGZus{}integers}\PYG{p}{(}\PYG{l+m+mi}{1}\PYG{p}{,} \PYG{l+m+mi}{6}\PYG{p}{,} \PYG{l+m+mi}{1000}\PYG{p}{)}
\PYG{g+gp}{\PYGZgt{}\PYGZgt{}\PYGZgt{} }\PYG{n}{d2} \PYG{o}{=} \PYG{n}{np}\PYG{o}{.}\PYG{n}{random}\PYG{o}{.}\PYG{n}{random\PYGZus{}integers}\PYG{p}{(}\PYG{l+m+mi}{1}\PYG{p}{,} \PYG{l+m+mi}{6}\PYG{p}{,} \PYG{l+m+mi}{1000}\PYG{p}{)}
\PYG{g+gp}{\PYGZgt{}\PYGZgt{}\PYGZgt{} }\PYG{n}{dsums} \PYG{o}{=} \PYG{n}{d1} \PYG{o}{+} \PYG{n}{d2}
\end{Verbatim}

Display results as a histogram:

\begin{Verbatim}[commandchars=\\\{\}]
\PYG{g+gp}{\PYGZgt{}\PYGZgt{}\PYGZgt{} }\PYG{k+kn}{import} \PYG{n+nn}{matplotlib.pyplot} \PYG{k+kn}{as} \PYG{n+nn}{plt}
\PYG{g+gp}{\PYGZgt{}\PYGZgt{}\PYGZgt{} }\PYG{n}{count}\PYG{p}{,} \PYG{n}{bins}\PYG{p}{,} \PYG{n}{ignored} \PYG{o}{=} \PYG{n}{plt}\PYG{o}{.}\PYG{n}{hist}\PYG{p}{(}\PYG{n}{dsums}\PYG{p}{,} \PYG{l+m+mi}{11}\PYG{p}{,} \PYG{n}{normed}\PYG{o}{=}\PYG{n+nb+bp}{True}\PYG{p}{)}
\PYG{g+gp}{\PYGZgt{}\PYGZgt{}\PYGZgt{} }\PYG{n}{plt}\PYG{o}{.}\PYG{n}{show}\PYG{p}{(}\PYG{p}{)}
\end{Verbatim}

\end{fulllineitems}

\index{random\_sample() (in module lib.graph.raf)}

\begin{fulllineitems}
\phantomsection\label{lib.graph:lib.graph.raf.random_sample}\pysiglinewithargsret{\code{lib.graph.raf.}\bfcode{random\_sample}}{\emph{size=None}}{}
Return random floats in the half-open interval {[}0.0, 1.0).

Results are from the ``continuous uniform'' distribution over the
stated interval.  To sample \(Unif[a, b), b > a\) multiply
the output of \emph{random\_sample} by \emph{(b-a)} and add \emph{a}:

\begin{Verbatim}[commandchars=\\\{\}]
\PYG{p}{(}\PYG{n}{b} \PYG{o}{\PYGZhy{}} \PYG{n}{a}\PYG{p}{)} \PYG{o}{*} \PYG{n}{random\PYGZus{}sample}\PYG{p}{(}\PYG{p}{)} \PYG{o}{+} \PYG{n}{a}
\end{Verbatim}
\begin{description}
\item[{size}] \leavevmode{[}int or tuple of ints, optional{]}
Defines the shape of the returned array of random floats. If None
(the default), returns a single float.

\end{description}
\begin{description}
\item[{out}] \leavevmode{[}float or ndarray of floats{]}
Array of random floats of shape \emph{size} (unless \code{size=None}, in which
case a single float is returned).

\end{description}

\begin{Verbatim}[commandchars=\\\{\}]
\PYG{g+gp}{\PYGZgt{}\PYGZgt{}\PYGZgt{} }\PYG{n}{np}\PYG{o}{.}\PYG{n}{random}\PYG{o}{.}\PYG{n}{random\PYGZus{}sample}\PYG{p}{(}\PYG{p}{)}
\PYG{g+go}{0.47108547995356098}
\PYG{g+gp}{\PYGZgt{}\PYGZgt{}\PYGZgt{} }\PYG{n+nb}{type}\PYG{p}{(}\PYG{n}{np}\PYG{o}{.}\PYG{n}{random}\PYG{o}{.}\PYG{n}{random\PYGZus{}sample}\PYG{p}{(}\PYG{p}{)}\PYG{p}{)}
\PYG{g+go}{\PYGZlt{}type \PYGZsq{}float\PYGZsq{}\PYGZgt{}}
\PYG{g+gp}{\PYGZgt{}\PYGZgt{}\PYGZgt{} }\PYG{n}{np}\PYG{o}{.}\PYG{n}{random}\PYG{o}{.}\PYG{n}{random\PYGZus{}sample}\PYG{p}{(}\PYG{p}{(}\PYG{l+m+mi}{5}\PYG{p}{,}\PYG{p}{)}\PYG{p}{)}
\PYG{g+go}{array([ 0.30220482,  0.86820401,  0.1654503 ,  0.11659149,  0.54323428])}
\end{Verbatim}

Three-by-two array of random numbers from {[}-5, 0):

\begin{Verbatim}[commandchars=\\\{\}]
\PYG{g+gp}{\PYGZgt{}\PYGZgt{}\PYGZgt{} }\PYG{l+m+mi}{5} \PYG{o}{*} \PYG{n}{np}\PYG{o}{.}\PYG{n}{random}\PYG{o}{.}\PYG{n}{random\PYGZus{}sample}\PYG{p}{(}\PYG{p}{(}\PYG{l+m+mi}{3}\PYG{p}{,} \PYG{l+m+mi}{2}\PYG{p}{)}\PYG{p}{)} \PYG{o}{\PYGZhy{}} \PYG{l+m+mi}{5}
\PYG{g+go}{array([[\PYGZhy{}3.99149989, \PYGZhy{}0.52338984],}
\PYG{g+go}{       [\PYGZhy{}2.99091858, \PYGZhy{}0.79479508],}
\PYG{g+go}{       [\PYGZhy{}1.23204345, \PYGZhy{}1.75224494]])}
\end{Verbatim}

\end{fulllineitems}

\index{ranf() (in module lib.graph.raf)}

\begin{fulllineitems}
\phantomsection\label{lib.graph:lib.graph.raf.ranf}\pysiglinewithargsret{\code{lib.graph.raf.}\bfcode{ranf}}{}{}
random\_sample(size=None)

Return random floats in the half-open interval {[}0.0, 1.0).

Results are from the ``continuous uniform'' distribution over the
stated interval.  To sample \(Unif[a, b), b > a\) multiply
the output of \emph{random\_sample} by \emph{(b-a)} and add \emph{a}:

\begin{Verbatim}[commandchars=\\\{\}]
\PYG{p}{(}\PYG{n}{b} \PYG{o}{\PYGZhy{}} \PYG{n}{a}\PYG{p}{)} \PYG{o}{*} \PYG{n}{random\PYGZus{}sample}\PYG{p}{(}\PYG{p}{)} \PYG{o}{+} \PYG{n}{a}
\end{Verbatim}
\begin{description}
\item[{size}] \leavevmode{[}int or tuple of ints, optional{]}
Defines the shape of the returned array of random floats. If None
(the default), returns a single float.

\end{description}
\begin{description}
\item[{out}] \leavevmode{[}float or ndarray of floats{]}
Array of random floats of shape \emph{size} (unless \code{size=None}, in which
case a single float is returned).

\end{description}

\begin{Verbatim}[commandchars=\\\{\}]
\PYG{g+gp}{\PYGZgt{}\PYGZgt{}\PYGZgt{} }\PYG{n}{np}\PYG{o}{.}\PYG{n}{random}\PYG{o}{.}\PYG{n}{random\PYGZus{}sample}\PYG{p}{(}\PYG{p}{)}
\PYG{g+go}{0.47108547995356098}
\PYG{g+gp}{\PYGZgt{}\PYGZgt{}\PYGZgt{} }\PYG{n+nb}{type}\PYG{p}{(}\PYG{n}{np}\PYG{o}{.}\PYG{n}{random}\PYG{o}{.}\PYG{n}{random\PYGZus{}sample}\PYG{p}{(}\PYG{p}{)}\PYG{p}{)}
\PYG{g+go}{\PYGZlt{}type \PYGZsq{}float\PYGZsq{}\PYGZgt{}}
\PYG{g+gp}{\PYGZgt{}\PYGZgt{}\PYGZgt{} }\PYG{n}{np}\PYG{o}{.}\PYG{n}{random}\PYG{o}{.}\PYG{n}{random\PYGZus{}sample}\PYG{p}{(}\PYG{p}{(}\PYG{l+m+mi}{5}\PYG{p}{,}\PYG{p}{)}\PYG{p}{)}
\PYG{g+go}{array([ 0.30220482,  0.86820401,  0.1654503 ,  0.11659149,  0.54323428])}
\end{Verbatim}

Three-by-two array of random numbers from {[}-5, 0):

\begin{Verbatim}[commandchars=\\\{\}]
\PYG{g+gp}{\PYGZgt{}\PYGZgt{}\PYGZgt{} }\PYG{l+m+mi}{5} \PYG{o}{*} \PYG{n}{np}\PYG{o}{.}\PYG{n}{random}\PYG{o}{.}\PYG{n}{random\PYGZus{}sample}\PYG{p}{(}\PYG{p}{(}\PYG{l+m+mi}{3}\PYG{p}{,} \PYG{l+m+mi}{2}\PYG{p}{)}\PYG{p}{)} \PYG{o}{\PYGZhy{}} \PYG{l+m+mi}{5}
\PYG{g+go}{array([[\PYGZhy{}3.99149989, \PYGZhy{}0.52338984],}
\PYG{g+go}{       [\PYGZhy{}2.99091858, \PYGZhy{}0.79479508],}
\PYG{g+go}{       [\PYGZhy{}1.23204345, \PYGZhy{}1.75224494]])}
\end{Verbatim}

\end{fulllineitems}

\index{rayleigh() (in module lib.graph.raf)}

\begin{fulllineitems}
\phantomsection\label{lib.graph:lib.graph.raf.rayleigh}\pysiglinewithargsret{\code{lib.graph.raf.}\bfcode{rayleigh}}{\emph{scale=1.0}, \emph{size=None}}{}
Draw samples from a Rayleigh distribution.

The \(\chi\) and Weibull distributions are generalizations of the
Rayleigh.
\begin{description}
\item[{scale}] \leavevmode{[}scalar{]}
Scale, also equals the mode. Should be \textgreater{}= 0.

\item[{size}] \leavevmode{[}int or tuple of ints, optional{]}
Shape of the output. Default is None, in which case a single
value is returned.

\end{description}

The probability density function for the Rayleigh distribution is
\begin{gather}
\begin{split}P(x;scale) = \frac{x}{scale^2}e^{\frac{-x^2}{2 \cdotp scale^2}}\end{split}\notag
\end{gather}
The Rayleigh distribution arises if the wind speed and wind direction are
both gaussian variables, then the vector wind velocity forms a Rayleigh
distribution. The Rayleigh distribution is used to model the expected
output from wind turbines.

Draw values from the distribution and plot the histogram

\begin{Verbatim}[commandchars=\\\{\}]
\PYG{g+gp}{\PYGZgt{}\PYGZgt{}\PYGZgt{} }\PYG{n}{values} \PYG{o}{=} \PYG{n}{hist}\PYG{p}{(}\PYG{n}{np}\PYG{o}{.}\PYG{n}{random}\PYG{o}{.}\PYG{n}{rayleigh}\PYG{p}{(}\PYG{l+m+mi}{3}\PYG{p}{,} \PYG{l+m+mi}{100000}\PYG{p}{)}\PYG{p}{,} \PYG{n}{bins}\PYG{o}{=}\PYG{l+m+mi}{200}\PYG{p}{,} \PYG{n}{normed}\PYG{o}{=}\PYG{n+nb+bp}{True}\PYG{p}{)}
\end{Verbatim}

Wave heights tend to follow a Rayleigh distribution. If the mean wave
height is 1 meter, what fraction of waves are likely to be larger than 3
meters?

\begin{Verbatim}[commandchars=\\\{\}]
\PYG{g+gp}{\PYGZgt{}\PYGZgt{}\PYGZgt{} }\PYG{n}{meanvalue} \PYG{o}{=} \PYG{l+m+mi}{1}
\PYG{g+gp}{\PYGZgt{}\PYGZgt{}\PYGZgt{} }\PYG{n}{modevalue} \PYG{o}{=} \PYG{n}{np}\PYG{o}{.}\PYG{n}{sqrt}\PYG{p}{(}\PYG{l+m+mi}{2} \PYG{o}{/} \PYG{n}{np}\PYG{o}{.}\PYG{n}{pi}\PYG{p}{)} \PYG{o}{*} \PYG{n}{meanvalue}
\PYG{g+gp}{\PYGZgt{}\PYGZgt{}\PYGZgt{} }\PYG{n}{s} \PYG{o}{=} \PYG{n}{np}\PYG{o}{.}\PYG{n}{random}\PYG{o}{.}\PYG{n}{rayleigh}\PYG{p}{(}\PYG{n}{modevalue}\PYG{p}{,} \PYG{l+m+mi}{1000000}\PYG{p}{)}
\end{Verbatim}

The percentage of waves larger than 3 meters is:

\begin{Verbatim}[commandchars=\\\{\}]
\PYG{g+gp}{\PYGZgt{}\PYGZgt{}\PYGZgt{} }\PYG{l+m+mf}{100.}\PYG{o}{*}\PYG{n+nb}{sum}\PYG{p}{(}\PYG{n}{s}\PYG{o}{\PYGZgt{}}\PYG{l+m+mi}{3}\PYG{p}{)}\PYG{o}{/}\PYG{l+m+mf}{1000000.}
\PYG{g+go}{0.087300000000000003}
\end{Verbatim}

\end{fulllineitems}

\index{sample() (in module lib.graph.raf)}

\begin{fulllineitems}
\phantomsection\label{lib.graph:lib.graph.raf.sample}\pysiglinewithargsret{\code{lib.graph.raf.}\bfcode{sample}}{}{}
random\_sample(size=None)

Return random floats in the half-open interval {[}0.0, 1.0).

Results are from the ``continuous uniform'' distribution over the
stated interval.  To sample \(Unif[a, b), b > a\) multiply
the output of \emph{random\_sample} by \emph{(b-a)} and add \emph{a}:

\begin{Verbatim}[commandchars=\\\{\}]
\PYG{p}{(}\PYG{n}{b} \PYG{o}{\PYGZhy{}} \PYG{n}{a}\PYG{p}{)} \PYG{o}{*} \PYG{n}{random\PYGZus{}sample}\PYG{p}{(}\PYG{p}{)} \PYG{o}{+} \PYG{n}{a}
\end{Verbatim}
\begin{description}
\item[{size}] \leavevmode{[}int or tuple of ints, optional{]}
Defines the shape of the returned array of random floats. If None
(the default), returns a single float.

\end{description}
\begin{description}
\item[{out}] \leavevmode{[}float or ndarray of floats{]}
Array of random floats of shape \emph{size} (unless \code{size=None}, in which
case a single float is returned).

\end{description}

\begin{Verbatim}[commandchars=\\\{\}]
\PYG{g+gp}{\PYGZgt{}\PYGZgt{}\PYGZgt{} }\PYG{n}{np}\PYG{o}{.}\PYG{n}{random}\PYG{o}{.}\PYG{n}{random\PYGZus{}sample}\PYG{p}{(}\PYG{p}{)}
\PYG{g+go}{0.47108547995356098}
\PYG{g+gp}{\PYGZgt{}\PYGZgt{}\PYGZgt{} }\PYG{n+nb}{type}\PYG{p}{(}\PYG{n}{np}\PYG{o}{.}\PYG{n}{random}\PYG{o}{.}\PYG{n}{random\PYGZus{}sample}\PYG{p}{(}\PYG{p}{)}\PYG{p}{)}
\PYG{g+go}{\PYGZlt{}type \PYGZsq{}float\PYGZsq{}\PYGZgt{}}
\PYG{g+gp}{\PYGZgt{}\PYGZgt{}\PYGZgt{} }\PYG{n}{np}\PYG{o}{.}\PYG{n}{random}\PYG{o}{.}\PYG{n}{random\PYGZus{}sample}\PYG{p}{(}\PYG{p}{(}\PYG{l+m+mi}{5}\PYG{p}{,}\PYG{p}{)}\PYG{p}{)}
\PYG{g+go}{array([ 0.30220482,  0.86820401,  0.1654503 ,  0.11659149,  0.54323428])}
\end{Verbatim}

Three-by-two array of random numbers from {[}-5, 0):

\begin{Verbatim}[commandchars=\\\{\}]
\PYG{g+gp}{\PYGZgt{}\PYGZgt{}\PYGZgt{} }\PYG{l+m+mi}{5} \PYG{o}{*} \PYG{n}{np}\PYG{o}{.}\PYG{n}{random}\PYG{o}{.}\PYG{n}{random\PYGZus{}sample}\PYG{p}{(}\PYG{p}{(}\PYG{l+m+mi}{3}\PYG{p}{,} \PYG{l+m+mi}{2}\PYG{p}{)}\PYG{p}{)} \PYG{o}{\PYGZhy{}} \PYG{l+m+mi}{5}
\PYG{g+go}{array([[\PYGZhy{}3.99149989, \PYGZhy{}0.52338984],}
\PYG{g+go}{       [\PYGZhy{}2.99091858, \PYGZhy{}0.79479508],}
\PYG{g+go}{       [\PYGZhy{}1.23204345, \PYGZhy{}1.75224494]])}
\end{Verbatim}

\end{fulllineitems}

\index{seed() (in module lib.graph.raf)}

\begin{fulllineitems}
\phantomsection\label{lib.graph:lib.graph.raf.seed}\pysiglinewithargsret{\code{lib.graph.raf.}\bfcode{seed}}{\emph{seed=None}}{}
Seed the generator.

This method is called when \emph{RandomState} is initialized. It can be
called again to re-seed the generator. For details, see \emph{RandomState}.
\begin{description}
\item[{seed}] \leavevmode{[}int or array\_like, optional{]}
Seed for \emph{RandomState}.

\end{description}

RandomState

\end{fulllineitems}

\index{set\_state() (in module lib.graph.raf)}

\begin{fulllineitems}
\phantomsection\label{lib.graph:lib.graph.raf.set_state}\pysiglinewithargsret{\code{lib.graph.raf.}\bfcode{set\_state}}{\emph{state}}{}
Set the internal state of the generator from a tuple.

For use if one has reason to manually (re-)set the internal state of the
``Mersenne Twister''{\color{red}\bfseries{}{[}1{]}\_} pseudo-random number generating algorithm.
\begin{description}
\item[{state}] \leavevmode{[}tuple(str, ndarray of 624 uints, int, int, float){]}
The \emph{state} tuple has the following items:
\begin{enumerate}
\item {} 
the string `MT19937', specifying the Mersenne Twister algorithm.

\item {} 
a 1-D array of 624 unsigned integers \code{keys}.

\item {} 
an integer \code{pos}.

\item {} 
an integer \code{has\_gauss}.

\item {} 
a float \code{cached\_gaussian}.

\end{enumerate}

\end{description}
\begin{description}
\item[{out}] \leavevmode{[}None{]}
Returns `None' on success.

\end{description}

get\_state

\emph{set\_state} and \emph{get\_state} are not needed to work with any of the
random distributions in NumPy. If the internal state is manually altered,
the user should know exactly what he/she is doing.

For backwards compatibility, the form (str, array of 624 uints, int) is
also accepted although it is missing some information about the cached
Gaussian value: \code{state = ('MT19937', keys, pos)}.

\end{fulllineitems}

\index{shuffle() (in module lib.graph.raf)}

\begin{fulllineitems}
\phantomsection\label{lib.graph:lib.graph.raf.shuffle}\pysiglinewithargsret{\code{lib.graph.raf.}\bfcode{shuffle}}{\emph{x}}{}
Modify a sequence in-place by shuffling its contents.
\begin{description}
\item[{x}] \leavevmode{[}array\_like{]}
The array or list to be shuffled.

\end{description}

None

\begin{Verbatim}[commandchars=\\\{\}]
\PYG{g+gp}{\PYGZgt{}\PYGZgt{}\PYGZgt{} }\PYG{n}{arr} \PYG{o}{=} \PYG{n}{np}\PYG{o}{.}\PYG{n}{arange}\PYG{p}{(}\PYG{l+m+mi}{10}\PYG{p}{)}
\PYG{g+gp}{\PYGZgt{}\PYGZgt{}\PYGZgt{} }\PYG{n}{np}\PYG{o}{.}\PYG{n}{random}\PYG{o}{.}\PYG{n}{shuffle}\PYG{p}{(}\PYG{n}{arr}\PYG{p}{)}
\PYG{g+gp}{\PYGZgt{}\PYGZgt{}\PYGZgt{} }\PYG{n}{arr}
\PYG{g+go}{[1 7 5 2 9 4 3 6 0 8]}
\end{Verbatim}

This function only shuffles the array along the first index of a
multi-dimensional array:

\begin{Verbatim}[commandchars=\\\{\}]
\PYG{g+gp}{\PYGZgt{}\PYGZgt{}\PYGZgt{} }\PYG{n}{arr} \PYG{o}{=} \PYG{n}{np}\PYG{o}{.}\PYG{n}{arange}\PYG{p}{(}\PYG{l+m+mi}{9}\PYG{p}{)}\PYG{o}{.}\PYG{n}{reshape}\PYG{p}{(}\PYG{p}{(}\PYG{l+m+mi}{3}\PYG{p}{,} \PYG{l+m+mi}{3}\PYG{p}{)}\PYG{p}{)}
\PYG{g+gp}{\PYGZgt{}\PYGZgt{}\PYGZgt{} }\PYG{n}{np}\PYG{o}{.}\PYG{n}{random}\PYG{o}{.}\PYG{n}{shuffle}\PYG{p}{(}\PYG{n}{arr}\PYG{p}{)}
\PYG{g+gp}{\PYGZgt{}\PYGZgt{}\PYGZgt{} }\PYG{n}{arr}
\PYG{g+go}{array([[3, 4, 5],}
\PYG{g+go}{       [6, 7, 8],}
\PYG{g+go}{       [0, 1, 2]])}
\end{Verbatim}

\end{fulllineitems}

\index{standard\_cauchy() (in module lib.graph.raf)}

\begin{fulllineitems}
\phantomsection\label{lib.graph:lib.graph.raf.standard_cauchy}\pysiglinewithargsret{\code{lib.graph.raf.}\bfcode{standard\_cauchy}}{\emph{size=None}}{}
Standard Cauchy distribution with mode = 0.

Also known as the Lorentz distribution.
\begin{description}
\item[{size}] \leavevmode{[}int or tuple of ints{]}
Shape of the output.

\end{description}
\begin{description}
\item[{samples}] \leavevmode{[}ndarray or scalar{]}
The drawn samples.

\end{description}

The probability density function for the full Cauchy distribution is
\begin{gather}
\begin{split}P(x; x_0, \gamma) = \frac{1}{\pi \gamma \bigl[ 1+
(\frac{x-x_0}{\gamma})^2 \bigr] }\end{split}\notag
\end{gather}
and the Standard Cauchy distribution just sets \(x_0=0\) and
\(\gamma=1\)

The Cauchy distribution arises in the solution to the driven harmonic
oscillator problem, and also describes spectral line broadening. It
also describes the distribution of values at which a line tilted at
a random angle will cut the x axis.

When studying hypothesis tests that assume normality, seeing how the
tests perform on data from a Cauchy distribution is a good indicator of
their sensitivity to a heavy-tailed distribution, since the Cauchy looks
very much like a Gaussian distribution, but with heavier tails.

Draw samples and plot the distribution:

\begin{Verbatim}[commandchars=\\\{\}]
\PYG{g+gp}{\PYGZgt{}\PYGZgt{}\PYGZgt{} }\PYG{n}{s} \PYG{o}{=} \PYG{n}{np}\PYG{o}{.}\PYG{n}{random}\PYG{o}{.}\PYG{n}{standard\PYGZus{}cauchy}\PYG{p}{(}\PYG{l+m+mi}{1000000}\PYG{p}{)}
\PYG{g+gp}{\PYGZgt{}\PYGZgt{}\PYGZgt{} }\PYG{n}{s} \PYG{o}{=} \PYG{n}{s}\PYG{p}{[}\PYG{p}{(}\PYG{n}{s}\PYG{o}{\PYGZgt{}}\PYG{o}{\PYGZhy{}}\PYG{l+m+mi}{25}\PYG{p}{)} \PYG{o}{\PYGZam{}} \PYG{p}{(}\PYG{n}{s}\PYG{o}{\PYGZlt{}}\PYG{l+m+mi}{25}\PYG{p}{)}\PYG{p}{]}  \PYG{c}{\PYGZsh{} truncate distribution so it plots well}
\PYG{g+gp}{\PYGZgt{}\PYGZgt{}\PYGZgt{} }\PYG{n}{plt}\PYG{o}{.}\PYG{n}{hist}\PYG{p}{(}\PYG{n}{s}\PYG{p}{,} \PYG{n}{bins}\PYG{o}{=}\PYG{l+m+mi}{100}\PYG{p}{)}
\PYG{g+gp}{\PYGZgt{}\PYGZgt{}\PYGZgt{} }\PYG{n}{plt}\PYG{o}{.}\PYG{n}{show}\PYG{p}{(}\PYG{p}{)}
\end{Verbatim}

\end{fulllineitems}

\index{standard\_exponential() (in module lib.graph.raf)}

\begin{fulllineitems}
\phantomsection\label{lib.graph:lib.graph.raf.standard_exponential}\pysiglinewithargsret{\code{lib.graph.raf.}\bfcode{standard\_exponential}}{\emph{size=None}}{}
Draw samples from the standard exponential distribution.

\emph{standard\_exponential} is identical to the exponential distribution
with a scale parameter of 1.
\begin{description}
\item[{size}] \leavevmode{[}int or tuple of ints{]}
Shape of the output.

\end{description}
\begin{description}
\item[{out}] \leavevmode{[}float or ndarray{]}
Drawn samples.

\end{description}

Output a 3x8000 array:

\begin{Verbatim}[commandchars=\\\{\}]
\PYG{g+gp}{\PYGZgt{}\PYGZgt{}\PYGZgt{} }\PYG{n}{n} \PYG{o}{=} \PYG{n}{np}\PYG{o}{.}\PYG{n}{random}\PYG{o}{.}\PYG{n}{standard\PYGZus{}exponential}\PYG{p}{(}\PYG{p}{(}\PYG{l+m+mi}{3}\PYG{p}{,} \PYG{l+m+mi}{8000}\PYG{p}{)}\PYG{p}{)}
\end{Verbatim}

\end{fulllineitems}

\index{standard\_gamma() (in module lib.graph.raf)}

\begin{fulllineitems}
\phantomsection\label{lib.graph:lib.graph.raf.standard_gamma}\pysiglinewithargsret{\code{lib.graph.raf.}\bfcode{standard\_gamma}}{\emph{shape}, \emph{size=None}}{}
Draw samples from a Standard Gamma distribution.

Samples are drawn from a Gamma distribution with specified parameters,
shape (sometimes designated ``k'') and scale=1.
\begin{description}
\item[{shape}] \leavevmode{[}float{]}
Parameter, should be \textgreater{} 0.

\item[{size}] \leavevmode{[}int or tuple of ints{]}
Output shape.  If the given shape is, e.g., \code{(m, n, k)}, then
\code{m * n * k} samples are drawn.

\end{description}
\begin{description}
\item[{samples}] \leavevmode{[}ndarray or scalar{]}
The drawn samples.

\end{description}
\begin{description}
\item[{scipy.stats.distributions.gamma}] \leavevmode{[}probability density function,{]}
distribution or cumulative density function, etc.

\end{description}

The probability density for the Gamma distribution is
\begin{gather}
\begin{split}p(x) = x^{k-1}\frac{e^{-x/\theta}}{\theta^k\Gamma(k)},\end{split}\notag
\end{gather}
where \(k\) is the shape and \(\theta\) the scale,
and \(\Gamma\) is the Gamma function.

The Gamma distribution is often used to model the times to failure of
electronic components, and arises naturally in processes for which the
waiting times between Poisson distributed events are relevant.

Draw samples from the distribution:

\begin{Verbatim}[commandchars=\\\{\}]
\PYG{g+gp}{\PYGZgt{}\PYGZgt{}\PYGZgt{} }\PYG{n}{shape}\PYG{p}{,} \PYG{n}{scale} \PYG{o}{=} \PYG{l+m+mf}{2.}\PYG{p}{,} \PYG{l+m+mf}{1.} \PYG{c}{\PYGZsh{} mean and width}
\PYG{g+gp}{\PYGZgt{}\PYGZgt{}\PYGZgt{} }\PYG{n}{s} \PYG{o}{=} \PYG{n}{np}\PYG{o}{.}\PYG{n}{random}\PYG{o}{.}\PYG{n}{standard\PYGZus{}gamma}\PYG{p}{(}\PYG{n}{shape}\PYG{p}{,} \PYG{l+m+mi}{1000000}\PYG{p}{)}
\end{Verbatim}

Display the histogram of the samples, along with
the probability density function:

\begin{Verbatim}[commandchars=\\\{\}]
\PYG{g+gp}{\PYGZgt{}\PYGZgt{}\PYGZgt{} }\PYG{k+kn}{import} \PYG{n+nn}{matplotlib.pyplot} \PYG{k+kn}{as} \PYG{n+nn}{plt}
\PYG{g+gp}{\PYGZgt{}\PYGZgt{}\PYGZgt{} }\PYG{k+kn}{import} \PYG{n+nn}{scipy.special} \PYG{k+kn}{as} \PYG{n+nn}{sps}
\PYG{g+gp}{\PYGZgt{}\PYGZgt{}\PYGZgt{} }\PYG{n}{count}\PYG{p}{,} \PYG{n}{bins}\PYG{p}{,} \PYG{n}{ignored} \PYG{o}{=} \PYG{n}{plt}\PYG{o}{.}\PYG{n}{hist}\PYG{p}{(}\PYG{n}{s}\PYG{p}{,} \PYG{l+m+mi}{50}\PYG{p}{,} \PYG{n}{normed}\PYG{o}{=}\PYG{n+nb+bp}{True}\PYG{p}{)}
\PYG{g+gp}{\PYGZgt{}\PYGZgt{}\PYGZgt{} }\PYG{n}{y} \PYG{o}{=} \PYG{n}{bins}\PYG{o}{*}\PYG{o}{*}\PYG{p}{(}\PYG{n}{shape}\PYG{o}{\PYGZhy{}}\PYG{l+m+mi}{1}\PYG{p}{)} \PYG{o}{*} \PYG{p}{(}\PYG{p}{(}\PYG{n}{np}\PYG{o}{.}\PYG{n}{exp}\PYG{p}{(}\PYG{o}{\PYGZhy{}}\PYG{n}{bins}\PYG{o}{/}\PYG{n}{scale}\PYG{p}{)}\PYG{p}{)}\PYG{o}{/} \PYGZbs{}
\PYG{g+gp}{... }                      \PYG{p}{(}\PYG{n}{sps}\PYG{o}{.}\PYG{n}{gamma}\PYG{p}{(}\PYG{n}{shape}\PYG{p}{)} \PYG{o}{*} \PYG{n}{scale}\PYG{o}{*}\PYG{o}{*}\PYG{n}{shape}\PYG{p}{)}\PYG{p}{)}
\PYG{g+gp}{\PYGZgt{}\PYGZgt{}\PYGZgt{} }\PYG{n}{plt}\PYG{o}{.}\PYG{n}{plot}\PYG{p}{(}\PYG{n}{bins}\PYG{p}{,} \PYG{n}{y}\PYG{p}{,} \PYG{n}{linewidth}\PYG{o}{=}\PYG{l+m+mi}{2}\PYG{p}{,} \PYG{n}{color}\PYG{o}{=}\PYG{l+s}{\PYGZsq{}}\PYG{l+s}{r}\PYG{l+s}{\PYGZsq{}}\PYG{p}{)}
\PYG{g+gp}{\PYGZgt{}\PYGZgt{}\PYGZgt{} }\PYG{n}{plt}\PYG{o}{.}\PYG{n}{show}\PYG{p}{(}\PYG{p}{)}
\end{Verbatim}

\end{fulllineitems}

\index{standard\_normal() (in module lib.graph.raf)}

\begin{fulllineitems}
\phantomsection\label{lib.graph:lib.graph.raf.standard_normal}\pysiglinewithargsret{\code{lib.graph.raf.}\bfcode{standard\_normal}}{\emph{size=None}}{}
Returns samples from a Standard Normal distribution (mean=0, stdev=1).
\begin{description}
\item[{size}] \leavevmode{[}int or tuple of ints, optional{]}
Output shape. Default is None, in which case a single value is
returned.

\end{description}
\begin{description}
\item[{out}] \leavevmode{[}float or ndarray{]}
Drawn samples.

\end{description}

\begin{Verbatim}[commandchars=\\\{\}]
\PYG{g+gp}{\PYGZgt{}\PYGZgt{}\PYGZgt{} }\PYG{n}{s} \PYG{o}{=} \PYG{n}{np}\PYG{o}{.}\PYG{n}{random}\PYG{o}{.}\PYG{n}{standard\PYGZus{}normal}\PYG{p}{(}\PYG{l+m+mi}{8000}\PYG{p}{)}
\PYG{g+gp}{\PYGZgt{}\PYGZgt{}\PYGZgt{} }\PYG{n}{s}
\PYG{g+go}{array([ 0.6888893 ,  0.78096262, \PYGZhy{}0.89086505, ...,  0.49876311, \PYGZsh{}random}
\PYG{g+go}{       \PYGZhy{}0.38672696, \PYGZhy{}0.4685006 ])                               \PYGZsh{}random}
\PYG{g+gp}{\PYGZgt{}\PYGZgt{}\PYGZgt{} }\PYG{n}{s}\PYG{o}{.}\PYG{n}{shape}
\PYG{g+go}{(8000,)}
\PYG{g+gp}{\PYGZgt{}\PYGZgt{}\PYGZgt{} }\PYG{n}{s} \PYG{o}{=} \PYG{n}{np}\PYG{o}{.}\PYG{n}{random}\PYG{o}{.}\PYG{n}{standard\PYGZus{}normal}\PYG{p}{(}\PYG{n}{size}\PYG{o}{=}\PYG{p}{(}\PYG{l+m+mi}{3}\PYG{p}{,} \PYG{l+m+mi}{4}\PYG{p}{,} \PYG{l+m+mi}{2}\PYG{p}{)}\PYG{p}{)}
\PYG{g+gp}{\PYGZgt{}\PYGZgt{}\PYGZgt{} }\PYG{n}{s}\PYG{o}{.}\PYG{n}{shape}
\PYG{g+go}{(3, 4, 2)}
\end{Verbatim}

\end{fulllineitems}

\index{standard\_t() (in module lib.graph.raf)}

\begin{fulllineitems}
\phantomsection\label{lib.graph:lib.graph.raf.standard_t}\pysiglinewithargsret{\code{lib.graph.raf.}\bfcode{standard\_t}}{\emph{df}, \emph{size=None}}{}
Standard Student's t distribution with df degrees of freedom.

A special case of the hyperbolic distribution.
As \emph{df} gets large, the result resembles that of the standard normal
distribution (\emph{standard\_normal}).
\begin{description}
\item[{df}] \leavevmode{[}int{]}
Degrees of freedom, should be \textgreater{} 0.

\item[{size}] \leavevmode{[}int or tuple of ints, optional{]}
Output shape. Default is None, in which case a single value is
returned.

\end{description}
\begin{description}
\item[{samples}] \leavevmode{[}ndarray or scalar{]}
Drawn samples.

\end{description}

The probability density function for the t distribution is
\begin{gather}
\begin{split}P(x, df) = \frac{\Gamma(\frac{df+1}{2})}{\sqrt{\pi df}
\Gamma(\frac{df}{2})}\Bigl( 1+\frac{x^2}{df} \Bigr)^{-(df+1)/2}\end{split}\notag
\end{gather}
The t test is based on an assumption that the data come from a Normal
distribution. The t test provides a way to test whether the sample mean
(that is the mean calculated from the data) is a good estimate of the true
mean.

The derivation of the t-distribution was forst published in 1908 by William
Gisset while working for the Guinness Brewery in Dublin. Due to proprietary
issues, he had to publish under a pseudonym, and so he used the name
Student.

From Dalgaard page 83 {\color{red}\bfseries{}{[}1{]}\_}, suppose the daily energy intake for 11
women in Kj is:

\begin{Verbatim}[commandchars=\\\{\}]
\PYG{g+gp}{\PYGZgt{}\PYGZgt{}\PYGZgt{} }\PYG{n}{intake} \PYG{o}{=} \PYG{n}{np}\PYG{o}{.}\PYG{n}{array}\PYG{p}{(}\PYG{p}{[}\PYG{l+m+mf}{5260.}\PYG{p}{,} \PYG{l+m+mi}{5470}\PYG{p}{,} \PYG{l+m+mi}{5640}\PYG{p}{,} \PYG{l+m+mi}{6180}\PYG{p}{,} \PYG{l+m+mi}{6390}\PYG{p}{,} \PYG{l+m+mi}{6515}\PYG{p}{,} \PYG{l+m+mi}{6805}\PYG{p}{,} \PYG{l+m+mi}{7515}\PYG{p}{,} \PYGZbs{}
\PYG{g+gp}{... }                   \PYG{l+m+mi}{7515}\PYG{p}{,} \PYG{l+m+mi}{8230}\PYG{p}{,} \PYG{l+m+mi}{8770}\PYG{p}{]}\PYG{p}{)}
\end{Verbatim}

Does their energy intake deviate systematically from the recommended
value of 7725 kJ?

We have 10 degrees of freedom, so is the sample mean within 95\% of the
recommended value?

\begin{Verbatim}[commandchars=\\\{\}]
\PYG{g+gp}{\PYGZgt{}\PYGZgt{}\PYGZgt{} }\PYG{n}{s} \PYG{o}{=} \PYG{n}{np}\PYG{o}{.}\PYG{n}{random}\PYG{o}{.}\PYG{n}{standard\PYGZus{}t}\PYG{p}{(}\PYG{l+m+mi}{10}\PYG{p}{,} \PYG{n}{size}\PYG{o}{=}\PYG{l+m+mi}{100000}\PYG{p}{)}
\PYG{g+gp}{\PYGZgt{}\PYGZgt{}\PYGZgt{} }\PYG{n}{np}\PYG{o}{.}\PYG{n}{mean}\PYG{p}{(}\PYG{n}{intake}\PYG{p}{)}
\PYG{g+go}{6753.636363636364}
\PYG{g+gp}{\PYGZgt{}\PYGZgt{}\PYGZgt{} }\PYG{n}{intake}\PYG{o}{.}\PYG{n}{std}\PYG{p}{(}\PYG{n}{ddof}\PYG{o}{=}\PYG{l+m+mi}{1}\PYG{p}{)}
\PYG{g+go}{1142.1232221373727}
\end{Verbatim}

Calculate the t statistic, setting the ddof parameter to the unbiased
value so the divisor in the standard deviation will be degrees of
freedom, N-1.

\begin{Verbatim}[commandchars=\\\{\}]
\PYG{g+gp}{\PYGZgt{}\PYGZgt{}\PYGZgt{} }\PYG{n}{t} \PYG{o}{=} \PYG{p}{(}\PYG{n}{np}\PYG{o}{.}\PYG{n}{mean}\PYG{p}{(}\PYG{n}{intake}\PYG{p}{)}\PYG{o}{\PYGZhy{}}\PYG{l+m+mi}{7725}\PYG{p}{)}\PYG{o}{/}\PYG{p}{(}\PYG{n}{intake}\PYG{o}{.}\PYG{n}{std}\PYG{p}{(}\PYG{n}{ddof}\PYG{o}{=}\PYG{l+m+mi}{1}\PYG{p}{)}\PYG{o}{/}\PYG{n}{np}\PYG{o}{.}\PYG{n}{sqrt}\PYG{p}{(}\PYG{n+nb}{len}\PYG{p}{(}\PYG{n}{intake}\PYG{p}{)}\PYG{p}{)}\PYG{p}{)}
\PYG{g+gp}{\PYGZgt{}\PYGZgt{}\PYGZgt{} }\PYG{k+kn}{import} \PYG{n+nn}{matplotlib.pyplot} \PYG{k+kn}{as} \PYG{n+nn}{plt}
\PYG{g+gp}{\PYGZgt{}\PYGZgt{}\PYGZgt{} }\PYG{n}{h} \PYG{o}{=} \PYG{n}{plt}\PYG{o}{.}\PYG{n}{hist}\PYG{p}{(}\PYG{n}{s}\PYG{p}{,} \PYG{n}{bins}\PYG{o}{=}\PYG{l+m+mi}{100}\PYG{p}{,} \PYG{n}{normed}\PYG{o}{=}\PYG{n+nb+bp}{True}\PYG{p}{)}
\end{Verbatim}

For a one-sided t-test, how far out in the distribution does the t
statistic appear?

\begin{Verbatim}[commandchars=\\\{\}]
\PYG{g+gp}{\PYGZgt{}\PYGZgt{}\PYGZgt{} }\PYG{o}{\PYGZgt{}\PYGZgt{}}\PYG{o}{\PYGZgt{}} \PYG{n}{np}\PYG{o}{.}\PYG{n}{sum}\PYG{p}{(}\PYG{n}{s}\PYG{o}{\PYGZlt{}}\PYG{n}{t}\PYG{p}{)} \PYG{o}{/} \PYG{n+nb}{float}\PYG{p}{(}\PYG{n+nb}{len}\PYG{p}{(}\PYG{n}{s}\PYG{p}{)}\PYG{p}{)}
\PYG{g+go}{0.0090699999999999999  \PYGZsh{}random}
\end{Verbatim}

So the p-value is about 0.009, which says the null hypothesis has a
probability of about 99\% of being true.

\end{fulllineitems}

\index{triangular() (in module lib.graph.raf)}

\begin{fulllineitems}
\phantomsection\label{lib.graph:lib.graph.raf.triangular}\pysiglinewithargsret{\code{lib.graph.raf.}\bfcode{triangular}}{\emph{left}, \emph{mode}, \emph{right}, \emph{size=None}}{}
Draw samples from the triangular distribution.

The triangular distribution is a continuous probability distribution with
lower limit left, peak at mode, and upper limit right. Unlike the other
distributions, these parameters directly define the shape of the pdf.
\begin{description}
\item[{left}] \leavevmode{[}scalar{]}
Lower limit.

\item[{mode}] \leavevmode{[}scalar{]}
The value where the peak of the distribution occurs.
The value should fulfill the condition \code{left \textless{}= mode \textless{}= right}.

\item[{right}] \leavevmode{[}scalar{]}
Upper limit, should be larger than \emph{left}.

\item[{size}] \leavevmode{[}int or tuple of ints, optional{]}
Output shape. Default is None, in which case a single value is
returned.

\end{description}
\begin{description}
\item[{samples}] \leavevmode{[}ndarray or scalar{]}
The returned samples all lie in the interval {[}left, right{]}.

\end{description}

The probability density function for the Triangular distribution is
\begin{gather}
\begin{split}P(x;l, m, r) = \begin{cases}
\frac{2(x-l)}{(r-l)(m-l)}& \text{for $l \leq x \leq m$},\\
\frac{2(m-x)}{(r-l)(r-m)}& \text{for $m \leq x \leq r$},\\
0& \text{otherwise}.
\end{cases}\end{split}\notag
\end{gather}
The triangular distribution is often used in ill-defined problems where the
underlying distribution is not known, but some knowledge of the limits and
mode exists. Often it is used in simulations.

Draw values from the distribution and plot the histogram:

\begin{Verbatim}[commandchars=\\\{\}]
\PYG{g+gp}{\PYGZgt{}\PYGZgt{}\PYGZgt{} }\PYG{k+kn}{import} \PYG{n+nn}{matplotlib.pyplot} \PYG{k+kn}{as} \PYG{n+nn}{plt}
\PYG{g+gp}{\PYGZgt{}\PYGZgt{}\PYGZgt{} }\PYG{n}{h} \PYG{o}{=} \PYG{n}{plt}\PYG{o}{.}\PYG{n}{hist}\PYG{p}{(}\PYG{n}{np}\PYG{o}{.}\PYG{n}{random}\PYG{o}{.}\PYG{n}{triangular}\PYG{p}{(}\PYG{o}{\PYGZhy{}}\PYG{l+m+mi}{3}\PYG{p}{,} \PYG{l+m+mi}{0}\PYG{p}{,} \PYG{l+m+mi}{8}\PYG{p}{,} \PYG{l+m+mi}{100000}\PYG{p}{)}\PYG{p}{,} \PYG{n}{bins}\PYG{o}{=}\PYG{l+m+mi}{200}\PYG{p}{,}
\PYG{g+gp}{... }             \PYG{n}{normed}\PYG{o}{=}\PYG{n+nb+bp}{True}\PYG{p}{)}
\PYG{g+gp}{\PYGZgt{}\PYGZgt{}\PYGZgt{} }\PYG{n}{plt}\PYG{o}{.}\PYG{n}{show}\PYG{p}{(}\PYG{p}{)}
\end{Verbatim}

\end{fulllineitems}

\index{uniform() (in module lib.graph.raf)}

\begin{fulllineitems}
\phantomsection\label{lib.graph:lib.graph.raf.uniform}\pysiglinewithargsret{\code{lib.graph.raf.}\bfcode{uniform}}{\emph{low=0.0}, \emph{high=1.0}, \emph{size=1}}{}
Draw samples from a uniform distribution.

Samples are uniformly distributed over the half-open interval
\code{{[}low, high)} (includes low, but excludes high).  In other words,
any value within the given interval is equally likely to be drawn
by \emph{uniform}.
\begin{description}
\item[{low}] \leavevmode{[}float, optional{]}
Lower boundary of the output interval.  All values generated will be
greater than or equal to low.  The default value is 0.

\item[{high}] \leavevmode{[}float{]}
Upper boundary of the output interval.  All values generated will be
less than high.  The default value is 1.0.

\item[{size}] \leavevmode{[}int or tuple of ints, optional{]}
Shape of output.  If the given size is, for example, (m,n,k),
m*n*k samples are generated.  If no shape is specified, a single sample
is returned.

\end{description}
\begin{description}
\item[{out}] \leavevmode{[}ndarray{]}
Drawn samples, with shape \emph{size}.

\end{description}

randint : Discrete uniform distribution, yielding integers.
random\_integers : Discrete uniform distribution over the closed
\begin{quote}

interval \code{{[}low, high{]}}.
\end{quote}

random\_sample : Floats uniformly distributed over \code{{[}0, 1)}.
random : Alias for \emph{random\_sample}.
rand : Convenience function that accepts dimensions as input, e.g.,
\begin{quote}

\code{rand(2,2)} would generate a 2-by-2 array of floats,
uniformly distributed over \code{{[}0, 1)}.
\end{quote}

The probability density function of the uniform distribution is
\begin{gather}
\begin{split}p(x) = \frac{1}{b - a}\end{split}\notag
\end{gather}
anywhere within the interval \code{{[}a, b)}, and zero elsewhere.

Draw samples from the distribution:

\begin{Verbatim}[commandchars=\\\{\}]
\PYG{g+gp}{\PYGZgt{}\PYGZgt{}\PYGZgt{} }\PYG{n}{s} \PYG{o}{=} \PYG{n}{np}\PYG{o}{.}\PYG{n}{random}\PYG{o}{.}\PYG{n}{uniform}\PYG{p}{(}\PYG{o}{\PYGZhy{}}\PYG{l+m+mi}{1}\PYG{p}{,}\PYG{l+m+mi}{0}\PYG{p}{,}\PYG{l+m+mi}{1000}\PYG{p}{)}
\end{Verbatim}

All values are within the given interval:

\begin{Verbatim}[commandchars=\\\{\}]
\PYG{g+gp}{\PYGZgt{}\PYGZgt{}\PYGZgt{} }\PYG{n}{np}\PYG{o}{.}\PYG{n}{all}\PYG{p}{(}\PYG{n}{s} \PYG{o}{\PYGZgt{}}\PYG{o}{=} \PYG{o}{\PYGZhy{}}\PYG{l+m+mi}{1}\PYG{p}{)}
\PYG{g+go}{True}
\PYG{g+gp}{\PYGZgt{}\PYGZgt{}\PYGZgt{} }\PYG{n}{np}\PYG{o}{.}\PYG{n}{all}\PYG{p}{(}\PYG{n}{s} \PYG{o}{\PYGZlt{}} \PYG{l+m+mi}{0}\PYG{p}{)}
\PYG{g+go}{True}
\end{Verbatim}

Display the histogram of the samples, along with the
probability density function:

\begin{Verbatim}[commandchars=\\\{\}]
\PYG{g+gp}{\PYGZgt{}\PYGZgt{}\PYGZgt{} }\PYG{k+kn}{import} \PYG{n+nn}{matplotlib.pyplot} \PYG{k+kn}{as} \PYG{n+nn}{plt}
\PYG{g+gp}{\PYGZgt{}\PYGZgt{}\PYGZgt{} }\PYG{n}{count}\PYG{p}{,} \PYG{n}{bins}\PYG{p}{,} \PYG{n}{ignored} \PYG{o}{=} \PYG{n}{plt}\PYG{o}{.}\PYG{n}{hist}\PYG{p}{(}\PYG{n}{s}\PYG{p}{,} \PYG{l+m+mi}{15}\PYG{p}{,} \PYG{n}{normed}\PYG{o}{=}\PYG{n+nb+bp}{True}\PYG{p}{)}
\PYG{g+gp}{\PYGZgt{}\PYGZgt{}\PYGZgt{} }\PYG{n}{plt}\PYG{o}{.}\PYG{n}{plot}\PYG{p}{(}\PYG{n}{bins}\PYG{p}{,} \PYG{n}{np}\PYG{o}{.}\PYG{n}{ones\PYGZus{}like}\PYG{p}{(}\PYG{n}{bins}\PYG{p}{)}\PYG{p}{,} \PYG{n}{linewidth}\PYG{o}{=}\PYG{l+m+mi}{2}\PYG{p}{,} \PYG{n}{color}\PYG{o}{=}\PYG{l+s}{\PYGZsq{}}\PYG{l+s}{r}\PYG{l+s}{\PYGZsq{}}\PYG{p}{)}
\PYG{g+gp}{\PYGZgt{}\PYGZgt{}\PYGZgt{} }\PYG{n}{plt}\PYG{o}{.}\PYG{n}{show}\PYG{p}{(}\PYG{p}{)}
\end{Verbatim}

\end{fulllineitems}

\index{vonmises() (in module lib.graph.raf)}

\begin{fulllineitems}
\phantomsection\label{lib.graph:lib.graph.raf.vonmises}\pysiglinewithargsret{\code{lib.graph.raf.}\bfcode{vonmises}}{\emph{mu}, \emph{kappa}, \emph{size=None}}{}
Draw samples from a von Mises distribution.

Samples are drawn from a von Mises distribution with specified mode
(mu) and dispersion (kappa), on the interval {[}-pi, pi{]}.

The von Mises distribution (also known as the circular normal
distribution) is a continuous probability distribution on the unit
circle.  It may be thought of as the circular analogue of the normal
distribution.
\begin{description}
\item[{mu}] \leavevmode{[}float{]}
Mode (``center'') of the distribution.

\item[{kappa}] \leavevmode{[}float{]}
Dispersion of the distribution, has to be \textgreater{}=0.

\item[{size}] \leavevmode{[}int or tuple of int{]}
Output shape.  If the given shape is, e.g., \code{(m, n, k)}, then
\code{m * n * k} samples are drawn.

\end{description}
\begin{description}
\item[{samples}] \leavevmode{[}scalar or ndarray{]}
The returned samples, which are in the interval {[}-pi, pi{]}.

\end{description}
\begin{description}
\item[{scipy.stats.distributions.vonmises}] \leavevmode{[}probability density function,{]}
distribution, or cumulative density function, etc.

\end{description}

The probability density for the von Mises distribution is
\begin{gather}
\begin{split}p(x) = \frac{e^{\kappa cos(x-\mu)}}{2\pi I_0(\kappa)},\end{split}\notag
\end{gather}
where \(\mu\) is the mode and \(\kappa\) the dispersion,
and \(I_0(\kappa)\) is the modified Bessel function of order 0.

The von Mises is named for Richard Edler von Mises, who was born in
Austria-Hungary, in what is now the Ukraine.  He fled to the United
States in 1939 and became a professor at Harvard.  He worked in
probability theory, aerodynamics, fluid mechanics, and philosophy of
science.

Abramowitz, M. and Stegun, I. A. (ed.), \emph{Handbook of Mathematical
Functions}, New York: Dover, 1965.

von Mises, R., \emph{Mathematical Theory of Probability and Statistics},
New York: Academic Press, 1964.

Draw samples from the distribution:

\begin{Verbatim}[commandchars=\\\{\}]
\PYG{g+gp}{\PYGZgt{}\PYGZgt{}\PYGZgt{} }\PYG{n}{mu}\PYG{p}{,} \PYG{n}{kappa} \PYG{o}{=} \PYG{l+m+mf}{0.0}\PYG{p}{,} \PYG{l+m+mf}{4.0} \PYG{c}{\PYGZsh{} mean and dispersion}
\PYG{g+gp}{\PYGZgt{}\PYGZgt{}\PYGZgt{} }\PYG{n}{s} \PYG{o}{=} \PYG{n}{np}\PYG{o}{.}\PYG{n}{random}\PYG{o}{.}\PYG{n}{vonmises}\PYG{p}{(}\PYG{n}{mu}\PYG{p}{,} \PYG{n}{kappa}\PYG{p}{,} \PYG{l+m+mi}{1000}\PYG{p}{)}
\end{Verbatim}

Display the histogram of the samples, along with
the probability density function:

\begin{Verbatim}[commandchars=\\\{\}]
\PYG{g+gp}{\PYGZgt{}\PYGZgt{}\PYGZgt{} }\PYG{k+kn}{import} \PYG{n+nn}{matplotlib.pyplot} \PYG{k+kn}{as} \PYG{n+nn}{plt}
\PYG{g+gp}{\PYGZgt{}\PYGZgt{}\PYGZgt{} }\PYG{k+kn}{import} \PYG{n+nn}{scipy.special} \PYG{k+kn}{as} \PYG{n+nn}{sps}
\PYG{g+gp}{\PYGZgt{}\PYGZgt{}\PYGZgt{} }\PYG{n}{count}\PYG{p}{,} \PYG{n}{bins}\PYG{p}{,} \PYG{n}{ignored} \PYG{o}{=} \PYG{n}{plt}\PYG{o}{.}\PYG{n}{hist}\PYG{p}{(}\PYG{n}{s}\PYG{p}{,} \PYG{l+m+mi}{50}\PYG{p}{,} \PYG{n}{normed}\PYG{o}{=}\PYG{n+nb+bp}{True}\PYG{p}{)}
\PYG{g+gp}{\PYGZgt{}\PYGZgt{}\PYGZgt{} }\PYG{n}{x} \PYG{o}{=} \PYG{n}{np}\PYG{o}{.}\PYG{n}{arange}\PYG{p}{(}\PYG{o}{\PYGZhy{}}\PYG{n}{np}\PYG{o}{.}\PYG{n}{pi}\PYG{p}{,} \PYG{n}{np}\PYG{o}{.}\PYG{n}{pi}\PYG{p}{,} \PYG{l+m+mi}{2}\PYG{o}{*}\PYG{n}{np}\PYG{o}{.}\PYG{n}{pi}\PYG{o}{/}\PYG{l+m+mf}{50.}\PYG{p}{)}
\PYG{g+gp}{\PYGZgt{}\PYGZgt{}\PYGZgt{} }\PYG{n}{y} \PYG{o}{=} \PYG{o}{\PYGZhy{}}\PYG{n}{np}\PYG{o}{.}\PYG{n}{exp}\PYG{p}{(}\PYG{n}{kappa}\PYG{o}{*}\PYG{n}{np}\PYG{o}{.}\PYG{n}{cos}\PYG{p}{(}\PYG{n}{x}\PYG{o}{\PYGZhy{}}\PYG{n}{mu}\PYG{p}{)}\PYG{p}{)}\PYG{o}{/}\PYG{p}{(}\PYG{l+m+mi}{2}\PYG{o}{*}\PYG{n}{np}\PYG{o}{.}\PYG{n}{pi}\PYG{o}{*}\PYG{n}{sps}\PYG{o}{.}\PYG{n}{jn}\PYG{p}{(}\PYG{l+m+mi}{0}\PYG{p}{,}\PYG{n}{kappa}\PYG{p}{)}\PYG{p}{)}
\PYG{g+gp}{\PYGZgt{}\PYGZgt{}\PYGZgt{} }\PYG{n}{plt}\PYG{o}{.}\PYG{n}{plot}\PYG{p}{(}\PYG{n}{x}\PYG{p}{,} \PYG{n}{y}\PYG{o}{/}\PYG{n+nb}{max}\PYG{p}{(}\PYG{n}{y}\PYG{p}{)}\PYG{p}{,} \PYG{n}{linewidth}\PYG{o}{=}\PYG{l+m+mi}{2}\PYG{p}{,} \PYG{n}{color}\PYG{o}{=}\PYG{l+s}{\PYGZsq{}}\PYG{l+s}{r}\PYG{l+s}{\PYGZsq{}}\PYG{p}{)}
\PYG{g+gp}{\PYGZgt{}\PYGZgt{}\PYGZgt{} }\PYG{n}{plt}\PYG{o}{.}\PYG{n}{show}\PYG{p}{(}\PYG{p}{)}
\end{Verbatim}

\end{fulllineitems}

\index{wald() (in module lib.graph.raf)}

\begin{fulllineitems}
\phantomsection\label{lib.graph:lib.graph.raf.wald}\pysiglinewithargsret{\code{lib.graph.raf.}\bfcode{wald}}{\emph{mean}, \emph{scale}, \emph{size=None}}{}
Draw samples from a Wald, or Inverse Gaussian, distribution.

As the scale approaches infinity, the distribution becomes more like a
Gaussian.

Some references claim that the Wald is an Inverse Gaussian with mean=1, but
this is by no means universal.

The Inverse Gaussian distribution was first studied in relationship to
Brownian motion. In 1956 M.C.K. Tweedie used the name Inverse Gaussian
because there is an inverse relationship between the time to cover a unit
distance and distance covered in unit time.
\begin{description}
\item[{mean}] \leavevmode{[}scalar{]}
Distribution mean, should be \textgreater{} 0.

\item[{scale}] \leavevmode{[}scalar{]}
Scale parameter, should be \textgreater{}= 0.

\item[{size}] \leavevmode{[}int or tuple of ints, optional{]}
Output shape. Default is None, in which case a single value is
returned.

\end{description}
\begin{description}
\item[{samples}] \leavevmode{[}ndarray or scalar{]}
Drawn sample, all greater than zero.

\end{description}

The probability density function for the Wald distribution is
\begin{gather}
\begin{split}P(x;mean,scale) = \sqrt{\frac{scale}{2\pi x^3}}e^
\frac{-scale(x-mean)^2}{2\cdotp mean^2x}\end{split}\notag
\end{gather}
As noted above the Inverse Gaussian distribution first arise from attempts
to model Brownian Motion. It is also a competitor to the Weibull for use in
reliability modeling and modeling stock returns and interest rate
processes.

Draw values from the distribution and plot the histogram:

\begin{Verbatim}[commandchars=\\\{\}]
\PYG{g+gp}{\PYGZgt{}\PYGZgt{}\PYGZgt{} }\PYG{k+kn}{import} \PYG{n+nn}{matplotlib.pyplot} \PYG{k+kn}{as} \PYG{n+nn}{plt}
\PYG{g+gp}{\PYGZgt{}\PYGZgt{}\PYGZgt{} }\PYG{n}{h} \PYG{o}{=} \PYG{n}{plt}\PYG{o}{.}\PYG{n}{hist}\PYG{p}{(}\PYG{n}{np}\PYG{o}{.}\PYG{n}{random}\PYG{o}{.}\PYG{n}{wald}\PYG{p}{(}\PYG{l+m+mi}{3}\PYG{p}{,} \PYG{l+m+mi}{2}\PYG{p}{,} \PYG{l+m+mi}{100000}\PYG{p}{)}\PYG{p}{,} \PYG{n}{bins}\PYG{o}{=}\PYG{l+m+mi}{200}\PYG{p}{,} \PYG{n}{normed}\PYG{o}{=}\PYG{n+nb+bp}{True}\PYG{p}{)}
\PYG{g+gp}{\PYGZgt{}\PYGZgt{}\PYGZgt{} }\PYG{n}{plt}\PYG{o}{.}\PYG{n}{show}\PYG{p}{(}\PYG{p}{)}
\end{Verbatim}

\end{fulllineitems}

\index{weibull() (in module lib.graph.raf)}

\begin{fulllineitems}
\phantomsection\label{lib.graph:lib.graph.raf.weibull}\pysiglinewithargsret{\code{lib.graph.raf.}\bfcode{weibull}}{\emph{a}, \emph{size=None}}{}
Weibull distribution.

Draw samples from a 1-parameter Weibull distribution with the given
shape parameter \emph{a}.
\begin{gather}
\begin{split}X = (-ln(U))^{1/a}\end{split}\notag
\end{gather}
Here, U is drawn from the uniform distribution over (0,1{]}.

The more common 2-parameter Weibull, including a scale parameter
\(\lambda\) is just \(X = \lambda(-ln(U))^{1/a}\).
\begin{description}
\item[{a}] \leavevmode{[}float{]}
Shape of the distribution.

\item[{size}] \leavevmode{[}tuple of ints{]}
Output shape.  If the given shape is, e.g., \code{(m, n, k)}, then
\code{m * n * k} samples are drawn.

\end{description}

scipy.stats.distributions.weibull\_max
scipy.stats.distributions.weibull\_min
scipy.stats.distributions.genextreme
gumbel

The Weibull (or Type III asymptotic extreme value distribution for smallest
values, SEV Type III, or Rosin-Rammler distribution) is one of a class of
Generalized Extreme Value (GEV) distributions used in modeling extreme
value problems.  This class includes the Gumbel and Frechet distributions.

The probability density for the Weibull distribution is
\begin{gather}
\begin{split}p(x) = \frac{a}
{\lambda}(\frac{x}{\lambda})^{a-1}e^{-(x/\lambda)^a},\end{split}\notag
\end{gather}
where \(a\) is the shape and \(\lambda\) the scale.

The function has its peak (the mode) at
\(\lambda(\frac{a-1}{a})^{1/a}\).

When \code{a = 1}, the Weibull distribution reduces to the exponential
distribution.

Draw samples from the distribution:

\begin{Verbatim}[commandchars=\\\{\}]
\PYG{g+gp}{\PYGZgt{}\PYGZgt{}\PYGZgt{} }\PYG{n}{a} \PYG{o}{=} \PYG{l+m+mf}{5.} \PYG{c}{\PYGZsh{} shape}
\PYG{g+gp}{\PYGZgt{}\PYGZgt{}\PYGZgt{} }\PYG{n}{s} \PYG{o}{=} \PYG{n}{np}\PYG{o}{.}\PYG{n}{random}\PYG{o}{.}\PYG{n}{weibull}\PYG{p}{(}\PYG{n}{a}\PYG{p}{,} \PYG{l+m+mi}{1000}\PYG{p}{)}
\end{Verbatim}

Display the histogram of the samples, along with
the probability density function:

\begin{Verbatim}[commandchars=\\\{\}]
\PYG{g+gp}{\PYGZgt{}\PYGZgt{}\PYGZgt{} }\PYG{k+kn}{import} \PYG{n+nn}{matplotlib.pyplot} \PYG{k+kn}{as} \PYG{n+nn}{plt}
\PYG{g+gp}{\PYGZgt{}\PYGZgt{}\PYGZgt{} }\PYG{n}{x} \PYG{o}{=} \PYG{n}{np}\PYG{o}{.}\PYG{n}{arange}\PYG{p}{(}\PYG{l+m+mi}{1}\PYG{p}{,}\PYG{l+m+mf}{100.}\PYG{p}{)}\PYG{o}{/}\PYG{l+m+mf}{50.}
\PYG{g+gp}{\PYGZgt{}\PYGZgt{}\PYGZgt{} }\PYG{k}{def} \PYG{n+nf}{weib}\PYG{p}{(}\PYG{n}{x}\PYG{p}{,}\PYG{n}{n}\PYG{p}{,}\PYG{n}{a}\PYG{p}{)}\PYG{p}{:}
\PYG{g+gp}{... }    \PYG{k}{return} \PYG{p}{(}\PYG{n}{a} \PYG{o}{/} \PYG{n}{n}\PYG{p}{)} \PYG{o}{*} \PYG{p}{(}\PYG{n}{x} \PYG{o}{/} \PYG{n}{n}\PYG{p}{)}\PYG{o}{*}\PYG{o}{*}\PYG{p}{(}\PYG{n}{a} \PYG{o}{\PYGZhy{}} \PYG{l+m+mi}{1}\PYG{p}{)} \PYG{o}{*} \PYG{n}{np}\PYG{o}{.}\PYG{n}{exp}\PYG{p}{(}\PYG{o}{\PYGZhy{}}\PYG{p}{(}\PYG{n}{x} \PYG{o}{/} \PYG{n}{n}\PYG{p}{)}\PYG{o}{*}\PYG{o}{*}\PYG{n}{a}\PYG{p}{)}
\end{Verbatim}

\begin{Verbatim}[commandchars=\\\{\}]
\PYG{g+gp}{\PYGZgt{}\PYGZgt{}\PYGZgt{} }\PYG{n}{count}\PYG{p}{,} \PYG{n}{bins}\PYG{p}{,} \PYG{n}{ignored} \PYG{o}{=} \PYG{n}{plt}\PYG{o}{.}\PYG{n}{hist}\PYG{p}{(}\PYG{n}{np}\PYG{o}{.}\PYG{n}{random}\PYG{o}{.}\PYG{n}{weibull}\PYG{p}{(}\PYG{l+m+mf}{5.}\PYG{p}{,}\PYG{l+m+mi}{1000}\PYG{p}{)}\PYG{p}{)}
\PYG{g+gp}{\PYGZgt{}\PYGZgt{}\PYGZgt{} }\PYG{n}{x} \PYG{o}{=} \PYG{n}{np}\PYG{o}{.}\PYG{n}{arange}\PYG{p}{(}\PYG{l+m+mi}{1}\PYG{p}{,}\PYG{l+m+mf}{100.}\PYG{p}{)}\PYG{o}{/}\PYG{l+m+mf}{50.}
\PYG{g+gp}{\PYGZgt{}\PYGZgt{}\PYGZgt{} }\PYG{n}{scale} \PYG{o}{=} \PYG{n}{count}\PYG{o}{.}\PYG{n}{max}\PYG{p}{(}\PYG{p}{)}\PYG{o}{/}\PYG{n}{weib}\PYG{p}{(}\PYG{n}{x}\PYG{p}{,} \PYG{l+m+mf}{1.}\PYG{p}{,} \PYG{l+m+mf}{5.}\PYG{p}{)}\PYG{o}{.}\PYG{n}{max}\PYG{p}{(}\PYG{p}{)}
\PYG{g+gp}{\PYGZgt{}\PYGZgt{}\PYGZgt{} }\PYG{n}{plt}\PYG{o}{.}\PYG{n}{plot}\PYG{p}{(}\PYG{n}{x}\PYG{p}{,} \PYG{n}{weib}\PYG{p}{(}\PYG{n}{x}\PYG{p}{,} \PYG{l+m+mf}{1.}\PYG{p}{,} \PYG{l+m+mf}{5.}\PYG{p}{)}\PYG{o}{*}\PYG{n}{scale}\PYG{p}{)}
\PYG{g+gp}{\PYGZgt{}\PYGZgt{}\PYGZgt{} }\PYG{n}{plt}\PYG{o}{.}\PYG{n}{show}\PYG{p}{(}\PYG{p}{)}
\end{Verbatim}

\end{fulllineitems}

\index{zipf() (in module lib.graph.raf)}

\begin{fulllineitems}
\phantomsection\label{lib.graph:lib.graph.raf.zipf}\pysiglinewithargsret{\code{lib.graph.raf.}\bfcode{zipf}}{\emph{a}, \emph{size=None}}{}
Draw samples from a Zipf distribution.

Samples are drawn from a Zipf distribution with specified parameter
\emph{a} \textgreater{} 1.

The Zipf distribution (also known as the zeta distribution) is a
continuous probability distribution that satisfies Zipf's law: the
frequency of an item is inversely proportional to its rank in a
frequency table.
\begin{description}
\item[{a}] \leavevmode{[}float \textgreater{} 1{]}
Distribution parameter.

\item[{size}] \leavevmode{[}int or tuple of int, optional{]}
Output shape.  If the given shape is, e.g., \code{(m, n, k)}, then
\code{m * n * k} samples are drawn; a single integer is equivalent in
its result to providing a mono-tuple, i.e., a 1-D array of length
\emph{size} is returned.  The default is None, in which case a single
scalar is returned.

\end{description}
\begin{description}
\item[{samples}] \leavevmode{[}scalar or ndarray{]}
The returned samples are greater than or equal to one.

\end{description}
\begin{description}
\item[{scipy.stats.distributions.zipf}] \leavevmode{[}probability density function,{]}
distribution, or cumulative density function, etc.

\end{description}

The probability density for the Zipf distribution is
\begin{gather}
\begin{split}p(x) = \frac{x^{-a}}{\zeta(a)},\end{split}\notag
\end{gather}
where \(\zeta\) is the Riemann Zeta function.

It is named for the American linguist George Kingsley Zipf, who noted
that the frequency of any word in a sample of a language is inversely
proportional to its rank in the frequency table.

Zipf, G. K., \emph{Selected Studies of the Principle of Relative Frequency
in Language}, Cambridge, MA: Harvard Univ. Press, 1932.

Draw samples from the distribution:

\begin{Verbatim}[commandchars=\\\{\}]
\PYG{g+gp}{\PYGZgt{}\PYGZgt{}\PYGZgt{} }\PYG{n}{a} \PYG{o}{=} \PYG{l+m+mf}{2.} \PYG{c}{\PYGZsh{} parameter}
\PYG{g+gp}{\PYGZgt{}\PYGZgt{}\PYGZgt{} }\PYG{n}{s} \PYG{o}{=} \PYG{n}{np}\PYG{o}{.}\PYG{n}{random}\PYG{o}{.}\PYG{n}{zipf}\PYG{p}{(}\PYG{n}{a}\PYG{p}{,} \PYG{l+m+mi}{1000}\PYG{p}{)}
\end{Verbatim}

Display the histogram of the samples, along with
the probability density function:

\begin{Verbatim}[commandchars=\\\{\}]
\PYG{g+gp}{\PYGZgt{}\PYGZgt{}\PYGZgt{} }\PYG{k+kn}{import} \PYG{n+nn}{matplotlib.pyplot} \PYG{k+kn}{as} \PYG{n+nn}{plt}
\PYG{g+gp}{\PYGZgt{}\PYGZgt{}\PYGZgt{} }\PYG{k+kn}{import} \PYG{n+nn}{scipy.special} \PYG{k+kn}{as} \PYG{n+nn}{sps}
\PYG{g+go}{Truncate s values at 50 so plot is interesting}
\PYG{g+gp}{\PYGZgt{}\PYGZgt{}\PYGZgt{} }\PYG{n}{count}\PYG{p}{,} \PYG{n}{bins}\PYG{p}{,} \PYG{n}{ignored} \PYG{o}{=} \PYG{n}{plt}\PYG{o}{.}\PYG{n}{hist}\PYG{p}{(}\PYG{n}{s}\PYG{p}{[}\PYG{n}{s}\PYG{o}{\PYGZlt{}}\PYG{l+m+mi}{50}\PYG{p}{]}\PYG{p}{,} \PYG{l+m+mi}{50}\PYG{p}{,} \PYG{n}{normed}\PYG{o}{=}\PYG{n+nb+bp}{True}\PYG{p}{)}
\PYG{g+gp}{\PYGZgt{}\PYGZgt{}\PYGZgt{} }\PYG{n}{x} \PYG{o}{=} \PYG{n}{np}\PYG{o}{.}\PYG{n}{arange}\PYG{p}{(}\PYG{l+m+mf}{1.}\PYG{p}{,} \PYG{l+m+mf}{50.}\PYG{p}{)}
\PYG{g+gp}{\PYGZgt{}\PYGZgt{}\PYGZgt{} }\PYG{n}{y} \PYG{o}{=} \PYG{n}{x}\PYG{o}{*}\PYG{o}{*}\PYG{p}{(}\PYG{o}{\PYGZhy{}}\PYG{n}{a}\PYG{p}{)}\PYG{o}{/}\PYG{n}{sps}\PYG{o}{.}\PYG{n}{zetac}\PYG{p}{(}\PYG{n}{a}\PYG{p}{)}
\PYG{g+gp}{\PYGZgt{}\PYGZgt{}\PYGZgt{} }\PYG{n}{plt}\PYG{o}{.}\PYG{n}{plot}\PYG{p}{(}\PYG{n}{x}\PYG{p}{,} \PYG{n}{y}\PYG{o}{/}\PYG{n+nb}{max}\PYG{p}{(}\PYG{n}{y}\PYG{p}{)}\PYG{p}{,} \PYG{n}{linewidth}\PYG{o}{=}\PYG{l+m+mi}{2}\PYG{p}{,} \PYG{n}{color}\PYG{o}{=}\PYG{l+s}{\PYGZsq{}}\PYG{l+s}{r}\PYG{l+s}{\PYGZsq{}}\PYG{p}{)}
\PYG{g+gp}{\PYGZgt{}\PYGZgt{}\PYGZgt{} }\PYG{n}{plt}\PYG{o}{.}\PYG{n}{show}\PYG{p}{(}\PYG{p}{)}
\end{Verbatim}

\end{fulllineitems}



\subsubsection{\texttt{scc} Module}
\label{lib.graph:module-lib.graph.scc}\label{lib.graph:scc-module}\index{lib.graph.scc (module)}\index{beta() (in module lib.graph.scc)}

\begin{fulllineitems}
\phantomsection\label{lib.graph:lib.graph.scc.beta}\pysiglinewithargsret{\code{lib.graph.scc.}\bfcode{beta}}{\emph{a}, \emph{b}, \emph{size=None}}{}
The Beta distribution over \code{{[}0, 1{]}}.

The Beta distribution is a special case of the Dirichlet distribution,
and is related to the Gamma distribution.  It has the probability
distribution function
\begin{gather}
\begin{split}f(x; a,b) = \frac{1}{B(\alpha, \beta)} x^{\alpha - 1}
(1 - x)^{\beta - 1},\end{split}\notag
\end{gather}
where the normalisation, B, is the beta function,
\begin{gather}
\begin{split}B(\alpha, \beta) = \int_0^1 t^{\alpha - 1}
(1 - t)^{\beta - 1} dt.\end{split}\notag
\end{gather}
It is often seen in Bayesian inference and order statistics.
\begin{description}
\item[{a}] \leavevmode{[}float{]}
Alpha, non-negative.

\item[{b}] \leavevmode{[}float{]}
Beta, non-negative.

\item[{size}] \leavevmode{[}tuple of ints, optional{]}
The number of samples to draw.  The output is packed according to
the size given.

\end{description}
\begin{description}
\item[{out}] \leavevmode{[}ndarray{]}
Array of the given shape, containing values drawn from a
Beta distribution.

\end{description}

\end{fulllineitems}

\index{binomial() (in module lib.graph.scc)}

\begin{fulllineitems}
\phantomsection\label{lib.graph:lib.graph.scc.binomial}\pysiglinewithargsret{\code{lib.graph.scc.}\bfcode{binomial}}{\emph{n}, \emph{p}, \emph{size=None}}{}
Draw samples from a binomial distribution.

Samples are drawn from a Binomial distribution with specified
parameters, n trials and p probability of success where
n an integer \textgreater{}= 0 and p is in the interval {[}0,1{]}. (n may be
input as a float, but it is truncated to an integer in use)
\begin{description}
\item[{n}] \leavevmode{[}float (but truncated to an integer){]}
parameter, \textgreater{}= 0.

\item[{p}] \leavevmode{[}float{]}
parameter, \textgreater{}= 0 and \textless{}=1.

\item[{size}] \leavevmode{[}\{tuple, int\}{]}
Output shape.  If the given shape is, e.g., \code{(m, n, k)}, then
\code{m * n * k} samples are drawn.

\end{description}
\begin{description}
\item[{samples}] \leavevmode{[}\{ndarray, scalar\}{]}
where the values are all integers in  {[}0, n{]}.

\end{description}
\begin{description}
\item[{scipy.stats.distributions.binom}] \leavevmode{[}probability density function,{]}
distribution or cumulative density function, etc.

\end{description}

The probability density for the Binomial distribution is
\begin{gather}
\begin{split}P(N) = \binom{n}{N}p^N(1-p)^{n-N},\end{split}\notag
\end{gather}
where \(n\) is the number of trials, \(p\) is the probability
of success, and \(N\) is the number of successes.

When estimating the standard error of a proportion in a population by
using a random sample, the normal distribution works well unless the
product p*n \textless{}=5, where p = population proportion estimate, and n =
number of samples, in which case the binomial distribution is used
instead. For example, a sample of 15 people shows 4 who are left
handed, and 11 who are right handed. Then p = 4/15 = 27\%. 0.27*15 = 4,
so the binomial distribution should be used in this case.

Draw samples from the distribution:

\begin{Verbatim}[commandchars=\\\{\}]
\PYG{g+gp}{\PYGZgt{}\PYGZgt{}\PYGZgt{} }\PYG{n}{n}\PYG{p}{,} \PYG{n}{p} \PYG{o}{=} \PYG{l+m+mi}{10}\PYG{p}{,} \PYG{o}{.}\PYG{l+m+mi}{5} \PYG{c}{\PYGZsh{} number of trials, probability of each trial}
\PYG{g+gp}{\PYGZgt{}\PYGZgt{}\PYGZgt{} }\PYG{n}{s} \PYG{o}{=} \PYG{n}{np}\PYG{o}{.}\PYG{n}{random}\PYG{o}{.}\PYG{n}{binomial}\PYG{p}{(}\PYG{n}{n}\PYG{p}{,} \PYG{n}{p}\PYG{p}{,} \PYG{l+m+mi}{1000}\PYG{p}{)}
\PYG{g+go}{\PYGZsh{} result of flipping a coin 10 times, tested 1000 times.}
\end{Verbatim}

A real world example. A company drills 9 wild-cat oil exploration
wells, each with an estimated probability of success of 0.1. All nine
wells fail. What is the probability of that happening?

Let's do 20,000 trials of the model, and count the number that
generate zero positive results.

\begin{Verbatim}[commandchars=\\\{\}]
\PYG{g+gp}{\PYGZgt{}\PYGZgt{}\PYGZgt{} }\PYG{n+nb}{sum}\PYG{p}{(}\PYG{n}{np}\PYG{o}{.}\PYG{n}{random}\PYG{o}{.}\PYG{n}{binomial}\PYG{p}{(}\PYG{l+m+mi}{9}\PYG{p}{,}\PYG{l+m+mf}{0.1}\PYG{p}{,}\PYG{l+m+mi}{20000}\PYG{p}{)}\PYG{o}{==}\PYG{l+m+mi}{0}\PYG{p}{)}\PYG{o}{/}\PYG{l+m+mf}{20000.}
\PYG{g+go}{answer = 0.38885, or 38\PYGZpc{}.}
\end{Verbatim}

\end{fulllineitems}

\index{checkMinimalSCCdimension() (in module lib.graph.scc)}

\begin{fulllineitems}
\phantomsection\label{lib.graph:lib.graph.scc.checkMinimalSCCdimension}\pysiglinewithargsret{\code{lib.graph.scc.}\bfcode{checkMinimalSCCdimension}}{\emph{tmpDig}, \emph{tmpMinDim}}{}
\end{fulllineitems}

\index{chisquare() (in module lib.graph.scc)}

\begin{fulllineitems}
\phantomsection\label{lib.graph:lib.graph.scc.chisquare}\pysiglinewithargsret{\code{lib.graph.scc.}\bfcode{chisquare}}{\emph{df}, \emph{size=None}}{}
Draw samples from a chi-square distribution.

When \emph{df} independent random variables, each with standard normal
distributions (mean 0, variance 1), are squared and summed, the
resulting distribution is chi-square (see Notes).  This distribution
is often used in hypothesis testing.
\begin{description}
\item[{df}] \leavevmode{[}int{]}
Number of degrees of freedom.

\item[{size}] \leavevmode{[}tuple of ints, int, optional{]}
Size of the returned array.  By default, a scalar is
returned.

\end{description}
\begin{description}
\item[{output}] \leavevmode{[}ndarray{]}
Samples drawn from the distribution, packed in a \emph{size}-shaped
array.

\end{description}
\begin{description}
\item[{ValueError}] \leavevmode
When \emph{df} \textless{}= 0 or when an inappropriate \emph{size} (e.g. \code{size=-1})
is given.

\end{description}

The variable obtained by summing the squares of \emph{df} independent,
standard normally distributed random variables:
\begin{gather}
\begin{split}Q = \sum_{i=0}^{\mathtt{df}} X^2_i\end{split}\notag
\end{gather}
is chi-square distributed, denoted
\begin{gather}
\begin{split}Q \sim \chi^2_k.\end{split}\notag
\end{gather}
The probability density function of the chi-squared distribution is
\begin{gather}
\begin{split}p(x) = \frac{(1/2)^{k/2}}{\Gamma(k/2)}
x^{k/2 - 1} e^{-x/2},\end{split}\notag
\end{gather}
where \(\Gamma\) is the gamma function,
\begin{gather}
\begin{split}\Gamma(x) = \int_0^{-\infty} t^{x - 1} e^{-t} dt.\end{split}\notag
\end{gather}
\href{http://www.itl.nist.gov/div898/handbook/eda/section3/eda3666.htm}{NIST/SEMATECH e-Handbook of Statistical Methods}

\begin{Verbatim}[commandchars=\\\{\}]
\PYG{g+gp}{\PYGZgt{}\PYGZgt{}\PYGZgt{} }\PYG{n}{np}\PYG{o}{.}\PYG{n}{random}\PYG{o}{.}\PYG{n}{chisquare}\PYG{p}{(}\PYG{l+m+mi}{2}\PYG{p}{,}\PYG{l+m+mi}{4}\PYG{p}{)}
\PYG{g+go}{array([ 1.89920014,  9.00867716,  3.13710533,  5.62318272])}
\end{Verbatim}

\end{fulllineitems}

\index{createNetXGraph() (in module lib.graph.scc)}

\begin{fulllineitems}
\phantomsection\label{lib.graph:lib.graph.scc.createNetXGraph}\pysiglinewithargsret{\code{lib.graph.scc.}\bfcode{createNetXGraph}}{\emph{tmpCstr}, \emph{tmpCats}}{}
{\color{red}\bfseries{}\textbar{}}- Cat -\textgreater{} Prod graph creation...

\end{fulllineitems}

\index{createNetXGraphForRAF() (in module lib.graph.scc)}

\begin{fulllineitems}
\phantomsection\label{lib.graph:lib.graph.scc.createNetXGraphForRAF}\pysiglinewithargsret{\code{lib.graph.scc.}\bfcode{createNetXGraphForRAF}}{\emph{tmpCstr}, \emph{tmpClosure}, \emph{tmpCats}}{}
{\color{red}\bfseries{}\textbar{}}- Cat -\textgreater{} Prod graph creation...

\end{fulllineitems}

\index{createSimpleGraph() (in module lib.graph.scc)}

\begin{fulllineitems}
\phantomsection\label{lib.graph:lib.graph.scc.createSimpleGraph}\pysiglinewithargsret{\code{lib.graph.scc.}\bfcode{createSimpleGraph}}{\emph{tmpCstr}, \emph{tmpCats}}{}
{\color{red}\bfseries{}\textbar{}}- Cat -\textgreater{} Prod graph creation...

\end{fulllineitems}

\index{diGraph\_netX\_stats() (in module lib.graph.scc)}

\begin{fulllineitems}
\phantomsection\label{lib.graph:lib.graph.scc.diGraph_netX_stats}\pysiglinewithargsret{\code{lib.graph.scc.}\bfcode{diGraph\_netX\_stats}}{\emph{tmpDig}}{}
\end{fulllineitems}

\index{exponential() (in module lib.graph.scc)}

\begin{fulllineitems}
\phantomsection\label{lib.graph:lib.graph.scc.exponential}\pysiglinewithargsret{\code{lib.graph.scc.}\bfcode{exponential}}{\emph{scale=1.0}, \emph{size=None}}{}
Exponential distribution.

Its probability density function is
\begin{gather}
\begin{split}f(x; \frac{1}{\beta}) = \frac{1}{\beta} \exp(-\frac{x}{\beta}),\end{split}\notag
\end{gather}
for \code{x \textgreater{} 0} and 0 elsewhere. \(\beta\) is the scale parameter,
which is the inverse of the rate parameter \(\lambda = 1/\beta\).
The rate parameter is an alternative, widely used parameterization
of the exponential distribution {\color{red}\bfseries{}{[}3{]}\_}.

The exponential distribution is a continuous analogue of the
geometric distribution.  It describes many common situations, such as
the size of raindrops measured over many rainstorms {\color{red}\bfseries{}{[}1{]}\_}, or the time
between page requests to Wikipedia {\color{red}\bfseries{}{[}2{]}\_}.
\begin{description}
\item[{scale}] \leavevmode{[}float{]}
The scale parameter, \(\beta = 1/\lambda\).

\item[{size}] \leavevmode{[}tuple of ints{]}
Number of samples to draw.  The output is shaped
according to \emph{size}.

\end{description}

\end{fulllineitems}

\index{f() (in module lib.graph.scc)}

\begin{fulllineitems}
\phantomsection\label{lib.graph:lib.graph.scc.f}\pysiglinewithargsret{\code{lib.graph.scc.}\bfcode{f}}{\emph{dfnum}, \emph{dfden}, \emph{size=None}}{}
Draw samples from a F distribution.

Samples are drawn from an F distribution with specified parameters,
\emph{dfnum} (degrees of freedom in numerator) and \emph{dfden} (degrees of freedom
in denominator), where both parameters should be greater than zero.

The random variate of the F distribution (also known as the
Fisher distribution) is a continuous probability distribution
that arises in ANOVA tests, and is the ratio of two chi-square
variates.
\begin{description}
\item[{dfnum}] \leavevmode{[}float{]}
Degrees of freedom in numerator. Should be greater than zero.

\item[{dfden}] \leavevmode{[}float{]}
Degrees of freedom in denominator. Should be greater than zero.

\item[{size}] \leavevmode{[}\{tuple, int\}, optional{]}
Output shape.  If the given shape is, e.g., \code{(m, n, k)},
then \code{m * n * k} samples are drawn. By default only one sample
is returned.

\end{description}
\begin{description}
\item[{samples}] \leavevmode{[}\{ndarray, scalar\}{]}
Samples from the Fisher distribution.

\end{description}
\begin{description}
\item[{scipy.stats.distributions.f}] \leavevmode{[}probability density function,{]}
distribution or cumulative density function, etc.

\end{description}

The F statistic is used to compare in-group variances to between-group
variances. Calculating the distribution depends on the sampling, and
so it is a function of the respective degrees of freedom in the
problem.  The variable \emph{dfnum} is the number of samples minus one, the
between-groups degrees of freedom, while \emph{dfden} is the within-groups
degrees of freedom, the sum of the number of samples in each group
minus the number of groups.

An example from Glantz{[}1{]}, pp 47-40.
Two groups, children of diabetics (25 people) and children from people
without diabetes (25 controls). Fasting blood glucose was measured,
case group had a mean value of 86.1, controls had a mean value of
82.2. Standard deviations were 2.09 and 2.49 respectively. Are these
data consistent with the null hypothesis that the parents diabetic
status does not affect their children's blood glucose levels?
Calculating the F statistic from the data gives a value of 36.01.

Draw samples from the distribution:

\begin{Verbatim}[commandchars=\\\{\}]
\PYG{g+gp}{\PYGZgt{}\PYGZgt{}\PYGZgt{} }\PYG{n}{dfnum} \PYG{o}{=} \PYG{l+m+mf}{1.} \PYG{c}{\PYGZsh{} between group degrees of freedom}
\PYG{g+gp}{\PYGZgt{}\PYGZgt{}\PYGZgt{} }\PYG{n}{dfden} \PYG{o}{=} \PYG{l+m+mf}{48.} \PYG{c}{\PYGZsh{} within groups degrees of freedom}
\PYG{g+gp}{\PYGZgt{}\PYGZgt{}\PYGZgt{} }\PYG{n}{s} \PYG{o}{=} \PYG{n}{np}\PYG{o}{.}\PYG{n}{random}\PYG{o}{.}\PYG{n}{f}\PYG{p}{(}\PYG{n}{dfnum}\PYG{p}{,} \PYG{n}{dfden}\PYG{p}{,} \PYG{l+m+mi}{1000}\PYG{p}{)}
\end{Verbatim}

The lower bound for the top 1\% of the samples is :

\begin{Verbatim}[commandchars=\\\{\}]
\PYG{g+gp}{\PYGZgt{}\PYGZgt{}\PYGZgt{} }\PYG{n}{sort}\PYG{p}{(}\PYG{n}{s}\PYG{p}{)}\PYG{p}{[}\PYG{o}{\PYGZhy{}}\PYG{l+m+mi}{10}\PYG{p}{]}
\PYG{g+go}{7.61988120985}
\end{Verbatim}

So there is about a 1\% chance that the F statistic will exceed 7.62,
the measured value is 36, so the null hypothesis is rejected at the 1\%
level.

\end{fulllineitems}

\index{gamma() (in module lib.graph.scc)}

\begin{fulllineitems}
\phantomsection\label{lib.graph:lib.graph.scc.gamma}\pysiglinewithargsret{\code{lib.graph.scc.}\bfcode{gamma}}{\emph{shape}, \emph{scale=1.0}, \emph{size=None}}{}
Draw samples from a Gamma distribution.

Samples are drawn from a Gamma distribution with specified parameters,
\emph{shape} (sometimes designated ``k'') and \emph{scale} (sometimes designated
``theta''), where both parameters are \textgreater{} 0.
\begin{description}
\item[{shape}] \leavevmode{[}scalar \textgreater{} 0{]}
The shape of the gamma distribution.

\item[{scale}] \leavevmode{[}scalar \textgreater{} 0, optional{]}
The scale of the gamma distribution.  Default is equal to 1.

\item[{size}] \leavevmode{[}shape\_tuple, optional{]}
Output shape.  If the given shape is, e.g., \code{(m, n, k)}, then
\code{m * n * k} samples are drawn.

\end{description}
\begin{description}
\item[{out}] \leavevmode{[}ndarray, float{]}
Returns one sample unless \emph{size} parameter is specified.

\end{description}
\begin{description}
\item[{scipy.stats.distributions.gamma}] \leavevmode{[}probability density function,{]}
distribution or cumulative density function, etc.

\end{description}

The probability density for the Gamma distribution is
\begin{gather}
\begin{split}p(x) = x^{k-1}\frac{e^{-x/\theta}}{\theta^k\Gamma(k)},\end{split}\notag
\end{gather}
where \(k\) is the shape and \(\theta\) the scale,
and \(\Gamma\) is the Gamma function.

The Gamma distribution is often used to model the times to failure of
electronic components, and arises naturally in processes for which the
waiting times between Poisson distributed events are relevant.

Draw samples from the distribution:

\begin{Verbatim}[commandchars=\\\{\}]
\PYG{g+gp}{\PYGZgt{}\PYGZgt{}\PYGZgt{} }\PYG{n}{shape}\PYG{p}{,} \PYG{n}{scale} \PYG{o}{=} \PYG{l+m+mf}{2.}\PYG{p}{,} \PYG{l+m+mf}{2.} \PYG{c}{\PYGZsh{} mean and dispersion}
\PYG{g+gp}{\PYGZgt{}\PYGZgt{}\PYGZgt{} }\PYG{n}{s} \PYG{o}{=} \PYG{n}{np}\PYG{o}{.}\PYG{n}{random}\PYG{o}{.}\PYG{n}{gamma}\PYG{p}{(}\PYG{n}{shape}\PYG{p}{,} \PYG{n}{scale}\PYG{p}{,} \PYG{l+m+mi}{1000}\PYG{p}{)}
\end{Verbatim}

Display the histogram of the samples, along with
the probability density function:

\begin{Verbatim}[commandchars=\\\{\}]
\PYG{g+gp}{\PYGZgt{}\PYGZgt{}\PYGZgt{} }\PYG{k+kn}{import} \PYG{n+nn}{matplotlib.pyplot} \PYG{k+kn}{as} \PYG{n+nn}{plt}
\PYG{g+gp}{\PYGZgt{}\PYGZgt{}\PYGZgt{} }\PYG{k+kn}{import} \PYG{n+nn}{scipy.special} \PYG{k+kn}{as} \PYG{n+nn}{sps}
\PYG{g+gp}{\PYGZgt{}\PYGZgt{}\PYGZgt{} }\PYG{n}{count}\PYG{p}{,} \PYG{n}{bins}\PYG{p}{,} \PYG{n}{ignored} \PYG{o}{=} \PYG{n}{plt}\PYG{o}{.}\PYG{n}{hist}\PYG{p}{(}\PYG{n}{s}\PYG{p}{,} \PYG{l+m+mi}{50}\PYG{p}{,} \PYG{n}{normed}\PYG{o}{=}\PYG{n+nb+bp}{True}\PYG{p}{)}
\PYG{g+gp}{\PYGZgt{}\PYGZgt{}\PYGZgt{} }\PYG{n}{y} \PYG{o}{=} \PYG{n}{bins}\PYG{o}{*}\PYG{o}{*}\PYG{p}{(}\PYG{n}{shape}\PYG{o}{\PYGZhy{}}\PYG{l+m+mi}{1}\PYG{p}{)}\PYG{o}{*}\PYG{p}{(}\PYG{n}{np}\PYG{o}{.}\PYG{n}{exp}\PYG{p}{(}\PYG{o}{\PYGZhy{}}\PYG{n}{bins}\PYG{o}{/}\PYG{n}{scale}\PYG{p}{)} \PYG{o}{/}
\PYG{g+gp}{... }                     \PYG{p}{(}\PYG{n}{sps}\PYG{o}{.}\PYG{n}{gamma}\PYG{p}{(}\PYG{n}{shape}\PYG{p}{)}\PYG{o}{*}\PYG{n}{scale}\PYG{o}{*}\PYG{o}{*}\PYG{n}{shape}\PYG{p}{)}\PYG{p}{)}
\PYG{g+gp}{\PYGZgt{}\PYGZgt{}\PYGZgt{} }\PYG{n}{plt}\PYG{o}{.}\PYG{n}{plot}\PYG{p}{(}\PYG{n}{bins}\PYG{p}{,} \PYG{n}{y}\PYG{p}{,} \PYG{n}{linewidth}\PYG{o}{=}\PYG{l+m+mi}{2}\PYG{p}{,} \PYG{n}{color}\PYG{o}{=}\PYG{l+s}{\PYGZsq{}}\PYG{l+s}{r}\PYG{l+s}{\PYGZsq{}}\PYG{p}{)}
\PYG{g+gp}{\PYGZgt{}\PYGZgt{}\PYGZgt{} }\PYG{n}{plt}\PYG{o}{.}\PYG{n}{show}\PYG{p}{(}\PYG{p}{)}
\end{Verbatim}

\end{fulllineitems}

\index{geometric() (in module lib.graph.scc)}

\begin{fulllineitems}
\phantomsection\label{lib.graph:lib.graph.scc.geometric}\pysiglinewithargsret{\code{lib.graph.scc.}\bfcode{geometric}}{\emph{p}, \emph{size=None}}{}
Draw samples from the geometric distribution.

Bernoulli trials are experiments with one of two outcomes:
success or failure (an example of such an experiment is flipping
a coin).  The geometric distribution models the number of trials
that must be run in order to achieve success.  It is therefore
supported on the positive integers, \code{k = 1, 2, ...}.

The probability mass function of the geometric distribution is
\begin{gather}
\begin{split}f(k) = (1 - p)^{k - 1} p\end{split}\notag
\end{gather}
where \emph{p} is the probability of success of an individual trial.
\begin{description}
\item[{p}] \leavevmode{[}float{]}
The probability of success of an individual trial.

\item[{size}] \leavevmode{[}tuple of ints{]}
Number of values to draw from the distribution.  The output
is shaped according to \emph{size}.

\end{description}
\begin{description}
\item[{out}] \leavevmode{[}ndarray{]}
Samples from the geometric distribution, shaped according to
\emph{size}.

\end{description}

Draw ten thousand values from the geometric distribution,
with the probability of an individual success equal to 0.35:

\begin{Verbatim}[commandchars=\\\{\}]
\PYG{g+gp}{\PYGZgt{}\PYGZgt{}\PYGZgt{} }\PYG{n}{z} \PYG{o}{=} \PYG{n}{np}\PYG{o}{.}\PYG{n}{random}\PYG{o}{.}\PYG{n}{geometric}\PYG{p}{(}\PYG{n}{p}\PYG{o}{=}\PYG{l+m+mf}{0.35}\PYG{p}{,} \PYG{n}{size}\PYG{o}{=}\PYG{l+m+mi}{10000}\PYG{p}{)}
\end{Verbatim}

How many trials succeeded after a single run?

\begin{Verbatim}[commandchars=\\\{\}]
\PYG{g+gp}{\PYGZgt{}\PYGZgt{}\PYGZgt{} }\PYG{p}{(}\PYG{n}{z} \PYG{o}{==} \PYG{l+m+mi}{1}\PYG{p}{)}\PYG{o}{.}\PYG{n}{sum}\PYG{p}{(}\PYG{p}{)} \PYG{o}{/} \PYG{l+m+mf}{10000.}
\PYG{g+go}{0.34889999999999999 \PYGZsh{}random}
\end{Verbatim}

\end{fulllineitems}

\index{get\_state() (in module lib.graph.scc)}

\begin{fulllineitems}
\phantomsection\label{lib.graph:lib.graph.scc.get_state}\pysiglinewithargsret{\code{lib.graph.scc.}\bfcode{get\_state}}{}{}
Return a tuple representing the internal state of the generator.

For more details, see \emph{set\_state}.
\begin{description}
\item[{out}] \leavevmode{[}tuple(str, ndarray of 624 uints, int, int, float){]}
The returned tuple has the following items:
\begin{enumerate}
\item {} 
the string `MT19937'.

\item {} 
a 1-D array of 624 unsigned integer keys.

\item {} 
an integer \code{pos}.

\item {} 
an integer \code{has\_gauss}.

\item {} 
a float \code{cached\_gaussian}.

\end{enumerate}

\end{description}

set\_state

\emph{set\_state} and \emph{get\_state} are not needed to work with any of the
random distributions in NumPy. If the internal state is manually altered,
the user should know exactly what he/she is doing.

\end{fulllineitems}

\index{gumbel() (in module lib.graph.scc)}

\begin{fulllineitems}
\phantomsection\label{lib.graph:lib.graph.scc.gumbel}\pysiglinewithargsret{\code{lib.graph.scc.}\bfcode{gumbel}}{\emph{loc=0.0}, \emph{scale=1.0}, \emph{size=None}}{}
Gumbel distribution.

Draw samples from a Gumbel distribution with specified location and scale.
For more information on the Gumbel distribution, see Notes and References
below.
\begin{description}
\item[{loc}] \leavevmode{[}float{]}
The location of the mode of the distribution.

\item[{scale}] \leavevmode{[}float{]}
The scale parameter of the distribution.

\item[{size}] \leavevmode{[}tuple of ints{]}
Output shape.  If the given shape is, e.g., \code{(m, n, k)}, then
\code{m * n * k} samples are drawn.

\end{description}
\begin{description}
\item[{out}] \leavevmode{[}ndarray{]}
The samples

\end{description}

scipy.stats.gumbel\_l
scipy.stats.gumbel\_r
scipy.stats.genextreme
\begin{quote}

probability density function, distribution, or cumulative density
function, etc. for each of the above
\end{quote}

weibull

The Gumbel (or Smallest Extreme Value (SEV) or the Smallest Extreme Value
Type I) distribution is one of a class of Generalized Extreme Value (GEV)
distributions used in modeling extreme value problems.  The Gumbel is a
special case of the Extreme Value Type I distribution for maximums from
distributions with ``exponential-like'' tails.

The probability density for the Gumbel distribution is
\begin{gather}
\begin{split}p(x) = \frac{e^{-(x - \mu)/ \beta}}{\beta} e^{ -e^{-(x - \mu)/
\beta}},\end{split}\notag
\end{gather}
where \(\mu\) is the mode, a location parameter, and \(\beta\) is
the scale parameter.

The Gumbel (named for German mathematician Emil Julius Gumbel) was used
very early in the hydrology literature, for modeling the occurrence of
flood events. It is also used for modeling maximum wind speed and rainfall
rates.  It is a ``fat-tailed'' distribution - the probability of an event in
the tail of the distribution is larger than if one used a Gaussian, hence
the surprisingly frequent occurrence of 100-year floods. Floods were
initially modeled as a Gaussian process, which underestimated the frequency
of extreme events.

It is one of a class of extreme value distributions, the Generalized
Extreme Value (GEV) distributions, which also includes the Weibull and
Frechet.

The function has a mean of \(\mu + 0.57721\beta\) and a variance of
\(\frac{\pi^2}{6}\beta^2\).

Gumbel, E. J., \emph{Statistics of Extremes}, New York: Columbia University
Press, 1958.

Reiss, R.-D. and Thomas, M., \emph{Statistical Analysis of Extreme Values from
Insurance, Finance, Hydrology and Other Fields}, Basel: Birkhauser Verlag,
2001.

Draw samples from the distribution:

\begin{Verbatim}[commandchars=\\\{\}]
\PYG{g+gp}{\PYGZgt{}\PYGZgt{}\PYGZgt{} }\PYG{n}{mu}\PYG{p}{,} \PYG{n}{beta} \PYG{o}{=} \PYG{l+m+mi}{0}\PYG{p}{,} \PYG{l+m+mf}{0.1} \PYG{c}{\PYGZsh{} location and scale}
\PYG{g+gp}{\PYGZgt{}\PYGZgt{}\PYGZgt{} }\PYG{n}{s} \PYG{o}{=} \PYG{n}{np}\PYG{o}{.}\PYG{n}{random}\PYG{o}{.}\PYG{n}{gumbel}\PYG{p}{(}\PYG{n}{mu}\PYG{p}{,} \PYG{n}{beta}\PYG{p}{,} \PYG{l+m+mi}{1000}\PYG{p}{)}
\end{Verbatim}

Display the histogram of the samples, along with
the probability density function:

\begin{Verbatim}[commandchars=\\\{\}]
\PYG{g+gp}{\PYGZgt{}\PYGZgt{}\PYGZgt{} }\PYG{k+kn}{import} \PYG{n+nn}{matplotlib.pyplot} \PYG{k+kn}{as} \PYG{n+nn}{plt}
\PYG{g+gp}{\PYGZgt{}\PYGZgt{}\PYGZgt{} }\PYG{n}{count}\PYG{p}{,} \PYG{n}{bins}\PYG{p}{,} \PYG{n}{ignored} \PYG{o}{=} \PYG{n}{plt}\PYG{o}{.}\PYG{n}{hist}\PYG{p}{(}\PYG{n}{s}\PYG{p}{,} \PYG{l+m+mi}{30}\PYG{p}{,} \PYG{n}{normed}\PYG{o}{=}\PYG{n+nb+bp}{True}\PYG{p}{)}
\PYG{g+gp}{\PYGZgt{}\PYGZgt{}\PYGZgt{} }\PYG{n}{plt}\PYG{o}{.}\PYG{n}{plot}\PYG{p}{(}\PYG{n}{bins}\PYG{p}{,} \PYG{p}{(}\PYG{l+m+mi}{1}\PYG{o}{/}\PYG{n}{beta}\PYG{p}{)}\PYG{o}{*}\PYG{n}{np}\PYG{o}{.}\PYG{n}{exp}\PYG{p}{(}\PYG{o}{\PYGZhy{}}\PYG{p}{(}\PYG{n}{bins} \PYG{o}{\PYGZhy{}} \PYG{n}{mu}\PYG{p}{)}\PYG{o}{/}\PYG{n}{beta}\PYG{p}{)}
\PYG{g+gp}{... }         \PYG{o}{*} \PYG{n}{np}\PYG{o}{.}\PYG{n}{exp}\PYG{p}{(} \PYG{o}{\PYGZhy{}}\PYG{n}{np}\PYG{o}{.}\PYG{n}{exp}\PYG{p}{(} \PYG{o}{\PYGZhy{}}\PYG{p}{(}\PYG{n}{bins} \PYG{o}{\PYGZhy{}} \PYG{n}{mu}\PYG{p}{)} \PYG{o}{/}\PYG{n}{beta}\PYG{p}{)} \PYG{p}{)}\PYG{p}{,}
\PYG{g+gp}{... }         \PYG{n}{linewidth}\PYG{o}{=}\PYG{l+m+mi}{2}\PYG{p}{,} \PYG{n}{color}\PYG{o}{=}\PYG{l+s}{\PYGZsq{}}\PYG{l+s}{r}\PYG{l+s}{\PYGZsq{}}\PYG{p}{)}
\PYG{g+gp}{\PYGZgt{}\PYGZgt{}\PYGZgt{} }\PYG{n}{plt}\PYG{o}{.}\PYG{n}{show}\PYG{p}{(}\PYG{p}{)}
\end{Verbatim}

Show how an extreme value distribution can arise from a Gaussian process
and compare to a Gaussian:

\begin{Verbatim}[commandchars=\\\{\}]
\PYG{g+gp}{\PYGZgt{}\PYGZgt{}\PYGZgt{} }\PYG{n}{means} \PYG{o}{=} \PYG{p}{[}\PYG{p}{]}
\PYG{g+gp}{\PYGZgt{}\PYGZgt{}\PYGZgt{} }\PYG{n}{maxima} \PYG{o}{=} \PYG{p}{[}\PYG{p}{]}
\PYG{g+gp}{\PYGZgt{}\PYGZgt{}\PYGZgt{} }\PYG{k}{for} \PYG{n}{i} \PYG{o+ow}{in} \PYG{n+nb}{range}\PYG{p}{(}\PYG{l+m+mi}{0}\PYG{p}{,}\PYG{l+m+mi}{1000}\PYG{p}{)} \PYG{p}{:}
\PYG{g+gp}{... }   \PYG{n}{a} \PYG{o}{=} \PYG{n}{np}\PYG{o}{.}\PYG{n}{random}\PYG{o}{.}\PYG{n}{normal}\PYG{p}{(}\PYG{n}{mu}\PYG{p}{,} \PYG{n}{beta}\PYG{p}{,} \PYG{l+m+mi}{1000}\PYG{p}{)}
\PYG{g+gp}{... }   \PYG{n}{means}\PYG{o}{.}\PYG{n}{append}\PYG{p}{(}\PYG{n}{a}\PYG{o}{.}\PYG{n}{mean}\PYG{p}{(}\PYG{p}{)}\PYG{p}{)}
\PYG{g+gp}{... }   \PYG{n}{maxima}\PYG{o}{.}\PYG{n}{append}\PYG{p}{(}\PYG{n}{a}\PYG{o}{.}\PYG{n}{max}\PYG{p}{(}\PYG{p}{)}\PYG{p}{)}
\PYG{g+gp}{\PYGZgt{}\PYGZgt{}\PYGZgt{} }\PYG{n}{count}\PYG{p}{,} \PYG{n}{bins}\PYG{p}{,} \PYG{n}{ignored} \PYG{o}{=} \PYG{n}{plt}\PYG{o}{.}\PYG{n}{hist}\PYG{p}{(}\PYG{n}{maxima}\PYG{p}{,} \PYG{l+m+mi}{30}\PYG{p}{,} \PYG{n}{normed}\PYG{o}{=}\PYG{n+nb+bp}{True}\PYG{p}{)}
\PYG{g+gp}{\PYGZgt{}\PYGZgt{}\PYGZgt{} }\PYG{n}{beta} \PYG{o}{=} \PYG{n}{np}\PYG{o}{.}\PYG{n}{std}\PYG{p}{(}\PYG{n}{maxima}\PYG{p}{)}\PYG{o}{*}\PYG{n}{np}\PYG{o}{.}\PYG{n}{pi}\PYG{o}{/}\PYG{n}{np}\PYG{o}{.}\PYG{n}{sqrt}\PYG{p}{(}\PYG{l+m+mi}{6}\PYG{p}{)}
\PYG{g+gp}{\PYGZgt{}\PYGZgt{}\PYGZgt{} }\PYG{n}{mu} \PYG{o}{=} \PYG{n}{np}\PYG{o}{.}\PYG{n}{mean}\PYG{p}{(}\PYG{n}{maxima}\PYG{p}{)} \PYG{o}{\PYGZhy{}} \PYG{l+m+mf}{0.57721}\PYG{o}{*}\PYG{n}{beta}
\PYG{g+gp}{\PYGZgt{}\PYGZgt{}\PYGZgt{} }\PYG{n}{plt}\PYG{o}{.}\PYG{n}{plot}\PYG{p}{(}\PYG{n}{bins}\PYG{p}{,} \PYG{p}{(}\PYG{l+m+mi}{1}\PYG{o}{/}\PYG{n}{beta}\PYG{p}{)}\PYG{o}{*}\PYG{n}{np}\PYG{o}{.}\PYG{n}{exp}\PYG{p}{(}\PYG{o}{\PYGZhy{}}\PYG{p}{(}\PYG{n}{bins} \PYG{o}{\PYGZhy{}} \PYG{n}{mu}\PYG{p}{)}\PYG{o}{/}\PYG{n}{beta}\PYG{p}{)}
\PYG{g+gp}{... }         \PYG{o}{*} \PYG{n}{np}\PYG{o}{.}\PYG{n}{exp}\PYG{p}{(}\PYG{o}{\PYGZhy{}}\PYG{n}{np}\PYG{o}{.}\PYG{n}{exp}\PYG{p}{(}\PYG{o}{\PYGZhy{}}\PYG{p}{(}\PYG{n}{bins} \PYG{o}{\PYGZhy{}} \PYG{n}{mu}\PYG{p}{)}\PYG{o}{/}\PYG{n}{beta}\PYG{p}{)}\PYG{p}{)}\PYG{p}{,}
\PYG{g+gp}{... }         \PYG{n}{linewidth}\PYG{o}{=}\PYG{l+m+mi}{2}\PYG{p}{,} \PYG{n}{color}\PYG{o}{=}\PYG{l+s}{\PYGZsq{}}\PYG{l+s}{r}\PYG{l+s}{\PYGZsq{}}\PYG{p}{)}
\PYG{g+gp}{\PYGZgt{}\PYGZgt{}\PYGZgt{} }\PYG{n}{plt}\PYG{o}{.}\PYG{n}{plot}\PYG{p}{(}\PYG{n}{bins}\PYG{p}{,} \PYG{l+m+mi}{1}\PYG{o}{/}\PYG{p}{(}\PYG{n}{beta} \PYG{o}{*} \PYG{n}{np}\PYG{o}{.}\PYG{n}{sqrt}\PYG{p}{(}\PYG{l+m+mi}{2} \PYG{o}{*} \PYG{n}{np}\PYG{o}{.}\PYG{n}{pi}\PYG{p}{)}\PYG{p}{)}
\PYG{g+gp}{... }         \PYG{o}{*} \PYG{n}{np}\PYG{o}{.}\PYG{n}{exp}\PYG{p}{(}\PYG{o}{\PYGZhy{}}\PYG{p}{(}\PYG{n}{bins} \PYG{o}{\PYGZhy{}} \PYG{n}{mu}\PYG{p}{)}\PYG{o}{*}\PYG{o}{*}\PYG{l+m+mi}{2} \PYG{o}{/} \PYG{p}{(}\PYG{l+m+mi}{2} \PYG{o}{*} \PYG{n}{beta}\PYG{o}{*}\PYG{o}{*}\PYG{l+m+mi}{2}\PYG{p}{)}\PYG{p}{)}\PYG{p}{,}
\PYG{g+gp}{... }         \PYG{n}{linewidth}\PYG{o}{=}\PYG{l+m+mi}{2}\PYG{p}{,} \PYG{n}{color}\PYG{o}{=}\PYG{l+s}{\PYGZsq{}}\PYG{l+s}{g}\PYG{l+s}{\PYGZsq{}}\PYG{p}{)}
\PYG{g+gp}{\PYGZgt{}\PYGZgt{}\PYGZgt{} }\PYG{n}{plt}\PYG{o}{.}\PYG{n}{show}\PYG{p}{(}\PYG{p}{)}
\end{Verbatim}

\end{fulllineitems}

\index{hypergeometric() (in module lib.graph.scc)}

\begin{fulllineitems}
\phantomsection\label{lib.graph:lib.graph.scc.hypergeometric}\pysiglinewithargsret{\code{lib.graph.scc.}\bfcode{hypergeometric}}{\emph{ngood}, \emph{nbad}, \emph{nsample}, \emph{size=None}}{}
Draw samples from a Hypergeometric distribution.

Samples are drawn from a Hypergeometric distribution with specified
parameters, ngood (ways to make a good selection), nbad (ways to make
a bad selection), and nsample = number of items sampled, which is less
than or equal to the sum ngood + nbad.
\begin{description}
\item[{ngood}] \leavevmode{[}int or array\_like{]}
Number of ways to make a good selection.  Must be nonnegative.

\item[{nbad}] \leavevmode{[}int or array\_like{]}
Number of ways to make a bad selection.  Must be nonnegative.

\item[{nsample}] \leavevmode{[}int or array\_like{]}
Number of items sampled.  Must be at least 1 and at most
\code{ngood + nbad}.

\item[{size}] \leavevmode{[}int or tuple of int{]}
Output shape.  If the given shape is, e.g., \code{(m, n, k)}, then
\code{m * n * k} samples are drawn.

\end{description}
\begin{description}
\item[{samples}] \leavevmode{[}ndarray or scalar{]}
The values are all integers in  {[}0, n{]}.

\end{description}
\begin{description}
\item[{scipy.stats.distributions.hypergeom}] \leavevmode{[}probability density function,{]}
distribution or cumulative density function, etc.

\end{description}

The probability density for the Hypergeometric distribution is
\begin{gather}
\begin{split}P(x) = \frac{\binom{m}{n}\binom{N-m}{n-x}}{\binom{N}{n}},\end{split}\notag
\end{gather}
where \(0 \le x \le m\) and \(n+m-N \le x \le n\)

for P(x) the probability of x successes, n = ngood, m = nbad, and
N = number of samples.

Consider an urn with black and white marbles in it, ngood of them
black and nbad are white. If you draw nsample balls without
replacement, then the Hypergeometric distribution describes the
distribution of black balls in the drawn sample.

Note that this distribution is very similar to the Binomial
distribution, except that in this case, samples are drawn without
replacement, whereas in the Binomial case samples are drawn with
replacement (or the sample space is infinite). As the sample space
becomes large, this distribution approaches the Binomial.

Draw samples from the distribution:

\begin{Verbatim}[commandchars=\\\{\}]
\PYG{g+gp}{\PYGZgt{}\PYGZgt{}\PYGZgt{} }\PYG{n}{ngood}\PYG{p}{,} \PYG{n}{nbad}\PYG{p}{,} \PYG{n}{nsamp} \PYG{o}{=} \PYG{l+m+mi}{100}\PYG{p}{,} \PYG{l+m+mi}{2}\PYG{p}{,} \PYG{l+m+mi}{10}
\PYG{g+go}{\PYGZsh{} number of good, number of bad, and number of samples}
\PYG{g+gp}{\PYGZgt{}\PYGZgt{}\PYGZgt{} }\PYG{n}{s} \PYG{o}{=} \PYG{n}{np}\PYG{o}{.}\PYG{n}{random}\PYG{o}{.}\PYG{n}{hypergeometric}\PYG{p}{(}\PYG{n}{ngood}\PYG{p}{,} \PYG{n}{nbad}\PYG{p}{,} \PYG{n}{nsamp}\PYG{p}{,} \PYG{l+m+mi}{1000}\PYG{p}{)}
\PYG{g+gp}{\PYGZgt{}\PYGZgt{}\PYGZgt{} }\PYG{n}{hist}\PYG{p}{(}\PYG{n}{s}\PYG{p}{)}
\PYG{g+go}{\PYGZsh{}   note that it is very unlikely to grab both bad items}
\end{Verbatim}

Suppose you have an urn with 15 white and 15 black marbles.
If you pull 15 marbles at random, how likely is it that
12 or more of them are one color?

\begin{Verbatim}[commandchars=\\\{\}]
\PYG{g+gp}{\PYGZgt{}\PYGZgt{}\PYGZgt{} }\PYG{n}{s} \PYG{o}{=} \PYG{n}{np}\PYG{o}{.}\PYG{n}{random}\PYG{o}{.}\PYG{n}{hypergeometric}\PYG{p}{(}\PYG{l+m+mi}{15}\PYG{p}{,} \PYG{l+m+mi}{15}\PYG{p}{,} \PYG{l+m+mi}{15}\PYG{p}{,} \PYG{l+m+mi}{100000}\PYG{p}{)}
\PYG{g+gp}{\PYGZgt{}\PYGZgt{}\PYGZgt{} }\PYG{n+nb}{sum}\PYG{p}{(}\PYG{n}{s}\PYG{o}{\PYGZgt{}}\PYG{o}{=}\PYG{l+m+mi}{12}\PYG{p}{)}\PYG{o}{/}\PYG{l+m+mf}{100000.} \PYG{o}{+} \PYG{n+nb}{sum}\PYG{p}{(}\PYG{n}{s}\PYG{o}{\PYGZlt{}}\PYG{o}{=}\PYG{l+m+mi}{3}\PYG{p}{)}\PYG{o}{/}\PYG{l+m+mf}{100000.}
\PYG{g+go}{\PYGZsh{}   answer = 0.003 ... pretty unlikely!}
\end{Verbatim}

\end{fulllineitems}

\index{laplace() (in module lib.graph.scc)}

\begin{fulllineitems}
\phantomsection\label{lib.graph:lib.graph.scc.laplace}\pysiglinewithargsret{\code{lib.graph.scc.}\bfcode{laplace}}{\emph{loc=0.0}, \emph{scale=1.0}, \emph{size=None}}{}
Draw samples from the Laplace or double exponential distribution with
specified location (or mean) and scale (decay).

The Laplace distribution is similar to the Gaussian/normal distribution,
but is sharper at the peak and has fatter tails. It represents the
difference between two independent, identically distributed exponential
random variables.
\begin{description}
\item[{loc}] \leavevmode{[}float{]}
The position, \(\mu\), of the distribution peak.

\item[{scale}] \leavevmode{[}float{]}
\(\lambda\), the exponential decay.

\end{description}

It has the probability density function
\begin{gather}
\begin{split}f(x; \mu, \lambda) = \frac{1}{2\lambda}
\exp\left(-\frac{|x - \mu|}{\lambda}\right).\end{split}\notag
\end{gather}
The first law of Laplace, from 1774, states that the frequency of an error
can be expressed as an exponential function of the absolute magnitude of
the error, which leads to the Laplace distribution. For many problems in
Economics and Health sciences, this distribution seems to model the data
better than the standard Gaussian distribution

Draw samples from the distribution

\begin{Verbatim}[commandchars=\\\{\}]
\PYG{g+gp}{\PYGZgt{}\PYGZgt{}\PYGZgt{} }\PYG{n}{loc}\PYG{p}{,} \PYG{n}{scale} \PYG{o}{=} \PYG{l+m+mf}{0.}\PYG{p}{,} \PYG{l+m+mf}{1.}
\PYG{g+gp}{\PYGZgt{}\PYGZgt{}\PYGZgt{} }\PYG{n}{s} \PYG{o}{=} \PYG{n}{np}\PYG{o}{.}\PYG{n}{random}\PYG{o}{.}\PYG{n}{laplace}\PYG{p}{(}\PYG{n}{loc}\PYG{p}{,} \PYG{n}{scale}\PYG{p}{,} \PYG{l+m+mi}{1000}\PYG{p}{)}
\end{Verbatim}

Display the histogram of the samples, along with
the probability density function:

\begin{Verbatim}[commandchars=\\\{\}]
\PYG{g+gp}{\PYGZgt{}\PYGZgt{}\PYGZgt{} }\PYG{k+kn}{import} \PYG{n+nn}{matplotlib.pyplot} \PYG{k+kn}{as} \PYG{n+nn}{plt}
\PYG{g+gp}{\PYGZgt{}\PYGZgt{}\PYGZgt{} }\PYG{n}{count}\PYG{p}{,} \PYG{n}{bins}\PYG{p}{,} \PYG{n}{ignored} \PYG{o}{=} \PYG{n}{plt}\PYG{o}{.}\PYG{n}{hist}\PYG{p}{(}\PYG{n}{s}\PYG{p}{,} \PYG{l+m+mi}{30}\PYG{p}{,} \PYG{n}{normed}\PYG{o}{=}\PYG{n+nb+bp}{True}\PYG{p}{)}
\PYG{g+gp}{\PYGZgt{}\PYGZgt{}\PYGZgt{} }\PYG{n}{x} \PYG{o}{=} \PYG{n}{np}\PYG{o}{.}\PYG{n}{arange}\PYG{p}{(}\PYG{o}{\PYGZhy{}}\PYG{l+m+mf}{8.}\PYG{p}{,} \PYG{l+m+mf}{8.}\PYG{p}{,} \PYG{o}{.}\PYG{l+m+mo}{01}\PYG{p}{)}
\PYG{g+gp}{\PYGZgt{}\PYGZgt{}\PYGZgt{} }\PYG{n}{pdf} \PYG{o}{=} \PYG{n}{np}\PYG{o}{.}\PYG{n}{exp}\PYG{p}{(}\PYG{o}{\PYGZhy{}}\PYG{n+nb}{abs}\PYG{p}{(}\PYG{n}{x}\PYG{o}{\PYGZhy{}}\PYG{n}{loc}\PYG{o}{/}\PYG{n}{scale}\PYG{p}{)}\PYG{p}{)}\PYG{o}{/}\PYG{p}{(}\PYG{l+m+mf}{2.}\PYG{o}{*}\PYG{n}{scale}\PYG{p}{)}
\PYG{g+gp}{\PYGZgt{}\PYGZgt{}\PYGZgt{} }\PYG{n}{plt}\PYG{o}{.}\PYG{n}{plot}\PYG{p}{(}\PYG{n}{x}\PYG{p}{,} \PYG{n}{pdf}\PYG{p}{)}
\end{Verbatim}

Plot Gaussian for comparison:

\begin{Verbatim}[commandchars=\\\{\}]
\PYG{g+gp}{\PYGZgt{}\PYGZgt{}\PYGZgt{} }\PYG{n}{g} \PYG{o}{=} \PYG{p}{(}\PYG{l+m+mi}{1}\PYG{o}{/}\PYG{p}{(}\PYG{n}{scale} \PYG{o}{*} \PYG{n}{np}\PYG{o}{.}\PYG{n}{sqrt}\PYG{p}{(}\PYG{l+m+mi}{2} \PYG{o}{*} \PYG{n}{np}\PYG{o}{.}\PYG{n}{pi}\PYG{p}{)}\PYG{p}{)} \PYG{o}{*} 
\PYG{g+gp}{... }     \PYG{n}{np}\PYG{o}{.}\PYG{n}{exp}\PYG{p}{(} \PYG{o}{\PYGZhy{}} \PYG{p}{(}\PYG{n}{x} \PYG{o}{\PYGZhy{}} \PYG{n}{loc}\PYG{p}{)}\PYG{o}{*}\PYG{o}{*}\PYG{l+m+mi}{2} \PYG{o}{/} \PYG{p}{(}\PYG{l+m+mi}{2} \PYG{o}{*} \PYG{n}{scale}\PYG{o}{*}\PYG{o}{*}\PYG{l+m+mi}{2}\PYG{p}{)} \PYG{p}{)}\PYG{p}{)}
\PYG{g+gp}{\PYGZgt{}\PYGZgt{}\PYGZgt{} }\PYG{n}{plt}\PYG{o}{.}\PYG{n}{plot}\PYG{p}{(}\PYG{n}{x}\PYG{p}{,}\PYG{n}{g}\PYG{p}{)}
\end{Verbatim}

\end{fulllineitems}

\index{logistic() (in module lib.graph.scc)}

\begin{fulllineitems}
\phantomsection\label{lib.graph:lib.graph.scc.logistic}\pysiglinewithargsret{\code{lib.graph.scc.}\bfcode{logistic}}{\emph{loc=0.0}, \emph{scale=1.0}, \emph{size=None}}{}
Draw samples from a Logistic distribution.

Samples are drawn from a Logistic distribution with specified
parameters, loc (location or mean, also median), and scale (\textgreater{}0).

loc : float

scale : float \textgreater{} 0.
\begin{description}
\item[{size}] \leavevmode{[}\{tuple, int\}{]}
Output shape.  If the given shape is, e.g., \code{(m, n, k)}, then
\code{m * n * k} samples are drawn.

\end{description}
\begin{description}
\item[{samples}] \leavevmode{[}\{ndarray, scalar\}{]}
where the values are all integers in  {[}0, n{]}.

\end{description}
\begin{description}
\item[{scipy.stats.distributions.logistic}] \leavevmode{[}probability density function,{]}
distribution or cumulative density function, etc.

\end{description}

The probability density for the Logistic distribution is
\begin{gather}
\begin{split}P(x) = P(x) = \frac{e^{-(x-\mu)/s}}{s(1+e^{-(x-\mu)/s})^2},\end{split}\notag
\end{gather}
where \(\mu\) = location and \(s\) = scale.

The Logistic distribution is used in Extreme Value problems where it
can act as a mixture of Gumbel distributions, in Epidemiology, and by
the World Chess Federation (FIDE) where it is used in the Elo ranking
system, assuming the performance of each player is a logistically
distributed random variable.

Draw samples from the distribution:

\begin{Verbatim}[commandchars=\\\{\}]
\PYG{g+gp}{\PYGZgt{}\PYGZgt{}\PYGZgt{} }\PYG{n}{loc}\PYG{p}{,} \PYG{n}{scale} \PYG{o}{=} \PYG{l+m+mi}{10}\PYG{p}{,} \PYG{l+m+mi}{1}
\PYG{g+gp}{\PYGZgt{}\PYGZgt{}\PYGZgt{} }\PYG{n}{s} \PYG{o}{=} \PYG{n}{np}\PYG{o}{.}\PYG{n}{random}\PYG{o}{.}\PYG{n}{logistic}\PYG{p}{(}\PYG{n}{loc}\PYG{p}{,} \PYG{n}{scale}\PYG{p}{,} \PYG{l+m+mi}{10000}\PYG{p}{)}
\PYG{g+gp}{\PYGZgt{}\PYGZgt{}\PYGZgt{} }\PYG{n}{count}\PYG{p}{,} \PYG{n}{bins}\PYG{p}{,} \PYG{n}{ignored} \PYG{o}{=} \PYG{n}{plt}\PYG{o}{.}\PYG{n}{hist}\PYG{p}{(}\PYG{n}{s}\PYG{p}{,} \PYG{n}{bins}\PYG{o}{=}\PYG{l+m+mi}{50}\PYG{p}{)}
\end{Verbatim}

\#   plot against distribution

\begin{Verbatim}[commandchars=\\\{\}]
\PYG{g+gp}{\PYGZgt{}\PYGZgt{}\PYGZgt{} }\PYG{k}{def} \PYG{n+nf}{logist}\PYG{p}{(}\PYG{n}{x}\PYG{p}{,} \PYG{n}{loc}\PYG{p}{,} \PYG{n}{scale}\PYG{p}{)}\PYG{p}{:}
\PYG{g+gp}{... }    \PYG{k}{return} \PYG{n}{exp}\PYG{p}{(}\PYG{p}{(}\PYG{n}{loc}\PYG{o}{\PYGZhy{}}\PYG{n}{x}\PYG{p}{)}\PYG{o}{/}\PYG{n}{scale}\PYG{p}{)}\PYG{o}{/}\PYG{p}{(}\PYG{n}{scale}\PYG{o}{*}\PYG{p}{(}\PYG{l+m+mi}{1}\PYG{o}{+}\PYG{n}{exp}\PYG{p}{(}\PYG{p}{(}\PYG{n}{loc}\PYG{o}{\PYGZhy{}}\PYG{n}{x}\PYG{p}{)}\PYG{o}{/}\PYG{n}{scale}\PYG{p}{)}\PYG{p}{)}\PYG{o}{*}\PYG{o}{*}\PYG{l+m+mi}{2}\PYG{p}{)}
\PYG{g+gp}{\PYGZgt{}\PYGZgt{}\PYGZgt{} }\PYG{n}{plt}\PYG{o}{.}\PYG{n}{plot}\PYG{p}{(}\PYG{n}{bins}\PYG{p}{,} \PYG{n}{logist}\PYG{p}{(}\PYG{n}{bins}\PYG{p}{,} \PYG{n}{loc}\PYG{p}{,} \PYG{n}{scale}\PYG{p}{)}\PYG{o}{*}\PYG{n}{count}\PYG{o}{.}\PYG{n}{max}\PYG{p}{(}\PYG{p}{)}\PYG{o}{/}\PYGZbs{}
\PYG{g+gp}{... }\PYG{n}{logist}\PYG{p}{(}\PYG{n}{bins}\PYG{p}{,} \PYG{n}{loc}\PYG{p}{,} \PYG{n}{scale}\PYG{p}{)}\PYG{o}{.}\PYG{n}{max}\PYG{p}{(}\PYG{p}{)}\PYG{p}{)}
\PYG{g+gp}{\PYGZgt{}\PYGZgt{}\PYGZgt{} }\PYG{n}{plt}\PYG{o}{.}\PYG{n}{show}\PYG{p}{(}\PYG{p}{)}
\end{Verbatim}

\end{fulllineitems}

\index{lognormal() (in module lib.graph.scc)}

\begin{fulllineitems}
\phantomsection\label{lib.graph:lib.graph.scc.lognormal}\pysiglinewithargsret{\code{lib.graph.scc.}\bfcode{lognormal}}{\emph{mean=0.0}, \emph{sigma=1.0}, \emph{size=None}}{}
Return samples drawn from a log-normal distribution.

Draw samples from a log-normal distribution with specified mean,
standard deviation, and array shape.  Note that the mean and standard
deviation are not the values for the distribution itself, but of the
underlying normal distribution it is derived from.
\begin{description}
\item[{mean}] \leavevmode{[}float{]}
Mean value of the underlying normal distribution

\item[{sigma}] \leavevmode{[}float, \textgreater{} 0.{]}
Standard deviation of the underlying normal distribution

\item[{size}] \leavevmode{[}tuple of ints{]}
Output shape.  If the given shape is, e.g., \code{(m, n, k)}, then
\code{m * n * k} samples are drawn.

\end{description}
\begin{description}
\item[{samples}] \leavevmode{[}ndarray or float{]}
The desired samples. An array of the same shape as \emph{size} if given,
if \emph{size} is None a float is returned.

\end{description}
\begin{description}
\item[{scipy.stats.lognorm}] \leavevmode{[}probability density function, distribution,{]}
cumulative density function, etc.

\end{description}

A variable \emph{x} has a log-normal distribution if \emph{log(x)} is normally
distributed.  The probability density function for the log-normal
distribution is:
\begin{gather}
\begin{split}p(x) = \frac{1}{\sigma x \sqrt{2\pi}}
e^{(-\frac{(ln(x)-\mu)^2}{2\sigma^2})}\end{split}\notag
\end{gather}
where \(\mu\) is the mean and \(\sigma\) is the standard
deviation of the normally distributed logarithm of the variable.
A log-normal distribution results if a random variable is the \emph{product}
of a large number of independent, identically-distributed variables in
the same way that a normal distribution results if the variable is the
\emph{sum} of a large number of independent, identically-distributed
variables.

Limpert, E., Stahel, W. A., and Abbt, M., ``Log-normal Distributions
across the Sciences: Keys and Clues,'' \emph{BioScience}, Vol. 51, No. 5,
May, 2001.  \href{http://stat.ethz.ch/~stahel/lognormal/bioscience.pdf}{http://stat.ethz.ch/\textasciitilde{}stahel/lognormal/bioscience.pdf}

Reiss, R.D. and Thomas, M., \emph{Statistical Analysis of Extreme Values},
Basel: Birkhauser Verlag, 2001, pp. 31-32.

Draw samples from the distribution:

\begin{Verbatim}[commandchars=\\\{\}]
\PYG{g+gp}{\PYGZgt{}\PYGZgt{}\PYGZgt{} }\PYG{n}{mu}\PYG{p}{,} \PYG{n}{sigma} \PYG{o}{=} \PYG{l+m+mf}{3.}\PYG{p}{,} \PYG{l+m+mf}{1.} \PYG{c}{\PYGZsh{} mean and standard deviation}
\PYG{g+gp}{\PYGZgt{}\PYGZgt{}\PYGZgt{} }\PYG{n}{s} \PYG{o}{=} \PYG{n}{np}\PYG{o}{.}\PYG{n}{random}\PYG{o}{.}\PYG{n}{lognormal}\PYG{p}{(}\PYG{n}{mu}\PYG{p}{,} \PYG{n}{sigma}\PYG{p}{,} \PYG{l+m+mi}{1000}\PYG{p}{)}
\end{Verbatim}

Display the histogram of the samples, along with
the probability density function:

\begin{Verbatim}[commandchars=\\\{\}]
\PYG{g+gp}{\PYGZgt{}\PYGZgt{}\PYGZgt{} }\PYG{k+kn}{import} \PYG{n+nn}{matplotlib.pyplot} \PYG{k+kn}{as} \PYG{n+nn}{plt}
\PYG{g+gp}{\PYGZgt{}\PYGZgt{}\PYGZgt{} }\PYG{n}{count}\PYG{p}{,} \PYG{n}{bins}\PYG{p}{,} \PYG{n}{ignored} \PYG{o}{=} \PYG{n}{plt}\PYG{o}{.}\PYG{n}{hist}\PYG{p}{(}\PYG{n}{s}\PYG{p}{,} \PYG{l+m+mi}{100}\PYG{p}{,} \PYG{n}{normed}\PYG{o}{=}\PYG{n+nb+bp}{True}\PYG{p}{,} \PYG{n}{align}\PYG{o}{=}\PYG{l+s}{\PYGZsq{}}\PYG{l+s}{mid}\PYG{l+s}{\PYGZsq{}}\PYG{p}{)}
\end{Verbatim}

\begin{Verbatim}[commandchars=\\\{\}]
\PYG{g+gp}{\PYGZgt{}\PYGZgt{}\PYGZgt{} }\PYG{n}{x} \PYG{o}{=} \PYG{n}{np}\PYG{o}{.}\PYG{n}{linspace}\PYG{p}{(}\PYG{n+nb}{min}\PYG{p}{(}\PYG{n}{bins}\PYG{p}{)}\PYG{p}{,} \PYG{n+nb}{max}\PYG{p}{(}\PYG{n}{bins}\PYG{p}{)}\PYG{p}{,} \PYG{l+m+mi}{10000}\PYG{p}{)}
\PYG{g+gp}{\PYGZgt{}\PYGZgt{}\PYGZgt{} }\PYG{n}{pdf} \PYG{o}{=} \PYG{p}{(}\PYG{n}{np}\PYG{o}{.}\PYG{n}{exp}\PYG{p}{(}\PYG{o}{\PYGZhy{}}\PYG{p}{(}\PYG{n}{np}\PYG{o}{.}\PYG{n}{log}\PYG{p}{(}\PYG{n}{x}\PYG{p}{)} \PYG{o}{\PYGZhy{}} \PYG{n}{mu}\PYG{p}{)}\PYG{o}{*}\PYG{o}{*}\PYG{l+m+mi}{2} \PYG{o}{/} \PYG{p}{(}\PYG{l+m+mi}{2} \PYG{o}{*} \PYG{n}{sigma}\PYG{o}{*}\PYG{o}{*}\PYG{l+m+mi}{2}\PYG{p}{)}\PYG{p}{)}
\PYG{g+gp}{... }       \PYG{o}{/} \PYG{p}{(}\PYG{n}{x} \PYG{o}{*} \PYG{n}{sigma} \PYG{o}{*} \PYG{n}{np}\PYG{o}{.}\PYG{n}{sqrt}\PYG{p}{(}\PYG{l+m+mi}{2} \PYG{o}{*} \PYG{n}{np}\PYG{o}{.}\PYG{n}{pi}\PYG{p}{)}\PYG{p}{)}\PYG{p}{)}
\end{Verbatim}

\begin{Verbatim}[commandchars=\\\{\}]
\PYG{g+gp}{\PYGZgt{}\PYGZgt{}\PYGZgt{} }\PYG{n}{plt}\PYG{o}{.}\PYG{n}{plot}\PYG{p}{(}\PYG{n}{x}\PYG{p}{,} \PYG{n}{pdf}\PYG{p}{,} \PYG{n}{linewidth}\PYG{o}{=}\PYG{l+m+mi}{2}\PYG{p}{,} \PYG{n}{color}\PYG{o}{=}\PYG{l+s}{\PYGZsq{}}\PYG{l+s}{r}\PYG{l+s}{\PYGZsq{}}\PYG{p}{)}
\PYG{g+gp}{\PYGZgt{}\PYGZgt{}\PYGZgt{} }\PYG{n}{plt}\PYG{o}{.}\PYG{n}{axis}\PYG{p}{(}\PYG{l+s}{\PYGZsq{}}\PYG{l+s}{tight}\PYG{l+s}{\PYGZsq{}}\PYG{p}{)}
\PYG{g+gp}{\PYGZgt{}\PYGZgt{}\PYGZgt{} }\PYG{n}{plt}\PYG{o}{.}\PYG{n}{show}\PYG{p}{(}\PYG{p}{)}
\end{Verbatim}

Demonstrate that taking the products of random samples from a uniform
distribution can be fit well by a log-normal probability density function.

\begin{Verbatim}[commandchars=\\\{\}]
\PYG{g+gp}{\PYGZgt{}\PYGZgt{}\PYGZgt{} }\PYG{c}{\PYGZsh{} Generate a thousand samples: each is the product of 100 random}
\PYG{g+gp}{\PYGZgt{}\PYGZgt{}\PYGZgt{} }\PYG{c}{\PYGZsh{} values, drawn from a normal distribution.}
\PYG{g+gp}{\PYGZgt{}\PYGZgt{}\PYGZgt{} }\PYG{n}{b} \PYG{o}{=} \PYG{p}{[}\PYG{p}{]}
\PYG{g+gp}{\PYGZgt{}\PYGZgt{}\PYGZgt{} }\PYG{k}{for} \PYG{n}{i} \PYG{o+ow}{in} \PYG{n+nb}{range}\PYG{p}{(}\PYG{l+m+mi}{1000}\PYG{p}{)}\PYG{p}{:}
\PYG{g+gp}{... }   \PYG{n}{a} \PYG{o}{=} \PYG{l+m+mf}{10.} \PYG{o}{+} \PYG{n}{np}\PYG{o}{.}\PYG{n}{random}\PYG{o}{.}\PYG{n}{random}\PYG{p}{(}\PYG{l+m+mi}{100}\PYG{p}{)}
\PYG{g+gp}{... }   \PYG{n}{b}\PYG{o}{.}\PYG{n}{append}\PYG{p}{(}\PYG{n}{np}\PYG{o}{.}\PYG{n}{product}\PYG{p}{(}\PYG{n}{a}\PYG{p}{)}\PYG{p}{)}
\end{Verbatim}

\begin{Verbatim}[commandchars=\\\{\}]
\PYG{g+gp}{\PYGZgt{}\PYGZgt{}\PYGZgt{} }\PYG{n}{b} \PYG{o}{=} \PYG{n}{np}\PYG{o}{.}\PYG{n}{array}\PYG{p}{(}\PYG{n}{b}\PYG{p}{)} \PYG{o}{/} \PYG{n}{np}\PYG{o}{.}\PYG{n}{min}\PYG{p}{(}\PYG{n}{b}\PYG{p}{)} \PYG{c}{\PYGZsh{} scale values to be positive}
\PYG{g+gp}{\PYGZgt{}\PYGZgt{}\PYGZgt{} }\PYG{n}{count}\PYG{p}{,} \PYG{n}{bins}\PYG{p}{,} \PYG{n}{ignored} \PYG{o}{=} \PYG{n}{plt}\PYG{o}{.}\PYG{n}{hist}\PYG{p}{(}\PYG{n}{b}\PYG{p}{,} \PYG{l+m+mi}{100}\PYG{p}{,} \PYG{n}{normed}\PYG{o}{=}\PYG{n+nb+bp}{True}\PYG{p}{,} \PYG{n}{align}\PYG{o}{=}\PYG{l+s}{\PYGZsq{}}\PYG{l+s}{center}\PYG{l+s}{\PYGZsq{}}\PYG{p}{)}
\PYG{g+gp}{\PYGZgt{}\PYGZgt{}\PYGZgt{} }\PYG{n}{sigma} \PYG{o}{=} \PYG{n}{np}\PYG{o}{.}\PYG{n}{std}\PYG{p}{(}\PYG{n}{np}\PYG{o}{.}\PYG{n}{log}\PYG{p}{(}\PYG{n}{b}\PYG{p}{)}\PYG{p}{)}
\PYG{g+gp}{\PYGZgt{}\PYGZgt{}\PYGZgt{} }\PYG{n}{mu} \PYG{o}{=} \PYG{n}{np}\PYG{o}{.}\PYG{n}{mean}\PYG{p}{(}\PYG{n}{np}\PYG{o}{.}\PYG{n}{log}\PYG{p}{(}\PYG{n}{b}\PYG{p}{)}\PYG{p}{)}
\end{Verbatim}

\begin{Verbatim}[commandchars=\\\{\}]
\PYG{g+gp}{\PYGZgt{}\PYGZgt{}\PYGZgt{} }\PYG{n}{x} \PYG{o}{=} \PYG{n}{np}\PYG{o}{.}\PYG{n}{linspace}\PYG{p}{(}\PYG{n+nb}{min}\PYG{p}{(}\PYG{n}{bins}\PYG{p}{)}\PYG{p}{,} \PYG{n+nb}{max}\PYG{p}{(}\PYG{n}{bins}\PYG{p}{)}\PYG{p}{,} \PYG{l+m+mi}{10000}\PYG{p}{)}
\PYG{g+gp}{\PYGZgt{}\PYGZgt{}\PYGZgt{} }\PYG{n}{pdf} \PYG{o}{=} \PYG{p}{(}\PYG{n}{np}\PYG{o}{.}\PYG{n}{exp}\PYG{p}{(}\PYG{o}{\PYGZhy{}}\PYG{p}{(}\PYG{n}{np}\PYG{o}{.}\PYG{n}{log}\PYG{p}{(}\PYG{n}{x}\PYG{p}{)} \PYG{o}{\PYGZhy{}} \PYG{n}{mu}\PYG{p}{)}\PYG{o}{*}\PYG{o}{*}\PYG{l+m+mi}{2} \PYG{o}{/} \PYG{p}{(}\PYG{l+m+mi}{2} \PYG{o}{*} \PYG{n}{sigma}\PYG{o}{*}\PYG{o}{*}\PYG{l+m+mi}{2}\PYG{p}{)}\PYG{p}{)}
\PYG{g+gp}{... }       \PYG{o}{/} \PYG{p}{(}\PYG{n}{x} \PYG{o}{*} \PYG{n}{sigma} \PYG{o}{*} \PYG{n}{np}\PYG{o}{.}\PYG{n}{sqrt}\PYG{p}{(}\PYG{l+m+mi}{2} \PYG{o}{*} \PYG{n}{np}\PYG{o}{.}\PYG{n}{pi}\PYG{p}{)}\PYG{p}{)}\PYG{p}{)}
\end{Verbatim}

\begin{Verbatim}[commandchars=\\\{\}]
\PYG{g+gp}{\PYGZgt{}\PYGZgt{}\PYGZgt{} }\PYG{n}{plt}\PYG{o}{.}\PYG{n}{plot}\PYG{p}{(}\PYG{n}{x}\PYG{p}{,} \PYG{n}{pdf}\PYG{p}{,} \PYG{n}{color}\PYG{o}{=}\PYG{l+s}{\PYGZsq{}}\PYG{l+s}{r}\PYG{l+s}{\PYGZsq{}}\PYG{p}{,} \PYG{n}{linewidth}\PYG{o}{=}\PYG{l+m+mi}{2}\PYG{p}{)}
\PYG{g+gp}{\PYGZgt{}\PYGZgt{}\PYGZgt{} }\PYG{n}{plt}\PYG{o}{.}\PYG{n}{show}\PYG{p}{(}\PYG{p}{)}
\end{Verbatim}

\end{fulllineitems}

\index{logseries() (in module lib.graph.scc)}

\begin{fulllineitems}
\phantomsection\label{lib.graph:lib.graph.scc.logseries}\pysiglinewithargsret{\code{lib.graph.scc.}\bfcode{logseries}}{\emph{p}, \emph{size=None}}{}
Draw samples from a Logarithmic Series distribution.

Samples are drawn from a Log Series distribution with specified
parameter, p (probability, 0 \textless{} p \textless{} 1).

loc : float

scale : float \textgreater{} 0.
\begin{description}
\item[{size}] \leavevmode{[}\{tuple, int\}{]}
Output shape.  If the given shape is, e.g., \code{(m, n, k)}, then
\code{m * n * k} samples are drawn.

\end{description}
\begin{description}
\item[{samples}] \leavevmode{[}\{ndarray, scalar\}{]}
where the values are all integers in  {[}0, n{]}.

\end{description}
\begin{description}
\item[{scipy.stats.distributions.logser}] \leavevmode{[}probability density function,{]}
distribution or cumulative density function, etc.

\end{description}

The probability density for the Log Series distribution is
\begin{gather}
\begin{split}P(k) = \frac{-p^k}{k \ln(1-p)},\end{split}\notag
\end{gather}
where p = probability.

The Log Series distribution is frequently used to represent species
richness and occurrence, first proposed by Fisher, Corbet, and
Williams in 1943 {[}2{]}.  It may also be used to model the numbers of
occupants seen in cars {[}3{]}.

Draw samples from the distribution:

\begin{Verbatim}[commandchars=\\\{\}]
\PYG{g+gp}{\PYGZgt{}\PYGZgt{}\PYGZgt{} }\PYG{n}{a} \PYG{o}{=} \PYG{o}{.}\PYG{l+m+mi}{6}
\PYG{g+gp}{\PYGZgt{}\PYGZgt{}\PYGZgt{} }\PYG{n}{s} \PYG{o}{=} \PYG{n}{np}\PYG{o}{.}\PYG{n}{random}\PYG{o}{.}\PYG{n}{logseries}\PYG{p}{(}\PYG{n}{a}\PYG{p}{,} \PYG{l+m+mi}{10000}\PYG{p}{)}
\PYG{g+gp}{\PYGZgt{}\PYGZgt{}\PYGZgt{} }\PYG{n}{count}\PYG{p}{,} \PYG{n}{bins}\PYG{p}{,} \PYG{n}{ignored} \PYG{o}{=} \PYG{n}{plt}\PYG{o}{.}\PYG{n}{hist}\PYG{p}{(}\PYG{n}{s}\PYG{p}{)}
\end{Verbatim}

\#   plot against distribution

\begin{Verbatim}[commandchars=\\\{\}]
\PYG{g+gp}{\PYGZgt{}\PYGZgt{}\PYGZgt{} }\PYG{k}{def} \PYG{n+nf}{logseries}\PYG{p}{(}\PYG{n}{k}\PYG{p}{,} \PYG{n}{p}\PYG{p}{)}\PYG{p}{:}
\PYG{g+gp}{... }    \PYG{k}{return} \PYG{o}{\PYGZhy{}}\PYG{n}{p}\PYG{o}{*}\PYG{o}{*}\PYG{n}{k}\PYG{o}{/}\PYG{p}{(}\PYG{n}{k}\PYG{o}{*}\PYG{n}{log}\PYG{p}{(}\PYG{l+m+mi}{1}\PYG{o}{\PYGZhy{}}\PYG{n}{p}\PYG{p}{)}\PYG{p}{)}
\PYG{g+gp}{\PYGZgt{}\PYGZgt{}\PYGZgt{} }\PYG{n}{plt}\PYG{o}{.}\PYG{n}{plot}\PYG{p}{(}\PYG{n}{bins}\PYG{p}{,} \PYG{n}{logseries}\PYG{p}{(}\PYG{n}{bins}\PYG{p}{,} \PYG{n}{a}\PYG{p}{)}\PYG{o}{*}\PYG{n}{count}\PYG{o}{.}\PYG{n}{max}\PYG{p}{(}\PYG{p}{)}\PYG{o}{/}
\PYG{g+go}{             logseries(bins, a).max(), \PYGZsq{}r\PYGZsq{})}
\PYG{g+gp}{\PYGZgt{}\PYGZgt{}\PYGZgt{} }\PYG{n}{plt}\PYG{o}{.}\PYG{n}{show}\PYG{p}{(}\PYG{p}{)}
\end{Verbatim}

\end{fulllineitems}

\index{multinomial() (in module lib.graph.scc)}

\begin{fulllineitems}
\phantomsection\label{lib.graph:lib.graph.scc.multinomial}\pysiglinewithargsret{\code{lib.graph.scc.}\bfcode{multinomial}}{\emph{n}, \emph{pvals}, \emph{size=None}}{}
Draw samples from a multinomial distribution.

The multinomial distribution is a multivariate generalisation of the
binomial distribution.  Take an experiment with one of \code{p}
possible outcomes.  An example of such an experiment is throwing a dice,
where the outcome can be 1 through 6.  Each sample drawn from the
distribution represents \emph{n} such experiments.  Its values,
\code{X\_i = {[}X\_0, X\_1, ..., X\_p{]}}, represent the number of times the outcome
was \code{i}.
\begin{description}
\item[{n}] \leavevmode{[}int{]}
Number of experiments.

\item[{pvals}] \leavevmode{[}sequence of floats, length p{]}
Probabilities of each of the \code{p} different outcomes.  These
should sum to 1 (however, the last element is always assumed to
account for the remaining probability, as long as
\code{sum(pvals{[}:-1{]}) \textless{}= 1)}.

\item[{size}] \leavevmode{[}tuple of ints{]}
Given a \emph{size} of \code{(M, N, K)}, then \code{M*N*K} samples are drawn,
and the output shape becomes \code{(M, N, K, p)}, since each sample
has shape \code{(p,)}.

\end{description}

Throw a dice 20 times:

\begin{Verbatim}[commandchars=\\\{\}]
\PYG{g+gp}{\PYGZgt{}\PYGZgt{}\PYGZgt{} }\PYG{n}{np}\PYG{o}{.}\PYG{n}{random}\PYG{o}{.}\PYG{n}{multinomial}\PYG{p}{(}\PYG{l+m+mi}{20}\PYG{p}{,} \PYG{p}{[}\PYG{l+m+mi}{1}\PYG{o}{/}\PYG{l+m+mf}{6.}\PYG{p}{]}\PYG{o}{*}\PYG{l+m+mi}{6}\PYG{p}{,} \PYG{n}{size}\PYG{o}{=}\PYG{l+m+mi}{1}\PYG{p}{)}
\PYG{g+go}{array([[4, 1, 7, 5, 2, 1]])}
\end{Verbatim}

It landed 4 times on 1, once on 2, etc.

Now, throw the dice 20 times, and 20 times again:

\begin{Verbatim}[commandchars=\\\{\}]
\PYG{g+gp}{\PYGZgt{}\PYGZgt{}\PYGZgt{} }\PYG{n}{np}\PYG{o}{.}\PYG{n}{random}\PYG{o}{.}\PYG{n}{multinomial}\PYG{p}{(}\PYG{l+m+mi}{20}\PYG{p}{,} \PYG{p}{[}\PYG{l+m+mi}{1}\PYG{o}{/}\PYG{l+m+mf}{6.}\PYG{p}{]}\PYG{o}{*}\PYG{l+m+mi}{6}\PYG{p}{,} \PYG{n}{size}\PYG{o}{=}\PYG{l+m+mi}{2}\PYG{p}{)}
\PYG{g+go}{array([[3, 4, 3, 3, 4, 3],}
\PYG{g+go}{       [2, 4, 3, 4, 0, 7]])}
\end{Verbatim}

For the first run, we threw 3 times 1, 4 times 2, etc.  For the second,
we threw 2 times 1, 4 times 2, etc.

A loaded dice is more likely to land on number 6:

\begin{Verbatim}[commandchars=\\\{\}]
\PYG{g+gp}{\PYGZgt{}\PYGZgt{}\PYGZgt{} }\PYG{n}{np}\PYG{o}{.}\PYG{n}{random}\PYG{o}{.}\PYG{n}{multinomial}\PYG{p}{(}\PYG{l+m+mi}{100}\PYG{p}{,} \PYG{p}{[}\PYG{l+m+mi}{1}\PYG{o}{/}\PYG{l+m+mf}{7.}\PYG{p}{]}\PYG{o}{*}\PYG{l+m+mi}{5}\PYG{p}{)}
\PYG{g+go}{array([13, 16, 13, 16, 42])}
\end{Verbatim}

\end{fulllineitems}

\index{multivariate\_normal() (in module lib.graph.scc)}

\begin{fulllineitems}
\phantomsection\label{lib.graph:lib.graph.scc.multivariate_normal}\pysiglinewithargsret{\code{lib.graph.scc.}\bfcode{multivariate\_normal}}{\emph{mean}, \emph{cov}\optional{, \emph{size}}}{}
Draw random samples from a multivariate normal distribution.

The multivariate normal, multinormal or Gaussian distribution is a
generalization of the one-dimensional normal distribution to higher
dimensions.  Such a distribution is specified by its mean and
covariance matrix.  These parameters are analogous to the mean
(average or ``center'') and variance (standard deviation, or ``width,''
squared) of the one-dimensional normal distribution.
\begin{description}
\item[{mean}] \leavevmode{[}1-D array\_like, of length N{]}
Mean of the N-dimensional distribution.

\item[{cov}] \leavevmode{[}2-D array\_like, of shape (N, N){]}
Covariance matrix of the distribution.  Must be symmetric and
positive semi-definite for ``physically meaningful'' results.

\item[{size}] \leavevmode{[}int or tuple of ints, optional{]}
Given a shape of, for example, \code{(m,n,k)}, \code{m*n*k} samples are
generated, and packed in an \emph{m}-by-\emph{n}-by-\emph{k} arrangement.  Because
each sample is \emph{N}-dimensional, the output shape is \code{(m,n,k,N)}.
If no shape is specified, a single (\emph{N}-D) sample is returned.

\end{description}
\begin{description}
\item[{out}] \leavevmode{[}ndarray{]}
The drawn samples, of shape \emph{size}, if that was provided.  If not,
the shape is \code{(N,)}.

In other words, each entry \code{out{[}i,j,...,:{]}} is an N-dimensional
value drawn from the distribution.

\end{description}

The mean is a coordinate in N-dimensional space, which represents the
location where samples are most likely to be generated.  This is
analogous to the peak of the bell curve for the one-dimensional or
univariate normal distribution.

Covariance indicates the level to which two variables vary together.
From the multivariate normal distribution, we draw N-dimensional
samples, \(X = [x_1, x_2, ... x_N]\).  The covariance matrix
element \(C_{ij}\) is the covariance of \(x_i\) and \(x_j\).
The element \(C_{ii}\) is the variance of \(x_i\) (i.e. its
``spread'').

Instead of specifying the full covariance matrix, popular
approximations include:
\begin{itemize}
\item {} 
Spherical covariance (\emph{cov} is a multiple of the identity matrix)

\item {} 
Diagonal covariance (\emph{cov} has non-negative elements, and only on
the diagonal)

\end{itemize}

This geometrical property can be seen in two dimensions by plotting
generated data-points:

\begin{Verbatim}[commandchars=\\\{\}]
\PYG{g+gp}{\PYGZgt{}\PYGZgt{}\PYGZgt{} }\PYG{n}{mean} \PYG{o}{=} \PYG{p}{[}\PYG{l+m+mi}{0}\PYG{p}{,}\PYG{l+m+mi}{0}\PYG{p}{]}
\PYG{g+gp}{\PYGZgt{}\PYGZgt{}\PYGZgt{} }\PYG{n}{cov} \PYG{o}{=} \PYG{p}{[}\PYG{p}{[}\PYG{l+m+mi}{1}\PYG{p}{,}\PYG{l+m+mi}{0}\PYG{p}{]}\PYG{p}{,}\PYG{p}{[}\PYG{l+m+mi}{0}\PYG{p}{,}\PYG{l+m+mi}{100}\PYG{p}{]}\PYG{p}{]} \PYG{c}{\PYGZsh{} diagonal covariance, points lie on x or y\PYGZhy{}axis}
\end{Verbatim}

\begin{Verbatim}[commandchars=\\\{\}]
\PYG{g+gp}{\PYGZgt{}\PYGZgt{}\PYGZgt{} }\PYG{k+kn}{import} \PYG{n+nn}{matplotlib.pyplot} \PYG{k+kn}{as} \PYG{n+nn}{plt}
\PYG{g+gp}{\PYGZgt{}\PYGZgt{}\PYGZgt{} }\PYG{n}{x}\PYG{p}{,}\PYG{n}{y} \PYG{o}{=} \PYG{n}{np}\PYG{o}{.}\PYG{n}{random}\PYG{o}{.}\PYG{n}{multivariate\PYGZus{}normal}\PYG{p}{(}\PYG{n}{mean}\PYG{p}{,}\PYG{n}{cov}\PYG{p}{,}\PYG{l+m+mi}{5000}\PYG{p}{)}\PYG{o}{.}\PYG{n}{T}
\PYG{g+gp}{\PYGZgt{}\PYGZgt{}\PYGZgt{} }\PYG{n}{plt}\PYG{o}{.}\PYG{n}{plot}\PYG{p}{(}\PYG{n}{x}\PYG{p}{,}\PYG{n}{y}\PYG{p}{,}\PYG{l+s}{\PYGZsq{}}\PYG{l+s}{x}\PYG{l+s}{\PYGZsq{}}\PYG{p}{)}\PYG{p}{;} \PYG{n}{plt}\PYG{o}{.}\PYG{n}{axis}\PYG{p}{(}\PYG{l+s}{\PYGZsq{}}\PYG{l+s}{equal}\PYG{l+s}{\PYGZsq{}}\PYG{p}{)}\PYG{p}{;} \PYG{n}{plt}\PYG{o}{.}\PYG{n}{show}\PYG{p}{(}\PYG{p}{)}
\end{Verbatim}

Note that the covariance matrix must be non-negative definite.

Papoulis, A., \emph{Probability, Random Variables, and Stochastic Processes},
3rd ed., New York: McGraw-Hill, 1991.

Duda, R. O., Hart, P. E., and Stork, D. G., \emph{Pattern Classification},
2nd ed., New York: Wiley, 2001.

\begin{Verbatim}[commandchars=\\\{\}]
\PYG{g+gp}{\PYGZgt{}\PYGZgt{}\PYGZgt{} }\PYG{n}{mean} \PYG{o}{=} \PYG{p}{(}\PYG{l+m+mi}{1}\PYG{p}{,}\PYG{l+m+mi}{2}\PYG{p}{)}
\PYG{g+gp}{\PYGZgt{}\PYGZgt{}\PYGZgt{} }\PYG{n}{cov} \PYG{o}{=} \PYG{p}{[}\PYG{p}{[}\PYG{l+m+mi}{1}\PYG{p}{,}\PYG{l+m+mi}{0}\PYG{p}{]}\PYG{p}{,}\PYG{p}{[}\PYG{l+m+mi}{1}\PYG{p}{,}\PYG{l+m+mi}{0}\PYG{p}{]}\PYG{p}{]}
\PYG{g+gp}{\PYGZgt{}\PYGZgt{}\PYGZgt{} }\PYG{n}{x} \PYG{o}{=} \PYG{n}{np}\PYG{o}{.}\PYG{n}{random}\PYG{o}{.}\PYG{n}{multivariate\PYGZus{}normal}\PYG{p}{(}\PYG{n}{mean}\PYG{p}{,}\PYG{n}{cov}\PYG{p}{,}\PYG{p}{(}\PYG{l+m+mi}{3}\PYG{p}{,}\PYG{l+m+mi}{3}\PYG{p}{)}\PYG{p}{)}
\PYG{g+gp}{\PYGZgt{}\PYGZgt{}\PYGZgt{} }\PYG{n}{x}\PYG{o}{.}\PYG{n}{shape}
\PYG{g+go}{(3, 3, 2)}
\end{Verbatim}

The following is probably true, given that 0.6 is roughly twice the
standard deviation:

\begin{Verbatim}[commandchars=\\\{\}]
\PYG{g+gp}{\PYGZgt{}\PYGZgt{}\PYGZgt{} }\PYG{k}{print} \PYG{n+nb}{list}\PYG{p}{(} \PYG{p}{(}\PYG{n}{x}\PYG{p}{[}\PYG{l+m+mi}{0}\PYG{p}{,}\PYG{l+m+mi}{0}\PYG{p}{,}\PYG{p}{:}\PYG{p}{]} \PYG{o}{\PYGZhy{}} \PYG{n}{mean}\PYG{p}{)} \PYG{o}{\PYGZlt{}} \PYG{l+m+mf}{0.6} \PYG{p}{)}
\PYG{g+go}{[True, True]}
\end{Verbatim}

\end{fulllineitems}

\index{negative\_binomial() (in module lib.graph.scc)}

\begin{fulllineitems}
\phantomsection\label{lib.graph:lib.graph.scc.negative_binomial}\pysiglinewithargsret{\code{lib.graph.scc.}\bfcode{negative\_binomial}}{\emph{n}, \emph{p}, \emph{size=None}}{}
Draw samples from a negative\_binomial distribution.

Samples are drawn from a negative\_Binomial distribution with specified
parameters, \emph{n} trials and \emph{p} probability of success where \emph{n} is an
integer \textgreater{} 0 and \emph{p} is in the interval {[}0, 1{]}.
\begin{description}
\item[{n}] \leavevmode{[}int{]}
Parameter, \textgreater{} 0.

\item[{p}] \leavevmode{[}float{]}
Parameter, \textgreater{}= 0 and \textless{}=1.

\item[{size}] \leavevmode{[}int or tuple of ints{]}
Output shape. If the given shape is, e.g., \code{(m, n, k)}, then
\code{m * n * k} samples are drawn.

\end{description}
\begin{description}
\item[{samples}] \leavevmode{[}int or ndarray of ints{]}
Drawn samples.

\end{description}

The probability density for the Negative Binomial distribution is
\begin{gather}
\begin{split}P(N;n,p) = \binom{N+n-1}{n-1}p^{n}(1-p)^{N},\end{split}\notag
\end{gather}
where \(n-1\) is the number of successes, \(p\) is the probability
of success, and \(N+n-1\) is the number of trials.

The negative binomial distribution gives the probability of n-1 successes
and N failures in N+n-1 trials, and success on the (N+n)th trial.

If one throws a die repeatedly until the third time a ``1'' appears, then the
probability distribution of the number of non-``1''s that appear before the
third ``1'' is a negative binomial distribution.

Draw samples from the distribution:

A real world example. A company drills wild-cat oil exploration wells, each
with an estimated probability of success of 0.1.  What is the probability
of having one success for each successive well, that is what is the
probability of a single success after drilling 5 wells, after 6 wells,
etc.?

\begin{Verbatim}[commandchars=\\\{\}]
\PYG{g+gp}{\PYGZgt{}\PYGZgt{}\PYGZgt{} }\PYG{n}{s} \PYG{o}{=} \PYG{n}{np}\PYG{o}{.}\PYG{n}{random}\PYG{o}{.}\PYG{n}{negative\PYGZus{}binomial}\PYG{p}{(}\PYG{l+m+mi}{1}\PYG{p}{,} \PYG{l+m+mf}{0.1}\PYG{p}{,} \PYG{l+m+mi}{100000}\PYG{p}{)}
\PYG{g+gp}{\PYGZgt{}\PYGZgt{}\PYGZgt{} }\PYG{k}{for} \PYG{n}{i} \PYG{o+ow}{in} \PYG{n+nb}{range}\PYG{p}{(}\PYG{l+m+mi}{1}\PYG{p}{,} \PYG{l+m+mi}{11}\PYG{p}{)}\PYG{p}{:}
\PYG{g+gp}{... }   \PYG{n}{probability} \PYG{o}{=} \PYG{n+nb}{sum}\PYG{p}{(}\PYG{n}{s}\PYG{o}{\PYGZlt{}}\PYG{n}{i}\PYG{p}{)} \PYG{o}{/} \PYG{l+m+mf}{100000.}
\PYG{g+gp}{... }   \PYG{k}{print} \PYG{n}{i}\PYG{p}{,} \PYG{l+s}{\PYGZdq{}}\PYG{l+s}{wells drilled, probability of one success =}\PYG{l+s}{\PYGZdq{}}\PYG{p}{,} \PYG{n}{probability}
\end{Verbatim}

\end{fulllineitems}

\index{noncentral\_chisquare() (in module lib.graph.scc)}

\begin{fulllineitems}
\phantomsection\label{lib.graph:lib.graph.scc.noncentral_chisquare}\pysiglinewithargsret{\code{lib.graph.scc.}\bfcode{noncentral\_chisquare}}{\emph{df}, \emph{nonc}, \emph{size=None}}{}
Draw samples from a noncentral chi-square distribution.

The noncentral \(\chi^2\) distribution is a generalisation of
the \(\chi^2\) distribution.
\begin{description}
\item[{df}] \leavevmode{[}int{]}
Degrees of freedom, should be \textgreater{}= 1.

\item[{nonc}] \leavevmode{[}float{]}
Non-centrality, should be \textgreater{} 0.

\item[{size}] \leavevmode{[}int or tuple of ints{]}
Shape of the output.

\end{description}

The probability density function for the noncentral Chi-square distribution
is
\begin{gather}
\begin{split}P(x;df,nonc) = \sum^{\infty}_{i=0}
\frac{e^{-nonc/2}(nonc/2)^{i}}{i!}P_{Y_{df+2i}}(x),\end{split}\notag
\end{gather}
where \(Y_{q}\) is the Chi-square with q degrees of freedom.

In Delhi (2007), it is noted that the noncentral chi-square is useful in
bombing and coverage problems, the probability of killing the point target
given by the noncentral chi-squared distribution.

Draw values from the distribution and plot the histogram

\begin{Verbatim}[commandchars=\\\{\}]
\PYG{g+gp}{\PYGZgt{}\PYGZgt{}\PYGZgt{} }\PYG{k+kn}{import} \PYG{n+nn}{matplotlib.pyplot} \PYG{k+kn}{as} \PYG{n+nn}{plt}
\PYG{g+gp}{\PYGZgt{}\PYGZgt{}\PYGZgt{} }\PYG{n}{values} \PYG{o}{=} \PYG{n}{plt}\PYG{o}{.}\PYG{n}{hist}\PYG{p}{(}\PYG{n}{np}\PYG{o}{.}\PYG{n}{random}\PYG{o}{.}\PYG{n}{noncentral\PYGZus{}chisquare}\PYG{p}{(}\PYG{l+m+mi}{3}\PYG{p}{,} \PYG{l+m+mi}{20}\PYG{p}{,} \PYG{l+m+mi}{100000}\PYG{p}{)}\PYG{p}{,}
\PYG{g+gp}{... }                  \PYG{n}{bins}\PYG{o}{=}\PYG{l+m+mi}{200}\PYG{p}{,} \PYG{n}{normed}\PYG{o}{=}\PYG{n+nb+bp}{True}\PYG{p}{)}
\PYG{g+gp}{\PYGZgt{}\PYGZgt{}\PYGZgt{} }\PYG{n}{plt}\PYG{o}{.}\PYG{n}{show}\PYG{p}{(}\PYG{p}{)}
\end{Verbatim}

Draw values from a noncentral chisquare with very small noncentrality,
and compare to a chisquare.

\begin{Verbatim}[commandchars=\\\{\}]
\PYG{g+gp}{\PYGZgt{}\PYGZgt{}\PYGZgt{} }\PYG{n}{plt}\PYG{o}{.}\PYG{n}{figure}\PYG{p}{(}\PYG{p}{)}
\PYG{g+gp}{\PYGZgt{}\PYGZgt{}\PYGZgt{} }\PYG{n}{values} \PYG{o}{=} \PYG{n}{plt}\PYG{o}{.}\PYG{n}{hist}\PYG{p}{(}\PYG{n}{np}\PYG{o}{.}\PYG{n}{random}\PYG{o}{.}\PYG{n}{noncentral\PYGZus{}chisquare}\PYG{p}{(}\PYG{l+m+mi}{3}\PYG{p}{,} \PYG{o}{.}\PYG{l+m+mo}{0000001}\PYG{p}{,} \PYG{l+m+mi}{100000}\PYG{p}{)}\PYG{p}{,}
\PYG{g+gp}{... }                  \PYG{n}{bins}\PYG{o}{=}\PYG{n}{np}\PYG{o}{.}\PYG{n}{arange}\PYG{p}{(}\PYG{l+m+mf}{0.}\PYG{p}{,} \PYG{l+m+mi}{25}\PYG{p}{,} \PYG{o}{.}\PYG{l+m+mi}{1}\PYG{p}{)}\PYG{p}{,} \PYG{n}{normed}\PYG{o}{=}\PYG{n+nb+bp}{True}\PYG{p}{)}
\PYG{g+gp}{\PYGZgt{}\PYGZgt{}\PYGZgt{} }\PYG{n}{values2} \PYG{o}{=} \PYG{n}{plt}\PYG{o}{.}\PYG{n}{hist}\PYG{p}{(}\PYG{n}{np}\PYG{o}{.}\PYG{n}{random}\PYG{o}{.}\PYG{n}{chisquare}\PYG{p}{(}\PYG{l+m+mi}{3}\PYG{p}{,} \PYG{l+m+mi}{100000}\PYG{p}{)}\PYG{p}{,}
\PYG{g+gp}{... }                   \PYG{n}{bins}\PYG{o}{=}\PYG{n}{np}\PYG{o}{.}\PYG{n}{arange}\PYG{p}{(}\PYG{l+m+mf}{0.}\PYG{p}{,} \PYG{l+m+mi}{25}\PYG{p}{,} \PYG{o}{.}\PYG{l+m+mi}{1}\PYG{p}{)}\PYG{p}{,} \PYG{n}{normed}\PYG{o}{=}\PYG{n+nb+bp}{True}\PYG{p}{)}
\PYG{g+gp}{\PYGZgt{}\PYGZgt{}\PYGZgt{} }\PYG{n}{plt}\PYG{o}{.}\PYG{n}{plot}\PYG{p}{(}\PYG{n}{values}\PYG{p}{[}\PYG{l+m+mi}{1}\PYG{p}{]}\PYG{p}{[}\PYG{l+m+mi}{0}\PYG{p}{:}\PYG{o}{\PYGZhy{}}\PYG{l+m+mi}{1}\PYG{p}{]}\PYG{p}{,} \PYG{n}{values}\PYG{p}{[}\PYG{l+m+mi}{0}\PYG{p}{]}\PYG{o}{\PYGZhy{}}\PYG{n}{values2}\PYG{p}{[}\PYG{l+m+mi}{0}\PYG{p}{]}\PYG{p}{,} \PYG{l+s}{\PYGZsq{}}\PYG{l+s}{ob}\PYG{l+s}{\PYGZsq{}}\PYG{p}{)}
\PYG{g+gp}{\PYGZgt{}\PYGZgt{}\PYGZgt{} }\PYG{n}{plt}\PYG{o}{.}\PYG{n}{show}\PYG{p}{(}\PYG{p}{)}
\end{Verbatim}

Demonstrate how large values of non-centrality lead to a more symmetric
distribution.

\begin{Verbatim}[commandchars=\\\{\}]
\PYG{g+gp}{\PYGZgt{}\PYGZgt{}\PYGZgt{} }\PYG{n}{plt}\PYG{o}{.}\PYG{n}{figure}\PYG{p}{(}\PYG{p}{)}
\PYG{g+gp}{\PYGZgt{}\PYGZgt{}\PYGZgt{} }\PYG{n}{values} \PYG{o}{=} \PYG{n}{plt}\PYG{o}{.}\PYG{n}{hist}\PYG{p}{(}\PYG{n}{np}\PYG{o}{.}\PYG{n}{random}\PYG{o}{.}\PYG{n}{noncentral\PYGZus{}chisquare}\PYG{p}{(}\PYG{l+m+mi}{3}\PYG{p}{,} \PYG{l+m+mi}{20}\PYG{p}{,} \PYG{l+m+mi}{100000}\PYG{p}{)}\PYG{p}{,}
\PYG{g+gp}{... }                  \PYG{n}{bins}\PYG{o}{=}\PYG{l+m+mi}{200}\PYG{p}{,} \PYG{n}{normed}\PYG{o}{=}\PYG{n+nb+bp}{True}\PYG{p}{)}
\PYG{g+gp}{\PYGZgt{}\PYGZgt{}\PYGZgt{} }\PYG{n}{plt}\PYG{o}{.}\PYG{n}{show}\PYG{p}{(}\PYG{p}{)}
\end{Verbatim}

\end{fulllineitems}

\index{noncentral\_f() (in module lib.graph.scc)}

\begin{fulllineitems}
\phantomsection\label{lib.graph:lib.graph.scc.noncentral_f}\pysiglinewithargsret{\code{lib.graph.scc.}\bfcode{noncentral\_f}}{\emph{dfnum}, \emph{dfden}, \emph{nonc}, \emph{size=None}}{}
Draw samples from the noncentral F distribution.

Samples are drawn from an F distribution with specified parameters,
\emph{dfnum} (degrees of freedom in numerator) and \emph{dfden} (degrees of
freedom in denominator), where both parameters \textgreater{} 1.
\emph{nonc} is the non-centrality parameter.
\begin{description}
\item[{dfnum}] \leavevmode{[}int{]}
Parameter, should be \textgreater{} 1.

\item[{dfden}] \leavevmode{[}int{]}
Parameter, should be \textgreater{} 1.

\item[{nonc}] \leavevmode{[}float{]}
Parameter, should be \textgreater{}= 0.

\item[{size}] \leavevmode{[}int or tuple of ints{]}
Output shape. If the given shape is, e.g., \code{(m, n, k)}, then
\code{m * n * k} samples are drawn.

\end{description}
\begin{description}
\item[{samples}] \leavevmode{[}scalar or ndarray{]}
Drawn samples.

\end{description}

When calculating the power of an experiment (power = probability of
rejecting the null hypothesis when a specific alternative is true) the
non-central F statistic becomes important.  When the null hypothesis is
true, the F statistic follows a central F distribution. When the null
hypothesis is not true, then it follows a non-central F statistic.

Weisstein, Eric W. ``Noncentral F-Distribution.'' From MathWorld--A Wolfram
Web Resource.  \href{http://mathworld.wolfram.com/NoncentralF-Distribution.html}{http://mathworld.wolfram.com/NoncentralF-Distribution.html}

Wikipedia, ``Noncentral F distribution'',
\href{http://en.wikipedia.org/wiki/Noncentral\_F-distribution}{http://en.wikipedia.org/wiki/Noncentral\_F-distribution}

In a study, testing for a specific alternative to the null hypothesis
requires use of the Noncentral F distribution. We need to calculate the
area in the tail of the distribution that exceeds the value of the F
distribution for the null hypothesis.  We'll plot the two probability
distributions for comparison.

\begin{Verbatim}[commandchars=\\\{\}]
\PYG{g+gp}{\PYGZgt{}\PYGZgt{}\PYGZgt{} }\PYG{n}{dfnum} \PYG{o}{=} \PYG{l+m+mi}{3} \PYG{c}{\PYGZsh{} between group deg of freedom}
\PYG{g+gp}{\PYGZgt{}\PYGZgt{}\PYGZgt{} }\PYG{n}{dfden} \PYG{o}{=} \PYG{l+m+mi}{20} \PYG{c}{\PYGZsh{} within groups degrees of freedom}
\PYG{g+gp}{\PYGZgt{}\PYGZgt{}\PYGZgt{} }\PYG{n}{nonc} \PYG{o}{=} \PYG{l+m+mf}{3.0}
\PYG{g+gp}{\PYGZgt{}\PYGZgt{}\PYGZgt{} }\PYG{n}{nc\PYGZus{}vals} \PYG{o}{=} \PYG{n}{np}\PYG{o}{.}\PYG{n}{random}\PYG{o}{.}\PYG{n}{noncentral\PYGZus{}f}\PYG{p}{(}\PYG{n}{dfnum}\PYG{p}{,} \PYG{n}{dfden}\PYG{p}{,} \PYG{n}{nonc}\PYG{p}{,} \PYG{l+m+mi}{1000000}\PYG{p}{)}
\PYG{g+gp}{\PYGZgt{}\PYGZgt{}\PYGZgt{} }\PYG{n}{NF} \PYG{o}{=} \PYG{n}{np}\PYG{o}{.}\PYG{n}{histogram}\PYG{p}{(}\PYG{n}{nc\PYGZus{}vals}\PYG{p}{,} \PYG{n}{bins}\PYG{o}{=}\PYG{l+m+mi}{50}\PYG{p}{,} \PYG{n}{normed}\PYG{o}{=}\PYG{n+nb+bp}{True}\PYG{p}{)}
\PYG{g+gp}{\PYGZgt{}\PYGZgt{}\PYGZgt{} }\PYG{n}{c\PYGZus{}vals} \PYG{o}{=} \PYG{n}{np}\PYG{o}{.}\PYG{n}{random}\PYG{o}{.}\PYG{n}{f}\PYG{p}{(}\PYG{n}{dfnum}\PYG{p}{,} \PYG{n}{dfden}\PYG{p}{,} \PYG{l+m+mi}{1000000}\PYG{p}{)}
\PYG{g+gp}{\PYGZgt{}\PYGZgt{}\PYGZgt{} }\PYG{n}{F} \PYG{o}{=} \PYG{n}{np}\PYG{o}{.}\PYG{n}{histogram}\PYG{p}{(}\PYG{n}{c\PYGZus{}vals}\PYG{p}{,} \PYG{n}{bins}\PYG{o}{=}\PYG{l+m+mi}{50}\PYG{p}{,} \PYG{n}{normed}\PYG{o}{=}\PYG{n+nb+bp}{True}\PYG{p}{)}
\PYG{g+gp}{\PYGZgt{}\PYGZgt{}\PYGZgt{} }\PYG{n}{plt}\PYG{o}{.}\PYG{n}{plot}\PYG{p}{(}\PYG{n}{F}\PYG{p}{[}\PYG{l+m+mi}{1}\PYG{p}{]}\PYG{p}{[}\PYG{l+m+mi}{1}\PYG{p}{:}\PYG{p}{]}\PYG{p}{,} \PYG{n}{F}\PYG{p}{[}\PYG{l+m+mi}{0}\PYG{p}{]}\PYG{p}{)}
\PYG{g+gp}{\PYGZgt{}\PYGZgt{}\PYGZgt{} }\PYG{n}{plt}\PYG{o}{.}\PYG{n}{plot}\PYG{p}{(}\PYG{n}{NF}\PYG{p}{[}\PYG{l+m+mi}{1}\PYG{p}{]}\PYG{p}{[}\PYG{l+m+mi}{1}\PYG{p}{:}\PYG{p}{]}\PYG{p}{,} \PYG{n}{NF}\PYG{p}{[}\PYG{l+m+mi}{0}\PYG{p}{]}\PYG{p}{)}
\PYG{g+gp}{\PYGZgt{}\PYGZgt{}\PYGZgt{} }\PYG{n}{plt}\PYG{o}{.}\PYG{n}{show}\PYG{p}{(}\PYG{p}{)}
\end{Verbatim}

\end{fulllineitems}

\index{normal() (in module lib.graph.scc)}

\begin{fulllineitems}
\phantomsection\label{lib.graph:lib.graph.scc.normal}\pysiglinewithargsret{\code{lib.graph.scc.}\bfcode{normal}}{\emph{loc=0.0}, \emph{scale=1.0}, \emph{size=None}}{}
Draw random samples from a normal (Gaussian) distribution.

The probability density function of the normal distribution, first
derived by De Moivre and 200 years later by both Gauss and Laplace
independently {\color{red}\bfseries{}{[}2{]}\_}, is often called the bell curve because of
its characteristic shape (see the example below).

The normal distributions occurs often in nature.  For example, it
describes the commonly occurring distribution of samples influenced
by a large number of tiny, random disturbances, each with its own
unique distribution {\color{red}\bfseries{}{[}2{]}\_}.
\begin{description}
\item[{loc}] \leavevmode{[}float{]}
Mean (``centre'') of the distribution.

\item[{scale}] \leavevmode{[}float{]}
Standard deviation (spread or ``width'') of the distribution.

\item[{size}] \leavevmode{[}tuple of ints{]}
Output shape.  If the given shape is, e.g., \code{(m, n, k)}, then
\code{m * n * k} samples are drawn.

\end{description}
\begin{description}
\item[{scipy.stats.distributions.norm}] \leavevmode{[}probability density function,{]}
distribution or cumulative density function, etc.

\end{description}

The probability density for the Gaussian distribution is
\begin{gather}
\begin{split}p(x) = \frac{1}{\sqrt{ 2 \pi \sigma^2 }}
e^{ - \frac{ (x - \mu)^2 } {2 \sigma^2} },\end{split}\notag
\end{gather}
where \(\mu\) is the mean and \(\sigma\) the standard deviation.
The square of the standard deviation, \(\sigma^2\), is called the
variance.

The function has its peak at the mean, and its ``spread'' increases with
the standard deviation (the function reaches 0.607 times its maximum at
\(x + \sigma\) and \(x - \sigma\) {\color{red}\bfseries{}{[}2{]}\_}).  This implies that
\emph{numpy.random.normal} is more likely to return samples lying close to the
mean, rather than those far away.

Draw samples from the distribution:

\begin{Verbatim}[commandchars=\\\{\}]
\PYG{g+gp}{\PYGZgt{}\PYGZgt{}\PYGZgt{} }\PYG{n}{mu}\PYG{p}{,} \PYG{n}{sigma} \PYG{o}{=} \PYG{l+m+mi}{0}\PYG{p}{,} \PYG{l+m+mf}{0.1} \PYG{c}{\PYGZsh{} mean and standard deviation}
\PYG{g+gp}{\PYGZgt{}\PYGZgt{}\PYGZgt{} }\PYG{n}{s} \PYG{o}{=} \PYG{n}{np}\PYG{o}{.}\PYG{n}{random}\PYG{o}{.}\PYG{n}{normal}\PYG{p}{(}\PYG{n}{mu}\PYG{p}{,} \PYG{n}{sigma}\PYG{p}{,} \PYG{l+m+mi}{1000}\PYG{p}{)}
\end{Verbatim}

Verify the mean and the variance:

\begin{Verbatim}[commandchars=\\\{\}]
\PYG{g+gp}{\PYGZgt{}\PYGZgt{}\PYGZgt{} }\PYG{n+nb}{abs}\PYG{p}{(}\PYG{n}{mu} \PYG{o}{\PYGZhy{}} \PYG{n}{np}\PYG{o}{.}\PYG{n}{mean}\PYG{p}{(}\PYG{n}{s}\PYG{p}{)}\PYG{p}{)} \PYG{o}{\PYGZlt{}} \PYG{l+m+mf}{0.01}
\PYG{g+go}{True}
\end{Verbatim}

\begin{Verbatim}[commandchars=\\\{\}]
\PYG{g+gp}{\PYGZgt{}\PYGZgt{}\PYGZgt{} }\PYG{n+nb}{abs}\PYG{p}{(}\PYG{n}{sigma} \PYG{o}{\PYGZhy{}} \PYG{n}{np}\PYG{o}{.}\PYG{n}{std}\PYG{p}{(}\PYG{n}{s}\PYG{p}{,} \PYG{n}{ddof}\PYG{o}{=}\PYG{l+m+mi}{1}\PYG{p}{)}\PYG{p}{)} \PYG{o}{\PYGZlt{}} \PYG{l+m+mf}{0.01}
\PYG{g+go}{True}
\end{Verbatim}

Display the histogram of the samples, along with
the probability density function:

\begin{Verbatim}[commandchars=\\\{\}]
\PYG{g+gp}{\PYGZgt{}\PYGZgt{}\PYGZgt{} }\PYG{k+kn}{import} \PYG{n+nn}{matplotlib.pyplot} \PYG{k+kn}{as} \PYG{n+nn}{plt}
\PYG{g+gp}{\PYGZgt{}\PYGZgt{}\PYGZgt{} }\PYG{n}{count}\PYG{p}{,} \PYG{n}{bins}\PYG{p}{,} \PYG{n}{ignored} \PYG{o}{=} \PYG{n}{plt}\PYG{o}{.}\PYG{n}{hist}\PYG{p}{(}\PYG{n}{s}\PYG{p}{,} \PYG{l+m+mi}{30}\PYG{p}{,} \PYG{n}{normed}\PYG{o}{=}\PYG{n+nb+bp}{True}\PYG{p}{)}
\PYG{g+gp}{\PYGZgt{}\PYGZgt{}\PYGZgt{} }\PYG{n}{plt}\PYG{o}{.}\PYG{n}{plot}\PYG{p}{(}\PYG{n}{bins}\PYG{p}{,} \PYG{l+m+mi}{1}\PYG{o}{/}\PYG{p}{(}\PYG{n}{sigma} \PYG{o}{*} \PYG{n}{np}\PYG{o}{.}\PYG{n}{sqrt}\PYG{p}{(}\PYG{l+m+mi}{2} \PYG{o}{*} \PYG{n}{np}\PYG{o}{.}\PYG{n}{pi}\PYG{p}{)}\PYG{p}{)} \PYG{o}{*}
\PYG{g+gp}{... }               \PYG{n}{np}\PYG{o}{.}\PYG{n}{exp}\PYG{p}{(} \PYG{o}{\PYGZhy{}} \PYG{p}{(}\PYG{n}{bins} \PYG{o}{\PYGZhy{}} \PYG{n}{mu}\PYG{p}{)}\PYG{o}{*}\PYG{o}{*}\PYG{l+m+mi}{2} \PYG{o}{/} \PYG{p}{(}\PYG{l+m+mi}{2} \PYG{o}{*} \PYG{n}{sigma}\PYG{o}{*}\PYG{o}{*}\PYG{l+m+mi}{2}\PYG{p}{)} \PYG{p}{)}\PYG{p}{,}
\PYG{g+gp}{... }         \PYG{n}{linewidth}\PYG{o}{=}\PYG{l+m+mi}{2}\PYG{p}{,} \PYG{n}{color}\PYG{o}{=}\PYG{l+s}{\PYGZsq{}}\PYG{l+s}{r}\PYG{l+s}{\PYGZsq{}}\PYG{p}{)}
\PYG{g+gp}{\PYGZgt{}\PYGZgt{}\PYGZgt{} }\PYG{n}{plt}\PYG{o}{.}\PYG{n}{show}\PYG{p}{(}\PYG{p}{)}
\end{Verbatim}

\end{fulllineitems}

\index{pareto() (in module lib.graph.scc)}

\begin{fulllineitems}
\phantomsection\label{lib.graph:lib.graph.scc.pareto}\pysiglinewithargsret{\code{lib.graph.scc.}\bfcode{pareto}}{\emph{a}, \emph{size=None}}{}
Draw samples from a Pareto II or Lomax distribution with specified shape.

The Lomax or Pareto II distribution is a shifted Pareto distribution. The
classical Pareto distribution can be obtained from the Lomax distribution
by adding the location parameter m, see below. The smallest value of the
Lomax distribution is zero while for the classical Pareto distribution it
is m, where the standard Pareto distribution has location m=1.
Lomax can also be considered as a simplified version of the Generalized
Pareto distribution (available in SciPy), with the scale set to one and
the location set to zero.

The Pareto distribution must be greater than zero, and is unbounded above.
It is also known as the ``80-20 rule''.  In this distribution, 80 percent of
the weights are in the lowest 20 percent of the range, while the other 20
percent fill the remaining 80 percent of the range.
\begin{description}
\item[{shape}] \leavevmode{[}float, \textgreater{} 0.{]}
Shape of the distribution.

\item[{size}] \leavevmode{[}tuple of ints{]}
Output shape.  If the given shape is, e.g., \code{(m, n, k)}, then
\code{m * n * k} samples are drawn.

\end{description}
\begin{description}
\item[{scipy.stats.distributions.lomax.pdf}] \leavevmode{[}probability density function,{]}
distribution or cumulative density function, etc.

\item[{scipy.stats.distributions.genpareto.pdf}] \leavevmode{[}probability density function,{]}
distribution or cumulative density function, etc.

\end{description}

The probability density for the Pareto distribution is
\begin{gather}
\begin{split}p(x) = \frac{am^a}{x^{a+1}}\end{split}\notag
\end{gather}
where \(a\) is the shape and \(m\) the location

The Pareto distribution, named after the Italian economist Vilfredo Pareto,
is a power law probability distribution useful in many real world problems.
Outside the field of economics it is generally referred to as the Bradford
distribution. Pareto developed the distribution to describe the
distribution of wealth in an economy.  It has also found use in insurance,
web page access statistics, oil field sizes, and many other problems,
including the download frequency for projects in Sourceforge {[}1{]}.  It is
one of the so-called ``fat-tailed'' distributions.

Draw samples from the distribution:

\begin{Verbatim}[commandchars=\\\{\}]
\PYG{g+gp}{\PYGZgt{}\PYGZgt{}\PYGZgt{} }\PYG{n}{a}\PYG{p}{,} \PYG{n}{m} \PYG{o}{=} \PYG{l+m+mf}{3.}\PYG{p}{,} \PYG{l+m+mf}{1.} \PYG{c}{\PYGZsh{} shape and mode}
\PYG{g+gp}{\PYGZgt{}\PYGZgt{}\PYGZgt{} }\PYG{n}{s} \PYG{o}{=} \PYG{n}{np}\PYG{o}{.}\PYG{n}{random}\PYG{o}{.}\PYG{n}{pareto}\PYG{p}{(}\PYG{n}{a}\PYG{p}{,} \PYG{l+m+mi}{1000}\PYG{p}{)} \PYG{o}{+} \PYG{n}{m}
\end{Verbatim}

Display the histogram of the samples, along with
the probability density function:

\begin{Verbatim}[commandchars=\\\{\}]
\PYG{g+gp}{\PYGZgt{}\PYGZgt{}\PYGZgt{} }\PYG{k+kn}{import} \PYG{n+nn}{matplotlib.pyplot} \PYG{k+kn}{as} \PYG{n+nn}{plt}
\PYG{g+gp}{\PYGZgt{}\PYGZgt{}\PYGZgt{} }\PYG{n}{count}\PYG{p}{,} \PYG{n}{bins}\PYG{p}{,} \PYG{n}{ignored} \PYG{o}{=} \PYG{n}{plt}\PYG{o}{.}\PYG{n}{hist}\PYG{p}{(}\PYG{n}{s}\PYG{p}{,} \PYG{l+m+mi}{100}\PYG{p}{,} \PYG{n}{normed}\PYG{o}{=}\PYG{n+nb+bp}{True}\PYG{p}{,} \PYG{n}{align}\PYG{o}{=}\PYG{l+s}{\PYGZsq{}}\PYG{l+s}{center}\PYG{l+s}{\PYGZsq{}}\PYG{p}{)}
\PYG{g+gp}{\PYGZgt{}\PYGZgt{}\PYGZgt{} }\PYG{n}{fit} \PYG{o}{=} \PYG{n}{a}\PYG{o}{*}\PYG{n}{m}\PYG{o}{*}\PYG{o}{*}\PYG{n}{a}\PYG{o}{/}\PYG{n}{bins}\PYG{o}{*}\PYG{o}{*}\PYG{p}{(}\PYG{n}{a}\PYG{o}{+}\PYG{l+m+mi}{1}\PYG{p}{)}
\PYG{g+gp}{\PYGZgt{}\PYGZgt{}\PYGZgt{} }\PYG{n}{plt}\PYG{o}{.}\PYG{n}{plot}\PYG{p}{(}\PYG{n}{bins}\PYG{p}{,} \PYG{n+nb}{max}\PYG{p}{(}\PYG{n}{count}\PYG{p}{)}\PYG{o}{*}\PYG{n}{fit}\PYG{o}{/}\PYG{n+nb}{max}\PYG{p}{(}\PYG{n}{fit}\PYG{p}{)}\PYG{p}{,}\PYG{n}{linewidth}\PYG{o}{=}\PYG{l+m+mi}{2}\PYG{p}{,} \PYG{n}{color}\PYG{o}{=}\PYG{l+s}{\PYGZsq{}}\PYG{l+s}{r}\PYG{l+s}{\PYGZsq{}}\PYG{p}{)}
\PYG{g+gp}{\PYGZgt{}\PYGZgt{}\PYGZgt{} }\PYG{n}{plt}\PYG{o}{.}\PYG{n}{show}\PYG{p}{(}\PYG{p}{)}
\end{Verbatim}

\end{fulllineitems}

\index{permutation() (in module lib.graph.scc)}

\begin{fulllineitems}
\phantomsection\label{lib.graph:lib.graph.scc.permutation}\pysiglinewithargsret{\code{lib.graph.scc.}\bfcode{permutation}}{\emph{x}}{}
Randomly permute a sequence, or return a permuted range.

If \emph{x} is a multi-dimensional array, it is only shuffled along its
first index.
\begin{description}
\item[{x}] \leavevmode{[}int or array\_like{]}
If \emph{x} is an integer, randomly permute \code{np.arange(x)}.
If \emph{x} is an array, make a copy and shuffle the elements
randomly.

\end{description}
\begin{description}
\item[{out}] \leavevmode{[}ndarray{]}
Permuted sequence or array range.

\end{description}

\begin{Verbatim}[commandchars=\\\{\}]
\PYG{g+gp}{\PYGZgt{}\PYGZgt{}\PYGZgt{} }\PYG{n}{np}\PYG{o}{.}\PYG{n}{random}\PYG{o}{.}\PYG{n}{permutation}\PYG{p}{(}\PYG{l+m+mi}{10}\PYG{p}{)}
\PYG{g+go}{array([1, 7, 4, 3, 0, 9, 2, 5, 8, 6])}
\end{Verbatim}

\begin{Verbatim}[commandchars=\\\{\}]
\PYG{g+gp}{\PYGZgt{}\PYGZgt{}\PYGZgt{} }\PYG{n}{np}\PYG{o}{.}\PYG{n}{random}\PYG{o}{.}\PYG{n}{permutation}\PYG{p}{(}\PYG{p}{[}\PYG{l+m+mi}{1}\PYG{p}{,} \PYG{l+m+mi}{4}\PYG{p}{,} \PYG{l+m+mi}{9}\PYG{p}{,} \PYG{l+m+mi}{12}\PYG{p}{,} \PYG{l+m+mi}{15}\PYG{p}{]}\PYG{p}{)}
\PYG{g+go}{array([15,  1,  9,  4, 12])}
\end{Verbatim}

\begin{Verbatim}[commandchars=\\\{\}]
\PYG{g+gp}{\PYGZgt{}\PYGZgt{}\PYGZgt{} }\PYG{n}{arr} \PYG{o}{=} \PYG{n}{np}\PYG{o}{.}\PYG{n}{arange}\PYG{p}{(}\PYG{l+m+mi}{9}\PYG{p}{)}\PYG{o}{.}\PYG{n}{reshape}\PYG{p}{(}\PYG{p}{(}\PYG{l+m+mi}{3}\PYG{p}{,} \PYG{l+m+mi}{3}\PYG{p}{)}\PYG{p}{)}
\PYG{g+gp}{\PYGZgt{}\PYGZgt{}\PYGZgt{} }\PYG{n}{np}\PYG{o}{.}\PYG{n}{random}\PYG{o}{.}\PYG{n}{permutation}\PYG{p}{(}\PYG{n}{arr}\PYG{p}{)}
\PYG{g+go}{array([[6, 7, 8],}
\PYG{g+go}{       [0, 1, 2],}
\PYG{g+go}{       [3, 4, 5]])}
\end{Verbatim}

\end{fulllineitems}

\index{poisson() (in module lib.graph.scc)}

\begin{fulllineitems}
\phantomsection\label{lib.graph:lib.graph.scc.poisson}\pysiglinewithargsret{\code{lib.graph.scc.}\bfcode{poisson}}{\emph{lam=1.0}, \emph{size=None}}{}
Draw samples from a Poisson distribution.

The Poisson distribution is the limit of the Binomial
distribution for large N.
\begin{description}
\item[{lam}] \leavevmode{[}float{]}
Expectation of interval, should be \textgreater{}= 0.

\item[{size}] \leavevmode{[}int or tuple of ints, optional{]}
Output shape. If the given shape is, e.g., \code{(m, n, k)}, then
\code{m * n * k} samples are drawn.

\end{description}

The Poisson distribution
\begin{gather}
\begin{split}f(k; \lambda)=\frac{\lambda^k e^{-\lambda}}{k!}\end{split}\notag
\end{gather}
For events with an expected separation \(\lambda\) the Poisson
distribution \(f(k; \lambda)\) describes the probability of
\(k\) events occurring within the observed interval \(\lambda\).

Because the output is limited to the range of the C long type, a
ValueError is raised when \emph{lam} is within 10 sigma of the maximum
representable value.

Draw samples from the distribution:

\begin{Verbatim}[commandchars=\\\{\}]
\PYG{g+gp}{\PYGZgt{}\PYGZgt{}\PYGZgt{} }\PYG{k+kn}{import} \PYG{n+nn}{numpy} \PYG{k+kn}{as} \PYG{n+nn}{np}
\PYG{g+gp}{\PYGZgt{}\PYGZgt{}\PYGZgt{} }\PYG{n}{s} \PYG{o}{=} \PYG{n}{np}\PYG{o}{.}\PYG{n}{random}\PYG{o}{.}\PYG{n}{poisson}\PYG{p}{(}\PYG{l+m+mi}{5}\PYG{p}{,} \PYG{l+m+mi}{10000}\PYG{p}{)}
\end{Verbatim}

Display histogram of the sample:

\begin{Verbatim}[commandchars=\\\{\}]
\PYG{g+gp}{\PYGZgt{}\PYGZgt{}\PYGZgt{} }\PYG{k+kn}{import} \PYG{n+nn}{matplotlib.pyplot} \PYG{k+kn}{as} \PYG{n+nn}{plt}
\PYG{g+gp}{\PYGZgt{}\PYGZgt{}\PYGZgt{} }\PYG{n}{count}\PYG{p}{,} \PYG{n}{bins}\PYG{p}{,} \PYG{n}{ignored} \PYG{o}{=} \PYG{n}{plt}\PYG{o}{.}\PYG{n}{hist}\PYG{p}{(}\PYG{n}{s}\PYG{p}{,} \PYG{l+m+mi}{14}\PYG{p}{,} \PYG{n}{normed}\PYG{o}{=}\PYG{n+nb+bp}{True}\PYG{p}{)}
\PYG{g+gp}{\PYGZgt{}\PYGZgt{}\PYGZgt{} }\PYG{n}{plt}\PYG{o}{.}\PYG{n}{show}\PYG{p}{(}\PYG{p}{)}
\end{Verbatim}

\end{fulllineitems}

\index{power() (in module lib.graph.scc)}

\begin{fulllineitems}
\phantomsection\label{lib.graph:lib.graph.scc.power}\pysiglinewithargsret{\code{lib.graph.scc.}\bfcode{power}}{\emph{a}, \emph{size=None}}{}
Draws samples in {[}0, 1{]} from a power distribution with positive
exponent a - 1.

Also known as the power function distribution.
\begin{description}
\item[{a}] \leavevmode{[}float{]}
parameter, \textgreater{} 0

\item[{size}] \leavevmode{[}tuple of ints{]}\begin{description}
\item[{Output shape.  If the given shape is, e.g., \code{(m, n, k)}, then}] \leavevmode
\code{m * n * k} samples are drawn.

\end{description}

\end{description}
\begin{description}
\item[{samples}] \leavevmode{[}\{ndarray, scalar\}{]}
The returned samples lie in {[}0, 1{]}.

\end{description}
\begin{description}
\item[{ValueError}] \leavevmode
If a\textless{}1.

\end{description}

The probability density function is
\begin{gather}
\begin{split}P(x; a) = ax^{a-1}, 0 \le x \le 1, a>0.\end{split}\notag
\end{gather}
The power function distribution is just the inverse of the Pareto
distribution. It may also be seen as a special case of the Beta
distribution.

It is used, for example, in modeling the over-reporting of insurance
claims.

Draw samples from the distribution:

\begin{Verbatim}[commandchars=\\\{\}]
\PYG{g+gp}{\PYGZgt{}\PYGZgt{}\PYGZgt{} }\PYG{n}{a} \PYG{o}{=} \PYG{l+m+mf}{5.} \PYG{c}{\PYGZsh{} shape}
\PYG{g+gp}{\PYGZgt{}\PYGZgt{}\PYGZgt{} }\PYG{n}{samples} \PYG{o}{=} \PYG{l+m+mi}{1000}
\PYG{g+gp}{\PYGZgt{}\PYGZgt{}\PYGZgt{} }\PYG{n}{s} \PYG{o}{=} \PYG{n}{np}\PYG{o}{.}\PYG{n}{random}\PYG{o}{.}\PYG{n}{power}\PYG{p}{(}\PYG{n}{a}\PYG{p}{,} \PYG{n}{samples}\PYG{p}{)}
\end{Verbatim}

Display the histogram of the samples, along with
the probability density function:

\begin{Verbatim}[commandchars=\\\{\}]
\PYG{g+gp}{\PYGZgt{}\PYGZgt{}\PYGZgt{} }\PYG{k+kn}{import} \PYG{n+nn}{matplotlib.pyplot} \PYG{k+kn}{as} \PYG{n+nn}{plt}
\PYG{g+gp}{\PYGZgt{}\PYGZgt{}\PYGZgt{} }\PYG{n}{count}\PYG{p}{,} \PYG{n}{bins}\PYG{p}{,} \PYG{n}{ignored} \PYG{o}{=} \PYG{n}{plt}\PYG{o}{.}\PYG{n}{hist}\PYG{p}{(}\PYG{n}{s}\PYG{p}{,} \PYG{n}{bins}\PYG{o}{=}\PYG{l+m+mi}{30}\PYG{p}{)}
\PYG{g+gp}{\PYGZgt{}\PYGZgt{}\PYGZgt{} }\PYG{n}{x} \PYG{o}{=} \PYG{n}{np}\PYG{o}{.}\PYG{n}{linspace}\PYG{p}{(}\PYG{l+m+mi}{0}\PYG{p}{,} \PYG{l+m+mi}{1}\PYG{p}{,} \PYG{l+m+mi}{100}\PYG{p}{)}
\PYG{g+gp}{\PYGZgt{}\PYGZgt{}\PYGZgt{} }\PYG{n}{y} \PYG{o}{=} \PYG{n}{a}\PYG{o}{*}\PYG{n}{x}\PYG{o}{*}\PYG{o}{*}\PYG{p}{(}\PYG{n}{a}\PYG{o}{\PYGZhy{}}\PYG{l+m+mf}{1.}\PYG{p}{)}
\PYG{g+gp}{\PYGZgt{}\PYGZgt{}\PYGZgt{} }\PYG{n}{normed\PYGZus{}y} \PYG{o}{=} \PYG{n}{samples}\PYG{o}{*}\PYG{n}{np}\PYG{o}{.}\PYG{n}{diff}\PYG{p}{(}\PYG{n}{bins}\PYG{p}{)}\PYG{p}{[}\PYG{l+m+mi}{0}\PYG{p}{]}\PYG{o}{*}\PYG{n}{y}
\PYG{g+gp}{\PYGZgt{}\PYGZgt{}\PYGZgt{} }\PYG{n}{plt}\PYG{o}{.}\PYG{n}{plot}\PYG{p}{(}\PYG{n}{x}\PYG{p}{,} \PYG{n}{normed\PYGZus{}y}\PYG{p}{)}
\PYG{g+gp}{\PYGZgt{}\PYGZgt{}\PYGZgt{} }\PYG{n}{plt}\PYG{o}{.}\PYG{n}{show}\PYG{p}{(}\PYG{p}{)}
\end{Verbatim}

Compare the power function distribution to the inverse of the Pareto.

\begin{Verbatim}[commandchars=\\\{\}]
\PYG{g+gp}{\PYGZgt{}\PYGZgt{}\PYGZgt{} }\PYG{k+kn}{from} \PYG{n+nn}{scipy} \PYG{k+kn}{import} \PYG{n}{stats}
\PYG{g+gp}{\PYGZgt{}\PYGZgt{}\PYGZgt{} }\PYG{n}{rvs} \PYG{o}{=} \PYG{n}{np}\PYG{o}{.}\PYG{n}{random}\PYG{o}{.}\PYG{n}{power}\PYG{p}{(}\PYG{l+m+mi}{5}\PYG{p}{,} \PYG{l+m+mi}{1000000}\PYG{p}{)}
\PYG{g+gp}{\PYGZgt{}\PYGZgt{}\PYGZgt{} }\PYG{n}{rvsp} \PYG{o}{=} \PYG{n}{np}\PYG{o}{.}\PYG{n}{random}\PYG{o}{.}\PYG{n}{pareto}\PYG{p}{(}\PYG{l+m+mi}{5}\PYG{p}{,} \PYG{l+m+mi}{1000000}\PYG{p}{)}
\PYG{g+gp}{\PYGZgt{}\PYGZgt{}\PYGZgt{} }\PYG{n}{xx} \PYG{o}{=} \PYG{n}{np}\PYG{o}{.}\PYG{n}{linspace}\PYG{p}{(}\PYG{l+m+mi}{0}\PYG{p}{,}\PYG{l+m+mi}{1}\PYG{p}{,}\PYG{l+m+mi}{100}\PYG{p}{)}
\PYG{g+gp}{\PYGZgt{}\PYGZgt{}\PYGZgt{} }\PYG{n}{powpdf} \PYG{o}{=} \PYG{n}{stats}\PYG{o}{.}\PYG{n}{powerlaw}\PYG{o}{.}\PYG{n}{pdf}\PYG{p}{(}\PYG{n}{xx}\PYG{p}{,}\PYG{l+m+mi}{5}\PYG{p}{)}
\end{Verbatim}

\begin{Verbatim}[commandchars=\\\{\}]
\PYG{g+gp}{\PYGZgt{}\PYGZgt{}\PYGZgt{} }\PYG{n}{plt}\PYG{o}{.}\PYG{n}{figure}\PYG{p}{(}\PYG{p}{)}
\PYG{g+gp}{\PYGZgt{}\PYGZgt{}\PYGZgt{} }\PYG{n}{plt}\PYG{o}{.}\PYG{n}{hist}\PYG{p}{(}\PYG{n}{rvs}\PYG{p}{,} \PYG{n}{bins}\PYG{o}{=}\PYG{l+m+mi}{50}\PYG{p}{,} \PYG{n}{normed}\PYG{o}{=}\PYG{n+nb+bp}{True}\PYG{p}{)}
\PYG{g+gp}{\PYGZgt{}\PYGZgt{}\PYGZgt{} }\PYG{n}{plt}\PYG{o}{.}\PYG{n}{plot}\PYG{p}{(}\PYG{n}{xx}\PYG{p}{,}\PYG{n}{powpdf}\PYG{p}{,}\PYG{l+s}{\PYGZsq{}}\PYG{l+s}{r\PYGZhy{}}\PYG{l+s}{\PYGZsq{}}\PYG{p}{)}
\PYG{g+gp}{\PYGZgt{}\PYGZgt{}\PYGZgt{} }\PYG{n}{plt}\PYG{o}{.}\PYG{n}{title}\PYG{p}{(}\PYG{l+s}{\PYGZsq{}}\PYG{l+s}{np.random.power(5)}\PYG{l+s}{\PYGZsq{}}\PYG{p}{)}
\end{Verbatim}

\begin{Verbatim}[commandchars=\\\{\}]
\PYG{g+gp}{\PYGZgt{}\PYGZgt{}\PYGZgt{} }\PYG{n}{plt}\PYG{o}{.}\PYG{n}{figure}\PYG{p}{(}\PYG{p}{)}
\PYG{g+gp}{\PYGZgt{}\PYGZgt{}\PYGZgt{} }\PYG{n}{plt}\PYG{o}{.}\PYG{n}{hist}\PYG{p}{(}\PYG{l+m+mf}{1.}\PYG{o}{/}\PYG{p}{(}\PYG{l+m+mf}{1.}\PYG{o}{+}\PYG{n}{rvsp}\PYG{p}{)}\PYG{p}{,} \PYG{n}{bins}\PYG{o}{=}\PYG{l+m+mi}{50}\PYG{p}{,} \PYG{n}{normed}\PYG{o}{=}\PYG{n+nb+bp}{True}\PYG{p}{)}
\PYG{g+gp}{\PYGZgt{}\PYGZgt{}\PYGZgt{} }\PYG{n}{plt}\PYG{o}{.}\PYG{n}{plot}\PYG{p}{(}\PYG{n}{xx}\PYG{p}{,}\PYG{n}{powpdf}\PYG{p}{,}\PYG{l+s}{\PYGZsq{}}\PYG{l+s}{r\PYGZhy{}}\PYG{l+s}{\PYGZsq{}}\PYG{p}{)}
\PYG{g+gp}{\PYGZgt{}\PYGZgt{}\PYGZgt{} }\PYG{n}{plt}\PYG{o}{.}\PYG{n}{title}\PYG{p}{(}\PYG{l+s}{\PYGZsq{}}\PYG{l+s}{inverse of 1 + np.random.pareto(5)}\PYG{l+s}{\PYGZsq{}}\PYG{p}{)}
\end{Verbatim}

\begin{Verbatim}[commandchars=\\\{\}]
\PYG{g+gp}{\PYGZgt{}\PYGZgt{}\PYGZgt{} }\PYG{n}{plt}\PYG{o}{.}\PYG{n}{figure}\PYG{p}{(}\PYG{p}{)}
\PYG{g+gp}{\PYGZgt{}\PYGZgt{}\PYGZgt{} }\PYG{n}{plt}\PYG{o}{.}\PYG{n}{hist}\PYG{p}{(}\PYG{l+m+mf}{1.}\PYG{o}{/}\PYG{p}{(}\PYG{l+m+mf}{1.}\PYG{o}{+}\PYG{n}{rvsp}\PYG{p}{)}\PYG{p}{,} \PYG{n}{bins}\PYG{o}{=}\PYG{l+m+mi}{50}\PYG{p}{,} \PYG{n}{normed}\PYG{o}{=}\PYG{n+nb+bp}{True}\PYG{p}{)}
\PYG{g+gp}{\PYGZgt{}\PYGZgt{}\PYGZgt{} }\PYG{n}{plt}\PYG{o}{.}\PYG{n}{plot}\PYG{p}{(}\PYG{n}{xx}\PYG{p}{,}\PYG{n}{powpdf}\PYG{p}{,}\PYG{l+s}{\PYGZsq{}}\PYG{l+s}{r\PYGZhy{}}\PYG{l+s}{\PYGZsq{}}\PYG{p}{)}
\PYG{g+gp}{\PYGZgt{}\PYGZgt{}\PYGZgt{} }\PYG{n}{plt}\PYG{o}{.}\PYG{n}{title}\PYG{p}{(}\PYG{l+s}{\PYGZsq{}}\PYG{l+s}{inverse of stats.pareto(5)}\PYG{l+s}{\PYGZsq{}}\PYG{p}{)}
\end{Verbatim}

\end{fulllineitems}

\index{printSCConFile() (in module lib.graph.scc)}

\begin{fulllineitems}
\phantomsection\label{lib.graph:lib.graph.scc.printSCConFile}\pysiglinewithargsret{\code{lib.graph.scc.}\bfcode{printSCConFile}}{\emph{tmpSCCL}, \emph{tmpfolderName}, \emph{filePrefix}}{}
\end{fulllineitems}

\index{rand() (in module lib.graph.scc)}

\begin{fulllineitems}
\phantomsection\label{lib.graph:lib.graph.scc.rand}\pysiglinewithargsret{\code{lib.graph.scc.}\bfcode{rand}}{\emph{d0}, \emph{d1}, \emph{...}, \emph{dn}}{}
Random values in a given shape.

Create an array of the given shape and propagate it with
random samples from a uniform distribution
over \code{{[}0, 1)}.
\begin{description}
\item[{d0, d1, ..., dn}] \leavevmode{[}int, optional{]}
The dimensions of the returned array, should all be positive.
If no argument is given a single Python float is returned.

\end{description}
\begin{description}
\item[{out}] \leavevmode{[}ndarray, shape \code{(d0, d1, ..., dn)}{]}
Random values.

\end{description}

random

This is a convenience function. If you want an interface that
takes a shape-tuple as the first argument, refer to
np.random.random\_sample .

\begin{Verbatim}[commandchars=\\\{\}]
\PYG{g+gp}{\PYGZgt{}\PYGZgt{}\PYGZgt{} }\PYG{n}{np}\PYG{o}{.}\PYG{n}{random}\PYG{o}{.}\PYG{n}{rand}\PYG{p}{(}\PYG{l+m+mi}{3}\PYG{p}{,}\PYG{l+m+mi}{2}\PYG{p}{)}
\PYG{g+go}{array([[ 0.14022471,  0.96360618],  \PYGZsh{}random}
\PYG{g+go}{       [ 0.37601032,  0.25528411],  \PYGZsh{}random}
\PYG{g+go}{       [ 0.49313049,  0.94909878]]) \PYGZsh{}random}
\end{Verbatim}

\end{fulllineitems}

\index{randint() (in module lib.graph.scc)}

\begin{fulllineitems}
\phantomsection\label{lib.graph:lib.graph.scc.randint}\pysiglinewithargsret{\code{lib.graph.scc.}\bfcode{randint}}{\emph{low}, \emph{high=None}, \emph{size=None}}{}
Return random integers from \emph{low} (inclusive) to \emph{high} (exclusive).

Return random integers from the ``discrete uniform'' distribution in the
``half-open'' interval {[}\emph{low}, \emph{high}). If \emph{high} is None (the default),
then results are from {[}0, \emph{low}).
\begin{description}
\item[{low}] \leavevmode{[}int{]}
Lowest (signed) integer to be drawn from the distribution (unless
\code{high=None}, in which case this parameter is the \emph{highest} such
integer).

\item[{high}] \leavevmode{[}int, optional{]}
If provided, one above the largest (signed) integer to be drawn
from the distribution (see above for behavior if \code{high=None}).

\item[{size}] \leavevmode{[}int or tuple of ints, optional{]}
Output shape. Default is None, in which case a single int is
returned.

\end{description}
\begin{description}
\item[{out}] \leavevmode{[}int or ndarray of ints{]}
\emph{size}-shaped array of random integers from the appropriate
distribution, or a single such random int if \emph{size} not provided.

\end{description}
\begin{description}
\item[{random.random\_integers}] \leavevmode{[}similar to \emph{randint}, only for the closed{]}
interval {[}\emph{low}, \emph{high}{]}, and 1 is the lowest value if \emph{high} is
omitted. In particular, this other one is the one to use to generate
uniformly distributed discrete non-integers.

\end{description}

\begin{Verbatim}[commandchars=\\\{\}]
\PYG{g+gp}{\PYGZgt{}\PYGZgt{}\PYGZgt{} }\PYG{n}{np}\PYG{o}{.}\PYG{n}{random}\PYG{o}{.}\PYG{n}{randint}\PYG{p}{(}\PYG{l+m+mi}{2}\PYG{p}{,} \PYG{n}{size}\PYG{o}{=}\PYG{l+m+mi}{10}\PYG{p}{)}
\PYG{g+go}{array([1, 0, 0, 0, 1, 1, 0, 0, 1, 0])}
\PYG{g+gp}{\PYGZgt{}\PYGZgt{}\PYGZgt{} }\PYG{n}{np}\PYG{o}{.}\PYG{n}{random}\PYG{o}{.}\PYG{n}{randint}\PYG{p}{(}\PYG{l+m+mi}{1}\PYG{p}{,} \PYG{n}{size}\PYG{o}{=}\PYG{l+m+mi}{10}\PYG{p}{)}
\PYG{g+go}{array([0, 0, 0, 0, 0, 0, 0, 0, 0, 0])}
\end{Verbatim}

Generate a 2 x 4 array of ints between 0 and 4, inclusive:

\begin{Verbatim}[commandchars=\\\{\}]
\PYG{g+gp}{\PYGZgt{}\PYGZgt{}\PYGZgt{} }\PYG{n}{np}\PYG{o}{.}\PYG{n}{random}\PYG{o}{.}\PYG{n}{randint}\PYG{p}{(}\PYG{l+m+mi}{5}\PYG{p}{,} \PYG{n}{size}\PYG{o}{=}\PYG{p}{(}\PYG{l+m+mi}{2}\PYG{p}{,} \PYG{l+m+mi}{4}\PYG{p}{)}\PYG{p}{)}
\PYG{g+go}{array([[4, 0, 2, 1],}
\PYG{g+go}{       [3, 2, 2, 0]])}
\end{Verbatim}

\end{fulllineitems}

\index{randn() (in module lib.graph.scc)}

\begin{fulllineitems}
\phantomsection\label{lib.graph:lib.graph.scc.randn}\pysiglinewithargsret{\code{lib.graph.scc.}\bfcode{randn}}{\emph{d0}, \emph{d1}, \emph{...}, \emph{dn}}{}
Return a sample (or samples) from the ``standard normal'' distribution.

If positive, int\_like or int-convertible arguments are provided,
\emph{randn} generates an array of shape \code{(d0, d1, ..., dn)}, filled
with random floats sampled from a univariate ``normal'' (Gaussian)
distribution of mean 0 and variance 1 (if any of the \(d_i\) are
floats, they are first converted to integers by truncation). A single
float randomly sampled from the distribution is returned if no
argument is provided.

This is a convenience function.  If you want an interface that takes a
tuple as the first argument, use \emph{numpy.random.standard\_normal} instead.
\begin{description}
\item[{d0, d1, ..., dn}] \leavevmode{[}int, optional{]}
The dimensions of the returned array, should be all positive.
If no argument is given a single Python float is returned.

\end{description}
\begin{description}
\item[{Z}] \leavevmode{[}ndarray or float{]}
A \code{(d0, d1, ..., dn)}-shaped array of floating-point samples from
the standard normal distribution, or a single such float if
no parameters were supplied.

\end{description}

random.standard\_normal : Similar, but takes a tuple as its argument.

For random samples from \(N(\mu, \sigma^2)\), use:

\code{sigma * np.random.randn(...) + mu}

\begin{Verbatim}[commandchars=\\\{\}]
\PYG{g+gp}{\PYGZgt{}\PYGZgt{}\PYGZgt{} }\PYG{n}{np}\PYG{o}{.}\PYG{n}{random}\PYG{o}{.}\PYG{n}{randn}\PYG{p}{(}\PYG{p}{)}
\PYG{g+go}{2.1923875335537315 \PYGZsh{}random}
\end{Verbatim}

Two-by-four array of samples from N(3, 6.25):

\begin{Verbatim}[commandchars=\\\{\}]
\PYG{g+gp}{\PYGZgt{}\PYGZgt{}\PYGZgt{} }\PYG{l+m+mf}{2.5} \PYG{o}{*} \PYG{n}{np}\PYG{o}{.}\PYG{n}{random}\PYG{o}{.}\PYG{n}{randn}\PYG{p}{(}\PYG{l+m+mi}{2}\PYG{p}{,} \PYG{l+m+mi}{4}\PYG{p}{)} \PYG{o}{+} \PYG{l+m+mi}{3}
\PYG{g+go}{array([[\PYGZhy{}4.49401501,  4.00950034, \PYGZhy{}1.81814867,  7.29718677],  \PYGZsh{}random}
\PYG{g+go}{       [ 0.39924804,  4.68456316,  4.99394529,  4.84057254]]) \PYGZsh{}random}
\end{Verbatim}

\end{fulllineitems}

\index{random() (in module lib.graph.scc)}

\begin{fulllineitems}
\phantomsection\label{lib.graph:lib.graph.scc.random}\pysiglinewithargsret{\code{lib.graph.scc.}\bfcode{random}}{}{}
random\_sample(size=None)

Return random floats in the half-open interval {[}0.0, 1.0).

Results are from the ``continuous uniform'' distribution over the
stated interval.  To sample \(Unif[a, b), b > a\) multiply
the output of \emph{random\_sample} by \emph{(b-a)} and add \emph{a}:

\begin{Verbatim}[commandchars=\\\{\}]
\PYG{p}{(}\PYG{n}{b} \PYG{o}{\PYGZhy{}} \PYG{n}{a}\PYG{p}{)} \PYG{o}{*} \PYG{n}{random\PYGZus{}sample}\PYG{p}{(}\PYG{p}{)} \PYG{o}{+} \PYG{n}{a}
\end{Verbatim}
\begin{description}
\item[{size}] \leavevmode{[}int or tuple of ints, optional{]}
Defines the shape of the returned array of random floats. If None
(the default), returns a single float.

\end{description}
\begin{description}
\item[{out}] \leavevmode{[}float or ndarray of floats{]}
Array of random floats of shape \emph{size} (unless \code{size=None}, in which
case a single float is returned).

\end{description}

\begin{Verbatim}[commandchars=\\\{\}]
\PYG{g+gp}{\PYGZgt{}\PYGZgt{}\PYGZgt{} }\PYG{n}{np}\PYG{o}{.}\PYG{n}{random}\PYG{o}{.}\PYG{n}{random\PYGZus{}sample}\PYG{p}{(}\PYG{p}{)}
\PYG{g+go}{0.47108547995356098}
\PYG{g+gp}{\PYGZgt{}\PYGZgt{}\PYGZgt{} }\PYG{n+nb}{type}\PYG{p}{(}\PYG{n}{np}\PYG{o}{.}\PYG{n}{random}\PYG{o}{.}\PYG{n}{random\PYGZus{}sample}\PYG{p}{(}\PYG{p}{)}\PYG{p}{)}
\PYG{g+go}{\PYGZlt{}type \PYGZsq{}float\PYGZsq{}\PYGZgt{}}
\PYG{g+gp}{\PYGZgt{}\PYGZgt{}\PYGZgt{} }\PYG{n}{np}\PYG{o}{.}\PYG{n}{random}\PYG{o}{.}\PYG{n}{random\PYGZus{}sample}\PYG{p}{(}\PYG{p}{(}\PYG{l+m+mi}{5}\PYG{p}{,}\PYG{p}{)}\PYG{p}{)}
\PYG{g+go}{array([ 0.30220482,  0.86820401,  0.1654503 ,  0.11659149,  0.54323428])}
\end{Verbatim}

Three-by-two array of random numbers from {[}-5, 0):

\begin{Verbatim}[commandchars=\\\{\}]
\PYG{g+gp}{\PYGZgt{}\PYGZgt{}\PYGZgt{} }\PYG{l+m+mi}{5} \PYG{o}{*} \PYG{n}{np}\PYG{o}{.}\PYG{n}{random}\PYG{o}{.}\PYG{n}{random\PYGZus{}sample}\PYG{p}{(}\PYG{p}{(}\PYG{l+m+mi}{3}\PYG{p}{,} \PYG{l+m+mi}{2}\PYG{p}{)}\PYG{p}{)} \PYG{o}{\PYGZhy{}} \PYG{l+m+mi}{5}
\PYG{g+go}{array([[\PYGZhy{}3.99149989, \PYGZhy{}0.52338984],}
\PYG{g+go}{       [\PYGZhy{}2.99091858, \PYGZhy{}0.79479508],}
\PYG{g+go}{       [\PYGZhy{}1.23204345, \PYGZhy{}1.75224494]])}
\end{Verbatim}

\end{fulllineitems}

\index{random\_integers() (in module lib.graph.scc)}

\begin{fulllineitems}
\phantomsection\label{lib.graph:lib.graph.scc.random_integers}\pysiglinewithargsret{\code{lib.graph.scc.}\bfcode{random\_integers}}{\emph{low}, \emph{high=None}, \emph{size=None}}{}
Return random integers between \emph{low} and \emph{high}, inclusive.

Return random integers from the ``discrete uniform'' distribution in the
closed interval {[}\emph{low}, \emph{high}{]}.  If \emph{high} is None (the default),
then results are from {[}1, \emph{low}{]}.
\begin{description}
\item[{low}] \leavevmode{[}int{]}
Lowest (signed) integer to be drawn from the distribution (unless
\code{high=None}, in which case this parameter is the \emph{highest} such
integer).

\item[{high}] \leavevmode{[}int, optional{]}
If provided, the largest (signed) integer to be drawn from the
distribution (see above for behavior if \code{high=None}).

\item[{size}] \leavevmode{[}int or tuple of ints, optional{]}
Output shape. Default is None, in which case a single int is returned.

\end{description}
\begin{description}
\item[{out}] \leavevmode{[}int or ndarray of ints{]}
\emph{size}-shaped array of random integers from the appropriate
distribution, or a single such random int if \emph{size} not provided.

\end{description}
\begin{description}
\item[{random.randint}] \leavevmode{[}Similar to \emph{random\_integers}, only for the half-open{]}
interval {[}\emph{low}, \emph{high}), and 0 is the lowest value if \emph{high} is
omitted.

\end{description}

To sample from N evenly spaced floating-point numbers between a and b,
use:

\begin{Verbatim}[commandchars=\\\{\}]
\PYG{n}{a} \PYG{o}{+} \PYG{p}{(}\PYG{n}{b} \PYG{o}{\PYGZhy{}} \PYG{n}{a}\PYG{p}{)} \PYG{o}{*} \PYG{p}{(}\PYG{n}{np}\PYG{o}{.}\PYG{n}{random}\PYG{o}{.}\PYG{n}{random\PYGZus{}integers}\PYG{p}{(}\PYG{n}{N}\PYG{p}{)} \PYG{o}{\PYGZhy{}} \PYG{l+m+mi}{1}\PYG{p}{)} \PYG{o}{/} \PYG{p}{(}\PYG{n}{N} \PYG{o}{\PYGZhy{}} \PYG{l+m+mf}{1.}\PYG{p}{)}
\end{Verbatim}

\begin{Verbatim}[commandchars=\\\{\}]
\PYG{g+gp}{\PYGZgt{}\PYGZgt{}\PYGZgt{} }\PYG{n}{np}\PYG{o}{.}\PYG{n}{random}\PYG{o}{.}\PYG{n}{random\PYGZus{}integers}\PYG{p}{(}\PYG{l+m+mi}{5}\PYG{p}{)}
\PYG{g+go}{4}
\PYG{g+gp}{\PYGZgt{}\PYGZgt{}\PYGZgt{} }\PYG{n+nb}{type}\PYG{p}{(}\PYG{n}{np}\PYG{o}{.}\PYG{n}{random}\PYG{o}{.}\PYG{n}{random\PYGZus{}integers}\PYG{p}{(}\PYG{l+m+mi}{5}\PYG{p}{)}\PYG{p}{)}
\PYG{g+go}{\PYGZlt{}type \PYGZsq{}int\PYGZsq{}\PYGZgt{}}
\PYG{g+gp}{\PYGZgt{}\PYGZgt{}\PYGZgt{} }\PYG{n}{np}\PYG{o}{.}\PYG{n}{random}\PYG{o}{.}\PYG{n}{random\PYGZus{}integers}\PYG{p}{(}\PYG{l+m+mi}{5}\PYG{p}{,} \PYG{n}{size}\PYG{o}{=}\PYG{p}{(}\PYG{l+m+mf}{3.}\PYG{p}{,}\PYG{l+m+mf}{2.}\PYG{p}{)}\PYG{p}{)}
\PYG{g+go}{array([[5, 4],}
\PYG{g+go}{       [3, 3],}
\PYG{g+go}{       [4, 5]])}
\end{Verbatim}

Choose five random numbers from the set of five evenly-spaced
numbers between 0 and 2.5, inclusive (\emph{i.e.}, from the set
\({0, 5/8, 10/8, 15/8, 20/8}\)):

\begin{Verbatim}[commandchars=\\\{\}]
\PYG{g+gp}{\PYGZgt{}\PYGZgt{}\PYGZgt{} }\PYG{l+m+mf}{2.5} \PYG{o}{*} \PYG{p}{(}\PYG{n}{np}\PYG{o}{.}\PYG{n}{random}\PYG{o}{.}\PYG{n}{random\PYGZus{}integers}\PYG{p}{(}\PYG{l+m+mi}{5}\PYG{p}{,} \PYG{n}{size}\PYG{o}{=}\PYG{p}{(}\PYG{l+m+mi}{5}\PYG{p}{,}\PYG{p}{)}\PYG{p}{)} \PYG{o}{\PYGZhy{}} \PYG{l+m+mi}{1}\PYG{p}{)} \PYG{o}{/} \PYG{l+m+mf}{4.}
\PYG{g+go}{array([ 0.625,  1.25 ,  0.625,  0.625,  2.5  ])}
\end{Verbatim}

Roll two six sided dice 1000 times and sum the results:

\begin{Verbatim}[commandchars=\\\{\}]
\PYG{g+gp}{\PYGZgt{}\PYGZgt{}\PYGZgt{} }\PYG{n}{d1} \PYG{o}{=} \PYG{n}{np}\PYG{o}{.}\PYG{n}{random}\PYG{o}{.}\PYG{n}{random\PYGZus{}integers}\PYG{p}{(}\PYG{l+m+mi}{1}\PYG{p}{,} \PYG{l+m+mi}{6}\PYG{p}{,} \PYG{l+m+mi}{1000}\PYG{p}{)}
\PYG{g+gp}{\PYGZgt{}\PYGZgt{}\PYGZgt{} }\PYG{n}{d2} \PYG{o}{=} \PYG{n}{np}\PYG{o}{.}\PYG{n}{random}\PYG{o}{.}\PYG{n}{random\PYGZus{}integers}\PYG{p}{(}\PYG{l+m+mi}{1}\PYG{p}{,} \PYG{l+m+mi}{6}\PYG{p}{,} \PYG{l+m+mi}{1000}\PYG{p}{)}
\PYG{g+gp}{\PYGZgt{}\PYGZgt{}\PYGZgt{} }\PYG{n}{dsums} \PYG{o}{=} \PYG{n}{d1} \PYG{o}{+} \PYG{n}{d2}
\end{Verbatim}

Display results as a histogram:

\begin{Verbatim}[commandchars=\\\{\}]
\PYG{g+gp}{\PYGZgt{}\PYGZgt{}\PYGZgt{} }\PYG{k+kn}{import} \PYG{n+nn}{matplotlib.pyplot} \PYG{k+kn}{as} \PYG{n+nn}{plt}
\PYG{g+gp}{\PYGZgt{}\PYGZgt{}\PYGZgt{} }\PYG{n}{count}\PYG{p}{,} \PYG{n}{bins}\PYG{p}{,} \PYG{n}{ignored} \PYG{o}{=} \PYG{n}{plt}\PYG{o}{.}\PYG{n}{hist}\PYG{p}{(}\PYG{n}{dsums}\PYG{p}{,} \PYG{l+m+mi}{11}\PYG{p}{,} \PYG{n}{normed}\PYG{o}{=}\PYG{n+nb+bp}{True}\PYG{p}{)}
\PYG{g+gp}{\PYGZgt{}\PYGZgt{}\PYGZgt{} }\PYG{n}{plt}\PYG{o}{.}\PYG{n}{show}\PYG{p}{(}\PYG{p}{)}
\end{Verbatim}

\end{fulllineitems}

\index{random\_sample() (in module lib.graph.scc)}

\begin{fulllineitems}
\phantomsection\label{lib.graph:lib.graph.scc.random_sample}\pysiglinewithargsret{\code{lib.graph.scc.}\bfcode{random\_sample}}{\emph{size=None}}{}
Return random floats in the half-open interval {[}0.0, 1.0).

Results are from the ``continuous uniform'' distribution over the
stated interval.  To sample \(Unif[a, b), b > a\) multiply
the output of \emph{random\_sample} by \emph{(b-a)} and add \emph{a}:

\begin{Verbatim}[commandchars=\\\{\}]
\PYG{p}{(}\PYG{n}{b} \PYG{o}{\PYGZhy{}} \PYG{n}{a}\PYG{p}{)} \PYG{o}{*} \PYG{n}{random\PYGZus{}sample}\PYG{p}{(}\PYG{p}{)} \PYG{o}{+} \PYG{n}{a}
\end{Verbatim}
\begin{description}
\item[{size}] \leavevmode{[}int or tuple of ints, optional{]}
Defines the shape of the returned array of random floats. If None
(the default), returns a single float.

\end{description}
\begin{description}
\item[{out}] \leavevmode{[}float or ndarray of floats{]}
Array of random floats of shape \emph{size} (unless \code{size=None}, in which
case a single float is returned).

\end{description}

\begin{Verbatim}[commandchars=\\\{\}]
\PYG{g+gp}{\PYGZgt{}\PYGZgt{}\PYGZgt{} }\PYG{n}{np}\PYG{o}{.}\PYG{n}{random}\PYG{o}{.}\PYG{n}{random\PYGZus{}sample}\PYG{p}{(}\PYG{p}{)}
\PYG{g+go}{0.47108547995356098}
\PYG{g+gp}{\PYGZgt{}\PYGZgt{}\PYGZgt{} }\PYG{n+nb}{type}\PYG{p}{(}\PYG{n}{np}\PYG{o}{.}\PYG{n}{random}\PYG{o}{.}\PYG{n}{random\PYGZus{}sample}\PYG{p}{(}\PYG{p}{)}\PYG{p}{)}
\PYG{g+go}{\PYGZlt{}type \PYGZsq{}float\PYGZsq{}\PYGZgt{}}
\PYG{g+gp}{\PYGZgt{}\PYGZgt{}\PYGZgt{} }\PYG{n}{np}\PYG{o}{.}\PYG{n}{random}\PYG{o}{.}\PYG{n}{random\PYGZus{}sample}\PYG{p}{(}\PYG{p}{(}\PYG{l+m+mi}{5}\PYG{p}{,}\PYG{p}{)}\PYG{p}{)}
\PYG{g+go}{array([ 0.30220482,  0.86820401,  0.1654503 ,  0.11659149,  0.54323428])}
\end{Verbatim}

Three-by-two array of random numbers from {[}-5, 0):

\begin{Verbatim}[commandchars=\\\{\}]
\PYG{g+gp}{\PYGZgt{}\PYGZgt{}\PYGZgt{} }\PYG{l+m+mi}{5} \PYG{o}{*} \PYG{n}{np}\PYG{o}{.}\PYG{n}{random}\PYG{o}{.}\PYG{n}{random\PYGZus{}sample}\PYG{p}{(}\PYG{p}{(}\PYG{l+m+mi}{3}\PYG{p}{,} \PYG{l+m+mi}{2}\PYG{p}{)}\PYG{p}{)} \PYG{o}{\PYGZhy{}} \PYG{l+m+mi}{5}
\PYG{g+go}{array([[\PYGZhy{}3.99149989, \PYGZhy{}0.52338984],}
\PYG{g+go}{       [\PYGZhy{}2.99091858, \PYGZhy{}0.79479508],}
\PYG{g+go}{       [\PYGZhy{}1.23204345, \PYGZhy{}1.75224494]])}
\end{Verbatim}

\end{fulllineitems}

\index{ranf() (in module lib.graph.scc)}

\begin{fulllineitems}
\phantomsection\label{lib.graph:lib.graph.scc.ranf}\pysiglinewithargsret{\code{lib.graph.scc.}\bfcode{ranf}}{}{}
random\_sample(size=None)

Return random floats in the half-open interval {[}0.0, 1.0).

Results are from the ``continuous uniform'' distribution over the
stated interval.  To sample \(Unif[a, b), b > a\) multiply
the output of \emph{random\_sample} by \emph{(b-a)} and add \emph{a}:

\begin{Verbatim}[commandchars=\\\{\}]
\PYG{p}{(}\PYG{n}{b} \PYG{o}{\PYGZhy{}} \PYG{n}{a}\PYG{p}{)} \PYG{o}{*} \PYG{n}{random\PYGZus{}sample}\PYG{p}{(}\PYG{p}{)} \PYG{o}{+} \PYG{n}{a}
\end{Verbatim}
\begin{description}
\item[{size}] \leavevmode{[}int or tuple of ints, optional{]}
Defines the shape of the returned array of random floats. If None
(the default), returns a single float.

\end{description}
\begin{description}
\item[{out}] \leavevmode{[}float or ndarray of floats{]}
Array of random floats of shape \emph{size} (unless \code{size=None}, in which
case a single float is returned).

\end{description}

\begin{Verbatim}[commandchars=\\\{\}]
\PYG{g+gp}{\PYGZgt{}\PYGZgt{}\PYGZgt{} }\PYG{n}{np}\PYG{o}{.}\PYG{n}{random}\PYG{o}{.}\PYG{n}{random\PYGZus{}sample}\PYG{p}{(}\PYG{p}{)}
\PYG{g+go}{0.47108547995356098}
\PYG{g+gp}{\PYGZgt{}\PYGZgt{}\PYGZgt{} }\PYG{n+nb}{type}\PYG{p}{(}\PYG{n}{np}\PYG{o}{.}\PYG{n}{random}\PYG{o}{.}\PYG{n}{random\PYGZus{}sample}\PYG{p}{(}\PYG{p}{)}\PYG{p}{)}
\PYG{g+go}{\PYGZlt{}type \PYGZsq{}float\PYGZsq{}\PYGZgt{}}
\PYG{g+gp}{\PYGZgt{}\PYGZgt{}\PYGZgt{} }\PYG{n}{np}\PYG{o}{.}\PYG{n}{random}\PYG{o}{.}\PYG{n}{random\PYGZus{}sample}\PYG{p}{(}\PYG{p}{(}\PYG{l+m+mi}{5}\PYG{p}{,}\PYG{p}{)}\PYG{p}{)}
\PYG{g+go}{array([ 0.30220482,  0.86820401,  0.1654503 ,  0.11659149,  0.54323428])}
\end{Verbatim}

Three-by-two array of random numbers from {[}-5, 0):

\begin{Verbatim}[commandchars=\\\{\}]
\PYG{g+gp}{\PYGZgt{}\PYGZgt{}\PYGZgt{} }\PYG{l+m+mi}{5} \PYG{o}{*} \PYG{n}{np}\PYG{o}{.}\PYG{n}{random}\PYG{o}{.}\PYG{n}{random\PYGZus{}sample}\PYG{p}{(}\PYG{p}{(}\PYG{l+m+mi}{3}\PYG{p}{,} \PYG{l+m+mi}{2}\PYG{p}{)}\PYG{p}{)} \PYG{o}{\PYGZhy{}} \PYG{l+m+mi}{5}
\PYG{g+go}{array([[\PYGZhy{}3.99149989, \PYGZhy{}0.52338984],}
\PYG{g+go}{       [\PYGZhy{}2.99091858, \PYGZhy{}0.79479508],}
\PYG{g+go}{       [\PYGZhy{}1.23204345, \PYGZhy{}1.75224494]])}
\end{Verbatim}

\end{fulllineitems}

\index{rayleigh() (in module lib.graph.scc)}

\begin{fulllineitems}
\phantomsection\label{lib.graph:lib.graph.scc.rayleigh}\pysiglinewithargsret{\code{lib.graph.scc.}\bfcode{rayleigh}}{\emph{scale=1.0}, \emph{size=None}}{}
Draw samples from a Rayleigh distribution.

The \(\chi\) and Weibull distributions are generalizations of the
Rayleigh.
\begin{description}
\item[{scale}] \leavevmode{[}scalar{]}
Scale, also equals the mode. Should be \textgreater{}= 0.

\item[{size}] \leavevmode{[}int or tuple of ints, optional{]}
Shape of the output. Default is None, in which case a single
value is returned.

\end{description}

The probability density function for the Rayleigh distribution is
\begin{gather}
\begin{split}P(x;scale) = \frac{x}{scale^2}e^{\frac{-x^2}{2 \cdotp scale^2}}\end{split}\notag
\end{gather}
The Rayleigh distribution arises if the wind speed and wind direction are
both gaussian variables, then the vector wind velocity forms a Rayleigh
distribution. The Rayleigh distribution is used to model the expected
output from wind turbines.

Draw values from the distribution and plot the histogram

\begin{Verbatim}[commandchars=\\\{\}]
\PYG{g+gp}{\PYGZgt{}\PYGZgt{}\PYGZgt{} }\PYG{n}{values} \PYG{o}{=} \PYG{n}{hist}\PYG{p}{(}\PYG{n}{np}\PYG{o}{.}\PYG{n}{random}\PYG{o}{.}\PYG{n}{rayleigh}\PYG{p}{(}\PYG{l+m+mi}{3}\PYG{p}{,} \PYG{l+m+mi}{100000}\PYG{p}{)}\PYG{p}{,} \PYG{n}{bins}\PYG{o}{=}\PYG{l+m+mi}{200}\PYG{p}{,} \PYG{n}{normed}\PYG{o}{=}\PYG{n+nb+bp}{True}\PYG{p}{)}
\end{Verbatim}

Wave heights tend to follow a Rayleigh distribution. If the mean wave
height is 1 meter, what fraction of waves are likely to be larger than 3
meters?

\begin{Verbatim}[commandchars=\\\{\}]
\PYG{g+gp}{\PYGZgt{}\PYGZgt{}\PYGZgt{} }\PYG{n}{meanvalue} \PYG{o}{=} \PYG{l+m+mi}{1}
\PYG{g+gp}{\PYGZgt{}\PYGZgt{}\PYGZgt{} }\PYG{n}{modevalue} \PYG{o}{=} \PYG{n}{np}\PYG{o}{.}\PYG{n}{sqrt}\PYG{p}{(}\PYG{l+m+mi}{2} \PYG{o}{/} \PYG{n}{np}\PYG{o}{.}\PYG{n}{pi}\PYG{p}{)} \PYG{o}{*} \PYG{n}{meanvalue}
\PYG{g+gp}{\PYGZgt{}\PYGZgt{}\PYGZgt{} }\PYG{n}{s} \PYG{o}{=} \PYG{n}{np}\PYG{o}{.}\PYG{n}{random}\PYG{o}{.}\PYG{n}{rayleigh}\PYG{p}{(}\PYG{n}{modevalue}\PYG{p}{,} \PYG{l+m+mi}{1000000}\PYG{p}{)}
\end{Verbatim}

The percentage of waves larger than 3 meters is:

\begin{Verbatim}[commandchars=\\\{\}]
\PYG{g+gp}{\PYGZgt{}\PYGZgt{}\PYGZgt{} }\PYG{l+m+mf}{100.}\PYG{o}{*}\PYG{n+nb}{sum}\PYG{p}{(}\PYG{n}{s}\PYG{o}{\PYGZgt{}}\PYG{l+m+mi}{3}\PYG{p}{)}\PYG{o}{/}\PYG{l+m+mf}{1000000.}
\PYG{g+go}{0.087300000000000003}
\end{Verbatim}

\end{fulllineitems}

\index{sample() (in module lib.graph.scc)}

\begin{fulllineitems}
\phantomsection\label{lib.graph:lib.graph.scc.sample}\pysiglinewithargsret{\code{lib.graph.scc.}\bfcode{sample}}{}{}
random\_sample(size=None)

Return random floats in the half-open interval {[}0.0, 1.0).

Results are from the ``continuous uniform'' distribution over the
stated interval.  To sample \(Unif[a, b), b > a\) multiply
the output of \emph{random\_sample} by \emph{(b-a)} and add \emph{a}:

\begin{Verbatim}[commandchars=\\\{\}]
\PYG{p}{(}\PYG{n}{b} \PYG{o}{\PYGZhy{}} \PYG{n}{a}\PYG{p}{)} \PYG{o}{*} \PYG{n}{random\PYGZus{}sample}\PYG{p}{(}\PYG{p}{)} \PYG{o}{+} \PYG{n}{a}
\end{Verbatim}
\begin{description}
\item[{size}] \leavevmode{[}int or tuple of ints, optional{]}
Defines the shape of the returned array of random floats. If None
(the default), returns a single float.

\end{description}
\begin{description}
\item[{out}] \leavevmode{[}float or ndarray of floats{]}
Array of random floats of shape \emph{size} (unless \code{size=None}, in which
case a single float is returned).

\end{description}

\begin{Verbatim}[commandchars=\\\{\}]
\PYG{g+gp}{\PYGZgt{}\PYGZgt{}\PYGZgt{} }\PYG{n}{np}\PYG{o}{.}\PYG{n}{random}\PYG{o}{.}\PYG{n}{random\PYGZus{}sample}\PYG{p}{(}\PYG{p}{)}
\PYG{g+go}{0.47108547995356098}
\PYG{g+gp}{\PYGZgt{}\PYGZgt{}\PYGZgt{} }\PYG{n+nb}{type}\PYG{p}{(}\PYG{n}{np}\PYG{o}{.}\PYG{n}{random}\PYG{o}{.}\PYG{n}{random\PYGZus{}sample}\PYG{p}{(}\PYG{p}{)}\PYG{p}{)}
\PYG{g+go}{\PYGZlt{}type \PYGZsq{}float\PYGZsq{}\PYGZgt{}}
\PYG{g+gp}{\PYGZgt{}\PYGZgt{}\PYGZgt{} }\PYG{n}{np}\PYG{o}{.}\PYG{n}{random}\PYG{o}{.}\PYG{n}{random\PYGZus{}sample}\PYG{p}{(}\PYG{p}{(}\PYG{l+m+mi}{5}\PYG{p}{,}\PYG{p}{)}\PYG{p}{)}
\PYG{g+go}{array([ 0.30220482,  0.86820401,  0.1654503 ,  0.11659149,  0.54323428])}
\end{Verbatim}

Three-by-two array of random numbers from {[}-5, 0):

\begin{Verbatim}[commandchars=\\\{\}]
\PYG{g+gp}{\PYGZgt{}\PYGZgt{}\PYGZgt{} }\PYG{l+m+mi}{5} \PYG{o}{*} \PYG{n}{np}\PYG{o}{.}\PYG{n}{random}\PYG{o}{.}\PYG{n}{random\PYGZus{}sample}\PYG{p}{(}\PYG{p}{(}\PYG{l+m+mi}{3}\PYG{p}{,} \PYG{l+m+mi}{2}\PYG{p}{)}\PYG{p}{)} \PYG{o}{\PYGZhy{}} \PYG{l+m+mi}{5}
\PYG{g+go}{array([[\PYGZhy{}3.99149989, \PYGZhy{}0.52338984],}
\PYG{g+go}{       [\PYGZhy{}2.99091858, \PYGZhy{}0.79479508],}
\PYG{g+go}{       [\PYGZhy{}1.23204345, \PYGZhy{}1.75224494]])}
\end{Verbatim}

\end{fulllineitems}

\index{seed() (in module lib.graph.scc)}

\begin{fulllineitems}
\phantomsection\label{lib.graph:lib.graph.scc.seed}\pysiglinewithargsret{\code{lib.graph.scc.}\bfcode{seed}}{\emph{seed=None}}{}
Seed the generator.

This method is called when \emph{RandomState} is initialized. It can be
called again to re-seed the generator. For details, see \emph{RandomState}.
\begin{description}
\item[{seed}] \leavevmode{[}int or array\_like, optional{]}
Seed for \emph{RandomState}.

\end{description}

RandomState

\end{fulllineitems}

\index{set\_state() (in module lib.graph.scc)}

\begin{fulllineitems}
\phantomsection\label{lib.graph:lib.graph.scc.set_state}\pysiglinewithargsret{\code{lib.graph.scc.}\bfcode{set\_state}}{\emph{state}}{}
Set the internal state of the generator from a tuple.

For use if one has reason to manually (re-)set the internal state of the
``Mersenne Twister''{\color{red}\bfseries{}{[}1{]}\_} pseudo-random number generating algorithm.
\begin{description}
\item[{state}] \leavevmode{[}tuple(str, ndarray of 624 uints, int, int, float){]}
The \emph{state} tuple has the following items:
\begin{enumerate}
\item {} 
the string `MT19937', specifying the Mersenne Twister algorithm.

\item {} 
a 1-D array of 624 unsigned integers \code{keys}.

\item {} 
an integer \code{pos}.

\item {} 
an integer \code{has\_gauss}.

\item {} 
a float \code{cached\_gaussian}.

\end{enumerate}

\end{description}
\begin{description}
\item[{out}] \leavevmode{[}None{]}
Returns `None' on success.

\end{description}

get\_state

\emph{set\_state} and \emph{get\_state} are not needed to work with any of the
random distributions in NumPy. If the internal state is manually altered,
the user should know exactly what he/she is doing.

For backwards compatibility, the form (str, array of 624 uints, int) is
also accepted although it is missing some information about the cached
Gaussian value: \code{state = ('MT19937', keys, pos)}.

\end{fulllineitems}

\index{shuffle() (in module lib.graph.scc)}

\begin{fulllineitems}
\phantomsection\label{lib.graph:lib.graph.scc.shuffle}\pysiglinewithargsret{\code{lib.graph.scc.}\bfcode{shuffle}}{\emph{x}}{}
Modify a sequence in-place by shuffling its contents.
\begin{description}
\item[{x}] \leavevmode{[}array\_like{]}
The array or list to be shuffled.

\end{description}

None

\begin{Verbatim}[commandchars=\\\{\}]
\PYG{g+gp}{\PYGZgt{}\PYGZgt{}\PYGZgt{} }\PYG{n}{arr} \PYG{o}{=} \PYG{n}{np}\PYG{o}{.}\PYG{n}{arange}\PYG{p}{(}\PYG{l+m+mi}{10}\PYG{p}{)}
\PYG{g+gp}{\PYGZgt{}\PYGZgt{}\PYGZgt{} }\PYG{n}{np}\PYG{o}{.}\PYG{n}{random}\PYG{o}{.}\PYG{n}{shuffle}\PYG{p}{(}\PYG{n}{arr}\PYG{p}{)}
\PYG{g+gp}{\PYGZgt{}\PYGZgt{}\PYGZgt{} }\PYG{n}{arr}
\PYG{g+go}{[1 7 5 2 9 4 3 6 0 8]}
\end{Verbatim}

This function only shuffles the array along the first index of a
multi-dimensional array:

\begin{Verbatim}[commandchars=\\\{\}]
\PYG{g+gp}{\PYGZgt{}\PYGZgt{}\PYGZgt{} }\PYG{n}{arr} \PYG{o}{=} \PYG{n}{np}\PYG{o}{.}\PYG{n}{arange}\PYG{p}{(}\PYG{l+m+mi}{9}\PYG{p}{)}\PYG{o}{.}\PYG{n}{reshape}\PYG{p}{(}\PYG{p}{(}\PYG{l+m+mi}{3}\PYG{p}{,} \PYG{l+m+mi}{3}\PYG{p}{)}\PYG{p}{)}
\PYG{g+gp}{\PYGZgt{}\PYGZgt{}\PYGZgt{} }\PYG{n}{np}\PYG{o}{.}\PYG{n}{random}\PYG{o}{.}\PYG{n}{shuffle}\PYG{p}{(}\PYG{n}{arr}\PYG{p}{)}
\PYG{g+gp}{\PYGZgt{}\PYGZgt{}\PYGZgt{} }\PYG{n}{arr}
\PYG{g+go}{array([[3, 4, 5],}
\PYG{g+go}{       [6, 7, 8],}
\PYG{g+go}{       [0, 1, 2]])}
\end{Verbatim}

\end{fulllineitems}

\index{standard\_cauchy() (in module lib.graph.scc)}

\begin{fulllineitems}
\phantomsection\label{lib.graph:lib.graph.scc.standard_cauchy}\pysiglinewithargsret{\code{lib.graph.scc.}\bfcode{standard\_cauchy}}{\emph{size=None}}{}
Standard Cauchy distribution with mode = 0.

Also known as the Lorentz distribution.
\begin{description}
\item[{size}] \leavevmode{[}int or tuple of ints{]}
Shape of the output.

\end{description}
\begin{description}
\item[{samples}] \leavevmode{[}ndarray or scalar{]}
The drawn samples.

\end{description}

The probability density function for the full Cauchy distribution is
\begin{gather}
\begin{split}P(x; x_0, \gamma) = \frac{1}{\pi \gamma \bigl[ 1+
(\frac{x-x_0}{\gamma})^2 \bigr] }\end{split}\notag
\end{gather}
and the Standard Cauchy distribution just sets \(x_0=0\) and
\(\gamma=1\)

The Cauchy distribution arises in the solution to the driven harmonic
oscillator problem, and also describes spectral line broadening. It
also describes the distribution of values at which a line tilted at
a random angle will cut the x axis.

When studying hypothesis tests that assume normality, seeing how the
tests perform on data from a Cauchy distribution is a good indicator of
their sensitivity to a heavy-tailed distribution, since the Cauchy looks
very much like a Gaussian distribution, but with heavier tails.

Draw samples and plot the distribution:

\begin{Verbatim}[commandchars=\\\{\}]
\PYG{g+gp}{\PYGZgt{}\PYGZgt{}\PYGZgt{} }\PYG{n}{s} \PYG{o}{=} \PYG{n}{np}\PYG{o}{.}\PYG{n}{random}\PYG{o}{.}\PYG{n}{standard\PYGZus{}cauchy}\PYG{p}{(}\PYG{l+m+mi}{1000000}\PYG{p}{)}
\PYG{g+gp}{\PYGZgt{}\PYGZgt{}\PYGZgt{} }\PYG{n}{s} \PYG{o}{=} \PYG{n}{s}\PYG{p}{[}\PYG{p}{(}\PYG{n}{s}\PYG{o}{\PYGZgt{}}\PYG{o}{\PYGZhy{}}\PYG{l+m+mi}{25}\PYG{p}{)} \PYG{o}{\PYGZam{}} \PYG{p}{(}\PYG{n}{s}\PYG{o}{\PYGZlt{}}\PYG{l+m+mi}{25}\PYG{p}{)}\PYG{p}{]}  \PYG{c}{\PYGZsh{} truncate distribution so it plots well}
\PYG{g+gp}{\PYGZgt{}\PYGZgt{}\PYGZgt{} }\PYG{n}{plt}\PYG{o}{.}\PYG{n}{hist}\PYG{p}{(}\PYG{n}{s}\PYG{p}{,} \PYG{n}{bins}\PYG{o}{=}\PYG{l+m+mi}{100}\PYG{p}{)}
\PYG{g+gp}{\PYGZgt{}\PYGZgt{}\PYGZgt{} }\PYG{n}{plt}\PYG{o}{.}\PYG{n}{show}\PYG{p}{(}\PYG{p}{)}
\end{Verbatim}

\end{fulllineitems}

\index{standard\_exponential() (in module lib.graph.scc)}

\begin{fulllineitems}
\phantomsection\label{lib.graph:lib.graph.scc.standard_exponential}\pysiglinewithargsret{\code{lib.graph.scc.}\bfcode{standard\_exponential}}{\emph{size=None}}{}
Draw samples from the standard exponential distribution.

\emph{standard\_exponential} is identical to the exponential distribution
with a scale parameter of 1.
\begin{description}
\item[{size}] \leavevmode{[}int or tuple of ints{]}
Shape of the output.

\end{description}
\begin{description}
\item[{out}] \leavevmode{[}float or ndarray{]}
Drawn samples.

\end{description}

Output a 3x8000 array:

\begin{Verbatim}[commandchars=\\\{\}]
\PYG{g+gp}{\PYGZgt{}\PYGZgt{}\PYGZgt{} }\PYG{n}{n} \PYG{o}{=} \PYG{n}{np}\PYG{o}{.}\PYG{n}{random}\PYG{o}{.}\PYG{n}{standard\PYGZus{}exponential}\PYG{p}{(}\PYG{p}{(}\PYG{l+m+mi}{3}\PYG{p}{,} \PYG{l+m+mi}{8000}\PYG{p}{)}\PYG{p}{)}
\end{Verbatim}

\end{fulllineitems}

\index{standard\_gamma() (in module lib.graph.scc)}

\begin{fulllineitems}
\phantomsection\label{lib.graph:lib.graph.scc.standard_gamma}\pysiglinewithargsret{\code{lib.graph.scc.}\bfcode{standard\_gamma}}{\emph{shape}, \emph{size=None}}{}
Draw samples from a Standard Gamma distribution.

Samples are drawn from a Gamma distribution with specified parameters,
shape (sometimes designated ``k'') and scale=1.
\begin{description}
\item[{shape}] \leavevmode{[}float{]}
Parameter, should be \textgreater{} 0.

\item[{size}] \leavevmode{[}int or tuple of ints{]}
Output shape.  If the given shape is, e.g., \code{(m, n, k)}, then
\code{m * n * k} samples are drawn.

\end{description}
\begin{description}
\item[{samples}] \leavevmode{[}ndarray or scalar{]}
The drawn samples.

\end{description}
\begin{description}
\item[{scipy.stats.distributions.gamma}] \leavevmode{[}probability density function,{]}
distribution or cumulative density function, etc.

\end{description}

The probability density for the Gamma distribution is
\begin{gather}
\begin{split}p(x) = x^{k-1}\frac{e^{-x/\theta}}{\theta^k\Gamma(k)},\end{split}\notag
\end{gather}
where \(k\) is the shape and \(\theta\) the scale,
and \(\Gamma\) is the Gamma function.

The Gamma distribution is often used to model the times to failure of
electronic components, and arises naturally in processes for which the
waiting times between Poisson distributed events are relevant.

Draw samples from the distribution:

\begin{Verbatim}[commandchars=\\\{\}]
\PYG{g+gp}{\PYGZgt{}\PYGZgt{}\PYGZgt{} }\PYG{n}{shape}\PYG{p}{,} \PYG{n}{scale} \PYG{o}{=} \PYG{l+m+mf}{2.}\PYG{p}{,} \PYG{l+m+mf}{1.} \PYG{c}{\PYGZsh{} mean and width}
\PYG{g+gp}{\PYGZgt{}\PYGZgt{}\PYGZgt{} }\PYG{n}{s} \PYG{o}{=} \PYG{n}{np}\PYG{o}{.}\PYG{n}{random}\PYG{o}{.}\PYG{n}{standard\PYGZus{}gamma}\PYG{p}{(}\PYG{n}{shape}\PYG{p}{,} \PYG{l+m+mi}{1000000}\PYG{p}{)}
\end{Verbatim}

Display the histogram of the samples, along with
the probability density function:

\begin{Verbatim}[commandchars=\\\{\}]
\PYG{g+gp}{\PYGZgt{}\PYGZgt{}\PYGZgt{} }\PYG{k+kn}{import} \PYG{n+nn}{matplotlib.pyplot} \PYG{k+kn}{as} \PYG{n+nn}{plt}
\PYG{g+gp}{\PYGZgt{}\PYGZgt{}\PYGZgt{} }\PYG{k+kn}{import} \PYG{n+nn}{scipy.special} \PYG{k+kn}{as} \PYG{n+nn}{sps}
\PYG{g+gp}{\PYGZgt{}\PYGZgt{}\PYGZgt{} }\PYG{n}{count}\PYG{p}{,} \PYG{n}{bins}\PYG{p}{,} \PYG{n}{ignored} \PYG{o}{=} \PYG{n}{plt}\PYG{o}{.}\PYG{n}{hist}\PYG{p}{(}\PYG{n}{s}\PYG{p}{,} \PYG{l+m+mi}{50}\PYG{p}{,} \PYG{n}{normed}\PYG{o}{=}\PYG{n+nb+bp}{True}\PYG{p}{)}
\PYG{g+gp}{\PYGZgt{}\PYGZgt{}\PYGZgt{} }\PYG{n}{y} \PYG{o}{=} \PYG{n}{bins}\PYG{o}{*}\PYG{o}{*}\PYG{p}{(}\PYG{n}{shape}\PYG{o}{\PYGZhy{}}\PYG{l+m+mi}{1}\PYG{p}{)} \PYG{o}{*} \PYG{p}{(}\PYG{p}{(}\PYG{n}{np}\PYG{o}{.}\PYG{n}{exp}\PYG{p}{(}\PYG{o}{\PYGZhy{}}\PYG{n}{bins}\PYG{o}{/}\PYG{n}{scale}\PYG{p}{)}\PYG{p}{)}\PYG{o}{/} \PYGZbs{}
\PYG{g+gp}{... }                      \PYG{p}{(}\PYG{n}{sps}\PYG{o}{.}\PYG{n}{gamma}\PYG{p}{(}\PYG{n}{shape}\PYG{p}{)} \PYG{o}{*} \PYG{n}{scale}\PYG{o}{*}\PYG{o}{*}\PYG{n}{shape}\PYG{p}{)}\PYG{p}{)}
\PYG{g+gp}{\PYGZgt{}\PYGZgt{}\PYGZgt{} }\PYG{n}{plt}\PYG{o}{.}\PYG{n}{plot}\PYG{p}{(}\PYG{n}{bins}\PYG{p}{,} \PYG{n}{y}\PYG{p}{,} \PYG{n}{linewidth}\PYG{o}{=}\PYG{l+m+mi}{2}\PYG{p}{,} \PYG{n}{color}\PYG{o}{=}\PYG{l+s}{\PYGZsq{}}\PYG{l+s}{r}\PYG{l+s}{\PYGZsq{}}\PYG{p}{)}
\PYG{g+gp}{\PYGZgt{}\PYGZgt{}\PYGZgt{} }\PYG{n}{plt}\PYG{o}{.}\PYG{n}{show}\PYG{p}{(}\PYG{p}{)}
\end{Verbatim}

\end{fulllineitems}

\index{standard\_normal() (in module lib.graph.scc)}

\begin{fulllineitems}
\phantomsection\label{lib.graph:lib.graph.scc.standard_normal}\pysiglinewithargsret{\code{lib.graph.scc.}\bfcode{standard\_normal}}{\emph{size=None}}{}
Returns samples from a Standard Normal distribution (mean=0, stdev=1).
\begin{description}
\item[{size}] \leavevmode{[}int or tuple of ints, optional{]}
Output shape. Default is None, in which case a single value is
returned.

\end{description}
\begin{description}
\item[{out}] \leavevmode{[}float or ndarray{]}
Drawn samples.

\end{description}

\begin{Verbatim}[commandchars=\\\{\}]
\PYG{g+gp}{\PYGZgt{}\PYGZgt{}\PYGZgt{} }\PYG{n}{s} \PYG{o}{=} \PYG{n}{np}\PYG{o}{.}\PYG{n}{random}\PYG{o}{.}\PYG{n}{standard\PYGZus{}normal}\PYG{p}{(}\PYG{l+m+mi}{8000}\PYG{p}{)}
\PYG{g+gp}{\PYGZgt{}\PYGZgt{}\PYGZgt{} }\PYG{n}{s}
\PYG{g+go}{array([ 0.6888893 ,  0.78096262, \PYGZhy{}0.89086505, ...,  0.49876311, \PYGZsh{}random}
\PYG{g+go}{       \PYGZhy{}0.38672696, \PYGZhy{}0.4685006 ])                               \PYGZsh{}random}
\PYG{g+gp}{\PYGZgt{}\PYGZgt{}\PYGZgt{} }\PYG{n}{s}\PYG{o}{.}\PYG{n}{shape}
\PYG{g+go}{(8000,)}
\PYG{g+gp}{\PYGZgt{}\PYGZgt{}\PYGZgt{} }\PYG{n}{s} \PYG{o}{=} \PYG{n}{np}\PYG{o}{.}\PYG{n}{random}\PYG{o}{.}\PYG{n}{standard\PYGZus{}normal}\PYG{p}{(}\PYG{n}{size}\PYG{o}{=}\PYG{p}{(}\PYG{l+m+mi}{3}\PYG{p}{,} \PYG{l+m+mi}{4}\PYG{p}{,} \PYG{l+m+mi}{2}\PYG{p}{)}\PYG{p}{)}
\PYG{g+gp}{\PYGZgt{}\PYGZgt{}\PYGZgt{} }\PYG{n}{s}\PYG{o}{.}\PYG{n}{shape}
\PYG{g+go}{(3, 4, 2)}
\end{Verbatim}

\end{fulllineitems}

\index{standard\_t() (in module lib.graph.scc)}

\begin{fulllineitems}
\phantomsection\label{lib.graph:lib.graph.scc.standard_t}\pysiglinewithargsret{\code{lib.graph.scc.}\bfcode{standard\_t}}{\emph{df}, \emph{size=None}}{}
Standard Student's t distribution with df degrees of freedom.

A special case of the hyperbolic distribution.
As \emph{df} gets large, the result resembles that of the standard normal
distribution (\emph{standard\_normal}).
\begin{description}
\item[{df}] \leavevmode{[}int{]}
Degrees of freedom, should be \textgreater{} 0.

\item[{size}] \leavevmode{[}int or tuple of ints, optional{]}
Output shape. Default is None, in which case a single value is
returned.

\end{description}
\begin{description}
\item[{samples}] \leavevmode{[}ndarray or scalar{]}
Drawn samples.

\end{description}

The probability density function for the t distribution is
\begin{gather}
\begin{split}P(x, df) = \frac{\Gamma(\frac{df+1}{2})}{\sqrt{\pi df}
\Gamma(\frac{df}{2})}\Bigl( 1+\frac{x^2}{df} \Bigr)^{-(df+1)/2}\end{split}\notag
\end{gather}
The t test is based on an assumption that the data come from a Normal
distribution. The t test provides a way to test whether the sample mean
(that is the mean calculated from the data) is a good estimate of the true
mean.

The derivation of the t-distribution was forst published in 1908 by William
Gisset while working for the Guinness Brewery in Dublin. Due to proprietary
issues, he had to publish under a pseudonym, and so he used the name
Student.

From Dalgaard page 83 {\color{red}\bfseries{}{[}1{]}\_}, suppose the daily energy intake for 11
women in Kj is:

\begin{Verbatim}[commandchars=\\\{\}]
\PYG{g+gp}{\PYGZgt{}\PYGZgt{}\PYGZgt{} }\PYG{n}{intake} \PYG{o}{=} \PYG{n}{np}\PYG{o}{.}\PYG{n}{array}\PYG{p}{(}\PYG{p}{[}\PYG{l+m+mf}{5260.}\PYG{p}{,} \PYG{l+m+mi}{5470}\PYG{p}{,} \PYG{l+m+mi}{5640}\PYG{p}{,} \PYG{l+m+mi}{6180}\PYG{p}{,} \PYG{l+m+mi}{6390}\PYG{p}{,} \PYG{l+m+mi}{6515}\PYG{p}{,} \PYG{l+m+mi}{6805}\PYG{p}{,} \PYG{l+m+mi}{7515}\PYG{p}{,} \PYGZbs{}
\PYG{g+gp}{... }                   \PYG{l+m+mi}{7515}\PYG{p}{,} \PYG{l+m+mi}{8230}\PYG{p}{,} \PYG{l+m+mi}{8770}\PYG{p}{]}\PYG{p}{)}
\end{Verbatim}

Does their energy intake deviate systematically from the recommended
value of 7725 kJ?

We have 10 degrees of freedom, so is the sample mean within 95\% of the
recommended value?

\begin{Verbatim}[commandchars=\\\{\}]
\PYG{g+gp}{\PYGZgt{}\PYGZgt{}\PYGZgt{} }\PYG{n}{s} \PYG{o}{=} \PYG{n}{np}\PYG{o}{.}\PYG{n}{random}\PYG{o}{.}\PYG{n}{standard\PYGZus{}t}\PYG{p}{(}\PYG{l+m+mi}{10}\PYG{p}{,} \PYG{n}{size}\PYG{o}{=}\PYG{l+m+mi}{100000}\PYG{p}{)}
\PYG{g+gp}{\PYGZgt{}\PYGZgt{}\PYGZgt{} }\PYG{n}{np}\PYG{o}{.}\PYG{n}{mean}\PYG{p}{(}\PYG{n}{intake}\PYG{p}{)}
\PYG{g+go}{6753.636363636364}
\PYG{g+gp}{\PYGZgt{}\PYGZgt{}\PYGZgt{} }\PYG{n}{intake}\PYG{o}{.}\PYG{n}{std}\PYG{p}{(}\PYG{n}{ddof}\PYG{o}{=}\PYG{l+m+mi}{1}\PYG{p}{)}
\PYG{g+go}{1142.1232221373727}
\end{Verbatim}

Calculate the t statistic, setting the ddof parameter to the unbiased
value so the divisor in the standard deviation will be degrees of
freedom, N-1.

\begin{Verbatim}[commandchars=\\\{\}]
\PYG{g+gp}{\PYGZgt{}\PYGZgt{}\PYGZgt{} }\PYG{n}{t} \PYG{o}{=} \PYG{p}{(}\PYG{n}{np}\PYG{o}{.}\PYG{n}{mean}\PYG{p}{(}\PYG{n}{intake}\PYG{p}{)}\PYG{o}{\PYGZhy{}}\PYG{l+m+mi}{7725}\PYG{p}{)}\PYG{o}{/}\PYG{p}{(}\PYG{n}{intake}\PYG{o}{.}\PYG{n}{std}\PYG{p}{(}\PYG{n}{ddof}\PYG{o}{=}\PYG{l+m+mi}{1}\PYG{p}{)}\PYG{o}{/}\PYG{n}{np}\PYG{o}{.}\PYG{n}{sqrt}\PYG{p}{(}\PYG{n+nb}{len}\PYG{p}{(}\PYG{n}{intake}\PYG{p}{)}\PYG{p}{)}\PYG{p}{)}
\PYG{g+gp}{\PYGZgt{}\PYGZgt{}\PYGZgt{} }\PYG{k+kn}{import} \PYG{n+nn}{matplotlib.pyplot} \PYG{k+kn}{as} \PYG{n+nn}{plt}
\PYG{g+gp}{\PYGZgt{}\PYGZgt{}\PYGZgt{} }\PYG{n}{h} \PYG{o}{=} \PYG{n}{plt}\PYG{o}{.}\PYG{n}{hist}\PYG{p}{(}\PYG{n}{s}\PYG{p}{,} \PYG{n}{bins}\PYG{o}{=}\PYG{l+m+mi}{100}\PYG{p}{,} \PYG{n}{normed}\PYG{o}{=}\PYG{n+nb+bp}{True}\PYG{p}{)}
\end{Verbatim}

For a one-sided t-test, how far out in the distribution does the t
statistic appear?

\begin{Verbatim}[commandchars=\\\{\}]
\PYG{g+gp}{\PYGZgt{}\PYGZgt{}\PYGZgt{} }\PYG{o}{\PYGZgt{}\PYGZgt{}}\PYG{o}{\PYGZgt{}} \PYG{n}{np}\PYG{o}{.}\PYG{n}{sum}\PYG{p}{(}\PYG{n}{s}\PYG{o}{\PYGZlt{}}\PYG{n}{t}\PYG{p}{)} \PYG{o}{/} \PYG{n+nb}{float}\PYG{p}{(}\PYG{n+nb}{len}\PYG{p}{(}\PYG{n}{s}\PYG{p}{)}\PYG{p}{)}
\PYG{g+go}{0.0090699999999999999  \PYGZsh{}random}
\end{Verbatim}

So the p-value is about 0.009, which says the null hypothesis has a
probability of about 99\% of being true.

\end{fulllineitems}

\index{triangular() (in module lib.graph.scc)}

\begin{fulllineitems}
\phantomsection\label{lib.graph:lib.graph.scc.triangular}\pysiglinewithargsret{\code{lib.graph.scc.}\bfcode{triangular}}{\emph{left}, \emph{mode}, \emph{right}, \emph{size=None}}{}
Draw samples from the triangular distribution.

The triangular distribution is a continuous probability distribution with
lower limit left, peak at mode, and upper limit right. Unlike the other
distributions, these parameters directly define the shape of the pdf.
\begin{description}
\item[{left}] \leavevmode{[}scalar{]}
Lower limit.

\item[{mode}] \leavevmode{[}scalar{]}
The value where the peak of the distribution occurs.
The value should fulfill the condition \code{left \textless{}= mode \textless{}= right}.

\item[{right}] \leavevmode{[}scalar{]}
Upper limit, should be larger than \emph{left}.

\item[{size}] \leavevmode{[}int or tuple of ints, optional{]}
Output shape. Default is None, in which case a single value is
returned.

\end{description}
\begin{description}
\item[{samples}] \leavevmode{[}ndarray or scalar{]}
The returned samples all lie in the interval {[}left, right{]}.

\end{description}

The probability density function for the Triangular distribution is
\begin{gather}
\begin{split}P(x;l, m, r) = \begin{cases}
\frac{2(x-l)}{(r-l)(m-l)}& \text{for $l \leq x \leq m$},\\
\frac{2(m-x)}{(r-l)(r-m)}& \text{for $m \leq x \leq r$},\\
0& \text{otherwise}.
\end{cases}\end{split}\notag
\end{gather}
The triangular distribution is often used in ill-defined problems where the
underlying distribution is not known, but some knowledge of the limits and
mode exists. Often it is used in simulations.

Draw values from the distribution and plot the histogram:

\begin{Verbatim}[commandchars=\\\{\}]
\PYG{g+gp}{\PYGZgt{}\PYGZgt{}\PYGZgt{} }\PYG{k+kn}{import} \PYG{n+nn}{matplotlib.pyplot} \PYG{k+kn}{as} \PYG{n+nn}{plt}
\PYG{g+gp}{\PYGZgt{}\PYGZgt{}\PYGZgt{} }\PYG{n}{h} \PYG{o}{=} \PYG{n}{plt}\PYG{o}{.}\PYG{n}{hist}\PYG{p}{(}\PYG{n}{np}\PYG{o}{.}\PYG{n}{random}\PYG{o}{.}\PYG{n}{triangular}\PYG{p}{(}\PYG{o}{\PYGZhy{}}\PYG{l+m+mi}{3}\PYG{p}{,} \PYG{l+m+mi}{0}\PYG{p}{,} \PYG{l+m+mi}{8}\PYG{p}{,} \PYG{l+m+mi}{100000}\PYG{p}{)}\PYG{p}{,} \PYG{n}{bins}\PYG{o}{=}\PYG{l+m+mi}{200}\PYG{p}{,}
\PYG{g+gp}{... }             \PYG{n}{normed}\PYG{o}{=}\PYG{n+nb+bp}{True}\PYG{p}{)}
\PYG{g+gp}{\PYGZgt{}\PYGZgt{}\PYGZgt{} }\PYG{n}{plt}\PYG{o}{.}\PYG{n}{show}\PYG{p}{(}\PYG{p}{)}
\end{Verbatim}

\end{fulllineitems}

\index{uniform() (in module lib.graph.scc)}

\begin{fulllineitems}
\phantomsection\label{lib.graph:lib.graph.scc.uniform}\pysiglinewithargsret{\code{lib.graph.scc.}\bfcode{uniform}}{\emph{low=0.0}, \emph{high=1.0}, \emph{size=1}}{}
Draw samples from a uniform distribution.

Samples are uniformly distributed over the half-open interval
\code{{[}low, high)} (includes low, but excludes high).  In other words,
any value within the given interval is equally likely to be drawn
by \emph{uniform}.
\begin{description}
\item[{low}] \leavevmode{[}float, optional{]}
Lower boundary of the output interval.  All values generated will be
greater than or equal to low.  The default value is 0.

\item[{high}] \leavevmode{[}float{]}
Upper boundary of the output interval.  All values generated will be
less than high.  The default value is 1.0.

\item[{size}] \leavevmode{[}int or tuple of ints, optional{]}
Shape of output.  If the given size is, for example, (m,n,k),
m*n*k samples are generated.  If no shape is specified, a single sample
is returned.

\end{description}
\begin{description}
\item[{out}] \leavevmode{[}ndarray{]}
Drawn samples, with shape \emph{size}.

\end{description}

randint : Discrete uniform distribution, yielding integers.
random\_integers : Discrete uniform distribution over the closed
\begin{quote}

interval \code{{[}low, high{]}}.
\end{quote}

random\_sample : Floats uniformly distributed over \code{{[}0, 1)}.
random : Alias for \emph{random\_sample}.
rand : Convenience function that accepts dimensions as input, e.g.,
\begin{quote}

\code{rand(2,2)} would generate a 2-by-2 array of floats,
uniformly distributed over \code{{[}0, 1)}.
\end{quote}

The probability density function of the uniform distribution is
\begin{gather}
\begin{split}p(x) = \frac{1}{b - a}\end{split}\notag
\end{gather}
anywhere within the interval \code{{[}a, b)}, and zero elsewhere.

Draw samples from the distribution:

\begin{Verbatim}[commandchars=\\\{\}]
\PYG{g+gp}{\PYGZgt{}\PYGZgt{}\PYGZgt{} }\PYG{n}{s} \PYG{o}{=} \PYG{n}{np}\PYG{o}{.}\PYG{n}{random}\PYG{o}{.}\PYG{n}{uniform}\PYG{p}{(}\PYG{o}{\PYGZhy{}}\PYG{l+m+mi}{1}\PYG{p}{,}\PYG{l+m+mi}{0}\PYG{p}{,}\PYG{l+m+mi}{1000}\PYG{p}{)}
\end{Verbatim}

All values are within the given interval:

\begin{Verbatim}[commandchars=\\\{\}]
\PYG{g+gp}{\PYGZgt{}\PYGZgt{}\PYGZgt{} }\PYG{n}{np}\PYG{o}{.}\PYG{n}{all}\PYG{p}{(}\PYG{n}{s} \PYG{o}{\PYGZgt{}}\PYG{o}{=} \PYG{o}{\PYGZhy{}}\PYG{l+m+mi}{1}\PYG{p}{)}
\PYG{g+go}{True}
\PYG{g+gp}{\PYGZgt{}\PYGZgt{}\PYGZgt{} }\PYG{n}{np}\PYG{o}{.}\PYG{n}{all}\PYG{p}{(}\PYG{n}{s} \PYG{o}{\PYGZlt{}} \PYG{l+m+mi}{0}\PYG{p}{)}
\PYG{g+go}{True}
\end{Verbatim}

Display the histogram of the samples, along with the
probability density function:

\begin{Verbatim}[commandchars=\\\{\}]
\PYG{g+gp}{\PYGZgt{}\PYGZgt{}\PYGZgt{} }\PYG{k+kn}{import} \PYG{n+nn}{matplotlib.pyplot} \PYG{k+kn}{as} \PYG{n+nn}{plt}
\PYG{g+gp}{\PYGZgt{}\PYGZgt{}\PYGZgt{} }\PYG{n}{count}\PYG{p}{,} \PYG{n}{bins}\PYG{p}{,} \PYG{n}{ignored} \PYG{o}{=} \PYG{n}{plt}\PYG{o}{.}\PYG{n}{hist}\PYG{p}{(}\PYG{n}{s}\PYG{p}{,} \PYG{l+m+mi}{15}\PYG{p}{,} \PYG{n}{normed}\PYG{o}{=}\PYG{n+nb+bp}{True}\PYG{p}{)}
\PYG{g+gp}{\PYGZgt{}\PYGZgt{}\PYGZgt{} }\PYG{n}{plt}\PYG{o}{.}\PYG{n}{plot}\PYG{p}{(}\PYG{n}{bins}\PYG{p}{,} \PYG{n}{np}\PYG{o}{.}\PYG{n}{ones\PYGZus{}like}\PYG{p}{(}\PYG{n}{bins}\PYG{p}{)}\PYG{p}{,} \PYG{n}{linewidth}\PYG{o}{=}\PYG{l+m+mi}{2}\PYG{p}{,} \PYG{n}{color}\PYG{o}{=}\PYG{l+s}{\PYGZsq{}}\PYG{l+s}{r}\PYG{l+s}{\PYGZsq{}}\PYG{p}{)}
\PYG{g+gp}{\PYGZgt{}\PYGZgt{}\PYGZgt{} }\PYG{n}{plt}\PYG{o}{.}\PYG{n}{show}\PYG{p}{(}\PYG{p}{)}
\end{Verbatim}

\end{fulllineitems}

\index{vonmises() (in module lib.graph.scc)}

\begin{fulllineitems}
\phantomsection\label{lib.graph:lib.graph.scc.vonmises}\pysiglinewithargsret{\code{lib.graph.scc.}\bfcode{vonmises}}{\emph{mu}, \emph{kappa}, \emph{size=None}}{}
Draw samples from a von Mises distribution.

Samples are drawn from a von Mises distribution with specified mode
(mu) and dispersion (kappa), on the interval {[}-pi, pi{]}.

The von Mises distribution (also known as the circular normal
distribution) is a continuous probability distribution on the unit
circle.  It may be thought of as the circular analogue of the normal
distribution.
\begin{description}
\item[{mu}] \leavevmode{[}float{]}
Mode (``center'') of the distribution.

\item[{kappa}] \leavevmode{[}float{]}
Dispersion of the distribution, has to be \textgreater{}=0.

\item[{size}] \leavevmode{[}int or tuple of int{]}
Output shape.  If the given shape is, e.g., \code{(m, n, k)}, then
\code{m * n * k} samples are drawn.

\end{description}
\begin{description}
\item[{samples}] \leavevmode{[}scalar or ndarray{]}
The returned samples, which are in the interval {[}-pi, pi{]}.

\end{description}
\begin{description}
\item[{scipy.stats.distributions.vonmises}] \leavevmode{[}probability density function,{]}
distribution, or cumulative density function, etc.

\end{description}

The probability density for the von Mises distribution is
\begin{gather}
\begin{split}p(x) = \frac{e^{\kappa cos(x-\mu)}}{2\pi I_0(\kappa)},\end{split}\notag
\end{gather}
where \(\mu\) is the mode and \(\kappa\) the dispersion,
and \(I_0(\kappa)\) is the modified Bessel function of order 0.

The von Mises is named for Richard Edler von Mises, who was born in
Austria-Hungary, in what is now the Ukraine.  He fled to the United
States in 1939 and became a professor at Harvard.  He worked in
probability theory, aerodynamics, fluid mechanics, and philosophy of
science.

Abramowitz, M. and Stegun, I. A. (ed.), \emph{Handbook of Mathematical
Functions}, New York: Dover, 1965.

von Mises, R., \emph{Mathematical Theory of Probability and Statistics},
New York: Academic Press, 1964.

Draw samples from the distribution:

\begin{Verbatim}[commandchars=\\\{\}]
\PYG{g+gp}{\PYGZgt{}\PYGZgt{}\PYGZgt{} }\PYG{n}{mu}\PYG{p}{,} \PYG{n}{kappa} \PYG{o}{=} \PYG{l+m+mf}{0.0}\PYG{p}{,} \PYG{l+m+mf}{4.0} \PYG{c}{\PYGZsh{} mean and dispersion}
\PYG{g+gp}{\PYGZgt{}\PYGZgt{}\PYGZgt{} }\PYG{n}{s} \PYG{o}{=} \PYG{n}{np}\PYG{o}{.}\PYG{n}{random}\PYG{o}{.}\PYG{n}{vonmises}\PYG{p}{(}\PYG{n}{mu}\PYG{p}{,} \PYG{n}{kappa}\PYG{p}{,} \PYG{l+m+mi}{1000}\PYG{p}{)}
\end{Verbatim}

Display the histogram of the samples, along with
the probability density function:

\begin{Verbatim}[commandchars=\\\{\}]
\PYG{g+gp}{\PYGZgt{}\PYGZgt{}\PYGZgt{} }\PYG{k+kn}{import} \PYG{n+nn}{matplotlib.pyplot} \PYG{k+kn}{as} \PYG{n+nn}{plt}
\PYG{g+gp}{\PYGZgt{}\PYGZgt{}\PYGZgt{} }\PYG{k+kn}{import} \PYG{n+nn}{scipy.special} \PYG{k+kn}{as} \PYG{n+nn}{sps}
\PYG{g+gp}{\PYGZgt{}\PYGZgt{}\PYGZgt{} }\PYG{n}{count}\PYG{p}{,} \PYG{n}{bins}\PYG{p}{,} \PYG{n}{ignored} \PYG{o}{=} \PYG{n}{plt}\PYG{o}{.}\PYG{n}{hist}\PYG{p}{(}\PYG{n}{s}\PYG{p}{,} \PYG{l+m+mi}{50}\PYG{p}{,} \PYG{n}{normed}\PYG{o}{=}\PYG{n+nb+bp}{True}\PYG{p}{)}
\PYG{g+gp}{\PYGZgt{}\PYGZgt{}\PYGZgt{} }\PYG{n}{x} \PYG{o}{=} \PYG{n}{np}\PYG{o}{.}\PYG{n}{arange}\PYG{p}{(}\PYG{o}{\PYGZhy{}}\PYG{n}{np}\PYG{o}{.}\PYG{n}{pi}\PYG{p}{,} \PYG{n}{np}\PYG{o}{.}\PYG{n}{pi}\PYG{p}{,} \PYG{l+m+mi}{2}\PYG{o}{*}\PYG{n}{np}\PYG{o}{.}\PYG{n}{pi}\PYG{o}{/}\PYG{l+m+mf}{50.}\PYG{p}{)}
\PYG{g+gp}{\PYGZgt{}\PYGZgt{}\PYGZgt{} }\PYG{n}{y} \PYG{o}{=} \PYG{o}{\PYGZhy{}}\PYG{n}{np}\PYG{o}{.}\PYG{n}{exp}\PYG{p}{(}\PYG{n}{kappa}\PYG{o}{*}\PYG{n}{np}\PYG{o}{.}\PYG{n}{cos}\PYG{p}{(}\PYG{n}{x}\PYG{o}{\PYGZhy{}}\PYG{n}{mu}\PYG{p}{)}\PYG{p}{)}\PYG{o}{/}\PYG{p}{(}\PYG{l+m+mi}{2}\PYG{o}{*}\PYG{n}{np}\PYG{o}{.}\PYG{n}{pi}\PYG{o}{*}\PYG{n}{sps}\PYG{o}{.}\PYG{n}{jn}\PYG{p}{(}\PYG{l+m+mi}{0}\PYG{p}{,}\PYG{n}{kappa}\PYG{p}{)}\PYG{p}{)}
\PYG{g+gp}{\PYGZgt{}\PYGZgt{}\PYGZgt{} }\PYG{n}{plt}\PYG{o}{.}\PYG{n}{plot}\PYG{p}{(}\PYG{n}{x}\PYG{p}{,} \PYG{n}{y}\PYG{o}{/}\PYG{n+nb}{max}\PYG{p}{(}\PYG{n}{y}\PYG{p}{)}\PYG{p}{,} \PYG{n}{linewidth}\PYG{o}{=}\PYG{l+m+mi}{2}\PYG{p}{,} \PYG{n}{color}\PYG{o}{=}\PYG{l+s}{\PYGZsq{}}\PYG{l+s}{r}\PYG{l+s}{\PYGZsq{}}\PYG{p}{)}
\PYG{g+gp}{\PYGZgt{}\PYGZgt{}\PYGZgt{} }\PYG{n}{plt}\PYG{o}{.}\PYG{n}{show}\PYG{p}{(}\PYG{p}{)}
\end{Verbatim}

\end{fulllineitems}

\index{wald() (in module lib.graph.scc)}

\begin{fulllineitems}
\phantomsection\label{lib.graph:lib.graph.scc.wald}\pysiglinewithargsret{\code{lib.graph.scc.}\bfcode{wald}}{\emph{mean}, \emph{scale}, \emph{size=None}}{}
Draw samples from a Wald, or Inverse Gaussian, distribution.

As the scale approaches infinity, the distribution becomes more like a
Gaussian.

Some references claim that the Wald is an Inverse Gaussian with mean=1, but
this is by no means universal.

The Inverse Gaussian distribution was first studied in relationship to
Brownian motion. In 1956 M.C.K. Tweedie used the name Inverse Gaussian
because there is an inverse relationship between the time to cover a unit
distance and distance covered in unit time.
\begin{description}
\item[{mean}] \leavevmode{[}scalar{]}
Distribution mean, should be \textgreater{} 0.

\item[{scale}] \leavevmode{[}scalar{]}
Scale parameter, should be \textgreater{}= 0.

\item[{size}] \leavevmode{[}int or tuple of ints, optional{]}
Output shape. Default is None, in which case a single value is
returned.

\end{description}
\begin{description}
\item[{samples}] \leavevmode{[}ndarray or scalar{]}
Drawn sample, all greater than zero.

\end{description}

The probability density function for the Wald distribution is
\begin{gather}
\begin{split}P(x;mean,scale) = \sqrt{\frac{scale}{2\pi x^3}}e^
\frac{-scale(x-mean)^2}{2\cdotp mean^2x}\end{split}\notag
\end{gather}
As noted above the Inverse Gaussian distribution first arise from attempts
to model Brownian Motion. It is also a competitor to the Weibull for use in
reliability modeling and modeling stock returns and interest rate
processes.

Draw values from the distribution and plot the histogram:

\begin{Verbatim}[commandchars=\\\{\}]
\PYG{g+gp}{\PYGZgt{}\PYGZgt{}\PYGZgt{} }\PYG{k+kn}{import} \PYG{n+nn}{matplotlib.pyplot} \PYG{k+kn}{as} \PYG{n+nn}{plt}
\PYG{g+gp}{\PYGZgt{}\PYGZgt{}\PYGZgt{} }\PYG{n}{h} \PYG{o}{=} \PYG{n}{plt}\PYG{o}{.}\PYG{n}{hist}\PYG{p}{(}\PYG{n}{np}\PYG{o}{.}\PYG{n}{random}\PYG{o}{.}\PYG{n}{wald}\PYG{p}{(}\PYG{l+m+mi}{3}\PYG{p}{,} \PYG{l+m+mi}{2}\PYG{p}{,} \PYG{l+m+mi}{100000}\PYG{p}{)}\PYG{p}{,} \PYG{n}{bins}\PYG{o}{=}\PYG{l+m+mi}{200}\PYG{p}{,} \PYG{n}{normed}\PYG{o}{=}\PYG{n+nb+bp}{True}\PYG{p}{)}
\PYG{g+gp}{\PYGZgt{}\PYGZgt{}\PYGZgt{} }\PYG{n}{plt}\PYG{o}{.}\PYG{n}{show}\PYG{p}{(}\PYG{p}{)}
\end{Verbatim}

\end{fulllineitems}

\index{weibull() (in module lib.graph.scc)}

\begin{fulllineitems}
\phantomsection\label{lib.graph:lib.graph.scc.weibull}\pysiglinewithargsret{\code{lib.graph.scc.}\bfcode{weibull}}{\emph{a}, \emph{size=None}}{}
Weibull distribution.

Draw samples from a 1-parameter Weibull distribution with the given
shape parameter \emph{a}.
\begin{gather}
\begin{split}X = (-ln(U))^{1/a}\end{split}\notag
\end{gather}
Here, U is drawn from the uniform distribution over (0,1{]}.

The more common 2-parameter Weibull, including a scale parameter
\(\lambda\) is just \(X = \lambda(-ln(U))^{1/a}\).
\begin{description}
\item[{a}] \leavevmode{[}float{]}
Shape of the distribution.

\item[{size}] \leavevmode{[}tuple of ints{]}
Output shape.  If the given shape is, e.g., \code{(m, n, k)}, then
\code{m * n * k} samples are drawn.

\end{description}

scipy.stats.distributions.weibull\_max
scipy.stats.distributions.weibull\_min
scipy.stats.distributions.genextreme
gumbel

The Weibull (or Type III asymptotic extreme value distribution for smallest
values, SEV Type III, or Rosin-Rammler distribution) is one of a class of
Generalized Extreme Value (GEV) distributions used in modeling extreme
value problems.  This class includes the Gumbel and Frechet distributions.

The probability density for the Weibull distribution is
\begin{gather}
\begin{split}p(x) = \frac{a}
{\lambda}(\frac{x}{\lambda})^{a-1}e^{-(x/\lambda)^a},\end{split}\notag
\end{gather}
where \(a\) is the shape and \(\lambda\) the scale.

The function has its peak (the mode) at
\(\lambda(\frac{a-1}{a})^{1/a}\).

When \code{a = 1}, the Weibull distribution reduces to the exponential
distribution.

Draw samples from the distribution:

\begin{Verbatim}[commandchars=\\\{\}]
\PYG{g+gp}{\PYGZgt{}\PYGZgt{}\PYGZgt{} }\PYG{n}{a} \PYG{o}{=} \PYG{l+m+mf}{5.} \PYG{c}{\PYGZsh{} shape}
\PYG{g+gp}{\PYGZgt{}\PYGZgt{}\PYGZgt{} }\PYG{n}{s} \PYG{o}{=} \PYG{n}{np}\PYG{o}{.}\PYG{n}{random}\PYG{o}{.}\PYG{n}{weibull}\PYG{p}{(}\PYG{n}{a}\PYG{p}{,} \PYG{l+m+mi}{1000}\PYG{p}{)}
\end{Verbatim}

Display the histogram of the samples, along with
the probability density function:

\begin{Verbatim}[commandchars=\\\{\}]
\PYG{g+gp}{\PYGZgt{}\PYGZgt{}\PYGZgt{} }\PYG{k+kn}{import} \PYG{n+nn}{matplotlib.pyplot} \PYG{k+kn}{as} \PYG{n+nn}{plt}
\PYG{g+gp}{\PYGZgt{}\PYGZgt{}\PYGZgt{} }\PYG{n}{x} \PYG{o}{=} \PYG{n}{np}\PYG{o}{.}\PYG{n}{arange}\PYG{p}{(}\PYG{l+m+mi}{1}\PYG{p}{,}\PYG{l+m+mf}{100.}\PYG{p}{)}\PYG{o}{/}\PYG{l+m+mf}{50.}
\PYG{g+gp}{\PYGZgt{}\PYGZgt{}\PYGZgt{} }\PYG{k}{def} \PYG{n+nf}{weib}\PYG{p}{(}\PYG{n}{x}\PYG{p}{,}\PYG{n}{n}\PYG{p}{,}\PYG{n}{a}\PYG{p}{)}\PYG{p}{:}
\PYG{g+gp}{... }    \PYG{k}{return} \PYG{p}{(}\PYG{n}{a} \PYG{o}{/} \PYG{n}{n}\PYG{p}{)} \PYG{o}{*} \PYG{p}{(}\PYG{n}{x} \PYG{o}{/} \PYG{n}{n}\PYG{p}{)}\PYG{o}{*}\PYG{o}{*}\PYG{p}{(}\PYG{n}{a} \PYG{o}{\PYGZhy{}} \PYG{l+m+mi}{1}\PYG{p}{)} \PYG{o}{*} \PYG{n}{np}\PYG{o}{.}\PYG{n}{exp}\PYG{p}{(}\PYG{o}{\PYGZhy{}}\PYG{p}{(}\PYG{n}{x} \PYG{o}{/} \PYG{n}{n}\PYG{p}{)}\PYG{o}{*}\PYG{o}{*}\PYG{n}{a}\PYG{p}{)}
\end{Verbatim}

\begin{Verbatim}[commandchars=\\\{\}]
\PYG{g+gp}{\PYGZgt{}\PYGZgt{}\PYGZgt{} }\PYG{n}{count}\PYG{p}{,} \PYG{n}{bins}\PYG{p}{,} \PYG{n}{ignored} \PYG{o}{=} \PYG{n}{plt}\PYG{o}{.}\PYG{n}{hist}\PYG{p}{(}\PYG{n}{np}\PYG{o}{.}\PYG{n}{random}\PYG{o}{.}\PYG{n}{weibull}\PYG{p}{(}\PYG{l+m+mf}{5.}\PYG{p}{,}\PYG{l+m+mi}{1000}\PYG{p}{)}\PYG{p}{)}
\PYG{g+gp}{\PYGZgt{}\PYGZgt{}\PYGZgt{} }\PYG{n}{x} \PYG{o}{=} \PYG{n}{np}\PYG{o}{.}\PYG{n}{arange}\PYG{p}{(}\PYG{l+m+mi}{1}\PYG{p}{,}\PYG{l+m+mf}{100.}\PYG{p}{)}\PYG{o}{/}\PYG{l+m+mf}{50.}
\PYG{g+gp}{\PYGZgt{}\PYGZgt{}\PYGZgt{} }\PYG{n}{scale} \PYG{o}{=} \PYG{n}{count}\PYG{o}{.}\PYG{n}{max}\PYG{p}{(}\PYG{p}{)}\PYG{o}{/}\PYG{n}{weib}\PYG{p}{(}\PYG{n}{x}\PYG{p}{,} \PYG{l+m+mf}{1.}\PYG{p}{,} \PYG{l+m+mf}{5.}\PYG{p}{)}\PYG{o}{.}\PYG{n}{max}\PYG{p}{(}\PYG{p}{)}
\PYG{g+gp}{\PYGZgt{}\PYGZgt{}\PYGZgt{} }\PYG{n}{plt}\PYG{o}{.}\PYG{n}{plot}\PYG{p}{(}\PYG{n}{x}\PYG{p}{,} \PYG{n}{weib}\PYG{p}{(}\PYG{n}{x}\PYG{p}{,} \PYG{l+m+mf}{1.}\PYG{p}{,} \PYG{l+m+mf}{5.}\PYG{p}{)}\PYG{o}{*}\PYG{n}{scale}\PYG{p}{)}
\PYG{g+gp}{\PYGZgt{}\PYGZgt{}\PYGZgt{} }\PYG{n}{plt}\PYG{o}{.}\PYG{n}{show}\PYG{p}{(}\PYG{p}{)}
\end{Verbatim}

\end{fulllineitems}

\index{zipf() (in module lib.graph.scc)}

\begin{fulllineitems}
\phantomsection\label{lib.graph:lib.graph.scc.zipf}\pysiglinewithargsret{\code{lib.graph.scc.}\bfcode{zipf}}{\emph{a}, \emph{size=None}}{}
Draw samples from a Zipf distribution.

Samples are drawn from a Zipf distribution with specified parameter
\emph{a} \textgreater{} 1.

The Zipf distribution (also known as the zeta distribution) is a
continuous probability distribution that satisfies Zipf's law: the
frequency of an item is inversely proportional to its rank in a
frequency table.
\begin{description}
\item[{a}] \leavevmode{[}float \textgreater{} 1{]}
Distribution parameter.

\item[{size}] \leavevmode{[}int or tuple of int, optional{]}
Output shape.  If the given shape is, e.g., \code{(m, n, k)}, then
\code{m * n * k} samples are drawn; a single integer is equivalent in
its result to providing a mono-tuple, i.e., a 1-D array of length
\emph{size} is returned.  The default is None, in which case a single
scalar is returned.

\end{description}
\begin{description}
\item[{samples}] \leavevmode{[}scalar or ndarray{]}
The returned samples are greater than or equal to one.

\end{description}
\begin{description}
\item[{scipy.stats.distributions.zipf}] \leavevmode{[}probability density function,{]}
distribution, or cumulative density function, etc.

\end{description}

The probability density for the Zipf distribution is
\begin{gather}
\begin{split}p(x) = \frac{x^{-a}}{\zeta(a)},\end{split}\notag
\end{gather}
where \(\zeta\) is the Riemann Zeta function.

It is named for the American linguist George Kingsley Zipf, who noted
that the frequency of any word in a sample of a language is inversely
proportional to its rank in the frequency table.

Zipf, G. K., \emph{Selected Studies of the Principle of Relative Frequency
in Language}, Cambridge, MA: Harvard Univ. Press, 1932.

Draw samples from the distribution:

\begin{Verbatim}[commandchars=\\\{\}]
\PYG{g+gp}{\PYGZgt{}\PYGZgt{}\PYGZgt{} }\PYG{n}{a} \PYG{o}{=} \PYG{l+m+mf}{2.} \PYG{c}{\PYGZsh{} parameter}
\PYG{g+gp}{\PYGZgt{}\PYGZgt{}\PYGZgt{} }\PYG{n}{s} \PYG{o}{=} \PYG{n}{np}\PYG{o}{.}\PYG{n}{random}\PYG{o}{.}\PYG{n}{zipf}\PYG{p}{(}\PYG{n}{a}\PYG{p}{,} \PYG{l+m+mi}{1000}\PYG{p}{)}
\end{Verbatim}

Display the histogram of the samples, along with
the probability density function:

\begin{Verbatim}[commandchars=\\\{\}]
\PYG{g+gp}{\PYGZgt{}\PYGZgt{}\PYGZgt{} }\PYG{k+kn}{import} \PYG{n+nn}{matplotlib.pyplot} \PYG{k+kn}{as} \PYG{n+nn}{plt}
\PYG{g+gp}{\PYGZgt{}\PYGZgt{}\PYGZgt{} }\PYG{k+kn}{import} \PYG{n+nn}{scipy.special} \PYG{k+kn}{as} \PYG{n+nn}{sps}
\PYG{g+go}{Truncate s values at 50 so plot is interesting}
\PYG{g+gp}{\PYGZgt{}\PYGZgt{}\PYGZgt{} }\PYG{n}{count}\PYG{p}{,} \PYG{n}{bins}\PYG{p}{,} \PYG{n}{ignored} \PYG{o}{=} \PYG{n}{plt}\PYG{o}{.}\PYG{n}{hist}\PYG{p}{(}\PYG{n}{s}\PYG{p}{[}\PYG{n}{s}\PYG{o}{\PYGZlt{}}\PYG{l+m+mi}{50}\PYG{p}{]}\PYG{p}{,} \PYG{l+m+mi}{50}\PYG{p}{,} \PYG{n}{normed}\PYG{o}{=}\PYG{n+nb+bp}{True}\PYG{p}{)}
\PYG{g+gp}{\PYGZgt{}\PYGZgt{}\PYGZgt{} }\PYG{n}{x} \PYG{o}{=} \PYG{n}{np}\PYG{o}{.}\PYG{n}{arange}\PYG{p}{(}\PYG{l+m+mf}{1.}\PYG{p}{,} \PYG{l+m+mf}{50.}\PYG{p}{)}
\PYG{g+gp}{\PYGZgt{}\PYGZgt{}\PYGZgt{} }\PYG{n}{y} \PYG{o}{=} \PYG{n}{x}\PYG{o}{*}\PYG{o}{*}\PYG{p}{(}\PYG{o}{\PYGZhy{}}\PYG{n}{a}\PYG{p}{)}\PYG{o}{/}\PYG{n}{sps}\PYG{o}{.}\PYG{n}{zetac}\PYG{p}{(}\PYG{n}{a}\PYG{p}{)}
\PYG{g+gp}{\PYGZgt{}\PYGZgt{}\PYGZgt{} }\PYG{n}{plt}\PYG{o}{.}\PYG{n}{plot}\PYG{p}{(}\PYG{n}{x}\PYG{p}{,} \PYG{n}{y}\PYG{o}{/}\PYG{n+nb}{max}\PYG{p}{(}\PYG{n}{y}\PYG{p}{)}\PYG{p}{,} \PYG{n}{linewidth}\PYG{o}{=}\PYG{l+m+mi}{2}\PYG{p}{,} \PYG{n}{color}\PYG{o}{=}\PYG{l+s}{\PYGZsq{}}\PYG{l+s}{r}\PYG{l+s}{\PYGZsq{}}\PYG{p}{)}
\PYG{g+gp}{\PYGZgt{}\PYGZgt{}\PYGZgt{} }\PYG{n}{plt}\PYG{o}{.}\PYG{n}{show}\PYG{p}{(}\PYG{p}{)}
\end{Verbatim}

\end{fulllineitems}



\subsection{model Package}
\label{lib.model::doc}\label{lib.model:model-package}

\subsubsection{\texttt{reactions} Module}
\label{lib.model:reactions-module}\label{lib.model:module-lib.model.reactions}\index{lib.model.reactions (module)}\index{createRandomCleavage() (in module lib.model.reactions)}

\begin{fulllineitems}
\phantomsection\label{lib.model:lib.model.reactions.createRandomCleavage}\pysiglinewithargsret{\code{lib.model.reactions.}\bfcode{createRandomCleavage}}{\emph{tmpSpecies}, \emph{alphabet}, \emph{tmpInitLMax}}{}
\end{fulllineitems}

\index{createRandomCleavageForCompleteFiringDisk() (in module lib.model.reactions)}

\begin{fulllineitems}
\phantomsection\label{lib.model:lib.model.reactions.createRandomCleavageForCompleteFiringDisk}\pysiglinewithargsret{\code{lib.model.reactions.}\bfcode{createRandomCleavageForCompleteFiringDisk}}{\emph{tmpSpecies}, \emph{alphabet}, \emph{tmpInitLMax}}{}
\end{fulllineitems}

\index{createRandomCondensation() (in module lib.model.reactions)}

\begin{fulllineitems}
\phantomsection\label{lib.model:lib.model.reactions.createRandomCondensation}\pysiglinewithargsret{\code{lib.model.reactions.}\bfcode{createRandomCondensation}}{\emph{tmpSpecies}, \emph{tmpInitLMax}}{}
\end{fulllineitems}

\index{getNumOfCleavages() (in module lib.model.reactions)}

\begin{fulllineitems}
\phantomsection\label{lib.model:lib.model.reactions.getNumOfCleavages}\pysiglinewithargsret{\code{lib.model.reactions.}\bfcode{getNumOfCleavages}}{\emph{tmpSpecies}}{}
\end{fulllineitems}

\index{getNumOfCondensations() (in module lib.model.reactions)}

\begin{fulllineitems}
\phantomsection\label{lib.model:lib.model.reactions.getNumOfCondensations}\pysiglinewithargsret{\code{lib.model.reactions.}\bfcode{getNumOfCondensations}}{\emph{N}}{}
\end{fulllineitems}



\subsubsection{\texttt{species} Module}
\label{lib.model:species-module}\label{lib.model:module-lib.model.species}\index{lib.model.species (module)}\index{createCompleteSpeciesPopulation() (in module lib.model.species)}

\begin{fulllineitems}
\phantomsection\label{lib.model:lib.model.species.createCompleteSpeciesPopulation}\pysiglinewithargsret{\code{lib.model.species.}\bfcode{createCompleteSpeciesPopulation}}{\emph{M}, \emph{alphabet}}{}
\end{fulllineitems}

\index{createFileSpecies() (in module lib.model.species)}

\begin{fulllineitems}
\phantomsection\label{lib.model:lib.model.species.createFileSpecies}\pysiglinewithargsret{\code{lib.model.species.}\bfcode{createFileSpecies}}{\emph{tmpFolder}, \emph{args}, \emph{pars}, \emph{tmpScale=1}, \emph{specieslist=None}, \emph{tmpCatInRAF=None}, \emph{tmpRafCatContribute2C=1}}{}
\end{fulllineitems}

\index{getTotNumberOfSpeciesFromCompletePop() (in module lib.model.species)}

\begin{fulllineitems}
\phantomsection\label{lib.model:lib.model.species.getTotNumberOfSpeciesFromCompletePop}\pysiglinewithargsret{\code{lib.model.species.}\bfcode{getTotNumberOfSpeciesFromCompletePop}}{\emph{M}}{}
\end{fulllineitems}

\index{weightedChoice() (in module lib.model.species)}

\begin{fulllineitems}
\phantomsection\label{lib.model:lib.model.species.weightedChoice}\pysiglinewithargsret{\code{lib.model.species.}\bfcode{weightedChoice}}{\emph{weights}, \emph{objects}}{}
Return a random item from objects, with the weighting defined by weights 
(which must sum to 1).

\end{fulllineitems}



\chapter{main Module}
\label{main:module-main}\label{main:main-module}\label{main::doc}\index{main (module)}
MAIN analysis script package for CaRNeSS simulations
Main file analysis
\index{beta() (in module main)}

\begin{fulllineitems}
\phantomsection\label{main:main.beta}\pysiglinewithargsret{\code{main.}\bfcode{beta}}{\emph{a}, \emph{b}, \emph{size=None}}{}
The Beta distribution over \code{{[}0, 1{]}}.

The Beta distribution is a special case of the Dirichlet distribution,
and is related to the Gamma distribution.  It has the probability
distribution function
\begin{gather}
\begin{split}f(x; a,b) = \frac{1}{B(\alpha, \beta)} x^{\alpha - 1}
(1 - x)^{\beta - 1},\end{split}\notag
\end{gather}
where the normalisation, B, is the beta function,
\begin{gather}
\begin{split}B(\alpha, \beta) = \int_0^1 t^{\alpha - 1}
(1 - t)^{\beta - 1} dt.\end{split}\notag
\end{gather}
It is often seen in Bayesian inference and order statistics.
\begin{description}
\item[{a}] \leavevmode{[}float{]}
Alpha, non-negative.

\item[{b}] \leavevmode{[}float{]}
Beta, non-negative.

\item[{size}] \leavevmode{[}tuple of ints, optional{]}
The number of samples to draw.  The output is packed according to
the size given.

\end{description}
\begin{description}
\item[{out}] \leavevmode{[}ndarray{]}
Array of the given shape, containing values drawn from a
Beta distribution.

\end{description}

\end{fulllineitems}

\index{binomial() (in module main)}

\begin{fulllineitems}
\phantomsection\label{main:main.binomial}\pysiglinewithargsret{\code{main.}\bfcode{binomial}}{\emph{n}, \emph{p}, \emph{size=None}}{}
Draw samples from a binomial distribution.

Samples are drawn from a Binomial distribution with specified
parameters, n trials and p probability of success where
n an integer \textgreater{}= 0 and p is in the interval {[}0,1{]}. (n may be
input as a float, but it is truncated to an integer in use)
\begin{description}
\item[{n}] \leavevmode{[}float (but truncated to an integer){]}
parameter, \textgreater{}= 0.

\item[{p}] \leavevmode{[}float{]}
parameter, \textgreater{}= 0 and \textless{}=1.

\item[{size}] \leavevmode{[}\{tuple, int\}{]}
Output shape.  If the given shape is, e.g., \code{(m, n, k)}, then
\code{m * n * k} samples are drawn.

\end{description}
\begin{description}
\item[{samples}] \leavevmode{[}\{ndarray, scalar\}{]}
where the values are all integers in  {[}0, n{]}.

\end{description}
\begin{description}
\item[{scipy.stats.distributions.binom}] \leavevmode{[}probability density function,{]}
distribution or cumulative density function, etc.

\end{description}

The probability density for the Binomial distribution is
\begin{gather}
\begin{split}P(N) = \binom{n}{N}p^N(1-p)^{n-N},\end{split}\notag
\end{gather}
where \(n\) is the number of trials, \(p\) is the probability
of success, and \(N\) is the number of successes.

When estimating the standard error of a proportion in a population by
using a random sample, the normal distribution works well unless the
product p*n \textless{}=5, where p = population proportion estimate, and n =
number of samples, in which case the binomial distribution is used
instead. For example, a sample of 15 people shows 4 who are left
handed, and 11 who are right handed. Then p = 4/15 = 27\%. 0.27*15 = 4,
so the binomial distribution should be used in this case.

Draw samples from the distribution:

\begin{Verbatim}[commandchars=\\\{\}]
\PYG{g+gp}{\PYGZgt{}\PYGZgt{}\PYGZgt{} }\PYG{n}{n}\PYG{p}{,} \PYG{n}{p} \PYG{o}{=} \PYG{l+m+mi}{10}\PYG{p}{,} \PYG{o}{.}\PYG{l+m+mi}{5} \PYG{c}{\PYGZsh{} number of trials, probability of each trial}
\PYG{g+gp}{\PYGZgt{}\PYGZgt{}\PYGZgt{} }\PYG{n}{s} \PYG{o}{=} \PYG{n}{np}\PYG{o}{.}\PYG{n}{random}\PYG{o}{.}\PYG{n}{binomial}\PYG{p}{(}\PYG{n}{n}\PYG{p}{,} \PYG{n}{p}\PYG{p}{,} \PYG{l+m+mi}{1000}\PYG{p}{)}
\PYG{g+go}{\PYGZsh{} result of flipping a coin 10 times, tested 1000 times.}
\end{Verbatim}

A real world example. A company drills 9 wild-cat oil exploration
wells, each with an estimated probability of success of 0.1. All nine
wells fail. What is the probability of that happening?

Let's do 20,000 trials of the model, and count the number that
generate zero positive results.

\begin{Verbatim}[commandchars=\\\{\}]
\PYG{g+gp}{\PYGZgt{}\PYGZgt{}\PYGZgt{} }\PYG{n+nb}{sum}\PYG{p}{(}\PYG{n}{np}\PYG{o}{.}\PYG{n}{random}\PYG{o}{.}\PYG{n}{binomial}\PYG{p}{(}\PYG{l+m+mi}{9}\PYG{p}{,}\PYG{l+m+mf}{0.1}\PYG{p}{,}\PYG{l+m+mi}{20000}\PYG{p}{)}\PYG{o}{==}\PYG{l+m+mi}{0}\PYG{p}{)}\PYG{o}{/}\PYG{l+m+mf}{20000.}
\PYG{g+go}{answer = 0.38885, or 38\PYGZpc{}.}
\end{Verbatim}

\end{fulllineitems}

\index{chisquare() (in module main)}

\begin{fulllineitems}
\phantomsection\label{main:main.chisquare}\pysiglinewithargsret{\code{main.}\bfcode{chisquare}}{\emph{df}, \emph{size=None}}{}
Draw samples from a chi-square distribution.

When \emph{df} independent random variables, each with standard normal
distributions (mean 0, variance 1), are squared and summed, the
resulting distribution is chi-square (see Notes).  This distribution
is often used in hypothesis testing.
\begin{description}
\item[{df}] \leavevmode{[}int{]}
Number of degrees of freedom.

\item[{size}] \leavevmode{[}tuple of ints, int, optional{]}
Size of the returned array.  By default, a scalar is
returned.

\end{description}
\begin{description}
\item[{output}] \leavevmode{[}ndarray{]}
Samples drawn from the distribution, packed in a \emph{size}-shaped
array.

\end{description}
\begin{description}
\item[{ValueError}] \leavevmode
When \emph{df} \textless{}= 0 or when an inappropriate \emph{size} (e.g. \code{size=-1})
is given.

\end{description}

The variable obtained by summing the squares of \emph{df} independent,
standard normally distributed random variables:
\begin{gather}
\begin{split}Q = \sum_{i=0}^{\mathtt{df}} X^2_i\end{split}\notag
\end{gather}
is chi-square distributed, denoted
\begin{gather}
\begin{split}Q \sim \chi^2_k.\end{split}\notag
\end{gather}
The probability density function of the chi-squared distribution is
\begin{gather}
\begin{split}p(x) = \frac{(1/2)^{k/2}}{\Gamma(k/2)}
x^{k/2 - 1} e^{-x/2},\end{split}\notag
\end{gather}
where \(\Gamma\) is the gamma function,
\begin{gather}
\begin{split}\Gamma(x) = \int_0^{-\infty} t^{x - 1} e^{-t} dt.\end{split}\notag
\end{gather}
\href{http://www.itl.nist.gov/div898/handbook/eda/section3/eda3666.htm}{NIST/SEMATECH e-Handbook of Statistical Methods}

\begin{Verbatim}[commandchars=\\\{\}]
\PYG{g+gp}{\PYGZgt{}\PYGZgt{}\PYGZgt{} }\PYG{n}{np}\PYG{o}{.}\PYG{n}{random}\PYG{o}{.}\PYG{n}{chisquare}\PYG{p}{(}\PYG{l+m+mi}{2}\PYG{p}{,}\PYG{l+m+mi}{4}\PYG{p}{)}
\PYG{g+go}{array([ 1.89920014,  9.00867716,  3.13710533,  5.62318272])}
\end{Verbatim}

\end{fulllineitems}

\index{exponential() (in module main)}

\begin{fulllineitems}
\phantomsection\label{main:main.exponential}\pysiglinewithargsret{\code{main.}\bfcode{exponential}}{\emph{scale=1.0}, \emph{size=None}}{}
Exponential distribution.

Its probability density function is
\begin{gather}
\begin{split}f(x; \frac{1}{\beta}) = \frac{1}{\beta} \exp(-\frac{x}{\beta}),\end{split}\notag
\end{gather}
for \code{x \textgreater{} 0} and 0 elsewhere. \(\beta\) is the scale parameter,
which is the inverse of the rate parameter \(\lambda = 1/\beta\).
The rate parameter is an alternative, widely used parameterization
of the exponential distribution {\color{red}\bfseries{}{[}3{]}\_}.

The exponential distribution is a continuous analogue of the
geometric distribution.  It describes many common situations, such as
the size of raindrops measured over many rainstorms {\color{red}\bfseries{}{[}1{]}\_}, or the time
between page requests to Wikipedia {\color{red}\bfseries{}{[}2{]}\_}.
\begin{description}
\item[{scale}] \leavevmode{[}float{]}
The scale parameter, \(\beta = 1/\lambda\).

\item[{size}] \leavevmode{[}tuple of ints{]}
Number of samples to draw.  The output is shaped
according to \emph{size}.

\end{description}

\end{fulllineitems}

\index{f() (in module main)}

\begin{fulllineitems}
\phantomsection\label{main:main.f}\pysiglinewithargsret{\code{main.}\bfcode{f}}{\emph{dfnum}, \emph{dfden}, \emph{size=None}}{}
Draw samples from a F distribution.

Samples are drawn from an F distribution with specified parameters,
\emph{dfnum} (degrees of freedom in numerator) and \emph{dfden} (degrees of freedom
in denominator), where both parameters should be greater than zero.

The random variate of the F distribution (also known as the
Fisher distribution) is a continuous probability distribution
that arises in ANOVA tests, and is the ratio of two chi-square
variates.
\begin{description}
\item[{dfnum}] \leavevmode{[}float{]}
Degrees of freedom in numerator. Should be greater than zero.

\item[{dfden}] \leavevmode{[}float{]}
Degrees of freedom in denominator. Should be greater than zero.

\item[{size}] \leavevmode{[}\{tuple, int\}, optional{]}
Output shape.  If the given shape is, e.g., \code{(m, n, k)},
then \code{m * n * k} samples are drawn. By default only one sample
is returned.

\end{description}
\begin{description}
\item[{samples}] \leavevmode{[}\{ndarray, scalar\}{]}
Samples from the Fisher distribution.

\end{description}
\begin{description}
\item[{scipy.stats.distributions.f}] \leavevmode{[}probability density function,{]}
distribution or cumulative density function, etc.

\end{description}

The F statistic is used to compare in-group variances to between-group
variances. Calculating the distribution depends on the sampling, and
so it is a function of the respective degrees of freedom in the
problem.  The variable \emph{dfnum} is the number of samples minus one, the
between-groups degrees of freedom, while \emph{dfden} is the within-groups
degrees of freedom, the sum of the number of samples in each group
minus the number of groups.

An example from Glantz{[}1{]}, pp 47-40.
Two groups, children of diabetics (25 people) and children from people
without diabetes (25 controls). Fasting blood glucose was measured,
case group had a mean value of 86.1, controls had a mean value of
82.2. Standard deviations were 2.09 and 2.49 respectively. Are these
data consistent with the null hypothesis that the parents diabetic
status does not affect their children's blood glucose levels?
Calculating the F statistic from the data gives a value of 36.01.

Draw samples from the distribution:

\begin{Verbatim}[commandchars=\\\{\}]
\PYG{g+gp}{\PYGZgt{}\PYGZgt{}\PYGZgt{} }\PYG{n}{dfnum} \PYG{o}{=} \PYG{l+m+mf}{1.} \PYG{c}{\PYGZsh{} between group degrees of freedom}
\PYG{g+gp}{\PYGZgt{}\PYGZgt{}\PYGZgt{} }\PYG{n}{dfden} \PYG{o}{=} \PYG{l+m+mf}{48.} \PYG{c}{\PYGZsh{} within groups degrees of freedom}
\PYG{g+gp}{\PYGZgt{}\PYGZgt{}\PYGZgt{} }\PYG{n}{s} \PYG{o}{=} \PYG{n}{np}\PYG{o}{.}\PYG{n}{random}\PYG{o}{.}\PYG{n}{f}\PYG{p}{(}\PYG{n}{dfnum}\PYG{p}{,} \PYG{n}{dfden}\PYG{p}{,} \PYG{l+m+mi}{1000}\PYG{p}{)}
\end{Verbatim}

The lower bound for the top 1\% of the samples is :

\begin{Verbatim}[commandchars=\\\{\}]
\PYG{g+gp}{\PYGZgt{}\PYGZgt{}\PYGZgt{} }\PYG{n}{sort}\PYG{p}{(}\PYG{n}{s}\PYG{p}{)}\PYG{p}{[}\PYG{o}{\PYGZhy{}}\PYG{l+m+mi}{10}\PYG{p}{]}
\PYG{g+go}{7.61988120985}
\end{Verbatim}

So there is about a 1\% chance that the F statistic will exceed 7.62,
the measured value is 36, so the null hypothesis is rejected at the 1\%
level.

\end{fulllineitems}

\index{gamma() (in module main)}

\begin{fulllineitems}
\phantomsection\label{main:main.gamma}\pysiglinewithargsret{\code{main.}\bfcode{gamma}}{\emph{shape}, \emph{scale=1.0}, \emph{size=None}}{}
Draw samples from a Gamma distribution.

Samples are drawn from a Gamma distribution with specified parameters,
\emph{shape} (sometimes designated ``k'') and \emph{scale} (sometimes designated
``theta''), where both parameters are \textgreater{} 0.
\begin{description}
\item[{shape}] \leavevmode{[}scalar \textgreater{} 0{]}
The shape of the gamma distribution.

\item[{scale}] \leavevmode{[}scalar \textgreater{} 0, optional{]}
The scale of the gamma distribution.  Default is equal to 1.

\item[{size}] \leavevmode{[}shape\_tuple, optional{]}
Output shape.  If the given shape is, e.g., \code{(m, n, k)}, then
\code{m * n * k} samples are drawn.

\end{description}
\begin{description}
\item[{out}] \leavevmode{[}ndarray, float{]}
Returns one sample unless \emph{size} parameter is specified.

\end{description}
\begin{description}
\item[{scipy.stats.distributions.gamma}] \leavevmode{[}probability density function,{]}
distribution or cumulative density function, etc.

\end{description}

The probability density for the Gamma distribution is
\begin{gather}
\begin{split}p(x) = x^{k-1}\frac{e^{-x/\theta}}{\theta^k\Gamma(k)},\end{split}\notag
\end{gather}
where \(k\) is the shape and \(\theta\) the scale,
and \(\Gamma\) is the Gamma function.

The Gamma distribution is often used to model the times to failure of
electronic components, and arises naturally in processes for which the
waiting times between Poisson distributed events are relevant.

Draw samples from the distribution:

\begin{Verbatim}[commandchars=\\\{\}]
\PYG{g+gp}{\PYGZgt{}\PYGZgt{}\PYGZgt{} }\PYG{n}{shape}\PYG{p}{,} \PYG{n}{scale} \PYG{o}{=} \PYG{l+m+mf}{2.}\PYG{p}{,} \PYG{l+m+mf}{2.} \PYG{c}{\PYGZsh{} mean and dispersion}
\PYG{g+gp}{\PYGZgt{}\PYGZgt{}\PYGZgt{} }\PYG{n}{s} \PYG{o}{=} \PYG{n}{np}\PYG{o}{.}\PYG{n}{random}\PYG{o}{.}\PYG{n}{gamma}\PYG{p}{(}\PYG{n}{shape}\PYG{p}{,} \PYG{n}{scale}\PYG{p}{,} \PYG{l+m+mi}{1000}\PYG{p}{)}
\end{Verbatim}

Display the histogram of the samples, along with
the probability density function:

\begin{Verbatim}[commandchars=\\\{\}]
\PYG{g+gp}{\PYGZgt{}\PYGZgt{}\PYGZgt{} }\PYG{k+kn}{import} \PYG{n+nn}{matplotlib.pyplot} \PYG{k+kn}{as} \PYG{n+nn}{plt}
\PYG{g+gp}{\PYGZgt{}\PYGZgt{}\PYGZgt{} }\PYG{k+kn}{import} \PYG{n+nn}{scipy.special} \PYG{k+kn}{as} \PYG{n+nn}{sps}
\PYG{g+gp}{\PYGZgt{}\PYGZgt{}\PYGZgt{} }\PYG{n}{count}\PYG{p}{,} \PYG{n}{bins}\PYG{p}{,} \PYG{n}{ignored} \PYG{o}{=} \PYG{n}{plt}\PYG{o}{.}\PYG{n}{hist}\PYG{p}{(}\PYG{n}{s}\PYG{p}{,} \PYG{l+m+mi}{50}\PYG{p}{,} \PYG{n}{normed}\PYG{o}{=}\PYG{n+nb+bp}{True}\PYG{p}{)}
\PYG{g+gp}{\PYGZgt{}\PYGZgt{}\PYGZgt{} }\PYG{n}{y} \PYG{o}{=} \PYG{n}{bins}\PYG{o}{*}\PYG{o}{*}\PYG{p}{(}\PYG{n}{shape}\PYG{o}{\PYGZhy{}}\PYG{l+m+mi}{1}\PYG{p}{)}\PYG{o}{*}\PYG{p}{(}\PYG{n}{np}\PYG{o}{.}\PYG{n}{exp}\PYG{p}{(}\PYG{o}{\PYGZhy{}}\PYG{n}{bins}\PYG{o}{/}\PYG{n}{scale}\PYG{p}{)} \PYG{o}{/}
\PYG{g+gp}{... }                     \PYG{p}{(}\PYG{n}{sps}\PYG{o}{.}\PYG{n}{gamma}\PYG{p}{(}\PYG{n}{shape}\PYG{p}{)}\PYG{o}{*}\PYG{n}{scale}\PYG{o}{*}\PYG{o}{*}\PYG{n}{shape}\PYG{p}{)}\PYG{p}{)}
\PYG{g+gp}{\PYGZgt{}\PYGZgt{}\PYGZgt{} }\PYG{n}{plt}\PYG{o}{.}\PYG{n}{plot}\PYG{p}{(}\PYG{n}{bins}\PYG{p}{,} \PYG{n}{y}\PYG{p}{,} \PYG{n}{linewidth}\PYG{o}{=}\PYG{l+m+mi}{2}\PYG{p}{,} \PYG{n}{color}\PYG{o}{=}\PYG{l+s}{\PYGZsq{}}\PYG{l+s}{r}\PYG{l+s}{\PYGZsq{}}\PYG{p}{)}
\PYG{g+gp}{\PYGZgt{}\PYGZgt{}\PYGZgt{} }\PYG{n}{plt}\PYG{o}{.}\PYG{n}{show}\PYG{p}{(}\PYG{p}{)}
\end{Verbatim}

\end{fulllineitems}

\index{geometric() (in module main)}

\begin{fulllineitems}
\phantomsection\label{main:main.geometric}\pysiglinewithargsret{\code{main.}\bfcode{geometric}}{\emph{p}, \emph{size=None}}{}
Draw samples from the geometric distribution.

Bernoulli trials are experiments with one of two outcomes:
success or failure (an example of such an experiment is flipping
a coin).  The geometric distribution models the number of trials
that must be run in order to achieve success.  It is therefore
supported on the positive integers, \code{k = 1, 2, ...}.

The probability mass function of the geometric distribution is
\begin{gather}
\begin{split}f(k) = (1 - p)^{k - 1} p\end{split}\notag
\end{gather}
where \emph{p} is the probability of success of an individual trial.
\begin{description}
\item[{p}] \leavevmode{[}float{]}
The probability of success of an individual trial.

\item[{size}] \leavevmode{[}tuple of ints{]}
Number of values to draw from the distribution.  The output
is shaped according to \emph{size}.

\end{description}
\begin{description}
\item[{out}] \leavevmode{[}ndarray{]}
Samples from the geometric distribution, shaped according to
\emph{size}.

\end{description}

Draw ten thousand values from the geometric distribution,
with the probability of an individual success equal to 0.35:

\begin{Verbatim}[commandchars=\\\{\}]
\PYG{g+gp}{\PYGZgt{}\PYGZgt{}\PYGZgt{} }\PYG{n}{z} \PYG{o}{=} \PYG{n}{np}\PYG{o}{.}\PYG{n}{random}\PYG{o}{.}\PYG{n}{geometric}\PYG{p}{(}\PYG{n}{p}\PYG{o}{=}\PYG{l+m+mf}{0.35}\PYG{p}{,} \PYG{n}{size}\PYG{o}{=}\PYG{l+m+mi}{10000}\PYG{p}{)}
\end{Verbatim}

How many trials succeeded after a single run?

\begin{Verbatim}[commandchars=\\\{\}]
\PYG{g+gp}{\PYGZgt{}\PYGZgt{}\PYGZgt{} }\PYG{p}{(}\PYG{n}{z} \PYG{o}{==} \PYG{l+m+mi}{1}\PYG{p}{)}\PYG{o}{.}\PYG{n}{sum}\PYG{p}{(}\PYG{p}{)} \PYG{o}{/} \PYG{l+m+mf}{10000.}
\PYG{g+go}{0.34889999999999999 \PYGZsh{}random}
\end{Verbatim}

\end{fulllineitems}

\index{get\_state() (in module main)}

\begin{fulllineitems}
\phantomsection\label{main:main.get_state}\pysiglinewithargsret{\code{main.}\bfcode{get\_state}}{}{}
Return a tuple representing the internal state of the generator.

For more details, see \emph{set\_state}.
\begin{description}
\item[{out}] \leavevmode{[}tuple(str, ndarray of 624 uints, int, int, float){]}
The returned tuple has the following items:
\begin{enumerate}
\item {} 
the string `MT19937'.

\item {} 
a 1-D array of 624 unsigned integer keys.

\item {} 
an integer \code{pos}.

\item {} 
an integer \code{has\_gauss}.

\item {} 
a float \code{cached\_gaussian}.

\end{enumerate}

\end{description}

set\_state

\emph{set\_state} and \emph{get\_state} are not needed to work with any of the
random distributions in NumPy. If the internal state is manually altered,
the user should know exactly what he/she is doing.

\end{fulllineitems}

\index{gumbel() (in module main)}

\begin{fulllineitems}
\phantomsection\label{main:main.gumbel}\pysiglinewithargsret{\code{main.}\bfcode{gumbel}}{\emph{loc=0.0}, \emph{scale=1.0}, \emph{size=None}}{}
Gumbel distribution.

Draw samples from a Gumbel distribution with specified location and scale.
For more information on the Gumbel distribution, see Notes and References
below.
\begin{description}
\item[{loc}] \leavevmode{[}float{]}
The location of the mode of the distribution.

\item[{scale}] \leavevmode{[}float{]}
The scale parameter of the distribution.

\item[{size}] \leavevmode{[}tuple of ints{]}
Output shape.  If the given shape is, e.g., \code{(m, n, k)}, then
\code{m * n * k} samples are drawn.

\end{description}
\begin{description}
\item[{out}] \leavevmode{[}ndarray{]}
The samples

\end{description}

scipy.stats.gumbel\_l
scipy.stats.gumbel\_r
scipy.stats.genextreme
\begin{quote}

probability density function, distribution, or cumulative density
function, etc. for each of the above
\end{quote}

weibull

The Gumbel (or Smallest Extreme Value (SEV) or the Smallest Extreme Value
Type I) distribution is one of a class of Generalized Extreme Value (GEV)
distributions used in modeling extreme value problems.  The Gumbel is a
special case of the Extreme Value Type I distribution for maximums from
distributions with ``exponential-like'' tails.

The probability density for the Gumbel distribution is
\begin{gather}
\begin{split}p(x) = \frac{e^{-(x - \mu)/ \beta}}{\beta} e^{ -e^{-(x - \mu)/
\beta}},\end{split}\notag
\end{gather}
where \(\mu\) is the mode, a location parameter, and \(\beta\) is
the scale parameter.

The Gumbel (named for German mathematician Emil Julius Gumbel) was used
very early in the hydrology literature, for modeling the occurrence of
flood events. It is also used for modeling maximum wind speed and rainfall
rates.  It is a ``fat-tailed'' distribution - the probability of an event in
the tail of the distribution is larger than if one used a Gaussian, hence
the surprisingly frequent occurrence of 100-year floods. Floods were
initially modeled as a Gaussian process, which underestimated the frequency
of extreme events.

It is one of a class of extreme value distributions, the Generalized
Extreme Value (GEV) distributions, which also includes the Weibull and
Frechet.

The function has a mean of \(\mu + 0.57721\beta\) and a variance of
\(\frac{\pi^2}{6}\beta^2\).

Gumbel, E. J., \emph{Statistics of Extremes}, New York: Columbia University
Press, 1958.

Reiss, R.-D. and Thomas, M., \emph{Statistical Analysis of Extreme Values from
Insurance, Finance, Hydrology and Other Fields}, Basel: Birkhauser Verlag,
2001.

Draw samples from the distribution:

\begin{Verbatim}[commandchars=\\\{\}]
\PYG{g+gp}{\PYGZgt{}\PYGZgt{}\PYGZgt{} }\PYG{n}{mu}\PYG{p}{,} \PYG{n}{beta} \PYG{o}{=} \PYG{l+m+mi}{0}\PYG{p}{,} \PYG{l+m+mf}{0.1} \PYG{c}{\PYGZsh{} location and scale}
\PYG{g+gp}{\PYGZgt{}\PYGZgt{}\PYGZgt{} }\PYG{n}{s} \PYG{o}{=} \PYG{n}{np}\PYG{o}{.}\PYG{n}{random}\PYG{o}{.}\PYG{n}{gumbel}\PYG{p}{(}\PYG{n}{mu}\PYG{p}{,} \PYG{n}{beta}\PYG{p}{,} \PYG{l+m+mi}{1000}\PYG{p}{)}
\end{Verbatim}

Display the histogram of the samples, along with
the probability density function:

\begin{Verbatim}[commandchars=\\\{\}]
\PYG{g+gp}{\PYGZgt{}\PYGZgt{}\PYGZgt{} }\PYG{k+kn}{import} \PYG{n+nn}{matplotlib.pyplot} \PYG{k+kn}{as} \PYG{n+nn}{plt}
\PYG{g+gp}{\PYGZgt{}\PYGZgt{}\PYGZgt{} }\PYG{n}{count}\PYG{p}{,} \PYG{n}{bins}\PYG{p}{,} \PYG{n}{ignored} \PYG{o}{=} \PYG{n}{plt}\PYG{o}{.}\PYG{n}{hist}\PYG{p}{(}\PYG{n}{s}\PYG{p}{,} \PYG{l+m+mi}{30}\PYG{p}{,} \PYG{n}{normed}\PYG{o}{=}\PYG{n+nb+bp}{True}\PYG{p}{)}
\PYG{g+gp}{\PYGZgt{}\PYGZgt{}\PYGZgt{} }\PYG{n}{plt}\PYG{o}{.}\PYG{n}{plot}\PYG{p}{(}\PYG{n}{bins}\PYG{p}{,} \PYG{p}{(}\PYG{l+m+mi}{1}\PYG{o}{/}\PYG{n}{beta}\PYG{p}{)}\PYG{o}{*}\PYG{n}{np}\PYG{o}{.}\PYG{n}{exp}\PYG{p}{(}\PYG{o}{\PYGZhy{}}\PYG{p}{(}\PYG{n}{bins} \PYG{o}{\PYGZhy{}} \PYG{n}{mu}\PYG{p}{)}\PYG{o}{/}\PYG{n}{beta}\PYG{p}{)}
\PYG{g+gp}{... }         \PYG{o}{*} \PYG{n}{np}\PYG{o}{.}\PYG{n}{exp}\PYG{p}{(} \PYG{o}{\PYGZhy{}}\PYG{n}{np}\PYG{o}{.}\PYG{n}{exp}\PYG{p}{(} \PYG{o}{\PYGZhy{}}\PYG{p}{(}\PYG{n}{bins} \PYG{o}{\PYGZhy{}} \PYG{n}{mu}\PYG{p}{)} \PYG{o}{/}\PYG{n}{beta}\PYG{p}{)} \PYG{p}{)}\PYG{p}{,}
\PYG{g+gp}{... }         \PYG{n}{linewidth}\PYG{o}{=}\PYG{l+m+mi}{2}\PYG{p}{,} \PYG{n}{color}\PYG{o}{=}\PYG{l+s}{\PYGZsq{}}\PYG{l+s}{r}\PYG{l+s}{\PYGZsq{}}\PYG{p}{)}
\PYG{g+gp}{\PYGZgt{}\PYGZgt{}\PYGZgt{} }\PYG{n}{plt}\PYG{o}{.}\PYG{n}{show}\PYG{p}{(}\PYG{p}{)}
\end{Verbatim}

Show how an extreme value distribution can arise from a Gaussian process
and compare to a Gaussian:

\begin{Verbatim}[commandchars=\\\{\}]
\PYG{g+gp}{\PYGZgt{}\PYGZgt{}\PYGZgt{} }\PYG{n}{means} \PYG{o}{=} \PYG{p}{[}\PYG{p}{]}
\PYG{g+gp}{\PYGZgt{}\PYGZgt{}\PYGZgt{} }\PYG{n}{maxima} \PYG{o}{=} \PYG{p}{[}\PYG{p}{]}
\PYG{g+gp}{\PYGZgt{}\PYGZgt{}\PYGZgt{} }\PYG{k}{for} \PYG{n}{i} \PYG{o+ow}{in} \PYG{n+nb}{range}\PYG{p}{(}\PYG{l+m+mi}{0}\PYG{p}{,}\PYG{l+m+mi}{1000}\PYG{p}{)} \PYG{p}{:}
\PYG{g+gp}{... }   \PYG{n}{a} \PYG{o}{=} \PYG{n}{np}\PYG{o}{.}\PYG{n}{random}\PYG{o}{.}\PYG{n}{normal}\PYG{p}{(}\PYG{n}{mu}\PYG{p}{,} \PYG{n}{beta}\PYG{p}{,} \PYG{l+m+mi}{1000}\PYG{p}{)}
\PYG{g+gp}{... }   \PYG{n}{means}\PYG{o}{.}\PYG{n}{append}\PYG{p}{(}\PYG{n}{a}\PYG{o}{.}\PYG{n}{mean}\PYG{p}{(}\PYG{p}{)}\PYG{p}{)}
\PYG{g+gp}{... }   \PYG{n}{maxima}\PYG{o}{.}\PYG{n}{append}\PYG{p}{(}\PYG{n}{a}\PYG{o}{.}\PYG{n}{max}\PYG{p}{(}\PYG{p}{)}\PYG{p}{)}
\PYG{g+gp}{\PYGZgt{}\PYGZgt{}\PYGZgt{} }\PYG{n}{count}\PYG{p}{,} \PYG{n}{bins}\PYG{p}{,} \PYG{n}{ignored} \PYG{o}{=} \PYG{n}{plt}\PYG{o}{.}\PYG{n}{hist}\PYG{p}{(}\PYG{n}{maxima}\PYG{p}{,} \PYG{l+m+mi}{30}\PYG{p}{,} \PYG{n}{normed}\PYG{o}{=}\PYG{n+nb+bp}{True}\PYG{p}{)}
\PYG{g+gp}{\PYGZgt{}\PYGZgt{}\PYGZgt{} }\PYG{n}{beta} \PYG{o}{=} \PYG{n}{np}\PYG{o}{.}\PYG{n}{std}\PYG{p}{(}\PYG{n}{maxima}\PYG{p}{)}\PYG{o}{*}\PYG{n}{np}\PYG{o}{.}\PYG{n}{pi}\PYG{o}{/}\PYG{n}{np}\PYG{o}{.}\PYG{n}{sqrt}\PYG{p}{(}\PYG{l+m+mi}{6}\PYG{p}{)}
\PYG{g+gp}{\PYGZgt{}\PYGZgt{}\PYGZgt{} }\PYG{n}{mu} \PYG{o}{=} \PYG{n}{np}\PYG{o}{.}\PYG{n}{mean}\PYG{p}{(}\PYG{n}{maxima}\PYG{p}{)} \PYG{o}{\PYGZhy{}} \PYG{l+m+mf}{0.57721}\PYG{o}{*}\PYG{n}{beta}
\PYG{g+gp}{\PYGZgt{}\PYGZgt{}\PYGZgt{} }\PYG{n}{plt}\PYG{o}{.}\PYG{n}{plot}\PYG{p}{(}\PYG{n}{bins}\PYG{p}{,} \PYG{p}{(}\PYG{l+m+mi}{1}\PYG{o}{/}\PYG{n}{beta}\PYG{p}{)}\PYG{o}{*}\PYG{n}{np}\PYG{o}{.}\PYG{n}{exp}\PYG{p}{(}\PYG{o}{\PYGZhy{}}\PYG{p}{(}\PYG{n}{bins} \PYG{o}{\PYGZhy{}} \PYG{n}{mu}\PYG{p}{)}\PYG{o}{/}\PYG{n}{beta}\PYG{p}{)}
\PYG{g+gp}{... }         \PYG{o}{*} \PYG{n}{np}\PYG{o}{.}\PYG{n}{exp}\PYG{p}{(}\PYG{o}{\PYGZhy{}}\PYG{n}{np}\PYG{o}{.}\PYG{n}{exp}\PYG{p}{(}\PYG{o}{\PYGZhy{}}\PYG{p}{(}\PYG{n}{bins} \PYG{o}{\PYGZhy{}} \PYG{n}{mu}\PYG{p}{)}\PYG{o}{/}\PYG{n}{beta}\PYG{p}{)}\PYG{p}{)}\PYG{p}{,}
\PYG{g+gp}{... }         \PYG{n}{linewidth}\PYG{o}{=}\PYG{l+m+mi}{2}\PYG{p}{,} \PYG{n}{color}\PYG{o}{=}\PYG{l+s}{\PYGZsq{}}\PYG{l+s}{r}\PYG{l+s}{\PYGZsq{}}\PYG{p}{)}
\PYG{g+gp}{\PYGZgt{}\PYGZgt{}\PYGZgt{} }\PYG{n}{plt}\PYG{o}{.}\PYG{n}{plot}\PYG{p}{(}\PYG{n}{bins}\PYG{p}{,} \PYG{l+m+mi}{1}\PYG{o}{/}\PYG{p}{(}\PYG{n}{beta} \PYG{o}{*} \PYG{n}{np}\PYG{o}{.}\PYG{n}{sqrt}\PYG{p}{(}\PYG{l+m+mi}{2} \PYG{o}{*} \PYG{n}{np}\PYG{o}{.}\PYG{n}{pi}\PYG{p}{)}\PYG{p}{)}
\PYG{g+gp}{... }         \PYG{o}{*} \PYG{n}{np}\PYG{o}{.}\PYG{n}{exp}\PYG{p}{(}\PYG{o}{\PYGZhy{}}\PYG{p}{(}\PYG{n}{bins} \PYG{o}{\PYGZhy{}} \PYG{n}{mu}\PYG{p}{)}\PYG{o}{*}\PYG{o}{*}\PYG{l+m+mi}{2} \PYG{o}{/} \PYG{p}{(}\PYG{l+m+mi}{2} \PYG{o}{*} \PYG{n}{beta}\PYG{o}{*}\PYG{o}{*}\PYG{l+m+mi}{2}\PYG{p}{)}\PYG{p}{)}\PYG{p}{,}
\PYG{g+gp}{... }         \PYG{n}{linewidth}\PYG{o}{=}\PYG{l+m+mi}{2}\PYG{p}{,} \PYG{n}{color}\PYG{o}{=}\PYG{l+s}{\PYGZsq{}}\PYG{l+s}{g}\PYG{l+s}{\PYGZsq{}}\PYG{p}{)}
\PYG{g+gp}{\PYGZgt{}\PYGZgt{}\PYGZgt{} }\PYG{n}{plt}\PYG{o}{.}\PYG{n}{show}\PYG{p}{(}\PYG{p}{)}
\end{Verbatim}

\end{fulllineitems}

\index{hypergeometric() (in module main)}

\begin{fulllineitems}
\phantomsection\label{main:main.hypergeometric}\pysiglinewithargsret{\code{main.}\bfcode{hypergeometric}}{\emph{ngood}, \emph{nbad}, \emph{nsample}, \emph{size=None}}{}
Draw samples from a Hypergeometric distribution.

Samples are drawn from a Hypergeometric distribution with specified
parameters, ngood (ways to make a good selection), nbad (ways to make
a bad selection), and nsample = number of items sampled, which is less
than or equal to the sum ngood + nbad.
\begin{description}
\item[{ngood}] \leavevmode{[}int or array\_like{]}
Number of ways to make a good selection.  Must be nonnegative.

\item[{nbad}] \leavevmode{[}int or array\_like{]}
Number of ways to make a bad selection.  Must be nonnegative.

\item[{nsample}] \leavevmode{[}int or array\_like{]}
Number of items sampled.  Must be at least 1 and at most
\code{ngood + nbad}.

\item[{size}] \leavevmode{[}int or tuple of int{]}
Output shape.  If the given shape is, e.g., \code{(m, n, k)}, then
\code{m * n * k} samples are drawn.

\end{description}
\begin{description}
\item[{samples}] \leavevmode{[}ndarray or scalar{]}
The values are all integers in  {[}0, n{]}.

\end{description}
\begin{description}
\item[{scipy.stats.distributions.hypergeom}] \leavevmode{[}probability density function,{]}
distribution or cumulative density function, etc.

\end{description}

The probability density for the Hypergeometric distribution is
\begin{gather}
\begin{split}P(x) = \frac{\binom{m}{n}\binom{N-m}{n-x}}{\binom{N}{n}},\end{split}\notag
\end{gather}
where \(0 \le x \le m\) and \(n+m-N \le x \le n\)

for P(x) the probability of x successes, n = ngood, m = nbad, and
N = number of samples.

Consider an urn with black and white marbles in it, ngood of them
black and nbad are white. If you draw nsample balls without
replacement, then the Hypergeometric distribution describes the
distribution of black balls in the drawn sample.

Note that this distribution is very similar to the Binomial
distribution, except that in this case, samples are drawn without
replacement, whereas in the Binomial case samples are drawn with
replacement (or the sample space is infinite). As the sample space
becomes large, this distribution approaches the Binomial.

Draw samples from the distribution:

\begin{Verbatim}[commandchars=\\\{\}]
\PYG{g+gp}{\PYGZgt{}\PYGZgt{}\PYGZgt{} }\PYG{n}{ngood}\PYG{p}{,} \PYG{n}{nbad}\PYG{p}{,} \PYG{n}{nsamp} \PYG{o}{=} \PYG{l+m+mi}{100}\PYG{p}{,} \PYG{l+m+mi}{2}\PYG{p}{,} \PYG{l+m+mi}{10}
\PYG{g+go}{\PYGZsh{} number of good, number of bad, and number of samples}
\PYG{g+gp}{\PYGZgt{}\PYGZgt{}\PYGZgt{} }\PYG{n}{s} \PYG{o}{=} \PYG{n}{np}\PYG{o}{.}\PYG{n}{random}\PYG{o}{.}\PYG{n}{hypergeometric}\PYG{p}{(}\PYG{n}{ngood}\PYG{p}{,} \PYG{n}{nbad}\PYG{p}{,} \PYG{n}{nsamp}\PYG{p}{,} \PYG{l+m+mi}{1000}\PYG{p}{)}
\PYG{g+gp}{\PYGZgt{}\PYGZgt{}\PYGZgt{} }\PYG{n}{hist}\PYG{p}{(}\PYG{n}{s}\PYG{p}{)}
\PYG{g+go}{\PYGZsh{}   note that it is very unlikely to grab both bad items}
\end{Verbatim}

Suppose you have an urn with 15 white and 15 black marbles.
If you pull 15 marbles at random, how likely is it that
12 or more of them are one color?

\begin{Verbatim}[commandchars=\\\{\}]
\PYG{g+gp}{\PYGZgt{}\PYGZgt{}\PYGZgt{} }\PYG{n}{s} \PYG{o}{=} \PYG{n}{np}\PYG{o}{.}\PYG{n}{random}\PYG{o}{.}\PYG{n}{hypergeometric}\PYG{p}{(}\PYG{l+m+mi}{15}\PYG{p}{,} \PYG{l+m+mi}{15}\PYG{p}{,} \PYG{l+m+mi}{15}\PYG{p}{,} \PYG{l+m+mi}{100000}\PYG{p}{)}
\PYG{g+gp}{\PYGZgt{}\PYGZgt{}\PYGZgt{} }\PYG{n+nb}{sum}\PYG{p}{(}\PYG{n}{s}\PYG{o}{\PYGZgt{}}\PYG{o}{=}\PYG{l+m+mi}{12}\PYG{p}{)}\PYG{o}{/}\PYG{l+m+mf}{100000.} \PYG{o}{+} \PYG{n+nb}{sum}\PYG{p}{(}\PYG{n}{s}\PYG{o}{\PYGZlt{}}\PYG{o}{=}\PYG{l+m+mi}{3}\PYG{p}{)}\PYG{o}{/}\PYG{l+m+mf}{100000.}
\PYG{g+go}{\PYGZsh{}   answer = 0.003 ... pretty unlikely!}
\end{Verbatim}

\end{fulllineitems}

\index{laplace() (in module main)}

\begin{fulllineitems}
\phantomsection\label{main:main.laplace}\pysiglinewithargsret{\code{main.}\bfcode{laplace}}{\emph{loc=0.0}, \emph{scale=1.0}, \emph{size=None}}{}
Draw samples from the Laplace or double exponential distribution with
specified location (or mean) and scale (decay).

The Laplace distribution is similar to the Gaussian/normal distribution,
but is sharper at the peak and has fatter tails. It represents the
difference between two independent, identically distributed exponential
random variables.
\begin{description}
\item[{loc}] \leavevmode{[}float{]}
The position, \(\mu\), of the distribution peak.

\item[{scale}] \leavevmode{[}float{]}
\(\lambda\), the exponential decay.

\end{description}

It has the probability density function
\begin{gather}
\begin{split}f(x; \mu, \lambda) = \frac{1}{2\lambda}
\exp\left(-\frac{|x - \mu|}{\lambda}\right).\end{split}\notag
\end{gather}
The first law of Laplace, from 1774, states that the frequency of an error
can be expressed as an exponential function of the absolute magnitude of
the error, which leads to the Laplace distribution. For many problems in
Economics and Health sciences, this distribution seems to model the data
better than the standard Gaussian distribution

Draw samples from the distribution

\begin{Verbatim}[commandchars=\\\{\}]
\PYG{g+gp}{\PYGZgt{}\PYGZgt{}\PYGZgt{} }\PYG{n}{loc}\PYG{p}{,} \PYG{n}{scale} \PYG{o}{=} \PYG{l+m+mf}{0.}\PYG{p}{,} \PYG{l+m+mf}{1.}
\PYG{g+gp}{\PYGZgt{}\PYGZgt{}\PYGZgt{} }\PYG{n}{s} \PYG{o}{=} \PYG{n}{np}\PYG{o}{.}\PYG{n}{random}\PYG{o}{.}\PYG{n}{laplace}\PYG{p}{(}\PYG{n}{loc}\PYG{p}{,} \PYG{n}{scale}\PYG{p}{,} \PYG{l+m+mi}{1000}\PYG{p}{)}
\end{Verbatim}

Display the histogram of the samples, along with
the probability density function:

\begin{Verbatim}[commandchars=\\\{\}]
\PYG{g+gp}{\PYGZgt{}\PYGZgt{}\PYGZgt{} }\PYG{k+kn}{import} \PYG{n+nn}{matplotlib.pyplot} \PYG{k+kn}{as} \PYG{n+nn}{plt}
\PYG{g+gp}{\PYGZgt{}\PYGZgt{}\PYGZgt{} }\PYG{n}{count}\PYG{p}{,} \PYG{n}{bins}\PYG{p}{,} \PYG{n}{ignored} \PYG{o}{=} \PYG{n}{plt}\PYG{o}{.}\PYG{n}{hist}\PYG{p}{(}\PYG{n}{s}\PYG{p}{,} \PYG{l+m+mi}{30}\PYG{p}{,} \PYG{n}{normed}\PYG{o}{=}\PYG{n+nb+bp}{True}\PYG{p}{)}
\PYG{g+gp}{\PYGZgt{}\PYGZgt{}\PYGZgt{} }\PYG{n}{x} \PYG{o}{=} \PYG{n}{np}\PYG{o}{.}\PYG{n}{arange}\PYG{p}{(}\PYG{o}{\PYGZhy{}}\PYG{l+m+mf}{8.}\PYG{p}{,} \PYG{l+m+mf}{8.}\PYG{p}{,} \PYG{o}{.}\PYG{l+m+mo}{01}\PYG{p}{)}
\PYG{g+gp}{\PYGZgt{}\PYGZgt{}\PYGZgt{} }\PYG{n}{pdf} \PYG{o}{=} \PYG{n}{np}\PYG{o}{.}\PYG{n}{exp}\PYG{p}{(}\PYG{o}{\PYGZhy{}}\PYG{n+nb}{abs}\PYG{p}{(}\PYG{n}{x}\PYG{o}{\PYGZhy{}}\PYG{n}{loc}\PYG{o}{/}\PYG{n}{scale}\PYG{p}{)}\PYG{p}{)}\PYG{o}{/}\PYG{p}{(}\PYG{l+m+mf}{2.}\PYG{o}{*}\PYG{n}{scale}\PYG{p}{)}
\PYG{g+gp}{\PYGZgt{}\PYGZgt{}\PYGZgt{} }\PYG{n}{plt}\PYG{o}{.}\PYG{n}{plot}\PYG{p}{(}\PYG{n}{x}\PYG{p}{,} \PYG{n}{pdf}\PYG{p}{)}
\end{Verbatim}

Plot Gaussian for comparison:

\begin{Verbatim}[commandchars=\\\{\}]
\PYG{g+gp}{\PYGZgt{}\PYGZgt{}\PYGZgt{} }\PYG{n}{g} \PYG{o}{=} \PYG{p}{(}\PYG{l+m+mi}{1}\PYG{o}{/}\PYG{p}{(}\PYG{n}{scale} \PYG{o}{*} \PYG{n}{np}\PYG{o}{.}\PYG{n}{sqrt}\PYG{p}{(}\PYG{l+m+mi}{2} \PYG{o}{*} \PYG{n}{np}\PYG{o}{.}\PYG{n}{pi}\PYG{p}{)}\PYG{p}{)} \PYG{o}{*} 
\PYG{g+gp}{... }     \PYG{n}{np}\PYG{o}{.}\PYG{n}{exp}\PYG{p}{(} \PYG{o}{\PYGZhy{}} \PYG{p}{(}\PYG{n}{x} \PYG{o}{\PYGZhy{}} \PYG{n}{loc}\PYG{p}{)}\PYG{o}{*}\PYG{o}{*}\PYG{l+m+mi}{2} \PYG{o}{/} \PYG{p}{(}\PYG{l+m+mi}{2} \PYG{o}{*} \PYG{n}{scale}\PYG{o}{*}\PYG{o}{*}\PYG{l+m+mi}{2}\PYG{p}{)} \PYG{p}{)}\PYG{p}{)}
\PYG{g+gp}{\PYGZgt{}\PYGZgt{}\PYGZgt{} }\PYG{n}{plt}\PYG{o}{.}\PYG{n}{plot}\PYG{p}{(}\PYG{n}{x}\PYG{p}{,}\PYG{n}{g}\PYG{p}{)}
\end{Verbatim}

\end{fulllineitems}

\index{logistic() (in module main)}

\begin{fulllineitems}
\phantomsection\label{main:main.logistic}\pysiglinewithargsret{\code{main.}\bfcode{logistic}}{\emph{loc=0.0}, \emph{scale=1.0}, \emph{size=None}}{}
Draw samples from a Logistic distribution.

Samples are drawn from a Logistic distribution with specified
parameters, loc (location or mean, also median), and scale (\textgreater{}0).

loc : float

scale : float \textgreater{} 0.
\begin{description}
\item[{size}] \leavevmode{[}\{tuple, int\}{]}
Output shape.  If the given shape is, e.g., \code{(m, n, k)}, then
\code{m * n * k} samples are drawn.

\end{description}
\begin{description}
\item[{samples}] \leavevmode{[}\{ndarray, scalar\}{]}
where the values are all integers in  {[}0, n{]}.

\end{description}
\begin{description}
\item[{scipy.stats.distributions.logistic}] \leavevmode{[}probability density function,{]}
distribution or cumulative density function, etc.

\end{description}

The probability density for the Logistic distribution is
\begin{gather}
\begin{split}P(x) = P(x) = \frac{e^{-(x-\mu)/s}}{s(1+e^{-(x-\mu)/s})^2},\end{split}\notag
\end{gather}
where \(\mu\) = location and \(s\) = scale.

The Logistic distribution is used in Extreme Value problems where it
can act as a mixture of Gumbel distributions, in Epidemiology, and by
the World Chess Federation (FIDE) where it is used in the Elo ranking
system, assuming the performance of each player is a logistically
distributed random variable.

Draw samples from the distribution:

\begin{Verbatim}[commandchars=\\\{\}]
\PYG{g+gp}{\PYGZgt{}\PYGZgt{}\PYGZgt{} }\PYG{n}{loc}\PYG{p}{,} \PYG{n}{scale} \PYG{o}{=} \PYG{l+m+mi}{10}\PYG{p}{,} \PYG{l+m+mi}{1}
\PYG{g+gp}{\PYGZgt{}\PYGZgt{}\PYGZgt{} }\PYG{n}{s} \PYG{o}{=} \PYG{n}{np}\PYG{o}{.}\PYG{n}{random}\PYG{o}{.}\PYG{n}{logistic}\PYG{p}{(}\PYG{n}{loc}\PYG{p}{,} \PYG{n}{scale}\PYG{p}{,} \PYG{l+m+mi}{10000}\PYG{p}{)}
\PYG{g+gp}{\PYGZgt{}\PYGZgt{}\PYGZgt{} }\PYG{n}{count}\PYG{p}{,} \PYG{n}{bins}\PYG{p}{,} \PYG{n}{ignored} \PYG{o}{=} \PYG{n}{plt}\PYG{o}{.}\PYG{n}{hist}\PYG{p}{(}\PYG{n}{s}\PYG{p}{,} \PYG{n}{bins}\PYG{o}{=}\PYG{l+m+mi}{50}\PYG{p}{)}
\end{Verbatim}

\#   plot against distribution

\begin{Verbatim}[commandchars=\\\{\}]
\PYG{g+gp}{\PYGZgt{}\PYGZgt{}\PYGZgt{} }\PYG{k}{def} \PYG{n+nf}{logist}\PYG{p}{(}\PYG{n}{x}\PYG{p}{,} \PYG{n}{loc}\PYG{p}{,} \PYG{n}{scale}\PYG{p}{)}\PYG{p}{:}
\PYG{g+gp}{... }    \PYG{k}{return} \PYG{n}{exp}\PYG{p}{(}\PYG{p}{(}\PYG{n}{loc}\PYG{o}{\PYGZhy{}}\PYG{n}{x}\PYG{p}{)}\PYG{o}{/}\PYG{n}{scale}\PYG{p}{)}\PYG{o}{/}\PYG{p}{(}\PYG{n}{scale}\PYG{o}{*}\PYG{p}{(}\PYG{l+m+mi}{1}\PYG{o}{+}\PYG{n}{exp}\PYG{p}{(}\PYG{p}{(}\PYG{n}{loc}\PYG{o}{\PYGZhy{}}\PYG{n}{x}\PYG{p}{)}\PYG{o}{/}\PYG{n}{scale}\PYG{p}{)}\PYG{p}{)}\PYG{o}{*}\PYG{o}{*}\PYG{l+m+mi}{2}\PYG{p}{)}
\PYG{g+gp}{\PYGZgt{}\PYGZgt{}\PYGZgt{} }\PYG{n}{plt}\PYG{o}{.}\PYG{n}{plot}\PYG{p}{(}\PYG{n}{bins}\PYG{p}{,} \PYG{n}{logist}\PYG{p}{(}\PYG{n}{bins}\PYG{p}{,} \PYG{n}{loc}\PYG{p}{,} \PYG{n}{scale}\PYG{p}{)}\PYG{o}{*}\PYG{n}{count}\PYG{o}{.}\PYG{n}{max}\PYG{p}{(}\PYG{p}{)}\PYG{o}{/}\PYGZbs{}
\PYG{g+gp}{... }\PYG{n}{logist}\PYG{p}{(}\PYG{n}{bins}\PYG{p}{,} \PYG{n}{loc}\PYG{p}{,} \PYG{n}{scale}\PYG{p}{)}\PYG{o}{.}\PYG{n}{max}\PYG{p}{(}\PYG{p}{)}\PYG{p}{)}
\PYG{g+gp}{\PYGZgt{}\PYGZgt{}\PYGZgt{} }\PYG{n}{plt}\PYG{o}{.}\PYG{n}{show}\PYG{p}{(}\PYG{p}{)}
\end{Verbatim}

\end{fulllineitems}

\index{lognormal() (in module main)}

\begin{fulllineitems}
\phantomsection\label{main:main.lognormal}\pysiglinewithargsret{\code{main.}\bfcode{lognormal}}{\emph{mean=0.0}, \emph{sigma=1.0}, \emph{size=None}}{}
Return samples drawn from a log-normal distribution.

Draw samples from a log-normal distribution with specified mean,
standard deviation, and array shape.  Note that the mean and standard
deviation are not the values for the distribution itself, but of the
underlying normal distribution it is derived from.
\begin{description}
\item[{mean}] \leavevmode{[}float{]}
Mean value of the underlying normal distribution

\item[{sigma}] \leavevmode{[}float, \textgreater{} 0.{]}
Standard deviation of the underlying normal distribution

\item[{size}] \leavevmode{[}tuple of ints{]}
Output shape.  If the given shape is, e.g., \code{(m, n, k)}, then
\code{m * n * k} samples are drawn.

\end{description}
\begin{description}
\item[{samples}] \leavevmode{[}ndarray or float{]}
The desired samples. An array of the same shape as \emph{size} if given,
if \emph{size} is None a float is returned.

\end{description}
\begin{description}
\item[{scipy.stats.lognorm}] \leavevmode{[}probability density function, distribution,{]}
cumulative density function, etc.

\end{description}

A variable \emph{x} has a log-normal distribution if \emph{log(x)} is normally
distributed.  The probability density function for the log-normal
distribution is:
\begin{gather}
\begin{split}p(x) = \frac{1}{\sigma x \sqrt{2\pi}}
e^{(-\frac{(ln(x)-\mu)^2}{2\sigma^2})}\end{split}\notag
\end{gather}
where \(\mu\) is the mean and \(\sigma\) is the standard
deviation of the normally distributed logarithm of the variable.
A log-normal distribution results if a random variable is the \emph{product}
of a large number of independent, identically-distributed variables in
the same way that a normal distribution results if the variable is the
\emph{sum} of a large number of independent, identically-distributed
variables.

Limpert, E., Stahel, W. A., and Abbt, M., ``Log-normal Distributions
across the Sciences: Keys and Clues,'' \emph{BioScience}, Vol. 51, No. 5,
May, 2001.  \href{http://stat.ethz.ch/~stahel/lognormal/bioscience.pdf}{http://stat.ethz.ch/\textasciitilde{}stahel/lognormal/bioscience.pdf}

Reiss, R.D. and Thomas, M., \emph{Statistical Analysis of Extreme Values},
Basel: Birkhauser Verlag, 2001, pp. 31-32.

Draw samples from the distribution:

\begin{Verbatim}[commandchars=\\\{\}]
\PYG{g+gp}{\PYGZgt{}\PYGZgt{}\PYGZgt{} }\PYG{n}{mu}\PYG{p}{,} \PYG{n}{sigma} \PYG{o}{=} \PYG{l+m+mf}{3.}\PYG{p}{,} \PYG{l+m+mf}{1.} \PYG{c}{\PYGZsh{} mean and standard deviation}
\PYG{g+gp}{\PYGZgt{}\PYGZgt{}\PYGZgt{} }\PYG{n}{s} \PYG{o}{=} \PYG{n}{np}\PYG{o}{.}\PYG{n}{random}\PYG{o}{.}\PYG{n}{lognormal}\PYG{p}{(}\PYG{n}{mu}\PYG{p}{,} \PYG{n}{sigma}\PYG{p}{,} \PYG{l+m+mi}{1000}\PYG{p}{)}
\end{Verbatim}

Display the histogram of the samples, along with
the probability density function:

\begin{Verbatim}[commandchars=\\\{\}]
\PYG{g+gp}{\PYGZgt{}\PYGZgt{}\PYGZgt{} }\PYG{k+kn}{import} \PYG{n+nn}{matplotlib.pyplot} \PYG{k+kn}{as} \PYG{n+nn}{plt}
\PYG{g+gp}{\PYGZgt{}\PYGZgt{}\PYGZgt{} }\PYG{n}{count}\PYG{p}{,} \PYG{n}{bins}\PYG{p}{,} \PYG{n}{ignored} \PYG{o}{=} \PYG{n}{plt}\PYG{o}{.}\PYG{n}{hist}\PYG{p}{(}\PYG{n}{s}\PYG{p}{,} \PYG{l+m+mi}{100}\PYG{p}{,} \PYG{n}{normed}\PYG{o}{=}\PYG{n+nb+bp}{True}\PYG{p}{,} \PYG{n}{align}\PYG{o}{=}\PYG{l+s}{\PYGZsq{}}\PYG{l+s}{mid}\PYG{l+s}{\PYGZsq{}}\PYG{p}{)}
\end{Verbatim}

\begin{Verbatim}[commandchars=\\\{\}]
\PYG{g+gp}{\PYGZgt{}\PYGZgt{}\PYGZgt{} }\PYG{n}{x} \PYG{o}{=} \PYG{n}{np}\PYG{o}{.}\PYG{n}{linspace}\PYG{p}{(}\PYG{n+nb}{min}\PYG{p}{(}\PYG{n}{bins}\PYG{p}{)}\PYG{p}{,} \PYG{n+nb}{max}\PYG{p}{(}\PYG{n}{bins}\PYG{p}{)}\PYG{p}{,} \PYG{l+m+mi}{10000}\PYG{p}{)}
\PYG{g+gp}{\PYGZgt{}\PYGZgt{}\PYGZgt{} }\PYG{n}{pdf} \PYG{o}{=} \PYG{p}{(}\PYG{n}{np}\PYG{o}{.}\PYG{n}{exp}\PYG{p}{(}\PYG{o}{\PYGZhy{}}\PYG{p}{(}\PYG{n}{np}\PYG{o}{.}\PYG{n}{log}\PYG{p}{(}\PYG{n}{x}\PYG{p}{)} \PYG{o}{\PYGZhy{}} \PYG{n}{mu}\PYG{p}{)}\PYG{o}{*}\PYG{o}{*}\PYG{l+m+mi}{2} \PYG{o}{/} \PYG{p}{(}\PYG{l+m+mi}{2} \PYG{o}{*} \PYG{n}{sigma}\PYG{o}{*}\PYG{o}{*}\PYG{l+m+mi}{2}\PYG{p}{)}\PYG{p}{)}
\PYG{g+gp}{... }       \PYG{o}{/} \PYG{p}{(}\PYG{n}{x} \PYG{o}{*} \PYG{n}{sigma} \PYG{o}{*} \PYG{n}{np}\PYG{o}{.}\PYG{n}{sqrt}\PYG{p}{(}\PYG{l+m+mi}{2} \PYG{o}{*} \PYG{n}{np}\PYG{o}{.}\PYG{n}{pi}\PYG{p}{)}\PYG{p}{)}\PYG{p}{)}
\end{Verbatim}

\begin{Verbatim}[commandchars=\\\{\}]
\PYG{g+gp}{\PYGZgt{}\PYGZgt{}\PYGZgt{} }\PYG{n}{plt}\PYG{o}{.}\PYG{n}{plot}\PYG{p}{(}\PYG{n}{x}\PYG{p}{,} \PYG{n}{pdf}\PYG{p}{,} \PYG{n}{linewidth}\PYG{o}{=}\PYG{l+m+mi}{2}\PYG{p}{,} \PYG{n}{color}\PYG{o}{=}\PYG{l+s}{\PYGZsq{}}\PYG{l+s}{r}\PYG{l+s}{\PYGZsq{}}\PYG{p}{)}
\PYG{g+gp}{\PYGZgt{}\PYGZgt{}\PYGZgt{} }\PYG{n}{plt}\PYG{o}{.}\PYG{n}{axis}\PYG{p}{(}\PYG{l+s}{\PYGZsq{}}\PYG{l+s}{tight}\PYG{l+s}{\PYGZsq{}}\PYG{p}{)}
\PYG{g+gp}{\PYGZgt{}\PYGZgt{}\PYGZgt{} }\PYG{n}{plt}\PYG{o}{.}\PYG{n}{show}\PYG{p}{(}\PYG{p}{)}
\end{Verbatim}

Demonstrate that taking the products of random samples from a uniform
distribution can be fit well by a log-normal probability density function.

\begin{Verbatim}[commandchars=\\\{\}]
\PYG{g+gp}{\PYGZgt{}\PYGZgt{}\PYGZgt{} }\PYG{c}{\PYGZsh{} Generate a thousand samples: each is the product of 100 random}
\PYG{g+gp}{\PYGZgt{}\PYGZgt{}\PYGZgt{} }\PYG{c}{\PYGZsh{} values, drawn from a normal distribution.}
\PYG{g+gp}{\PYGZgt{}\PYGZgt{}\PYGZgt{} }\PYG{n}{b} \PYG{o}{=} \PYG{p}{[}\PYG{p}{]}
\PYG{g+gp}{\PYGZgt{}\PYGZgt{}\PYGZgt{} }\PYG{k}{for} \PYG{n}{i} \PYG{o+ow}{in} \PYG{n+nb}{range}\PYG{p}{(}\PYG{l+m+mi}{1000}\PYG{p}{)}\PYG{p}{:}
\PYG{g+gp}{... }   \PYG{n}{a} \PYG{o}{=} \PYG{l+m+mf}{10.} \PYG{o}{+} \PYG{n}{np}\PYG{o}{.}\PYG{n}{random}\PYG{o}{.}\PYG{n}{random}\PYG{p}{(}\PYG{l+m+mi}{100}\PYG{p}{)}
\PYG{g+gp}{... }   \PYG{n}{b}\PYG{o}{.}\PYG{n}{append}\PYG{p}{(}\PYG{n}{np}\PYG{o}{.}\PYG{n}{product}\PYG{p}{(}\PYG{n}{a}\PYG{p}{)}\PYG{p}{)}
\end{Verbatim}

\begin{Verbatim}[commandchars=\\\{\}]
\PYG{g+gp}{\PYGZgt{}\PYGZgt{}\PYGZgt{} }\PYG{n}{b} \PYG{o}{=} \PYG{n}{np}\PYG{o}{.}\PYG{n}{array}\PYG{p}{(}\PYG{n}{b}\PYG{p}{)} \PYG{o}{/} \PYG{n}{np}\PYG{o}{.}\PYG{n}{min}\PYG{p}{(}\PYG{n}{b}\PYG{p}{)} \PYG{c}{\PYGZsh{} scale values to be positive}
\PYG{g+gp}{\PYGZgt{}\PYGZgt{}\PYGZgt{} }\PYG{n}{count}\PYG{p}{,} \PYG{n}{bins}\PYG{p}{,} \PYG{n}{ignored} \PYG{o}{=} \PYG{n}{plt}\PYG{o}{.}\PYG{n}{hist}\PYG{p}{(}\PYG{n}{b}\PYG{p}{,} \PYG{l+m+mi}{100}\PYG{p}{,} \PYG{n}{normed}\PYG{o}{=}\PYG{n+nb+bp}{True}\PYG{p}{,} \PYG{n}{align}\PYG{o}{=}\PYG{l+s}{\PYGZsq{}}\PYG{l+s}{center}\PYG{l+s}{\PYGZsq{}}\PYG{p}{)}
\PYG{g+gp}{\PYGZgt{}\PYGZgt{}\PYGZgt{} }\PYG{n}{sigma} \PYG{o}{=} \PYG{n}{np}\PYG{o}{.}\PYG{n}{std}\PYG{p}{(}\PYG{n}{np}\PYG{o}{.}\PYG{n}{log}\PYG{p}{(}\PYG{n}{b}\PYG{p}{)}\PYG{p}{)}
\PYG{g+gp}{\PYGZgt{}\PYGZgt{}\PYGZgt{} }\PYG{n}{mu} \PYG{o}{=} \PYG{n}{np}\PYG{o}{.}\PYG{n}{mean}\PYG{p}{(}\PYG{n}{np}\PYG{o}{.}\PYG{n}{log}\PYG{p}{(}\PYG{n}{b}\PYG{p}{)}\PYG{p}{)}
\end{Verbatim}

\begin{Verbatim}[commandchars=\\\{\}]
\PYG{g+gp}{\PYGZgt{}\PYGZgt{}\PYGZgt{} }\PYG{n}{x} \PYG{o}{=} \PYG{n}{np}\PYG{o}{.}\PYG{n}{linspace}\PYG{p}{(}\PYG{n+nb}{min}\PYG{p}{(}\PYG{n}{bins}\PYG{p}{)}\PYG{p}{,} \PYG{n+nb}{max}\PYG{p}{(}\PYG{n}{bins}\PYG{p}{)}\PYG{p}{,} \PYG{l+m+mi}{10000}\PYG{p}{)}
\PYG{g+gp}{\PYGZgt{}\PYGZgt{}\PYGZgt{} }\PYG{n}{pdf} \PYG{o}{=} \PYG{p}{(}\PYG{n}{np}\PYG{o}{.}\PYG{n}{exp}\PYG{p}{(}\PYG{o}{\PYGZhy{}}\PYG{p}{(}\PYG{n}{np}\PYG{o}{.}\PYG{n}{log}\PYG{p}{(}\PYG{n}{x}\PYG{p}{)} \PYG{o}{\PYGZhy{}} \PYG{n}{mu}\PYG{p}{)}\PYG{o}{*}\PYG{o}{*}\PYG{l+m+mi}{2} \PYG{o}{/} \PYG{p}{(}\PYG{l+m+mi}{2} \PYG{o}{*} \PYG{n}{sigma}\PYG{o}{*}\PYG{o}{*}\PYG{l+m+mi}{2}\PYG{p}{)}\PYG{p}{)}
\PYG{g+gp}{... }       \PYG{o}{/} \PYG{p}{(}\PYG{n}{x} \PYG{o}{*} \PYG{n}{sigma} \PYG{o}{*} \PYG{n}{np}\PYG{o}{.}\PYG{n}{sqrt}\PYG{p}{(}\PYG{l+m+mi}{2} \PYG{o}{*} \PYG{n}{np}\PYG{o}{.}\PYG{n}{pi}\PYG{p}{)}\PYG{p}{)}\PYG{p}{)}
\end{Verbatim}

\begin{Verbatim}[commandchars=\\\{\}]
\PYG{g+gp}{\PYGZgt{}\PYGZgt{}\PYGZgt{} }\PYG{n}{plt}\PYG{o}{.}\PYG{n}{plot}\PYG{p}{(}\PYG{n}{x}\PYG{p}{,} \PYG{n}{pdf}\PYG{p}{,} \PYG{n}{color}\PYG{o}{=}\PYG{l+s}{\PYGZsq{}}\PYG{l+s}{r}\PYG{l+s}{\PYGZsq{}}\PYG{p}{,} \PYG{n}{linewidth}\PYG{o}{=}\PYG{l+m+mi}{2}\PYG{p}{)}
\PYG{g+gp}{\PYGZgt{}\PYGZgt{}\PYGZgt{} }\PYG{n}{plt}\PYG{o}{.}\PYG{n}{show}\PYG{p}{(}\PYG{p}{)}
\end{Verbatim}

\end{fulllineitems}

\index{logseries() (in module main)}

\begin{fulllineitems}
\phantomsection\label{main:main.logseries}\pysiglinewithargsret{\code{main.}\bfcode{logseries}}{\emph{p}, \emph{size=None}}{}
Draw samples from a Logarithmic Series distribution.

Samples are drawn from a Log Series distribution with specified
parameter, p (probability, 0 \textless{} p \textless{} 1).

loc : float

scale : float \textgreater{} 0.
\begin{description}
\item[{size}] \leavevmode{[}\{tuple, int\}{]}
Output shape.  If the given shape is, e.g., \code{(m, n, k)}, then
\code{m * n * k} samples are drawn.

\end{description}
\begin{description}
\item[{samples}] \leavevmode{[}\{ndarray, scalar\}{]}
where the values are all integers in  {[}0, n{]}.

\end{description}
\begin{description}
\item[{scipy.stats.distributions.logser}] \leavevmode{[}probability density function,{]}
distribution or cumulative density function, etc.

\end{description}

The probability density for the Log Series distribution is
\begin{gather}
\begin{split}P(k) = \frac{-p^k}{k \ln(1-p)},\end{split}\notag
\end{gather}
where p = probability.

The Log Series distribution is frequently used to represent species
richness and occurrence, first proposed by Fisher, Corbet, and
Williams in 1943 {[}2{]}.  It may also be used to model the numbers of
occupants seen in cars {[}3{]}.

Draw samples from the distribution:

\begin{Verbatim}[commandchars=\\\{\}]
\PYG{g+gp}{\PYGZgt{}\PYGZgt{}\PYGZgt{} }\PYG{n}{a} \PYG{o}{=} \PYG{o}{.}\PYG{l+m+mi}{6}
\PYG{g+gp}{\PYGZgt{}\PYGZgt{}\PYGZgt{} }\PYG{n}{s} \PYG{o}{=} \PYG{n}{np}\PYG{o}{.}\PYG{n}{random}\PYG{o}{.}\PYG{n}{logseries}\PYG{p}{(}\PYG{n}{a}\PYG{p}{,} \PYG{l+m+mi}{10000}\PYG{p}{)}
\PYG{g+gp}{\PYGZgt{}\PYGZgt{}\PYGZgt{} }\PYG{n}{count}\PYG{p}{,} \PYG{n}{bins}\PYG{p}{,} \PYG{n}{ignored} \PYG{o}{=} \PYG{n}{plt}\PYG{o}{.}\PYG{n}{hist}\PYG{p}{(}\PYG{n}{s}\PYG{p}{)}
\end{Verbatim}

\#   plot against distribution

\begin{Verbatim}[commandchars=\\\{\}]
\PYG{g+gp}{\PYGZgt{}\PYGZgt{}\PYGZgt{} }\PYG{k}{def} \PYG{n+nf}{logseries}\PYG{p}{(}\PYG{n}{k}\PYG{p}{,} \PYG{n}{p}\PYG{p}{)}\PYG{p}{:}
\PYG{g+gp}{... }    \PYG{k}{return} \PYG{o}{\PYGZhy{}}\PYG{n}{p}\PYG{o}{*}\PYG{o}{*}\PYG{n}{k}\PYG{o}{/}\PYG{p}{(}\PYG{n}{k}\PYG{o}{*}\PYG{n}{log}\PYG{p}{(}\PYG{l+m+mi}{1}\PYG{o}{\PYGZhy{}}\PYG{n}{p}\PYG{p}{)}\PYG{p}{)}
\PYG{g+gp}{\PYGZgt{}\PYGZgt{}\PYGZgt{} }\PYG{n}{plt}\PYG{o}{.}\PYG{n}{plot}\PYG{p}{(}\PYG{n}{bins}\PYG{p}{,} \PYG{n}{logseries}\PYG{p}{(}\PYG{n}{bins}\PYG{p}{,} \PYG{n}{a}\PYG{p}{)}\PYG{o}{*}\PYG{n}{count}\PYG{o}{.}\PYG{n}{max}\PYG{p}{(}\PYG{p}{)}\PYG{o}{/}
\PYG{g+go}{             logseries(bins, a).max(), \PYGZsq{}r\PYGZsq{})}
\PYG{g+gp}{\PYGZgt{}\PYGZgt{}\PYGZgt{} }\PYG{n}{plt}\PYG{o}{.}\PYG{n}{show}\PYG{p}{(}\PYG{p}{)}
\end{Verbatim}

\end{fulllineitems}

\index{multinomial() (in module main)}

\begin{fulllineitems}
\phantomsection\label{main:main.multinomial}\pysiglinewithargsret{\code{main.}\bfcode{multinomial}}{\emph{n}, \emph{pvals}, \emph{size=None}}{}
Draw samples from a multinomial distribution.

The multinomial distribution is a multivariate generalisation of the
binomial distribution.  Take an experiment with one of \code{p}
possible outcomes.  An example of such an experiment is throwing a dice,
where the outcome can be 1 through 6.  Each sample drawn from the
distribution represents \emph{n} such experiments.  Its values,
\code{X\_i = {[}X\_0, X\_1, ..., X\_p{]}}, represent the number of times the outcome
was \code{i}.
\begin{description}
\item[{n}] \leavevmode{[}int{]}
Number of experiments.

\item[{pvals}] \leavevmode{[}sequence of floats, length p{]}
Probabilities of each of the \code{p} different outcomes.  These
should sum to 1 (however, the last element is always assumed to
account for the remaining probability, as long as
\code{sum(pvals{[}:-1{]}) \textless{}= 1)}.

\item[{size}] \leavevmode{[}tuple of ints{]}
Given a \emph{size} of \code{(M, N, K)}, then \code{M*N*K} samples are drawn,
and the output shape becomes \code{(M, N, K, p)}, since each sample
has shape \code{(p,)}.

\end{description}

Throw a dice 20 times:

\begin{Verbatim}[commandchars=\\\{\}]
\PYG{g+gp}{\PYGZgt{}\PYGZgt{}\PYGZgt{} }\PYG{n}{np}\PYG{o}{.}\PYG{n}{random}\PYG{o}{.}\PYG{n}{multinomial}\PYG{p}{(}\PYG{l+m+mi}{20}\PYG{p}{,} \PYG{p}{[}\PYG{l+m+mi}{1}\PYG{o}{/}\PYG{l+m+mf}{6.}\PYG{p}{]}\PYG{o}{*}\PYG{l+m+mi}{6}\PYG{p}{,} \PYG{n}{size}\PYG{o}{=}\PYG{l+m+mi}{1}\PYG{p}{)}
\PYG{g+go}{array([[4, 1, 7, 5, 2, 1]])}
\end{Verbatim}

It landed 4 times on 1, once on 2, etc.

Now, throw the dice 20 times, and 20 times again:

\begin{Verbatim}[commandchars=\\\{\}]
\PYG{g+gp}{\PYGZgt{}\PYGZgt{}\PYGZgt{} }\PYG{n}{np}\PYG{o}{.}\PYG{n}{random}\PYG{o}{.}\PYG{n}{multinomial}\PYG{p}{(}\PYG{l+m+mi}{20}\PYG{p}{,} \PYG{p}{[}\PYG{l+m+mi}{1}\PYG{o}{/}\PYG{l+m+mf}{6.}\PYG{p}{]}\PYG{o}{*}\PYG{l+m+mi}{6}\PYG{p}{,} \PYG{n}{size}\PYG{o}{=}\PYG{l+m+mi}{2}\PYG{p}{)}
\PYG{g+go}{array([[3, 4, 3, 3, 4, 3],}
\PYG{g+go}{       [2, 4, 3, 4, 0, 7]])}
\end{Verbatim}

For the first run, we threw 3 times 1, 4 times 2, etc.  For the second,
we threw 2 times 1, 4 times 2, etc.

A loaded dice is more likely to land on number 6:

\begin{Verbatim}[commandchars=\\\{\}]
\PYG{g+gp}{\PYGZgt{}\PYGZgt{}\PYGZgt{} }\PYG{n}{np}\PYG{o}{.}\PYG{n}{random}\PYG{o}{.}\PYG{n}{multinomial}\PYG{p}{(}\PYG{l+m+mi}{100}\PYG{p}{,} \PYG{p}{[}\PYG{l+m+mi}{1}\PYG{o}{/}\PYG{l+m+mf}{7.}\PYG{p}{]}\PYG{o}{*}\PYG{l+m+mi}{5}\PYG{p}{)}
\PYG{g+go}{array([13, 16, 13, 16, 42])}
\end{Verbatim}

\end{fulllineitems}

\index{multivariate\_normal() (in module main)}

\begin{fulllineitems}
\phantomsection\label{main:main.multivariate_normal}\pysiglinewithargsret{\code{main.}\bfcode{multivariate\_normal}}{\emph{mean}, \emph{cov}\optional{, \emph{size}}}{}
Draw random samples from a multivariate normal distribution.

The multivariate normal, multinormal or Gaussian distribution is a
generalization of the one-dimensional normal distribution to higher
dimensions.  Such a distribution is specified by its mean and
covariance matrix.  These parameters are analogous to the mean
(average or ``center'') and variance (standard deviation, or ``width,''
squared) of the one-dimensional normal distribution.
\begin{description}
\item[{mean}] \leavevmode{[}1-D array\_like, of length N{]}
Mean of the N-dimensional distribution.

\item[{cov}] \leavevmode{[}2-D array\_like, of shape (N, N){]}
Covariance matrix of the distribution.  Must be symmetric and
positive semi-definite for ``physically meaningful'' results.

\item[{size}] \leavevmode{[}int or tuple of ints, optional{]}
Given a shape of, for example, \code{(m,n,k)}, \code{m*n*k} samples are
generated, and packed in an \emph{m}-by-\emph{n}-by-\emph{k} arrangement.  Because
each sample is \emph{N}-dimensional, the output shape is \code{(m,n,k,N)}.
If no shape is specified, a single (\emph{N}-D) sample is returned.

\end{description}
\begin{description}
\item[{out}] \leavevmode{[}ndarray{]}
The drawn samples, of shape \emph{size}, if that was provided.  If not,
the shape is \code{(N,)}.

In other words, each entry \code{out{[}i,j,...,:{]}} is an N-dimensional
value drawn from the distribution.

\end{description}

The mean is a coordinate in N-dimensional space, which represents the
location where samples are most likely to be generated.  This is
analogous to the peak of the bell curve for the one-dimensional or
univariate normal distribution.

Covariance indicates the level to which two variables vary together.
From the multivariate normal distribution, we draw N-dimensional
samples, \(X = [x_1, x_2, ... x_N]\).  The covariance matrix
element \(C_{ij}\) is the covariance of \(x_i\) and \(x_j\).
The element \(C_{ii}\) is the variance of \(x_i\) (i.e. its
``spread'').

Instead of specifying the full covariance matrix, popular
approximations include:
\begin{itemize}
\item {} 
Spherical covariance (\emph{cov} is a multiple of the identity matrix)

\item {} 
Diagonal covariance (\emph{cov} has non-negative elements, and only on
the diagonal)

\end{itemize}

This geometrical property can be seen in two dimensions by plotting
generated data-points:

\begin{Verbatim}[commandchars=\\\{\}]
\PYG{g+gp}{\PYGZgt{}\PYGZgt{}\PYGZgt{} }\PYG{n}{mean} \PYG{o}{=} \PYG{p}{[}\PYG{l+m+mi}{0}\PYG{p}{,}\PYG{l+m+mi}{0}\PYG{p}{]}
\PYG{g+gp}{\PYGZgt{}\PYGZgt{}\PYGZgt{} }\PYG{n}{cov} \PYG{o}{=} \PYG{p}{[}\PYG{p}{[}\PYG{l+m+mi}{1}\PYG{p}{,}\PYG{l+m+mi}{0}\PYG{p}{]}\PYG{p}{,}\PYG{p}{[}\PYG{l+m+mi}{0}\PYG{p}{,}\PYG{l+m+mi}{100}\PYG{p}{]}\PYG{p}{]} \PYG{c}{\PYGZsh{} diagonal covariance, points lie on x or y\PYGZhy{}axis}
\end{Verbatim}

\begin{Verbatim}[commandchars=\\\{\}]
\PYG{g+gp}{\PYGZgt{}\PYGZgt{}\PYGZgt{} }\PYG{k+kn}{import} \PYG{n+nn}{matplotlib.pyplot} \PYG{k+kn}{as} \PYG{n+nn}{plt}
\PYG{g+gp}{\PYGZgt{}\PYGZgt{}\PYGZgt{} }\PYG{n}{x}\PYG{p}{,}\PYG{n}{y} \PYG{o}{=} \PYG{n}{np}\PYG{o}{.}\PYG{n}{random}\PYG{o}{.}\PYG{n}{multivariate\PYGZus{}normal}\PYG{p}{(}\PYG{n}{mean}\PYG{p}{,}\PYG{n}{cov}\PYG{p}{,}\PYG{l+m+mi}{5000}\PYG{p}{)}\PYG{o}{.}\PYG{n}{T}
\PYG{g+gp}{\PYGZgt{}\PYGZgt{}\PYGZgt{} }\PYG{n}{plt}\PYG{o}{.}\PYG{n}{plot}\PYG{p}{(}\PYG{n}{x}\PYG{p}{,}\PYG{n}{y}\PYG{p}{,}\PYG{l+s}{\PYGZsq{}}\PYG{l+s}{x}\PYG{l+s}{\PYGZsq{}}\PYG{p}{)}\PYG{p}{;} \PYG{n}{plt}\PYG{o}{.}\PYG{n}{axis}\PYG{p}{(}\PYG{l+s}{\PYGZsq{}}\PYG{l+s}{equal}\PYG{l+s}{\PYGZsq{}}\PYG{p}{)}\PYG{p}{;} \PYG{n}{plt}\PYG{o}{.}\PYG{n}{show}\PYG{p}{(}\PYG{p}{)}
\end{Verbatim}

Note that the covariance matrix must be non-negative definite.

Papoulis, A., \emph{Probability, Random Variables, and Stochastic Processes},
3rd ed., New York: McGraw-Hill, 1991.

Duda, R. O., Hart, P. E., and Stork, D. G., \emph{Pattern Classification},
2nd ed., New York: Wiley, 2001.

\begin{Verbatim}[commandchars=\\\{\}]
\PYG{g+gp}{\PYGZgt{}\PYGZgt{}\PYGZgt{} }\PYG{n}{mean} \PYG{o}{=} \PYG{p}{(}\PYG{l+m+mi}{1}\PYG{p}{,}\PYG{l+m+mi}{2}\PYG{p}{)}
\PYG{g+gp}{\PYGZgt{}\PYGZgt{}\PYGZgt{} }\PYG{n}{cov} \PYG{o}{=} \PYG{p}{[}\PYG{p}{[}\PYG{l+m+mi}{1}\PYG{p}{,}\PYG{l+m+mi}{0}\PYG{p}{]}\PYG{p}{,}\PYG{p}{[}\PYG{l+m+mi}{1}\PYG{p}{,}\PYG{l+m+mi}{0}\PYG{p}{]}\PYG{p}{]}
\PYG{g+gp}{\PYGZgt{}\PYGZgt{}\PYGZgt{} }\PYG{n}{x} \PYG{o}{=} \PYG{n}{np}\PYG{o}{.}\PYG{n}{random}\PYG{o}{.}\PYG{n}{multivariate\PYGZus{}normal}\PYG{p}{(}\PYG{n}{mean}\PYG{p}{,}\PYG{n}{cov}\PYG{p}{,}\PYG{p}{(}\PYG{l+m+mi}{3}\PYG{p}{,}\PYG{l+m+mi}{3}\PYG{p}{)}\PYG{p}{)}
\PYG{g+gp}{\PYGZgt{}\PYGZgt{}\PYGZgt{} }\PYG{n}{x}\PYG{o}{.}\PYG{n}{shape}
\PYG{g+go}{(3, 3, 2)}
\end{Verbatim}

The following is probably true, given that 0.6 is roughly twice the
standard deviation:

\begin{Verbatim}[commandchars=\\\{\}]
\PYG{g+gp}{\PYGZgt{}\PYGZgt{}\PYGZgt{} }\PYG{k}{print} \PYG{n+nb}{list}\PYG{p}{(} \PYG{p}{(}\PYG{n}{x}\PYG{p}{[}\PYG{l+m+mi}{0}\PYG{p}{,}\PYG{l+m+mi}{0}\PYG{p}{,}\PYG{p}{:}\PYG{p}{]} \PYG{o}{\PYGZhy{}} \PYG{n}{mean}\PYG{p}{)} \PYG{o}{\PYGZlt{}} \PYG{l+m+mf}{0.6} \PYG{p}{)}
\PYG{g+go}{[True, True]}
\end{Verbatim}

\end{fulllineitems}

\index{negative\_binomial() (in module main)}

\begin{fulllineitems}
\phantomsection\label{main:main.negative_binomial}\pysiglinewithargsret{\code{main.}\bfcode{negative\_binomial}}{\emph{n}, \emph{p}, \emph{size=None}}{}
Draw samples from a negative\_binomial distribution.

Samples are drawn from a negative\_Binomial distribution with specified
parameters, \emph{n} trials and \emph{p} probability of success where \emph{n} is an
integer \textgreater{} 0 and \emph{p} is in the interval {[}0, 1{]}.
\begin{description}
\item[{n}] \leavevmode{[}int{]}
Parameter, \textgreater{} 0.

\item[{p}] \leavevmode{[}float{]}
Parameter, \textgreater{}= 0 and \textless{}=1.

\item[{size}] \leavevmode{[}int or tuple of ints{]}
Output shape. If the given shape is, e.g., \code{(m, n, k)}, then
\code{m * n * k} samples are drawn.

\end{description}
\begin{description}
\item[{samples}] \leavevmode{[}int or ndarray of ints{]}
Drawn samples.

\end{description}

The probability density for the Negative Binomial distribution is
\begin{gather}
\begin{split}P(N;n,p) = \binom{N+n-1}{n-1}p^{n}(1-p)^{N},\end{split}\notag
\end{gather}
where \(n-1\) is the number of successes, \(p\) is the probability
of success, and \(N+n-1\) is the number of trials.

The negative binomial distribution gives the probability of n-1 successes
and N failures in N+n-1 trials, and success on the (N+n)th trial.

If one throws a die repeatedly until the third time a ``1'' appears, then the
probability distribution of the number of non-``1''s that appear before the
third ``1'' is a negative binomial distribution.

Draw samples from the distribution:

A real world example. A company drills wild-cat oil exploration wells, each
with an estimated probability of success of 0.1.  What is the probability
of having one success for each successive well, that is what is the
probability of a single success after drilling 5 wells, after 6 wells,
etc.?

\begin{Verbatim}[commandchars=\\\{\}]
\PYG{g+gp}{\PYGZgt{}\PYGZgt{}\PYGZgt{} }\PYG{n}{s} \PYG{o}{=} \PYG{n}{np}\PYG{o}{.}\PYG{n}{random}\PYG{o}{.}\PYG{n}{negative\PYGZus{}binomial}\PYG{p}{(}\PYG{l+m+mi}{1}\PYG{p}{,} \PYG{l+m+mf}{0.1}\PYG{p}{,} \PYG{l+m+mi}{100000}\PYG{p}{)}
\PYG{g+gp}{\PYGZgt{}\PYGZgt{}\PYGZgt{} }\PYG{k}{for} \PYG{n}{i} \PYG{o+ow}{in} \PYG{n+nb}{range}\PYG{p}{(}\PYG{l+m+mi}{1}\PYG{p}{,} \PYG{l+m+mi}{11}\PYG{p}{)}\PYG{p}{:}
\PYG{g+gp}{... }   \PYG{n}{probability} \PYG{o}{=} \PYG{n+nb}{sum}\PYG{p}{(}\PYG{n}{s}\PYG{o}{\PYGZlt{}}\PYG{n}{i}\PYG{p}{)} \PYG{o}{/} \PYG{l+m+mf}{100000.}
\PYG{g+gp}{... }   \PYG{k}{print} \PYG{n}{i}\PYG{p}{,} \PYG{l+s}{\PYGZdq{}}\PYG{l+s}{wells drilled, probability of one success =}\PYG{l+s}{\PYGZdq{}}\PYG{p}{,} \PYG{n}{probability}
\end{Verbatim}

\end{fulllineitems}

\index{noncentral\_chisquare() (in module main)}

\begin{fulllineitems}
\phantomsection\label{main:main.noncentral_chisquare}\pysiglinewithargsret{\code{main.}\bfcode{noncentral\_chisquare}}{\emph{df}, \emph{nonc}, \emph{size=None}}{}
Draw samples from a noncentral chi-square distribution.

The noncentral \(\chi^2\) distribution is a generalisation of
the \(\chi^2\) distribution.
\begin{description}
\item[{df}] \leavevmode{[}int{]}
Degrees of freedom, should be \textgreater{}= 1.

\item[{nonc}] \leavevmode{[}float{]}
Non-centrality, should be \textgreater{} 0.

\item[{size}] \leavevmode{[}int or tuple of ints{]}
Shape of the output.

\end{description}

The probability density function for the noncentral Chi-square distribution
is
\begin{gather}
\begin{split}P(x;df,nonc) = \sum^{\infty}_{i=0}
\frac{e^{-nonc/2}(nonc/2)^{i}}{i!}P_{Y_{df+2i}}(x),\end{split}\notag
\end{gather}
where \(Y_{q}\) is the Chi-square with q degrees of freedom.

In Delhi (2007), it is noted that the noncentral chi-square is useful in
bombing and coverage problems, the probability of killing the point target
given by the noncentral chi-squared distribution.

Draw values from the distribution and plot the histogram

\begin{Verbatim}[commandchars=\\\{\}]
\PYG{g+gp}{\PYGZgt{}\PYGZgt{}\PYGZgt{} }\PYG{k+kn}{import} \PYG{n+nn}{matplotlib.pyplot} \PYG{k+kn}{as} \PYG{n+nn}{plt}
\PYG{g+gp}{\PYGZgt{}\PYGZgt{}\PYGZgt{} }\PYG{n}{values} \PYG{o}{=} \PYG{n}{plt}\PYG{o}{.}\PYG{n}{hist}\PYG{p}{(}\PYG{n}{np}\PYG{o}{.}\PYG{n}{random}\PYG{o}{.}\PYG{n}{noncentral\PYGZus{}chisquare}\PYG{p}{(}\PYG{l+m+mi}{3}\PYG{p}{,} \PYG{l+m+mi}{20}\PYG{p}{,} \PYG{l+m+mi}{100000}\PYG{p}{)}\PYG{p}{,}
\PYG{g+gp}{... }                  \PYG{n}{bins}\PYG{o}{=}\PYG{l+m+mi}{200}\PYG{p}{,} \PYG{n}{normed}\PYG{o}{=}\PYG{n+nb+bp}{True}\PYG{p}{)}
\PYG{g+gp}{\PYGZgt{}\PYGZgt{}\PYGZgt{} }\PYG{n}{plt}\PYG{o}{.}\PYG{n}{show}\PYG{p}{(}\PYG{p}{)}
\end{Verbatim}

Draw values from a noncentral chisquare with very small noncentrality,
and compare to a chisquare.

\begin{Verbatim}[commandchars=\\\{\}]
\PYG{g+gp}{\PYGZgt{}\PYGZgt{}\PYGZgt{} }\PYG{n}{plt}\PYG{o}{.}\PYG{n}{figure}\PYG{p}{(}\PYG{p}{)}
\PYG{g+gp}{\PYGZgt{}\PYGZgt{}\PYGZgt{} }\PYG{n}{values} \PYG{o}{=} \PYG{n}{plt}\PYG{o}{.}\PYG{n}{hist}\PYG{p}{(}\PYG{n}{np}\PYG{o}{.}\PYG{n}{random}\PYG{o}{.}\PYG{n}{noncentral\PYGZus{}chisquare}\PYG{p}{(}\PYG{l+m+mi}{3}\PYG{p}{,} \PYG{o}{.}\PYG{l+m+mo}{0000001}\PYG{p}{,} \PYG{l+m+mi}{100000}\PYG{p}{)}\PYG{p}{,}
\PYG{g+gp}{... }                  \PYG{n}{bins}\PYG{o}{=}\PYG{n}{np}\PYG{o}{.}\PYG{n}{arange}\PYG{p}{(}\PYG{l+m+mf}{0.}\PYG{p}{,} \PYG{l+m+mi}{25}\PYG{p}{,} \PYG{o}{.}\PYG{l+m+mi}{1}\PYG{p}{)}\PYG{p}{,} \PYG{n}{normed}\PYG{o}{=}\PYG{n+nb+bp}{True}\PYG{p}{)}
\PYG{g+gp}{\PYGZgt{}\PYGZgt{}\PYGZgt{} }\PYG{n}{values2} \PYG{o}{=} \PYG{n}{plt}\PYG{o}{.}\PYG{n}{hist}\PYG{p}{(}\PYG{n}{np}\PYG{o}{.}\PYG{n}{random}\PYG{o}{.}\PYG{n}{chisquare}\PYG{p}{(}\PYG{l+m+mi}{3}\PYG{p}{,} \PYG{l+m+mi}{100000}\PYG{p}{)}\PYG{p}{,}
\PYG{g+gp}{... }                   \PYG{n}{bins}\PYG{o}{=}\PYG{n}{np}\PYG{o}{.}\PYG{n}{arange}\PYG{p}{(}\PYG{l+m+mf}{0.}\PYG{p}{,} \PYG{l+m+mi}{25}\PYG{p}{,} \PYG{o}{.}\PYG{l+m+mi}{1}\PYG{p}{)}\PYG{p}{,} \PYG{n}{normed}\PYG{o}{=}\PYG{n+nb+bp}{True}\PYG{p}{)}
\PYG{g+gp}{\PYGZgt{}\PYGZgt{}\PYGZgt{} }\PYG{n}{plt}\PYG{o}{.}\PYG{n}{plot}\PYG{p}{(}\PYG{n}{values}\PYG{p}{[}\PYG{l+m+mi}{1}\PYG{p}{]}\PYG{p}{[}\PYG{l+m+mi}{0}\PYG{p}{:}\PYG{o}{\PYGZhy{}}\PYG{l+m+mi}{1}\PYG{p}{]}\PYG{p}{,} \PYG{n}{values}\PYG{p}{[}\PYG{l+m+mi}{0}\PYG{p}{]}\PYG{o}{\PYGZhy{}}\PYG{n}{values2}\PYG{p}{[}\PYG{l+m+mi}{0}\PYG{p}{]}\PYG{p}{,} \PYG{l+s}{\PYGZsq{}}\PYG{l+s}{ob}\PYG{l+s}{\PYGZsq{}}\PYG{p}{)}
\PYG{g+gp}{\PYGZgt{}\PYGZgt{}\PYGZgt{} }\PYG{n}{plt}\PYG{o}{.}\PYG{n}{show}\PYG{p}{(}\PYG{p}{)}
\end{Verbatim}

Demonstrate how large values of non-centrality lead to a more symmetric
distribution.

\begin{Verbatim}[commandchars=\\\{\}]
\PYG{g+gp}{\PYGZgt{}\PYGZgt{}\PYGZgt{} }\PYG{n}{plt}\PYG{o}{.}\PYG{n}{figure}\PYG{p}{(}\PYG{p}{)}
\PYG{g+gp}{\PYGZgt{}\PYGZgt{}\PYGZgt{} }\PYG{n}{values} \PYG{o}{=} \PYG{n}{plt}\PYG{o}{.}\PYG{n}{hist}\PYG{p}{(}\PYG{n}{np}\PYG{o}{.}\PYG{n}{random}\PYG{o}{.}\PYG{n}{noncentral\PYGZus{}chisquare}\PYG{p}{(}\PYG{l+m+mi}{3}\PYG{p}{,} \PYG{l+m+mi}{20}\PYG{p}{,} \PYG{l+m+mi}{100000}\PYG{p}{)}\PYG{p}{,}
\PYG{g+gp}{... }                  \PYG{n}{bins}\PYG{o}{=}\PYG{l+m+mi}{200}\PYG{p}{,} \PYG{n}{normed}\PYG{o}{=}\PYG{n+nb+bp}{True}\PYG{p}{)}
\PYG{g+gp}{\PYGZgt{}\PYGZgt{}\PYGZgt{} }\PYG{n}{plt}\PYG{o}{.}\PYG{n}{show}\PYG{p}{(}\PYG{p}{)}
\end{Verbatim}

\end{fulllineitems}

\index{noncentral\_f() (in module main)}

\begin{fulllineitems}
\phantomsection\label{main:main.noncentral_f}\pysiglinewithargsret{\code{main.}\bfcode{noncentral\_f}}{\emph{dfnum}, \emph{dfden}, \emph{nonc}, \emph{size=None}}{}
Draw samples from the noncentral F distribution.

Samples are drawn from an F distribution with specified parameters,
\emph{dfnum} (degrees of freedom in numerator) and \emph{dfden} (degrees of
freedom in denominator), where both parameters \textgreater{} 1.
\emph{nonc} is the non-centrality parameter.
\begin{description}
\item[{dfnum}] \leavevmode{[}int{]}
Parameter, should be \textgreater{} 1.

\item[{dfden}] \leavevmode{[}int{]}
Parameter, should be \textgreater{} 1.

\item[{nonc}] \leavevmode{[}float{]}
Parameter, should be \textgreater{}= 0.

\item[{size}] \leavevmode{[}int or tuple of ints{]}
Output shape. If the given shape is, e.g., \code{(m, n, k)}, then
\code{m * n * k} samples are drawn.

\end{description}
\begin{description}
\item[{samples}] \leavevmode{[}scalar or ndarray{]}
Drawn samples.

\end{description}

When calculating the power of an experiment (power = probability of
rejecting the null hypothesis when a specific alternative is true) the
non-central F statistic becomes important.  When the null hypothesis is
true, the F statistic follows a central F distribution. When the null
hypothesis is not true, then it follows a non-central F statistic.

Weisstein, Eric W. ``Noncentral F-Distribution.'' From MathWorld--A Wolfram
Web Resource.  \href{http://mathworld.wolfram.com/NoncentralF-Distribution.html}{http://mathworld.wolfram.com/NoncentralF-Distribution.html}

Wikipedia, ``Noncentral F distribution'',
\href{http://en.wikipedia.org/wiki/Noncentral\_F-distribution}{http://en.wikipedia.org/wiki/Noncentral\_F-distribution}

In a study, testing for a specific alternative to the null hypothesis
requires use of the Noncentral F distribution. We need to calculate the
area in the tail of the distribution that exceeds the value of the F
distribution for the null hypothesis.  We'll plot the two probability
distributions for comparison.

\begin{Verbatim}[commandchars=\\\{\}]
\PYG{g+gp}{\PYGZgt{}\PYGZgt{}\PYGZgt{} }\PYG{n}{dfnum} \PYG{o}{=} \PYG{l+m+mi}{3} \PYG{c}{\PYGZsh{} between group deg of freedom}
\PYG{g+gp}{\PYGZgt{}\PYGZgt{}\PYGZgt{} }\PYG{n}{dfden} \PYG{o}{=} \PYG{l+m+mi}{20} \PYG{c}{\PYGZsh{} within groups degrees of freedom}
\PYG{g+gp}{\PYGZgt{}\PYGZgt{}\PYGZgt{} }\PYG{n}{nonc} \PYG{o}{=} \PYG{l+m+mf}{3.0}
\PYG{g+gp}{\PYGZgt{}\PYGZgt{}\PYGZgt{} }\PYG{n}{nc\PYGZus{}vals} \PYG{o}{=} \PYG{n}{np}\PYG{o}{.}\PYG{n}{random}\PYG{o}{.}\PYG{n}{noncentral\PYGZus{}f}\PYG{p}{(}\PYG{n}{dfnum}\PYG{p}{,} \PYG{n}{dfden}\PYG{p}{,} \PYG{n}{nonc}\PYG{p}{,} \PYG{l+m+mi}{1000000}\PYG{p}{)}
\PYG{g+gp}{\PYGZgt{}\PYGZgt{}\PYGZgt{} }\PYG{n}{NF} \PYG{o}{=} \PYG{n}{np}\PYG{o}{.}\PYG{n}{histogram}\PYG{p}{(}\PYG{n}{nc\PYGZus{}vals}\PYG{p}{,} \PYG{n}{bins}\PYG{o}{=}\PYG{l+m+mi}{50}\PYG{p}{,} \PYG{n}{normed}\PYG{o}{=}\PYG{n+nb+bp}{True}\PYG{p}{)}
\PYG{g+gp}{\PYGZgt{}\PYGZgt{}\PYGZgt{} }\PYG{n}{c\PYGZus{}vals} \PYG{o}{=} \PYG{n}{np}\PYG{o}{.}\PYG{n}{random}\PYG{o}{.}\PYG{n}{f}\PYG{p}{(}\PYG{n}{dfnum}\PYG{p}{,} \PYG{n}{dfden}\PYG{p}{,} \PYG{l+m+mi}{1000000}\PYG{p}{)}
\PYG{g+gp}{\PYGZgt{}\PYGZgt{}\PYGZgt{} }\PYG{n}{F} \PYG{o}{=} \PYG{n}{np}\PYG{o}{.}\PYG{n}{histogram}\PYG{p}{(}\PYG{n}{c\PYGZus{}vals}\PYG{p}{,} \PYG{n}{bins}\PYG{o}{=}\PYG{l+m+mi}{50}\PYG{p}{,} \PYG{n}{normed}\PYG{o}{=}\PYG{n+nb+bp}{True}\PYG{p}{)}
\PYG{g+gp}{\PYGZgt{}\PYGZgt{}\PYGZgt{} }\PYG{n}{plt}\PYG{o}{.}\PYG{n}{plot}\PYG{p}{(}\PYG{n}{F}\PYG{p}{[}\PYG{l+m+mi}{1}\PYG{p}{]}\PYG{p}{[}\PYG{l+m+mi}{1}\PYG{p}{:}\PYG{p}{]}\PYG{p}{,} \PYG{n}{F}\PYG{p}{[}\PYG{l+m+mi}{0}\PYG{p}{]}\PYG{p}{)}
\PYG{g+gp}{\PYGZgt{}\PYGZgt{}\PYGZgt{} }\PYG{n}{plt}\PYG{o}{.}\PYG{n}{plot}\PYG{p}{(}\PYG{n}{NF}\PYG{p}{[}\PYG{l+m+mi}{1}\PYG{p}{]}\PYG{p}{[}\PYG{l+m+mi}{1}\PYG{p}{:}\PYG{p}{]}\PYG{p}{,} \PYG{n}{NF}\PYG{p}{[}\PYG{l+m+mi}{0}\PYG{p}{]}\PYG{p}{)}
\PYG{g+gp}{\PYGZgt{}\PYGZgt{}\PYGZgt{} }\PYG{n}{plt}\PYG{o}{.}\PYG{n}{show}\PYG{p}{(}\PYG{p}{)}
\end{Verbatim}

\end{fulllineitems}

\index{normal() (in module main)}

\begin{fulllineitems}
\phantomsection\label{main:main.normal}\pysiglinewithargsret{\code{main.}\bfcode{normal}}{\emph{loc=0.0}, \emph{scale=1.0}, \emph{size=None}}{}
Draw random samples from a normal (Gaussian) distribution.

The probability density function of the normal distribution, first
derived by De Moivre and 200 years later by both Gauss and Laplace
independently {\color{red}\bfseries{}{[}2{]}\_}, is often called the bell curve because of
its characteristic shape (see the example below).

The normal distributions occurs often in nature.  For example, it
describes the commonly occurring distribution of samples influenced
by a large number of tiny, random disturbances, each with its own
unique distribution {\color{red}\bfseries{}{[}2{]}\_}.
\begin{description}
\item[{loc}] \leavevmode{[}float{]}
Mean (``centre'') of the distribution.

\item[{scale}] \leavevmode{[}float{]}
Standard deviation (spread or ``width'') of the distribution.

\item[{size}] \leavevmode{[}tuple of ints{]}
Output shape.  If the given shape is, e.g., \code{(m, n, k)}, then
\code{m * n * k} samples are drawn.

\end{description}
\begin{description}
\item[{scipy.stats.distributions.norm}] \leavevmode{[}probability density function,{]}
distribution or cumulative density function, etc.

\end{description}

The probability density for the Gaussian distribution is
\begin{gather}
\begin{split}p(x) = \frac{1}{\sqrt{ 2 \pi \sigma^2 }}
e^{ - \frac{ (x - \mu)^2 } {2 \sigma^2} },\end{split}\notag
\end{gather}
where \(\mu\) is the mean and \(\sigma\) the standard deviation.
The square of the standard deviation, \(\sigma^2\), is called the
variance.

The function has its peak at the mean, and its ``spread'' increases with
the standard deviation (the function reaches 0.607 times its maximum at
\(x + \sigma\) and \(x - \sigma\) {\color{red}\bfseries{}{[}2{]}\_}).  This implies that
\emph{numpy.random.normal} is more likely to return samples lying close to the
mean, rather than those far away.

Draw samples from the distribution:

\begin{Verbatim}[commandchars=\\\{\}]
\PYG{g+gp}{\PYGZgt{}\PYGZgt{}\PYGZgt{} }\PYG{n}{mu}\PYG{p}{,} \PYG{n}{sigma} \PYG{o}{=} \PYG{l+m+mi}{0}\PYG{p}{,} \PYG{l+m+mf}{0.1} \PYG{c}{\PYGZsh{} mean and standard deviation}
\PYG{g+gp}{\PYGZgt{}\PYGZgt{}\PYGZgt{} }\PYG{n}{s} \PYG{o}{=} \PYG{n}{np}\PYG{o}{.}\PYG{n}{random}\PYG{o}{.}\PYG{n}{normal}\PYG{p}{(}\PYG{n}{mu}\PYG{p}{,} \PYG{n}{sigma}\PYG{p}{,} \PYG{l+m+mi}{1000}\PYG{p}{)}
\end{Verbatim}

Verify the mean and the variance:

\begin{Verbatim}[commandchars=\\\{\}]
\PYG{g+gp}{\PYGZgt{}\PYGZgt{}\PYGZgt{} }\PYG{n+nb}{abs}\PYG{p}{(}\PYG{n}{mu} \PYG{o}{\PYGZhy{}} \PYG{n}{np}\PYG{o}{.}\PYG{n}{mean}\PYG{p}{(}\PYG{n}{s}\PYG{p}{)}\PYG{p}{)} \PYG{o}{\PYGZlt{}} \PYG{l+m+mf}{0.01}
\PYG{g+go}{True}
\end{Verbatim}

\begin{Verbatim}[commandchars=\\\{\}]
\PYG{g+gp}{\PYGZgt{}\PYGZgt{}\PYGZgt{} }\PYG{n+nb}{abs}\PYG{p}{(}\PYG{n}{sigma} \PYG{o}{\PYGZhy{}} \PYG{n}{np}\PYG{o}{.}\PYG{n}{std}\PYG{p}{(}\PYG{n}{s}\PYG{p}{,} \PYG{n}{ddof}\PYG{o}{=}\PYG{l+m+mi}{1}\PYG{p}{)}\PYG{p}{)} \PYG{o}{\PYGZlt{}} \PYG{l+m+mf}{0.01}
\PYG{g+go}{True}
\end{Verbatim}

Display the histogram of the samples, along with
the probability density function:

\begin{Verbatim}[commandchars=\\\{\}]
\PYG{g+gp}{\PYGZgt{}\PYGZgt{}\PYGZgt{} }\PYG{k+kn}{import} \PYG{n+nn}{matplotlib.pyplot} \PYG{k+kn}{as} \PYG{n+nn}{plt}
\PYG{g+gp}{\PYGZgt{}\PYGZgt{}\PYGZgt{} }\PYG{n}{count}\PYG{p}{,} \PYG{n}{bins}\PYG{p}{,} \PYG{n}{ignored} \PYG{o}{=} \PYG{n}{plt}\PYG{o}{.}\PYG{n}{hist}\PYG{p}{(}\PYG{n}{s}\PYG{p}{,} \PYG{l+m+mi}{30}\PYG{p}{,} \PYG{n}{normed}\PYG{o}{=}\PYG{n+nb+bp}{True}\PYG{p}{)}
\PYG{g+gp}{\PYGZgt{}\PYGZgt{}\PYGZgt{} }\PYG{n}{plt}\PYG{o}{.}\PYG{n}{plot}\PYG{p}{(}\PYG{n}{bins}\PYG{p}{,} \PYG{l+m+mi}{1}\PYG{o}{/}\PYG{p}{(}\PYG{n}{sigma} \PYG{o}{*} \PYG{n}{np}\PYG{o}{.}\PYG{n}{sqrt}\PYG{p}{(}\PYG{l+m+mi}{2} \PYG{o}{*} \PYG{n}{np}\PYG{o}{.}\PYG{n}{pi}\PYG{p}{)}\PYG{p}{)} \PYG{o}{*}
\PYG{g+gp}{... }               \PYG{n}{np}\PYG{o}{.}\PYG{n}{exp}\PYG{p}{(} \PYG{o}{\PYGZhy{}} \PYG{p}{(}\PYG{n}{bins} \PYG{o}{\PYGZhy{}} \PYG{n}{mu}\PYG{p}{)}\PYG{o}{*}\PYG{o}{*}\PYG{l+m+mi}{2} \PYG{o}{/} \PYG{p}{(}\PYG{l+m+mi}{2} \PYG{o}{*} \PYG{n}{sigma}\PYG{o}{*}\PYG{o}{*}\PYG{l+m+mi}{2}\PYG{p}{)} \PYG{p}{)}\PYG{p}{,}
\PYG{g+gp}{... }         \PYG{n}{linewidth}\PYG{o}{=}\PYG{l+m+mi}{2}\PYG{p}{,} \PYG{n}{color}\PYG{o}{=}\PYG{l+s}{\PYGZsq{}}\PYG{l+s}{r}\PYG{l+s}{\PYGZsq{}}\PYG{p}{)}
\PYG{g+gp}{\PYGZgt{}\PYGZgt{}\PYGZgt{} }\PYG{n}{plt}\PYG{o}{.}\PYG{n}{show}\PYG{p}{(}\PYG{p}{)}
\end{Verbatim}

\end{fulllineitems}

\index{pareto() (in module main)}

\begin{fulllineitems}
\phantomsection\label{main:main.pareto}\pysiglinewithargsret{\code{main.}\bfcode{pareto}}{\emph{a}, \emph{size=None}}{}
Draw samples from a Pareto II or Lomax distribution with specified shape.

The Lomax or Pareto II distribution is a shifted Pareto distribution. The
classical Pareto distribution can be obtained from the Lomax distribution
by adding the location parameter m, see below. The smallest value of the
Lomax distribution is zero while for the classical Pareto distribution it
is m, where the standard Pareto distribution has location m=1.
Lomax can also be considered as a simplified version of the Generalized
Pareto distribution (available in SciPy), with the scale set to one and
the location set to zero.

The Pareto distribution must be greater than zero, and is unbounded above.
It is also known as the ``80-20 rule''.  In this distribution, 80 percent of
the weights are in the lowest 20 percent of the range, while the other 20
percent fill the remaining 80 percent of the range.
\begin{description}
\item[{shape}] \leavevmode{[}float, \textgreater{} 0.{]}
Shape of the distribution.

\item[{size}] \leavevmode{[}tuple of ints{]}
Output shape.  If the given shape is, e.g., \code{(m, n, k)}, then
\code{m * n * k} samples are drawn.

\end{description}
\begin{description}
\item[{scipy.stats.distributions.lomax.pdf}] \leavevmode{[}probability density function,{]}
distribution or cumulative density function, etc.

\item[{scipy.stats.distributions.genpareto.pdf}] \leavevmode{[}probability density function,{]}
distribution or cumulative density function, etc.

\end{description}

The probability density for the Pareto distribution is
\begin{gather}
\begin{split}p(x) = \frac{am^a}{x^{a+1}}\end{split}\notag
\end{gather}
where \(a\) is the shape and \(m\) the location

The Pareto distribution, named after the Italian economist Vilfredo Pareto,
is a power law probability distribution useful in many real world problems.
Outside the field of economics it is generally referred to as the Bradford
distribution. Pareto developed the distribution to describe the
distribution of wealth in an economy.  It has also found use in insurance,
web page access statistics, oil field sizes, and many other problems,
including the download frequency for projects in Sourceforge {[}1{]}.  It is
one of the so-called ``fat-tailed'' distributions.

Draw samples from the distribution:

\begin{Verbatim}[commandchars=\\\{\}]
\PYG{g+gp}{\PYGZgt{}\PYGZgt{}\PYGZgt{} }\PYG{n}{a}\PYG{p}{,} \PYG{n}{m} \PYG{o}{=} \PYG{l+m+mf}{3.}\PYG{p}{,} \PYG{l+m+mf}{1.} \PYG{c}{\PYGZsh{} shape and mode}
\PYG{g+gp}{\PYGZgt{}\PYGZgt{}\PYGZgt{} }\PYG{n}{s} \PYG{o}{=} \PYG{n}{np}\PYG{o}{.}\PYG{n}{random}\PYG{o}{.}\PYG{n}{pareto}\PYG{p}{(}\PYG{n}{a}\PYG{p}{,} \PYG{l+m+mi}{1000}\PYG{p}{)} \PYG{o}{+} \PYG{n}{m}
\end{Verbatim}

Display the histogram of the samples, along with
the probability density function:

\begin{Verbatim}[commandchars=\\\{\}]
\PYG{g+gp}{\PYGZgt{}\PYGZgt{}\PYGZgt{} }\PYG{k+kn}{import} \PYG{n+nn}{matplotlib.pyplot} \PYG{k+kn}{as} \PYG{n+nn}{plt}
\PYG{g+gp}{\PYGZgt{}\PYGZgt{}\PYGZgt{} }\PYG{n}{count}\PYG{p}{,} \PYG{n}{bins}\PYG{p}{,} \PYG{n}{ignored} \PYG{o}{=} \PYG{n}{plt}\PYG{o}{.}\PYG{n}{hist}\PYG{p}{(}\PYG{n}{s}\PYG{p}{,} \PYG{l+m+mi}{100}\PYG{p}{,} \PYG{n}{normed}\PYG{o}{=}\PYG{n+nb+bp}{True}\PYG{p}{,} \PYG{n}{align}\PYG{o}{=}\PYG{l+s}{\PYGZsq{}}\PYG{l+s}{center}\PYG{l+s}{\PYGZsq{}}\PYG{p}{)}
\PYG{g+gp}{\PYGZgt{}\PYGZgt{}\PYGZgt{} }\PYG{n}{fit} \PYG{o}{=} \PYG{n}{a}\PYG{o}{*}\PYG{n}{m}\PYG{o}{*}\PYG{o}{*}\PYG{n}{a}\PYG{o}{/}\PYG{n}{bins}\PYG{o}{*}\PYG{o}{*}\PYG{p}{(}\PYG{n}{a}\PYG{o}{+}\PYG{l+m+mi}{1}\PYG{p}{)}
\PYG{g+gp}{\PYGZgt{}\PYGZgt{}\PYGZgt{} }\PYG{n}{plt}\PYG{o}{.}\PYG{n}{plot}\PYG{p}{(}\PYG{n}{bins}\PYG{p}{,} \PYG{n+nb}{max}\PYG{p}{(}\PYG{n}{count}\PYG{p}{)}\PYG{o}{*}\PYG{n}{fit}\PYG{o}{/}\PYG{n+nb}{max}\PYG{p}{(}\PYG{n}{fit}\PYG{p}{)}\PYG{p}{,}\PYG{n}{linewidth}\PYG{o}{=}\PYG{l+m+mi}{2}\PYG{p}{,} \PYG{n}{color}\PYG{o}{=}\PYG{l+s}{\PYGZsq{}}\PYG{l+s}{r}\PYG{l+s}{\PYGZsq{}}\PYG{p}{)}
\PYG{g+gp}{\PYGZgt{}\PYGZgt{}\PYGZgt{} }\PYG{n}{plt}\PYG{o}{.}\PYG{n}{show}\PYG{p}{(}\PYG{p}{)}
\end{Verbatim}

\end{fulllineitems}

\index{permutation() (in module main)}

\begin{fulllineitems}
\phantomsection\label{main:main.permutation}\pysiglinewithargsret{\code{main.}\bfcode{permutation}}{\emph{x}}{}
Randomly permute a sequence, or return a permuted range.

If \emph{x} is a multi-dimensional array, it is only shuffled along its
first index.
\begin{description}
\item[{x}] \leavevmode{[}int or array\_like{]}
If \emph{x} is an integer, randomly permute \code{np.arange(x)}.
If \emph{x} is an array, make a copy and shuffle the elements
randomly.

\end{description}
\begin{description}
\item[{out}] \leavevmode{[}ndarray{]}
Permuted sequence or array range.

\end{description}

\begin{Verbatim}[commandchars=\\\{\}]
\PYG{g+gp}{\PYGZgt{}\PYGZgt{}\PYGZgt{} }\PYG{n}{np}\PYG{o}{.}\PYG{n}{random}\PYG{o}{.}\PYG{n}{permutation}\PYG{p}{(}\PYG{l+m+mi}{10}\PYG{p}{)}
\PYG{g+go}{array([1, 7, 4, 3, 0, 9, 2, 5, 8, 6])}
\end{Verbatim}

\begin{Verbatim}[commandchars=\\\{\}]
\PYG{g+gp}{\PYGZgt{}\PYGZgt{}\PYGZgt{} }\PYG{n}{np}\PYG{o}{.}\PYG{n}{random}\PYG{o}{.}\PYG{n}{permutation}\PYG{p}{(}\PYG{p}{[}\PYG{l+m+mi}{1}\PYG{p}{,} \PYG{l+m+mi}{4}\PYG{p}{,} \PYG{l+m+mi}{9}\PYG{p}{,} \PYG{l+m+mi}{12}\PYG{p}{,} \PYG{l+m+mi}{15}\PYG{p}{]}\PYG{p}{)}
\PYG{g+go}{array([15,  1,  9,  4, 12])}
\end{Verbatim}

\begin{Verbatim}[commandchars=\\\{\}]
\PYG{g+gp}{\PYGZgt{}\PYGZgt{}\PYGZgt{} }\PYG{n}{arr} \PYG{o}{=} \PYG{n}{np}\PYG{o}{.}\PYG{n}{arange}\PYG{p}{(}\PYG{l+m+mi}{9}\PYG{p}{)}\PYG{o}{.}\PYG{n}{reshape}\PYG{p}{(}\PYG{p}{(}\PYG{l+m+mi}{3}\PYG{p}{,} \PYG{l+m+mi}{3}\PYG{p}{)}\PYG{p}{)}
\PYG{g+gp}{\PYGZgt{}\PYGZgt{}\PYGZgt{} }\PYG{n}{np}\PYG{o}{.}\PYG{n}{random}\PYG{o}{.}\PYG{n}{permutation}\PYG{p}{(}\PYG{n}{arr}\PYG{p}{)}
\PYG{g+go}{array([[6, 7, 8],}
\PYG{g+go}{       [0, 1, 2],}
\PYG{g+go}{       [3, 4, 5]])}
\end{Verbatim}

\end{fulllineitems}

\index{poisson() (in module main)}

\begin{fulllineitems}
\phantomsection\label{main:main.poisson}\pysiglinewithargsret{\code{main.}\bfcode{poisson}}{\emph{lam=1.0}, \emph{size=None}}{}
Draw samples from a Poisson distribution.

The Poisson distribution is the limit of the Binomial
distribution for large N.
\begin{description}
\item[{lam}] \leavevmode{[}float{]}
Expectation of interval, should be \textgreater{}= 0.

\item[{size}] \leavevmode{[}int or tuple of ints, optional{]}
Output shape. If the given shape is, e.g., \code{(m, n, k)}, then
\code{m * n * k} samples are drawn.

\end{description}

The Poisson distribution
\begin{gather}
\begin{split}f(k; \lambda)=\frac{\lambda^k e^{-\lambda}}{k!}\end{split}\notag
\end{gather}
For events with an expected separation \(\lambda\) the Poisson
distribution \(f(k; \lambda)\) describes the probability of
\(k\) events occurring within the observed interval \(\lambda\).

Because the output is limited to the range of the C long type, a
ValueError is raised when \emph{lam} is within 10 sigma of the maximum
representable value.

Draw samples from the distribution:

\begin{Verbatim}[commandchars=\\\{\}]
\PYG{g+gp}{\PYGZgt{}\PYGZgt{}\PYGZgt{} }\PYG{k+kn}{import} \PYG{n+nn}{numpy} \PYG{k+kn}{as} \PYG{n+nn}{np}
\PYG{g+gp}{\PYGZgt{}\PYGZgt{}\PYGZgt{} }\PYG{n}{s} \PYG{o}{=} \PYG{n}{np}\PYG{o}{.}\PYG{n}{random}\PYG{o}{.}\PYG{n}{poisson}\PYG{p}{(}\PYG{l+m+mi}{5}\PYG{p}{,} \PYG{l+m+mi}{10000}\PYG{p}{)}
\end{Verbatim}

Display histogram of the sample:

\begin{Verbatim}[commandchars=\\\{\}]
\PYG{g+gp}{\PYGZgt{}\PYGZgt{}\PYGZgt{} }\PYG{k+kn}{import} \PYG{n+nn}{matplotlib.pyplot} \PYG{k+kn}{as} \PYG{n+nn}{plt}
\PYG{g+gp}{\PYGZgt{}\PYGZgt{}\PYGZgt{} }\PYG{n}{count}\PYG{p}{,} \PYG{n}{bins}\PYG{p}{,} \PYG{n}{ignored} \PYG{o}{=} \PYG{n}{plt}\PYG{o}{.}\PYG{n}{hist}\PYG{p}{(}\PYG{n}{s}\PYG{p}{,} \PYG{l+m+mi}{14}\PYG{p}{,} \PYG{n}{normed}\PYG{o}{=}\PYG{n+nb+bp}{True}\PYG{p}{)}
\PYG{g+gp}{\PYGZgt{}\PYGZgt{}\PYGZgt{} }\PYG{n}{plt}\PYG{o}{.}\PYG{n}{show}\PYG{p}{(}\PYG{p}{)}
\end{Verbatim}

\end{fulllineitems}

\index{power() (in module main)}

\begin{fulllineitems}
\phantomsection\label{main:main.power}\pysiglinewithargsret{\code{main.}\bfcode{power}}{\emph{a}, \emph{size=None}}{}
Draws samples in {[}0, 1{]} from a power distribution with positive
exponent a - 1.

Also known as the power function distribution.
\begin{description}
\item[{a}] \leavevmode{[}float{]}
parameter, \textgreater{} 0

\item[{size}] \leavevmode{[}tuple of ints{]}\begin{description}
\item[{Output shape.  If the given shape is, e.g., \code{(m, n, k)}, then}] \leavevmode
\code{m * n * k} samples are drawn.

\end{description}

\end{description}
\begin{description}
\item[{samples}] \leavevmode{[}\{ndarray, scalar\}{]}
The returned samples lie in {[}0, 1{]}.

\end{description}
\begin{description}
\item[{ValueError}] \leavevmode
If a\textless{}1.

\end{description}

The probability density function is
\begin{gather}
\begin{split}P(x; a) = ax^{a-1}, 0 \le x \le 1, a>0.\end{split}\notag
\end{gather}
The power function distribution is just the inverse of the Pareto
distribution. It may also be seen as a special case of the Beta
distribution.

It is used, for example, in modeling the over-reporting of insurance
claims.

Draw samples from the distribution:

\begin{Verbatim}[commandchars=\\\{\}]
\PYG{g+gp}{\PYGZgt{}\PYGZgt{}\PYGZgt{} }\PYG{n}{a} \PYG{o}{=} \PYG{l+m+mf}{5.} \PYG{c}{\PYGZsh{} shape}
\PYG{g+gp}{\PYGZgt{}\PYGZgt{}\PYGZgt{} }\PYG{n}{samples} \PYG{o}{=} \PYG{l+m+mi}{1000}
\PYG{g+gp}{\PYGZgt{}\PYGZgt{}\PYGZgt{} }\PYG{n}{s} \PYG{o}{=} \PYG{n}{np}\PYG{o}{.}\PYG{n}{random}\PYG{o}{.}\PYG{n}{power}\PYG{p}{(}\PYG{n}{a}\PYG{p}{,} \PYG{n}{samples}\PYG{p}{)}
\end{Verbatim}

Display the histogram of the samples, along with
the probability density function:

\begin{Verbatim}[commandchars=\\\{\}]
\PYG{g+gp}{\PYGZgt{}\PYGZgt{}\PYGZgt{} }\PYG{k+kn}{import} \PYG{n+nn}{matplotlib.pyplot} \PYG{k+kn}{as} \PYG{n+nn}{plt}
\PYG{g+gp}{\PYGZgt{}\PYGZgt{}\PYGZgt{} }\PYG{n}{count}\PYG{p}{,} \PYG{n}{bins}\PYG{p}{,} \PYG{n}{ignored} \PYG{o}{=} \PYG{n}{plt}\PYG{o}{.}\PYG{n}{hist}\PYG{p}{(}\PYG{n}{s}\PYG{p}{,} \PYG{n}{bins}\PYG{o}{=}\PYG{l+m+mi}{30}\PYG{p}{)}
\PYG{g+gp}{\PYGZgt{}\PYGZgt{}\PYGZgt{} }\PYG{n}{x} \PYG{o}{=} \PYG{n}{np}\PYG{o}{.}\PYG{n}{linspace}\PYG{p}{(}\PYG{l+m+mi}{0}\PYG{p}{,} \PYG{l+m+mi}{1}\PYG{p}{,} \PYG{l+m+mi}{100}\PYG{p}{)}
\PYG{g+gp}{\PYGZgt{}\PYGZgt{}\PYGZgt{} }\PYG{n}{y} \PYG{o}{=} \PYG{n}{a}\PYG{o}{*}\PYG{n}{x}\PYG{o}{*}\PYG{o}{*}\PYG{p}{(}\PYG{n}{a}\PYG{o}{\PYGZhy{}}\PYG{l+m+mf}{1.}\PYG{p}{)}
\PYG{g+gp}{\PYGZgt{}\PYGZgt{}\PYGZgt{} }\PYG{n}{normed\PYGZus{}y} \PYG{o}{=} \PYG{n}{samples}\PYG{o}{*}\PYG{n}{np}\PYG{o}{.}\PYG{n}{diff}\PYG{p}{(}\PYG{n}{bins}\PYG{p}{)}\PYG{p}{[}\PYG{l+m+mi}{0}\PYG{p}{]}\PYG{o}{*}\PYG{n}{y}
\PYG{g+gp}{\PYGZgt{}\PYGZgt{}\PYGZgt{} }\PYG{n}{plt}\PYG{o}{.}\PYG{n}{plot}\PYG{p}{(}\PYG{n}{x}\PYG{p}{,} \PYG{n}{normed\PYGZus{}y}\PYG{p}{)}
\PYG{g+gp}{\PYGZgt{}\PYGZgt{}\PYGZgt{} }\PYG{n}{plt}\PYG{o}{.}\PYG{n}{show}\PYG{p}{(}\PYG{p}{)}
\end{Verbatim}

Compare the power function distribution to the inverse of the Pareto.

\begin{Verbatim}[commandchars=\\\{\}]
\PYG{g+gp}{\PYGZgt{}\PYGZgt{}\PYGZgt{} }\PYG{k+kn}{from} \PYG{n+nn}{scipy} \PYG{k+kn}{import} \PYG{n}{stats}
\PYG{g+gp}{\PYGZgt{}\PYGZgt{}\PYGZgt{} }\PYG{n}{rvs} \PYG{o}{=} \PYG{n}{np}\PYG{o}{.}\PYG{n}{random}\PYG{o}{.}\PYG{n}{power}\PYG{p}{(}\PYG{l+m+mi}{5}\PYG{p}{,} \PYG{l+m+mi}{1000000}\PYG{p}{)}
\PYG{g+gp}{\PYGZgt{}\PYGZgt{}\PYGZgt{} }\PYG{n}{rvsp} \PYG{o}{=} \PYG{n}{np}\PYG{o}{.}\PYG{n}{random}\PYG{o}{.}\PYG{n}{pareto}\PYG{p}{(}\PYG{l+m+mi}{5}\PYG{p}{,} \PYG{l+m+mi}{1000000}\PYG{p}{)}
\PYG{g+gp}{\PYGZgt{}\PYGZgt{}\PYGZgt{} }\PYG{n}{xx} \PYG{o}{=} \PYG{n}{np}\PYG{o}{.}\PYG{n}{linspace}\PYG{p}{(}\PYG{l+m+mi}{0}\PYG{p}{,}\PYG{l+m+mi}{1}\PYG{p}{,}\PYG{l+m+mi}{100}\PYG{p}{)}
\PYG{g+gp}{\PYGZgt{}\PYGZgt{}\PYGZgt{} }\PYG{n}{powpdf} \PYG{o}{=} \PYG{n}{stats}\PYG{o}{.}\PYG{n}{powerlaw}\PYG{o}{.}\PYG{n}{pdf}\PYG{p}{(}\PYG{n}{xx}\PYG{p}{,}\PYG{l+m+mi}{5}\PYG{p}{)}
\end{Verbatim}

\begin{Verbatim}[commandchars=\\\{\}]
\PYG{g+gp}{\PYGZgt{}\PYGZgt{}\PYGZgt{} }\PYG{n}{plt}\PYG{o}{.}\PYG{n}{figure}\PYG{p}{(}\PYG{p}{)}
\PYG{g+gp}{\PYGZgt{}\PYGZgt{}\PYGZgt{} }\PYG{n}{plt}\PYG{o}{.}\PYG{n}{hist}\PYG{p}{(}\PYG{n}{rvs}\PYG{p}{,} \PYG{n}{bins}\PYG{o}{=}\PYG{l+m+mi}{50}\PYG{p}{,} \PYG{n}{normed}\PYG{o}{=}\PYG{n+nb+bp}{True}\PYG{p}{)}
\PYG{g+gp}{\PYGZgt{}\PYGZgt{}\PYGZgt{} }\PYG{n}{plt}\PYG{o}{.}\PYG{n}{plot}\PYG{p}{(}\PYG{n}{xx}\PYG{p}{,}\PYG{n}{powpdf}\PYG{p}{,}\PYG{l+s}{\PYGZsq{}}\PYG{l+s}{r\PYGZhy{}}\PYG{l+s}{\PYGZsq{}}\PYG{p}{)}
\PYG{g+gp}{\PYGZgt{}\PYGZgt{}\PYGZgt{} }\PYG{n}{plt}\PYG{o}{.}\PYG{n}{title}\PYG{p}{(}\PYG{l+s}{\PYGZsq{}}\PYG{l+s}{np.random.power(5)}\PYG{l+s}{\PYGZsq{}}\PYG{p}{)}
\end{Verbatim}

\begin{Verbatim}[commandchars=\\\{\}]
\PYG{g+gp}{\PYGZgt{}\PYGZgt{}\PYGZgt{} }\PYG{n}{plt}\PYG{o}{.}\PYG{n}{figure}\PYG{p}{(}\PYG{p}{)}
\PYG{g+gp}{\PYGZgt{}\PYGZgt{}\PYGZgt{} }\PYG{n}{plt}\PYG{o}{.}\PYG{n}{hist}\PYG{p}{(}\PYG{l+m+mf}{1.}\PYG{o}{/}\PYG{p}{(}\PYG{l+m+mf}{1.}\PYG{o}{+}\PYG{n}{rvsp}\PYG{p}{)}\PYG{p}{,} \PYG{n}{bins}\PYG{o}{=}\PYG{l+m+mi}{50}\PYG{p}{,} \PYG{n}{normed}\PYG{o}{=}\PYG{n+nb+bp}{True}\PYG{p}{)}
\PYG{g+gp}{\PYGZgt{}\PYGZgt{}\PYGZgt{} }\PYG{n}{plt}\PYG{o}{.}\PYG{n}{plot}\PYG{p}{(}\PYG{n}{xx}\PYG{p}{,}\PYG{n}{powpdf}\PYG{p}{,}\PYG{l+s}{\PYGZsq{}}\PYG{l+s}{r\PYGZhy{}}\PYG{l+s}{\PYGZsq{}}\PYG{p}{)}
\PYG{g+gp}{\PYGZgt{}\PYGZgt{}\PYGZgt{} }\PYG{n}{plt}\PYG{o}{.}\PYG{n}{title}\PYG{p}{(}\PYG{l+s}{\PYGZsq{}}\PYG{l+s}{inverse of 1 + np.random.pareto(5)}\PYG{l+s}{\PYGZsq{}}\PYG{p}{)}
\end{Verbatim}

\begin{Verbatim}[commandchars=\\\{\}]
\PYG{g+gp}{\PYGZgt{}\PYGZgt{}\PYGZgt{} }\PYG{n}{plt}\PYG{o}{.}\PYG{n}{figure}\PYG{p}{(}\PYG{p}{)}
\PYG{g+gp}{\PYGZgt{}\PYGZgt{}\PYGZgt{} }\PYG{n}{plt}\PYG{o}{.}\PYG{n}{hist}\PYG{p}{(}\PYG{l+m+mf}{1.}\PYG{o}{/}\PYG{p}{(}\PYG{l+m+mf}{1.}\PYG{o}{+}\PYG{n}{rvsp}\PYG{p}{)}\PYG{p}{,} \PYG{n}{bins}\PYG{o}{=}\PYG{l+m+mi}{50}\PYG{p}{,} \PYG{n}{normed}\PYG{o}{=}\PYG{n+nb+bp}{True}\PYG{p}{)}
\PYG{g+gp}{\PYGZgt{}\PYGZgt{}\PYGZgt{} }\PYG{n}{plt}\PYG{o}{.}\PYG{n}{plot}\PYG{p}{(}\PYG{n}{xx}\PYG{p}{,}\PYG{n}{powpdf}\PYG{p}{,}\PYG{l+s}{\PYGZsq{}}\PYG{l+s}{r\PYGZhy{}}\PYG{l+s}{\PYGZsq{}}\PYG{p}{)}
\PYG{g+gp}{\PYGZgt{}\PYGZgt{}\PYGZgt{} }\PYG{n}{plt}\PYG{o}{.}\PYG{n}{title}\PYG{p}{(}\PYG{l+s}{\PYGZsq{}}\PYG{l+s}{inverse of stats.pareto(5)}\PYG{l+s}{\PYGZsq{}}\PYG{p}{)}
\end{Verbatim}

\end{fulllineitems}

\index{rand() (in module main)}

\begin{fulllineitems}
\phantomsection\label{main:main.rand}\pysiglinewithargsret{\code{main.}\bfcode{rand}}{\emph{d0}, \emph{d1}, \emph{...}, \emph{dn}}{}
Random values in a given shape.

Create an array of the given shape and propagate it with
random samples from a uniform distribution
over \code{{[}0, 1)}.
\begin{description}
\item[{d0, d1, ..., dn}] \leavevmode{[}int, optional{]}
The dimensions of the returned array, should all be positive.
If no argument is given a single Python float is returned.

\end{description}
\begin{description}
\item[{out}] \leavevmode{[}ndarray, shape \code{(d0, d1, ..., dn)}{]}
Random values.

\end{description}

random

This is a convenience function. If you want an interface that
takes a shape-tuple as the first argument, refer to
np.random.random\_sample .

\begin{Verbatim}[commandchars=\\\{\}]
\PYG{g+gp}{\PYGZgt{}\PYGZgt{}\PYGZgt{} }\PYG{n}{np}\PYG{o}{.}\PYG{n}{random}\PYG{o}{.}\PYG{n}{rand}\PYG{p}{(}\PYG{l+m+mi}{3}\PYG{p}{,}\PYG{l+m+mi}{2}\PYG{p}{)}
\PYG{g+go}{array([[ 0.14022471,  0.96360618],  \PYGZsh{}random}
\PYG{g+go}{       [ 0.37601032,  0.25528411],  \PYGZsh{}random}
\PYG{g+go}{       [ 0.49313049,  0.94909878]]) \PYGZsh{}random}
\end{Verbatim}

\end{fulllineitems}

\index{randint() (in module main)}

\begin{fulllineitems}
\phantomsection\label{main:main.randint}\pysiglinewithargsret{\code{main.}\bfcode{randint}}{\emph{low}, \emph{high=None}, \emph{size=None}}{}
Return random integers from \emph{low} (inclusive) to \emph{high} (exclusive).

Return random integers from the ``discrete uniform'' distribution in the
``half-open'' interval {[}\emph{low}, \emph{high}). If \emph{high} is None (the default),
then results are from {[}0, \emph{low}).
\begin{description}
\item[{low}] \leavevmode{[}int{]}
Lowest (signed) integer to be drawn from the distribution (unless
\code{high=None}, in which case this parameter is the \emph{highest} such
integer).

\item[{high}] \leavevmode{[}int, optional{]}
If provided, one above the largest (signed) integer to be drawn
from the distribution (see above for behavior if \code{high=None}).

\item[{size}] \leavevmode{[}int or tuple of ints, optional{]}
Output shape. Default is None, in which case a single int is
returned.

\end{description}
\begin{description}
\item[{out}] \leavevmode{[}int or ndarray of ints{]}
\emph{size}-shaped array of random integers from the appropriate
distribution, or a single such random int if \emph{size} not provided.

\end{description}
\begin{description}
\item[{random.random\_integers}] \leavevmode{[}similar to \emph{randint}, only for the closed{]}
interval {[}\emph{low}, \emph{high}{]}, and 1 is the lowest value if \emph{high} is
omitted. In particular, this other one is the one to use to generate
uniformly distributed discrete non-integers.

\end{description}

\begin{Verbatim}[commandchars=\\\{\}]
\PYG{g+gp}{\PYGZgt{}\PYGZgt{}\PYGZgt{} }\PYG{n}{np}\PYG{o}{.}\PYG{n}{random}\PYG{o}{.}\PYG{n}{randint}\PYG{p}{(}\PYG{l+m+mi}{2}\PYG{p}{,} \PYG{n}{size}\PYG{o}{=}\PYG{l+m+mi}{10}\PYG{p}{)}
\PYG{g+go}{array([1, 0, 0, 0, 1, 1, 0, 0, 1, 0])}
\PYG{g+gp}{\PYGZgt{}\PYGZgt{}\PYGZgt{} }\PYG{n}{np}\PYG{o}{.}\PYG{n}{random}\PYG{o}{.}\PYG{n}{randint}\PYG{p}{(}\PYG{l+m+mi}{1}\PYG{p}{,} \PYG{n}{size}\PYG{o}{=}\PYG{l+m+mi}{10}\PYG{p}{)}
\PYG{g+go}{array([0, 0, 0, 0, 0, 0, 0, 0, 0, 0])}
\end{Verbatim}

Generate a 2 x 4 array of ints between 0 and 4, inclusive:

\begin{Verbatim}[commandchars=\\\{\}]
\PYG{g+gp}{\PYGZgt{}\PYGZgt{}\PYGZgt{} }\PYG{n}{np}\PYG{o}{.}\PYG{n}{random}\PYG{o}{.}\PYG{n}{randint}\PYG{p}{(}\PYG{l+m+mi}{5}\PYG{p}{,} \PYG{n}{size}\PYG{o}{=}\PYG{p}{(}\PYG{l+m+mi}{2}\PYG{p}{,} \PYG{l+m+mi}{4}\PYG{p}{)}\PYG{p}{)}
\PYG{g+go}{array([[4, 0, 2, 1],}
\PYG{g+go}{       [3, 2, 2, 0]])}
\end{Verbatim}

\end{fulllineitems}

\index{randn() (in module main)}

\begin{fulllineitems}
\phantomsection\label{main:main.randn}\pysiglinewithargsret{\code{main.}\bfcode{randn}}{\emph{d0}, \emph{d1}, \emph{...}, \emph{dn}}{}
Return a sample (or samples) from the ``standard normal'' distribution.

If positive, int\_like or int-convertible arguments are provided,
\emph{randn} generates an array of shape \code{(d0, d1, ..., dn)}, filled
with random floats sampled from a univariate ``normal'' (Gaussian)
distribution of mean 0 and variance 1 (if any of the \(d_i\) are
floats, they are first converted to integers by truncation). A single
float randomly sampled from the distribution is returned if no
argument is provided.

This is a convenience function.  If you want an interface that takes a
tuple as the first argument, use \emph{numpy.random.standard\_normal} instead.
\begin{description}
\item[{d0, d1, ..., dn}] \leavevmode{[}int, optional{]}
The dimensions of the returned array, should be all positive.
If no argument is given a single Python float is returned.

\end{description}
\begin{description}
\item[{Z}] \leavevmode{[}ndarray or float{]}
A \code{(d0, d1, ..., dn)}-shaped array of floating-point samples from
the standard normal distribution, or a single such float if
no parameters were supplied.

\end{description}

random.standard\_normal : Similar, but takes a tuple as its argument.

For random samples from \(N(\mu, \sigma^2)\), use:

\code{sigma * np.random.randn(...) + mu}

\begin{Verbatim}[commandchars=\\\{\}]
\PYG{g+gp}{\PYGZgt{}\PYGZgt{}\PYGZgt{} }\PYG{n}{np}\PYG{o}{.}\PYG{n}{random}\PYG{o}{.}\PYG{n}{randn}\PYG{p}{(}\PYG{p}{)}
\PYG{g+go}{2.1923875335537315 \PYGZsh{}random}
\end{Verbatim}

Two-by-four array of samples from N(3, 6.25):

\begin{Verbatim}[commandchars=\\\{\}]
\PYG{g+gp}{\PYGZgt{}\PYGZgt{}\PYGZgt{} }\PYG{l+m+mf}{2.5} \PYG{o}{*} \PYG{n}{np}\PYG{o}{.}\PYG{n}{random}\PYG{o}{.}\PYG{n}{randn}\PYG{p}{(}\PYG{l+m+mi}{2}\PYG{p}{,} \PYG{l+m+mi}{4}\PYG{p}{)} \PYG{o}{+} \PYG{l+m+mi}{3}
\PYG{g+go}{array([[\PYGZhy{}4.49401501,  4.00950034, \PYGZhy{}1.81814867,  7.29718677],  \PYGZsh{}random}
\PYG{g+go}{       [ 0.39924804,  4.68456316,  4.99394529,  4.84057254]]) \PYGZsh{}random}
\end{Verbatim}

\end{fulllineitems}

\index{random() (in module main)}

\begin{fulllineitems}
\phantomsection\label{main:main.random}\pysiglinewithargsret{\code{main.}\bfcode{random}}{}{}
random\_sample(size=None)

Return random floats in the half-open interval {[}0.0, 1.0).

Results are from the ``continuous uniform'' distribution over the
stated interval.  To sample \(Unif[a, b), b > a\) multiply
the output of \emph{random\_sample} by \emph{(b-a)} and add \emph{a}:

\begin{Verbatim}[commandchars=\\\{\}]
\PYG{p}{(}\PYG{n}{b} \PYG{o}{\PYGZhy{}} \PYG{n}{a}\PYG{p}{)} \PYG{o}{*} \PYG{n}{random\PYGZus{}sample}\PYG{p}{(}\PYG{p}{)} \PYG{o}{+} \PYG{n}{a}
\end{Verbatim}
\begin{description}
\item[{size}] \leavevmode{[}int or tuple of ints, optional{]}
Defines the shape of the returned array of random floats. If None
(the default), returns a single float.

\end{description}
\begin{description}
\item[{out}] \leavevmode{[}float or ndarray of floats{]}
Array of random floats of shape \emph{size} (unless \code{size=None}, in which
case a single float is returned).

\end{description}

\begin{Verbatim}[commandchars=\\\{\}]
\PYG{g+gp}{\PYGZgt{}\PYGZgt{}\PYGZgt{} }\PYG{n}{np}\PYG{o}{.}\PYG{n}{random}\PYG{o}{.}\PYG{n}{random\PYGZus{}sample}\PYG{p}{(}\PYG{p}{)}
\PYG{g+go}{0.47108547995356098}
\PYG{g+gp}{\PYGZgt{}\PYGZgt{}\PYGZgt{} }\PYG{n+nb}{type}\PYG{p}{(}\PYG{n}{np}\PYG{o}{.}\PYG{n}{random}\PYG{o}{.}\PYG{n}{random\PYGZus{}sample}\PYG{p}{(}\PYG{p}{)}\PYG{p}{)}
\PYG{g+go}{\PYGZlt{}type \PYGZsq{}float\PYGZsq{}\PYGZgt{}}
\PYG{g+gp}{\PYGZgt{}\PYGZgt{}\PYGZgt{} }\PYG{n}{np}\PYG{o}{.}\PYG{n}{random}\PYG{o}{.}\PYG{n}{random\PYGZus{}sample}\PYG{p}{(}\PYG{p}{(}\PYG{l+m+mi}{5}\PYG{p}{,}\PYG{p}{)}\PYG{p}{)}
\PYG{g+go}{array([ 0.30220482,  0.86820401,  0.1654503 ,  0.11659149,  0.54323428])}
\end{Verbatim}

Three-by-two array of random numbers from {[}-5, 0):

\begin{Verbatim}[commandchars=\\\{\}]
\PYG{g+gp}{\PYGZgt{}\PYGZgt{}\PYGZgt{} }\PYG{l+m+mi}{5} \PYG{o}{*} \PYG{n}{np}\PYG{o}{.}\PYG{n}{random}\PYG{o}{.}\PYG{n}{random\PYGZus{}sample}\PYG{p}{(}\PYG{p}{(}\PYG{l+m+mi}{3}\PYG{p}{,} \PYG{l+m+mi}{2}\PYG{p}{)}\PYG{p}{)} \PYG{o}{\PYGZhy{}} \PYG{l+m+mi}{5}
\PYG{g+go}{array([[\PYGZhy{}3.99149989, \PYGZhy{}0.52338984],}
\PYG{g+go}{       [\PYGZhy{}2.99091858, \PYGZhy{}0.79479508],}
\PYG{g+go}{       [\PYGZhy{}1.23204345, \PYGZhy{}1.75224494]])}
\end{Verbatim}

\end{fulllineitems}

\index{random\_integers() (in module main)}

\begin{fulllineitems}
\phantomsection\label{main:main.random_integers}\pysiglinewithargsret{\code{main.}\bfcode{random\_integers}}{\emph{low}, \emph{high=None}, \emph{size=None}}{}
Return random integers between \emph{low} and \emph{high}, inclusive.

Return random integers from the ``discrete uniform'' distribution in the
closed interval {[}\emph{low}, \emph{high}{]}.  If \emph{high} is None (the default),
then results are from {[}1, \emph{low}{]}.
\begin{description}
\item[{low}] \leavevmode{[}int{]}
Lowest (signed) integer to be drawn from the distribution (unless
\code{high=None}, in which case this parameter is the \emph{highest} such
integer).

\item[{high}] \leavevmode{[}int, optional{]}
If provided, the largest (signed) integer to be drawn from the
distribution (see above for behavior if \code{high=None}).

\item[{size}] \leavevmode{[}int or tuple of ints, optional{]}
Output shape. Default is None, in which case a single int is returned.

\end{description}
\begin{description}
\item[{out}] \leavevmode{[}int or ndarray of ints{]}
\emph{size}-shaped array of random integers from the appropriate
distribution, or a single such random int if \emph{size} not provided.

\end{description}
\begin{description}
\item[{random.randint}] \leavevmode{[}Similar to \emph{random\_integers}, only for the half-open{]}
interval {[}\emph{low}, \emph{high}), and 0 is the lowest value if \emph{high} is
omitted.

\end{description}

To sample from N evenly spaced floating-point numbers between a and b,
use:

\begin{Verbatim}[commandchars=\\\{\}]
\PYG{n}{a} \PYG{o}{+} \PYG{p}{(}\PYG{n}{b} \PYG{o}{\PYGZhy{}} \PYG{n}{a}\PYG{p}{)} \PYG{o}{*} \PYG{p}{(}\PYG{n}{np}\PYG{o}{.}\PYG{n}{random}\PYG{o}{.}\PYG{n}{random\PYGZus{}integers}\PYG{p}{(}\PYG{n}{N}\PYG{p}{)} \PYG{o}{\PYGZhy{}} \PYG{l+m+mi}{1}\PYG{p}{)} \PYG{o}{/} \PYG{p}{(}\PYG{n}{N} \PYG{o}{\PYGZhy{}} \PYG{l+m+mf}{1.}\PYG{p}{)}
\end{Verbatim}

\begin{Verbatim}[commandchars=\\\{\}]
\PYG{g+gp}{\PYGZgt{}\PYGZgt{}\PYGZgt{} }\PYG{n}{np}\PYG{o}{.}\PYG{n}{random}\PYG{o}{.}\PYG{n}{random\PYGZus{}integers}\PYG{p}{(}\PYG{l+m+mi}{5}\PYG{p}{)}
\PYG{g+go}{4}
\PYG{g+gp}{\PYGZgt{}\PYGZgt{}\PYGZgt{} }\PYG{n+nb}{type}\PYG{p}{(}\PYG{n}{np}\PYG{o}{.}\PYG{n}{random}\PYG{o}{.}\PYG{n}{random\PYGZus{}integers}\PYG{p}{(}\PYG{l+m+mi}{5}\PYG{p}{)}\PYG{p}{)}
\PYG{g+go}{\PYGZlt{}type \PYGZsq{}int\PYGZsq{}\PYGZgt{}}
\PYG{g+gp}{\PYGZgt{}\PYGZgt{}\PYGZgt{} }\PYG{n}{np}\PYG{o}{.}\PYG{n}{random}\PYG{o}{.}\PYG{n}{random\PYGZus{}integers}\PYG{p}{(}\PYG{l+m+mi}{5}\PYG{p}{,} \PYG{n}{size}\PYG{o}{=}\PYG{p}{(}\PYG{l+m+mf}{3.}\PYG{p}{,}\PYG{l+m+mf}{2.}\PYG{p}{)}\PYG{p}{)}
\PYG{g+go}{array([[5, 4],}
\PYG{g+go}{       [3, 3],}
\PYG{g+go}{       [4, 5]])}
\end{Verbatim}

Choose five random numbers from the set of five evenly-spaced
numbers between 0 and 2.5, inclusive (\emph{i.e.}, from the set
\({0, 5/8, 10/8, 15/8, 20/8}\)):

\begin{Verbatim}[commandchars=\\\{\}]
\PYG{g+gp}{\PYGZgt{}\PYGZgt{}\PYGZgt{} }\PYG{l+m+mf}{2.5} \PYG{o}{*} \PYG{p}{(}\PYG{n}{np}\PYG{o}{.}\PYG{n}{random}\PYG{o}{.}\PYG{n}{random\PYGZus{}integers}\PYG{p}{(}\PYG{l+m+mi}{5}\PYG{p}{,} \PYG{n}{size}\PYG{o}{=}\PYG{p}{(}\PYG{l+m+mi}{5}\PYG{p}{,}\PYG{p}{)}\PYG{p}{)} \PYG{o}{\PYGZhy{}} \PYG{l+m+mi}{1}\PYG{p}{)} \PYG{o}{/} \PYG{l+m+mf}{4.}
\PYG{g+go}{array([ 0.625,  1.25 ,  0.625,  0.625,  2.5  ])}
\end{Verbatim}

Roll two six sided dice 1000 times and sum the results:

\begin{Verbatim}[commandchars=\\\{\}]
\PYG{g+gp}{\PYGZgt{}\PYGZgt{}\PYGZgt{} }\PYG{n}{d1} \PYG{o}{=} \PYG{n}{np}\PYG{o}{.}\PYG{n}{random}\PYG{o}{.}\PYG{n}{random\PYGZus{}integers}\PYG{p}{(}\PYG{l+m+mi}{1}\PYG{p}{,} \PYG{l+m+mi}{6}\PYG{p}{,} \PYG{l+m+mi}{1000}\PYG{p}{)}
\PYG{g+gp}{\PYGZgt{}\PYGZgt{}\PYGZgt{} }\PYG{n}{d2} \PYG{o}{=} \PYG{n}{np}\PYG{o}{.}\PYG{n}{random}\PYG{o}{.}\PYG{n}{random\PYGZus{}integers}\PYG{p}{(}\PYG{l+m+mi}{1}\PYG{p}{,} \PYG{l+m+mi}{6}\PYG{p}{,} \PYG{l+m+mi}{1000}\PYG{p}{)}
\PYG{g+gp}{\PYGZgt{}\PYGZgt{}\PYGZgt{} }\PYG{n}{dsums} \PYG{o}{=} \PYG{n}{d1} \PYG{o}{+} \PYG{n}{d2}
\end{Verbatim}

Display results as a histogram:

\begin{Verbatim}[commandchars=\\\{\}]
\PYG{g+gp}{\PYGZgt{}\PYGZgt{}\PYGZgt{} }\PYG{k+kn}{import} \PYG{n+nn}{matplotlib.pyplot} \PYG{k+kn}{as} \PYG{n+nn}{plt}
\PYG{g+gp}{\PYGZgt{}\PYGZgt{}\PYGZgt{} }\PYG{n}{count}\PYG{p}{,} \PYG{n}{bins}\PYG{p}{,} \PYG{n}{ignored} \PYG{o}{=} \PYG{n}{plt}\PYG{o}{.}\PYG{n}{hist}\PYG{p}{(}\PYG{n}{dsums}\PYG{p}{,} \PYG{l+m+mi}{11}\PYG{p}{,} \PYG{n}{normed}\PYG{o}{=}\PYG{n+nb+bp}{True}\PYG{p}{)}
\PYG{g+gp}{\PYGZgt{}\PYGZgt{}\PYGZgt{} }\PYG{n}{plt}\PYG{o}{.}\PYG{n}{show}\PYG{p}{(}\PYG{p}{)}
\end{Verbatim}

\end{fulllineitems}

\index{random\_sample() (in module main)}

\begin{fulllineitems}
\phantomsection\label{main:main.random_sample}\pysiglinewithargsret{\code{main.}\bfcode{random\_sample}}{\emph{size=None}}{}
Return random floats in the half-open interval {[}0.0, 1.0).

Results are from the ``continuous uniform'' distribution over the
stated interval.  To sample \(Unif[a, b), b > a\) multiply
the output of \emph{random\_sample} by \emph{(b-a)} and add \emph{a}:

\begin{Verbatim}[commandchars=\\\{\}]
\PYG{p}{(}\PYG{n}{b} \PYG{o}{\PYGZhy{}} \PYG{n}{a}\PYG{p}{)} \PYG{o}{*} \PYG{n}{random\PYGZus{}sample}\PYG{p}{(}\PYG{p}{)} \PYG{o}{+} \PYG{n}{a}
\end{Verbatim}
\begin{description}
\item[{size}] \leavevmode{[}int or tuple of ints, optional{]}
Defines the shape of the returned array of random floats. If None
(the default), returns a single float.

\end{description}
\begin{description}
\item[{out}] \leavevmode{[}float or ndarray of floats{]}
Array of random floats of shape \emph{size} (unless \code{size=None}, in which
case a single float is returned).

\end{description}

\begin{Verbatim}[commandchars=\\\{\}]
\PYG{g+gp}{\PYGZgt{}\PYGZgt{}\PYGZgt{} }\PYG{n}{np}\PYG{o}{.}\PYG{n}{random}\PYG{o}{.}\PYG{n}{random\PYGZus{}sample}\PYG{p}{(}\PYG{p}{)}
\PYG{g+go}{0.47108547995356098}
\PYG{g+gp}{\PYGZgt{}\PYGZgt{}\PYGZgt{} }\PYG{n+nb}{type}\PYG{p}{(}\PYG{n}{np}\PYG{o}{.}\PYG{n}{random}\PYG{o}{.}\PYG{n}{random\PYGZus{}sample}\PYG{p}{(}\PYG{p}{)}\PYG{p}{)}
\PYG{g+go}{\PYGZlt{}type \PYGZsq{}float\PYGZsq{}\PYGZgt{}}
\PYG{g+gp}{\PYGZgt{}\PYGZgt{}\PYGZgt{} }\PYG{n}{np}\PYG{o}{.}\PYG{n}{random}\PYG{o}{.}\PYG{n}{random\PYGZus{}sample}\PYG{p}{(}\PYG{p}{(}\PYG{l+m+mi}{5}\PYG{p}{,}\PYG{p}{)}\PYG{p}{)}
\PYG{g+go}{array([ 0.30220482,  0.86820401,  0.1654503 ,  0.11659149,  0.54323428])}
\end{Verbatim}

Three-by-two array of random numbers from {[}-5, 0):

\begin{Verbatim}[commandchars=\\\{\}]
\PYG{g+gp}{\PYGZgt{}\PYGZgt{}\PYGZgt{} }\PYG{l+m+mi}{5} \PYG{o}{*} \PYG{n}{np}\PYG{o}{.}\PYG{n}{random}\PYG{o}{.}\PYG{n}{random\PYGZus{}sample}\PYG{p}{(}\PYG{p}{(}\PYG{l+m+mi}{3}\PYG{p}{,} \PYG{l+m+mi}{2}\PYG{p}{)}\PYG{p}{)} \PYG{o}{\PYGZhy{}} \PYG{l+m+mi}{5}
\PYG{g+go}{array([[\PYGZhy{}3.99149989, \PYGZhy{}0.52338984],}
\PYG{g+go}{       [\PYGZhy{}2.99091858, \PYGZhy{}0.79479508],}
\PYG{g+go}{       [\PYGZhy{}1.23204345, \PYGZhy{}1.75224494]])}
\end{Verbatim}

\end{fulllineitems}

\index{ranf() (in module main)}

\begin{fulllineitems}
\phantomsection\label{main:main.ranf}\pysiglinewithargsret{\code{main.}\bfcode{ranf}}{}{}
random\_sample(size=None)

Return random floats in the half-open interval {[}0.0, 1.0).

Results are from the ``continuous uniform'' distribution over the
stated interval.  To sample \(Unif[a, b), b > a\) multiply
the output of \emph{random\_sample} by \emph{(b-a)} and add \emph{a}:

\begin{Verbatim}[commandchars=\\\{\}]
\PYG{p}{(}\PYG{n}{b} \PYG{o}{\PYGZhy{}} \PYG{n}{a}\PYG{p}{)} \PYG{o}{*} \PYG{n}{random\PYGZus{}sample}\PYG{p}{(}\PYG{p}{)} \PYG{o}{+} \PYG{n}{a}
\end{Verbatim}
\begin{description}
\item[{size}] \leavevmode{[}int or tuple of ints, optional{]}
Defines the shape of the returned array of random floats. If None
(the default), returns a single float.

\end{description}
\begin{description}
\item[{out}] \leavevmode{[}float or ndarray of floats{]}
Array of random floats of shape \emph{size} (unless \code{size=None}, in which
case a single float is returned).

\end{description}

\begin{Verbatim}[commandchars=\\\{\}]
\PYG{g+gp}{\PYGZgt{}\PYGZgt{}\PYGZgt{} }\PYG{n}{np}\PYG{o}{.}\PYG{n}{random}\PYG{o}{.}\PYG{n}{random\PYGZus{}sample}\PYG{p}{(}\PYG{p}{)}
\PYG{g+go}{0.47108547995356098}
\PYG{g+gp}{\PYGZgt{}\PYGZgt{}\PYGZgt{} }\PYG{n+nb}{type}\PYG{p}{(}\PYG{n}{np}\PYG{o}{.}\PYG{n}{random}\PYG{o}{.}\PYG{n}{random\PYGZus{}sample}\PYG{p}{(}\PYG{p}{)}\PYG{p}{)}
\PYG{g+go}{\PYGZlt{}type \PYGZsq{}float\PYGZsq{}\PYGZgt{}}
\PYG{g+gp}{\PYGZgt{}\PYGZgt{}\PYGZgt{} }\PYG{n}{np}\PYG{o}{.}\PYG{n}{random}\PYG{o}{.}\PYG{n}{random\PYGZus{}sample}\PYG{p}{(}\PYG{p}{(}\PYG{l+m+mi}{5}\PYG{p}{,}\PYG{p}{)}\PYG{p}{)}
\PYG{g+go}{array([ 0.30220482,  0.86820401,  0.1654503 ,  0.11659149,  0.54323428])}
\end{Verbatim}

Three-by-two array of random numbers from {[}-5, 0):

\begin{Verbatim}[commandchars=\\\{\}]
\PYG{g+gp}{\PYGZgt{}\PYGZgt{}\PYGZgt{} }\PYG{l+m+mi}{5} \PYG{o}{*} \PYG{n}{np}\PYG{o}{.}\PYG{n}{random}\PYG{o}{.}\PYG{n}{random\PYGZus{}sample}\PYG{p}{(}\PYG{p}{(}\PYG{l+m+mi}{3}\PYG{p}{,} \PYG{l+m+mi}{2}\PYG{p}{)}\PYG{p}{)} \PYG{o}{\PYGZhy{}} \PYG{l+m+mi}{5}
\PYG{g+go}{array([[\PYGZhy{}3.99149989, \PYGZhy{}0.52338984],}
\PYG{g+go}{       [\PYGZhy{}2.99091858, \PYGZhy{}0.79479508],}
\PYG{g+go}{       [\PYGZhy{}1.23204345, \PYGZhy{}1.75224494]])}
\end{Verbatim}

\end{fulllineitems}

\index{rayleigh() (in module main)}

\begin{fulllineitems}
\phantomsection\label{main:main.rayleigh}\pysiglinewithargsret{\code{main.}\bfcode{rayleigh}}{\emph{scale=1.0}, \emph{size=None}}{}
Draw samples from a Rayleigh distribution.

The \(\chi\) and Weibull distributions are generalizations of the
Rayleigh.
\begin{description}
\item[{scale}] \leavevmode{[}scalar{]}
Scale, also equals the mode. Should be \textgreater{}= 0.

\item[{size}] \leavevmode{[}int or tuple of ints, optional{]}
Shape of the output. Default is None, in which case a single
value is returned.

\end{description}

The probability density function for the Rayleigh distribution is
\begin{gather}
\begin{split}P(x;scale) = \frac{x}{scale^2}e^{\frac{-x^2}{2 \cdotp scale^2}}\end{split}\notag
\end{gather}
The Rayleigh distribution arises if the wind speed and wind direction are
both gaussian variables, then the vector wind velocity forms a Rayleigh
distribution. The Rayleigh distribution is used to model the expected
output from wind turbines.

Draw values from the distribution and plot the histogram

\begin{Verbatim}[commandchars=\\\{\}]
\PYG{g+gp}{\PYGZgt{}\PYGZgt{}\PYGZgt{} }\PYG{n}{values} \PYG{o}{=} \PYG{n}{hist}\PYG{p}{(}\PYG{n}{np}\PYG{o}{.}\PYG{n}{random}\PYG{o}{.}\PYG{n}{rayleigh}\PYG{p}{(}\PYG{l+m+mi}{3}\PYG{p}{,} \PYG{l+m+mi}{100000}\PYG{p}{)}\PYG{p}{,} \PYG{n}{bins}\PYG{o}{=}\PYG{l+m+mi}{200}\PYG{p}{,} \PYG{n}{normed}\PYG{o}{=}\PYG{n+nb+bp}{True}\PYG{p}{)}
\end{Verbatim}

Wave heights tend to follow a Rayleigh distribution. If the mean wave
height is 1 meter, what fraction of waves are likely to be larger than 3
meters?

\begin{Verbatim}[commandchars=\\\{\}]
\PYG{g+gp}{\PYGZgt{}\PYGZgt{}\PYGZgt{} }\PYG{n}{meanvalue} \PYG{o}{=} \PYG{l+m+mi}{1}
\PYG{g+gp}{\PYGZgt{}\PYGZgt{}\PYGZgt{} }\PYG{n}{modevalue} \PYG{o}{=} \PYG{n}{np}\PYG{o}{.}\PYG{n}{sqrt}\PYG{p}{(}\PYG{l+m+mi}{2} \PYG{o}{/} \PYG{n}{np}\PYG{o}{.}\PYG{n}{pi}\PYG{p}{)} \PYG{o}{*} \PYG{n}{meanvalue}
\PYG{g+gp}{\PYGZgt{}\PYGZgt{}\PYGZgt{} }\PYG{n}{s} \PYG{o}{=} \PYG{n}{np}\PYG{o}{.}\PYG{n}{random}\PYG{o}{.}\PYG{n}{rayleigh}\PYG{p}{(}\PYG{n}{modevalue}\PYG{p}{,} \PYG{l+m+mi}{1000000}\PYG{p}{)}
\end{Verbatim}

The percentage of waves larger than 3 meters is:

\begin{Verbatim}[commandchars=\\\{\}]
\PYG{g+gp}{\PYGZgt{}\PYGZgt{}\PYGZgt{} }\PYG{l+m+mf}{100.}\PYG{o}{*}\PYG{n+nb}{sum}\PYG{p}{(}\PYG{n}{s}\PYG{o}{\PYGZgt{}}\PYG{l+m+mi}{3}\PYG{p}{)}\PYG{o}{/}\PYG{l+m+mf}{1000000.}
\PYG{g+go}{0.087300000000000003}
\end{Verbatim}

\end{fulllineitems}

\index{sample() (in module main)}

\begin{fulllineitems}
\phantomsection\label{main:main.sample}\pysiglinewithargsret{\code{main.}\bfcode{sample}}{}{}
random\_sample(size=None)

Return random floats in the half-open interval {[}0.0, 1.0).

Results are from the ``continuous uniform'' distribution over the
stated interval.  To sample \(Unif[a, b), b > a\) multiply
the output of \emph{random\_sample} by \emph{(b-a)} and add \emph{a}:

\begin{Verbatim}[commandchars=\\\{\}]
\PYG{p}{(}\PYG{n}{b} \PYG{o}{\PYGZhy{}} \PYG{n}{a}\PYG{p}{)} \PYG{o}{*} \PYG{n}{random\PYGZus{}sample}\PYG{p}{(}\PYG{p}{)} \PYG{o}{+} \PYG{n}{a}
\end{Verbatim}
\begin{description}
\item[{size}] \leavevmode{[}int or tuple of ints, optional{]}
Defines the shape of the returned array of random floats. If None
(the default), returns a single float.

\end{description}
\begin{description}
\item[{out}] \leavevmode{[}float or ndarray of floats{]}
Array of random floats of shape \emph{size} (unless \code{size=None}, in which
case a single float is returned).

\end{description}

\begin{Verbatim}[commandchars=\\\{\}]
\PYG{g+gp}{\PYGZgt{}\PYGZgt{}\PYGZgt{} }\PYG{n}{np}\PYG{o}{.}\PYG{n}{random}\PYG{o}{.}\PYG{n}{random\PYGZus{}sample}\PYG{p}{(}\PYG{p}{)}
\PYG{g+go}{0.47108547995356098}
\PYG{g+gp}{\PYGZgt{}\PYGZgt{}\PYGZgt{} }\PYG{n+nb}{type}\PYG{p}{(}\PYG{n}{np}\PYG{o}{.}\PYG{n}{random}\PYG{o}{.}\PYG{n}{random\PYGZus{}sample}\PYG{p}{(}\PYG{p}{)}\PYG{p}{)}
\PYG{g+go}{\PYGZlt{}type \PYGZsq{}float\PYGZsq{}\PYGZgt{}}
\PYG{g+gp}{\PYGZgt{}\PYGZgt{}\PYGZgt{} }\PYG{n}{np}\PYG{o}{.}\PYG{n}{random}\PYG{o}{.}\PYG{n}{random\PYGZus{}sample}\PYG{p}{(}\PYG{p}{(}\PYG{l+m+mi}{5}\PYG{p}{,}\PYG{p}{)}\PYG{p}{)}
\PYG{g+go}{array([ 0.30220482,  0.86820401,  0.1654503 ,  0.11659149,  0.54323428])}
\end{Verbatim}

Three-by-two array of random numbers from {[}-5, 0):

\begin{Verbatim}[commandchars=\\\{\}]
\PYG{g+gp}{\PYGZgt{}\PYGZgt{}\PYGZgt{} }\PYG{l+m+mi}{5} \PYG{o}{*} \PYG{n}{np}\PYG{o}{.}\PYG{n}{random}\PYG{o}{.}\PYG{n}{random\PYGZus{}sample}\PYG{p}{(}\PYG{p}{(}\PYG{l+m+mi}{3}\PYG{p}{,} \PYG{l+m+mi}{2}\PYG{p}{)}\PYG{p}{)} \PYG{o}{\PYGZhy{}} \PYG{l+m+mi}{5}
\PYG{g+go}{array([[\PYGZhy{}3.99149989, \PYGZhy{}0.52338984],}
\PYG{g+go}{       [\PYGZhy{}2.99091858, \PYGZhy{}0.79479508],}
\PYG{g+go}{       [\PYGZhy{}1.23204345, \PYGZhy{}1.75224494]])}
\end{Verbatim}

\end{fulllineitems}

\index{seed() (in module main)}

\begin{fulllineitems}
\phantomsection\label{main:main.seed}\pysiglinewithargsret{\code{main.}\bfcode{seed}}{\emph{seed=None}}{}
Seed the generator.

This method is called when \emph{RandomState} is initialized. It can be
called again to re-seed the generator. For details, see \emph{RandomState}.
\begin{description}
\item[{seed}] \leavevmode{[}int or array\_like, optional{]}
Seed for \emph{RandomState}.

\end{description}

RandomState

\end{fulllineitems}

\index{set\_state() (in module main)}

\begin{fulllineitems}
\phantomsection\label{main:main.set_state}\pysiglinewithargsret{\code{main.}\bfcode{set\_state}}{\emph{state}}{}
Set the internal state of the generator from a tuple.

For use if one has reason to manually (re-)set the internal state of the
``Mersenne Twister''{\color{red}\bfseries{}{[}1{]}\_} pseudo-random number generating algorithm.
\begin{description}
\item[{state}] \leavevmode{[}tuple(str, ndarray of 624 uints, int, int, float){]}
The \emph{state} tuple has the following items:
\begin{enumerate}
\item {} 
the string `MT19937', specifying the Mersenne Twister algorithm.

\item {} 
a 1-D array of 624 unsigned integers \code{keys}.

\item {} 
an integer \code{pos}.

\item {} 
an integer \code{has\_gauss}.

\item {} 
a float \code{cached\_gaussian}.

\end{enumerate}

\end{description}
\begin{description}
\item[{out}] \leavevmode{[}None{]}
Returns `None' on success.

\end{description}

get\_state

\emph{set\_state} and \emph{get\_state} are not needed to work with any of the
random distributions in NumPy. If the internal state is manually altered,
the user should know exactly what he/she is doing.

For backwards compatibility, the form (str, array of 624 uints, int) is
also accepted although it is missing some information about the cached
Gaussian value: \code{state = ('MT19937', keys, pos)}.

\end{fulllineitems}

\index{shuffle() (in module main)}

\begin{fulllineitems}
\phantomsection\label{main:main.shuffle}\pysiglinewithargsret{\code{main.}\bfcode{shuffle}}{\emph{x}}{}
Modify a sequence in-place by shuffling its contents.
\begin{description}
\item[{x}] \leavevmode{[}array\_like{]}
The array or list to be shuffled.

\end{description}

None

\begin{Verbatim}[commandchars=\\\{\}]
\PYG{g+gp}{\PYGZgt{}\PYGZgt{}\PYGZgt{} }\PYG{n}{arr} \PYG{o}{=} \PYG{n}{np}\PYG{o}{.}\PYG{n}{arange}\PYG{p}{(}\PYG{l+m+mi}{10}\PYG{p}{)}
\PYG{g+gp}{\PYGZgt{}\PYGZgt{}\PYGZgt{} }\PYG{n}{np}\PYG{o}{.}\PYG{n}{random}\PYG{o}{.}\PYG{n}{shuffle}\PYG{p}{(}\PYG{n}{arr}\PYG{p}{)}
\PYG{g+gp}{\PYGZgt{}\PYGZgt{}\PYGZgt{} }\PYG{n}{arr}
\PYG{g+go}{[1 7 5 2 9 4 3 6 0 8]}
\end{Verbatim}

This function only shuffles the array along the first index of a
multi-dimensional array:

\begin{Verbatim}[commandchars=\\\{\}]
\PYG{g+gp}{\PYGZgt{}\PYGZgt{}\PYGZgt{} }\PYG{n}{arr} \PYG{o}{=} \PYG{n}{np}\PYG{o}{.}\PYG{n}{arange}\PYG{p}{(}\PYG{l+m+mi}{9}\PYG{p}{)}\PYG{o}{.}\PYG{n}{reshape}\PYG{p}{(}\PYG{p}{(}\PYG{l+m+mi}{3}\PYG{p}{,} \PYG{l+m+mi}{3}\PYG{p}{)}\PYG{p}{)}
\PYG{g+gp}{\PYGZgt{}\PYGZgt{}\PYGZgt{} }\PYG{n}{np}\PYG{o}{.}\PYG{n}{random}\PYG{o}{.}\PYG{n}{shuffle}\PYG{p}{(}\PYG{n}{arr}\PYG{p}{)}
\PYG{g+gp}{\PYGZgt{}\PYGZgt{}\PYGZgt{} }\PYG{n}{arr}
\PYG{g+go}{array([[3, 4, 5],}
\PYG{g+go}{       [6, 7, 8],}
\PYG{g+go}{       [0, 1, 2]])}
\end{Verbatim}

\end{fulllineitems}

\index{standard\_cauchy() (in module main)}

\begin{fulllineitems}
\phantomsection\label{main:main.standard_cauchy}\pysiglinewithargsret{\code{main.}\bfcode{standard\_cauchy}}{\emph{size=None}}{}
Standard Cauchy distribution with mode = 0.

Also known as the Lorentz distribution.
\begin{description}
\item[{size}] \leavevmode{[}int or tuple of ints{]}
Shape of the output.

\end{description}
\begin{description}
\item[{samples}] \leavevmode{[}ndarray or scalar{]}
The drawn samples.

\end{description}

The probability density function for the full Cauchy distribution is
\begin{gather}
\begin{split}P(x; x_0, \gamma) = \frac{1}{\pi \gamma \bigl[ 1+
(\frac{x-x_0}{\gamma})^2 \bigr] }\end{split}\notag
\end{gather}
and the Standard Cauchy distribution just sets \(x_0=0\) and
\(\gamma=1\)

The Cauchy distribution arises in the solution to the driven harmonic
oscillator problem, and also describes spectral line broadening. It
also describes the distribution of values at which a line tilted at
a random angle will cut the x axis.

When studying hypothesis tests that assume normality, seeing how the
tests perform on data from a Cauchy distribution is a good indicator of
their sensitivity to a heavy-tailed distribution, since the Cauchy looks
very much like a Gaussian distribution, but with heavier tails.

Draw samples and plot the distribution:

\begin{Verbatim}[commandchars=\\\{\}]
\PYG{g+gp}{\PYGZgt{}\PYGZgt{}\PYGZgt{} }\PYG{n}{s} \PYG{o}{=} \PYG{n}{np}\PYG{o}{.}\PYG{n}{random}\PYG{o}{.}\PYG{n}{standard\PYGZus{}cauchy}\PYG{p}{(}\PYG{l+m+mi}{1000000}\PYG{p}{)}
\PYG{g+gp}{\PYGZgt{}\PYGZgt{}\PYGZgt{} }\PYG{n}{s} \PYG{o}{=} \PYG{n}{s}\PYG{p}{[}\PYG{p}{(}\PYG{n}{s}\PYG{o}{\PYGZgt{}}\PYG{o}{\PYGZhy{}}\PYG{l+m+mi}{25}\PYG{p}{)} \PYG{o}{\PYGZam{}} \PYG{p}{(}\PYG{n}{s}\PYG{o}{\PYGZlt{}}\PYG{l+m+mi}{25}\PYG{p}{)}\PYG{p}{]}  \PYG{c}{\PYGZsh{} truncate distribution so it plots well}
\PYG{g+gp}{\PYGZgt{}\PYGZgt{}\PYGZgt{} }\PYG{n}{plt}\PYG{o}{.}\PYG{n}{hist}\PYG{p}{(}\PYG{n}{s}\PYG{p}{,} \PYG{n}{bins}\PYG{o}{=}\PYG{l+m+mi}{100}\PYG{p}{)}
\PYG{g+gp}{\PYGZgt{}\PYGZgt{}\PYGZgt{} }\PYG{n}{plt}\PYG{o}{.}\PYG{n}{show}\PYG{p}{(}\PYG{p}{)}
\end{Verbatim}

\end{fulllineitems}

\index{standard\_exponential() (in module main)}

\begin{fulllineitems}
\phantomsection\label{main:main.standard_exponential}\pysiglinewithargsret{\code{main.}\bfcode{standard\_exponential}}{\emph{size=None}}{}
Draw samples from the standard exponential distribution.

\emph{standard\_exponential} is identical to the exponential distribution
with a scale parameter of 1.
\begin{description}
\item[{size}] \leavevmode{[}int or tuple of ints{]}
Shape of the output.

\end{description}
\begin{description}
\item[{out}] \leavevmode{[}float or ndarray{]}
Drawn samples.

\end{description}

Output a 3x8000 array:

\begin{Verbatim}[commandchars=\\\{\}]
\PYG{g+gp}{\PYGZgt{}\PYGZgt{}\PYGZgt{} }\PYG{n}{n} \PYG{o}{=} \PYG{n}{np}\PYG{o}{.}\PYG{n}{random}\PYG{o}{.}\PYG{n}{standard\PYGZus{}exponential}\PYG{p}{(}\PYG{p}{(}\PYG{l+m+mi}{3}\PYG{p}{,} \PYG{l+m+mi}{8000}\PYG{p}{)}\PYG{p}{)}
\end{Verbatim}

\end{fulllineitems}

\index{standard\_gamma() (in module main)}

\begin{fulllineitems}
\phantomsection\label{main:main.standard_gamma}\pysiglinewithargsret{\code{main.}\bfcode{standard\_gamma}}{\emph{shape}, \emph{size=None}}{}
Draw samples from a Standard Gamma distribution.

Samples are drawn from a Gamma distribution with specified parameters,
shape (sometimes designated ``k'') and scale=1.
\begin{description}
\item[{shape}] \leavevmode{[}float{]}
Parameter, should be \textgreater{} 0.

\item[{size}] \leavevmode{[}int or tuple of ints{]}
Output shape.  If the given shape is, e.g., \code{(m, n, k)}, then
\code{m * n * k} samples are drawn.

\end{description}
\begin{description}
\item[{samples}] \leavevmode{[}ndarray or scalar{]}
The drawn samples.

\end{description}
\begin{description}
\item[{scipy.stats.distributions.gamma}] \leavevmode{[}probability density function,{]}
distribution or cumulative density function, etc.

\end{description}

The probability density for the Gamma distribution is
\begin{gather}
\begin{split}p(x) = x^{k-1}\frac{e^{-x/\theta}}{\theta^k\Gamma(k)},\end{split}\notag
\end{gather}
where \(k\) is the shape and \(\theta\) the scale,
and \(\Gamma\) is the Gamma function.

The Gamma distribution is often used to model the times to failure of
electronic components, and arises naturally in processes for which the
waiting times between Poisson distributed events are relevant.

Draw samples from the distribution:

\begin{Verbatim}[commandchars=\\\{\}]
\PYG{g+gp}{\PYGZgt{}\PYGZgt{}\PYGZgt{} }\PYG{n}{shape}\PYG{p}{,} \PYG{n}{scale} \PYG{o}{=} \PYG{l+m+mf}{2.}\PYG{p}{,} \PYG{l+m+mf}{1.} \PYG{c}{\PYGZsh{} mean and width}
\PYG{g+gp}{\PYGZgt{}\PYGZgt{}\PYGZgt{} }\PYG{n}{s} \PYG{o}{=} \PYG{n}{np}\PYG{o}{.}\PYG{n}{random}\PYG{o}{.}\PYG{n}{standard\PYGZus{}gamma}\PYG{p}{(}\PYG{n}{shape}\PYG{p}{,} \PYG{l+m+mi}{1000000}\PYG{p}{)}
\end{Verbatim}

Display the histogram of the samples, along with
the probability density function:

\begin{Verbatim}[commandchars=\\\{\}]
\PYG{g+gp}{\PYGZgt{}\PYGZgt{}\PYGZgt{} }\PYG{k+kn}{import} \PYG{n+nn}{matplotlib.pyplot} \PYG{k+kn}{as} \PYG{n+nn}{plt}
\PYG{g+gp}{\PYGZgt{}\PYGZgt{}\PYGZgt{} }\PYG{k+kn}{import} \PYG{n+nn}{scipy.special} \PYG{k+kn}{as} \PYG{n+nn}{sps}
\PYG{g+gp}{\PYGZgt{}\PYGZgt{}\PYGZgt{} }\PYG{n}{count}\PYG{p}{,} \PYG{n}{bins}\PYG{p}{,} \PYG{n}{ignored} \PYG{o}{=} \PYG{n}{plt}\PYG{o}{.}\PYG{n}{hist}\PYG{p}{(}\PYG{n}{s}\PYG{p}{,} \PYG{l+m+mi}{50}\PYG{p}{,} \PYG{n}{normed}\PYG{o}{=}\PYG{n+nb+bp}{True}\PYG{p}{)}
\PYG{g+gp}{\PYGZgt{}\PYGZgt{}\PYGZgt{} }\PYG{n}{y} \PYG{o}{=} \PYG{n}{bins}\PYG{o}{*}\PYG{o}{*}\PYG{p}{(}\PYG{n}{shape}\PYG{o}{\PYGZhy{}}\PYG{l+m+mi}{1}\PYG{p}{)} \PYG{o}{*} \PYG{p}{(}\PYG{p}{(}\PYG{n}{np}\PYG{o}{.}\PYG{n}{exp}\PYG{p}{(}\PYG{o}{\PYGZhy{}}\PYG{n}{bins}\PYG{o}{/}\PYG{n}{scale}\PYG{p}{)}\PYG{p}{)}\PYG{o}{/} \PYGZbs{}
\PYG{g+gp}{... }                      \PYG{p}{(}\PYG{n}{sps}\PYG{o}{.}\PYG{n}{gamma}\PYG{p}{(}\PYG{n}{shape}\PYG{p}{)} \PYG{o}{*} \PYG{n}{scale}\PYG{o}{*}\PYG{o}{*}\PYG{n}{shape}\PYG{p}{)}\PYG{p}{)}
\PYG{g+gp}{\PYGZgt{}\PYGZgt{}\PYGZgt{} }\PYG{n}{plt}\PYG{o}{.}\PYG{n}{plot}\PYG{p}{(}\PYG{n}{bins}\PYG{p}{,} \PYG{n}{y}\PYG{p}{,} \PYG{n}{linewidth}\PYG{o}{=}\PYG{l+m+mi}{2}\PYG{p}{,} \PYG{n}{color}\PYG{o}{=}\PYG{l+s}{\PYGZsq{}}\PYG{l+s}{r}\PYG{l+s}{\PYGZsq{}}\PYG{p}{)}
\PYG{g+gp}{\PYGZgt{}\PYGZgt{}\PYGZgt{} }\PYG{n}{plt}\PYG{o}{.}\PYG{n}{show}\PYG{p}{(}\PYG{p}{)}
\end{Verbatim}

\end{fulllineitems}

\index{standard\_normal() (in module main)}

\begin{fulllineitems}
\phantomsection\label{main:main.standard_normal}\pysiglinewithargsret{\code{main.}\bfcode{standard\_normal}}{\emph{size=None}}{}
Returns samples from a Standard Normal distribution (mean=0, stdev=1).
\begin{description}
\item[{size}] \leavevmode{[}int or tuple of ints, optional{]}
Output shape. Default is None, in which case a single value is
returned.

\end{description}
\begin{description}
\item[{out}] \leavevmode{[}float or ndarray{]}
Drawn samples.

\end{description}

\begin{Verbatim}[commandchars=\\\{\}]
\PYG{g+gp}{\PYGZgt{}\PYGZgt{}\PYGZgt{} }\PYG{n}{s} \PYG{o}{=} \PYG{n}{np}\PYG{o}{.}\PYG{n}{random}\PYG{o}{.}\PYG{n}{standard\PYGZus{}normal}\PYG{p}{(}\PYG{l+m+mi}{8000}\PYG{p}{)}
\PYG{g+gp}{\PYGZgt{}\PYGZgt{}\PYGZgt{} }\PYG{n}{s}
\PYG{g+go}{array([ 0.6888893 ,  0.78096262, \PYGZhy{}0.89086505, ...,  0.49876311, \PYGZsh{}random}
\PYG{g+go}{       \PYGZhy{}0.38672696, \PYGZhy{}0.4685006 ])                               \PYGZsh{}random}
\PYG{g+gp}{\PYGZgt{}\PYGZgt{}\PYGZgt{} }\PYG{n}{s}\PYG{o}{.}\PYG{n}{shape}
\PYG{g+go}{(8000,)}
\PYG{g+gp}{\PYGZgt{}\PYGZgt{}\PYGZgt{} }\PYG{n}{s} \PYG{o}{=} \PYG{n}{np}\PYG{o}{.}\PYG{n}{random}\PYG{o}{.}\PYG{n}{standard\PYGZus{}normal}\PYG{p}{(}\PYG{n}{size}\PYG{o}{=}\PYG{p}{(}\PYG{l+m+mi}{3}\PYG{p}{,} \PYG{l+m+mi}{4}\PYG{p}{,} \PYG{l+m+mi}{2}\PYG{p}{)}\PYG{p}{)}
\PYG{g+gp}{\PYGZgt{}\PYGZgt{}\PYGZgt{} }\PYG{n}{s}\PYG{o}{.}\PYG{n}{shape}
\PYG{g+go}{(3, 4, 2)}
\end{Verbatim}

\end{fulllineitems}

\index{standard\_t() (in module main)}

\begin{fulllineitems}
\phantomsection\label{main:main.standard_t}\pysiglinewithargsret{\code{main.}\bfcode{standard\_t}}{\emph{df}, \emph{size=None}}{}
Standard Student's t distribution with df degrees of freedom.

A special case of the hyperbolic distribution.
As \emph{df} gets large, the result resembles that of the standard normal
distribution (\emph{standard\_normal}).
\begin{description}
\item[{df}] \leavevmode{[}int{]}
Degrees of freedom, should be \textgreater{} 0.

\item[{size}] \leavevmode{[}int or tuple of ints, optional{]}
Output shape. Default is None, in which case a single value is
returned.

\end{description}
\begin{description}
\item[{samples}] \leavevmode{[}ndarray or scalar{]}
Drawn samples.

\end{description}

The probability density function for the t distribution is
\begin{gather}
\begin{split}P(x, df) = \frac{\Gamma(\frac{df+1}{2})}{\sqrt{\pi df}
\Gamma(\frac{df}{2})}\Bigl( 1+\frac{x^2}{df} \Bigr)^{-(df+1)/2}\end{split}\notag
\end{gather}
The t test is based on an assumption that the data come from a Normal
distribution. The t test provides a way to test whether the sample mean
(that is the mean calculated from the data) is a good estimate of the true
mean.

The derivation of the t-distribution was forst published in 1908 by William
Gisset while working for the Guinness Brewery in Dublin. Due to proprietary
issues, he had to publish under a pseudonym, and so he used the name
Student.

From Dalgaard page 83 {\color{red}\bfseries{}{[}1{]}\_}, suppose the daily energy intake for 11
women in Kj is:

\begin{Verbatim}[commandchars=\\\{\}]
\PYG{g+gp}{\PYGZgt{}\PYGZgt{}\PYGZgt{} }\PYG{n}{intake} \PYG{o}{=} \PYG{n}{np}\PYG{o}{.}\PYG{n}{array}\PYG{p}{(}\PYG{p}{[}\PYG{l+m+mf}{5260.}\PYG{p}{,} \PYG{l+m+mi}{5470}\PYG{p}{,} \PYG{l+m+mi}{5640}\PYG{p}{,} \PYG{l+m+mi}{6180}\PYG{p}{,} \PYG{l+m+mi}{6390}\PYG{p}{,} \PYG{l+m+mi}{6515}\PYG{p}{,} \PYG{l+m+mi}{6805}\PYG{p}{,} \PYG{l+m+mi}{7515}\PYG{p}{,} \PYGZbs{}
\PYG{g+gp}{... }                   \PYG{l+m+mi}{7515}\PYG{p}{,} \PYG{l+m+mi}{8230}\PYG{p}{,} \PYG{l+m+mi}{8770}\PYG{p}{]}\PYG{p}{)}
\end{Verbatim}

Does their energy intake deviate systematically from the recommended
value of 7725 kJ?

We have 10 degrees of freedom, so is the sample mean within 95\% of the
recommended value?

\begin{Verbatim}[commandchars=\\\{\}]
\PYG{g+gp}{\PYGZgt{}\PYGZgt{}\PYGZgt{} }\PYG{n}{s} \PYG{o}{=} \PYG{n}{np}\PYG{o}{.}\PYG{n}{random}\PYG{o}{.}\PYG{n}{standard\PYGZus{}t}\PYG{p}{(}\PYG{l+m+mi}{10}\PYG{p}{,} \PYG{n}{size}\PYG{o}{=}\PYG{l+m+mi}{100000}\PYG{p}{)}
\PYG{g+gp}{\PYGZgt{}\PYGZgt{}\PYGZgt{} }\PYG{n}{np}\PYG{o}{.}\PYG{n}{mean}\PYG{p}{(}\PYG{n}{intake}\PYG{p}{)}
\PYG{g+go}{6753.636363636364}
\PYG{g+gp}{\PYGZgt{}\PYGZgt{}\PYGZgt{} }\PYG{n}{intake}\PYG{o}{.}\PYG{n}{std}\PYG{p}{(}\PYG{n}{ddof}\PYG{o}{=}\PYG{l+m+mi}{1}\PYG{p}{)}
\PYG{g+go}{1142.1232221373727}
\end{Verbatim}

Calculate the t statistic, setting the ddof parameter to the unbiased
value so the divisor in the standard deviation will be degrees of
freedom, N-1.

\begin{Verbatim}[commandchars=\\\{\}]
\PYG{g+gp}{\PYGZgt{}\PYGZgt{}\PYGZgt{} }\PYG{n}{t} \PYG{o}{=} \PYG{p}{(}\PYG{n}{np}\PYG{o}{.}\PYG{n}{mean}\PYG{p}{(}\PYG{n}{intake}\PYG{p}{)}\PYG{o}{\PYGZhy{}}\PYG{l+m+mi}{7725}\PYG{p}{)}\PYG{o}{/}\PYG{p}{(}\PYG{n}{intake}\PYG{o}{.}\PYG{n}{std}\PYG{p}{(}\PYG{n}{ddof}\PYG{o}{=}\PYG{l+m+mi}{1}\PYG{p}{)}\PYG{o}{/}\PYG{n}{np}\PYG{o}{.}\PYG{n}{sqrt}\PYG{p}{(}\PYG{n+nb}{len}\PYG{p}{(}\PYG{n}{intake}\PYG{p}{)}\PYG{p}{)}\PYG{p}{)}
\PYG{g+gp}{\PYGZgt{}\PYGZgt{}\PYGZgt{} }\PYG{k+kn}{import} \PYG{n+nn}{matplotlib.pyplot} \PYG{k+kn}{as} \PYG{n+nn}{plt}
\PYG{g+gp}{\PYGZgt{}\PYGZgt{}\PYGZgt{} }\PYG{n}{h} \PYG{o}{=} \PYG{n}{plt}\PYG{o}{.}\PYG{n}{hist}\PYG{p}{(}\PYG{n}{s}\PYG{p}{,} \PYG{n}{bins}\PYG{o}{=}\PYG{l+m+mi}{100}\PYG{p}{,} \PYG{n}{normed}\PYG{o}{=}\PYG{n+nb+bp}{True}\PYG{p}{)}
\end{Verbatim}

For a one-sided t-test, how far out in the distribution does the t
statistic appear?

\begin{Verbatim}[commandchars=\\\{\}]
\PYG{g+gp}{\PYGZgt{}\PYGZgt{}\PYGZgt{} }\PYG{o}{\PYGZgt{}\PYGZgt{}}\PYG{o}{\PYGZgt{}} \PYG{n}{np}\PYG{o}{.}\PYG{n}{sum}\PYG{p}{(}\PYG{n}{s}\PYG{o}{\PYGZlt{}}\PYG{n}{t}\PYG{p}{)} \PYG{o}{/} \PYG{n+nb}{float}\PYG{p}{(}\PYG{n+nb}{len}\PYG{p}{(}\PYG{n}{s}\PYG{p}{)}\PYG{p}{)}
\PYG{g+go}{0.0090699999999999999  \PYGZsh{}random}
\end{Verbatim}

So the p-value is about 0.009, which says the null hypothesis has a
probability of about 99\% of being true.

\end{fulllineitems}

\index{triangular() (in module main)}

\begin{fulllineitems}
\phantomsection\label{main:main.triangular}\pysiglinewithargsret{\code{main.}\bfcode{triangular}}{\emph{left}, \emph{mode}, \emph{right}, \emph{size=None}}{}
Draw samples from the triangular distribution.

The triangular distribution is a continuous probability distribution with
lower limit left, peak at mode, and upper limit right. Unlike the other
distributions, these parameters directly define the shape of the pdf.
\begin{description}
\item[{left}] \leavevmode{[}scalar{]}
Lower limit.

\item[{mode}] \leavevmode{[}scalar{]}
The value where the peak of the distribution occurs.
The value should fulfill the condition \code{left \textless{}= mode \textless{}= right}.

\item[{right}] \leavevmode{[}scalar{]}
Upper limit, should be larger than \emph{left}.

\item[{size}] \leavevmode{[}int or tuple of ints, optional{]}
Output shape. Default is None, in which case a single value is
returned.

\end{description}
\begin{description}
\item[{samples}] \leavevmode{[}ndarray or scalar{]}
The returned samples all lie in the interval {[}left, right{]}.

\end{description}

The probability density function for the Triangular distribution is
\begin{gather}
\begin{split}P(x;l, m, r) = \begin{cases}
\frac{2(x-l)}{(r-l)(m-l)}& \text{for $l \leq x \leq m$},\\
\frac{2(m-x)}{(r-l)(r-m)}& \text{for $m \leq x \leq r$},\\
0& \text{otherwise}.
\end{cases}\end{split}\notag
\end{gather}
The triangular distribution is often used in ill-defined problems where the
underlying distribution is not known, but some knowledge of the limits and
mode exists. Often it is used in simulations.

Draw values from the distribution and plot the histogram:

\begin{Verbatim}[commandchars=\\\{\}]
\PYG{g+gp}{\PYGZgt{}\PYGZgt{}\PYGZgt{} }\PYG{k+kn}{import} \PYG{n+nn}{matplotlib.pyplot} \PYG{k+kn}{as} \PYG{n+nn}{plt}
\PYG{g+gp}{\PYGZgt{}\PYGZgt{}\PYGZgt{} }\PYG{n}{h} \PYG{o}{=} \PYG{n}{plt}\PYG{o}{.}\PYG{n}{hist}\PYG{p}{(}\PYG{n}{np}\PYG{o}{.}\PYG{n}{random}\PYG{o}{.}\PYG{n}{triangular}\PYG{p}{(}\PYG{o}{\PYGZhy{}}\PYG{l+m+mi}{3}\PYG{p}{,} \PYG{l+m+mi}{0}\PYG{p}{,} \PYG{l+m+mi}{8}\PYG{p}{,} \PYG{l+m+mi}{100000}\PYG{p}{)}\PYG{p}{,} \PYG{n}{bins}\PYG{o}{=}\PYG{l+m+mi}{200}\PYG{p}{,}
\PYG{g+gp}{... }             \PYG{n}{normed}\PYG{o}{=}\PYG{n+nb+bp}{True}\PYG{p}{)}
\PYG{g+gp}{\PYGZgt{}\PYGZgt{}\PYGZgt{} }\PYG{n}{plt}\PYG{o}{.}\PYG{n}{show}\PYG{p}{(}\PYG{p}{)}
\end{Verbatim}

\end{fulllineitems}

\index{uniform() (in module main)}

\begin{fulllineitems}
\phantomsection\label{main:main.uniform}\pysiglinewithargsret{\code{main.}\bfcode{uniform}}{\emph{low=0.0}, \emph{high=1.0}, \emph{size=1}}{}
Draw samples from a uniform distribution.

Samples are uniformly distributed over the half-open interval
\code{{[}low, high)} (includes low, but excludes high).  In other words,
any value within the given interval is equally likely to be drawn
by \emph{uniform}.
\begin{description}
\item[{low}] \leavevmode{[}float, optional{]}
Lower boundary of the output interval.  All values generated will be
greater than or equal to low.  The default value is 0.

\item[{high}] \leavevmode{[}float{]}
Upper boundary of the output interval.  All values generated will be
less than high.  The default value is 1.0.

\item[{size}] \leavevmode{[}int or tuple of ints, optional{]}
Shape of output.  If the given size is, for example, (m,n,k),
m*n*k samples are generated.  If no shape is specified, a single sample
is returned.

\end{description}
\begin{description}
\item[{out}] \leavevmode{[}ndarray{]}
Drawn samples, with shape \emph{size}.

\end{description}

randint : Discrete uniform distribution, yielding integers.
random\_integers : Discrete uniform distribution over the closed
\begin{quote}

interval \code{{[}low, high{]}}.
\end{quote}

random\_sample : Floats uniformly distributed over \code{{[}0, 1)}.
random : Alias for \emph{random\_sample}.
rand : Convenience function that accepts dimensions as input, e.g.,
\begin{quote}

\code{rand(2,2)} would generate a 2-by-2 array of floats,
uniformly distributed over \code{{[}0, 1)}.
\end{quote}

The probability density function of the uniform distribution is
\begin{gather}
\begin{split}p(x) = \frac{1}{b - a}\end{split}\notag
\end{gather}
anywhere within the interval \code{{[}a, b)}, and zero elsewhere.

Draw samples from the distribution:

\begin{Verbatim}[commandchars=\\\{\}]
\PYG{g+gp}{\PYGZgt{}\PYGZgt{}\PYGZgt{} }\PYG{n}{s} \PYG{o}{=} \PYG{n}{np}\PYG{o}{.}\PYG{n}{random}\PYG{o}{.}\PYG{n}{uniform}\PYG{p}{(}\PYG{o}{\PYGZhy{}}\PYG{l+m+mi}{1}\PYG{p}{,}\PYG{l+m+mi}{0}\PYG{p}{,}\PYG{l+m+mi}{1000}\PYG{p}{)}
\end{Verbatim}

All values are within the given interval:

\begin{Verbatim}[commandchars=\\\{\}]
\PYG{g+gp}{\PYGZgt{}\PYGZgt{}\PYGZgt{} }\PYG{n}{np}\PYG{o}{.}\PYG{n}{all}\PYG{p}{(}\PYG{n}{s} \PYG{o}{\PYGZgt{}}\PYG{o}{=} \PYG{o}{\PYGZhy{}}\PYG{l+m+mi}{1}\PYG{p}{)}
\PYG{g+go}{True}
\PYG{g+gp}{\PYGZgt{}\PYGZgt{}\PYGZgt{} }\PYG{n}{np}\PYG{o}{.}\PYG{n}{all}\PYG{p}{(}\PYG{n}{s} \PYG{o}{\PYGZlt{}} \PYG{l+m+mi}{0}\PYG{p}{)}
\PYG{g+go}{True}
\end{Verbatim}

Display the histogram of the samples, along with the
probability density function:

\begin{Verbatim}[commandchars=\\\{\}]
\PYG{g+gp}{\PYGZgt{}\PYGZgt{}\PYGZgt{} }\PYG{k+kn}{import} \PYG{n+nn}{matplotlib.pyplot} \PYG{k+kn}{as} \PYG{n+nn}{plt}
\PYG{g+gp}{\PYGZgt{}\PYGZgt{}\PYGZgt{} }\PYG{n}{count}\PYG{p}{,} \PYG{n}{bins}\PYG{p}{,} \PYG{n}{ignored} \PYG{o}{=} \PYG{n}{plt}\PYG{o}{.}\PYG{n}{hist}\PYG{p}{(}\PYG{n}{s}\PYG{p}{,} \PYG{l+m+mi}{15}\PYG{p}{,} \PYG{n}{normed}\PYG{o}{=}\PYG{n+nb+bp}{True}\PYG{p}{)}
\PYG{g+gp}{\PYGZgt{}\PYGZgt{}\PYGZgt{} }\PYG{n}{plt}\PYG{o}{.}\PYG{n}{plot}\PYG{p}{(}\PYG{n}{bins}\PYG{p}{,} \PYG{n}{np}\PYG{o}{.}\PYG{n}{ones\PYGZus{}like}\PYG{p}{(}\PYG{n}{bins}\PYG{p}{)}\PYG{p}{,} \PYG{n}{linewidth}\PYG{o}{=}\PYG{l+m+mi}{2}\PYG{p}{,} \PYG{n}{color}\PYG{o}{=}\PYG{l+s}{\PYGZsq{}}\PYG{l+s}{r}\PYG{l+s}{\PYGZsq{}}\PYG{p}{)}
\PYG{g+gp}{\PYGZgt{}\PYGZgt{}\PYGZgt{} }\PYG{n}{plt}\PYG{o}{.}\PYG{n}{show}\PYG{p}{(}\PYG{p}{)}
\end{Verbatim}

\end{fulllineitems}

\index{vonmises() (in module main)}

\begin{fulllineitems}
\phantomsection\label{main:main.vonmises}\pysiglinewithargsret{\code{main.}\bfcode{vonmises}}{\emph{mu}, \emph{kappa}, \emph{size=None}}{}
Draw samples from a von Mises distribution.

Samples are drawn from a von Mises distribution with specified mode
(mu) and dispersion (kappa), on the interval {[}-pi, pi{]}.

The von Mises distribution (also known as the circular normal
distribution) is a continuous probability distribution on the unit
circle.  It may be thought of as the circular analogue of the normal
distribution.
\begin{description}
\item[{mu}] \leavevmode{[}float{]}
Mode (``center'') of the distribution.

\item[{kappa}] \leavevmode{[}float{]}
Dispersion of the distribution, has to be \textgreater{}=0.

\item[{size}] \leavevmode{[}int or tuple of int{]}
Output shape.  If the given shape is, e.g., \code{(m, n, k)}, then
\code{m * n * k} samples are drawn.

\end{description}
\begin{description}
\item[{samples}] \leavevmode{[}scalar or ndarray{]}
The returned samples, which are in the interval {[}-pi, pi{]}.

\end{description}
\begin{description}
\item[{scipy.stats.distributions.vonmises}] \leavevmode{[}probability density function,{]}
distribution, or cumulative density function, etc.

\end{description}

The probability density for the von Mises distribution is
\begin{gather}
\begin{split}p(x) = \frac{e^{\kappa cos(x-\mu)}}{2\pi I_0(\kappa)},\end{split}\notag
\end{gather}
where \(\mu\) is the mode and \(\kappa\) the dispersion,
and \(I_0(\kappa)\) is the modified Bessel function of order 0.

The von Mises is named for Richard Edler von Mises, who was born in
Austria-Hungary, in what is now the Ukraine.  He fled to the United
States in 1939 and became a professor at Harvard.  He worked in
probability theory, aerodynamics, fluid mechanics, and philosophy of
science.

Abramowitz, M. and Stegun, I. A. (ed.), \emph{Handbook of Mathematical
Functions}, New York: Dover, 1965.

von Mises, R., \emph{Mathematical Theory of Probability and Statistics},
New York: Academic Press, 1964.

Draw samples from the distribution:

\begin{Verbatim}[commandchars=\\\{\}]
\PYG{g+gp}{\PYGZgt{}\PYGZgt{}\PYGZgt{} }\PYG{n}{mu}\PYG{p}{,} \PYG{n}{kappa} \PYG{o}{=} \PYG{l+m+mf}{0.0}\PYG{p}{,} \PYG{l+m+mf}{4.0} \PYG{c}{\PYGZsh{} mean and dispersion}
\PYG{g+gp}{\PYGZgt{}\PYGZgt{}\PYGZgt{} }\PYG{n}{s} \PYG{o}{=} \PYG{n}{np}\PYG{o}{.}\PYG{n}{random}\PYG{o}{.}\PYG{n}{vonmises}\PYG{p}{(}\PYG{n}{mu}\PYG{p}{,} \PYG{n}{kappa}\PYG{p}{,} \PYG{l+m+mi}{1000}\PYG{p}{)}
\end{Verbatim}

Display the histogram of the samples, along with
the probability density function:

\begin{Verbatim}[commandchars=\\\{\}]
\PYG{g+gp}{\PYGZgt{}\PYGZgt{}\PYGZgt{} }\PYG{k+kn}{import} \PYG{n+nn}{matplotlib.pyplot} \PYG{k+kn}{as} \PYG{n+nn}{plt}
\PYG{g+gp}{\PYGZgt{}\PYGZgt{}\PYGZgt{} }\PYG{k+kn}{import} \PYG{n+nn}{scipy.special} \PYG{k+kn}{as} \PYG{n+nn}{sps}
\PYG{g+gp}{\PYGZgt{}\PYGZgt{}\PYGZgt{} }\PYG{n}{count}\PYG{p}{,} \PYG{n}{bins}\PYG{p}{,} \PYG{n}{ignored} \PYG{o}{=} \PYG{n}{plt}\PYG{o}{.}\PYG{n}{hist}\PYG{p}{(}\PYG{n}{s}\PYG{p}{,} \PYG{l+m+mi}{50}\PYG{p}{,} \PYG{n}{normed}\PYG{o}{=}\PYG{n+nb+bp}{True}\PYG{p}{)}
\PYG{g+gp}{\PYGZgt{}\PYGZgt{}\PYGZgt{} }\PYG{n}{x} \PYG{o}{=} \PYG{n}{np}\PYG{o}{.}\PYG{n}{arange}\PYG{p}{(}\PYG{o}{\PYGZhy{}}\PYG{n}{np}\PYG{o}{.}\PYG{n}{pi}\PYG{p}{,} \PYG{n}{np}\PYG{o}{.}\PYG{n}{pi}\PYG{p}{,} \PYG{l+m+mi}{2}\PYG{o}{*}\PYG{n}{np}\PYG{o}{.}\PYG{n}{pi}\PYG{o}{/}\PYG{l+m+mf}{50.}\PYG{p}{)}
\PYG{g+gp}{\PYGZgt{}\PYGZgt{}\PYGZgt{} }\PYG{n}{y} \PYG{o}{=} \PYG{o}{\PYGZhy{}}\PYG{n}{np}\PYG{o}{.}\PYG{n}{exp}\PYG{p}{(}\PYG{n}{kappa}\PYG{o}{*}\PYG{n}{np}\PYG{o}{.}\PYG{n}{cos}\PYG{p}{(}\PYG{n}{x}\PYG{o}{\PYGZhy{}}\PYG{n}{mu}\PYG{p}{)}\PYG{p}{)}\PYG{o}{/}\PYG{p}{(}\PYG{l+m+mi}{2}\PYG{o}{*}\PYG{n}{np}\PYG{o}{.}\PYG{n}{pi}\PYG{o}{*}\PYG{n}{sps}\PYG{o}{.}\PYG{n}{jn}\PYG{p}{(}\PYG{l+m+mi}{0}\PYG{p}{,}\PYG{n}{kappa}\PYG{p}{)}\PYG{p}{)}
\PYG{g+gp}{\PYGZgt{}\PYGZgt{}\PYGZgt{} }\PYG{n}{plt}\PYG{o}{.}\PYG{n}{plot}\PYG{p}{(}\PYG{n}{x}\PYG{p}{,} \PYG{n}{y}\PYG{o}{/}\PYG{n+nb}{max}\PYG{p}{(}\PYG{n}{y}\PYG{p}{)}\PYG{p}{,} \PYG{n}{linewidth}\PYG{o}{=}\PYG{l+m+mi}{2}\PYG{p}{,} \PYG{n}{color}\PYG{o}{=}\PYG{l+s}{\PYGZsq{}}\PYG{l+s}{r}\PYG{l+s}{\PYGZsq{}}\PYG{p}{)}
\PYG{g+gp}{\PYGZgt{}\PYGZgt{}\PYGZgt{} }\PYG{n}{plt}\PYG{o}{.}\PYG{n}{show}\PYG{p}{(}\PYG{p}{)}
\end{Verbatim}

\end{fulllineitems}

\index{wald() (in module main)}

\begin{fulllineitems}
\phantomsection\label{main:main.wald}\pysiglinewithargsret{\code{main.}\bfcode{wald}}{\emph{mean}, \emph{scale}, \emph{size=None}}{}
Draw samples from a Wald, or Inverse Gaussian, distribution.

As the scale approaches infinity, the distribution becomes more like a
Gaussian.

Some references claim that the Wald is an Inverse Gaussian with mean=1, but
this is by no means universal.

The Inverse Gaussian distribution was first studied in relationship to
Brownian motion. In 1956 M.C.K. Tweedie used the name Inverse Gaussian
because there is an inverse relationship between the time to cover a unit
distance and distance covered in unit time.
\begin{description}
\item[{mean}] \leavevmode{[}scalar{]}
Distribution mean, should be \textgreater{} 0.

\item[{scale}] \leavevmode{[}scalar{]}
Scale parameter, should be \textgreater{}= 0.

\item[{size}] \leavevmode{[}int or tuple of ints, optional{]}
Output shape. Default is None, in which case a single value is
returned.

\end{description}
\begin{description}
\item[{samples}] \leavevmode{[}ndarray or scalar{]}
Drawn sample, all greater than zero.

\end{description}

The probability density function for the Wald distribution is
\begin{gather}
\begin{split}P(x;mean,scale) = \sqrt{\frac{scale}{2\pi x^3}}e^
\frac{-scale(x-mean)^2}{2\cdotp mean^2x}\end{split}\notag
\end{gather}
As noted above the Inverse Gaussian distribution first arise from attempts
to model Brownian Motion. It is also a competitor to the Weibull for use in
reliability modeling and modeling stock returns and interest rate
processes.

Draw values from the distribution and plot the histogram:

\begin{Verbatim}[commandchars=\\\{\}]
\PYG{g+gp}{\PYGZgt{}\PYGZgt{}\PYGZgt{} }\PYG{k+kn}{import} \PYG{n+nn}{matplotlib.pyplot} \PYG{k+kn}{as} \PYG{n+nn}{plt}
\PYG{g+gp}{\PYGZgt{}\PYGZgt{}\PYGZgt{} }\PYG{n}{h} \PYG{o}{=} \PYG{n}{plt}\PYG{o}{.}\PYG{n}{hist}\PYG{p}{(}\PYG{n}{np}\PYG{o}{.}\PYG{n}{random}\PYG{o}{.}\PYG{n}{wald}\PYG{p}{(}\PYG{l+m+mi}{3}\PYG{p}{,} \PYG{l+m+mi}{2}\PYG{p}{,} \PYG{l+m+mi}{100000}\PYG{p}{)}\PYG{p}{,} \PYG{n}{bins}\PYG{o}{=}\PYG{l+m+mi}{200}\PYG{p}{,} \PYG{n}{normed}\PYG{o}{=}\PYG{n+nb+bp}{True}\PYG{p}{)}
\PYG{g+gp}{\PYGZgt{}\PYGZgt{}\PYGZgt{} }\PYG{n}{plt}\PYG{o}{.}\PYG{n}{show}\PYG{p}{(}\PYG{p}{)}
\end{Verbatim}

\end{fulllineitems}

\index{weibull() (in module main)}

\begin{fulllineitems}
\phantomsection\label{main:main.weibull}\pysiglinewithargsret{\code{main.}\bfcode{weibull}}{\emph{a}, \emph{size=None}}{}
Weibull distribution.

Draw samples from a 1-parameter Weibull distribution with the given
shape parameter \emph{a}.
\begin{gather}
\begin{split}X = (-ln(U))^{1/a}\end{split}\notag
\end{gather}
Here, U is drawn from the uniform distribution over (0,1{]}.

The more common 2-parameter Weibull, including a scale parameter
\(\lambda\) is just \(X = \lambda(-ln(U))^{1/a}\).
\begin{description}
\item[{a}] \leavevmode{[}float{]}
Shape of the distribution.

\item[{size}] \leavevmode{[}tuple of ints{]}
Output shape.  If the given shape is, e.g., \code{(m, n, k)}, then
\code{m * n * k} samples are drawn.

\end{description}

scipy.stats.distributions.weibull\_max
scipy.stats.distributions.weibull\_min
scipy.stats.distributions.genextreme
gumbel

The Weibull (or Type III asymptotic extreme value distribution for smallest
values, SEV Type III, or Rosin-Rammler distribution) is one of a class of
Generalized Extreme Value (GEV) distributions used in modeling extreme
value problems.  This class includes the Gumbel and Frechet distributions.

The probability density for the Weibull distribution is
\begin{gather}
\begin{split}p(x) = \frac{a}
{\lambda}(\frac{x}{\lambda})^{a-1}e^{-(x/\lambda)^a},\end{split}\notag
\end{gather}
where \(a\) is the shape and \(\lambda\) the scale.

The function has its peak (the mode) at
\(\lambda(\frac{a-1}{a})^{1/a}\).

When \code{a = 1}, the Weibull distribution reduces to the exponential
distribution.

Draw samples from the distribution:

\begin{Verbatim}[commandchars=\\\{\}]
\PYG{g+gp}{\PYGZgt{}\PYGZgt{}\PYGZgt{} }\PYG{n}{a} \PYG{o}{=} \PYG{l+m+mf}{5.} \PYG{c}{\PYGZsh{} shape}
\PYG{g+gp}{\PYGZgt{}\PYGZgt{}\PYGZgt{} }\PYG{n}{s} \PYG{o}{=} \PYG{n}{np}\PYG{o}{.}\PYG{n}{random}\PYG{o}{.}\PYG{n}{weibull}\PYG{p}{(}\PYG{n}{a}\PYG{p}{,} \PYG{l+m+mi}{1000}\PYG{p}{)}
\end{Verbatim}

Display the histogram of the samples, along with
the probability density function:

\begin{Verbatim}[commandchars=\\\{\}]
\PYG{g+gp}{\PYGZgt{}\PYGZgt{}\PYGZgt{} }\PYG{k+kn}{import} \PYG{n+nn}{matplotlib.pyplot} \PYG{k+kn}{as} \PYG{n+nn}{plt}
\PYG{g+gp}{\PYGZgt{}\PYGZgt{}\PYGZgt{} }\PYG{n}{x} \PYG{o}{=} \PYG{n}{np}\PYG{o}{.}\PYG{n}{arange}\PYG{p}{(}\PYG{l+m+mi}{1}\PYG{p}{,}\PYG{l+m+mf}{100.}\PYG{p}{)}\PYG{o}{/}\PYG{l+m+mf}{50.}
\PYG{g+gp}{\PYGZgt{}\PYGZgt{}\PYGZgt{} }\PYG{k}{def} \PYG{n+nf}{weib}\PYG{p}{(}\PYG{n}{x}\PYG{p}{,}\PYG{n}{n}\PYG{p}{,}\PYG{n}{a}\PYG{p}{)}\PYG{p}{:}
\PYG{g+gp}{... }    \PYG{k}{return} \PYG{p}{(}\PYG{n}{a} \PYG{o}{/} \PYG{n}{n}\PYG{p}{)} \PYG{o}{*} \PYG{p}{(}\PYG{n}{x} \PYG{o}{/} \PYG{n}{n}\PYG{p}{)}\PYG{o}{*}\PYG{o}{*}\PYG{p}{(}\PYG{n}{a} \PYG{o}{\PYGZhy{}} \PYG{l+m+mi}{1}\PYG{p}{)} \PYG{o}{*} \PYG{n}{np}\PYG{o}{.}\PYG{n}{exp}\PYG{p}{(}\PYG{o}{\PYGZhy{}}\PYG{p}{(}\PYG{n}{x} \PYG{o}{/} \PYG{n}{n}\PYG{p}{)}\PYG{o}{*}\PYG{o}{*}\PYG{n}{a}\PYG{p}{)}
\end{Verbatim}

\begin{Verbatim}[commandchars=\\\{\}]
\PYG{g+gp}{\PYGZgt{}\PYGZgt{}\PYGZgt{} }\PYG{n}{count}\PYG{p}{,} \PYG{n}{bins}\PYG{p}{,} \PYG{n}{ignored} \PYG{o}{=} \PYG{n}{plt}\PYG{o}{.}\PYG{n}{hist}\PYG{p}{(}\PYG{n}{np}\PYG{o}{.}\PYG{n}{random}\PYG{o}{.}\PYG{n}{weibull}\PYG{p}{(}\PYG{l+m+mf}{5.}\PYG{p}{,}\PYG{l+m+mi}{1000}\PYG{p}{)}\PYG{p}{)}
\PYG{g+gp}{\PYGZgt{}\PYGZgt{}\PYGZgt{} }\PYG{n}{x} \PYG{o}{=} \PYG{n}{np}\PYG{o}{.}\PYG{n}{arange}\PYG{p}{(}\PYG{l+m+mi}{1}\PYG{p}{,}\PYG{l+m+mf}{100.}\PYG{p}{)}\PYG{o}{/}\PYG{l+m+mf}{50.}
\PYG{g+gp}{\PYGZgt{}\PYGZgt{}\PYGZgt{} }\PYG{n}{scale} \PYG{o}{=} \PYG{n}{count}\PYG{o}{.}\PYG{n}{max}\PYG{p}{(}\PYG{p}{)}\PYG{o}{/}\PYG{n}{weib}\PYG{p}{(}\PYG{n}{x}\PYG{p}{,} \PYG{l+m+mf}{1.}\PYG{p}{,} \PYG{l+m+mf}{5.}\PYG{p}{)}\PYG{o}{.}\PYG{n}{max}\PYG{p}{(}\PYG{p}{)}
\PYG{g+gp}{\PYGZgt{}\PYGZgt{}\PYGZgt{} }\PYG{n}{plt}\PYG{o}{.}\PYG{n}{plot}\PYG{p}{(}\PYG{n}{x}\PYG{p}{,} \PYG{n}{weib}\PYG{p}{(}\PYG{n}{x}\PYG{p}{,} \PYG{l+m+mf}{1.}\PYG{p}{,} \PYG{l+m+mf}{5.}\PYG{p}{)}\PYG{o}{*}\PYG{n}{scale}\PYG{p}{)}
\PYG{g+gp}{\PYGZgt{}\PYGZgt{}\PYGZgt{} }\PYG{n}{plt}\PYG{o}{.}\PYG{n}{show}\PYG{p}{(}\PYG{p}{)}
\end{Verbatim}

\end{fulllineitems}

\index{zipf() (in module main)}

\begin{fulllineitems}
\phantomsection\label{main:main.zipf}\pysiglinewithargsret{\code{main.}\bfcode{zipf}}{\emph{a}, \emph{size=None}}{}
Draw samples from a Zipf distribution.

Samples are drawn from a Zipf distribution with specified parameter
\emph{a} \textgreater{} 1.

The Zipf distribution (also known as the zeta distribution) is a
continuous probability distribution that satisfies Zipf's law: the
frequency of an item is inversely proportional to its rank in a
frequency table.
\begin{description}
\item[{a}] \leavevmode{[}float \textgreater{} 1{]}
Distribution parameter.

\item[{size}] \leavevmode{[}int or tuple of int, optional{]}
Output shape.  If the given shape is, e.g., \code{(m, n, k)}, then
\code{m * n * k} samples are drawn; a single integer is equivalent in
its result to providing a mono-tuple, i.e., a 1-D array of length
\emph{size} is returned.  The default is None, in which case a single
scalar is returned.

\end{description}
\begin{description}
\item[{samples}] \leavevmode{[}scalar or ndarray{]}
The returned samples are greater than or equal to one.

\end{description}
\begin{description}
\item[{scipy.stats.distributions.zipf}] \leavevmode{[}probability density function,{]}
distribution, or cumulative density function, etc.

\end{description}

The probability density for the Zipf distribution is
\begin{gather}
\begin{split}p(x) = \frac{x^{-a}}{\zeta(a)},\end{split}\notag
\end{gather}
where \(\zeta\) is the Riemann Zeta function.

It is named for the American linguist George Kingsley Zipf, who noted
that the frequency of any word in a sample of a language is inversely
proportional to its rank in the frequency table.

Zipf, G. K., \emph{Selected Studies of the Principle of Relative Frequency
in Language}, Cambridge, MA: Harvard Univ. Press, 1932.

Draw samples from the distribution:

\begin{Verbatim}[commandchars=\\\{\}]
\PYG{g+gp}{\PYGZgt{}\PYGZgt{}\PYGZgt{} }\PYG{n}{a} \PYG{o}{=} \PYG{l+m+mf}{2.} \PYG{c}{\PYGZsh{} parameter}
\PYG{g+gp}{\PYGZgt{}\PYGZgt{}\PYGZgt{} }\PYG{n}{s} \PYG{o}{=} \PYG{n}{np}\PYG{o}{.}\PYG{n}{random}\PYG{o}{.}\PYG{n}{zipf}\PYG{p}{(}\PYG{n}{a}\PYG{p}{,} \PYG{l+m+mi}{1000}\PYG{p}{)}
\end{Verbatim}

Display the histogram of the samples, along with
the probability density function:

\begin{Verbatim}[commandchars=\\\{\}]
\PYG{g+gp}{\PYGZgt{}\PYGZgt{}\PYGZgt{} }\PYG{k+kn}{import} \PYG{n+nn}{matplotlib.pyplot} \PYG{k+kn}{as} \PYG{n+nn}{plt}
\PYG{g+gp}{\PYGZgt{}\PYGZgt{}\PYGZgt{} }\PYG{k+kn}{import} \PYG{n+nn}{scipy.special} \PYG{k+kn}{as} \PYG{n+nn}{sps}
\PYG{g+go}{Truncate s values at 50 so plot is interesting}
\PYG{g+gp}{\PYGZgt{}\PYGZgt{}\PYGZgt{} }\PYG{n}{count}\PYG{p}{,} \PYG{n}{bins}\PYG{p}{,} \PYG{n}{ignored} \PYG{o}{=} \PYG{n}{plt}\PYG{o}{.}\PYG{n}{hist}\PYG{p}{(}\PYG{n}{s}\PYG{p}{[}\PYG{n}{s}\PYG{o}{\PYGZlt{}}\PYG{l+m+mi}{50}\PYG{p}{]}\PYG{p}{,} \PYG{l+m+mi}{50}\PYG{p}{,} \PYG{n}{normed}\PYG{o}{=}\PYG{n+nb+bp}{True}\PYG{p}{)}
\PYG{g+gp}{\PYGZgt{}\PYGZgt{}\PYGZgt{} }\PYG{n}{x} \PYG{o}{=} \PYG{n}{np}\PYG{o}{.}\PYG{n}{arange}\PYG{p}{(}\PYG{l+m+mf}{1.}\PYG{p}{,} \PYG{l+m+mf}{50.}\PYG{p}{)}
\PYG{g+gp}{\PYGZgt{}\PYGZgt{}\PYGZgt{} }\PYG{n}{y} \PYG{o}{=} \PYG{n}{x}\PYG{o}{*}\PYG{o}{*}\PYG{p}{(}\PYG{o}{\PYGZhy{}}\PYG{n}{a}\PYG{p}{)}\PYG{o}{/}\PYG{n}{sps}\PYG{o}{.}\PYG{n}{zetac}\PYG{p}{(}\PYG{n}{a}\PYG{p}{)}
\PYG{g+gp}{\PYGZgt{}\PYGZgt{}\PYGZgt{} }\PYG{n}{plt}\PYG{o}{.}\PYG{n}{plot}\PYG{p}{(}\PYG{n}{x}\PYG{p}{,} \PYG{n}{y}\PYG{o}{/}\PYG{n+nb}{max}\PYG{p}{(}\PYG{n}{y}\PYG{p}{)}\PYG{p}{,} \PYG{n}{linewidth}\PYG{o}{=}\PYG{l+m+mi}{2}\PYG{p}{,} \PYG{n}{color}\PYG{o}{=}\PYG{l+s}{\PYGZsq{}}\PYG{l+s}{r}\PYG{l+s}{\PYGZsq{}}\PYG{p}{)}
\PYG{g+gp}{\PYGZgt{}\PYGZgt{}\PYGZgt{} }\PYG{n}{plt}\PYG{o}{.}\PYG{n}{show}\PYG{p}{(}\PYG{p}{)}
\end{Verbatim}

\end{fulllineitems}



\chapter{prepareNewSim Module}
\label{prepareNewSim:preparenewsim-module}\label{prepareNewSim::doc}

\chapter{topology\_analysis Module}
\label{topology_analysis:topology-analysis-module}\label{topology_analysis::doc}\label{topology_analysis:module-topology_analysis}\index{topology\_analysis (module)}
This python program assess different network structures in term of topological (NON DYNAMICAL) RAF and SCC presence according 
to different structural parameters.
\index{beta() (in module topology\_analysis)}

\begin{fulllineitems}
\phantomsection\label{topology_analysis:topology_analysis.beta}\pysiglinewithargsret{\code{topology\_analysis.}\bfcode{beta}}{\emph{a}, \emph{b}, \emph{size=None}}{}
The Beta distribution over \code{{[}0, 1{]}}.

The Beta distribution is a special case of the Dirichlet distribution,
and is related to the Gamma distribution.  It has the probability
distribution function
\begin{gather}
\begin{split}f(x; a,b) = \frac{1}{B(\alpha, \beta)} x^{\alpha - 1}
(1 - x)^{\beta - 1},\end{split}\notag
\end{gather}
where the normalisation, B, is the beta function,
\begin{gather}
\begin{split}B(\alpha, \beta) = \int_0^1 t^{\alpha - 1}
(1 - t)^{\beta - 1} dt.\end{split}\notag
\end{gather}
It is often seen in Bayesian inference and order statistics.
\begin{description}
\item[{a}] \leavevmode{[}float{]}
Alpha, non-negative.

\item[{b}] \leavevmode{[}float{]}
Beta, non-negative.

\item[{size}] \leavevmode{[}tuple of ints, optional{]}
The number of samples to draw.  The output is packed according to
the size given.

\end{description}
\begin{description}
\item[{out}] \leavevmode{[}ndarray{]}
Array of the given shape, containing values drawn from a
Beta distribution.

\end{description}

\end{fulllineitems}

\index{binomial() (in module topology\_analysis)}

\begin{fulllineitems}
\phantomsection\label{topology_analysis:topology_analysis.binomial}\pysiglinewithargsret{\code{topology\_analysis.}\bfcode{binomial}}{\emph{n}, \emph{p}, \emph{size=None}}{}
Draw samples from a binomial distribution.

Samples are drawn from a Binomial distribution with specified
parameters, n trials and p probability of success where
n an integer \textgreater{}= 0 and p is in the interval {[}0,1{]}. (n may be
input as a float, but it is truncated to an integer in use)
\begin{description}
\item[{n}] \leavevmode{[}float (but truncated to an integer){]}
parameter, \textgreater{}= 0.

\item[{p}] \leavevmode{[}float{]}
parameter, \textgreater{}= 0 and \textless{}=1.

\item[{size}] \leavevmode{[}\{tuple, int\}{]}
Output shape.  If the given shape is, e.g., \code{(m, n, k)}, then
\code{m * n * k} samples are drawn.

\end{description}
\begin{description}
\item[{samples}] \leavevmode{[}\{ndarray, scalar\}{]}
where the values are all integers in  {[}0, n{]}.

\end{description}
\begin{description}
\item[{scipy.stats.distributions.binom}] \leavevmode{[}probability density function,{]}
distribution or cumulative density function, etc.

\end{description}

The probability density for the Binomial distribution is
\begin{gather}
\begin{split}P(N) = \binom{n}{N}p^N(1-p)^{n-N},\end{split}\notag
\end{gather}
where \(n\) is the number of trials, \(p\) is the probability
of success, and \(N\) is the number of successes.

When estimating the standard error of a proportion in a population by
using a random sample, the normal distribution works well unless the
product p*n \textless{}=5, where p = population proportion estimate, and n =
number of samples, in which case the binomial distribution is used
instead. For example, a sample of 15 people shows 4 who are left
handed, and 11 who are right handed. Then p = 4/15 = 27\%. 0.27*15 = 4,
so the binomial distribution should be used in this case.

Draw samples from the distribution:

\begin{Verbatim}[commandchars=\\\{\}]
\PYG{g+gp}{\PYGZgt{}\PYGZgt{}\PYGZgt{} }\PYG{n}{n}\PYG{p}{,} \PYG{n}{p} \PYG{o}{=} \PYG{l+m+mi}{10}\PYG{p}{,} \PYG{o}{.}\PYG{l+m+mi}{5} \PYG{c}{\PYGZsh{} number of trials, probability of each trial}
\PYG{g+gp}{\PYGZgt{}\PYGZgt{}\PYGZgt{} }\PYG{n}{s} \PYG{o}{=} \PYG{n}{np}\PYG{o}{.}\PYG{n}{random}\PYG{o}{.}\PYG{n}{binomial}\PYG{p}{(}\PYG{n}{n}\PYG{p}{,} \PYG{n}{p}\PYG{p}{,} \PYG{l+m+mi}{1000}\PYG{p}{)}
\PYG{g+go}{\PYGZsh{} result of flipping a coin 10 times, tested 1000 times.}
\end{Verbatim}

A real world example. A company drills 9 wild-cat oil exploration
wells, each with an estimated probability of success of 0.1. All nine
wells fail. What is the probability of that happening?

Let's do 20,000 trials of the model, and count the number that
generate zero positive results.

\begin{Verbatim}[commandchars=\\\{\}]
\PYG{g+gp}{\PYGZgt{}\PYGZgt{}\PYGZgt{} }\PYG{n+nb}{sum}\PYG{p}{(}\PYG{n}{np}\PYG{o}{.}\PYG{n}{random}\PYG{o}{.}\PYG{n}{binomial}\PYG{p}{(}\PYG{l+m+mi}{9}\PYG{p}{,}\PYG{l+m+mf}{0.1}\PYG{p}{,}\PYG{l+m+mi}{20000}\PYG{p}{)}\PYG{o}{==}\PYG{l+m+mi}{0}\PYG{p}{)}\PYG{o}{/}\PYG{l+m+mf}{20000.}
\PYG{g+go}{answer = 0.38885, or 38\PYGZpc{}.}
\end{Verbatim}

\end{fulllineitems}

\index{chisquare() (in module topology\_analysis)}

\begin{fulllineitems}
\phantomsection\label{topology_analysis:topology_analysis.chisquare}\pysiglinewithargsret{\code{topology\_analysis.}\bfcode{chisquare}}{\emph{df}, \emph{size=None}}{}
Draw samples from a chi-square distribution.

When \emph{df} independent random variables, each with standard normal
distributions (mean 0, variance 1), are squared and summed, the
resulting distribution is chi-square (see Notes).  This distribution
is often used in hypothesis testing.
\begin{description}
\item[{df}] \leavevmode{[}int{]}
Number of degrees of freedom.

\item[{size}] \leavevmode{[}tuple of ints, int, optional{]}
Size of the returned array.  By default, a scalar is
returned.

\end{description}
\begin{description}
\item[{output}] \leavevmode{[}ndarray{]}
Samples drawn from the distribution, packed in a \emph{size}-shaped
array.

\end{description}
\begin{description}
\item[{ValueError}] \leavevmode
When \emph{df} \textless{}= 0 or when an inappropriate \emph{size} (e.g. \code{size=-1})
is given.

\end{description}

The variable obtained by summing the squares of \emph{df} independent,
standard normally distributed random variables:
\begin{gather}
\begin{split}Q = \sum_{i=0}^{\mathtt{df}} X^2_i\end{split}\notag
\end{gather}
is chi-square distributed, denoted
\begin{gather}
\begin{split}Q \sim \chi^2_k.\end{split}\notag
\end{gather}
The probability density function of the chi-squared distribution is
\begin{gather}
\begin{split}p(x) = \frac{(1/2)^{k/2}}{\Gamma(k/2)}
x^{k/2 - 1} e^{-x/2},\end{split}\notag
\end{gather}
where \(\Gamma\) is the gamma function,
\begin{gather}
\begin{split}\Gamma(x) = \int_0^{-\infty} t^{x - 1} e^{-t} dt.\end{split}\notag
\end{gather}
\href{http://www.itl.nist.gov/div898/handbook/eda/section3/eda3666.htm}{NIST/SEMATECH e-Handbook of Statistical Methods}

\begin{Verbatim}[commandchars=\\\{\}]
\PYG{g+gp}{\PYGZgt{}\PYGZgt{}\PYGZgt{} }\PYG{n}{np}\PYG{o}{.}\PYG{n}{random}\PYG{o}{.}\PYG{n}{chisquare}\PYG{p}{(}\PYG{l+m+mi}{2}\PYG{p}{,}\PYG{l+m+mi}{4}\PYG{p}{)}
\PYG{g+go}{array([ 1.89920014,  9.00867716,  3.13710533,  5.62318272])}
\end{Verbatim}

\end{fulllineitems}

\index{exponential() (in module topology\_analysis)}

\begin{fulllineitems}
\phantomsection\label{topology_analysis:topology_analysis.exponential}\pysiglinewithargsret{\code{topology\_analysis.}\bfcode{exponential}}{\emph{scale=1.0}, \emph{size=None}}{}
Exponential distribution.

Its probability density function is
\begin{gather}
\begin{split}f(x; \frac{1}{\beta}) = \frac{1}{\beta} \exp(-\frac{x}{\beta}),\end{split}\notag
\end{gather}
for \code{x \textgreater{} 0} and 0 elsewhere. \(\beta\) is the scale parameter,
which is the inverse of the rate parameter \(\lambda = 1/\beta\).
The rate parameter is an alternative, widely used parameterization
of the exponential distribution {\color{red}\bfseries{}{[}3{]}\_}.

The exponential distribution is a continuous analogue of the
geometric distribution.  It describes many common situations, such as
the size of raindrops measured over many rainstorms {\color{red}\bfseries{}{[}1{]}\_}, or the time
between page requests to Wikipedia {\color{red}\bfseries{}{[}2{]}\_}.
\begin{description}
\item[{scale}] \leavevmode{[}float{]}
The scale parameter, \(\beta = 1/\lambda\).

\item[{size}] \leavevmode{[}tuple of ints{]}
Number of samples to draw.  The output is shaped
according to \emph{size}.

\end{description}

\end{fulllineitems}

\index{f() (in module topology\_analysis)}

\begin{fulllineitems}
\phantomsection\label{topology_analysis:topology_analysis.f}\pysiglinewithargsret{\code{topology\_analysis.}\bfcode{f}}{\emph{dfnum}, \emph{dfden}, \emph{size=None}}{}
Draw samples from a F distribution.

Samples are drawn from an F distribution with specified parameters,
\emph{dfnum} (degrees of freedom in numerator) and \emph{dfden} (degrees of freedom
in denominator), where both parameters should be greater than zero.

The random variate of the F distribution (also known as the
Fisher distribution) is a continuous probability distribution
that arises in ANOVA tests, and is the ratio of two chi-square
variates.
\begin{description}
\item[{dfnum}] \leavevmode{[}float{]}
Degrees of freedom in numerator. Should be greater than zero.

\item[{dfden}] \leavevmode{[}float{]}
Degrees of freedom in denominator. Should be greater than zero.

\item[{size}] \leavevmode{[}\{tuple, int\}, optional{]}
Output shape.  If the given shape is, e.g., \code{(m, n, k)},
then \code{m * n * k} samples are drawn. By default only one sample
is returned.

\end{description}
\begin{description}
\item[{samples}] \leavevmode{[}\{ndarray, scalar\}{]}
Samples from the Fisher distribution.

\end{description}
\begin{description}
\item[{scipy.stats.distributions.f}] \leavevmode{[}probability density function,{]}
distribution or cumulative density function, etc.

\end{description}

The F statistic is used to compare in-group variances to between-group
variances. Calculating the distribution depends on the sampling, and
so it is a function of the respective degrees of freedom in the
problem.  The variable \emph{dfnum} is the number of samples minus one, the
between-groups degrees of freedom, while \emph{dfden} is the within-groups
degrees of freedom, the sum of the number of samples in each group
minus the number of groups.

An example from Glantz{[}1{]}, pp 47-40.
Two groups, children of diabetics (25 people) and children from people
without diabetes (25 controls). Fasting blood glucose was measured,
case group had a mean value of 86.1, controls had a mean value of
82.2. Standard deviations were 2.09 and 2.49 respectively. Are these
data consistent with the null hypothesis that the parents diabetic
status does not affect their children's blood glucose levels?
Calculating the F statistic from the data gives a value of 36.01.

Draw samples from the distribution:

\begin{Verbatim}[commandchars=\\\{\}]
\PYG{g+gp}{\PYGZgt{}\PYGZgt{}\PYGZgt{} }\PYG{n}{dfnum} \PYG{o}{=} \PYG{l+m+mf}{1.} \PYG{c}{\PYGZsh{} between group degrees of freedom}
\PYG{g+gp}{\PYGZgt{}\PYGZgt{}\PYGZgt{} }\PYG{n}{dfden} \PYG{o}{=} \PYG{l+m+mf}{48.} \PYG{c}{\PYGZsh{} within groups degrees of freedom}
\PYG{g+gp}{\PYGZgt{}\PYGZgt{}\PYGZgt{} }\PYG{n}{s} \PYG{o}{=} \PYG{n}{np}\PYG{o}{.}\PYG{n}{random}\PYG{o}{.}\PYG{n}{f}\PYG{p}{(}\PYG{n}{dfnum}\PYG{p}{,} \PYG{n}{dfden}\PYG{p}{,} \PYG{l+m+mi}{1000}\PYG{p}{)}
\end{Verbatim}

The lower bound for the top 1\% of the samples is :

\begin{Verbatim}[commandchars=\\\{\}]
\PYG{g+gp}{\PYGZgt{}\PYGZgt{}\PYGZgt{} }\PYG{n}{sort}\PYG{p}{(}\PYG{n}{s}\PYG{p}{)}\PYG{p}{[}\PYG{o}{\PYGZhy{}}\PYG{l+m+mi}{10}\PYG{p}{]}
\PYG{g+go}{7.61988120985}
\end{Verbatim}

So there is about a 1\% chance that the F statistic will exceed 7.62,
the measured value is 36, so the null hypothesis is rejected at the 1\%
level.

\end{fulllineitems}

\index{gamma() (in module topology\_analysis)}

\begin{fulllineitems}
\phantomsection\label{topology_analysis:topology_analysis.gamma}\pysiglinewithargsret{\code{topology\_analysis.}\bfcode{gamma}}{\emph{shape}, \emph{scale=1.0}, \emph{size=None}}{}
Draw samples from a Gamma distribution.

Samples are drawn from a Gamma distribution with specified parameters,
\emph{shape} (sometimes designated ``k'') and \emph{scale} (sometimes designated
``theta''), where both parameters are \textgreater{} 0.
\begin{description}
\item[{shape}] \leavevmode{[}scalar \textgreater{} 0{]}
The shape of the gamma distribution.

\item[{scale}] \leavevmode{[}scalar \textgreater{} 0, optional{]}
The scale of the gamma distribution.  Default is equal to 1.

\item[{size}] \leavevmode{[}shape\_tuple, optional{]}
Output shape.  If the given shape is, e.g., \code{(m, n, k)}, then
\code{m * n * k} samples are drawn.

\end{description}
\begin{description}
\item[{out}] \leavevmode{[}ndarray, float{]}
Returns one sample unless \emph{size} parameter is specified.

\end{description}
\begin{description}
\item[{scipy.stats.distributions.gamma}] \leavevmode{[}probability density function,{]}
distribution or cumulative density function, etc.

\end{description}

The probability density for the Gamma distribution is
\begin{gather}
\begin{split}p(x) = x^{k-1}\frac{e^{-x/\theta}}{\theta^k\Gamma(k)},\end{split}\notag
\end{gather}
where \(k\) is the shape and \(\theta\) the scale,
and \(\Gamma\) is the Gamma function.

The Gamma distribution is often used to model the times to failure of
electronic components, and arises naturally in processes for which the
waiting times between Poisson distributed events are relevant.

Draw samples from the distribution:

\begin{Verbatim}[commandchars=\\\{\}]
\PYG{g+gp}{\PYGZgt{}\PYGZgt{}\PYGZgt{} }\PYG{n}{shape}\PYG{p}{,} \PYG{n}{scale} \PYG{o}{=} \PYG{l+m+mf}{2.}\PYG{p}{,} \PYG{l+m+mf}{2.} \PYG{c}{\PYGZsh{} mean and dispersion}
\PYG{g+gp}{\PYGZgt{}\PYGZgt{}\PYGZgt{} }\PYG{n}{s} \PYG{o}{=} \PYG{n}{np}\PYG{o}{.}\PYG{n}{random}\PYG{o}{.}\PYG{n}{gamma}\PYG{p}{(}\PYG{n}{shape}\PYG{p}{,} \PYG{n}{scale}\PYG{p}{,} \PYG{l+m+mi}{1000}\PYG{p}{)}
\end{Verbatim}

Display the histogram of the samples, along with
the probability density function:

\begin{Verbatim}[commandchars=\\\{\}]
\PYG{g+gp}{\PYGZgt{}\PYGZgt{}\PYGZgt{} }\PYG{k+kn}{import} \PYG{n+nn}{matplotlib.pyplot} \PYG{k+kn}{as} \PYG{n+nn}{plt}
\PYG{g+gp}{\PYGZgt{}\PYGZgt{}\PYGZgt{} }\PYG{k+kn}{import} \PYG{n+nn}{scipy.special} \PYG{k+kn}{as} \PYG{n+nn}{sps}
\PYG{g+gp}{\PYGZgt{}\PYGZgt{}\PYGZgt{} }\PYG{n}{count}\PYG{p}{,} \PYG{n}{bins}\PYG{p}{,} \PYG{n}{ignored} \PYG{o}{=} \PYG{n}{plt}\PYG{o}{.}\PYG{n}{hist}\PYG{p}{(}\PYG{n}{s}\PYG{p}{,} \PYG{l+m+mi}{50}\PYG{p}{,} \PYG{n}{normed}\PYG{o}{=}\PYG{n+nb+bp}{True}\PYG{p}{)}
\PYG{g+gp}{\PYGZgt{}\PYGZgt{}\PYGZgt{} }\PYG{n}{y} \PYG{o}{=} \PYG{n}{bins}\PYG{o}{*}\PYG{o}{*}\PYG{p}{(}\PYG{n}{shape}\PYG{o}{\PYGZhy{}}\PYG{l+m+mi}{1}\PYG{p}{)}\PYG{o}{*}\PYG{p}{(}\PYG{n}{np}\PYG{o}{.}\PYG{n}{exp}\PYG{p}{(}\PYG{o}{\PYGZhy{}}\PYG{n}{bins}\PYG{o}{/}\PYG{n}{scale}\PYG{p}{)} \PYG{o}{/}
\PYG{g+gp}{... }                     \PYG{p}{(}\PYG{n}{sps}\PYG{o}{.}\PYG{n}{gamma}\PYG{p}{(}\PYG{n}{shape}\PYG{p}{)}\PYG{o}{*}\PYG{n}{scale}\PYG{o}{*}\PYG{o}{*}\PYG{n}{shape}\PYG{p}{)}\PYG{p}{)}
\PYG{g+gp}{\PYGZgt{}\PYGZgt{}\PYGZgt{} }\PYG{n}{plt}\PYG{o}{.}\PYG{n}{plot}\PYG{p}{(}\PYG{n}{bins}\PYG{p}{,} \PYG{n}{y}\PYG{p}{,} \PYG{n}{linewidth}\PYG{o}{=}\PYG{l+m+mi}{2}\PYG{p}{,} \PYG{n}{color}\PYG{o}{=}\PYG{l+s}{\PYGZsq{}}\PYG{l+s}{r}\PYG{l+s}{\PYGZsq{}}\PYG{p}{)}
\PYG{g+gp}{\PYGZgt{}\PYGZgt{}\PYGZgt{} }\PYG{n}{plt}\PYG{o}{.}\PYG{n}{show}\PYG{p}{(}\PYG{p}{)}
\end{Verbatim}

\end{fulllineitems}

\index{geometric() (in module topology\_analysis)}

\begin{fulllineitems}
\phantomsection\label{topology_analysis:topology_analysis.geometric}\pysiglinewithargsret{\code{topology\_analysis.}\bfcode{geometric}}{\emph{p}, \emph{size=None}}{}
Draw samples from the geometric distribution.

Bernoulli trials are experiments with one of two outcomes:
success or failure (an example of such an experiment is flipping
a coin).  The geometric distribution models the number of trials
that must be run in order to achieve success.  It is therefore
supported on the positive integers, \code{k = 1, 2, ...}.

The probability mass function of the geometric distribution is
\begin{gather}
\begin{split}f(k) = (1 - p)^{k - 1} p\end{split}\notag
\end{gather}
where \emph{p} is the probability of success of an individual trial.
\begin{description}
\item[{p}] \leavevmode{[}float{]}
The probability of success of an individual trial.

\item[{size}] \leavevmode{[}tuple of ints{]}
Number of values to draw from the distribution.  The output
is shaped according to \emph{size}.

\end{description}
\begin{description}
\item[{out}] \leavevmode{[}ndarray{]}
Samples from the geometric distribution, shaped according to
\emph{size}.

\end{description}

Draw ten thousand values from the geometric distribution,
with the probability of an individual success equal to 0.35:

\begin{Verbatim}[commandchars=\\\{\}]
\PYG{g+gp}{\PYGZgt{}\PYGZgt{}\PYGZgt{} }\PYG{n}{z} \PYG{o}{=} \PYG{n}{np}\PYG{o}{.}\PYG{n}{random}\PYG{o}{.}\PYG{n}{geometric}\PYG{p}{(}\PYG{n}{p}\PYG{o}{=}\PYG{l+m+mf}{0.35}\PYG{p}{,} \PYG{n}{size}\PYG{o}{=}\PYG{l+m+mi}{10000}\PYG{p}{)}
\end{Verbatim}

How many trials succeeded after a single run?

\begin{Verbatim}[commandchars=\\\{\}]
\PYG{g+gp}{\PYGZgt{}\PYGZgt{}\PYGZgt{} }\PYG{p}{(}\PYG{n}{z} \PYG{o}{==} \PYG{l+m+mi}{1}\PYG{p}{)}\PYG{o}{.}\PYG{n}{sum}\PYG{p}{(}\PYG{p}{)} \PYG{o}{/} \PYG{l+m+mf}{10000.}
\PYG{g+go}{0.34889999999999999 \PYGZsh{}random}
\end{Verbatim}

\end{fulllineitems}

\index{get\_state() (in module topology\_analysis)}

\begin{fulllineitems}
\phantomsection\label{topology_analysis:topology_analysis.get_state}\pysiglinewithargsret{\code{topology\_analysis.}\bfcode{get\_state}}{}{}
Return a tuple representing the internal state of the generator.

For more details, see \emph{set\_state}.
\begin{description}
\item[{out}] \leavevmode{[}tuple(str, ndarray of 624 uints, int, int, float){]}
The returned tuple has the following items:
\begin{enumerate}
\item {} 
the string `MT19937'.

\item {} 
a 1-D array of 624 unsigned integer keys.

\item {} 
an integer \code{pos}.

\item {} 
an integer \code{has\_gauss}.

\item {} 
a float \code{cached\_gaussian}.

\end{enumerate}

\end{description}

set\_state

\emph{set\_state} and \emph{get\_state} are not needed to work with any of the
random distributions in NumPy. If the internal state is manually altered,
the user should know exactly what he/she is doing.

\end{fulllineitems}

\index{gumbel() (in module topology\_analysis)}

\begin{fulllineitems}
\phantomsection\label{topology_analysis:topology_analysis.gumbel}\pysiglinewithargsret{\code{topology\_analysis.}\bfcode{gumbel}}{\emph{loc=0.0}, \emph{scale=1.0}, \emph{size=None}}{}
Gumbel distribution.

Draw samples from a Gumbel distribution with specified location and scale.
For more information on the Gumbel distribution, see Notes and References
below.
\begin{description}
\item[{loc}] \leavevmode{[}float{]}
The location of the mode of the distribution.

\item[{scale}] \leavevmode{[}float{]}
The scale parameter of the distribution.

\item[{size}] \leavevmode{[}tuple of ints{]}
Output shape.  If the given shape is, e.g., \code{(m, n, k)}, then
\code{m * n * k} samples are drawn.

\end{description}
\begin{description}
\item[{out}] \leavevmode{[}ndarray{]}
The samples

\end{description}

scipy.stats.gumbel\_l
scipy.stats.gumbel\_r
scipy.stats.genextreme
\begin{quote}

probability density function, distribution, or cumulative density
function, etc. for each of the above
\end{quote}

weibull

The Gumbel (or Smallest Extreme Value (SEV) or the Smallest Extreme Value
Type I) distribution is one of a class of Generalized Extreme Value (GEV)
distributions used in modeling extreme value problems.  The Gumbel is a
special case of the Extreme Value Type I distribution for maximums from
distributions with ``exponential-like'' tails.

The probability density for the Gumbel distribution is
\begin{gather}
\begin{split}p(x) = \frac{e^{-(x - \mu)/ \beta}}{\beta} e^{ -e^{-(x - \mu)/
\beta}},\end{split}\notag
\end{gather}
where \(\mu\) is the mode, a location parameter, and \(\beta\) is
the scale parameter.

The Gumbel (named for German mathematician Emil Julius Gumbel) was used
very early in the hydrology literature, for modeling the occurrence of
flood events. It is also used for modeling maximum wind speed and rainfall
rates.  It is a ``fat-tailed'' distribution - the probability of an event in
the tail of the distribution is larger than if one used a Gaussian, hence
the surprisingly frequent occurrence of 100-year floods. Floods were
initially modeled as a Gaussian process, which underestimated the frequency
of extreme events.

It is one of a class of extreme value distributions, the Generalized
Extreme Value (GEV) distributions, which also includes the Weibull and
Frechet.

The function has a mean of \(\mu + 0.57721\beta\) and a variance of
\(\frac{\pi^2}{6}\beta^2\).

Gumbel, E. J., \emph{Statistics of Extremes}, New York: Columbia University
Press, 1958.

Reiss, R.-D. and Thomas, M., \emph{Statistical Analysis of Extreme Values from
Insurance, Finance, Hydrology and Other Fields}, Basel: Birkhauser Verlag,
2001.

Draw samples from the distribution:

\begin{Verbatim}[commandchars=\\\{\}]
\PYG{g+gp}{\PYGZgt{}\PYGZgt{}\PYGZgt{} }\PYG{n}{mu}\PYG{p}{,} \PYG{n}{beta} \PYG{o}{=} \PYG{l+m+mi}{0}\PYG{p}{,} \PYG{l+m+mf}{0.1} \PYG{c}{\PYGZsh{} location and scale}
\PYG{g+gp}{\PYGZgt{}\PYGZgt{}\PYGZgt{} }\PYG{n}{s} \PYG{o}{=} \PYG{n}{np}\PYG{o}{.}\PYG{n}{random}\PYG{o}{.}\PYG{n}{gumbel}\PYG{p}{(}\PYG{n}{mu}\PYG{p}{,} \PYG{n}{beta}\PYG{p}{,} \PYG{l+m+mi}{1000}\PYG{p}{)}
\end{Verbatim}

Display the histogram of the samples, along with
the probability density function:

\begin{Verbatim}[commandchars=\\\{\}]
\PYG{g+gp}{\PYGZgt{}\PYGZgt{}\PYGZgt{} }\PYG{k+kn}{import} \PYG{n+nn}{matplotlib.pyplot} \PYG{k+kn}{as} \PYG{n+nn}{plt}
\PYG{g+gp}{\PYGZgt{}\PYGZgt{}\PYGZgt{} }\PYG{n}{count}\PYG{p}{,} \PYG{n}{bins}\PYG{p}{,} \PYG{n}{ignored} \PYG{o}{=} \PYG{n}{plt}\PYG{o}{.}\PYG{n}{hist}\PYG{p}{(}\PYG{n}{s}\PYG{p}{,} \PYG{l+m+mi}{30}\PYG{p}{,} \PYG{n}{normed}\PYG{o}{=}\PYG{n+nb+bp}{True}\PYG{p}{)}
\PYG{g+gp}{\PYGZgt{}\PYGZgt{}\PYGZgt{} }\PYG{n}{plt}\PYG{o}{.}\PYG{n}{plot}\PYG{p}{(}\PYG{n}{bins}\PYG{p}{,} \PYG{p}{(}\PYG{l+m+mi}{1}\PYG{o}{/}\PYG{n}{beta}\PYG{p}{)}\PYG{o}{*}\PYG{n}{np}\PYG{o}{.}\PYG{n}{exp}\PYG{p}{(}\PYG{o}{\PYGZhy{}}\PYG{p}{(}\PYG{n}{bins} \PYG{o}{\PYGZhy{}} \PYG{n}{mu}\PYG{p}{)}\PYG{o}{/}\PYG{n}{beta}\PYG{p}{)}
\PYG{g+gp}{... }         \PYG{o}{*} \PYG{n}{np}\PYG{o}{.}\PYG{n}{exp}\PYG{p}{(} \PYG{o}{\PYGZhy{}}\PYG{n}{np}\PYG{o}{.}\PYG{n}{exp}\PYG{p}{(} \PYG{o}{\PYGZhy{}}\PYG{p}{(}\PYG{n}{bins} \PYG{o}{\PYGZhy{}} \PYG{n}{mu}\PYG{p}{)} \PYG{o}{/}\PYG{n}{beta}\PYG{p}{)} \PYG{p}{)}\PYG{p}{,}
\PYG{g+gp}{... }         \PYG{n}{linewidth}\PYG{o}{=}\PYG{l+m+mi}{2}\PYG{p}{,} \PYG{n}{color}\PYG{o}{=}\PYG{l+s}{\PYGZsq{}}\PYG{l+s}{r}\PYG{l+s}{\PYGZsq{}}\PYG{p}{)}
\PYG{g+gp}{\PYGZgt{}\PYGZgt{}\PYGZgt{} }\PYG{n}{plt}\PYG{o}{.}\PYG{n}{show}\PYG{p}{(}\PYG{p}{)}
\end{Verbatim}

Show how an extreme value distribution can arise from a Gaussian process
and compare to a Gaussian:

\begin{Verbatim}[commandchars=\\\{\}]
\PYG{g+gp}{\PYGZgt{}\PYGZgt{}\PYGZgt{} }\PYG{n}{means} \PYG{o}{=} \PYG{p}{[}\PYG{p}{]}
\PYG{g+gp}{\PYGZgt{}\PYGZgt{}\PYGZgt{} }\PYG{n}{maxima} \PYG{o}{=} \PYG{p}{[}\PYG{p}{]}
\PYG{g+gp}{\PYGZgt{}\PYGZgt{}\PYGZgt{} }\PYG{k}{for} \PYG{n}{i} \PYG{o+ow}{in} \PYG{n+nb}{range}\PYG{p}{(}\PYG{l+m+mi}{0}\PYG{p}{,}\PYG{l+m+mi}{1000}\PYG{p}{)} \PYG{p}{:}
\PYG{g+gp}{... }   \PYG{n}{a} \PYG{o}{=} \PYG{n}{np}\PYG{o}{.}\PYG{n}{random}\PYG{o}{.}\PYG{n}{normal}\PYG{p}{(}\PYG{n}{mu}\PYG{p}{,} \PYG{n}{beta}\PYG{p}{,} \PYG{l+m+mi}{1000}\PYG{p}{)}
\PYG{g+gp}{... }   \PYG{n}{means}\PYG{o}{.}\PYG{n}{append}\PYG{p}{(}\PYG{n}{a}\PYG{o}{.}\PYG{n}{mean}\PYG{p}{(}\PYG{p}{)}\PYG{p}{)}
\PYG{g+gp}{... }   \PYG{n}{maxima}\PYG{o}{.}\PYG{n}{append}\PYG{p}{(}\PYG{n}{a}\PYG{o}{.}\PYG{n}{max}\PYG{p}{(}\PYG{p}{)}\PYG{p}{)}
\PYG{g+gp}{\PYGZgt{}\PYGZgt{}\PYGZgt{} }\PYG{n}{count}\PYG{p}{,} \PYG{n}{bins}\PYG{p}{,} \PYG{n}{ignored} \PYG{o}{=} \PYG{n}{plt}\PYG{o}{.}\PYG{n}{hist}\PYG{p}{(}\PYG{n}{maxima}\PYG{p}{,} \PYG{l+m+mi}{30}\PYG{p}{,} \PYG{n}{normed}\PYG{o}{=}\PYG{n+nb+bp}{True}\PYG{p}{)}
\PYG{g+gp}{\PYGZgt{}\PYGZgt{}\PYGZgt{} }\PYG{n}{beta} \PYG{o}{=} \PYG{n}{np}\PYG{o}{.}\PYG{n}{std}\PYG{p}{(}\PYG{n}{maxima}\PYG{p}{)}\PYG{o}{*}\PYG{n}{np}\PYG{o}{.}\PYG{n}{pi}\PYG{o}{/}\PYG{n}{np}\PYG{o}{.}\PYG{n}{sqrt}\PYG{p}{(}\PYG{l+m+mi}{6}\PYG{p}{)}
\PYG{g+gp}{\PYGZgt{}\PYGZgt{}\PYGZgt{} }\PYG{n}{mu} \PYG{o}{=} \PYG{n}{np}\PYG{o}{.}\PYG{n}{mean}\PYG{p}{(}\PYG{n}{maxima}\PYG{p}{)} \PYG{o}{\PYGZhy{}} \PYG{l+m+mf}{0.57721}\PYG{o}{*}\PYG{n}{beta}
\PYG{g+gp}{\PYGZgt{}\PYGZgt{}\PYGZgt{} }\PYG{n}{plt}\PYG{o}{.}\PYG{n}{plot}\PYG{p}{(}\PYG{n}{bins}\PYG{p}{,} \PYG{p}{(}\PYG{l+m+mi}{1}\PYG{o}{/}\PYG{n}{beta}\PYG{p}{)}\PYG{o}{*}\PYG{n}{np}\PYG{o}{.}\PYG{n}{exp}\PYG{p}{(}\PYG{o}{\PYGZhy{}}\PYG{p}{(}\PYG{n}{bins} \PYG{o}{\PYGZhy{}} \PYG{n}{mu}\PYG{p}{)}\PYG{o}{/}\PYG{n}{beta}\PYG{p}{)}
\PYG{g+gp}{... }         \PYG{o}{*} \PYG{n}{np}\PYG{o}{.}\PYG{n}{exp}\PYG{p}{(}\PYG{o}{\PYGZhy{}}\PYG{n}{np}\PYG{o}{.}\PYG{n}{exp}\PYG{p}{(}\PYG{o}{\PYGZhy{}}\PYG{p}{(}\PYG{n}{bins} \PYG{o}{\PYGZhy{}} \PYG{n}{mu}\PYG{p}{)}\PYG{o}{/}\PYG{n}{beta}\PYG{p}{)}\PYG{p}{)}\PYG{p}{,}
\PYG{g+gp}{... }         \PYG{n}{linewidth}\PYG{o}{=}\PYG{l+m+mi}{2}\PYG{p}{,} \PYG{n}{color}\PYG{o}{=}\PYG{l+s}{\PYGZsq{}}\PYG{l+s}{r}\PYG{l+s}{\PYGZsq{}}\PYG{p}{)}
\PYG{g+gp}{\PYGZgt{}\PYGZgt{}\PYGZgt{} }\PYG{n}{plt}\PYG{o}{.}\PYG{n}{plot}\PYG{p}{(}\PYG{n}{bins}\PYG{p}{,} \PYG{l+m+mi}{1}\PYG{o}{/}\PYG{p}{(}\PYG{n}{beta} \PYG{o}{*} \PYG{n}{np}\PYG{o}{.}\PYG{n}{sqrt}\PYG{p}{(}\PYG{l+m+mi}{2} \PYG{o}{*} \PYG{n}{np}\PYG{o}{.}\PYG{n}{pi}\PYG{p}{)}\PYG{p}{)}
\PYG{g+gp}{... }         \PYG{o}{*} \PYG{n}{np}\PYG{o}{.}\PYG{n}{exp}\PYG{p}{(}\PYG{o}{\PYGZhy{}}\PYG{p}{(}\PYG{n}{bins} \PYG{o}{\PYGZhy{}} \PYG{n}{mu}\PYG{p}{)}\PYG{o}{*}\PYG{o}{*}\PYG{l+m+mi}{2} \PYG{o}{/} \PYG{p}{(}\PYG{l+m+mi}{2} \PYG{o}{*} \PYG{n}{beta}\PYG{o}{*}\PYG{o}{*}\PYG{l+m+mi}{2}\PYG{p}{)}\PYG{p}{)}\PYG{p}{,}
\PYG{g+gp}{... }         \PYG{n}{linewidth}\PYG{o}{=}\PYG{l+m+mi}{2}\PYG{p}{,} \PYG{n}{color}\PYG{o}{=}\PYG{l+s}{\PYGZsq{}}\PYG{l+s}{g}\PYG{l+s}{\PYGZsq{}}\PYG{p}{)}
\PYG{g+gp}{\PYGZgt{}\PYGZgt{}\PYGZgt{} }\PYG{n}{plt}\PYG{o}{.}\PYG{n}{show}\PYG{p}{(}\PYG{p}{)}
\end{Verbatim}

\end{fulllineitems}

\index{hypergeometric() (in module topology\_analysis)}

\begin{fulllineitems}
\phantomsection\label{topology_analysis:topology_analysis.hypergeometric}\pysiglinewithargsret{\code{topology\_analysis.}\bfcode{hypergeometric}}{\emph{ngood}, \emph{nbad}, \emph{nsample}, \emph{size=None}}{}
Draw samples from a Hypergeometric distribution.

Samples are drawn from a Hypergeometric distribution with specified
parameters, ngood (ways to make a good selection), nbad (ways to make
a bad selection), and nsample = number of items sampled, which is less
than or equal to the sum ngood + nbad.
\begin{description}
\item[{ngood}] \leavevmode{[}int or array\_like{]}
Number of ways to make a good selection.  Must be nonnegative.

\item[{nbad}] \leavevmode{[}int or array\_like{]}
Number of ways to make a bad selection.  Must be nonnegative.

\item[{nsample}] \leavevmode{[}int or array\_like{]}
Number of items sampled.  Must be at least 1 and at most
\code{ngood + nbad}.

\item[{size}] \leavevmode{[}int or tuple of int{]}
Output shape.  If the given shape is, e.g., \code{(m, n, k)}, then
\code{m * n * k} samples are drawn.

\end{description}
\begin{description}
\item[{samples}] \leavevmode{[}ndarray or scalar{]}
The values are all integers in  {[}0, n{]}.

\end{description}
\begin{description}
\item[{scipy.stats.distributions.hypergeom}] \leavevmode{[}probability density function,{]}
distribution or cumulative density function, etc.

\end{description}

The probability density for the Hypergeometric distribution is
\begin{gather}
\begin{split}P(x) = \frac{\binom{m}{n}\binom{N-m}{n-x}}{\binom{N}{n}},\end{split}\notag
\end{gather}
where \(0 \le x \le m\) and \(n+m-N \le x \le n\)

for P(x) the probability of x successes, n = ngood, m = nbad, and
N = number of samples.

Consider an urn with black and white marbles in it, ngood of them
black and nbad are white. If you draw nsample balls without
replacement, then the Hypergeometric distribution describes the
distribution of black balls in the drawn sample.

Note that this distribution is very similar to the Binomial
distribution, except that in this case, samples are drawn without
replacement, whereas in the Binomial case samples are drawn with
replacement (or the sample space is infinite). As the sample space
becomes large, this distribution approaches the Binomial.

Draw samples from the distribution:

\begin{Verbatim}[commandchars=\\\{\}]
\PYG{g+gp}{\PYGZgt{}\PYGZgt{}\PYGZgt{} }\PYG{n}{ngood}\PYG{p}{,} \PYG{n}{nbad}\PYG{p}{,} \PYG{n}{nsamp} \PYG{o}{=} \PYG{l+m+mi}{100}\PYG{p}{,} \PYG{l+m+mi}{2}\PYG{p}{,} \PYG{l+m+mi}{10}
\PYG{g+go}{\PYGZsh{} number of good, number of bad, and number of samples}
\PYG{g+gp}{\PYGZgt{}\PYGZgt{}\PYGZgt{} }\PYG{n}{s} \PYG{o}{=} \PYG{n}{np}\PYG{o}{.}\PYG{n}{random}\PYG{o}{.}\PYG{n}{hypergeometric}\PYG{p}{(}\PYG{n}{ngood}\PYG{p}{,} \PYG{n}{nbad}\PYG{p}{,} \PYG{n}{nsamp}\PYG{p}{,} \PYG{l+m+mi}{1000}\PYG{p}{)}
\PYG{g+gp}{\PYGZgt{}\PYGZgt{}\PYGZgt{} }\PYG{n}{hist}\PYG{p}{(}\PYG{n}{s}\PYG{p}{)}
\PYG{g+go}{\PYGZsh{}   note that it is very unlikely to grab both bad items}
\end{Verbatim}

Suppose you have an urn with 15 white and 15 black marbles.
If you pull 15 marbles at random, how likely is it that
12 or more of them are one color?

\begin{Verbatim}[commandchars=\\\{\}]
\PYG{g+gp}{\PYGZgt{}\PYGZgt{}\PYGZgt{} }\PYG{n}{s} \PYG{o}{=} \PYG{n}{np}\PYG{o}{.}\PYG{n}{random}\PYG{o}{.}\PYG{n}{hypergeometric}\PYG{p}{(}\PYG{l+m+mi}{15}\PYG{p}{,} \PYG{l+m+mi}{15}\PYG{p}{,} \PYG{l+m+mi}{15}\PYG{p}{,} \PYG{l+m+mi}{100000}\PYG{p}{)}
\PYG{g+gp}{\PYGZgt{}\PYGZgt{}\PYGZgt{} }\PYG{n+nb}{sum}\PYG{p}{(}\PYG{n}{s}\PYG{o}{\PYGZgt{}}\PYG{o}{=}\PYG{l+m+mi}{12}\PYG{p}{)}\PYG{o}{/}\PYG{l+m+mf}{100000.} \PYG{o}{+} \PYG{n+nb}{sum}\PYG{p}{(}\PYG{n}{s}\PYG{o}{\PYGZlt{}}\PYG{o}{=}\PYG{l+m+mi}{3}\PYG{p}{)}\PYG{o}{/}\PYG{l+m+mf}{100000.}
\PYG{g+go}{\PYGZsh{}   answer = 0.003 ... pretty unlikely!}
\end{Verbatim}

\end{fulllineitems}

\index{laplace() (in module topology\_analysis)}

\begin{fulllineitems}
\phantomsection\label{topology_analysis:topology_analysis.laplace}\pysiglinewithargsret{\code{topology\_analysis.}\bfcode{laplace}}{\emph{loc=0.0}, \emph{scale=1.0}, \emph{size=None}}{}
Draw samples from the Laplace or double exponential distribution with
specified location (or mean) and scale (decay).

The Laplace distribution is similar to the Gaussian/normal distribution,
but is sharper at the peak and has fatter tails. It represents the
difference between two independent, identically distributed exponential
random variables.
\begin{description}
\item[{loc}] \leavevmode{[}float{]}
The position, \(\mu\), of the distribution peak.

\item[{scale}] \leavevmode{[}float{]}
\(\lambda\), the exponential decay.

\end{description}

It has the probability density function
\begin{gather}
\begin{split}f(x; \mu, \lambda) = \frac{1}{2\lambda}
\exp\left(-\frac{|x - \mu|}{\lambda}\right).\end{split}\notag
\end{gather}
The first law of Laplace, from 1774, states that the frequency of an error
can be expressed as an exponential function of the absolute magnitude of
the error, which leads to the Laplace distribution. For many problems in
Economics and Health sciences, this distribution seems to model the data
better than the standard Gaussian distribution

Draw samples from the distribution

\begin{Verbatim}[commandchars=\\\{\}]
\PYG{g+gp}{\PYGZgt{}\PYGZgt{}\PYGZgt{} }\PYG{n}{loc}\PYG{p}{,} \PYG{n}{scale} \PYG{o}{=} \PYG{l+m+mf}{0.}\PYG{p}{,} \PYG{l+m+mf}{1.}
\PYG{g+gp}{\PYGZgt{}\PYGZgt{}\PYGZgt{} }\PYG{n}{s} \PYG{o}{=} \PYG{n}{np}\PYG{o}{.}\PYG{n}{random}\PYG{o}{.}\PYG{n}{laplace}\PYG{p}{(}\PYG{n}{loc}\PYG{p}{,} \PYG{n}{scale}\PYG{p}{,} \PYG{l+m+mi}{1000}\PYG{p}{)}
\end{Verbatim}

Display the histogram of the samples, along with
the probability density function:

\begin{Verbatim}[commandchars=\\\{\}]
\PYG{g+gp}{\PYGZgt{}\PYGZgt{}\PYGZgt{} }\PYG{k+kn}{import} \PYG{n+nn}{matplotlib.pyplot} \PYG{k+kn}{as} \PYG{n+nn}{plt}
\PYG{g+gp}{\PYGZgt{}\PYGZgt{}\PYGZgt{} }\PYG{n}{count}\PYG{p}{,} \PYG{n}{bins}\PYG{p}{,} \PYG{n}{ignored} \PYG{o}{=} \PYG{n}{plt}\PYG{o}{.}\PYG{n}{hist}\PYG{p}{(}\PYG{n}{s}\PYG{p}{,} \PYG{l+m+mi}{30}\PYG{p}{,} \PYG{n}{normed}\PYG{o}{=}\PYG{n+nb+bp}{True}\PYG{p}{)}
\PYG{g+gp}{\PYGZgt{}\PYGZgt{}\PYGZgt{} }\PYG{n}{x} \PYG{o}{=} \PYG{n}{np}\PYG{o}{.}\PYG{n}{arange}\PYG{p}{(}\PYG{o}{\PYGZhy{}}\PYG{l+m+mf}{8.}\PYG{p}{,} \PYG{l+m+mf}{8.}\PYG{p}{,} \PYG{o}{.}\PYG{l+m+mo}{01}\PYG{p}{)}
\PYG{g+gp}{\PYGZgt{}\PYGZgt{}\PYGZgt{} }\PYG{n}{pdf} \PYG{o}{=} \PYG{n}{np}\PYG{o}{.}\PYG{n}{exp}\PYG{p}{(}\PYG{o}{\PYGZhy{}}\PYG{n+nb}{abs}\PYG{p}{(}\PYG{n}{x}\PYG{o}{\PYGZhy{}}\PYG{n}{loc}\PYG{o}{/}\PYG{n}{scale}\PYG{p}{)}\PYG{p}{)}\PYG{o}{/}\PYG{p}{(}\PYG{l+m+mf}{2.}\PYG{o}{*}\PYG{n}{scale}\PYG{p}{)}
\PYG{g+gp}{\PYGZgt{}\PYGZgt{}\PYGZgt{} }\PYG{n}{plt}\PYG{o}{.}\PYG{n}{plot}\PYG{p}{(}\PYG{n}{x}\PYG{p}{,} \PYG{n}{pdf}\PYG{p}{)}
\end{Verbatim}

Plot Gaussian for comparison:

\begin{Verbatim}[commandchars=\\\{\}]
\PYG{g+gp}{\PYGZgt{}\PYGZgt{}\PYGZgt{} }\PYG{n}{g} \PYG{o}{=} \PYG{p}{(}\PYG{l+m+mi}{1}\PYG{o}{/}\PYG{p}{(}\PYG{n}{scale} \PYG{o}{*} \PYG{n}{np}\PYG{o}{.}\PYG{n}{sqrt}\PYG{p}{(}\PYG{l+m+mi}{2} \PYG{o}{*} \PYG{n}{np}\PYG{o}{.}\PYG{n}{pi}\PYG{p}{)}\PYG{p}{)} \PYG{o}{*} 
\PYG{g+gp}{... }     \PYG{n}{np}\PYG{o}{.}\PYG{n}{exp}\PYG{p}{(} \PYG{o}{\PYGZhy{}} \PYG{p}{(}\PYG{n}{x} \PYG{o}{\PYGZhy{}} \PYG{n}{loc}\PYG{p}{)}\PYG{o}{*}\PYG{o}{*}\PYG{l+m+mi}{2} \PYG{o}{/} \PYG{p}{(}\PYG{l+m+mi}{2} \PYG{o}{*} \PYG{n}{scale}\PYG{o}{*}\PYG{o}{*}\PYG{l+m+mi}{2}\PYG{p}{)} \PYG{p}{)}\PYG{p}{)}
\PYG{g+gp}{\PYGZgt{}\PYGZgt{}\PYGZgt{} }\PYG{n}{plt}\PYG{o}{.}\PYG{n}{plot}\PYG{p}{(}\PYG{n}{x}\PYG{p}{,}\PYG{n}{g}\PYG{p}{)}
\end{Verbatim}

\end{fulllineitems}

\index{logistic() (in module topology\_analysis)}

\begin{fulllineitems}
\phantomsection\label{topology_analysis:topology_analysis.logistic}\pysiglinewithargsret{\code{topology\_analysis.}\bfcode{logistic}}{\emph{loc=0.0}, \emph{scale=1.0}, \emph{size=None}}{}
Draw samples from a Logistic distribution.

Samples are drawn from a Logistic distribution with specified
parameters, loc (location or mean, also median), and scale (\textgreater{}0).

loc : float

scale : float \textgreater{} 0.
\begin{description}
\item[{size}] \leavevmode{[}\{tuple, int\}{]}
Output shape.  If the given shape is, e.g., \code{(m, n, k)}, then
\code{m * n * k} samples are drawn.

\end{description}
\begin{description}
\item[{samples}] \leavevmode{[}\{ndarray, scalar\}{]}
where the values are all integers in  {[}0, n{]}.

\end{description}
\begin{description}
\item[{scipy.stats.distributions.logistic}] \leavevmode{[}probability density function,{]}
distribution or cumulative density function, etc.

\end{description}

The probability density for the Logistic distribution is
\begin{gather}
\begin{split}P(x) = P(x) = \frac{e^{-(x-\mu)/s}}{s(1+e^{-(x-\mu)/s})^2},\end{split}\notag
\end{gather}
where \(\mu\) = location and \(s\) = scale.

The Logistic distribution is used in Extreme Value problems where it
can act as a mixture of Gumbel distributions, in Epidemiology, and by
the World Chess Federation (FIDE) where it is used in the Elo ranking
system, assuming the performance of each player is a logistically
distributed random variable.

Draw samples from the distribution:

\begin{Verbatim}[commandchars=\\\{\}]
\PYG{g+gp}{\PYGZgt{}\PYGZgt{}\PYGZgt{} }\PYG{n}{loc}\PYG{p}{,} \PYG{n}{scale} \PYG{o}{=} \PYG{l+m+mi}{10}\PYG{p}{,} \PYG{l+m+mi}{1}
\PYG{g+gp}{\PYGZgt{}\PYGZgt{}\PYGZgt{} }\PYG{n}{s} \PYG{o}{=} \PYG{n}{np}\PYG{o}{.}\PYG{n}{random}\PYG{o}{.}\PYG{n}{logistic}\PYG{p}{(}\PYG{n}{loc}\PYG{p}{,} \PYG{n}{scale}\PYG{p}{,} \PYG{l+m+mi}{10000}\PYG{p}{)}
\PYG{g+gp}{\PYGZgt{}\PYGZgt{}\PYGZgt{} }\PYG{n}{count}\PYG{p}{,} \PYG{n}{bins}\PYG{p}{,} \PYG{n}{ignored} \PYG{o}{=} \PYG{n}{plt}\PYG{o}{.}\PYG{n}{hist}\PYG{p}{(}\PYG{n}{s}\PYG{p}{,} \PYG{n}{bins}\PYG{o}{=}\PYG{l+m+mi}{50}\PYG{p}{)}
\end{Verbatim}

\#   plot against distribution

\begin{Verbatim}[commandchars=\\\{\}]
\PYG{g+gp}{\PYGZgt{}\PYGZgt{}\PYGZgt{} }\PYG{k}{def} \PYG{n+nf}{logist}\PYG{p}{(}\PYG{n}{x}\PYG{p}{,} \PYG{n}{loc}\PYG{p}{,} \PYG{n}{scale}\PYG{p}{)}\PYG{p}{:}
\PYG{g+gp}{... }    \PYG{k}{return} \PYG{n}{exp}\PYG{p}{(}\PYG{p}{(}\PYG{n}{loc}\PYG{o}{\PYGZhy{}}\PYG{n}{x}\PYG{p}{)}\PYG{o}{/}\PYG{n}{scale}\PYG{p}{)}\PYG{o}{/}\PYG{p}{(}\PYG{n}{scale}\PYG{o}{*}\PYG{p}{(}\PYG{l+m+mi}{1}\PYG{o}{+}\PYG{n}{exp}\PYG{p}{(}\PYG{p}{(}\PYG{n}{loc}\PYG{o}{\PYGZhy{}}\PYG{n}{x}\PYG{p}{)}\PYG{o}{/}\PYG{n}{scale}\PYG{p}{)}\PYG{p}{)}\PYG{o}{*}\PYG{o}{*}\PYG{l+m+mi}{2}\PYG{p}{)}
\PYG{g+gp}{\PYGZgt{}\PYGZgt{}\PYGZgt{} }\PYG{n}{plt}\PYG{o}{.}\PYG{n}{plot}\PYG{p}{(}\PYG{n}{bins}\PYG{p}{,} \PYG{n}{logist}\PYG{p}{(}\PYG{n}{bins}\PYG{p}{,} \PYG{n}{loc}\PYG{p}{,} \PYG{n}{scale}\PYG{p}{)}\PYG{o}{*}\PYG{n}{count}\PYG{o}{.}\PYG{n}{max}\PYG{p}{(}\PYG{p}{)}\PYG{o}{/}\PYGZbs{}
\PYG{g+gp}{... }\PYG{n}{logist}\PYG{p}{(}\PYG{n}{bins}\PYG{p}{,} \PYG{n}{loc}\PYG{p}{,} \PYG{n}{scale}\PYG{p}{)}\PYG{o}{.}\PYG{n}{max}\PYG{p}{(}\PYG{p}{)}\PYG{p}{)}
\PYG{g+gp}{\PYGZgt{}\PYGZgt{}\PYGZgt{} }\PYG{n}{plt}\PYG{o}{.}\PYG{n}{show}\PYG{p}{(}\PYG{p}{)}
\end{Verbatim}

\end{fulllineitems}

\index{lognormal() (in module topology\_analysis)}

\begin{fulllineitems}
\phantomsection\label{topology_analysis:topology_analysis.lognormal}\pysiglinewithargsret{\code{topology\_analysis.}\bfcode{lognormal}}{\emph{mean=0.0}, \emph{sigma=1.0}, \emph{size=None}}{}
Return samples drawn from a log-normal distribution.

Draw samples from a log-normal distribution with specified mean,
standard deviation, and array shape.  Note that the mean and standard
deviation are not the values for the distribution itself, but of the
underlying normal distribution it is derived from.
\begin{description}
\item[{mean}] \leavevmode{[}float{]}
Mean value of the underlying normal distribution

\item[{sigma}] \leavevmode{[}float, \textgreater{} 0.{]}
Standard deviation of the underlying normal distribution

\item[{size}] \leavevmode{[}tuple of ints{]}
Output shape.  If the given shape is, e.g., \code{(m, n, k)}, then
\code{m * n * k} samples are drawn.

\end{description}
\begin{description}
\item[{samples}] \leavevmode{[}ndarray or float{]}
The desired samples. An array of the same shape as \emph{size} if given,
if \emph{size} is None a float is returned.

\end{description}
\begin{description}
\item[{scipy.stats.lognorm}] \leavevmode{[}probability density function, distribution,{]}
cumulative density function, etc.

\end{description}

A variable \emph{x} has a log-normal distribution if \emph{log(x)} is normally
distributed.  The probability density function for the log-normal
distribution is:
\begin{gather}
\begin{split}p(x) = \frac{1}{\sigma x \sqrt{2\pi}}
e^{(-\frac{(ln(x)-\mu)^2}{2\sigma^2})}\end{split}\notag
\end{gather}
where \(\mu\) is the mean and \(\sigma\) is the standard
deviation of the normally distributed logarithm of the variable.
A log-normal distribution results if a random variable is the \emph{product}
of a large number of independent, identically-distributed variables in
the same way that a normal distribution results if the variable is the
\emph{sum} of a large number of independent, identically-distributed
variables.

Limpert, E., Stahel, W. A., and Abbt, M., ``Log-normal Distributions
across the Sciences: Keys and Clues,'' \emph{BioScience}, Vol. 51, No. 5,
May, 2001.  \href{http://stat.ethz.ch/~stahel/lognormal/bioscience.pdf}{http://stat.ethz.ch/\textasciitilde{}stahel/lognormal/bioscience.pdf}

Reiss, R.D. and Thomas, M., \emph{Statistical Analysis of Extreme Values},
Basel: Birkhauser Verlag, 2001, pp. 31-32.

Draw samples from the distribution:

\begin{Verbatim}[commandchars=\\\{\}]
\PYG{g+gp}{\PYGZgt{}\PYGZgt{}\PYGZgt{} }\PYG{n}{mu}\PYG{p}{,} \PYG{n}{sigma} \PYG{o}{=} \PYG{l+m+mf}{3.}\PYG{p}{,} \PYG{l+m+mf}{1.} \PYG{c}{\PYGZsh{} mean and standard deviation}
\PYG{g+gp}{\PYGZgt{}\PYGZgt{}\PYGZgt{} }\PYG{n}{s} \PYG{o}{=} \PYG{n}{np}\PYG{o}{.}\PYG{n}{random}\PYG{o}{.}\PYG{n}{lognormal}\PYG{p}{(}\PYG{n}{mu}\PYG{p}{,} \PYG{n}{sigma}\PYG{p}{,} \PYG{l+m+mi}{1000}\PYG{p}{)}
\end{Verbatim}

Display the histogram of the samples, along with
the probability density function:

\begin{Verbatim}[commandchars=\\\{\}]
\PYG{g+gp}{\PYGZgt{}\PYGZgt{}\PYGZgt{} }\PYG{k+kn}{import} \PYG{n+nn}{matplotlib.pyplot} \PYG{k+kn}{as} \PYG{n+nn}{plt}
\PYG{g+gp}{\PYGZgt{}\PYGZgt{}\PYGZgt{} }\PYG{n}{count}\PYG{p}{,} \PYG{n}{bins}\PYG{p}{,} \PYG{n}{ignored} \PYG{o}{=} \PYG{n}{plt}\PYG{o}{.}\PYG{n}{hist}\PYG{p}{(}\PYG{n}{s}\PYG{p}{,} \PYG{l+m+mi}{100}\PYG{p}{,} \PYG{n}{normed}\PYG{o}{=}\PYG{n+nb+bp}{True}\PYG{p}{,} \PYG{n}{align}\PYG{o}{=}\PYG{l+s}{\PYGZsq{}}\PYG{l+s}{mid}\PYG{l+s}{\PYGZsq{}}\PYG{p}{)}
\end{Verbatim}

\begin{Verbatim}[commandchars=\\\{\}]
\PYG{g+gp}{\PYGZgt{}\PYGZgt{}\PYGZgt{} }\PYG{n}{x} \PYG{o}{=} \PYG{n}{np}\PYG{o}{.}\PYG{n}{linspace}\PYG{p}{(}\PYG{n+nb}{min}\PYG{p}{(}\PYG{n}{bins}\PYG{p}{)}\PYG{p}{,} \PYG{n+nb}{max}\PYG{p}{(}\PYG{n}{bins}\PYG{p}{)}\PYG{p}{,} \PYG{l+m+mi}{10000}\PYG{p}{)}
\PYG{g+gp}{\PYGZgt{}\PYGZgt{}\PYGZgt{} }\PYG{n}{pdf} \PYG{o}{=} \PYG{p}{(}\PYG{n}{np}\PYG{o}{.}\PYG{n}{exp}\PYG{p}{(}\PYG{o}{\PYGZhy{}}\PYG{p}{(}\PYG{n}{np}\PYG{o}{.}\PYG{n}{log}\PYG{p}{(}\PYG{n}{x}\PYG{p}{)} \PYG{o}{\PYGZhy{}} \PYG{n}{mu}\PYG{p}{)}\PYG{o}{*}\PYG{o}{*}\PYG{l+m+mi}{2} \PYG{o}{/} \PYG{p}{(}\PYG{l+m+mi}{2} \PYG{o}{*} \PYG{n}{sigma}\PYG{o}{*}\PYG{o}{*}\PYG{l+m+mi}{2}\PYG{p}{)}\PYG{p}{)}
\PYG{g+gp}{... }       \PYG{o}{/} \PYG{p}{(}\PYG{n}{x} \PYG{o}{*} \PYG{n}{sigma} \PYG{o}{*} \PYG{n}{np}\PYG{o}{.}\PYG{n}{sqrt}\PYG{p}{(}\PYG{l+m+mi}{2} \PYG{o}{*} \PYG{n}{np}\PYG{o}{.}\PYG{n}{pi}\PYG{p}{)}\PYG{p}{)}\PYG{p}{)}
\end{Verbatim}

\begin{Verbatim}[commandchars=\\\{\}]
\PYG{g+gp}{\PYGZgt{}\PYGZgt{}\PYGZgt{} }\PYG{n}{plt}\PYG{o}{.}\PYG{n}{plot}\PYG{p}{(}\PYG{n}{x}\PYG{p}{,} \PYG{n}{pdf}\PYG{p}{,} \PYG{n}{linewidth}\PYG{o}{=}\PYG{l+m+mi}{2}\PYG{p}{,} \PYG{n}{color}\PYG{o}{=}\PYG{l+s}{\PYGZsq{}}\PYG{l+s}{r}\PYG{l+s}{\PYGZsq{}}\PYG{p}{)}
\PYG{g+gp}{\PYGZgt{}\PYGZgt{}\PYGZgt{} }\PYG{n}{plt}\PYG{o}{.}\PYG{n}{axis}\PYG{p}{(}\PYG{l+s}{\PYGZsq{}}\PYG{l+s}{tight}\PYG{l+s}{\PYGZsq{}}\PYG{p}{)}
\PYG{g+gp}{\PYGZgt{}\PYGZgt{}\PYGZgt{} }\PYG{n}{plt}\PYG{o}{.}\PYG{n}{show}\PYG{p}{(}\PYG{p}{)}
\end{Verbatim}

Demonstrate that taking the products of random samples from a uniform
distribution can be fit well by a log-normal probability density function.

\begin{Verbatim}[commandchars=\\\{\}]
\PYG{g+gp}{\PYGZgt{}\PYGZgt{}\PYGZgt{} }\PYG{c}{\PYGZsh{} Generate a thousand samples: each is the product of 100 random}
\PYG{g+gp}{\PYGZgt{}\PYGZgt{}\PYGZgt{} }\PYG{c}{\PYGZsh{} values, drawn from a normal distribution.}
\PYG{g+gp}{\PYGZgt{}\PYGZgt{}\PYGZgt{} }\PYG{n}{b} \PYG{o}{=} \PYG{p}{[}\PYG{p}{]}
\PYG{g+gp}{\PYGZgt{}\PYGZgt{}\PYGZgt{} }\PYG{k}{for} \PYG{n}{i} \PYG{o+ow}{in} \PYG{n+nb}{range}\PYG{p}{(}\PYG{l+m+mi}{1000}\PYG{p}{)}\PYG{p}{:}
\PYG{g+gp}{... }   \PYG{n}{a} \PYG{o}{=} \PYG{l+m+mf}{10.} \PYG{o}{+} \PYG{n}{np}\PYG{o}{.}\PYG{n}{random}\PYG{o}{.}\PYG{n}{random}\PYG{p}{(}\PYG{l+m+mi}{100}\PYG{p}{)}
\PYG{g+gp}{... }   \PYG{n}{b}\PYG{o}{.}\PYG{n}{append}\PYG{p}{(}\PYG{n}{np}\PYG{o}{.}\PYG{n}{product}\PYG{p}{(}\PYG{n}{a}\PYG{p}{)}\PYG{p}{)}
\end{Verbatim}

\begin{Verbatim}[commandchars=\\\{\}]
\PYG{g+gp}{\PYGZgt{}\PYGZgt{}\PYGZgt{} }\PYG{n}{b} \PYG{o}{=} \PYG{n}{np}\PYG{o}{.}\PYG{n}{array}\PYG{p}{(}\PYG{n}{b}\PYG{p}{)} \PYG{o}{/} \PYG{n}{np}\PYG{o}{.}\PYG{n}{min}\PYG{p}{(}\PYG{n}{b}\PYG{p}{)} \PYG{c}{\PYGZsh{} scale values to be positive}
\PYG{g+gp}{\PYGZgt{}\PYGZgt{}\PYGZgt{} }\PYG{n}{count}\PYG{p}{,} \PYG{n}{bins}\PYG{p}{,} \PYG{n}{ignored} \PYG{o}{=} \PYG{n}{plt}\PYG{o}{.}\PYG{n}{hist}\PYG{p}{(}\PYG{n}{b}\PYG{p}{,} \PYG{l+m+mi}{100}\PYG{p}{,} \PYG{n}{normed}\PYG{o}{=}\PYG{n+nb+bp}{True}\PYG{p}{,} \PYG{n}{align}\PYG{o}{=}\PYG{l+s}{\PYGZsq{}}\PYG{l+s}{center}\PYG{l+s}{\PYGZsq{}}\PYG{p}{)}
\PYG{g+gp}{\PYGZgt{}\PYGZgt{}\PYGZgt{} }\PYG{n}{sigma} \PYG{o}{=} \PYG{n}{np}\PYG{o}{.}\PYG{n}{std}\PYG{p}{(}\PYG{n}{np}\PYG{o}{.}\PYG{n}{log}\PYG{p}{(}\PYG{n}{b}\PYG{p}{)}\PYG{p}{)}
\PYG{g+gp}{\PYGZgt{}\PYGZgt{}\PYGZgt{} }\PYG{n}{mu} \PYG{o}{=} \PYG{n}{np}\PYG{o}{.}\PYG{n}{mean}\PYG{p}{(}\PYG{n}{np}\PYG{o}{.}\PYG{n}{log}\PYG{p}{(}\PYG{n}{b}\PYG{p}{)}\PYG{p}{)}
\end{Verbatim}

\begin{Verbatim}[commandchars=\\\{\}]
\PYG{g+gp}{\PYGZgt{}\PYGZgt{}\PYGZgt{} }\PYG{n}{x} \PYG{o}{=} \PYG{n}{np}\PYG{o}{.}\PYG{n}{linspace}\PYG{p}{(}\PYG{n+nb}{min}\PYG{p}{(}\PYG{n}{bins}\PYG{p}{)}\PYG{p}{,} \PYG{n+nb}{max}\PYG{p}{(}\PYG{n}{bins}\PYG{p}{)}\PYG{p}{,} \PYG{l+m+mi}{10000}\PYG{p}{)}
\PYG{g+gp}{\PYGZgt{}\PYGZgt{}\PYGZgt{} }\PYG{n}{pdf} \PYG{o}{=} \PYG{p}{(}\PYG{n}{np}\PYG{o}{.}\PYG{n}{exp}\PYG{p}{(}\PYG{o}{\PYGZhy{}}\PYG{p}{(}\PYG{n}{np}\PYG{o}{.}\PYG{n}{log}\PYG{p}{(}\PYG{n}{x}\PYG{p}{)} \PYG{o}{\PYGZhy{}} \PYG{n}{mu}\PYG{p}{)}\PYG{o}{*}\PYG{o}{*}\PYG{l+m+mi}{2} \PYG{o}{/} \PYG{p}{(}\PYG{l+m+mi}{2} \PYG{o}{*} \PYG{n}{sigma}\PYG{o}{*}\PYG{o}{*}\PYG{l+m+mi}{2}\PYG{p}{)}\PYG{p}{)}
\PYG{g+gp}{... }       \PYG{o}{/} \PYG{p}{(}\PYG{n}{x} \PYG{o}{*} \PYG{n}{sigma} \PYG{o}{*} \PYG{n}{np}\PYG{o}{.}\PYG{n}{sqrt}\PYG{p}{(}\PYG{l+m+mi}{2} \PYG{o}{*} \PYG{n}{np}\PYG{o}{.}\PYG{n}{pi}\PYG{p}{)}\PYG{p}{)}\PYG{p}{)}
\end{Verbatim}

\begin{Verbatim}[commandchars=\\\{\}]
\PYG{g+gp}{\PYGZgt{}\PYGZgt{}\PYGZgt{} }\PYG{n}{plt}\PYG{o}{.}\PYG{n}{plot}\PYG{p}{(}\PYG{n}{x}\PYG{p}{,} \PYG{n}{pdf}\PYG{p}{,} \PYG{n}{color}\PYG{o}{=}\PYG{l+s}{\PYGZsq{}}\PYG{l+s}{r}\PYG{l+s}{\PYGZsq{}}\PYG{p}{,} \PYG{n}{linewidth}\PYG{o}{=}\PYG{l+m+mi}{2}\PYG{p}{)}
\PYG{g+gp}{\PYGZgt{}\PYGZgt{}\PYGZgt{} }\PYG{n}{plt}\PYG{o}{.}\PYG{n}{show}\PYG{p}{(}\PYG{p}{)}
\end{Verbatim}

\end{fulllineitems}

\index{logseries() (in module topology\_analysis)}

\begin{fulllineitems}
\phantomsection\label{topology_analysis:topology_analysis.logseries}\pysiglinewithargsret{\code{topology\_analysis.}\bfcode{logseries}}{\emph{p}, \emph{size=None}}{}
Draw samples from a Logarithmic Series distribution.

Samples are drawn from a Log Series distribution with specified
parameter, p (probability, 0 \textless{} p \textless{} 1).

loc : float

scale : float \textgreater{} 0.
\begin{description}
\item[{size}] \leavevmode{[}\{tuple, int\}{]}
Output shape.  If the given shape is, e.g., \code{(m, n, k)}, then
\code{m * n * k} samples are drawn.

\end{description}
\begin{description}
\item[{samples}] \leavevmode{[}\{ndarray, scalar\}{]}
where the values are all integers in  {[}0, n{]}.

\end{description}
\begin{description}
\item[{scipy.stats.distributions.logser}] \leavevmode{[}probability density function,{]}
distribution or cumulative density function, etc.

\end{description}

The probability density for the Log Series distribution is
\begin{gather}
\begin{split}P(k) = \frac{-p^k}{k \ln(1-p)},\end{split}\notag
\end{gather}
where p = probability.

The Log Series distribution is frequently used to represent species
richness and occurrence, first proposed by Fisher, Corbet, and
Williams in 1943 {[}2{]}.  It may also be used to model the numbers of
occupants seen in cars {[}3{]}.

Draw samples from the distribution:

\begin{Verbatim}[commandchars=\\\{\}]
\PYG{g+gp}{\PYGZgt{}\PYGZgt{}\PYGZgt{} }\PYG{n}{a} \PYG{o}{=} \PYG{o}{.}\PYG{l+m+mi}{6}
\PYG{g+gp}{\PYGZgt{}\PYGZgt{}\PYGZgt{} }\PYG{n}{s} \PYG{o}{=} \PYG{n}{np}\PYG{o}{.}\PYG{n}{random}\PYG{o}{.}\PYG{n}{logseries}\PYG{p}{(}\PYG{n}{a}\PYG{p}{,} \PYG{l+m+mi}{10000}\PYG{p}{)}
\PYG{g+gp}{\PYGZgt{}\PYGZgt{}\PYGZgt{} }\PYG{n}{count}\PYG{p}{,} \PYG{n}{bins}\PYG{p}{,} \PYG{n}{ignored} \PYG{o}{=} \PYG{n}{plt}\PYG{o}{.}\PYG{n}{hist}\PYG{p}{(}\PYG{n}{s}\PYG{p}{)}
\end{Verbatim}

\#   plot against distribution

\begin{Verbatim}[commandchars=\\\{\}]
\PYG{g+gp}{\PYGZgt{}\PYGZgt{}\PYGZgt{} }\PYG{k}{def} \PYG{n+nf}{logseries}\PYG{p}{(}\PYG{n}{k}\PYG{p}{,} \PYG{n}{p}\PYG{p}{)}\PYG{p}{:}
\PYG{g+gp}{... }    \PYG{k}{return} \PYG{o}{\PYGZhy{}}\PYG{n}{p}\PYG{o}{*}\PYG{o}{*}\PYG{n}{k}\PYG{o}{/}\PYG{p}{(}\PYG{n}{k}\PYG{o}{*}\PYG{n}{log}\PYG{p}{(}\PYG{l+m+mi}{1}\PYG{o}{\PYGZhy{}}\PYG{n}{p}\PYG{p}{)}\PYG{p}{)}
\PYG{g+gp}{\PYGZgt{}\PYGZgt{}\PYGZgt{} }\PYG{n}{plt}\PYG{o}{.}\PYG{n}{plot}\PYG{p}{(}\PYG{n}{bins}\PYG{p}{,} \PYG{n}{logseries}\PYG{p}{(}\PYG{n}{bins}\PYG{p}{,} \PYG{n}{a}\PYG{p}{)}\PYG{o}{*}\PYG{n}{count}\PYG{o}{.}\PYG{n}{max}\PYG{p}{(}\PYG{p}{)}\PYG{o}{/}
\PYG{g+go}{             logseries(bins, a).max(), \PYGZsq{}r\PYGZsq{})}
\PYG{g+gp}{\PYGZgt{}\PYGZgt{}\PYGZgt{} }\PYG{n}{plt}\PYG{o}{.}\PYG{n}{show}\PYG{p}{(}\PYG{p}{)}
\end{Verbatim}

\end{fulllineitems}

\index{multinomial() (in module topology\_analysis)}

\begin{fulllineitems}
\phantomsection\label{topology_analysis:topology_analysis.multinomial}\pysiglinewithargsret{\code{topology\_analysis.}\bfcode{multinomial}}{\emph{n}, \emph{pvals}, \emph{size=None}}{}
Draw samples from a multinomial distribution.

The multinomial distribution is a multivariate generalisation of the
binomial distribution.  Take an experiment with one of \code{p}
possible outcomes.  An example of such an experiment is throwing a dice,
where the outcome can be 1 through 6.  Each sample drawn from the
distribution represents \emph{n} such experiments.  Its values,
\code{X\_i = {[}X\_0, X\_1, ..., X\_p{]}}, represent the number of times the outcome
was \code{i}.
\begin{description}
\item[{n}] \leavevmode{[}int{]}
Number of experiments.

\item[{pvals}] \leavevmode{[}sequence of floats, length p{]}
Probabilities of each of the \code{p} different outcomes.  These
should sum to 1 (however, the last element is always assumed to
account for the remaining probability, as long as
\code{sum(pvals{[}:-1{]}) \textless{}= 1)}.

\item[{size}] \leavevmode{[}tuple of ints{]}
Given a \emph{size} of \code{(M, N, K)}, then \code{M*N*K} samples are drawn,
and the output shape becomes \code{(M, N, K, p)}, since each sample
has shape \code{(p,)}.

\end{description}

Throw a dice 20 times:

\begin{Verbatim}[commandchars=\\\{\}]
\PYG{g+gp}{\PYGZgt{}\PYGZgt{}\PYGZgt{} }\PYG{n}{np}\PYG{o}{.}\PYG{n}{random}\PYG{o}{.}\PYG{n}{multinomial}\PYG{p}{(}\PYG{l+m+mi}{20}\PYG{p}{,} \PYG{p}{[}\PYG{l+m+mi}{1}\PYG{o}{/}\PYG{l+m+mf}{6.}\PYG{p}{]}\PYG{o}{*}\PYG{l+m+mi}{6}\PYG{p}{,} \PYG{n}{size}\PYG{o}{=}\PYG{l+m+mi}{1}\PYG{p}{)}
\PYG{g+go}{array([[4, 1, 7, 5, 2, 1]])}
\end{Verbatim}

It landed 4 times on 1, once on 2, etc.

Now, throw the dice 20 times, and 20 times again:

\begin{Verbatim}[commandchars=\\\{\}]
\PYG{g+gp}{\PYGZgt{}\PYGZgt{}\PYGZgt{} }\PYG{n}{np}\PYG{o}{.}\PYG{n}{random}\PYG{o}{.}\PYG{n}{multinomial}\PYG{p}{(}\PYG{l+m+mi}{20}\PYG{p}{,} \PYG{p}{[}\PYG{l+m+mi}{1}\PYG{o}{/}\PYG{l+m+mf}{6.}\PYG{p}{]}\PYG{o}{*}\PYG{l+m+mi}{6}\PYG{p}{,} \PYG{n}{size}\PYG{o}{=}\PYG{l+m+mi}{2}\PYG{p}{)}
\PYG{g+go}{array([[3, 4, 3, 3, 4, 3],}
\PYG{g+go}{       [2, 4, 3, 4, 0, 7]])}
\end{Verbatim}

For the first run, we threw 3 times 1, 4 times 2, etc.  For the second,
we threw 2 times 1, 4 times 2, etc.

A loaded dice is more likely to land on number 6:

\begin{Verbatim}[commandchars=\\\{\}]
\PYG{g+gp}{\PYGZgt{}\PYGZgt{}\PYGZgt{} }\PYG{n}{np}\PYG{o}{.}\PYG{n}{random}\PYG{o}{.}\PYG{n}{multinomial}\PYG{p}{(}\PYG{l+m+mi}{100}\PYG{p}{,} \PYG{p}{[}\PYG{l+m+mi}{1}\PYG{o}{/}\PYG{l+m+mf}{7.}\PYG{p}{]}\PYG{o}{*}\PYG{l+m+mi}{5}\PYG{p}{)}
\PYG{g+go}{array([13, 16, 13, 16, 42])}
\end{Verbatim}

\end{fulllineitems}

\index{multivariate\_normal() (in module topology\_analysis)}

\begin{fulllineitems}
\phantomsection\label{topology_analysis:topology_analysis.multivariate_normal}\pysiglinewithargsret{\code{topology\_analysis.}\bfcode{multivariate\_normal}}{\emph{mean}, \emph{cov}\optional{, \emph{size}}}{}
Draw random samples from a multivariate normal distribution.

The multivariate normal, multinormal or Gaussian distribution is a
generalization of the one-dimensional normal distribution to higher
dimensions.  Such a distribution is specified by its mean and
covariance matrix.  These parameters are analogous to the mean
(average or ``center'') and variance (standard deviation, or ``width,''
squared) of the one-dimensional normal distribution.
\begin{description}
\item[{mean}] \leavevmode{[}1-D array\_like, of length N{]}
Mean of the N-dimensional distribution.

\item[{cov}] \leavevmode{[}2-D array\_like, of shape (N, N){]}
Covariance matrix of the distribution.  Must be symmetric and
positive semi-definite for ``physically meaningful'' results.

\item[{size}] \leavevmode{[}int or tuple of ints, optional{]}
Given a shape of, for example, \code{(m,n,k)}, \code{m*n*k} samples are
generated, and packed in an \emph{m}-by-\emph{n}-by-\emph{k} arrangement.  Because
each sample is \emph{N}-dimensional, the output shape is \code{(m,n,k,N)}.
If no shape is specified, a single (\emph{N}-D) sample is returned.

\end{description}
\begin{description}
\item[{out}] \leavevmode{[}ndarray{]}
The drawn samples, of shape \emph{size}, if that was provided.  If not,
the shape is \code{(N,)}.

In other words, each entry \code{out{[}i,j,...,:{]}} is an N-dimensional
value drawn from the distribution.

\end{description}

The mean is a coordinate in N-dimensional space, which represents the
location where samples are most likely to be generated.  This is
analogous to the peak of the bell curve for the one-dimensional or
univariate normal distribution.

Covariance indicates the level to which two variables vary together.
From the multivariate normal distribution, we draw N-dimensional
samples, \(X = [x_1, x_2, ... x_N]\).  The covariance matrix
element \(C_{ij}\) is the covariance of \(x_i\) and \(x_j\).
The element \(C_{ii}\) is the variance of \(x_i\) (i.e. its
``spread'').

Instead of specifying the full covariance matrix, popular
approximations include:
\begin{itemize}
\item {} 
Spherical covariance (\emph{cov} is a multiple of the identity matrix)

\item {} 
Diagonal covariance (\emph{cov} has non-negative elements, and only on
the diagonal)

\end{itemize}

This geometrical property can be seen in two dimensions by plotting
generated data-points:

\begin{Verbatim}[commandchars=\\\{\}]
\PYG{g+gp}{\PYGZgt{}\PYGZgt{}\PYGZgt{} }\PYG{n}{mean} \PYG{o}{=} \PYG{p}{[}\PYG{l+m+mi}{0}\PYG{p}{,}\PYG{l+m+mi}{0}\PYG{p}{]}
\PYG{g+gp}{\PYGZgt{}\PYGZgt{}\PYGZgt{} }\PYG{n}{cov} \PYG{o}{=} \PYG{p}{[}\PYG{p}{[}\PYG{l+m+mi}{1}\PYG{p}{,}\PYG{l+m+mi}{0}\PYG{p}{]}\PYG{p}{,}\PYG{p}{[}\PYG{l+m+mi}{0}\PYG{p}{,}\PYG{l+m+mi}{100}\PYG{p}{]}\PYG{p}{]} \PYG{c}{\PYGZsh{} diagonal covariance, points lie on x or y\PYGZhy{}axis}
\end{Verbatim}

\begin{Verbatim}[commandchars=\\\{\}]
\PYG{g+gp}{\PYGZgt{}\PYGZgt{}\PYGZgt{} }\PYG{k+kn}{import} \PYG{n+nn}{matplotlib.pyplot} \PYG{k+kn}{as} \PYG{n+nn}{plt}
\PYG{g+gp}{\PYGZgt{}\PYGZgt{}\PYGZgt{} }\PYG{n}{x}\PYG{p}{,}\PYG{n}{y} \PYG{o}{=} \PYG{n}{np}\PYG{o}{.}\PYG{n}{random}\PYG{o}{.}\PYG{n}{multivariate\PYGZus{}normal}\PYG{p}{(}\PYG{n}{mean}\PYG{p}{,}\PYG{n}{cov}\PYG{p}{,}\PYG{l+m+mi}{5000}\PYG{p}{)}\PYG{o}{.}\PYG{n}{T}
\PYG{g+gp}{\PYGZgt{}\PYGZgt{}\PYGZgt{} }\PYG{n}{plt}\PYG{o}{.}\PYG{n}{plot}\PYG{p}{(}\PYG{n}{x}\PYG{p}{,}\PYG{n}{y}\PYG{p}{,}\PYG{l+s}{\PYGZsq{}}\PYG{l+s}{x}\PYG{l+s}{\PYGZsq{}}\PYG{p}{)}\PYG{p}{;} \PYG{n}{plt}\PYG{o}{.}\PYG{n}{axis}\PYG{p}{(}\PYG{l+s}{\PYGZsq{}}\PYG{l+s}{equal}\PYG{l+s}{\PYGZsq{}}\PYG{p}{)}\PYG{p}{;} \PYG{n}{plt}\PYG{o}{.}\PYG{n}{show}\PYG{p}{(}\PYG{p}{)}
\end{Verbatim}

Note that the covariance matrix must be non-negative definite.

Papoulis, A., \emph{Probability, Random Variables, and Stochastic Processes},
3rd ed., New York: McGraw-Hill, 1991.

Duda, R. O., Hart, P. E., and Stork, D. G., \emph{Pattern Classification},
2nd ed., New York: Wiley, 2001.

\begin{Verbatim}[commandchars=\\\{\}]
\PYG{g+gp}{\PYGZgt{}\PYGZgt{}\PYGZgt{} }\PYG{n}{mean} \PYG{o}{=} \PYG{p}{(}\PYG{l+m+mi}{1}\PYG{p}{,}\PYG{l+m+mi}{2}\PYG{p}{)}
\PYG{g+gp}{\PYGZgt{}\PYGZgt{}\PYGZgt{} }\PYG{n}{cov} \PYG{o}{=} \PYG{p}{[}\PYG{p}{[}\PYG{l+m+mi}{1}\PYG{p}{,}\PYG{l+m+mi}{0}\PYG{p}{]}\PYG{p}{,}\PYG{p}{[}\PYG{l+m+mi}{1}\PYG{p}{,}\PYG{l+m+mi}{0}\PYG{p}{]}\PYG{p}{]}
\PYG{g+gp}{\PYGZgt{}\PYGZgt{}\PYGZgt{} }\PYG{n}{x} \PYG{o}{=} \PYG{n}{np}\PYG{o}{.}\PYG{n}{random}\PYG{o}{.}\PYG{n}{multivariate\PYGZus{}normal}\PYG{p}{(}\PYG{n}{mean}\PYG{p}{,}\PYG{n}{cov}\PYG{p}{,}\PYG{p}{(}\PYG{l+m+mi}{3}\PYG{p}{,}\PYG{l+m+mi}{3}\PYG{p}{)}\PYG{p}{)}
\PYG{g+gp}{\PYGZgt{}\PYGZgt{}\PYGZgt{} }\PYG{n}{x}\PYG{o}{.}\PYG{n}{shape}
\PYG{g+go}{(3, 3, 2)}
\end{Verbatim}

The following is probably true, given that 0.6 is roughly twice the
standard deviation:

\begin{Verbatim}[commandchars=\\\{\}]
\PYG{g+gp}{\PYGZgt{}\PYGZgt{}\PYGZgt{} }\PYG{k}{print} \PYG{n+nb}{list}\PYG{p}{(} \PYG{p}{(}\PYG{n}{x}\PYG{p}{[}\PYG{l+m+mi}{0}\PYG{p}{,}\PYG{l+m+mi}{0}\PYG{p}{,}\PYG{p}{:}\PYG{p}{]} \PYG{o}{\PYGZhy{}} \PYG{n}{mean}\PYG{p}{)} \PYG{o}{\PYGZlt{}} \PYG{l+m+mf}{0.6} \PYG{p}{)}
\PYG{g+go}{[True, True]}
\end{Verbatim}

\end{fulllineitems}

\index{negative\_binomial() (in module topology\_analysis)}

\begin{fulllineitems}
\phantomsection\label{topology_analysis:topology_analysis.negative_binomial}\pysiglinewithargsret{\code{topology\_analysis.}\bfcode{negative\_binomial}}{\emph{n}, \emph{p}, \emph{size=None}}{}
Draw samples from a negative\_binomial distribution.

Samples are drawn from a negative\_Binomial distribution with specified
parameters, \emph{n} trials and \emph{p} probability of success where \emph{n} is an
integer \textgreater{} 0 and \emph{p} is in the interval {[}0, 1{]}.
\begin{description}
\item[{n}] \leavevmode{[}int{]}
Parameter, \textgreater{} 0.

\item[{p}] \leavevmode{[}float{]}
Parameter, \textgreater{}= 0 and \textless{}=1.

\item[{size}] \leavevmode{[}int or tuple of ints{]}
Output shape. If the given shape is, e.g., \code{(m, n, k)}, then
\code{m * n * k} samples are drawn.

\end{description}
\begin{description}
\item[{samples}] \leavevmode{[}int or ndarray of ints{]}
Drawn samples.

\end{description}

The probability density for the Negative Binomial distribution is
\begin{gather}
\begin{split}P(N;n,p) = \binom{N+n-1}{n-1}p^{n}(1-p)^{N},\end{split}\notag
\end{gather}
where \(n-1\) is the number of successes, \(p\) is the probability
of success, and \(N+n-1\) is the number of trials.

The negative binomial distribution gives the probability of n-1 successes
and N failures in N+n-1 trials, and success on the (N+n)th trial.

If one throws a die repeatedly until the third time a ``1'' appears, then the
probability distribution of the number of non-``1''s that appear before the
third ``1'' is a negative binomial distribution.

Draw samples from the distribution:

A real world example. A company drills wild-cat oil exploration wells, each
with an estimated probability of success of 0.1.  What is the probability
of having one success for each successive well, that is what is the
probability of a single success after drilling 5 wells, after 6 wells,
etc.?

\begin{Verbatim}[commandchars=\\\{\}]
\PYG{g+gp}{\PYGZgt{}\PYGZgt{}\PYGZgt{} }\PYG{n}{s} \PYG{o}{=} \PYG{n}{np}\PYG{o}{.}\PYG{n}{random}\PYG{o}{.}\PYG{n}{negative\PYGZus{}binomial}\PYG{p}{(}\PYG{l+m+mi}{1}\PYG{p}{,} \PYG{l+m+mf}{0.1}\PYG{p}{,} \PYG{l+m+mi}{100000}\PYG{p}{)}
\PYG{g+gp}{\PYGZgt{}\PYGZgt{}\PYGZgt{} }\PYG{k}{for} \PYG{n}{i} \PYG{o+ow}{in} \PYG{n+nb}{range}\PYG{p}{(}\PYG{l+m+mi}{1}\PYG{p}{,} \PYG{l+m+mi}{11}\PYG{p}{)}\PYG{p}{:}
\PYG{g+gp}{... }   \PYG{n}{probability} \PYG{o}{=} \PYG{n+nb}{sum}\PYG{p}{(}\PYG{n}{s}\PYG{o}{\PYGZlt{}}\PYG{n}{i}\PYG{p}{)} \PYG{o}{/} \PYG{l+m+mf}{100000.}
\PYG{g+gp}{... }   \PYG{k}{print} \PYG{n}{i}\PYG{p}{,} \PYG{l+s}{\PYGZdq{}}\PYG{l+s}{wells drilled, probability of one success =}\PYG{l+s}{\PYGZdq{}}\PYG{p}{,} \PYG{n}{probability}
\end{Verbatim}

\end{fulllineitems}

\index{noncentral\_chisquare() (in module topology\_analysis)}

\begin{fulllineitems}
\phantomsection\label{topology_analysis:topology_analysis.noncentral_chisquare}\pysiglinewithargsret{\code{topology\_analysis.}\bfcode{noncentral\_chisquare}}{\emph{df}, \emph{nonc}, \emph{size=None}}{}
Draw samples from a noncentral chi-square distribution.

The noncentral \(\chi^2\) distribution is a generalisation of
the \(\chi^2\) distribution.
\begin{description}
\item[{df}] \leavevmode{[}int{]}
Degrees of freedom, should be \textgreater{}= 1.

\item[{nonc}] \leavevmode{[}float{]}
Non-centrality, should be \textgreater{} 0.

\item[{size}] \leavevmode{[}int or tuple of ints{]}
Shape of the output.

\end{description}

The probability density function for the noncentral Chi-square distribution
is
\begin{gather}
\begin{split}P(x;df,nonc) = \sum^{\infty}_{i=0}
\frac{e^{-nonc/2}(nonc/2)^{i}}{i!}P_{Y_{df+2i}}(x),\end{split}\notag
\end{gather}
where \(Y_{q}\) is the Chi-square with q degrees of freedom.

In Delhi (2007), it is noted that the noncentral chi-square is useful in
bombing and coverage problems, the probability of killing the point target
given by the noncentral chi-squared distribution.

Draw values from the distribution and plot the histogram

\begin{Verbatim}[commandchars=\\\{\}]
\PYG{g+gp}{\PYGZgt{}\PYGZgt{}\PYGZgt{} }\PYG{k+kn}{import} \PYG{n+nn}{matplotlib.pyplot} \PYG{k+kn}{as} \PYG{n+nn}{plt}
\PYG{g+gp}{\PYGZgt{}\PYGZgt{}\PYGZgt{} }\PYG{n}{values} \PYG{o}{=} \PYG{n}{plt}\PYG{o}{.}\PYG{n}{hist}\PYG{p}{(}\PYG{n}{np}\PYG{o}{.}\PYG{n}{random}\PYG{o}{.}\PYG{n}{noncentral\PYGZus{}chisquare}\PYG{p}{(}\PYG{l+m+mi}{3}\PYG{p}{,} \PYG{l+m+mi}{20}\PYG{p}{,} \PYG{l+m+mi}{100000}\PYG{p}{)}\PYG{p}{,}
\PYG{g+gp}{... }                  \PYG{n}{bins}\PYG{o}{=}\PYG{l+m+mi}{200}\PYG{p}{,} \PYG{n}{normed}\PYG{o}{=}\PYG{n+nb+bp}{True}\PYG{p}{)}
\PYG{g+gp}{\PYGZgt{}\PYGZgt{}\PYGZgt{} }\PYG{n}{plt}\PYG{o}{.}\PYG{n}{show}\PYG{p}{(}\PYG{p}{)}
\end{Verbatim}

Draw values from a noncentral chisquare with very small noncentrality,
and compare to a chisquare.

\begin{Verbatim}[commandchars=\\\{\}]
\PYG{g+gp}{\PYGZgt{}\PYGZgt{}\PYGZgt{} }\PYG{n}{plt}\PYG{o}{.}\PYG{n}{figure}\PYG{p}{(}\PYG{p}{)}
\PYG{g+gp}{\PYGZgt{}\PYGZgt{}\PYGZgt{} }\PYG{n}{values} \PYG{o}{=} \PYG{n}{plt}\PYG{o}{.}\PYG{n}{hist}\PYG{p}{(}\PYG{n}{np}\PYG{o}{.}\PYG{n}{random}\PYG{o}{.}\PYG{n}{noncentral\PYGZus{}chisquare}\PYG{p}{(}\PYG{l+m+mi}{3}\PYG{p}{,} \PYG{o}{.}\PYG{l+m+mo}{0000001}\PYG{p}{,} \PYG{l+m+mi}{100000}\PYG{p}{)}\PYG{p}{,}
\PYG{g+gp}{... }                  \PYG{n}{bins}\PYG{o}{=}\PYG{n}{np}\PYG{o}{.}\PYG{n}{arange}\PYG{p}{(}\PYG{l+m+mf}{0.}\PYG{p}{,} \PYG{l+m+mi}{25}\PYG{p}{,} \PYG{o}{.}\PYG{l+m+mi}{1}\PYG{p}{)}\PYG{p}{,} \PYG{n}{normed}\PYG{o}{=}\PYG{n+nb+bp}{True}\PYG{p}{)}
\PYG{g+gp}{\PYGZgt{}\PYGZgt{}\PYGZgt{} }\PYG{n}{values2} \PYG{o}{=} \PYG{n}{plt}\PYG{o}{.}\PYG{n}{hist}\PYG{p}{(}\PYG{n}{np}\PYG{o}{.}\PYG{n}{random}\PYG{o}{.}\PYG{n}{chisquare}\PYG{p}{(}\PYG{l+m+mi}{3}\PYG{p}{,} \PYG{l+m+mi}{100000}\PYG{p}{)}\PYG{p}{,}
\PYG{g+gp}{... }                   \PYG{n}{bins}\PYG{o}{=}\PYG{n}{np}\PYG{o}{.}\PYG{n}{arange}\PYG{p}{(}\PYG{l+m+mf}{0.}\PYG{p}{,} \PYG{l+m+mi}{25}\PYG{p}{,} \PYG{o}{.}\PYG{l+m+mi}{1}\PYG{p}{)}\PYG{p}{,} \PYG{n}{normed}\PYG{o}{=}\PYG{n+nb+bp}{True}\PYG{p}{)}
\PYG{g+gp}{\PYGZgt{}\PYGZgt{}\PYGZgt{} }\PYG{n}{plt}\PYG{o}{.}\PYG{n}{plot}\PYG{p}{(}\PYG{n}{values}\PYG{p}{[}\PYG{l+m+mi}{1}\PYG{p}{]}\PYG{p}{[}\PYG{l+m+mi}{0}\PYG{p}{:}\PYG{o}{\PYGZhy{}}\PYG{l+m+mi}{1}\PYG{p}{]}\PYG{p}{,} \PYG{n}{values}\PYG{p}{[}\PYG{l+m+mi}{0}\PYG{p}{]}\PYG{o}{\PYGZhy{}}\PYG{n}{values2}\PYG{p}{[}\PYG{l+m+mi}{0}\PYG{p}{]}\PYG{p}{,} \PYG{l+s}{\PYGZsq{}}\PYG{l+s}{ob}\PYG{l+s}{\PYGZsq{}}\PYG{p}{)}
\PYG{g+gp}{\PYGZgt{}\PYGZgt{}\PYGZgt{} }\PYG{n}{plt}\PYG{o}{.}\PYG{n}{show}\PYG{p}{(}\PYG{p}{)}
\end{Verbatim}

Demonstrate how large values of non-centrality lead to a more symmetric
distribution.

\begin{Verbatim}[commandchars=\\\{\}]
\PYG{g+gp}{\PYGZgt{}\PYGZgt{}\PYGZgt{} }\PYG{n}{plt}\PYG{o}{.}\PYG{n}{figure}\PYG{p}{(}\PYG{p}{)}
\PYG{g+gp}{\PYGZgt{}\PYGZgt{}\PYGZgt{} }\PYG{n}{values} \PYG{o}{=} \PYG{n}{plt}\PYG{o}{.}\PYG{n}{hist}\PYG{p}{(}\PYG{n}{np}\PYG{o}{.}\PYG{n}{random}\PYG{o}{.}\PYG{n}{noncentral\PYGZus{}chisquare}\PYG{p}{(}\PYG{l+m+mi}{3}\PYG{p}{,} \PYG{l+m+mi}{20}\PYG{p}{,} \PYG{l+m+mi}{100000}\PYG{p}{)}\PYG{p}{,}
\PYG{g+gp}{... }                  \PYG{n}{bins}\PYG{o}{=}\PYG{l+m+mi}{200}\PYG{p}{,} \PYG{n}{normed}\PYG{o}{=}\PYG{n+nb+bp}{True}\PYG{p}{)}
\PYG{g+gp}{\PYGZgt{}\PYGZgt{}\PYGZgt{} }\PYG{n}{plt}\PYG{o}{.}\PYG{n}{show}\PYG{p}{(}\PYG{p}{)}
\end{Verbatim}

\end{fulllineitems}

\index{noncentral\_f() (in module topology\_analysis)}

\begin{fulllineitems}
\phantomsection\label{topology_analysis:topology_analysis.noncentral_f}\pysiglinewithargsret{\code{topology\_analysis.}\bfcode{noncentral\_f}}{\emph{dfnum}, \emph{dfden}, \emph{nonc}, \emph{size=None}}{}
Draw samples from the noncentral F distribution.

Samples are drawn from an F distribution with specified parameters,
\emph{dfnum} (degrees of freedom in numerator) and \emph{dfden} (degrees of
freedom in denominator), where both parameters \textgreater{} 1.
\emph{nonc} is the non-centrality parameter.
\begin{description}
\item[{dfnum}] \leavevmode{[}int{]}
Parameter, should be \textgreater{} 1.

\item[{dfden}] \leavevmode{[}int{]}
Parameter, should be \textgreater{} 1.

\item[{nonc}] \leavevmode{[}float{]}
Parameter, should be \textgreater{}= 0.

\item[{size}] \leavevmode{[}int or tuple of ints{]}
Output shape. If the given shape is, e.g., \code{(m, n, k)}, then
\code{m * n * k} samples are drawn.

\end{description}
\begin{description}
\item[{samples}] \leavevmode{[}scalar or ndarray{]}
Drawn samples.

\end{description}

When calculating the power of an experiment (power = probability of
rejecting the null hypothesis when a specific alternative is true) the
non-central F statistic becomes important.  When the null hypothesis is
true, the F statistic follows a central F distribution. When the null
hypothesis is not true, then it follows a non-central F statistic.

Weisstein, Eric W. ``Noncentral F-Distribution.'' From MathWorld--A Wolfram
Web Resource.  \href{http://mathworld.wolfram.com/NoncentralF-Distribution.html}{http://mathworld.wolfram.com/NoncentralF-Distribution.html}

Wikipedia, ``Noncentral F distribution'',
\href{http://en.wikipedia.org/wiki/Noncentral\_F-distribution}{http://en.wikipedia.org/wiki/Noncentral\_F-distribution}

In a study, testing for a specific alternative to the null hypothesis
requires use of the Noncentral F distribution. We need to calculate the
area in the tail of the distribution that exceeds the value of the F
distribution for the null hypothesis.  We'll plot the two probability
distributions for comparison.

\begin{Verbatim}[commandchars=\\\{\}]
\PYG{g+gp}{\PYGZgt{}\PYGZgt{}\PYGZgt{} }\PYG{n}{dfnum} \PYG{o}{=} \PYG{l+m+mi}{3} \PYG{c}{\PYGZsh{} between group deg of freedom}
\PYG{g+gp}{\PYGZgt{}\PYGZgt{}\PYGZgt{} }\PYG{n}{dfden} \PYG{o}{=} \PYG{l+m+mi}{20} \PYG{c}{\PYGZsh{} within groups degrees of freedom}
\PYG{g+gp}{\PYGZgt{}\PYGZgt{}\PYGZgt{} }\PYG{n}{nonc} \PYG{o}{=} \PYG{l+m+mf}{3.0}
\PYG{g+gp}{\PYGZgt{}\PYGZgt{}\PYGZgt{} }\PYG{n}{nc\PYGZus{}vals} \PYG{o}{=} \PYG{n}{np}\PYG{o}{.}\PYG{n}{random}\PYG{o}{.}\PYG{n}{noncentral\PYGZus{}f}\PYG{p}{(}\PYG{n}{dfnum}\PYG{p}{,} \PYG{n}{dfden}\PYG{p}{,} \PYG{n}{nonc}\PYG{p}{,} \PYG{l+m+mi}{1000000}\PYG{p}{)}
\PYG{g+gp}{\PYGZgt{}\PYGZgt{}\PYGZgt{} }\PYG{n}{NF} \PYG{o}{=} \PYG{n}{np}\PYG{o}{.}\PYG{n}{histogram}\PYG{p}{(}\PYG{n}{nc\PYGZus{}vals}\PYG{p}{,} \PYG{n}{bins}\PYG{o}{=}\PYG{l+m+mi}{50}\PYG{p}{,} \PYG{n}{normed}\PYG{o}{=}\PYG{n+nb+bp}{True}\PYG{p}{)}
\PYG{g+gp}{\PYGZgt{}\PYGZgt{}\PYGZgt{} }\PYG{n}{c\PYGZus{}vals} \PYG{o}{=} \PYG{n}{np}\PYG{o}{.}\PYG{n}{random}\PYG{o}{.}\PYG{n}{f}\PYG{p}{(}\PYG{n}{dfnum}\PYG{p}{,} \PYG{n}{dfden}\PYG{p}{,} \PYG{l+m+mi}{1000000}\PYG{p}{)}
\PYG{g+gp}{\PYGZgt{}\PYGZgt{}\PYGZgt{} }\PYG{n}{F} \PYG{o}{=} \PYG{n}{np}\PYG{o}{.}\PYG{n}{histogram}\PYG{p}{(}\PYG{n}{c\PYGZus{}vals}\PYG{p}{,} \PYG{n}{bins}\PYG{o}{=}\PYG{l+m+mi}{50}\PYG{p}{,} \PYG{n}{normed}\PYG{o}{=}\PYG{n+nb+bp}{True}\PYG{p}{)}
\PYG{g+gp}{\PYGZgt{}\PYGZgt{}\PYGZgt{} }\PYG{n}{plt}\PYG{o}{.}\PYG{n}{plot}\PYG{p}{(}\PYG{n}{F}\PYG{p}{[}\PYG{l+m+mi}{1}\PYG{p}{]}\PYG{p}{[}\PYG{l+m+mi}{1}\PYG{p}{:}\PYG{p}{]}\PYG{p}{,} \PYG{n}{F}\PYG{p}{[}\PYG{l+m+mi}{0}\PYG{p}{]}\PYG{p}{)}
\PYG{g+gp}{\PYGZgt{}\PYGZgt{}\PYGZgt{} }\PYG{n}{plt}\PYG{o}{.}\PYG{n}{plot}\PYG{p}{(}\PYG{n}{NF}\PYG{p}{[}\PYG{l+m+mi}{1}\PYG{p}{]}\PYG{p}{[}\PYG{l+m+mi}{1}\PYG{p}{:}\PYG{p}{]}\PYG{p}{,} \PYG{n}{NF}\PYG{p}{[}\PYG{l+m+mi}{0}\PYG{p}{]}\PYG{p}{)}
\PYG{g+gp}{\PYGZgt{}\PYGZgt{}\PYGZgt{} }\PYG{n}{plt}\PYG{o}{.}\PYG{n}{show}\PYG{p}{(}\PYG{p}{)}
\end{Verbatim}

\end{fulllineitems}

\index{normal() (in module topology\_analysis)}

\begin{fulllineitems}
\phantomsection\label{topology_analysis:topology_analysis.normal}\pysiglinewithargsret{\code{topology\_analysis.}\bfcode{normal}}{\emph{loc=0.0}, \emph{scale=1.0}, \emph{size=None}}{}
Draw random samples from a normal (Gaussian) distribution.

The probability density function of the normal distribution, first
derived by De Moivre and 200 years later by both Gauss and Laplace
independently {\color{red}\bfseries{}{[}2{]}\_}, is often called the bell curve because of
its characteristic shape (see the example below).

The normal distributions occurs often in nature.  For example, it
describes the commonly occurring distribution of samples influenced
by a large number of tiny, random disturbances, each with its own
unique distribution {\color{red}\bfseries{}{[}2{]}\_}.
\begin{description}
\item[{loc}] \leavevmode{[}float{]}
Mean (``centre'') of the distribution.

\item[{scale}] \leavevmode{[}float{]}
Standard deviation (spread or ``width'') of the distribution.

\item[{size}] \leavevmode{[}tuple of ints{]}
Output shape.  If the given shape is, e.g., \code{(m, n, k)}, then
\code{m * n * k} samples are drawn.

\end{description}
\begin{description}
\item[{scipy.stats.distributions.norm}] \leavevmode{[}probability density function,{]}
distribution or cumulative density function, etc.

\end{description}

The probability density for the Gaussian distribution is
\begin{gather}
\begin{split}p(x) = \frac{1}{\sqrt{ 2 \pi \sigma^2 }}
e^{ - \frac{ (x - \mu)^2 } {2 \sigma^2} },\end{split}\notag
\end{gather}
where \(\mu\) is the mean and \(\sigma\) the standard deviation.
The square of the standard deviation, \(\sigma^2\), is called the
variance.

The function has its peak at the mean, and its ``spread'' increases with
the standard deviation (the function reaches 0.607 times its maximum at
\(x + \sigma\) and \(x - \sigma\) {\color{red}\bfseries{}{[}2{]}\_}).  This implies that
\emph{numpy.random.normal} is more likely to return samples lying close to the
mean, rather than those far away.

Draw samples from the distribution:

\begin{Verbatim}[commandchars=\\\{\}]
\PYG{g+gp}{\PYGZgt{}\PYGZgt{}\PYGZgt{} }\PYG{n}{mu}\PYG{p}{,} \PYG{n}{sigma} \PYG{o}{=} \PYG{l+m+mi}{0}\PYG{p}{,} \PYG{l+m+mf}{0.1} \PYG{c}{\PYGZsh{} mean and standard deviation}
\PYG{g+gp}{\PYGZgt{}\PYGZgt{}\PYGZgt{} }\PYG{n}{s} \PYG{o}{=} \PYG{n}{np}\PYG{o}{.}\PYG{n}{random}\PYG{o}{.}\PYG{n}{normal}\PYG{p}{(}\PYG{n}{mu}\PYG{p}{,} \PYG{n}{sigma}\PYG{p}{,} \PYG{l+m+mi}{1000}\PYG{p}{)}
\end{Verbatim}

Verify the mean and the variance:

\begin{Verbatim}[commandchars=\\\{\}]
\PYG{g+gp}{\PYGZgt{}\PYGZgt{}\PYGZgt{} }\PYG{n+nb}{abs}\PYG{p}{(}\PYG{n}{mu} \PYG{o}{\PYGZhy{}} \PYG{n}{np}\PYG{o}{.}\PYG{n}{mean}\PYG{p}{(}\PYG{n}{s}\PYG{p}{)}\PYG{p}{)} \PYG{o}{\PYGZlt{}} \PYG{l+m+mf}{0.01}
\PYG{g+go}{True}
\end{Verbatim}

\begin{Verbatim}[commandchars=\\\{\}]
\PYG{g+gp}{\PYGZgt{}\PYGZgt{}\PYGZgt{} }\PYG{n+nb}{abs}\PYG{p}{(}\PYG{n}{sigma} \PYG{o}{\PYGZhy{}} \PYG{n}{np}\PYG{o}{.}\PYG{n}{std}\PYG{p}{(}\PYG{n}{s}\PYG{p}{,} \PYG{n}{ddof}\PYG{o}{=}\PYG{l+m+mi}{1}\PYG{p}{)}\PYG{p}{)} \PYG{o}{\PYGZlt{}} \PYG{l+m+mf}{0.01}
\PYG{g+go}{True}
\end{Verbatim}

Display the histogram of the samples, along with
the probability density function:

\begin{Verbatim}[commandchars=\\\{\}]
\PYG{g+gp}{\PYGZgt{}\PYGZgt{}\PYGZgt{} }\PYG{k+kn}{import} \PYG{n+nn}{matplotlib.pyplot} \PYG{k+kn}{as} \PYG{n+nn}{plt}
\PYG{g+gp}{\PYGZgt{}\PYGZgt{}\PYGZgt{} }\PYG{n}{count}\PYG{p}{,} \PYG{n}{bins}\PYG{p}{,} \PYG{n}{ignored} \PYG{o}{=} \PYG{n}{plt}\PYG{o}{.}\PYG{n}{hist}\PYG{p}{(}\PYG{n}{s}\PYG{p}{,} \PYG{l+m+mi}{30}\PYG{p}{,} \PYG{n}{normed}\PYG{o}{=}\PYG{n+nb+bp}{True}\PYG{p}{)}
\PYG{g+gp}{\PYGZgt{}\PYGZgt{}\PYGZgt{} }\PYG{n}{plt}\PYG{o}{.}\PYG{n}{plot}\PYG{p}{(}\PYG{n}{bins}\PYG{p}{,} \PYG{l+m+mi}{1}\PYG{o}{/}\PYG{p}{(}\PYG{n}{sigma} \PYG{o}{*} \PYG{n}{np}\PYG{o}{.}\PYG{n}{sqrt}\PYG{p}{(}\PYG{l+m+mi}{2} \PYG{o}{*} \PYG{n}{np}\PYG{o}{.}\PYG{n}{pi}\PYG{p}{)}\PYG{p}{)} \PYG{o}{*}
\PYG{g+gp}{... }               \PYG{n}{np}\PYG{o}{.}\PYG{n}{exp}\PYG{p}{(} \PYG{o}{\PYGZhy{}} \PYG{p}{(}\PYG{n}{bins} \PYG{o}{\PYGZhy{}} \PYG{n}{mu}\PYG{p}{)}\PYG{o}{*}\PYG{o}{*}\PYG{l+m+mi}{2} \PYG{o}{/} \PYG{p}{(}\PYG{l+m+mi}{2} \PYG{o}{*} \PYG{n}{sigma}\PYG{o}{*}\PYG{o}{*}\PYG{l+m+mi}{2}\PYG{p}{)} \PYG{p}{)}\PYG{p}{,}
\PYG{g+gp}{... }         \PYG{n}{linewidth}\PYG{o}{=}\PYG{l+m+mi}{2}\PYG{p}{,} \PYG{n}{color}\PYG{o}{=}\PYG{l+s}{\PYGZsq{}}\PYG{l+s}{r}\PYG{l+s}{\PYGZsq{}}\PYG{p}{)}
\PYG{g+gp}{\PYGZgt{}\PYGZgt{}\PYGZgt{} }\PYG{n}{plt}\PYG{o}{.}\PYG{n}{show}\PYG{p}{(}\PYG{p}{)}
\end{Verbatim}

\end{fulllineitems}

\index{pareto() (in module topology\_analysis)}

\begin{fulllineitems}
\phantomsection\label{topology_analysis:topology_analysis.pareto}\pysiglinewithargsret{\code{topology\_analysis.}\bfcode{pareto}}{\emph{a}, \emph{size=None}}{}
Draw samples from a Pareto II or Lomax distribution with specified shape.

The Lomax or Pareto II distribution is a shifted Pareto distribution. The
classical Pareto distribution can be obtained from the Lomax distribution
by adding the location parameter m, see below. The smallest value of the
Lomax distribution is zero while for the classical Pareto distribution it
is m, where the standard Pareto distribution has location m=1.
Lomax can also be considered as a simplified version of the Generalized
Pareto distribution (available in SciPy), with the scale set to one and
the location set to zero.

The Pareto distribution must be greater than zero, and is unbounded above.
It is also known as the ``80-20 rule''.  In this distribution, 80 percent of
the weights are in the lowest 20 percent of the range, while the other 20
percent fill the remaining 80 percent of the range.
\begin{description}
\item[{shape}] \leavevmode{[}float, \textgreater{} 0.{]}
Shape of the distribution.

\item[{size}] \leavevmode{[}tuple of ints{]}
Output shape.  If the given shape is, e.g., \code{(m, n, k)}, then
\code{m * n * k} samples are drawn.

\end{description}
\begin{description}
\item[{scipy.stats.distributions.lomax.pdf}] \leavevmode{[}probability density function,{]}
distribution or cumulative density function, etc.

\item[{scipy.stats.distributions.genpareto.pdf}] \leavevmode{[}probability density function,{]}
distribution or cumulative density function, etc.

\end{description}

The probability density for the Pareto distribution is
\begin{gather}
\begin{split}p(x) = \frac{am^a}{x^{a+1}}\end{split}\notag
\end{gather}
where \(a\) is the shape and \(m\) the location

The Pareto distribution, named after the Italian economist Vilfredo Pareto,
is a power law probability distribution useful in many real world problems.
Outside the field of economics it is generally referred to as the Bradford
distribution. Pareto developed the distribution to describe the
distribution of wealth in an economy.  It has also found use in insurance,
web page access statistics, oil field sizes, and many other problems,
including the download frequency for projects in Sourceforge {[}1{]}.  It is
one of the so-called ``fat-tailed'' distributions.

Draw samples from the distribution:

\begin{Verbatim}[commandchars=\\\{\}]
\PYG{g+gp}{\PYGZgt{}\PYGZgt{}\PYGZgt{} }\PYG{n}{a}\PYG{p}{,} \PYG{n}{m} \PYG{o}{=} \PYG{l+m+mf}{3.}\PYG{p}{,} \PYG{l+m+mf}{1.} \PYG{c}{\PYGZsh{} shape and mode}
\PYG{g+gp}{\PYGZgt{}\PYGZgt{}\PYGZgt{} }\PYG{n}{s} \PYG{o}{=} \PYG{n}{np}\PYG{o}{.}\PYG{n}{random}\PYG{o}{.}\PYG{n}{pareto}\PYG{p}{(}\PYG{n}{a}\PYG{p}{,} \PYG{l+m+mi}{1000}\PYG{p}{)} \PYG{o}{+} \PYG{n}{m}
\end{Verbatim}

Display the histogram of the samples, along with
the probability density function:

\begin{Verbatim}[commandchars=\\\{\}]
\PYG{g+gp}{\PYGZgt{}\PYGZgt{}\PYGZgt{} }\PYG{k+kn}{import} \PYG{n+nn}{matplotlib.pyplot} \PYG{k+kn}{as} \PYG{n+nn}{plt}
\PYG{g+gp}{\PYGZgt{}\PYGZgt{}\PYGZgt{} }\PYG{n}{count}\PYG{p}{,} \PYG{n}{bins}\PYG{p}{,} \PYG{n}{ignored} \PYG{o}{=} \PYG{n}{plt}\PYG{o}{.}\PYG{n}{hist}\PYG{p}{(}\PYG{n}{s}\PYG{p}{,} \PYG{l+m+mi}{100}\PYG{p}{,} \PYG{n}{normed}\PYG{o}{=}\PYG{n+nb+bp}{True}\PYG{p}{,} \PYG{n}{align}\PYG{o}{=}\PYG{l+s}{\PYGZsq{}}\PYG{l+s}{center}\PYG{l+s}{\PYGZsq{}}\PYG{p}{)}
\PYG{g+gp}{\PYGZgt{}\PYGZgt{}\PYGZgt{} }\PYG{n}{fit} \PYG{o}{=} \PYG{n}{a}\PYG{o}{*}\PYG{n}{m}\PYG{o}{*}\PYG{o}{*}\PYG{n}{a}\PYG{o}{/}\PYG{n}{bins}\PYG{o}{*}\PYG{o}{*}\PYG{p}{(}\PYG{n}{a}\PYG{o}{+}\PYG{l+m+mi}{1}\PYG{p}{)}
\PYG{g+gp}{\PYGZgt{}\PYGZgt{}\PYGZgt{} }\PYG{n}{plt}\PYG{o}{.}\PYG{n}{plot}\PYG{p}{(}\PYG{n}{bins}\PYG{p}{,} \PYG{n+nb}{max}\PYG{p}{(}\PYG{n}{count}\PYG{p}{)}\PYG{o}{*}\PYG{n}{fit}\PYG{o}{/}\PYG{n+nb}{max}\PYG{p}{(}\PYG{n}{fit}\PYG{p}{)}\PYG{p}{,}\PYG{n}{linewidth}\PYG{o}{=}\PYG{l+m+mi}{2}\PYG{p}{,} \PYG{n}{color}\PYG{o}{=}\PYG{l+s}{\PYGZsq{}}\PYG{l+s}{r}\PYG{l+s}{\PYGZsq{}}\PYG{p}{)}
\PYG{g+gp}{\PYGZgt{}\PYGZgt{}\PYGZgt{} }\PYG{n}{plt}\PYG{o}{.}\PYG{n}{show}\PYG{p}{(}\PYG{p}{)}
\end{Verbatim}

\end{fulllineitems}

\index{permutation() (in module topology\_analysis)}

\begin{fulllineitems}
\phantomsection\label{topology_analysis:topology_analysis.permutation}\pysiglinewithargsret{\code{topology\_analysis.}\bfcode{permutation}}{\emph{x}}{}
Randomly permute a sequence, or return a permuted range.

If \emph{x} is a multi-dimensional array, it is only shuffled along its
first index.
\begin{description}
\item[{x}] \leavevmode{[}int or array\_like{]}
If \emph{x} is an integer, randomly permute \code{np.arange(x)}.
If \emph{x} is an array, make a copy and shuffle the elements
randomly.

\end{description}
\begin{description}
\item[{out}] \leavevmode{[}ndarray{]}
Permuted sequence or array range.

\end{description}

\begin{Verbatim}[commandchars=\\\{\}]
\PYG{g+gp}{\PYGZgt{}\PYGZgt{}\PYGZgt{} }\PYG{n}{np}\PYG{o}{.}\PYG{n}{random}\PYG{o}{.}\PYG{n}{permutation}\PYG{p}{(}\PYG{l+m+mi}{10}\PYG{p}{)}
\PYG{g+go}{array([1, 7, 4, 3, 0, 9, 2, 5, 8, 6])}
\end{Verbatim}

\begin{Verbatim}[commandchars=\\\{\}]
\PYG{g+gp}{\PYGZgt{}\PYGZgt{}\PYGZgt{} }\PYG{n}{np}\PYG{o}{.}\PYG{n}{random}\PYG{o}{.}\PYG{n}{permutation}\PYG{p}{(}\PYG{p}{[}\PYG{l+m+mi}{1}\PYG{p}{,} \PYG{l+m+mi}{4}\PYG{p}{,} \PYG{l+m+mi}{9}\PYG{p}{,} \PYG{l+m+mi}{12}\PYG{p}{,} \PYG{l+m+mi}{15}\PYG{p}{]}\PYG{p}{)}
\PYG{g+go}{array([15,  1,  9,  4, 12])}
\end{Verbatim}

\begin{Verbatim}[commandchars=\\\{\}]
\PYG{g+gp}{\PYGZgt{}\PYGZgt{}\PYGZgt{} }\PYG{n}{arr} \PYG{o}{=} \PYG{n}{np}\PYG{o}{.}\PYG{n}{arange}\PYG{p}{(}\PYG{l+m+mi}{9}\PYG{p}{)}\PYG{o}{.}\PYG{n}{reshape}\PYG{p}{(}\PYG{p}{(}\PYG{l+m+mi}{3}\PYG{p}{,} \PYG{l+m+mi}{3}\PYG{p}{)}\PYG{p}{)}
\PYG{g+gp}{\PYGZgt{}\PYGZgt{}\PYGZgt{} }\PYG{n}{np}\PYG{o}{.}\PYG{n}{random}\PYG{o}{.}\PYG{n}{permutation}\PYG{p}{(}\PYG{n}{arr}\PYG{p}{)}
\PYG{g+go}{array([[6, 7, 8],}
\PYG{g+go}{       [0, 1, 2],}
\PYG{g+go}{       [3, 4, 5]])}
\end{Verbatim}

\end{fulllineitems}

\index{poisson() (in module topology\_analysis)}

\begin{fulllineitems}
\phantomsection\label{topology_analysis:topology_analysis.poisson}\pysiglinewithargsret{\code{topology\_analysis.}\bfcode{poisson}}{\emph{lam=1.0}, \emph{size=None}}{}
Draw samples from a Poisson distribution.

The Poisson distribution is the limit of the Binomial
distribution for large N.
\begin{description}
\item[{lam}] \leavevmode{[}float{]}
Expectation of interval, should be \textgreater{}= 0.

\item[{size}] \leavevmode{[}int or tuple of ints, optional{]}
Output shape. If the given shape is, e.g., \code{(m, n, k)}, then
\code{m * n * k} samples are drawn.

\end{description}

The Poisson distribution
\begin{gather}
\begin{split}f(k; \lambda)=\frac{\lambda^k e^{-\lambda}}{k!}\end{split}\notag
\end{gather}
For events with an expected separation \(\lambda\) the Poisson
distribution \(f(k; \lambda)\) describes the probability of
\(k\) events occurring within the observed interval \(\lambda\).

Because the output is limited to the range of the C long type, a
ValueError is raised when \emph{lam} is within 10 sigma of the maximum
representable value.

Draw samples from the distribution:

\begin{Verbatim}[commandchars=\\\{\}]
\PYG{g+gp}{\PYGZgt{}\PYGZgt{}\PYGZgt{} }\PYG{k+kn}{import} \PYG{n+nn}{numpy} \PYG{k+kn}{as} \PYG{n+nn}{np}
\PYG{g+gp}{\PYGZgt{}\PYGZgt{}\PYGZgt{} }\PYG{n}{s} \PYG{o}{=} \PYG{n}{np}\PYG{o}{.}\PYG{n}{random}\PYG{o}{.}\PYG{n}{poisson}\PYG{p}{(}\PYG{l+m+mi}{5}\PYG{p}{,} \PYG{l+m+mi}{10000}\PYG{p}{)}
\end{Verbatim}

Display histogram of the sample:

\begin{Verbatim}[commandchars=\\\{\}]
\PYG{g+gp}{\PYGZgt{}\PYGZgt{}\PYGZgt{} }\PYG{k+kn}{import} \PYG{n+nn}{matplotlib.pyplot} \PYG{k+kn}{as} \PYG{n+nn}{plt}
\PYG{g+gp}{\PYGZgt{}\PYGZgt{}\PYGZgt{} }\PYG{n}{count}\PYG{p}{,} \PYG{n}{bins}\PYG{p}{,} \PYG{n}{ignored} \PYG{o}{=} \PYG{n}{plt}\PYG{o}{.}\PYG{n}{hist}\PYG{p}{(}\PYG{n}{s}\PYG{p}{,} \PYG{l+m+mi}{14}\PYG{p}{,} \PYG{n}{normed}\PYG{o}{=}\PYG{n+nb+bp}{True}\PYG{p}{)}
\PYG{g+gp}{\PYGZgt{}\PYGZgt{}\PYGZgt{} }\PYG{n}{plt}\PYG{o}{.}\PYG{n}{show}\PYG{p}{(}\PYG{p}{)}
\end{Verbatim}

\end{fulllineitems}

\index{power() (in module topology\_analysis)}

\begin{fulllineitems}
\phantomsection\label{topology_analysis:topology_analysis.power}\pysiglinewithargsret{\code{topology\_analysis.}\bfcode{power}}{\emph{a}, \emph{size=None}}{}
Draws samples in {[}0, 1{]} from a power distribution with positive
exponent a - 1.

Also known as the power function distribution.
\begin{description}
\item[{a}] \leavevmode{[}float{]}
parameter, \textgreater{} 0

\item[{size}] \leavevmode{[}tuple of ints{]}\begin{description}
\item[{Output shape.  If the given shape is, e.g., \code{(m, n, k)}, then}] \leavevmode
\code{m * n * k} samples are drawn.

\end{description}

\end{description}
\begin{description}
\item[{samples}] \leavevmode{[}\{ndarray, scalar\}{]}
The returned samples lie in {[}0, 1{]}.

\end{description}
\begin{description}
\item[{ValueError}] \leavevmode
If a\textless{}1.

\end{description}

The probability density function is
\begin{gather}
\begin{split}P(x; a) = ax^{a-1}, 0 \le x \le 1, a>0.\end{split}\notag
\end{gather}
The power function distribution is just the inverse of the Pareto
distribution. It may also be seen as a special case of the Beta
distribution.

It is used, for example, in modeling the over-reporting of insurance
claims.

Draw samples from the distribution:

\begin{Verbatim}[commandchars=\\\{\}]
\PYG{g+gp}{\PYGZgt{}\PYGZgt{}\PYGZgt{} }\PYG{n}{a} \PYG{o}{=} \PYG{l+m+mf}{5.} \PYG{c}{\PYGZsh{} shape}
\PYG{g+gp}{\PYGZgt{}\PYGZgt{}\PYGZgt{} }\PYG{n}{samples} \PYG{o}{=} \PYG{l+m+mi}{1000}
\PYG{g+gp}{\PYGZgt{}\PYGZgt{}\PYGZgt{} }\PYG{n}{s} \PYG{o}{=} \PYG{n}{np}\PYG{o}{.}\PYG{n}{random}\PYG{o}{.}\PYG{n}{power}\PYG{p}{(}\PYG{n}{a}\PYG{p}{,} \PYG{n}{samples}\PYG{p}{)}
\end{Verbatim}

Display the histogram of the samples, along with
the probability density function:

\begin{Verbatim}[commandchars=\\\{\}]
\PYG{g+gp}{\PYGZgt{}\PYGZgt{}\PYGZgt{} }\PYG{k+kn}{import} \PYG{n+nn}{matplotlib.pyplot} \PYG{k+kn}{as} \PYG{n+nn}{plt}
\PYG{g+gp}{\PYGZgt{}\PYGZgt{}\PYGZgt{} }\PYG{n}{count}\PYG{p}{,} \PYG{n}{bins}\PYG{p}{,} \PYG{n}{ignored} \PYG{o}{=} \PYG{n}{plt}\PYG{o}{.}\PYG{n}{hist}\PYG{p}{(}\PYG{n}{s}\PYG{p}{,} \PYG{n}{bins}\PYG{o}{=}\PYG{l+m+mi}{30}\PYG{p}{)}
\PYG{g+gp}{\PYGZgt{}\PYGZgt{}\PYGZgt{} }\PYG{n}{x} \PYG{o}{=} \PYG{n}{np}\PYG{o}{.}\PYG{n}{linspace}\PYG{p}{(}\PYG{l+m+mi}{0}\PYG{p}{,} \PYG{l+m+mi}{1}\PYG{p}{,} \PYG{l+m+mi}{100}\PYG{p}{)}
\PYG{g+gp}{\PYGZgt{}\PYGZgt{}\PYGZgt{} }\PYG{n}{y} \PYG{o}{=} \PYG{n}{a}\PYG{o}{*}\PYG{n}{x}\PYG{o}{*}\PYG{o}{*}\PYG{p}{(}\PYG{n}{a}\PYG{o}{\PYGZhy{}}\PYG{l+m+mf}{1.}\PYG{p}{)}
\PYG{g+gp}{\PYGZgt{}\PYGZgt{}\PYGZgt{} }\PYG{n}{normed\PYGZus{}y} \PYG{o}{=} \PYG{n}{samples}\PYG{o}{*}\PYG{n}{np}\PYG{o}{.}\PYG{n}{diff}\PYG{p}{(}\PYG{n}{bins}\PYG{p}{)}\PYG{p}{[}\PYG{l+m+mi}{0}\PYG{p}{]}\PYG{o}{*}\PYG{n}{y}
\PYG{g+gp}{\PYGZgt{}\PYGZgt{}\PYGZgt{} }\PYG{n}{plt}\PYG{o}{.}\PYG{n}{plot}\PYG{p}{(}\PYG{n}{x}\PYG{p}{,} \PYG{n}{normed\PYGZus{}y}\PYG{p}{)}
\PYG{g+gp}{\PYGZgt{}\PYGZgt{}\PYGZgt{} }\PYG{n}{plt}\PYG{o}{.}\PYG{n}{show}\PYG{p}{(}\PYG{p}{)}
\end{Verbatim}

Compare the power function distribution to the inverse of the Pareto.

\begin{Verbatim}[commandchars=\\\{\}]
\PYG{g+gp}{\PYGZgt{}\PYGZgt{}\PYGZgt{} }\PYG{k+kn}{from} \PYG{n+nn}{scipy} \PYG{k+kn}{import} \PYG{n}{stats}
\PYG{g+gp}{\PYGZgt{}\PYGZgt{}\PYGZgt{} }\PYG{n}{rvs} \PYG{o}{=} \PYG{n}{np}\PYG{o}{.}\PYG{n}{random}\PYG{o}{.}\PYG{n}{power}\PYG{p}{(}\PYG{l+m+mi}{5}\PYG{p}{,} \PYG{l+m+mi}{1000000}\PYG{p}{)}
\PYG{g+gp}{\PYGZgt{}\PYGZgt{}\PYGZgt{} }\PYG{n}{rvsp} \PYG{o}{=} \PYG{n}{np}\PYG{o}{.}\PYG{n}{random}\PYG{o}{.}\PYG{n}{pareto}\PYG{p}{(}\PYG{l+m+mi}{5}\PYG{p}{,} \PYG{l+m+mi}{1000000}\PYG{p}{)}
\PYG{g+gp}{\PYGZgt{}\PYGZgt{}\PYGZgt{} }\PYG{n}{xx} \PYG{o}{=} \PYG{n}{np}\PYG{o}{.}\PYG{n}{linspace}\PYG{p}{(}\PYG{l+m+mi}{0}\PYG{p}{,}\PYG{l+m+mi}{1}\PYG{p}{,}\PYG{l+m+mi}{100}\PYG{p}{)}
\PYG{g+gp}{\PYGZgt{}\PYGZgt{}\PYGZgt{} }\PYG{n}{powpdf} \PYG{o}{=} \PYG{n}{stats}\PYG{o}{.}\PYG{n}{powerlaw}\PYG{o}{.}\PYG{n}{pdf}\PYG{p}{(}\PYG{n}{xx}\PYG{p}{,}\PYG{l+m+mi}{5}\PYG{p}{)}
\end{Verbatim}

\begin{Verbatim}[commandchars=\\\{\}]
\PYG{g+gp}{\PYGZgt{}\PYGZgt{}\PYGZgt{} }\PYG{n}{plt}\PYG{o}{.}\PYG{n}{figure}\PYG{p}{(}\PYG{p}{)}
\PYG{g+gp}{\PYGZgt{}\PYGZgt{}\PYGZgt{} }\PYG{n}{plt}\PYG{o}{.}\PYG{n}{hist}\PYG{p}{(}\PYG{n}{rvs}\PYG{p}{,} \PYG{n}{bins}\PYG{o}{=}\PYG{l+m+mi}{50}\PYG{p}{,} \PYG{n}{normed}\PYG{o}{=}\PYG{n+nb+bp}{True}\PYG{p}{)}
\PYG{g+gp}{\PYGZgt{}\PYGZgt{}\PYGZgt{} }\PYG{n}{plt}\PYG{o}{.}\PYG{n}{plot}\PYG{p}{(}\PYG{n}{xx}\PYG{p}{,}\PYG{n}{powpdf}\PYG{p}{,}\PYG{l+s}{\PYGZsq{}}\PYG{l+s}{r\PYGZhy{}}\PYG{l+s}{\PYGZsq{}}\PYG{p}{)}
\PYG{g+gp}{\PYGZgt{}\PYGZgt{}\PYGZgt{} }\PYG{n}{plt}\PYG{o}{.}\PYG{n}{title}\PYG{p}{(}\PYG{l+s}{\PYGZsq{}}\PYG{l+s}{np.random.power(5)}\PYG{l+s}{\PYGZsq{}}\PYG{p}{)}
\end{Verbatim}

\begin{Verbatim}[commandchars=\\\{\}]
\PYG{g+gp}{\PYGZgt{}\PYGZgt{}\PYGZgt{} }\PYG{n}{plt}\PYG{o}{.}\PYG{n}{figure}\PYG{p}{(}\PYG{p}{)}
\PYG{g+gp}{\PYGZgt{}\PYGZgt{}\PYGZgt{} }\PYG{n}{plt}\PYG{o}{.}\PYG{n}{hist}\PYG{p}{(}\PYG{l+m+mf}{1.}\PYG{o}{/}\PYG{p}{(}\PYG{l+m+mf}{1.}\PYG{o}{+}\PYG{n}{rvsp}\PYG{p}{)}\PYG{p}{,} \PYG{n}{bins}\PYG{o}{=}\PYG{l+m+mi}{50}\PYG{p}{,} \PYG{n}{normed}\PYG{o}{=}\PYG{n+nb+bp}{True}\PYG{p}{)}
\PYG{g+gp}{\PYGZgt{}\PYGZgt{}\PYGZgt{} }\PYG{n}{plt}\PYG{o}{.}\PYG{n}{plot}\PYG{p}{(}\PYG{n}{xx}\PYG{p}{,}\PYG{n}{powpdf}\PYG{p}{,}\PYG{l+s}{\PYGZsq{}}\PYG{l+s}{r\PYGZhy{}}\PYG{l+s}{\PYGZsq{}}\PYG{p}{)}
\PYG{g+gp}{\PYGZgt{}\PYGZgt{}\PYGZgt{} }\PYG{n}{plt}\PYG{o}{.}\PYG{n}{title}\PYG{p}{(}\PYG{l+s}{\PYGZsq{}}\PYG{l+s}{inverse of 1 + np.random.pareto(5)}\PYG{l+s}{\PYGZsq{}}\PYG{p}{)}
\end{Verbatim}

\begin{Verbatim}[commandchars=\\\{\}]
\PYG{g+gp}{\PYGZgt{}\PYGZgt{}\PYGZgt{} }\PYG{n}{plt}\PYG{o}{.}\PYG{n}{figure}\PYG{p}{(}\PYG{p}{)}
\PYG{g+gp}{\PYGZgt{}\PYGZgt{}\PYGZgt{} }\PYG{n}{plt}\PYG{o}{.}\PYG{n}{hist}\PYG{p}{(}\PYG{l+m+mf}{1.}\PYG{o}{/}\PYG{p}{(}\PYG{l+m+mf}{1.}\PYG{o}{+}\PYG{n}{rvsp}\PYG{p}{)}\PYG{p}{,} \PYG{n}{bins}\PYG{o}{=}\PYG{l+m+mi}{50}\PYG{p}{,} \PYG{n}{normed}\PYG{o}{=}\PYG{n+nb+bp}{True}\PYG{p}{)}
\PYG{g+gp}{\PYGZgt{}\PYGZgt{}\PYGZgt{} }\PYG{n}{plt}\PYG{o}{.}\PYG{n}{plot}\PYG{p}{(}\PYG{n}{xx}\PYG{p}{,}\PYG{n}{powpdf}\PYG{p}{,}\PYG{l+s}{\PYGZsq{}}\PYG{l+s}{r\PYGZhy{}}\PYG{l+s}{\PYGZsq{}}\PYG{p}{)}
\PYG{g+gp}{\PYGZgt{}\PYGZgt{}\PYGZgt{} }\PYG{n}{plt}\PYG{o}{.}\PYG{n}{title}\PYG{p}{(}\PYG{l+s}{\PYGZsq{}}\PYG{l+s}{inverse of stats.pareto(5)}\PYG{l+s}{\PYGZsq{}}\PYG{p}{)}
\end{Verbatim}

\end{fulllineitems}

\index{rand() (in module topology\_analysis)}

\begin{fulllineitems}
\phantomsection\label{topology_analysis:topology_analysis.rand}\pysiglinewithargsret{\code{topology\_analysis.}\bfcode{rand}}{\emph{d0}, \emph{d1}, \emph{...}, \emph{dn}}{}
Random values in a given shape.

Create an array of the given shape and propagate it with
random samples from a uniform distribution
over \code{{[}0, 1)}.
\begin{description}
\item[{d0, d1, ..., dn}] \leavevmode{[}int, optional{]}
The dimensions of the returned array, should all be positive.
If no argument is given a single Python float is returned.

\end{description}
\begin{description}
\item[{out}] \leavevmode{[}ndarray, shape \code{(d0, d1, ..., dn)}{]}
Random values.

\end{description}

random

This is a convenience function. If you want an interface that
takes a shape-tuple as the first argument, refer to
np.random.random\_sample .

\begin{Verbatim}[commandchars=\\\{\}]
\PYG{g+gp}{\PYGZgt{}\PYGZgt{}\PYGZgt{} }\PYG{n}{np}\PYG{o}{.}\PYG{n}{random}\PYG{o}{.}\PYG{n}{rand}\PYG{p}{(}\PYG{l+m+mi}{3}\PYG{p}{,}\PYG{l+m+mi}{2}\PYG{p}{)}
\PYG{g+go}{array([[ 0.14022471,  0.96360618],  \PYGZsh{}random}
\PYG{g+go}{       [ 0.37601032,  0.25528411],  \PYGZsh{}random}
\PYG{g+go}{       [ 0.49313049,  0.94909878]]) \PYGZsh{}random}
\end{Verbatim}

\end{fulllineitems}

\index{randint() (in module topology\_analysis)}

\begin{fulllineitems}
\phantomsection\label{topology_analysis:topology_analysis.randint}\pysiglinewithargsret{\code{topology\_analysis.}\bfcode{randint}}{\emph{low}, \emph{high=None}, \emph{size=None}}{}
Return random integers from \emph{low} (inclusive) to \emph{high} (exclusive).

Return random integers from the ``discrete uniform'' distribution in the
``half-open'' interval {[}\emph{low}, \emph{high}). If \emph{high} is None (the default),
then results are from {[}0, \emph{low}).
\begin{description}
\item[{low}] \leavevmode{[}int{]}
Lowest (signed) integer to be drawn from the distribution (unless
\code{high=None}, in which case this parameter is the \emph{highest} such
integer).

\item[{high}] \leavevmode{[}int, optional{]}
If provided, one above the largest (signed) integer to be drawn
from the distribution (see above for behavior if \code{high=None}).

\item[{size}] \leavevmode{[}int or tuple of ints, optional{]}
Output shape. Default is None, in which case a single int is
returned.

\end{description}
\begin{description}
\item[{out}] \leavevmode{[}int or ndarray of ints{]}
\emph{size}-shaped array of random integers from the appropriate
distribution, or a single such random int if \emph{size} not provided.

\end{description}
\begin{description}
\item[{random.random\_integers}] \leavevmode{[}similar to \emph{randint}, only for the closed{]}
interval {[}\emph{low}, \emph{high}{]}, and 1 is the lowest value if \emph{high} is
omitted. In particular, this other one is the one to use to generate
uniformly distributed discrete non-integers.

\end{description}

\begin{Verbatim}[commandchars=\\\{\}]
\PYG{g+gp}{\PYGZgt{}\PYGZgt{}\PYGZgt{} }\PYG{n}{np}\PYG{o}{.}\PYG{n}{random}\PYG{o}{.}\PYG{n}{randint}\PYG{p}{(}\PYG{l+m+mi}{2}\PYG{p}{,} \PYG{n}{size}\PYG{o}{=}\PYG{l+m+mi}{10}\PYG{p}{)}
\PYG{g+go}{array([1, 0, 0, 0, 1, 1, 0, 0, 1, 0])}
\PYG{g+gp}{\PYGZgt{}\PYGZgt{}\PYGZgt{} }\PYG{n}{np}\PYG{o}{.}\PYG{n}{random}\PYG{o}{.}\PYG{n}{randint}\PYG{p}{(}\PYG{l+m+mi}{1}\PYG{p}{,} \PYG{n}{size}\PYG{o}{=}\PYG{l+m+mi}{10}\PYG{p}{)}
\PYG{g+go}{array([0, 0, 0, 0, 0, 0, 0, 0, 0, 0])}
\end{Verbatim}

Generate a 2 x 4 array of ints between 0 and 4, inclusive:

\begin{Verbatim}[commandchars=\\\{\}]
\PYG{g+gp}{\PYGZgt{}\PYGZgt{}\PYGZgt{} }\PYG{n}{np}\PYG{o}{.}\PYG{n}{random}\PYG{o}{.}\PYG{n}{randint}\PYG{p}{(}\PYG{l+m+mi}{5}\PYG{p}{,} \PYG{n}{size}\PYG{o}{=}\PYG{p}{(}\PYG{l+m+mi}{2}\PYG{p}{,} \PYG{l+m+mi}{4}\PYG{p}{)}\PYG{p}{)}
\PYG{g+go}{array([[4, 0, 2, 1],}
\PYG{g+go}{       [3, 2, 2, 0]])}
\end{Verbatim}

\end{fulllineitems}

\index{randn() (in module topology\_analysis)}

\begin{fulllineitems}
\phantomsection\label{topology_analysis:topology_analysis.randn}\pysiglinewithargsret{\code{topology\_analysis.}\bfcode{randn}}{\emph{d0}, \emph{d1}, \emph{...}, \emph{dn}}{}
Return a sample (or samples) from the ``standard normal'' distribution.

If positive, int\_like or int-convertible arguments are provided,
\emph{randn} generates an array of shape \code{(d0, d1, ..., dn)}, filled
with random floats sampled from a univariate ``normal'' (Gaussian)
distribution of mean 0 and variance 1 (if any of the \(d_i\) are
floats, they are first converted to integers by truncation). A single
float randomly sampled from the distribution is returned if no
argument is provided.

This is a convenience function.  If you want an interface that takes a
tuple as the first argument, use \emph{numpy.random.standard\_normal} instead.
\begin{description}
\item[{d0, d1, ..., dn}] \leavevmode{[}int, optional{]}
The dimensions of the returned array, should be all positive.
If no argument is given a single Python float is returned.

\end{description}
\begin{description}
\item[{Z}] \leavevmode{[}ndarray or float{]}
A \code{(d0, d1, ..., dn)}-shaped array of floating-point samples from
the standard normal distribution, or a single such float if
no parameters were supplied.

\end{description}

random.standard\_normal : Similar, but takes a tuple as its argument.

For random samples from \(N(\mu, \sigma^2)\), use:

\code{sigma * np.random.randn(...) + mu}

\begin{Verbatim}[commandchars=\\\{\}]
\PYG{g+gp}{\PYGZgt{}\PYGZgt{}\PYGZgt{} }\PYG{n}{np}\PYG{o}{.}\PYG{n}{random}\PYG{o}{.}\PYG{n}{randn}\PYG{p}{(}\PYG{p}{)}
\PYG{g+go}{2.1923875335537315 \PYGZsh{}random}
\end{Verbatim}

Two-by-four array of samples from N(3, 6.25):

\begin{Verbatim}[commandchars=\\\{\}]
\PYG{g+gp}{\PYGZgt{}\PYGZgt{}\PYGZgt{} }\PYG{l+m+mf}{2.5} \PYG{o}{*} \PYG{n}{np}\PYG{o}{.}\PYG{n}{random}\PYG{o}{.}\PYG{n}{randn}\PYG{p}{(}\PYG{l+m+mi}{2}\PYG{p}{,} \PYG{l+m+mi}{4}\PYG{p}{)} \PYG{o}{+} \PYG{l+m+mi}{3}
\PYG{g+go}{array([[\PYGZhy{}4.49401501,  4.00950034, \PYGZhy{}1.81814867,  7.29718677],  \PYGZsh{}random}
\PYG{g+go}{       [ 0.39924804,  4.68456316,  4.99394529,  4.84057254]]) \PYGZsh{}random}
\end{Verbatim}

\end{fulllineitems}

\index{random() (in module topology\_analysis)}

\begin{fulllineitems}
\phantomsection\label{topology_analysis:topology_analysis.random}\pysiglinewithargsret{\code{topology\_analysis.}\bfcode{random}}{}{}
random\_sample(size=None)

Return random floats in the half-open interval {[}0.0, 1.0).

Results are from the ``continuous uniform'' distribution over the
stated interval.  To sample \(Unif[a, b), b > a\) multiply
the output of \emph{random\_sample} by \emph{(b-a)} and add \emph{a}:

\begin{Verbatim}[commandchars=\\\{\}]
\PYG{p}{(}\PYG{n}{b} \PYG{o}{\PYGZhy{}} \PYG{n}{a}\PYG{p}{)} \PYG{o}{*} \PYG{n}{random\PYGZus{}sample}\PYG{p}{(}\PYG{p}{)} \PYG{o}{+} \PYG{n}{a}
\end{Verbatim}
\begin{description}
\item[{size}] \leavevmode{[}int or tuple of ints, optional{]}
Defines the shape of the returned array of random floats. If None
(the default), returns a single float.

\end{description}
\begin{description}
\item[{out}] \leavevmode{[}float or ndarray of floats{]}
Array of random floats of shape \emph{size} (unless \code{size=None}, in which
case a single float is returned).

\end{description}

\begin{Verbatim}[commandchars=\\\{\}]
\PYG{g+gp}{\PYGZgt{}\PYGZgt{}\PYGZgt{} }\PYG{n}{np}\PYG{o}{.}\PYG{n}{random}\PYG{o}{.}\PYG{n}{random\PYGZus{}sample}\PYG{p}{(}\PYG{p}{)}
\PYG{g+go}{0.47108547995356098}
\PYG{g+gp}{\PYGZgt{}\PYGZgt{}\PYGZgt{} }\PYG{n+nb}{type}\PYG{p}{(}\PYG{n}{np}\PYG{o}{.}\PYG{n}{random}\PYG{o}{.}\PYG{n}{random\PYGZus{}sample}\PYG{p}{(}\PYG{p}{)}\PYG{p}{)}
\PYG{g+go}{\PYGZlt{}type \PYGZsq{}float\PYGZsq{}\PYGZgt{}}
\PYG{g+gp}{\PYGZgt{}\PYGZgt{}\PYGZgt{} }\PYG{n}{np}\PYG{o}{.}\PYG{n}{random}\PYG{o}{.}\PYG{n}{random\PYGZus{}sample}\PYG{p}{(}\PYG{p}{(}\PYG{l+m+mi}{5}\PYG{p}{,}\PYG{p}{)}\PYG{p}{)}
\PYG{g+go}{array([ 0.30220482,  0.86820401,  0.1654503 ,  0.11659149,  0.54323428])}
\end{Verbatim}

Three-by-two array of random numbers from {[}-5, 0):

\begin{Verbatim}[commandchars=\\\{\}]
\PYG{g+gp}{\PYGZgt{}\PYGZgt{}\PYGZgt{} }\PYG{l+m+mi}{5} \PYG{o}{*} \PYG{n}{np}\PYG{o}{.}\PYG{n}{random}\PYG{o}{.}\PYG{n}{random\PYGZus{}sample}\PYG{p}{(}\PYG{p}{(}\PYG{l+m+mi}{3}\PYG{p}{,} \PYG{l+m+mi}{2}\PYG{p}{)}\PYG{p}{)} \PYG{o}{\PYGZhy{}} \PYG{l+m+mi}{5}
\PYG{g+go}{array([[\PYGZhy{}3.99149989, \PYGZhy{}0.52338984],}
\PYG{g+go}{       [\PYGZhy{}2.99091858, \PYGZhy{}0.79479508],}
\PYG{g+go}{       [\PYGZhy{}1.23204345, \PYGZhy{}1.75224494]])}
\end{Verbatim}

\end{fulllineitems}

\index{random\_integers() (in module topology\_analysis)}

\begin{fulllineitems}
\phantomsection\label{topology_analysis:topology_analysis.random_integers}\pysiglinewithargsret{\code{topology\_analysis.}\bfcode{random\_integers}}{\emph{low}, \emph{high=None}, \emph{size=None}}{}
Return random integers between \emph{low} and \emph{high}, inclusive.

Return random integers from the ``discrete uniform'' distribution in the
closed interval {[}\emph{low}, \emph{high}{]}.  If \emph{high} is None (the default),
then results are from {[}1, \emph{low}{]}.
\begin{description}
\item[{low}] \leavevmode{[}int{]}
Lowest (signed) integer to be drawn from the distribution (unless
\code{high=None}, in which case this parameter is the \emph{highest} such
integer).

\item[{high}] \leavevmode{[}int, optional{]}
If provided, the largest (signed) integer to be drawn from the
distribution (see above for behavior if \code{high=None}).

\item[{size}] \leavevmode{[}int or tuple of ints, optional{]}
Output shape. Default is None, in which case a single int is returned.

\end{description}
\begin{description}
\item[{out}] \leavevmode{[}int or ndarray of ints{]}
\emph{size}-shaped array of random integers from the appropriate
distribution, or a single such random int if \emph{size} not provided.

\end{description}
\begin{description}
\item[{random.randint}] \leavevmode{[}Similar to \emph{random\_integers}, only for the half-open{]}
interval {[}\emph{low}, \emph{high}), and 0 is the lowest value if \emph{high} is
omitted.

\end{description}

To sample from N evenly spaced floating-point numbers between a and b,
use:

\begin{Verbatim}[commandchars=\\\{\}]
\PYG{n}{a} \PYG{o}{+} \PYG{p}{(}\PYG{n}{b} \PYG{o}{\PYGZhy{}} \PYG{n}{a}\PYG{p}{)} \PYG{o}{*} \PYG{p}{(}\PYG{n}{np}\PYG{o}{.}\PYG{n}{random}\PYG{o}{.}\PYG{n}{random\PYGZus{}integers}\PYG{p}{(}\PYG{n}{N}\PYG{p}{)} \PYG{o}{\PYGZhy{}} \PYG{l+m+mi}{1}\PYG{p}{)} \PYG{o}{/} \PYG{p}{(}\PYG{n}{N} \PYG{o}{\PYGZhy{}} \PYG{l+m+mf}{1.}\PYG{p}{)}
\end{Verbatim}

\begin{Verbatim}[commandchars=\\\{\}]
\PYG{g+gp}{\PYGZgt{}\PYGZgt{}\PYGZgt{} }\PYG{n}{np}\PYG{o}{.}\PYG{n}{random}\PYG{o}{.}\PYG{n}{random\PYGZus{}integers}\PYG{p}{(}\PYG{l+m+mi}{5}\PYG{p}{)}
\PYG{g+go}{4}
\PYG{g+gp}{\PYGZgt{}\PYGZgt{}\PYGZgt{} }\PYG{n+nb}{type}\PYG{p}{(}\PYG{n}{np}\PYG{o}{.}\PYG{n}{random}\PYG{o}{.}\PYG{n}{random\PYGZus{}integers}\PYG{p}{(}\PYG{l+m+mi}{5}\PYG{p}{)}\PYG{p}{)}
\PYG{g+go}{\PYGZlt{}type \PYGZsq{}int\PYGZsq{}\PYGZgt{}}
\PYG{g+gp}{\PYGZgt{}\PYGZgt{}\PYGZgt{} }\PYG{n}{np}\PYG{o}{.}\PYG{n}{random}\PYG{o}{.}\PYG{n}{random\PYGZus{}integers}\PYG{p}{(}\PYG{l+m+mi}{5}\PYG{p}{,} \PYG{n}{size}\PYG{o}{=}\PYG{p}{(}\PYG{l+m+mf}{3.}\PYG{p}{,}\PYG{l+m+mf}{2.}\PYG{p}{)}\PYG{p}{)}
\PYG{g+go}{array([[5, 4],}
\PYG{g+go}{       [3, 3],}
\PYG{g+go}{       [4, 5]])}
\end{Verbatim}

Choose five random numbers from the set of five evenly-spaced
numbers between 0 and 2.5, inclusive (\emph{i.e.}, from the set
\({0, 5/8, 10/8, 15/8, 20/8}\)):

\begin{Verbatim}[commandchars=\\\{\}]
\PYG{g+gp}{\PYGZgt{}\PYGZgt{}\PYGZgt{} }\PYG{l+m+mf}{2.5} \PYG{o}{*} \PYG{p}{(}\PYG{n}{np}\PYG{o}{.}\PYG{n}{random}\PYG{o}{.}\PYG{n}{random\PYGZus{}integers}\PYG{p}{(}\PYG{l+m+mi}{5}\PYG{p}{,} \PYG{n}{size}\PYG{o}{=}\PYG{p}{(}\PYG{l+m+mi}{5}\PYG{p}{,}\PYG{p}{)}\PYG{p}{)} \PYG{o}{\PYGZhy{}} \PYG{l+m+mi}{1}\PYG{p}{)} \PYG{o}{/} \PYG{l+m+mf}{4.}
\PYG{g+go}{array([ 0.625,  1.25 ,  0.625,  0.625,  2.5  ])}
\end{Verbatim}

Roll two six sided dice 1000 times and sum the results:

\begin{Verbatim}[commandchars=\\\{\}]
\PYG{g+gp}{\PYGZgt{}\PYGZgt{}\PYGZgt{} }\PYG{n}{d1} \PYG{o}{=} \PYG{n}{np}\PYG{o}{.}\PYG{n}{random}\PYG{o}{.}\PYG{n}{random\PYGZus{}integers}\PYG{p}{(}\PYG{l+m+mi}{1}\PYG{p}{,} \PYG{l+m+mi}{6}\PYG{p}{,} \PYG{l+m+mi}{1000}\PYG{p}{)}
\PYG{g+gp}{\PYGZgt{}\PYGZgt{}\PYGZgt{} }\PYG{n}{d2} \PYG{o}{=} \PYG{n}{np}\PYG{o}{.}\PYG{n}{random}\PYG{o}{.}\PYG{n}{random\PYGZus{}integers}\PYG{p}{(}\PYG{l+m+mi}{1}\PYG{p}{,} \PYG{l+m+mi}{6}\PYG{p}{,} \PYG{l+m+mi}{1000}\PYG{p}{)}
\PYG{g+gp}{\PYGZgt{}\PYGZgt{}\PYGZgt{} }\PYG{n}{dsums} \PYG{o}{=} \PYG{n}{d1} \PYG{o}{+} \PYG{n}{d2}
\end{Verbatim}

Display results as a histogram:

\begin{Verbatim}[commandchars=\\\{\}]
\PYG{g+gp}{\PYGZgt{}\PYGZgt{}\PYGZgt{} }\PYG{k+kn}{import} \PYG{n+nn}{matplotlib.pyplot} \PYG{k+kn}{as} \PYG{n+nn}{plt}
\PYG{g+gp}{\PYGZgt{}\PYGZgt{}\PYGZgt{} }\PYG{n}{count}\PYG{p}{,} \PYG{n}{bins}\PYG{p}{,} \PYG{n}{ignored} \PYG{o}{=} \PYG{n}{plt}\PYG{o}{.}\PYG{n}{hist}\PYG{p}{(}\PYG{n}{dsums}\PYG{p}{,} \PYG{l+m+mi}{11}\PYG{p}{,} \PYG{n}{normed}\PYG{o}{=}\PYG{n+nb+bp}{True}\PYG{p}{)}
\PYG{g+gp}{\PYGZgt{}\PYGZgt{}\PYGZgt{} }\PYG{n}{plt}\PYG{o}{.}\PYG{n}{show}\PYG{p}{(}\PYG{p}{)}
\end{Verbatim}

\end{fulllineitems}

\index{random\_sample() (in module topology\_analysis)}

\begin{fulllineitems}
\phantomsection\label{topology_analysis:topology_analysis.random_sample}\pysiglinewithargsret{\code{topology\_analysis.}\bfcode{random\_sample}}{\emph{size=None}}{}
Return random floats in the half-open interval {[}0.0, 1.0).

Results are from the ``continuous uniform'' distribution over the
stated interval.  To sample \(Unif[a, b), b > a\) multiply
the output of \emph{random\_sample} by \emph{(b-a)} and add \emph{a}:

\begin{Verbatim}[commandchars=\\\{\}]
\PYG{p}{(}\PYG{n}{b} \PYG{o}{\PYGZhy{}} \PYG{n}{a}\PYG{p}{)} \PYG{o}{*} \PYG{n}{random\PYGZus{}sample}\PYG{p}{(}\PYG{p}{)} \PYG{o}{+} \PYG{n}{a}
\end{Verbatim}
\begin{description}
\item[{size}] \leavevmode{[}int or tuple of ints, optional{]}
Defines the shape of the returned array of random floats. If None
(the default), returns a single float.

\end{description}
\begin{description}
\item[{out}] \leavevmode{[}float or ndarray of floats{]}
Array of random floats of shape \emph{size} (unless \code{size=None}, in which
case a single float is returned).

\end{description}

\begin{Verbatim}[commandchars=\\\{\}]
\PYG{g+gp}{\PYGZgt{}\PYGZgt{}\PYGZgt{} }\PYG{n}{np}\PYG{o}{.}\PYG{n}{random}\PYG{o}{.}\PYG{n}{random\PYGZus{}sample}\PYG{p}{(}\PYG{p}{)}
\PYG{g+go}{0.47108547995356098}
\PYG{g+gp}{\PYGZgt{}\PYGZgt{}\PYGZgt{} }\PYG{n+nb}{type}\PYG{p}{(}\PYG{n}{np}\PYG{o}{.}\PYG{n}{random}\PYG{o}{.}\PYG{n}{random\PYGZus{}sample}\PYG{p}{(}\PYG{p}{)}\PYG{p}{)}
\PYG{g+go}{\PYGZlt{}type \PYGZsq{}float\PYGZsq{}\PYGZgt{}}
\PYG{g+gp}{\PYGZgt{}\PYGZgt{}\PYGZgt{} }\PYG{n}{np}\PYG{o}{.}\PYG{n}{random}\PYG{o}{.}\PYG{n}{random\PYGZus{}sample}\PYG{p}{(}\PYG{p}{(}\PYG{l+m+mi}{5}\PYG{p}{,}\PYG{p}{)}\PYG{p}{)}
\PYG{g+go}{array([ 0.30220482,  0.86820401,  0.1654503 ,  0.11659149,  0.54323428])}
\end{Verbatim}

Three-by-two array of random numbers from {[}-5, 0):

\begin{Verbatim}[commandchars=\\\{\}]
\PYG{g+gp}{\PYGZgt{}\PYGZgt{}\PYGZgt{} }\PYG{l+m+mi}{5} \PYG{o}{*} \PYG{n}{np}\PYG{o}{.}\PYG{n}{random}\PYG{o}{.}\PYG{n}{random\PYGZus{}sample}\PYG{p}{(}\PYG{p}{(}\PYG{l+m+mi}{3}\PYG{p}{,} \PYG{l+m+mi}{2}\PYG{p}{)}\PYG{p}{)} \PYG{o}{\PYGZhy{}} \PYG{l+m+mi}{5}
\PYG{g+go}{array([[\PYGZhy{}3.99149989, \PYGZhy{}0.52338984],}
\PYG{g+go}{       [\PYGZhy{}2.99091858, \PYGZhy{}0.79479508],}
\PYG{g+go}{       [\PYGZhy{}1.23204345, \PYGZhy{}1.75224494]])}
\end{Verbatim}

\end{fulllineitems}

\index{ranf() (in module topology\_analysis)}

\begin{fulllineitems}
\phantomsection\label{topology_analysis:topology_analysis.ranf}\pysiglinewithargsret{\code{topology\_analysis.}\bfcode{ranf}}{}{}
random\_sample(size=None)

Return random floats in the half-open interval {[}0.0, 1.0).

Results are from the ``continuous uniform'' distribution over the
stated interval.  To sample \(Unif[a, b), b > a\) multiply
the output of \emph{random\_sample} by \emph{(b-a)} and add \emph{a}:

\begin{Verbatim}[commandchars=\\\{\}]
\PYG{p}{(}\PYG{n}{b} \PYG{o}{\PYGZhy{}} \PYG{n}{a}\PYG{p}{)} \PYG{o}{*} \PYG{n}{random\PYGZus{}sample}\PYG{p}{(}\PYG{p}{)} \PYG{o}{+} \PYG{n}{a}
\end{Verbatim}
\begin{description}
\item[{size}] \leavevmode{[}int or tuple of ints, optional{]}
Defines the shape of the returned array of random floats. If None
(the default), returns a single float.

\end{description}
\begin{description}
\item[{out}] \leavevmode{[}float or ndarray of floats{]}
Array of random floats of shape \emph{size} (unless \code{size=None}, in which
case a single float is returned).

\end{description}

\begin{Verbatim}[commandchars=\\\{\}]
\PYG{g+gp}{\PYGZgt{}\PYGZgt{}\PYGZgt{} }\PYG{n}{np}\PYG{o}{.}\PYG{n}{random}\PYG{o}{.}\PYG{n}{random\PYGZus{}sample}\PYG{p}{(}\PYG{p}{)}
\PYG{g+go}{0.47108547995356098}
\PYG{g+gp}{\PYGZgt{}\PYGZgt{}\PYGZgt{} }\PYG{n+nb}{type}\PYG{p}{(}\PYG{n}{np}\PYG{o}{.}\PYG{n}{random}\PYG{o}{.}\PYG{n}{random\PYGZus{}sample}\PYG{p}{(}\PYG{p}{)}\PYG{p}{)}
\PYG{g+go}{\PYGZlt{}type \PYGZsq{}float\PYGZsq{}\PYGZgt{}}
\PYG{g+gp}{\PYGZgt{}\PYGZgt{}\PYGZgt{} }\PYG{n}{np}\PYG{o}{.}\PYG{n}{random}\PYG{o}{.}\PYG{n}{random\PYGZus{}sample}\PYG{p}{(}\PYG{p}{(}\PYG{l+m+mi}{5}\PYG{p}{,}\PYG{p}{)}\PYG{p}{)}
\PYG{g+go}{array([ 0.30220482,  0.86820401,  0.1654503 ,  0.11659149,  0.54323428])}
\end{Verbatim}

Three-by-two array of random numbers from {[}-5, 0):

\begin{Verbatim}[commandchars=\\\{\}]
\PYG{g+gp}{\PYGZgt{}\PYGZgt{}\PYGZgt{} }\PYG{l+m+mi}{5} \PYG{o}{*} \PYG{n}{np}\PYG{o}{.}\PYG{n}{random}\PYG{o}{.}\PYG{n}{random\PYGZus{}sample}\PYG{p}{(}\PYG{p}{(}\PYG{l+m+mi}{3}\PYG{p}{,} \PYG{l+m+mi}{2}\PYG{p}{)}\PYG{p}{)} \PYG{o}{\PYGZhy{}} \PYG{l+m+mi}{5}
\PYG{g+go}{array([[\PYGZhy{}3.99149989, \PYGZhy{}0.52338984],}
\PYG{g+go}{       [\PYGZhy{}2.99091858, \PYGZhy{}0.79479508],}
\PYG{g+go}{       [\PYGZhy{}1.23204345, \PYGZhy{}1.75224494]])}
\end{Verbatim}

\end{fulllineitems}

\index{rayleigh() (in module topology\_analysis)}

\begin{fulllineitems}
\phantomsection\label{topology_analysis:topology_analysis.rayleigh}\pysiglinewithargsret{\code{topology\_analysis.}\bfcode{rayleigh}}{\emph{scale=1.0}, \emph{size=None}}{}
Draw samples from a Rayleigh distribution.

The \(\chi\) and Weibull distributions are generalizations of the
Rayleigh.
\begin{description}
\item[{scale}] \leavevmode{[}scalar{]}
Scale, also equals the mode. Should be \textgreater{}= 0.

\item[{size}] \leavevmode{[}int or tuple of ints, optional{]}
Shape of the output. Default is None, in which case a single
value is returned.

\end{description}

The probability density function for the Rayleigh distribution is
\begin{gather}
\begin{split}P(x;scale) = \frac{x}{scale^2}e^{\frac{-x^2}{2 \cdotp scale^2}}\end{split}\notag
\end{gather}
The Rayleigh distribution arises if the wind speed and wind direction are
both gaussian variables, then the vector wind velocity forms a Rayleigh
distribution. The Rayleigh distribution is used to model the expected
output from wind turbines.

Draw values from the distribution and plot the histogram

\begin{Verbatim}[commandchars=\\\{\}]
\PYG{g+gp}{\PYGZgt{}\PYGZgt{}\PYGZgt{} }\PYG{n}{values} \PYG{o}{=} \PYG{n}{hist}\PYG{p}{(}\PYG{n}{np}\PYG{o}{.}\PYG{n}{random}\PYG{o}{.}\PYG{n}{rayleigh}\PYG{p}{(}\PYG{l+m+mi}{3}\PYG{p}{,} \PYG{l+m+mi}{100000}\PYG{p}{)}\PYG{p}{,} \PYG{n}{bins}\PYG{o}{=}\PYG{l+m+mi}{200}\PYG{p}{,} \PYG{n}{normed}\PYG{o}{=}\PYG{n+nb+bp}{True}\PYG{p}{)}
\end{Verbatim}

Wave heights tend to follow a Rayleigh distribution. If the mean wave
height is 1 meter, what fraction of waves are likely to be larger than 3
meters?

\begin{Verbatim}[commandchars=\\\{\}]
\PYG{g+gp}{\PYGZgt{}\PYGZgt{}\PYGZgt{} }\PYG{n}{meanvalue} \PYG{o}{=} \PYG{l+m+mi}{1}
\PYG{g+gp}{\PYGZgt{}\PYGZgt{}\PYGZgt{} }\PYG{n}{modevalue} \PYG{o}{=} \PYG{n}{np}\PYG{o}{.}\PYG{n}{sqrt}\PYG{p}{(}\PYG{l+m+mi}{2} \PYG{o}{/} \PYG{n}{np}\PYG{o}{.}\PYG{n}{pi}\PYG{p}{)} \PYG{o}{*} \PYG{n}{meanvalue}
\PYG{g+gp}{\PYGZgt{}\PYGZgt{}\PYGZgt{} }\PYG{n}{s} \PYG{o}{=} \PYG{n}{np}\PYG{o}{.}\PYG{n}{random}\PYG{o}{.}\PYG{n}{rayleigh}\PYG{p}{(}\PYG{n}{modevalue}\PYG{p}{,} \PYG{l+m+mi}{1000000}\PYG{p}{)}
\end{Verbatim}

The percentage of waves larger than 3 meters is:

\begin{Verbatim}[commandchars=\\\{\}]
\PYG{g+gp}{\PYGZgt{}\PYGZgt{}\PYGZgt{} }\PYG{l+m+mf}{100.}\PYG{o}{*}\PYG{n+nb}{sum}\PYG{p}{(}\PYG{n}{s}\PYG{o}{\PYGZgt{}}\PYG{l+m+mi}{3}\PYG{p}{)}\PYG{o}{/}\PYG{l+m+mf}{1000000.}
\PYG{g+go}{0.087300000000000003}
\end{Verbatim}

\end{fulllineitems}

\index{sample() (in module topology\_analysis)}

\begin{fulllineitems}
\phantomsection\label{topology_analysis:topology_analysis.sample}\pysiglinewithargsret{\code{topology\_analysis.}\bfcode{sample}}{}{}
random\_sample(size=None)

Return random floats in the half-open interval {[}0.0, 1.0).

Results are from the ``continuous uniform'' distribution over the
stated interval.  To sample \(Unif[a, b), b > a\) multiply
the output of \emph{random\_sample} by \emph{(b-a)} and add \emph{a}:

\begin{Verbatim}[commandchars=\\\{\}]
\PYG{p}{(}\PYG{n}{b} \PYG{o}{\PYGZhy{}} \PYG{n}{a}\PYG{p}{)} \PYG{o}{*} \PYG{n}{random\PYGZus{}sample}\PYG{p}{(}\PYG{p}{)} \PYG{o}{+} \PYG{n}{a}
\end{Verbatim}
\begin{description}
\item[{size}] \leavevmode{[}int or tuple of ints, optional{]}
Defines the shape of the returned array of random floats. If None
(the default), returns a single float.

\end{description}
\begin{description}
\item[{out}] \leavevmode{[}float or ndarray of floats{]}
Array of random floats of shape \emph{size} (unless \code{size=None}, in which
case a single float is returned).

\end{description}

\begin{Verbatim}[commandchars=\\\{\}]
\PYG{g+gp}{\PYGZgt{}\PYGZgt{}\PYGZgt{} }\PYG{n}{np}\PYG{o}{.}\PYG{n}{random}\PYG{o}{.}\PYG{n}{random\PYGZus{}sample}\PYG{p}{(}\PYG{p}{)}
\PYG{g+go}{0.47108547995356098}
\PYG{g+gp}{\PYGZgt{}\PYGZgt{}\PYGZgt{} }\PYG{n+nb}{type}\PYG{p}{(}\PYG{n}{np}\PYG{o}{.}\PYG{n}{random}\PYG{o}{.}\PYG{n}{random\PYGZus{}sample}\PYG{p}{(}\PYG{p}{)}\PYG{p}{)}
\PYG{g+go}{\PYGZlt{}type \PYGZsq{}float\PYGZsq{}\PYGZgt{}}
\PYG{g+gp}{\PYGZgt{}\PYGZgt{}\PYGZgt{} }\PYG{n}{np}\PYG{o}{.}\PYG{n}{random}\PYG{o}{.}\PYG{n}{random\PYGZus{}sample}\PYG{p}{(}\PYG{p}{(}\PYG{l+m+mi}{5}\PYG{p}{,}\PYG{p}{)}\PYG{p}{)}
\PYG{g+go}{array([ 0.30220482,  0.86820401,  0.1654503 ,  0.11659149,  0.54323428])}
\end{Verbatim}

Three-by-two array of random numbers from {[}-5, 0):

\begin{Verbatim}[commandchars=\\\{\}]
\PYG{g+gp}{\PYGZgt{}\PYGZgt{}\PYGZgt{} }\PYG{l+m+mi}{5} \PYG{o}{*} \PYG{n}{np}\PYG{o}{.}\PYG{n}{random}\PYG{o}{.}\PYG{n}{random\PYGZus{}sample}\PYG{p}{(}\PYG{p}{(}\PYG{l+m+mi}{3}\PYG{p}{,} \PYG{l+m+mi}{2}\PYG{p}{)}\PYG{p}{)} \PYG{o}{\PYGZhy{}} \PYG{l+m+mi}{5}
\PYG{g+go}{array([[\PYGZhy{}3.99149989, \PYGZhy{}0.52338984],}
\PYG{g+go}{       [\PYGZhy{}2.99091858, \PYGZhy{}0.79479508],}
\PYG{g+go}{       [\PYGZhy{}1.23204345, \PYGZhy{}1.75224494]])}
\end{Verbatim}

\end{fulllineitems}

\index{seed() (in module topology\_analysis)}

\begin{fulllineitems}
\phantomsection\label{topology_analysis:topology_analysis.seed}\pysiglinewithargsret{\code{topology\_analysis.}\bfcode{seed}}{\emph{seed=None}}{}
Seed the generator.

This method is called when \emph{RandomState} is initialized. It can be
called again to re-seed the generator. For details, see \emph{RandomState}.
\begin{description}
\item[{seed}] \leavevmode{[}int or array\_like, optional{]}
Seed for \emph{RandomState}.

\end{description}

RandomState

\end{fulllineitems}

\index{set\_state() (in module topology\_analysis)}

\begin{fulllineitems}
\phantomsection\label{topology_analysis:topology_analysis.set_state}\pysiglinewithargsret{\code{topology\_analysis.}\bfcode{set\_state}}{\emph{state}}{}
Set the internal state of the generator from a tuple.

For use if one has reason to manually (re-)set the internal state of the
``Mersenne Twister''{\color{red}\bfseries{}{[}1{]}\_} pseudo-random number generating algorithm.
\begin{description}
\item[{state}] \leavevmode{[}tuple(str, ndarray of 624 uints, int, int, float){]}
The \emph{state} tuple has the following items:
\begin{enumerate}
\item {} 
the string `MT19937', specifying the Mersenne Twister algorithm.

\item {} 
a 1-D array of 624 unsigned integers \code{keys}.

\item {} 
an integer \code{pos}.

\item {} 
an integer \code{has\_gauss}.

\item {} 
a float \code{cached\_gaussian}.

\end{enumerate}

\end{description}
\begin{description}
\item[{out}] \leavevmode{[}None{]}
Returns `None' on success.

\end{description}

get\_state

\emph{set\_state} and \emph{get\_state} are not needed to work with any of the
random distributions in NumPy. If the internal state is manually altered,
the user should know exactly what he/she is doing.

For backwards compatibility, the form (str, array of 624 uints, int) is
also accepted although it is missing some information about the cached
Gaussian value: \code{state = ('MT19937', keys, pos)}.

\end{fulllineitems}

\index{shuffle() (in module topology\_analysis)}

\begin{fulllineitems}
\phantomsection\label{topology_analysis:topology_analysis.shuffle}\pysiglinewithargsret{\code{topology\_analysis.}\bfcode{shuffle}}{\emph{x}}{}
Modify a sequence in-place by shuffling its contents.
\begin{description}
\item[{x}] \leavevmode{[}array\_like{]}
The array or list to be shuffled.

\end{description}

None

\begin{Verbatim}[commandchars=\\\{\}]
\PYG{g+gp}{\PYGZgt{}\PYGZgt{}\PYGZgt{} }\PYG{n}{arr} \PYG{o}{=} \PYG{n}{np}\PYG{o}{.}\PYG{n}{arange}\PYG{p}{(}\PYG{l+m+mi}{10}\PYG{p}{)}
\PYG{g+gp}{\PYGZgt{}\PYGZgt{}\PYGZgt{} }\PYG{n}{np}\PYG{o}{.}\PYG{n}{random}\PYG{o}{.}\PYG{n}{shuffle}\PYG{p}{(}\PYG{n}{arr}\PYG{p}{)}
\PYG{g+gp}{\PYGZgt{}\PYGZgt{}\PYGZgt{} }\PYG{n}{arr}
\PYG{g+go}{[1 7 5 2 9 4 3 6 0 8]}
\end{Verbatim}

This function only shuffles the array along the first index of a
multi-dimensional array:

\begin{Verbatim}[commandchars=\\\{\}]
\PYG{g+gp}{\PYGZgt{}\PYGZgt{}\PYGZgt{} }\PYG{n}{arr} \PYG{o}{=} \PYG{n}{np}\PYG{o}{.}\PYG{n}{arange}\PYG{p}{(}\PYG{l+m+mi}{9}\PYG{p}{)}\PYG{o}{.}\PYG{n}{reshape}\PYG{p}{(}\PYG{p}{(}\PYG{l+m+mi}{3}\PYG{p}{,} \PYG{l+m+mi}{3}\PYG{p}{)}\PYG{p}{)}
\PYG{g+gp}{\PYGZgt{}\PYGZgt{}\PYGZgt{} }\PYG{n}{np}\PYG{o}{.}\PYG{n}{random}\PYG{o}{.}\PYG{n}{shuffle}\PYG{p}{(}\PYG{n}{arr}\PYG{p}{)}
\PYG{g+gp}{\PYGZgt{}\PYGZgt{}\PYGZgt{} }\PYG{n}{arr}
\PYG{g+go}{array([[3, 4, 5],}
\PYG{g+go}{       [6, 7, 8],}
\PYG{g+go}{       [0, 1, 2]])}
\end{Verbatim}

\end{fulllineitems}

\index{standard\_cauchy() (in module topology\_analysis)}

\begin{fulllineitems}
\phantomsection\label{topology_analysis:topology_analysis.standard_cauchy}\pysiglinewithargsret{\code{topology\_analysis.}\bfcode{standard\_cauchy}}{\emph{size=None}}{}
Standard Cauchy distribution with mode = 0.

Also known as the Lorentz distribution.
\begin{description}
\item[{size}] \leavevmode{[}int or tuple of ints{]}
Shape of the output.

\end{description}
\begin{description}
\item[{samples}] \leavevmode{[}ndarray or scalar{]}
The drawn samples.

\end{description}

The probability density function for the full Cauchy distribution is
\begin{gather}
\begin{split}P(x; x_0, \gamma) = \frac{1}{\pi \gamma \bigl[ 1+
(\frac{x-x_0}{\gamma})^2 \bigr] }\end{split}\notag
\end{gather}
and the Standard Cauchy distribution just sets \(x_0=0\) and
\(\gamma=1\)

The Cauchy distribution arises in the solution to the driven harmonic
oscillator problem, and also describes spectral line broadening. It
also describes the distribution of values at which a line tilted at
a random angle will cut the x axis.

When studying hypothesis tests that assume normality, seeing how the
tests perform on data from a Cauchy distribution is a good indicator of
their sensitivity to a heavy-tailed distribution, since the Cauchy looks
very much like a Gaussian distribution, but with heavier tails.

Draw samples and plot the distribution:

\begin{Verbatim}[commandchars=\\\{\}]
\PYG{g+gp}{\PYGZgt{}\PYGZgt{}\PYGZgt{} }\PYG{n}{s} \PYG{o}{=} \PYG{n}{np}\PYG{o}{.}\PYG{n}{random}\PYG{o}{.}\PYG{n}{standard\PYGZus{}cauchy}\PYG{p}{(}\PYG{l+m+mi}{1000000}\PYG{p}{)}
\PYG{g+gp}{\PYGZgt{}\PYGZgt{}\PYGZgt{} }\PYG{n}{s} \PYG{o}{=} \PYG{n}{s}\PYG{p}{[}\PYG{p}{(}\PYG{n}{s}\PYG{o}{\PYGZgt{}}\PYG{o}{\PYGZhy{}}\PYG{l+m+mi}{25}\PYG{p}{)} \PYG{o}{\PYGZam{}} \PYG{p}{(}\PYG{n}{s}\PYG{o}{\PYGZlt{}}\PYG{l+m+mi}{25}\PYG{p}{)}\PYG{p}{]}  \PYG{c}{\PYGZsh{} truncate distribution so it plots well}
\PYG{g+gp}{\PYGZgt{}\PYGZgt{}\PYGZgt{} }\PYG{n}{plt}\PYG{o}{.}\PYG{n}{hist}\PYG{p}{(}\PYG{n}{s}\PYG{p}{,} \PYG{n}{bins}\PYG{o}{=}\PYG{l+m+mi}{100}\PYG{p}{)}
\PYG{g+gp}{\PYGZgt{}\PYGZgt{}\PYGZgt{} }\PYG{n}{plt}\PYG{o}{.}\PYG{n}{show}\PYG{p}{(}\PYG{p}{)}
\end{Verbatim}

\end{fulllineitems}

\index{standard\_exponential() (in module topology\_analysis)}

\begin{fulllineitems}
\phantomsection\label{topology_analysis:topology_analysis.standard_exponential}\pysiglinewithargsret{\code{topology\_analysis.}\bfcode{standard\_exponential}}{\emph{size=None}}{}
Draw samples from the standard exponential distribution.

\emph{standard\_exponential} is identical to the exponential distribution
with a scale parameter of 1.
\begin{description}
\item[{size}] \leavevmode{[}int or tuple of ints{]}
Shape of the output.

\end{description}
\begin{description}
\item[{out}] \leavevmode{[}float or ndarray{]}
Drawn samples.

\end{description}

Output a 3x8000 array:

\begin{Verbatim}[commandchars=\\\{\}]
\PYG{g+gp}{\PYGZgt{}\PYGZgt{}\PYGZgt{} }\PYG{n}{n} \PYG{o}{=} \PYG{n}{np}\PYG{o}{.}\PYG{n}{random}\PYG{o}{.}\PYG{n}{standard\PYGZus{}exponential}\PYG{p}{(}\PYG{p}{(}\PYG{l+m+mi}{3}\PYG{p}{,} \PYG{l+m+mi}{8000}\PYG{p}{)}\PYG{p}{)}
\end{Verbatim}

\end{fulllineitems}

\index{standard\_gamma() (in module topology\_analysis)}

\begin{fulllineitems}
\phantomsection\label{topology_analysis:topology_analysis.standard_gamma}\pysiglinewithargsret{\code{topology\_analysis.}\bfcode{standard\_gamma}}{\emph{shape}, \emph{size=None}}{}
Draw samples from a Standard Gamma distribution.

Samples are drawn from a Gamma distribution with specified parameters,
shape (sometimes designated ``k'') and scale=1.
\begin{description}
\item[{shape}] \leavevmode{[}float{]}
Parameter, should be \textgreater{} 0.

\item[{size}] \leavevmode{[}int or tuple of ints{]}
Output shape.  If the given shape is, e.g., \code{(m, n, k)}, then
\code{m * n * k} samples are drawn.

\end{description}
\begin{description}
\item[{samples}] \leavevmode{[}ndarray or scalar{]}
The drawn samples.

\end{description}
\begin{description}
\item[{scipy.stats.distributions.gamma}] \leavevmode{[}probability density function,{]}
distribution or cumulative density function, etc.

\end{description}

The probability density for the Gamma distribution is
\begin{gather}
\begin{split}p(x) = x^{k-1}\frac{e^{-x/\theta}}{\theta^k\Gamma(k)},\end{split}\notag
\end{gather}
where \(k\) is the shape and \(\theta\) the scale,
and \(\Gamma\) is the Gamma function.

The Gamma distribution is often used to model the times to failure of
electronic components, and arises naturally in processes for which the
waiting times between Poisson distributed events are relevant.

Draw samples from the distribution:

\begin{Verbatim}[commandchars=\\\{\}]
\PYG{g+gp}{\PYGZgt{}\PYGZgt{}\PYGZgt{} }\PYG{n}{shape}\PYG{p}{,} \PYG{n}{scale} \PYG{o}{=} \PYG{l+m+mf}{2.}\PYG{p}{,} \PYG{l+m+mf}{1.} \PYG{c}{\PYGZsh{} mean and width}
\PYG{g+gp}{\PYGZgt{}\PYGZgt{}\PYGZgt{} }\PYG{n}{s} \PYG{o}{=} \PYG{n}{np}\PYG{o}{.}\PYG{n}{random}\PYG{o}{.}\PYG{n}{standard\PYGZus{}gamma}\PYG{p}{(}\PYG{n}{shape}\PYG{p}{,} \PYG{l+m+mi}{1000000}\PYG{p}{)}
\end{Verbatim}

Display the histogram of the samples, along with
the probability density function:

\begin{Verbatim}[commandchars=\\\{\}]
\PYG{g+gp}{\PYGZgt{}\PYGZgt{}\PYGZgt{} }\PYG{k+kn}{import} \PYG{n+nn}{matplotlib.pyplot} \PYG{k+kn}{as} \PYG{n+nn}{plt}
\PYG{g+gp}{\PYGZgt{}\PYGZgt{}\PYGZgt{} }\PYG{k+kn}{import} \PYG{n+nn}{scipy.special} \PYG{k+kn}{as} \PYG{n+nn}{sps}
\PYG{g+gp}{\PYGZgt{}\PYGZgt{}\PYGZgt{} }\PYG{n}{count}\PYG{p}{,} \PYG{n}{bins}\PYG{p}{,} \PYG{n}{ignored} \PYG{o}{=} \PYG{n}{plt}\PYG{o}{.}\PYG{n}{hist}\PYG{p}{(}\PYG{n}{s}\PYG{p}{,} \PYG{l+m+mi}{50}\PYG{p}{,} \PYG{n}{normed}\PYG{o}{=}\PYG{n+nb+bp}{True}\PYG{p}{)}
\PYG{g+gp}{\PYGZgt{}\PYGZgt{}\PYGZgt{} }\PYG{n}{y} \PYG{o}{=} \PYG{n}{bins}\PYG{o}{*}\PYG{o}{*}\PYG{p}{(}\PYG{n}{shape}\PYG{o}{\PYGZhy{}}\PYG{l+m+mi}{1}\PYG{p}{)} \PYG{o}{*} \PYG{p}{(}\PYG{p}{(}\PYG{n}{np}\PYG{o}{.}\PYG{n}{exp}\PYG{p}{(}\PYG{o}{\PYGZhy{}}\PYG{n}{bins}\PYG{o}{/}\PYG{n}{scale}\PYG{p}{)}\PYG{p}{)}\PYG{o}{/} \PYGZbs{}
\PYG{g+gp}{... }                      \PYG{p}{(}\PYG{n}{sps}\PYG{o}{.}\PYG{n}{gamma}\PYG{p}{(}\PYG{n}{shape}\PYG{p}{)} \PYG{o}{*} \PYG{n}{scale}\PYG{o}{*}\PYG{o}{*}\PYG{n}{shape}\PYG{p}{)}\PYG{p}{)}
\PYG{g+gp}{\PYGZgt{}\PYGZgt{}\PYGZgt{} }\PYG{n}{plt}\PYG{o}{.}\PYG{n}{plot}\PYG{p}{(}\PYG{n}{bins}\PYG{p}{,} \PYG{n}{y}\PYG{p}{,} \PYG{n}{linewidth}\PYG{o}{=}\PYG{l+m+mi}{2}\PYG{p}{,} \PYG{n}{color}\PYG{o}{=}\PYG{l+s}{\PYGZsq{}}\PYG{l+s}{r}\PYG{l+s}{\PYGZsq{}}\PYG{p}{)}
\PYG{g+gp}{\PYGZgt{}\PYGZgt{}\PYGZgt{} }\PYG{n}{plt}\PYG{o}{.}\PYG{n}{show}\PYG{p}{(}\PYG{p}{)}
\end{Verbatim}

\end{fulllineitems}

\index{standard\_normal() (in module topology\_analysis)}

\begin{fulllineitems}
\phantomsection\label{topology_analysis:topology_analysis.standard_normal}\pysiglinewithargsret{\code{topology\_analysis.}\bfcode{standard\_normal}}{\emph{size=None}}{}
Returns samples from a Standard Normal distribution (mean=0, stdev=1).
\begin{description}
\item[{size}] \leavevmode{[}int or tuple of ints, optional{]}
Output shape. Default is None, in which case a single value is
returned.

\end{description}
\begin{description}
\item[{out}] \leavevmode{[}float or ndarray{]}
Drawn samples.

\end{description}

\begin{Verbatim}[commandchars=\\\{\}]
\PYG{g+gp}{\PYGZgt{}\PYGZgt{}\PYGZgt{} }\PYG{n}{s} \PYG{o}{=} \PYG{n}{np}\PYG{o}{.}\PYG{n}{random}\PYG{o}{.}\PYG{n}{standard\PYGZus{}normal}\PYG{p}{(}\PYG{l+m+mi}{8000}\PYG{p}{)}
\PYG{g+gp}{\PYGZgt{}\PYGZgt{}\PYGZgt{} }\PYG{n}{s}
\PYG{g+go}{array([ 0.6888893 ,  0.78096262, \PYGZhy{}0.89086505, ...,  0.49876311, \PYGZsh{}random}
\PYG{g+go}{       \PYGZhy{}0.38672696, \PYGZhy{}0.4685006 ])                               \PYGZsh{}random}
\PYG{g+gp}{\PYGZgt{}\PYGZgt{}\PYGZgt{} }\PYG{n}{s}\PYG{o}{.}\PYG{n}{shape}
\PYG{g+go}{(8000,)}
\PYG{g+gp}{\PYGZgt{}\PYGZgt{}\PYGZgt{} }\PYG{n}{s} \PYG{o}{=} \PYG{n}{np}\PYG{o}{.}\PYG{n}{random}\PYG{o}{.}\PYG{n}{standard\PYGZus{}normal}\PYG{p}{(}\PYG{n}{size}\PYG{o}{=}\PYG{p}{(}\PYG{l+m+mi}{3}\PYG{p}{,} \PYG{l+m+mi}{4}\PYG{p}{,} \PYG{l+m+mi}{2}\PYG{p}{)}\PYG{p}{)}
\PYG{g+gp}{\PYGZgt{}\PYGZgt{}\PYGZgt{} }\PYG{n}{s}\PYG{o}{.}\PYG{n}{shape}
\PYG{g+go}{(3, 4, 2)}
\end{Verbatim}

\end{fulllineitems}

\index{standard\_t() (in module topology\_analysis)}

\begin{fulllineitems}
\phantomsection\label{topology_analysis:topology_analysis.standard_t}\pysiglinewithargsret{\code{topology\_analysis.}\bfcode{standard\_t}}{\emph{df}, \emph{size=None}}{}
Standard Student's t distribution with df degrees of freedom.

A special case of the hyperbolic distribution.
As \emph{df} gets large, the result resembles that of the standard normal
distribution (\emph{standard\_normal}).
\begin{description}
\item[{df}] \leavevmode{[}int{]}
Degrees of freedom, should be \textgreater{} 0.

\item[{size}] \leavevmode{[}int or tuple of ints, optional{]}
Output shape. Default is None, in which case a single value is
returned.

\end{description}
\begin{description}
\item[{samples}] \leavevmode{[}ndarray or scalar{]}
Drawn samples.

\end{description}

The probability density function for the t distribution is
\begin{gather}
\begin{split}P(x, df) = \frac{\Gamma(\frac{df+1}{2})}{\sqrt{\pi df}
\Gamma(\frac{df}{2})}\Bigl( 1+\frac{x^2}{df} \Bigr)^{-(df+1)/2}\end{split}\notag
\end{gather}
The t test is based on an assumption that the data come from a Normal
distribution. The t test provides a way to test whether the sample mean
(that is the mean calculated from the data) is a good estimate of the true
mean.

The derivation of the t-distribution was forst published in 1908 by William
Gisset while working for the Guinness Brewery in Dublin. Due to proprietary
issues, he had to publish under a pseudonym, and so he used the name
Student.

From Dalgaard page 83 {\color{red}\bfseries{}{[}1{]}\_}, suppose the daily energy intake for 11
women in Kj is:

\begin{Verbatim}[commandchars=\\\{\}]
\PYG{g+gp}{\PYGZgt{}\PYGZgt{}\PYGZgt{} }\PYG{n}{intake} \PYG{o}{=} \PYG{n}{np}\PYG{o}{.}\PYG{n}{array}\PYG{p}{(}\PYG{p}{[}\PYG{l+m+mf}{5260.}\PYG{p}{,} \PYG{l+m+mi}{5470}\PYG{p}{,} \PYG{l+m+mi}{5640}\PYG{p}{,} \PYG{l+m+mi}{6180}\PYG{p}{,} \PYG{l+m+mi}{6390}\PYG{p}{,} \PYG{l+m+mi}{6515}\PYG{p}{,} \PYG{l+m+mi}{6805}\PYG{p}{,} \PYG{l+m+mi}{7515}\PYG{p}{,} \PYGZbs{}
\PYG{g+gp}{... }                   \PYG{l+m+mi}{7515}\PYG{p}{,} \PYG{l+m+mi}{8230}\PYG{p}{,} \PYG{l+m+mi}{8770}\PYG{p}{]}\PYG{p}{)}
\end{Verbatim}

Does their energy intake deviate systematically from the recommended
value of 7725 kJ?

We have 10 degrees of freedom, so is the sample mean within 95\% of the
recommended value?

\begin{Verbatim}[commandchars=\\\{\}]
\PYG{g+gp}{\PYGZgt{}\PYGZgt{}\PYGZgt{} }\PYG{n}{s} \PYG{o}{=} \PYG{n}{np}\PYG{o}{.}\PYG{n}{random}\PYG{o}{.}\PYG{n}{standard\PYGZus{}t}\PYG{p}{(}\PYG{l+m+mi}{10}\PYG{p}{,} \PYG{n}{size}\PYG{o}{=}\PYG{l+m+mi}{100000}\PYG{p}{)}
\PYG{g+gp}{\PYGZgt{}\PYGZgt{}\PYGZgt{} }\PYG{n}{np}\PYG{o}{.}\PYG{n}{mean}\PYG{p}{(}\PYG{n}{intake}\PYG{p}{)}
\PYG{g+go}{6753.636363636364}
\PYG{g+gp}{\PYGZgt{}\PYGZgt{}\PYGZgt{} }\PYG{n}{intake}\PYG{o}{.}\PYG{n}{std}\PYG{p}{(}\PYG{n}{ddof}\PYG{o}{=}\PYG{l+m+mi}{1}\PYG{p}{)}
\PYG{g+go}{1142.1232221373727}
\end{Verbatim}

Calculate the t statistic, setting the ddof parameter to the unbiased
value so the divisor in the standard deviation will be degrees of
freedom, N-1.

\begin{Verbatim}[commandchars=\\\{\}]
\PYG{g+gp}{\PYGZgt{}\PYGZgt{}\PYGZgt{} }\PYG{n}{t} \PYG{o}{=} \PYG{p}{(}\PYG{n}{np}\PYG{o}{.}\PYG{n}{mean}\PYG{p}{(}\PYG{n}{intake}\PYG{p}{)}\PYG{o}{\PYGZhy{}}\PYG{l+m+mi}{7725}\PYG{p}{)}\PYG{o}{/}\PYG{p}{(}\PYG{n}{intake}\PYG{o}{.}\PYG{n}{std}\PYG{p}{(}\PYG{n}{ddof}\PYG{o}{=}\PYG{l+m+mi}{1}\PYG{p}{)}\PYG{o}{/}\PYG{n}{np}\PYG{o}{.}\PYG{n}{sqrt}\PYG{p}{(}\PYG{n+nb}{len}\PYG{p}{(}\PYG{n}{intake}\PYG{p}{)}\PYG{p}{)}\PYG{p}{)}
\PYG{g+gp}{\PYGZgt{}\PYGZgt{}\PYGZgt{} }\PYG{k+kn}{import} \PYG{n+nn}{matplotlib.pyplot} \PYG{k+kn}{as} \PYG{n+nn}{plt}
\PYG{g+gp}{\PYGZgt{}\PYGZgt{}\PYGZgt{} }\PYG{n}{h} \PYG{o}{=} \PYG{n}{plt}\PYG{o}{.}\PYG{n}{hist}\PYG{p}{(}\PYG{n}{s}\PYG{p}{,} \PYG{n}{bins}\PYG{o}{=}\PYG{l+m+mi}{100}\PYG{p}{,} \PYG{n}{normed}\PYG{o}{=}\PYG{n+nb+bp}{True}\PYG{p}{)}
\end{Verbatim}

For a one-sided t-test, how far out in the distribution does the t
statistic appear?

\begin{Verbatim}[commandchars=\\\{\}]
\PYG{g+gp}{\PYGZgt{}\PYGZgt{}\PYGZgt{} }\PYG{o}{\PYGZgt{}\PYGZgt{}}\PYG{o}{\PYGZgt{}} \PYG{n}{np}\PYG{o}{.}\PYG{n}{sum}\PYG{p}{(}\PYG{n}{s}\PYG{o}{\PYGZlt{}}\PYG{n}{t}\PYG{p}{)} \PYG{o}{/} \PYG{n+nb}{float}\PYG{p}{(}\PYG{n+nb}{len}\PYG{p}{(}\PYG{n}{s}\PYG{p}{)}\PYG{p}{)}
\PYG{g+go}{0.0090699999999999999  \PYGZsh{}random}
\end{Verbatim}

So the p-value is about 0.009, which says the null hypothesis has a
probability of about 99\% of being true.

\end{fulllineitems}

\index{triangular() (in module topology\_analysis)}

\begin{fulllineitems}
\phantomsection\label{topology_analysis:topology_analysis.triangular}\pysiglinewithargsret{\code{topology\_analysis.}\bfcode{triangular}}{\emph{left}, \emph{mode}, \emph{right}, \emph{size=None}}{}
Draw samples from the triangular distribution.

The triangular distribution is a continuous probability distribution with
lower limit left, peak at mode, and upper limit right. Unlike the other
distributions, these parameters directly define the shape of the pdf.
\begin{description}
\item[{left}] \leavevmode{[}scalar{]}
Lower limit.

\item[{mode}] \leavevmode{[}scalar{]}
The value where the peak of the distribution occurs.
The value should fulfill the condition \code{left \textless{}= mode \textless{}= right}.

\item[{right}] \leavevmode{[}scalar{]}
Upper limit, should be larger than \emph{left}.

\item[{size}] \leavevmode{[}int or tuple of ints, optional{]}
Output shape. Default is None, in which case a single value is
returned.

\end{description}
\begin{description}
\item[{samples}] \leavevmode{[}ndarray or scalar{]}
The returned samples all lie in the interval {[}left, right{]}.

\end{description}

The probability density function for the Triangular distribution is
\begin{gather}
\begin{split}P(x;l, m, r) = \begin{cases}
\frac{2(x-l)}{(r-l)(m-l)}& \text{for $l \leq x \leq m$},\\
\frac{2(m-x)}{(r-l)(r-m)}& \text{for $m \leq x \leq r$},\\
0& \text{otherwise}.
\end{cases}\end{split}\notag
\end{gather}
The triangular distribution is often used in ill-defined problems where the
underlying distribution is not known, but some knowledge of the limits and
mode exists. Often it is used in simulations.

Draw values from the distribution and plot the histogram:

\begin{Verbatim}[commandchars=\\\{\}]
\PYG{g+gp}{\PYGZgt{}\PYGZgt{}\PYGZgt{} }\PYG{k+kn}{import} \PYG{n+nn}{matplotlib.pyplot} \PYG{k+kn}{as} \PYG{n+nn}{plt}
\PYG{g+gp}{\PYGZgt{}\PYGZgt{}\PYGZgt{} }\PYG{n}{h} \PYG{o}{=} \PYG{n}{plt}\PYG{o}{.}\PYG{n}{hist}\PYG{p}{(}\PYG{n}{np}\PYG{o}{.}\PYG{n}{random}\PYG{o}{.}\PYG{n}{triangular}\PYG{p}{(}\PYG{o}{\PYGZhy{}}\PYG{l+m+mi}{3}\PYG{p}{,} \PYG{l+m+mi}{0}\PYG{p}{,} \PYG{l+m+mi}{8}\PYG{p}{,} \PYG{l+m+mi}{100000}\PYG{p}{)}\PYG{p}{,} \PYG{n}{bins}\PYG{o}{=}\PYG{l+m+mi}{200}\PYG{p}{,}
\PYG{g+gp}{... }             \PYG{n}{normed}\PYG{o}{=}\PYG{n+nb+bp}{True}\PYG{p}{)}
\PYG{g+gp}{\PYGZgt{}\PYGZgt{}\PYGZgt{} }\PYG{n}{plt}\PYG{o}{.}\PYG{n}{show}\PYG{p}{(}\PYG{p}{)}
\end{Verbatim}

\end{fulllineitems}

\index{uniform() (in module topology\_analysis)}

\begin{fulllineitems}
\phantomsection\label{topology_analysis:topology_analysis.uniform}\pysiglinewithargsret{\code{topology\_analysis.}\bfcode{uniform}}{\emph{low=0.0}, \emph{high=1.0}, \emph{size=1}}{}
Draw samples from a uniform distribution.

Samples are uniformly distributed over the half-open interval
\code{{[}low, high)} (includes low, but excludes high).  In other words,
any value within the given interval is equally likely to be drawn
by \emph{uniform}.
\begin{description}
\item[{low}] \leavevmode{[}float, optional{]}
Lower boundary of the output interval.  All values generated will be
greater than or equal to low.  The default value is 0.

\item[{high}] \leavevmode{[}float{]}
Upper boundary of the output interval.  All values generated will be
less than high.  The default value is 1.0.

\item[{size}] \leavevmode{[}int or tuple of ints, optional{]}
Shape of output.  If the given size is, for example, (m,n,k),
m*n*k samples are generated.  If no shape is specified, a single sample
is returned.

\end{description}
\begin{description}
\item[{out}] \leavevmode{[}ndarray{]}
Drawn samples, with shape \emph{size}.

\end{description}

randint : Discrete uniform distribution, yielding integers.
random\_integers : Discrete uniform distribution over the closed
\begin{quote}

interval \code{{[}low, high{]}}.
\end{quote}

random\_sample : Floats uniformly distributed over \code{{[}0, 1)}.
random : Alias for \emph{random\_sample}.
rand : Convenience function that accepts dimensions as input, e.g.,
\begin{quote}

\code{rand(2,2)} would generate a 2-by-2 array of floats,
uniformly distributed over \code{{[}0, 1)}.
\end{quote}

The probability density function of the uniform distribution is
\begin{gather}
\begin{split}p(x) = \frac{1}{b - a}\end{split}\notag
\end{gather}
anywhere within the interval \code{{[}a, b)}, and zero elsewhere.

Draw samples from the distribution:

\begin{Verbatim}[commandchars=\\\{\}]
\PYG{g+gp}{\PYGZgt{}\PYGZgt{}\PYGZgt{} }\PYG{n}{s} \PYG{o}{=} \PYG{n}{np}\PYG{o}{.}\PYG{n}{random}\PYG{o}{.}\PYG{n}{uniform}\PYG{p}{(}\PYG{o}{\PYGZhy{}}\PYG{l+m+mi}{1}\PYG{p}{,}\PYG{l+m+mi}{0}\PYG{p}{,}\PYG{l+m+mi}{1000}\PYG{p}{)}
\end{Verbatim}

All values are within the given interval:

\begin{Verbatim}[commandchars=\\\{\}]
\PYG{g+gp}{\PYGZgt{}\PYGZgt{}\PYGZgt{} }\PYG{n}{np}\PYG{o}{.}\PYG{n}{all}\PYG{p}{(}\PYG{n}{s} \PYG{o}{\PYGZgt{}}\PYG{o}{=} \PYG{o}{\PYGZhy{}}\PYG{l+m+mi}{1}\PYG{p}{)}
\PYG{g+go}{True}
\PYG{g+gp}{\PYGZgt{}\PYGZgt{}\PYGZgt{} }\PYG{n}{np}\PYG{o}{.}\PYG{n}{all}\PYG{p}{(}\PYG{n}{s} \PYG{o}{\PYGZlt{}} \PYG{l+m+mi}{0}\PYG{p}{)}
\PYG{g+go}{True}
\end{Verbatim}

Display the histogram of the samples, along with the
probability density function:

\begin{Verbatim}[commandchars=\\\{\}]
\PYG{g+gp}{\PYGZgt{}\PYGZgt{}\PYGZgt{} }\PYG{k+kn}{import} \PYG{n+nn}{matplotlib.pyplot} \PYG{k+kn}{as} \PYG{n+nn}{plt}
\PYG{g+gp}{\PYGZgt{}\PYGZgt{}\PYGZgt{} }\PYG{n}{count}\PYG{p}{,} \PYG{n}{bins}\PYG{p}{,} \PYG{n}{ignored} \PYG{o}{=} \PYG{n}{plt}\PYG{o}{.}\PYG{n}{hist}\PYG{p}{(}\PYG{n}{s}\PYG{p}{,} \PYG{l+m+mi}{15}\PYG{p}{,} \PYG{n}{normed}\PYG{o}{=}\PYG{n+nb+bp}{True}\PYG{p}{)}
\PYG{g+gp}{\PYGZgt{}\PYGZgt{}\PYGZgt{} }\PYG{n}{plt}\PYG{o}{.}\PYG{n}{plot}\PYG{p}{(}\PYG{n}{bins}\PYG{p}{,} \PYG{n}{np}\PYG{o}{.}\PYG{n}{ones\PYGZus{}like}\PYG{p}{(}\PYG{n}{bins}\PYG{p}{)}\PYG{p}{,} \PYG{n}{linewidth}\PYG{o}{=}\PYG{l+m+mi}{2}\PYG{p}{,} \PYG{n}{color}\PYG{o}{=}\PYG{l+s}{\PYGZsq{}}\PYG{l+s}{r}\PYG{l+s}{\PYGZsq{}}\PYG{p}{)}
\PYG{g+gp}{\PYGZgt{}\PYGZgt{}\PYGZgt{} }\PYG{n}{plt}\PYG{o}{.}\PYG{n}{show}\PYG{p}{(}\PYG{p}{)}
\end{Verbatim}

\end{fulllineitems}

\index{vonmises() (in module topology\_analysis)}

\begin{fulllineitems}
\phantomsection\label{topology_analysis:topology_analysis.vonmises}\pysiglinewithargsret{\code{topology\_analysis.}\bfcode{vonmises}}{\emph{mu}, \emph{kappa}, \emph{size=None}}{}
Draw samples from a von Mises distribution.

Samples are drawn from a von Mises distribution with specified mode
(mu) and dispersion (kappa), on the interval {[}-pi, pi{]}.

The von Mises distribution (also known as the circular normal
distribution) is a continuous probability distribution on the unit
circle.  It may be thought of as the circular analogue of the normal
distribution.
\begin{description}
\item[{mu}] \leavevmode{[}float{]}
Mode (``center'') of the distribution.

\item[{kappa}] \leavevmode{[}float{]}
Dispersion of the distribution, has to be \textgreater{}=0.

\item[{size}] \leavevmode{[}int or tuple of int{]}
Output shape.  If the given shape is, e.g., \code{(m, n, k)}, then
\code{m * n * k} samples are drawn.

\end{description}
\begin{description}
\item[{samples}] \leavevmode{[}scalar or ndarray{]}
The returned samples, which are in the interval {[}-pi, pi{]}.

\end{description}
\begin{description}
\item[{scipy.stats.distributions.vonmises}] \leavevmode{[}probability density function,{]}
distribution, or cumulative density function, etc.

\end{description}

The probability density for the von Mises distribution is
\begin{gather}
\begin{split}p(x) = \frac{e^{\kappa cos(x-\mu)}}{2\pi I_0(\kappa)},\end{split}\notag
\end{gather}
where \(\mu\) is the mode and \(\kappa\) the dispersion,
and \(I_0(\kappa)\) is the modified Bessel function of order 0.

The von Mises is named for Richard Edler von Mises, who was born in
Austria-Hungary, in what is now the Ukraine.  He fled to the United
States in 1939 and became a professor at Harvard.  He worked in
probability theory, aerodynamics, fluid mechanics, and philosophy of
science.

Abramowitz, M. and Stegun, I. A. (ed.), \emph{Handbook of Mathematical
Functions}, New York: Dover, 1965.

von Mises, R., \emph{Mathematical Theory of Probability and Statistics},
New York: Academic Press, 1964.

Draw samples from the distribution:

\begin{Verbatim}[commandchars=\\\{\}]
\PYG{g+gp}{\PYGZgt{}\PYGZgt{}\PYGZgt{} }\PYG{n}{mu}\PYG{p}{,} \PYG{n}{kappa} \PYG{o}{=} \PYG{l+m+mf}{0.0}\PYG{p}{,} \PYG{l+m+mf}{4.0} \PYG{c}{\PYGZsh{} mean and dispersion}
\PYG{g+gp}{\PYGZgt{}\PYGZgt{}\PYGZgt{} }\PYG{n}{s} \PYG{o}{=} \PYG{n}{np}\PYG{o}{.}\PYG{n}{random}\PYG{o}{.}\PYG{n}{vonmises}\PYG{p}{(}\PYG{n}{mu}\PYG{p}{,} \PYG{n}{kappa}\PYG{p}{,} \PYG{l+m+mi}{1000}\PYG{p}{)}
\end{Verbatim}

Display the histogram of the samples, along with
the probability density function:

\begin{Verbatim}[commandchars=\\\{\}]
\PYG{g+gp}{\PYGZgt{}\PYGZgt{}\PYGZgt{} }\PYG{k+kn}{import} \PYG{n+nn}{matplotlib.pyplot} \PYG{k+kn}{as} \PYG{n+nn}{plt}
\PYG{g+gp}{\PYGZgt{}\PYGZgt{}\PYGZgt{} }\PYG{k+kn}{import} \PYG{n+nn}{scipy.special} \PYG{k+kn}{as} \PYG{n+nn}{sps}
\PYG{g+gp}{\PYGZgt{}\PYGZgt{}\PYGZgt{} }\PYG{n}{count}\PYG{p}{,} \PYG{n}{bins}\PYG{p}{,} \PYG{n}{ignored} \PYG{o}{=} \PYG{n}{plt}\PYG{o}{.}\PYG{n}{hist}\PYG{p}{(}\PYG{n}{s}\PYG{p}{,} \PYG{l+m+mi}{50}\PYG{p}{,} \PYG{n}{normed}\PYG{o}{=}\PYG{n+nb+bp}{True}\PYG{p}{)}
\PYG{g+gp}{\PYGZgt{}\PYGZgt{}\PYGZgt{} }\PYG{n}{x} \PYG{o}{=} \PYG{n}{np}\PYG{o}{.}\PYG{n}{arange}\PYG{p}{(}\PYG{o}{\PYGZhy{}}\PYG{n}{np}\PYG{o}{.}\PYG{n}{pi}\PYG{p}{,} \PYG{n}{np}\PYG{o}{.}\PYG{n}{pi}\PYG{p}{,} \PYG{l+m+mi}{2}\PYG{o}{*}\PYG{n}{np}\PYG{o}{.}\PYG{n}{pi}\PYG{o}{/}\PYG{l+m+mf}{50.}\PYG{p}{)}
\PYG{g+gp}{\PYGZgt{}\PYGZgt{}\PYGZgt{} }\PYG{n}{y} \PYG{o}{=} \PYG{o}{\PYGZhy{}}\PYG{n}{np}\PYG{o}{.}\PYG{n}{exp}\PYG{p}{(}\PYG{n}{kappa}\PYG{o}{*}\PYG{n}{np}\PYG{o}{.}\PYG{n}{cos}\PYG{p}{(}\PYG{n}{x}\PYG{o}{\PYGZhy{}}\PYG{n}{mu}\PYG{p}{)}\PYG{p}{)}\PYG{o}{/}\PYG{p}{(}\PYG{l+m+mi}{2}\PYG{o}{*}\PYG{n}{np}\PYG{o}{.}\PYG{n}{pi}\PYG{o}{*}\PYG{n}{sps}\PYG{o}{.}\PYG{n}{jn}\PYG{p}{(}\PYG{l+m+mi}{0}\PYG{p}{,}\PYG{n}{kappa}\PYG{p}{)}\PYG{p}{)}
\PYG{g+gp}{\PYGZgt{}\PYGZgt{}\PYGZgt{} }\PYG{n}{plt}\PYG{o}{.}\PYG{n}{plot}\PYG{p}{(}\PYG{n}{x}\PYG{p}{,} \PYG{n}{y}\PYG{o}{/}\PYG{n+nb}{max}\PYG{p}{(}\PYG{n}{y}\PYG{p}{)}\PYG{p}{,} \PYG{n}{linewidth}\PYG{o}{=}\PYG{l+m+mi}{2}\PYG{p}{,} \PYG{n}{color}\PYG{o}{=}\PYG{l+s}{\PYGZsq{}}\PYG{l+s}{r}\PYG{l+s}{\PYGZsq{}}\PYG{p}{)}
\PYG{g+gp}{\PYGZgt{}\PYGZgt{}\PYGZgt{} }\PYG{n}{plt}\PYG{o}{.}\PYG{n}{show}\PYG{p}{(}\PYG{p}{)}
\end{Verbatim}

\end{fulllineitems}

\index{wald() (in module topology\_analysis)}

\begin{fulllineitems}
\phantomsection\label{topology_analysis:topology_analysis.wald}\pysiglinewithargsret{\code{topology\_analysis.}\bfcode{wald}}{\emph{mean}, \emph{scale}, \emph{size=None}}{}
Draw samples from a Wald, or Inverse Gaussian, distribution.

As the scale approaches infinity, the distribution becomes more like a
Gaussian.

Some references claim that the Wald is an Inverse Gaussian with mean=1, but
this is by no means universal.

The Inverse Gaussian distribution was first studied in relationship to
Brownian motion. In 1956 M.C.K. Tweedie used the name Inverse Gaussian
because there is an inverse relationship between the time to cover a unit
distance and distance covered in unit time.
\begin{description}
\item[{mean}] \leavevmode{[}scalar{]}
Distribution mean, should be \textgreater{} 0.

\item[{scale}] \leavevmode{[}scalar{]}
Scale parameter, should be \textgreater{}= 0.

\item[{size}] \leavevmode{[}int or tuple of ints, optional{]}
Output shape. Default is None, in which case a single value is
returned.

\end{description}
\begin{description}
\item[{samples}] \leavevmode{[}ndarray or scalar{]}
Drawn sample, all greater than zero.

\end{description}

The probability density function for the Wald distribution is
\begin{gather}
\begin{split}P(x;mean,scale) = \sqrt{\frac{scale}{2\pi x^3}}e^
\frac{-scale(x-mean)^2}{2\cdotp mean^2x}\end{split}\notag
\end{gather}
As noted above the Inverse Gaussian distribution first arise from attempts
to model Brownian Motion. It is also a competitor to the Weibull for use in
reliability modeling and modeling stock returns and interest rate
processes.

Draw values from the distribution and plot the histogram:

\begin{Verbatim}[commandchars=\\\{\}]
\PYG{g+gp}{\PYGZgt{}\PYGZgt{}\PYGZgt{} }\PYG{k+kn}{import} \PYG{n+nn}{matplotlib.pyplot} \PYG{k+kn}{as} \PYG{n+nn}{plt}
\PYG{g+gp}{\PYGZgt{}\PYGZgt{}\PYGZgt{} }\PYG{n}{h} \PYG{o}{=} \PYG{n}{plt}\PYG{o}{.}\PYG{n}{hist}\PYG{p}{(}\PYG{n}{np}\PYG{o}{.}\PYG{n}{random}\PYG{o}{.}\PYG{n}{wald}\PYG{p}{(}\PYG{l+m+mi}{3}\PYG{p}{,} \PYG{l+m+mi}{2}\PYG{p}{,} \PYG{l+m+mi}{100000}\PYG{p}{)}\PYG{p}{,} \PYG{n}{bins}\PYG{o}{=}\PYG{l+m+mi}{200}\PYG{p}{,} \PYG{n}{normed}\PYG{o}{=}\PYG{n+nb+bp}{True}\PYG{p}{)}
\PYG{g+gp}{\PYGZgt{}\PYGZgt{}\PYGZgt{} }\PYG{n}{plt}\PYG{o}{.}\PYG{n}{show}\PYG{p}{(}\PYG{p}{)}
\end{Verbatim}

\end{fulllineitems}

\index{weibull() (in module topology\_analysis)}

\begin{fulllineitems}
\phantomsection\label{topology_analysis:topology_analysis.weibull}\pysiglinewithargsret{\code{topology\_analysis.}\bfcode{weibull}}{\emph{a}, \emph{size=None}}{}
Weibull distribution.

Draw samples from a 1-parameter Weibull distribution with the given
shape parameter \emph{a}.
\begin{gather}
\begin{split}X = (-ln(U))^{1/a}\end{split}\notag
\end{gather}
Here, U is drawn from the uniform distribution over (0,1{]}.

The more common 2-parameter Weibull, including a scale parameter
\(\lambda\) is just \(X = \lambda(-ln(U))^{1/a}\).
\begin{description}
\item[{a}] \leavevmode{[}float{]}
Shape of the distribution.

\item[{size}] \leavevmode{[}tuple of ints{]}
Output shape.  If the given shape is, e.g., \code{(m, n, k)}, then
\code{m * n * k} samples are drawn.

\end{description}

scipy.stats.distributions.weibull\_max
scipy.stats.distributions.weibull\_min
scipy.stats.distributions.genextreme
gumbel

The Weibull (or Type III asymptotic extreme value distribution for smallest
values, SEV Type III, or Rosin-Rammler distribution) is one of a class of
Generalized Extreme Value (GEV) distributions used in modeling extreme
value problems.  This class includes the Gumbel and Frechet distributions.

The probability density for the Weibull distribution is
\begin{gather}
\begin{split}p(x) = \frac{a}
{\lambda}(\frac{x}{\lambda})^{a-1}e^{-(x/\lambda)^a},\end{split}\notag
\end{gather}
where \(a\) is the shape and \(\lambda\) the scale.

The function has its peak (the mode) at
\(\lambda(\frac{a-1}{a})^{1/a}\).

When \code{a = 1}, the Weibull distribution reduces to the exponential
distribution.

Draw samples from the distribution:

\begin{Verbatim}[commandchars=\\\{\}]
\PYG{g+gp}{\PYGZgt{}\PYGZgt{}\PYGZgt{} }\PYG{n}{a} \PYG{o}{=} \PYG{l+m+mf}{5.} \PYG{c}{\PYGZsh{} shape}
\PYG{g+gp}{\PYGZgt{}\PYGZgt{}\PYGZgt{} }\PYG{n}{s} \PYG{o}{=} \PYG{n}{np}\PYG{o}{.}\PYG{n}{random}\PYG{o}{.}\PYG{n}{weibull}\PYG{p}{(}\PYG{n}{a}\PYG{p}{,} \PYG{l+m+mi}{1000}\PYG{p}{)}
\end{Verbatim}

Display the histogram of the samples, along with
the probability density function:

\begin{Verbatim}[commandchars=\\\{\}]
\PYG{g+gp}{\PYGZgt{}\PYGZgt{}\PYGZgt{} }\PYG{k+kn}{import} \PYG{n+nn}{matplotlib.pyplot} \PYG{k+kn}{as} \PYG{n+nn}{plt}
\PYG{g+gp}{\PYGZgt{}\PYGZgt{}\PYGZgt{} }\PYG{n}{x} \PYG{o}{=} \PYG{n}{np}\PYG{o}{.}\PYG{n}{arange}\PYG{p}{(}\PYG{l+m+mi}{1}\PYG{p}{,}\PYG{l+m+mf}{100.}\PYG{p}{)}\PYG{o}{/}\PYG{l+m+mf}{50.}
\PYG{g+gp}{\PYGZgt{}\PYGZgt{}\PYGZgt{} }\PYG{k}{def} \PYG{n+nf}{weib}\PYG{p}{(}\PYG{n}{x}\PYG{p}{,}\PYG{n}{n}\PYG{p}{,}\PYG{n}{a}\PYG{p}{)}\PYG{p}{:}
\PYG{g+gp}{... }    \PYG{k}{return} \PYG{p}{(}\PYG{n}{a} \PYG{o}{/} \PYG{n}{n}\PYG{p}{)} \PYG{o}{*} \PYG{p}{(}\PYG{n}{x} \PYG{o}{/} \PYG{n}{n}\PYG{p}{)}\PYG{o}{*}\PYG{o}{*}\PYG{p}{(}\PYG{n}{a} \PYG{o}{\PYGZhy{}} \PYG{l+m+mi}{1}\PYG{p}{)} \PYG{o}{*} \PYG{n}{np}\PYG{o}{.}\PYG{n}{exp}\PYG{p}{(}\PYG{o}{\PYGZhy{}}\PYG{p}{(}\PYG{n}{x} \PYG{o}{/} \PYG{n}{n}\PYG{p}{)}\PYG{o}{*}\PYG{o}{*}\PYG{n}{a}\PYG{p}{)}
\end{Verbatim}

\begin{Verbatim}[commandchars=\\\{\}]
\PYG{g+gp}{\PYGZgt{}\PYGZgt{}\PYGZgt{} }\PYG{n}{count}\PYG{p}{,} \PYG{n}{bins}\PYG{p}{,} \PYG{n}{ignored} \PYG{o}{=} \PYG{n}{plt}\PYG{o}{.}\PYG{n}{hist}\PYG{p}{(}\PYG{n}{np}\PYG{o}{.}\PYG{n}{random}\PYG{o}{.}\PYG{n}{weibull}\PYG{p}{(}\PYG{l+m+mf}{5.}\PYG{p}{,}\PYG{l+m+mi}{1000}\PYG{p}{)}\PYG{p}{)}
\PYG{g+gp}{\PYGZgt{}\PYGZgt{}\PYGZgt{} }\PYG{n}{x} \PYG{o}{=} \PYG{n}{np}\PYG{o}{.}\PYG{n}{arange}\PYG{p}{(}\PYG{l+m+mi}{1}\PYG{p}{,}\PYG{l+m+mf}{100.}\PYG{p}{)}\PYG{o}{/}\PYG{l+m+mf}{50.}
\PYG{g+gp}{\PYGZgt{}\PYGZgt{}\PYGZgt{} }\PYG{n}{scale} \PYG{o}{=} \PYG{n}{count}\PYG{o}{.}\PYG{n}{max}\PYG{p}{(}\PYG{p}{)}\PYG{o}{/}\PYG{n}{weib}\PYG{p}{(}\PYG{n}{x}\PYG{p}{,} \PYG{l+m+mf}{1.}\PYG{p}{,} \PYG{l+m+mf}{5.}\PYG{p}{)}\PYG{o}{.}\PYG{n}{max}\PYG{p}{(}\PYG{p}{)}
\PYG{g+gp}{\PYGZgt{}\PYGZgt{}\PYGZgt{} }\PYG{n}{plt}\PYG{o}{.}\PYG{n}{plot}\PYG{p}{(}\PYG{n}{x}\PYG{p}{,} \PYG{n}{weib}\PYG{p}{(}\PYG{n}{x}\PYG{p}{,} \PYG{l+m+mf}{1.}\PYG{p}{,} \PYG{l+m+mf}{5.}\PYG{p}{)}\PYG{o}{*}\PYG{n}{scale}\PYG{p}{)}
\PYG{g+gp}{\PYGZgt{}\PYGZgt{}\PYGZgt{} }\PYG{n}{plt}\PYG{o}{.}\PYG{n}{show}\PYG{p}{(}\PYG{p}{)}
\end{Verbatim}

\end{fulllineitems}

\index{zipf() (in module topology\_analysis)}

\begin{fulllineitems}
\phantomsection\label{topology_analysis:topology_analysis.zipf}\pysiglinewithargsret{\code{topology\_analysis.}\bfcode{zipf}}{\emph{a}, \emph{size=None}}{}
Draw samples from a Zipf distribution.

Samples are drawn from a Zipf distribution with specified parameter
\emph{a} \textgreater{} 1.

The Zipf distribution (also known as the zeta distribution) is a
continuous probability distribution that satisfies Zipf's law: the
frequency of an item is inversely proportional to its rank in a
frequency table.
\begin{description}
\item[{a}] \leavevmode{[}float \textgreater{} 1{]}
Distribution parameter.

\item[{size}] \leavevmode{[}int or tuple of int, optional{]}
Output shape.  If the given shape is, e.g., \code{(m, n, k)}, then
\code{m * n * k} samples are drawn; a single integer is equivalent in
its result to providing a mono-tuple, i.e., a 1-D array of length
\emph{size} is returned.  The default is None, in which case a single
scalar is returned.

\end{description}
\begin{description}
\item[{samples}] \leavevmode{[}scalar or ndarray{]}
The returned samples are greater than or equal to one.

\end{description}
\begin{description}
\item[{scipy.stats.distributions.zipf}] \leavevmode{[}probability density function,{]}
distribution, or cumulative density function, etc.

\end{description}

The probability density for the Zipf distribution is
\begin{gather}
\begin{split}p(x) = \frac{x^{-a}}{\zeta(a)},\end{split}\notag
\end{gather}
where \(\zeta\) is the Riemann Zeta function.

It is named for the American linguist George Kingsley Zipf, who noted
that the frequency of any word in a sample of a language is inversely
proportional to its rank in the frequency table.

Zipf, G. K., \emph{Selected Studies of the Principle of Relative Frequency
in Language}, Cambridge, MA: Harvard Univ. Press, 1932.

Draw samples from the distribution:

\begin{Verbatim}[commandchars=\\\{\}]
\PYG{g+gp}{\PYGZgt{}\PYGZgt{}\PYGZgt{} }\PYG{n}{a} \PYG{o}{=} \PYG{l+m+mf}{2.} \PYG{c}{\PYGZsh{} parameter}
\PYG{g+gp}{\PYGZgt{}\PYGZgt{}\PYGZgt{} }\PYG{n}{s} \PYG{o}{=} \PYG{n}{np}\PYG{o}{.}\PYG{n}{random}\PYG{o}{.}\PYG{n}{zipf}\PYG{p}{(}\PYG{n}{a}\PYG{p}{,} \PYG{l+m+mi}{1000}\PYG{p}{)}
\end{Verbatim}

Display the histogram of the samples, along with
the probability density function:

\begin{Verbatim}[commandchars=\\\{\}]
\PYG{g+gp}{\PYGZgt{}\PYGZgt{}\PYGZgt{} }\PYG{k+kn}{import} \PYG{n+nn}{matplotlib.pyplot} \PYG{k+kn}{as} \PYG{n+nn}{plt}
\PYG{g+gp}{\PYGZgt{}\PYGZgt{}\PYGZgt{} }\PYG{k+kn}{import} \PYG{n+nn}{scipy.special} \PYG{k+kn}{as} \PYG{n+nn}{sps}
\PYG{g+go}{Truncate s values at 50 so plot is interesting}
\PYG{g+gp}{\PYGZgt{}\PYGZgt{}\PYGZgt{} }\PYG{n}{count}\PYG{p}{,} \PYG{n}{bins}\PYG{p}{,} \PYG{n}{ignored} \PYG{o}{=} \PYG{n}{plt}\PYG{o}{.}\PYG{n}{hist}\PYG{p}{(}\PYG{n}{s}\PYG{p}{[}\PYG{n}{s}\PYG{o}{\PYGZlt{}}\PYG{l+m+mi}{50}\PYG{p}{]}\PYG{p}{,} \PYG{l+m+mi}{50}\PYG{p}{,} \PYG{n}{normed}\PYG{o}{=}\PYG{n+nb+bp}{True}\PYG{p}{)}
\PYG{g+gp}{\PYGZgt{}\PYGZgt{}\PYGZgt{} }\PYG{n}{x} \PYG{o}{=} \PYG{n}{np}\PYG{o}{.}\PYG{n}{arange}\PYG{p}{(}\PYG{l+m+mf}{1.}\PYG{p}{,} \PYG{l+m+mf}{50.}\PYG{p}{)}
\PYG{g+gp}{\PYGZgt{}\PYGZgt{}\PYGZgt{} }\PYG{n}{y} \PYG{o}{=} \PYG{n}{x}\PYG{o}{*}\PYG{o}{*}\PYG{p}{(}\PYG{o}{\PYGZhy{}}\PYG{n}{a}\PYG{p}{)}\PYG{o}{/}\PYG{n}{sps}\PYG{o}{.}\PYG{n}{zetac}\PYG{p}{(}\PYG{n}{a}\PYG{p}{)}
\PYG{g+gp}{\PYGZgt{}\PYGZgt{}\PYGZgt{} }\PYG{n}{plt}\PYG{o}{.}\PYG{n}{plot}\PYG{p}{(}\PYG{n}{x}\PYG{p}{,} \PYG{n}{y}\PYG{o}{/}\PYG{n+nb}{max}\PYG{p}{(}\PYG{n}{y}\PYG{p}{)}\PYG{p}{,} \PYG{n}{linewidth}\PYG{o}{=}\PYG{l+m+mi}{2}\PYG{p}{,} \PYG{n}{color}\PYG{o}{=}\PYG{l+s}{\PYGZsq{}}\PYG{l+s}{r}\PYG{l+s}{\PYGZsq{}}\PYG{p}{)}
\PYG{g+gp}{\PYGZgt{}\PYGZgt{}\PYGZgt{} }\PYG{n}{plt}\PYG{o}{.}\PYG{n}{show}\PYG{p}{(}\PYG{p}{)}
\end{Verbatim}

\end{fulllineitems}



\chapter{Indices and tables}
\label{index:indices-and-tables}\begin{itemize}
\item {} 
\emph{genindex}

\item {} 
\emph{modindex}

\item {} 
\emph{search}

\end{itemize}


\renewcommand{\indexname}{Python Module Index}
\begin{theindex}
\def\bigletter#1{{\Large\sffamily#1}\nopagebreak\vspace{1mm}}
\bigletter{a}
\item {\texttt{acsAttractorAnalysis}}, \pageref{acsAttractorAnalysis:module-acsAttractorAnalysis}
\item {\texttt{acsAttractorAnalysisInTime}}, \pageref{acsAttractorAnalysisInTime:module-acsAttractorAnalysisInTime}
\item {\texttt{acsBufferedFluxes}}, \pageref{acsBufferedFluxes:module-acsBufferedFluxes}
\item {\texttt{acsDynStatInTime}}, \pageref{acsDynStatInTime:module-acsDynStatInTime}
\item {\texttt{acsFromWim2Carness}}, \pageref{acsFromWim2Carness:module-acsFromWim2Carness}
\item {\texttt{acsSCCanalysis}}, \pageref{acsSCCanalysis:module-acsSCCanalysis}
\item {\texttt{acsSpeciesActivities}}, \pageref{acsSpeciesActivities:module-acsSpeciesActivities}
\item {\texttt{acsStatesAnalysis}}, \pageref{acsStatesAnalysis:module-acsStatesAnalysis}
\indexspace
\bigletter{g}
\item {\texttt{graph\_chemistry\_analysis}}, \pageref{graph_chemistry_analysis:module-graph_chemistry_analysis}
\indexspace
\bigletter{i}
\item {\texttt{initializator}}, \pageref{initializator:module-initializator}
\indexspace
\bigletter{l}
\item {\texttt{lib.dyn.dynamics}}, \pageref{lib.dyn:module-lib.dyn.dynamics}
\item {\texttt{lib.graph.network}}, \pageref{lib.graph:module-lib.graph.network}
\item {\texttt{lib.graph.raf}}, \pageref{lib.graph:module-lib.graph.raf}
\item {\texttt{lib.graph.scc}}, \pageref{lib.graph:module-lib.graph.scc}
\item {\texttt{lib.IO}}, \pageref{lib.IO:module-lib.IO}
\item {\texttt{lib.IO.readfiles}}, \pageref{lib.IO:module-lib.IO.readfiles}
\item {\texttt{lib.IO.writefiles}}, \pageref{lib.IO:module-lib.IO.writefiles}
\item {\texttt{lib.model.reactions}}, \pageref{lib.model:module-lib.model.reactions}
\item {\texttt{lib.model.species}}, \pageref{lib.model:module-lib.model.species}
\indexspace
\bigletter{m}
\item {\texttt{main}}, \pageref{main:module-main}
\indexspace
\bigletter{t}
\item {\texttt{topology\_analysis}}, \pageref{topology_analysis:module-topology_analysis}
\end{theindex}

\renewcommand{\indexname}{Index}
\printindex
\end{document}
